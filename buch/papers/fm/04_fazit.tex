%
% teil3.tex -- Beispiel-File für Teil 3
%
% (c) 2020 Prof Dr Andreas Müller, Hochschule Rapperswil
%
\section{Fazit
\label{fm:section:fazit}}
\rhead{Zusamenfassend}
Ohne die Besselfunktion könnte man die Einzelen Peaks der Fm nichicht sounabängig von einander berchenen und herausfinden.
Da die Besselfunktion schnell abklingt, brauchte es auch wenige Besselkoeffizente um das Nachrichten Signal wieder zurückzugewinnen.

TODO  Anwendungen erklären und Sinn des Ganzen.


