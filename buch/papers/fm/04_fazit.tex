%
% teil3.tex -- Beispiel-File für Teil 3
%
% (c) 2020 Prof Dr Andreas Müller, Hochschule Rapperswil
%
\section{Fazit
\label{fm:section:fazit}}
\rhead{Zusamenfassend}
Ohne die Bessel-Funktion könnte man die einzelen Peaks der Frequenz
nicht so unabängig von einander berechnen.
Da die Bessel-Funktion schnell abklingt, brauchte es auch nur wenige
Bessel-Koeffizenten um das Nachrichtensignal wieder zurückzugewinnen.


