%
% main.tex -- Paper zum Thema <fm>
%
% (c) 2020 Hochschule Rapperswil
% 
% !TeX root = buch.tex
%\begin {document}
\chapter{Thema\label{chapter:fm}}
\lhead{Thema}
\begin{refsection}

\chapterauthor{Joshua Bär}

Ein paar Hinweise für die korrekte Formatierung des Textes
\begin{itemize}
\item
Absätze werden gebildet, indem man eine Leerzeile einfügt.
Die Verwendung von \verb+\\+ ist nur in Tabellen und Arrays gestattet.
\item
Die explizite Platzierung von Bildern ist nicht erlaubt, entsprechende
Optionen werden gelöscht. 
Verwenden Sie Labels und Verweise, um auf Bilder hinzuweisen.
\item
Beginnen Sie jeden Satz auf einer neuen Zeile. 
Damit ermöglichen Sie dem Versionsverwaltungssysteme, Änderungen
in verschiedenen Sätzen von verschiedenen Autoren ohne Konflikt 
anzuwenden.
\item 
Bilden Sie auch für Formeln kurze Zeilen, einerseits der besseren
Übersicht wegen, aber auch um GIT die Arbeit zu erleichtern.
\end{itemize}

%
% teil0.tex -- Definition
%
% (c) 2020 Prof Dr Andreas Müller, Hochschule Rapperswil
%
\section{Definition\label{fresnel:section:teil0}}
\kopfrechts{Definition}
Die Funktion $e^{x^2}$ hat bekanntermassen keine elementare Stammfunktion,
weshalb die Fehlerfunktion als Stammfunktion definiert wurde.
Die Funktionen $\cos x^2$ und $\sin x^2$ sind eng mit $e^{x^2}$
verwandt, es ist daher nicht überraschend, dass sie ebenfalls
keine elementare Stammfunktionen haben.
Dies rechtfertigt die Definition der Fresnel-Integrale als neue spezielle
Funktionen.

\begin{definition}
Die Funktionen 
\begin{align*}
C(x) &= \int_0^x \cos\biggl(\frac{\pi}2 t^2\biggr)\,dt
\\
S(x) &= \int_0^x \sin\biggl(\frac{\pi}2 t^2\biggr)\,dt
\end{align*}
heissen die Fresnel-Integrale.
\end{definition}

Der Faktor $\frac{\pi}2$ ist einigermassen willkürlich, man könnte
daher noch allgemeiner die Funktionen
\begin{align*}
C_a(x) &= \int_0^x \cos(at^2)\,dt
\\
S_a(x) &= \int_0^x \sin(at^2)\,dt
\end{align*}
definieren, so dass die Funktionen $C(x)$ und $S(x)$ der Fall
$a=\frac{\pi}2$ werden, also
\[
\begin{aligned}
C(x) &= C_{\frac{\pi}2}(x),
&
S(x) &= S_{\frac{\pi}2}(x).
\end{aligned}
\]
Durch eine Substitution $t=bs$ erhält man
\begin{align*}
C_a(x)
&=
\int_0^x \cos(at^2)\,dt
=
b
\int_0^{\frac{x}b} \cos(ab^2s^2)\,ds
=
b
C_{ab^2}\biggl(\frac{x}b\biggr)
\\
S_a(x)
&=
\int_0^x \sin(at^2)\,dt
=
b
\int_0^{\frac{x}b} \sin(ab^2s^2)\,ds
=
b
S_{ab^2}\biggl(\frac{x}b\biggr).
\end{align*}
Indem man $ab^2=\frac{\pi}2$ setzt, also
\[
b
=
\sqrt{\frac{\pi}{2a}}
,
\]
kann man die Funktionen $C_a(x)$ und $S_a(x)$ durch $C(x)$ und $S(x)$
ausdrücken:
\begin{align}
C_a(x)
&=
\sqrt{\frac{\pi}{2a}}
C\biggl(x
\sqrt{\frac{2a}{\pi}}
\biggr)
&&\text{und}&
S_a(x)
&=
\sqrt{\frac{\pi}{2a}}
S\biggl(x
\sqrt{\frac{2a}{\pi}}
\biggr).
\label{fresnel:equation:arg}
\end{align}
Im Folgenden werden wir meistens nur den Fall $a=1$, also die Funktionen
$C_1(x)$ und $S_1(x)$ betrachten, da in diesem Fall die Formeln einfacher
werden.
\begin{figure}
\centering
\includegraphics{papers/fresnel/images/fresnelgraph.pdf}
\caption{Graph der Funktionen $C(x)$ ({\color{red}rot}) 
und $S(x)$ ({\color{blue}blau})
\label{fresnel:figure:plot}}
\end{figure}
Die Abbildung~\ref{fresnel:figure:plot} zeigt die Graphen der
Funktion $C(x)$ und $S(x)$.


%
% teil1.tex -- Beispiel-File für das Paper
%
% (c) 2020 Prof Dr Andreas Müller, Hochschule Rapperswil
%
\section{Ordnungsstatistik und Beta-Funktion
\label{dreieck:section:ordnungsstatistik}}
\rhead{}




%
% teil2.tex -- Beispiel-File für teil2 
%
% (c) 2020 Prof Dr Andreas Müller, Hochschule Rapperswil
%
\section{Anwendung in der Physik 
\label{parzyl:section:teil2}}
\rhead{Anwendung in der Physik}

Die parabolischen Zylinderkoordinaten tauchen auf, wenn man das elektrische Feld einer semi-infiniten Platte, wie in Abbildung \ref{parzyl:fig:leiterplatte} gezeigt, finden will.
\begin{figure}
	 \centering
	\includegraphics[width=0.9\textwidth]{papers/parzyl/img/plane.pdf}
	\caption{Semi-infinite Leiterplatte}
	\label{parzyl:fig:leiterplatte}
\end{figure}
Das dies so ist kann im zwei Dimensionalen mit Hilfe von komplexen Funktionen gezeigt werden. Die Platte ist dann nur eine Linie, was man in Abbildung TODO sieht.
Jede komplexe Funktion $F(z)$ kann geschrieben werden als
\begin{equation}
	F(s) = U(x,y) + iV(x,y) \qquad s \in \mathbb{C}; x,y \in \mathbb{R}.
\end{equation}  
Dabei muss gelten, falls die Funktion differenzierbar ist, dass
\begin{equation}
	\frac{\partial U(x,y)}{\partial x} 
	=
	\frac{\partial V(x,y)}{\partial y} 
	\qquad
	\frac{\partial V(x,y)}{\partial x}
	=
	-\frac{\partial U(x,y)}{\partial y}.
\end{equation}
Aus dieser Bedingung folgt 
\begin{equation}
	\label{parzyl_e_feld_zweite_ab}
	\underbrace{
	\frac{\partial^2 U(x,y)}{\partial x^2}
	+ 
	\frac{\partial^2 U(x,y)}{\partial y^2}
	=
	0
	}_{\displaystyle{\nabla^2U(x,y)=0}}
	\qquad
	\underbrace{
	\frac{\partial^2 V(x,y)}{\partial x^2}
	+
	\frac{\partial^2 V(x,y)}{\partial y^2}
	=
	0
	}_{\displaystyle{\nabla^2V(x,y) = 0}}.
\end{equation}
Zusätzlich kann auch gezeigt werden, dass die Funktion $F(z)$ eine winkeltreue Abbildung ist. 
Der Zusammenhang zum elektrischen Feld ist jetzt, dass das Potential an einem quellenfreien Punkt gegeben ist als 
\begin{equation}
	\nabla^2\phi(x,y) = 0.
\end{equation}
Dies ist eine Bedingung welche differenzierbare Funktionen, wie in Gleichung \ref{parzyl_e_feld_zweite_ab} gezeigt wird, bereits besitzen. 
Nun kann zum Beispiel $U(x,y)$ als das Potential angeschaut werden
\begin{equation}
	\phi(x,y) = U(x,y).
\end{equation}
Orthogonal zum Potential ist das elektrische Feld
\begin{equation}
	E(x,y) = V(x,y).
\end{equation}
Um nun zu den parabolische Zylinderkoordinaten zu gelangen muss nur noch eine geeignete komplexe Funktion $F(s)$ gefunden werden, 
welche eine semi-infinite Platte beschreiben kann.
Die gesuchte Funktion in diesem Fall ist
\begin{equation}
	F(s) 
	= 
	\sqrt{s} 
	= 
	\sqrt{x + iy}.
\end{equation}
Dies kann umgeformt werden zu
\begin{equation}
	F(s) 
	= 
	\underbrace{\sqrt{\frac{\sqrt{x^2+y^2} + x}{2}}}_{U(x,y)} 
	+ 
	i\underbrace{\sqrt{\frac{\sqrt{x^2+y^2} - x}{2}}}_{V(x,y)}
	.
\end{equation}
Die Äquipotentialflächen können nun betrachtet werden, indem man die Funktion welche das Potential beschreibt gleich eine Konstante setzt,
\begin{equation}
	\sigma = U(x,y) = \sqrt{\frac{\sqrt{x^2+y^2} + x}{2}},
\end{equation}
und die Flächen mit der gleichen elektrischen Feldstärke können als
\begin{equation}
	\tau = V(x,y) = \sqrt{\frac{\sqrt{x^2+y^2} - x}{2}}
\end{equation}
beschrieben werden. Diese zwei Gleichungen zeigen nun wie man vom kartesischen Koordinatensystem ins parabolische Zylinderkoordinatensystem kommt. Werden diese Formeln nun nach x und y aufgelöst so beschreibe sie, wie man aus dem parabolischen Zylinderkoordinatensystem zurück ins kartesische rechnen kann
\begin{equation}
	x = \sigma \tau,
\end{equation}
\begin{equation}
	y = \frac{1}{2}\left ( \tau^2 - \sigma^2 \right )
\end{equation}






%
% teil3.tex -- Resultate und Ausblick
%
% (c) 2022 Fabian Dünki, Hochschule Rapperswil
%
\section{Auswertung
\label{0f1:section:teil3}}
\rhead{Resultate}
Im Verlauf dieser Arbeit hat sich gezeigt, 
das ein einfacher mathematischer Algorithmus zu implementieren gar nicht so einfach ist.
So haben alle drei umgesetzten Ansätze Probleme mit grossen negativen $z$ in der Funktion $\mathstrut_0F_1(;c;z)$.
Ebenso kann festgestellt werden, dass je grösser der Wert $z$ in $\mathstrut_0F_1(;c;z)$ wird, desto mehr weichen die berechneten Resultate von den Erwarteten \cite{0f1:wolfram-0f1} ab.

\subsection{Konvergenz
\label{0f1:subsection:konvergenz}}
Es zeigt sich in Abbildung \ref{0f1:ausblick:plot:airy:konvergenz}, dass schon nach drei Iterationen ($k = 3$) die Funktionen schon genaue Resultate im Bereich von $-2$ bis $2$ liefert. Ebenso kann festgestellt werden, dass der Kettenbruch schneller konvergiert und im positiven Bereich sogar mit der Referenzfunktion $\operatorname{Ai}(x)$ übereinstimmt. Da die Rekursionsformel eine Abwandlung des Kettenbruches ist, verhalten sich die Funktionen in diesem Fall gleich.

Erst wenn mehrerer Iterationen gemacht werden, um die Genauigkeit zu verbessern, ist der Kettenbruch den anderen zwei Algorithmen, bezüglich Konvergenz überlegen. 
Interessant ist auch, dass die Rekursionsformel nahezu gleich schnell wie die Potenzreihe konvergiert, aber sich danach, wie in Abbildung \ref{0f1:ausblick:plot:konvergenz:positiv} zu beobachten ist, einschwingt. Dieses Verhalten ist auch bei grösseren $z$ zu beobachten, allerdings ist dann die Differenz zwischen dem ersten lokalen Minimum von k bis zum Abbruch kleiner.
Dieses Phänomen ist auf die Lösung der Rekursionsformel \eqref{0f1:math:loesung:eq} zurück zu führen. Da im Gegensatz die ganz kleinen Werte nicht zu einer Konvergenz wie beim Kettenbruch führen, sondern sich noch eine Zeit lang durch die Multiplikation aufschwingen.

Ist $z$ negativ wie im Abbildung \ref{0f1:ausblick:plot:konvergenz:negativ}, führt dies zu einer Gegenseitigen Kompensation von negativen und positiven Termen so bricht die Rekursionsformel hier zusammen mit der Potenzreihe ab.
Die ansteigende Differenz mit anschliessender, ist aufgrund der sich alternierenden Termen mit wechselnden Vorzeichens zu erklären.

\subsection{Stabilität
\label{0f1:subsection:Stabilitaet}}
Verändert sich der Wert von z in $\mathstrut_0F_1(;c;z)$ gegen grössere positive Werte, wie zum Beispiel $c = 800$ liefert die Kettenbruch-Funktion (Listing \ref{0f1:listing:kettenbruchIterativ}) \verb+inf+ zurück. Dies könnte durch ein Abbruchkriterien abgefangen werden. Allerdings würde das, bei grossen Werten zulasten der Genauigkeit gehen. Trotzdem könnte, je nach Anwendung, auf ein paar Nachkommastellen verzichtet werden.

Wohingegen die Potenzreihe (Listing \ref{0f1:listing:potenzreihe}) das Problem hat, dass je mehr Terme berechnet werden, desto schneller wächst die Fakultät und irgendwann gibt es eine Bereichsüberschreitung von \verb+double+. Schlussendlich gibt das Unterprogramm das Resultat \verb+-nan(ind)+ zurück.
Die Rekursionformel \eqref{0f1:listing:kettenbruchRekursion} liefert für sehr grosse positive Werte die genausten Ergebnisse, verglichen mit der GNU Scientific Library. Wie schon vermutet ist die Rekursionsformel, im positivem Bereich, der stabilste Algorithmus. Um die Stabilität zu gewährleisten, muss wie in Abbildung \ref{0f1:ausblick:plot:konvergenz:positiv} dargestellt, die Iterationstiefe $k$ genug gross gewählt werden.

Im negativem Bereich sind alle gewählten und umgesetzten Ansätze instabil. Grund dafür ist die Potenz von z, was zum Phänomen der Auslöschung \cite{0f1:SeminarNumerik} führt. Schön zu beobachten ist dies in der Abbildung \ref{0f1:ausblick:plot:airy:stabilitaet} mit der Airy-Funktion als Test. So sind sowohl der Kettenbruch, als auch die Rekursionsformel bis ungefähr $\frac{-15^3}{9}$ stabil. Dies macht auch Sinn, da beide auf der gleichen mathematischen Grundlage basieren. Danach verhält sich allerdings die Instabilität unterschiedlich. Das unterschiedliche Verhalten kann damit erklärt werden, dass beim Kettenbruch jeweils eine zusätzliche Division stattfindet. Diese Unterschiede sind auch in Abbildung \ref{0f1:ausblick:plot:konvergenz:positiv} festzustellen.



\begin{figure}
    \centering
    \includegraphics[width=0.8\textwidth]{papers/0f1/images/konvergenzAiry.pdf}
    \caption{Konvergenz nach drei Iterationen, dargestellt anhand der Airy Funktion zu den Anfangsbedingungen $\operatorname{Ai}(0)=1$ und $\operatorname{Ai}'(0)=0$.
    \label{0f1:ausblick:plot:airy:konvergenz}}
\end{figure}

\begin{figure}
    \centering
    \includegraphics[width=0.8\textwidth]{papers/0f1/images/konvergenzPositiv.pdf}
    \caption{Konvergenz mit positivem z; Logarithmisch dargestellte Differenz vom erwarteten Endresultat.
    \label{0f1:ausblick:plot:konvergenz:positiv}}
\end{figure}

\begin{figure}
    \centering
    \includegraphics[width=0.8\textwidth]{papers/0f1/images/konvergenzNegativ.pdf}
    \caption{Konvergenz mit negativem z; Logarithmisch dargestellte Differenz vom erwarteten Endresultat.
    \label{0f1:ausblick:plot:konvergenz:negativ}}
\end{figure}

\begin{figure}
    \centering
    \includegraphics[width=1\textwidth]{papers/0f1/images/stabilitaet.pdf}
    \caption{Stabilität der 3 Algorithmen verglichen mit der Referenz Funktion $\operatorname{Ai}(x)$.
    \label{0f1:ausblick:plot:airy:stabilitaet}}
\end{figure}



\printbibliography[heading=subbibliography]
\end{refsection}

%\end {document}
