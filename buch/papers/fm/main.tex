%
% main.tex -- Paper zum Thema <fm>
%
% (c) 2020 Hochschule Rapperswil
% 

\chapter{FM  Bessel\label{chapter:fm}}
\lhead{FM}
\begin{refsection}

\chapterauthor{Joshua Bär}
%$\with$
Die Frequenzmodulation ist eine Modulation die man auch schon im alten Radio findet. 
Falls du dich an die Zeit erinnerst, konnte man zwischen \textit{FM-AM} Umschalten, 
dies bedeutete so viel wie: \textit{F}requenz-\textit{M}odulation und \textit{A}mplituden-\textit{M}odulation.
Durch die Modulation wird ein Nachrichtensignal \(m(t)\) auf ein Trägersignal (z.B. ein Sinus- oder Rechtecksignal) abgebildet (kombiniert).
Durch dieses Auftragen vom Nachrichtensignal \(m(t)\) kann das modulierte Signal in einem gewünschten Frequenzbereich übertragen werden.
Der ursprünglich Frequenzbereich des Nachrichtensignal \(m(t)\) erstreckt sich typischerweise von 0 Hz bis zur Bandbreite \(B_m\).
\newline
Beim Empfänger wird dann durch Demodulation das ursprüngliche Nachrichtensignal \(m(t)\) so originalgetreu wie möglich zurückgewonnen.
\newline
Beim Trägersignal \(x_c(t)\) handelt es sich um ein informationsloses Hilfssignal.
Durch die Modulation mit dem Nachrichtensignal \(m(t)\) wird es zum modulierten zu übertragenden Signal.
Für alle Erklärungen wird ein sinusförmiges Trägersignal benutzt, jedoch kann auch ein Rechtecksignal,
welches Digital einfach umzusetzten ist, 
genauso als Trägersignal genutzt werden kann.
Zuerst wird erklärt was \textit{FM-AM} ist, danach wie sich diese im Frequenzspektrum verhalten.
Erst dann erklär ich dir wie die Besselfunktion mit der Frequenzmodulation( acro?) zusammenhängt.
Nun zur Modulation im nächsten Abschnitt.

%
% einleitung.tex -- Beispiel-File für die Einleitung
%
% (c) 2020 Prof Dr Andreas Müller, Hochschule Rapperswil
%
\section{AM - FM\label{fm:section:teil0}}
\rhead{AM- FM}

Das sinusförmige Trägersignal hat die übliche Form: 
\(x_c(t) = A_c \cdot \cos(\omega_c(t)+\varphi)\).
Wobei die konstanten Amplitude \(A_c\) und Phase \(\varphi\) vom Nachrichtensignal \(m(t)\) verändert wird.
Der Parameter \(\omega_c\), die Trägerkreisfrequenz bzw. die Trägerfrequenz \(f_c = \frac{\omega_c}{2\pi}\),
steht nicht für die modulation zur verfügung, statt dessen kann durch ihn die Frequenzachse frei gewählt werden.
\newblockpunct
Jedoch ist das für die Vielfalt der Modulationsarten keine Einschrenkung.
Ein Nachrichtensignal kann auch über die Momentanfrequenz (instantenous frequency) \(\omega_i\) eines trägers verändert werden.
Mathematisch wird dann daraus
\[
    \omega_i = \omega_c + \frac{d \varphi(t)}{dt}
\]
mit der Ableitung der Phase\cite{fm:NAT}.
Mit diesen drei parameter ergeben sich auch drei modulationsarten, die Amplitudenmodulation welche \(A_c\) benutzt, 
die Phasenmodulation \(\varphi\) und dann noch die Momentankreisfrequenz \(\omega_i\):
\newline
\newline
To do: Bilder jeder Modulationsart

\subsection{AM - Amplitudenmodulation}
Das Ziel ist FM zu verstehen doch dazu wird zuerst AM erklärt welches einwenig einfacher zu verstehen ist und erst dann übertragen wir die Ideeen in FM.
Nun zur Amplitudenmodulation verwenden wir das bevorzugte Trägersignal
\[
    x_c(t) = A_c \cdot \cos(\omega_ct).
\]
Dies bringt den grossen Vorteil das, dass modulierend Signal sämtliche Anteile im Frequenzspektrum inanspruch nimmt 
und das Trägersignal nur  zwei komplexe Schwingungen besitzt. 
Dies sieht man besonders in der Eulerischen Formel
\[
    x_c(t) = \frac{A_c}{2} \cdot e^{j\omega_ct}\;+\;\frac{A_c}{2} \cdot e^{-j\omega_ct}.
\]
Dabei ist die negative Frequenz der zweiten komplexen Schwingung zwingend erforderlich, damit in der Summe immer ein reelwertiges Trägersignal ergibt.
Nun wird der parameter \(A_c\) durch das  Moduierende Signal \(m(t)\) ersetzt, wobei so \(m(t) \leqslant |1|\) normiert wurde.
\newline
\newline
TODO:
Hier beschrieib ich was AmplitudenModulation ist und mache dan den link zu Frequenzmodulation inkl Formel \[\cos( \cos x)\]
so wird beschrieben das daraus eigentlich \(x_c(t) = A_c \cdot \cos(\omega_i)\) wird und somit \(x_c(t) = A_c \cdot \cos(\omega_c + \frac{d \varphi(t)}{dt})\).
Da \(\sin \) abgeleitet \(\cos \) ergibt, so wird aus dem \(m(t)\) ein \( \frac{d \varphi(t)}{dt}\)  in der momentan frequenz. \[ \Rightarrow \cos( \cos x) \]

\input{papers/fm/02_frequenzyspectrum.tex}
%
% teil2.tex -- Beispiel-File für teil2 
%
% (c) 2020 Prof Dr Andreas Müller, Hochschule Rapperswil
%
\section{FM und Besselfunktion 
\label{fm:section:proof}}
\rhead{Herleitung}
Die momentane Trägerkreisfrequenz \(\omega_i\) wie schon in (ref) beschrieben ist, bringt die Vorigen Kapittel beschreiben. (Ableitung \(\frac{d \varphi(t)}{dt}\) mit sich).
Diese wiederum kann durch \(\beta\sin(\omega_mt)\) ausgedrückt werden, wobei es das Modulierende Signal \(m(t)\) ist.
Somit haben wir unser \(x_c\) welches 
\[
\cos(\omega_c t+\beta\sin(\omega_mt))
\]
ist.

\subsection{Herleitung}
Das Ziel ist es unser moduliertes Signal mit der Besselfunktion so auszudrücken:
\begin{align}
    x_c(t)
    = 
    \cos(\omega_ct+\beta\sin(\omega_mt))
    &=
    \sum_{k= -\infty}^\infty J_{k}(\beta) \cos((\omega_c+k\omega_m)t)
    \label{fm:eq:proof}
\end{align}
\subsubsection{Hilfsmittel}
Doch dazu brauchen wir die Hilfe der Additionsthoerme 
\begin{align}
    \cos(A + B) 
    &= 
    \cos(A)\cos(B)-\sin(A)\sin(B)
    \label{fm:eq:addth1}
    \\
    2\cos (A)\cos (B)
    &=
    \cos(A-B)+\cos(A+B)
    \label{fm:eq:addth2}
    \\
    2\sin(A)\sin(B)
    &=
    \cos(A-B)-\cos(A+B)
    \label{fm:eq:addth3}
\end{align}
und die drei Besselfunktions indentitäten,
\begin{align}
    \cos(\beta\sin\phi)
    &=
    J_0(\beta) + 2\sum_{k=1}^\infty J_{2k}(\beta) \cos(2k\phi)
    \label{fm:eq:besselid1}
    \\
    \sin(\beta\sin\phi)
    &=
    J_0(\beta) + 2\sum_{k=1}^\infty J_{2k+1}(\beta) \cos((2k+1)\phi)
    \label{fm:eq:besselid2}
    \\
    J_{-n}(\beta) &= (-1)^n J_n(\beta)
    \label{fm:eq:besselid3}
\end{align}
welche man im Kapitel (ref), ref, ref findet.

\subsubsection{Anwenden des Additionstheorem}
Mit dem \eqref{fm:eq:addth1} wird aus dem modulierten Signal
\[
    x_c(t) 
    =
    \cos(\omega_c t + \beta\sin(\omega_mt))
    =
    \cos(\omega_c t)\cos(\beta\sin(\omega_m t))-\sin(\omega_c)\sin(\beta\sin(\omega_m t)).
    \label{fm:eq:start}
\]
\subsubsection{Cos-Teil}
Zu beginn wird der Cos-Teil
\[
    \cos(\omega_c)\cos(\beta\sin(\omega_mt))  
\]
mit hilfe der Bessel indentität \eqref{fm:eq:besselid1} zum
\begin{align*}
    \cos(\omega_c t) \cdot [\, J_0(\beta) + 2\sum_{k=1}^\infty J_{2k}(\beta) \cos(2k\omega_m t)\, ]
    &=\\
    J_0(\beta)\cos(\omega_c t) + \sum_{k=1}^\infty J_{2k}(\beta) 
    \underbrace{2\cos(\omega_c t)\cos(2k\omega_m t)}_{Additionstheorem}
\end{align*}
wobei mit dem Additionstheorem \eqref{fm:eq:addth2} \(A = \omega_c t\) und \(B = 2k\omega_m t \) zum
\[
    J_0(\beta)\cdot \cos(\omega_c t) + \sum_{k=1}^\infty J_{2k}(\beta) \{ \cos((\omega_c - 2k\omega_m) t)+\cos((\omega_c + 2k\omega_m) t) \}
\]
wird.
Wenn dabei \(2k\) durch alle geraden Zahlen von \(-\infty \to \infty\) mit \(n\) substituiert erhält man den vereinfachten Term
\[
    \sum_{n\, \text{gerade}} J_{n}(\beta) \cos((\omega_c + n\omega_m) t),
    \label{fm:eq:gerade}
\]
dabei gehen nun die Terme von \(-\infty \to \infty\), dabei bleibt n Ganzzahlig.

\subsubsection{Sin-Teil}
Nun zum zweiten Teil des Term \eqref{fm:eq:start}, den Sin-Teil
\[
    \sin(\omega_c)\sin(\beta\sin(\omega_m t)).
\]
Dieser wird mit der \eqref{fm:eq:besselid2} Bessel indentität zu
\begin{align*}
    \sin(\omega_c t) \cdot [J_0(\beta)  \sin(\omega_c t) + 2\sum_{k=1}^\infty J_{2k+1}(\beta) \cos((2k+1)\omega_m t)]
    &=\\
    J_0(\beta) \cdot \sin(\omega_c t) + \sum_{k=1}^\infty J_{2k+1}(\beta) \underbrace{2\sin(\omega_c t)\cos((2k+1)\omega_m t)}_{Additionstheorem}.
\end{align*}
Auch hier wird ein Additionstheorem \eqref{fm:eq:addth3} gebraucht, dabei ist \(A = \omega_c t\) und \(B = (2k+1)\omega_m t \), 
somit wird daraus
\[
    J_0(\beta) \cdot \sin(\omega_c) + \sum_{k=1}^\infty J_{2k+1}(\beta) \{ \underbrace{\cos((\omega_c-(2k+1)\omega_m) t)}_{neg.Teil} - \cos((\omega_c+(2k+1)\omega_m) t) \}
\]dieser Term.
Wenn dabei \(2k +1\) durch alle ungeraden Zahlen von \(-\infty \to \infty\) mit \(n\) substituiert.
Zusätzlich dabei noch die letzte Bessel indentität \eqref{fm:eq:besselid3} brauchen, ist bei allen ungeraden negativen \(n : J_{-n}(\beta) = -1\cdot J_n(\beta)\).
Somit wird negTeil zum Term \(-\cos((\omega_c+(2k+1)\omega_m) t)\)und die Summe vereinfacht sich zu
\[
     \sum_{n\, \text{ungerade}} -1 \cdot J_{n}(\beta) \cos((\omega_c + n\omega_m) t).
     \label{fm:eq:ungerade}
\]
Substituiert man nun noch \(n \text{mit} -n \) so fällt das \(-1\) weg.

\subsubsection{Summe Zusammenführen}
Beide Teile \eqref{fm:eq:gerade} Gerade 
\[
    \sum_{n\, \text{gerade}} J_{n}(\beta) \cos((\omega_c + n\omega_m) t)
\]und \eqref{fm:eq:ungerade} Ungerade 
\[
    \sum_{n\, \text{ungerade}} J_{n}(\beta) \cos((\omega_c + n\omega_m) t)
\]
ergeben zusammen
\[
    \cos(\omega_ct+\beta\sin(\omega_mt))
    =
    \sum_{k= -\infty}^\infty J_{k}(\beta) \cos((\omega_c+k\omega_m)t).
\]
Somit ist \eqref{fm:eq:proof} bewiesen.
\newpage

%----------------------------------------------------------------------------
\subsection{Bessel und Frequenzspektrum}
Um sich das ganze noch einwenig Bildlicher vorzustellenhier einmal die Besselfunktion \(J_{k}(\beta)\) in geplottet.
\begin{figure}
	\centering
%	%% Creator: Matplotlib, PGF backend
%%
%% To include the figure in your LaTeX document, write
%%   \input{<filename>.pgf}
%%
%% Make sure the required packages are loaded in your preamble
%%   \usepackage{pgf}
%%
%% Also ensure that all the required font packages are loaded; for instance,
%% the lmodern package is sometimes necessary when using math font.
%%   \usepackage{lmodern}
%%
%% Figures using additional raster images can only be included by \input if
%% they are in the same directory as the main LaTeX file. For loading figures
%% from other directories you can use the `import` package
%%   \usepackage{import}
%%
%% and then include the figures with
%%   \import{<path to file>}{<filename>.pgf}
%%
%% Matplotlib used the following preamble
%%
\begingroup%
\makeatletter%
\begin{pgfpicture}%
\pgfpathrectangle{\pgfpointorigin}{\pgfqpoint{6.000000in}{4.000000in}}%
\pgfusepath{use as bounding box, clip}%
\begin{pgfscope}%
\pgfsetbuttcap%
\pgfsetmiterjoin%
\pgfsetlinewidth{0.000000pt}%
\definecolor{currentstroke}{rgb}{1.000000,1.000000,1.000000}%
\pgfsetstrokecolor{currentstroke}%
\pgfsetstrokeopacity{0.000000}%
\pgfsetdash{}{0pt}%
\pgfpathmoveto{\pgfqpoint{0.000000in}{0.000000in}}%
\pgfpathlineto{\pgfqpoint{6.000000in}{0.000000in}}%
\pgfpathlineto{\pgfqpoint{6.000000in}{4.000000in}}%
\pgfpathlineto{\pgfqpoint{0.000000in}{4.000000in}}%
\pgfpathlineto{\pgfqpoint{0.000000in}{0.000000in}}%
\pgfpathclose%
\pgfusepath{}%
\end{pgfscope}%
\begin{pgfscope}%
\pgfsetbuttcap%
\pgfsetmiterjoin%
\definecolor{currentfill}{rgb}{1.000000,1.000000,1.000000}%
\pgfsetfillcolor{currentfill}%
\pgfsetlinewidth{0.000000pt}%
\definecolor{currentstroke}{rgb}{0.000000,0.000000,0.000000}%
\pgfsetstrokecolor{currentstroke}%
\pgfsetstrokeopacity{0.000000}%
\pgfsetdash{}{0pt}%
\pgfpathmoveto{\pgfqpoint{0.750000in}{0.500000in}}%
\pgfpathlineto{\pgfqpoint{5.400000in}{0.500000in}}%
\pgfpathlineto{\pgfqpoint{5.400000in}{3.520000in}}%
\pgfpathlineto{\pgfqpoint{0.750000in}{3.520000in}}%
\pgfpathlineto{\pgfqpoint{0.750000in}{0.500000in}}%
\pgfpathclose%
\pgfusepath{fill}%
\end{pgfscope}%
\begin{pgfscope}%
\pgfpathrectangle{\pgfqpoint{0.750000in}{0.500000in}}{\pgfqpoint{4.650000in}{3.020000in}}%
\pgfusepath{clip}%
\pgfsetrectcap%
\pgfsetroundjoin%
\pgfsetlinewidth{0.803000pt}%
\definecolor{currentstroke}{rgb}{0.690196,0.690196,0.690196}%
\pgfsetstrokecolor{currentstroke}%
\pgfsetdash{}{0pt}%
\pgfpathmoveto{\pgfqpoint{0.750000in}{0.500000in}}%
\pgfpathlineto{\pgfqpoint{0.750000in}{3.520000in}}%
\pgfusepath{stroke}%
\end{pgfscope}%
\begin{pgfscope}%
\pgfsetbuttcap%
\pgfsetroundjoin%
\definecolor{currentfill}{rgb}{0.000000,0.000000,0.000000}%
\pgfsetfillcolor{currentfill}%
\pgfsetlinewidth{0.803000pt}%
\definecolor{currentstroke}{rgb}{0.000000,0.000000,0.000000}%
\pgfsetstrokecolor{currentstroke}%
\pgfsetdash{}{0pt}%
\pgfsys@defobject{currentmarker}{\pgfqpoint{0.000000in}{-0.048611in}}{\pgfqpoint{0.000000in}{0.000000in}}{%
\pgfpathmoveto{\pgfqpoint{0.000000in}{0.000000in}}%
\pgfpathlineto{\pgfqpoint{0.000000in}{-0.048611in}}%
\pgfusepath{stroke,fill}%
}%
\begin{pgfscope}%
\pgfsys@transformshift{0.750000in}{0.500000in}%
\pgfsys@useobject{currentmarker}{}%
\end{pgfscope}%
\end{pgfscope}%
\begin{pgfscope}%
\definecolor{textcolor}{rgb}{0.000000,0.000000,0.000000}%
\pgfsetstrokecolor{textcolor}%
\pgfsetfillcolor{textcolor}%
\pgftext[x=0.750000in,y=0.402778in,,top]{\color{textcolor}\rmfamily\fontsize{10.000000}{12.000000}\selectfont \(\displaystyle {\ensuremath{-}5.0}\)}%
\end{pgfscope}%
\begin{pgfscope}%
\pgfpathrectangle{\pgfqpoint{0.750000in}{0.500000in}}{\pgfqpoint{4.650000in}{3.020000in}}%
\pgfusepath{clip}%
\pgfsetrectcap%
\pgfsetroundjoin%
\pgfsetlinewidth{0.803000pt}%
\definecolor{currentstroke}{rgb}{0.690196,0.690196,0.690196}%
\pgfsetstrokecolor{currentstroke}%
\pgfsetdash{}{0pt}%
\pgfpathmoveto{\pgfqpoint{1.331250in}{0.500000in}}%
\pgfpathlineto{\pgfqpoint{1.331250in}{3.520000in}}%
\pgfusepath{stroke}%
\end{pgfscope}%
\begin{pgfscope}%
\pgfsetbuttcap%
\pgfsetroundjoin%
\definecolor{currentfill}{rgb}{0.000000,0.000000,0.000000}%
\pgfsetfillcolor{currentfill}%
\pgfsetlinewidth{0.803000pt}%
\definecolor{currentstroke}{rgb}{0.000000,0.000000,0.000000}%
\pgfsetstrokecolor{currentstroke}%
\pgfsetdash{}{0pt}%
\pgfsys@defobject{currentmarker}{\pgfqpoint{0.000000in}{-0.048611in}}{\pgfqpoint{0.000000in}{0.000000in}}{%
\pgfpathmoveto{\pgfqpoint{0.000000in}{0.000000in}}%
\pgfpathlineto{\pgfqpoint{0.000000in}{-0.048611in}}%
\pgfusepath{stroke,fill}%
}%
\begin{pgfscope}%
\pgfsys@transformshift{1.331250in}{0.500000in}%
\pgfsys@useobject{currentmarker}{}%
\end{pgfscope}%
\end{pgfscope}%
\begin{pgfscope}%
\definecolor{textcolor}{rgb}{0.000000,0.000000,0.000000}%
\pgfsetstrokecolor{textcolor}%
\pgfsetfillcolor{textcolor}%
\pgftext[x=1.331250in,y=0.402778in,,top]{\color{textcolor}\rmfamily\fontsize{10.000000}{12.000000}\selectfont \(\displaystyle {\ensuremath{-}2.5}\)}%
\end{pgfscope}%
\begin{pgfscope}%
\pgfpathrectangle{\pgfqpoint{0.750000in}{0.500000in}}{\pgfqpoint{4.650000in}{3.020000in}}%
\pgfusepath{clip}%
\pgfsetrectcap%
\pgfsetroundjoin%
\pgfsetlinewidth{0.803000pt}%
\definecolor{currentstroke}{rgb}{0.690196,0.690196,0.690196}%
\pgfsetstrokecolor{currentstroke}%
\pgfsetdash{}{0pt}%
\pgfpathmoveto{\pgfqpoint{1.912500in}{0.500000in}}%
\pgfpathlineto{\pgfqpoint{1.912500in}{3.520000in}}%
\pgfusepath{stroke}%
\end{pgfscope}%
\begin{pgfscope}%
\pgfsetbuttcap%
\pgfsetroundjoin%
\definecolor{currentfill}{rgb}{0.000000,0.000000,0.000000}%
\pgfsetfillcolor{currentfill}%
\pgfsetlinewidth{0.803000pt}%
\definecolor{currentstroke}{rgb}{0.000000,0.000000,0.000000}%
\pgfsetstrokecolor{currentstroke}%
\pgfsetdash{}{0pt}%
\pgfsys@defobject{currentmarker}{\pgfqpoint{0.000000in}{-0.048611in}}{\pgfqpoint{0.000000in}{0.000000in}}{%
\pgfpathmoveto{\pgfqpoint{0.000000in}{0.000000in}}%
\pgfpathlineto{\pgfqpoint{0.000000in}{-0.048611in}}%
\pgfusepath{stroke,fill}%
}%
\begin{pgfscope}%
\pgfsys@transformshift{1.912500in}{0.500000in}%
\pgfsys@useobject{currentmarker}{}%
\end{pgfscope}%
\end{pgfscope}%
\begin{pgfscope}%
\definecolor{textcolor}{rgb}{0.000000,0.000000,0.000000}%
\pgfsetstrokecolor{textcolor}%
\pgfsetfillcolor{textcolor}%
\pgftext[x=1.912500in,y=0.402778in,,top]{\color{textcolor}\rmfamily\fontsize{10.000000}{12.000000}\selectfont \(\displaystyle {0.0}\)}%
\end{pgfscope}%
\begin{pgfscope}%
\pgfpathrectangle{\pgfqpoint{0.750000in}{0.500000in}}{\pgfqpoint{4.650000in}{3.020000in}}%
\pgfusepath{clip}%
\pgfsetrectcap%
\pgfsetroundjoin%
\pgfsetlinewidth{0.803000pt}%
\definecolor{currentstroke}{rgb}{0.690196,0.690196,0.690196}%
\pgfsetstrokecolor{currentstroke}%
\pgfsetdash{}{0pt}%
\pgfpathmoveto{\pgfqpoint{2.493750in}{0.500000in}}%
\pgfpathlineto{\pgfqpoint{2.493750in}{3.520000in}}%
\pgfusepath{stroke}%
\end{pgfscope}%
\begin{pgfscope}%
\pgfsetbuttcap%
\pgfsetroundjoin%
\definecolor{currentfill}{rgb}{0.000000,0.000000,0.000000}%
\pgfsetfillcolor{currentfill}%
\pgfsetlinewidth{0.803000pt}%
\definecolor{currentstroke}{rgb}{0.000000,0.000000,0.000000}%
\pgfsetstrokecolor{currentstroke}%
\pgfsetdash{}{0pt}%
\pgfsys@defobject{currentmarker}{\pgfqpoint{0.000000in}{-0.048611in}}{\pgfqpoint{0.000000in}{0.000000in}}{%
\pgfpathmoveto{\pgfqpoint{0.000000in}{0.000000in}}%
\pgfpathlineto{\pgfqpoint{0.000000in}{-0.048611in}}%
\pgfusepath{stroke,fill}%
}%
\begin{pgfscope}%
\pgfsys@transformshift{2.493750in}{0.500000in}%
\pgfsys@useobject{currentmarker}{}%
\end{pgfscope}%
\end{pgfscope}%
\begin{pgfscope}%
\definecolor{textcolor}{rgb}{0.000000,0.000000,0.000000}%
\pgfsetstrokecolor{textcolor}%
\pgfsetfillcolor{textcolor}%
\pgftext[x=2.493750in,y=0.402778in,,top]{\color{textcolor}\rmfamily\fontsize{10.000000}{12.000000}\selectfont \(\displaystyle {2.5}\)}%
\end{pgfscope}%
\begin{pgfscope}%
\pgfpathrectangle{\pgfqpoint{0.750000in}{0.500000in}}{\pgfqpoint{4.650000in}{3.020000in}}%
\pgfusepath{clip}%
\pgfsetrectcap%
\pgfsetroundjoin%
\pgfsetlinewidth{0.803000pt}%
\definecolor{currentstroke}{rgb}{0.690196,0.690196,0.690196}%
\pgfsetstrokecolor{currentstroke}%
\pgfsetdash{}{0pt}%
\pgfpathmoveto{\pgfqpoint{3.075000in}{0.500000in}}%
\pgfpathlineto{\pgfqpoint{3.075000in}{3.520000in}}%
\pgfusepath{stroke}%
\end{pgfscope}%
\begin{pgfscope}%
\pgfsetbuttcap%
\pgfsetroundjoin%
\definecolor{currentfill}{rgb}{0.000000,0.000000,0.000000}%
\pgfsetfillcolor{currentfill}%
\pgfsetlinewidth{0.803000pt}%
\definecolor{currentstroke}{rgb}{0.000000,0.000000,0.000000}%
\pgfsetstrokecolor{currentstroke}%
\pgfsetdash{}{0pt}%
\pgfsys@defobject{currentmarker}{\pgfqpoint{0.000000in}{-0.048611in}}{\pgfqpoint{0.000000in}{0.000000in}}{%
\pgfpathmoveto{\pgfqpoint{0.000000in}{0.000000in}}%
\pgfpathlineto{\pgfqpoint{0.000000in}{-0.048611in}}%
\pgfusepath{stroke,fill}%
}%
\begin{pgfscope}%
\pgfsys@transformshift{3.075000in}{0.500000in}%
\pgfsys@useobject{currentmarker}{}%
\end{pgfscope}%
\end{pgfscope}%
\begin{pgfscope}%
\definecolor{textcolor}{rgb}{0.000000,0.000000,0.000000}%
\pgfsetstrokecolor{textcolor}%
\pgfsetfillcolor{textcolor}%
\pgftext[x=3.075000in,y=0.402778in,,top]{\color{textcolor}\rmfamily\fontsize{10.000000}{12.000000}\selectfont \(\displaystyle {5.0}\)}%
\end{pgfscope}%
\begin{pgfscope}%
\pgfpathrectangle{\pgfqpoint{0.750000in}{0.500000in}}{\pgfqpoint{4.650000in}{3.020000in}}%
\pgfusepath{clip}%
\pgfsetrectcap%
\pgfsetroundjoin%
\pgfsetlinewidth{0.803000pt}%
\definecolor{currentstroke}{rgb}{0.690196,0.690196,0.690196}%
\pgfsetstrokecolor{currentstroke}%
\pgfsetdash{}{0pt}%
\pgfpathmoveto{\pgfqpoint{3.656250in}{0.500000in}}%
\pgfpathlineto{\pgfqpoint{3.656250in}{3.520000in}}%
\pgfusepath{stroke}%
\end{pgfscope}%
\begin{pgfscope}%
\pgfsetbuttcap%
\pgfsetroundjoin%
\definecolor{currentfill}{rgb}{0.000000,0.000000,0.000000}%
\pgfsetfillcolor{currentfill}%
\pgfsetlinewidth{0.803000pt}%
\definecolor{currentstroke}{rgb}{0.000000,0.000000,0.000000}%
\pgfsetstrokecolor{currentstroke}%
\pgfsetdash{}{0pt}%
\pgfsys@defobject{currentmarker}{\pgfqpoint{0.000000in}{-0.048611in}}{\pgfqpoint{0.000000in}{0.000000in}}{%
\pgfpathmoveto{\pgfqpoint{0.000000in}{0.000000in}}%
\pgfpathlineto{\pgfqpoint{0.000000in}{-0.048611in}}%
\pgfusepath{stroke,fill}%
}%
\begin{pgfscope}%
\pgfsys@transformshift{3.656250in}{0.500000in}%
\pgfsys@useobject{currentmarker}{}%
\end{pgfscope}%
\end{pgfscope}%
\begin{pgfscope}%
\definecolor{textcolor}{rgb}{0.000000,0.000000,0.000000}%
\pgfsetstrokecolor{textcolor}%
\pgfsetfillcolor{textcolor}%
\pgftext[x=3.656250in,y=0.402778in,,top]{\color{textcolor}\rmfamily\fontsize{10.000000}{12.000000}\selectfont \(\displaystyle {7.5}\)}%
\end{pgfscope}%
\begin{pgfscope}%
\pgfpathrectangle{\pgfqpoint{0.750000in}{0.500000in}}{\pgfqpoint{4.650000in}{3.020000in}}%
\pgfusepath{clip}%
\pgfsetrectcap%
\pgfsetroundjoin%
\pgfsetlinewidth{0.803000pt}%
\definecolor{currentstroke}{rgb}{0.690196,0.690196,0.690196}%
\pgfsetstrokecolor{currentstroke}%
\pgfsetdash{}{0pt}%
\pgfpathmoveto{\pgfqpoint{4.237500in}{0.500000in}}%
\pgfpathlineto{\pgfqpoint{4.237500in}{3.520000in}}%
\pgfusepath{stroke}%
\end{pgfscope}%
\begin{pgfscope}%
\pgfsetbuttcap%
\pgfsetroundjoin%
\definecolor{currentfill}{rgb}{0.000000,0.000000,0.000000}%
\pgfsetfillcolor{currentfill}%
\pgfsetlinewidth{0.803000pt}%
\definecolor{currentstroke}{rgb}{0.000000,0.000000,0.000000}%
\pgfsetstrokecolor{currentstroke}%
\pgfsetdash{}{0pt}%
\pgfsys@defobject{currentmarker}{\pgfqpoint{0.000000in}{-0.048611in}}{\pgfqpoint{0.000000in}{0.000000in}}{%
\pgfpathmoveto{\pgfqpoint{0.000000in}{0.000000in}}%
\pgfpathlineto{\pgfqpoint{0.000000in}{-0.048611in}}%
\pgfusepath{stroke,fill}%
}%
\begin{pgfscope}%
\pgfsys@transformshift{4.237500in}{0.500000in}%
\pgfsys@useobject{currentmarker}{}%
\end{pgfscope}%
\end{pgfscope}%
\begin{pgfscope}%
\definecolor{textcolor}{rgb}{0.000000,0.000000,0.000000}%
\pgfsetstrokecolor{textcolor}%
\pgfsetfillcolor{textcolor}%
\pgftext[x=4.237500in,y=0.402778in,,top]{\color{textcolor}\rmfamily\fontsize{10.000000}{12.000000}\selectfont \(\displaystyle {10.0}\)}%
\end{pgfscope}%
\begin{pgfscope}%
\pgfpathrectangle{\pgfqpoint{0.750000in}{0.500000in}}{\pgfqpoint{4.650000in}{3.020000in}}%
\pgfusepath{clip}%
\pgfsetrectcap%
\pgfsetroundjoin%
\pgfsetlinewidth{0.803000pt}%
\definecolor{currentstroke}{rgb}{0.690196,0.690196,0.690196}%
\pgfsetstrokecolor{currentstroke}%
\pgfsetdash{}{0pt}%
\pgfpathmoveto{\pgfqpoint{4.818750in}{0.500000in}}%
\pgfpathlineto{\pgfqpoint{4.818750in}{3.520000in}}%
\pgfusepath{stroke}%
\end{pgfscope}%
\begin{pgfscope}%
\pgfsetbuttcap%
\pgfsetroundjoin%
\definecolor{currentfill}{rgb}{0.000000,0.000000,0.000000}%
\pgfsetfillcolor{currentfill}%
\pgfsetlinewidth{0.803000pt}%
\definecolor{currentstroke}{rgb}{0.000000,0.000000,0.000000}%
\pgfsetstrokecolor{currentstroke}%
\pgfsetdash{}{0pt}%
\pgfsys@defobject{currentmarker}{\pgfqpoint{0.000000in}{-0.048611in}}{\pgfqpoint{0.000000in}{0.000000in}}{%
\pgfpathmoveto{\pgfqpoint{0.000000in}{0.000000in}}%
\pgfpathlineto{\pgfqpoint{0.000000in}{-0.048611in}}%
\pgfusepath{stroke,fill}%
}%
\begin{pgfscope}%
\pgfsys@transformshift{4.818750in}{0.500000in}%
\pgfsys@useobject{currentmarker}{}%
\end{pgfscope}%
\end{pgfscope}%
\begin{pgfscope}%
\definecolor{textcolor}{rgb}{0.000000,0.000000,0.000000}%
\pgfsetstrokecolor{textcolor}%
\pgfsetfillcolor{textcolor}%
\pgftext[x=4.818750in,y=0.402778in,,top]{\color{textcolor}\rmfamily\fontsize{10.000000}{12.000000}\selectfont \(\displaystyle {12.5}\)}%
\end{pgfscope}%
\begin{pgfscope}%
\pgfpathrectangle{\pgfqpoint{0.750000in}{0.500000in}}{\pgfqpoint{4.650000in}{3.020000in}}%
\pgfusepath{clip}%
\pgfsetrectcap%
\pgfsetroundjoin%
\pgfsetlinewidth{0.803000pt}%
\definecolor{currentstroke}{rgb}{0.690196,0.690196,0.690196}%
\pgfsetstrokecolor{currentstroke}%
\pgfsetdash{}{0pt}%
\pgfpathmoveto{\pgfqpoint{5.400000in}{0.500000in}}%
\pgfpathlineto{\pgfqpoint{5.400000in}{3.520000in}}%
\pgfusepath{stroke}%
\end{pgfscope}%
\begin{pgfscope}%
\pgfsetbuttcap%
\pgfsetroundjoin%
\definecolor{currentfill}{rgb}{0.000000,0.000000,0.000000}%
\pgfsetfillcolor{currentfill}%
\pgfsetlinewidth{0.803000pt}%
\definecolor{currentstroke}{rgb}{0.000000,0.000000,0.000000}%
\pgfsetstrokecolor{currentstroke}%
\pgfsetdash{}{0pt}%
\pgfsys@defobject{currentmarker}{\pgfqpoint{0.000000in}{-0.048611in}}{\pgfqpoint{0.000000in}{0.000000in}}{%
\pgfpathmoveto{\pgfqpoint{0.000000in}{0.000000in}}%
\pgfpathlineto{\pgfqpoint{0.000000in}{-0.048611in}}%
\pgfusepath{stroke,fill}%
}%
\begin{pgfscope}%
\pgfsys@transformshift{5.400000in}{0.500000in}%
\pgfsys@useobject{currentmarker}{}%
\end{pgfscope}%
\end{pgfscope}%
\begin{pgfscope}%
\definecolor{textcolor}{rgb}{0.000000,0.000000,0.000000}%
\pgfsetstrokecolor{textcolor}%
\pgfsetfillcolor{textcolor}%
\pgftext[x=5.400000in,y=0.402778in,,top]{\color{textcolor}\rmfamily\fontsize{10.000000}{12.000000}\selectfont \(\displaystyle {15.0}\)}%
\end{pgfscope}%
\begin{pgfscope}%
\definecolor{textcolor}{rgb}{0.000000,0.000000,0.000000}%
\pgfsetstrokecolor{textcolor}%
\pgfsetfillcolor{textcolor}%
\pgftext[x=3.075000in,y=0.223766in,,top]{\color{textcolor}\rmfamily\fontsize{10.000000}{12.000000}\selectfont  \(\displaystyle  \beta \) }%
\end{pgfscope}%
\begin{pgfscope}%
\pgfpathrectangle{\pgfqpoint{0.750000in}{0.500000in}}{\pgfqpoint{4.650000in}{3.020000in}}%
\pgfusepath{clip}%
\pgfsetrectcap%
\pgfsetroundjoin%
\pgfsetlinewidth{0.803000pt}%
\definecolor{currentstroke}{rgb}{0.690196,0.690196,0.690196}%
\pgfsetstrokecolor{currentstroke}%
\pgfsetdash{}{0pt}%
\pgfpathmoveto{\pgfqpoint{0.750000in}{0.605798in}}%
\pgfpathlineto{\pgfqpoint{5.400000in}{0.605798in}}%
\pgfusepath{stroke}%
\end{pgfscope}%
\begin{pgfscope}%
\pgfsetbuttcap%
\pgfsetroundjoin%
\definecolor{currentfill}{rgb}{0.000000,0.000000,0.000000}%
\pgfsetfillcolor{currentfill}%
\pgfsetlinewidth{0.803000pt}%
\definecolor{currentstroke}{rgb}{0.000000,0.000000,0.000000}%
\pgfsetstrokecolor{currentstroke}%
\pgfsetdash{}{0pt}%
\pgfsys@defobject{currentmarker}{\pgfqpoint{-0.048611in}{0.000000in}}{\pgfqpoint{-0.000000in}{0.000000in}}{%
\pgfpathmoveto{\pgfqpoint{-0.000000in}{0.000000in}}%
\pgfpathlineto{\pgfqpoint{-0.048611in}{0.000000in}}%
\pgfusepath{stroke,fill}%
}%
\begin{pgfscope}%
\pgfsys@transformshift{0.750000in}{0.605798in}%
\pgfsys@useobject{currentmarker}{}%
\end{pgfscope}%
\end{pgfscope}%
\begin{pgfscope}%
\definecolor{textcolor}{rgb}{0.000000,0.000000,0.000000}%
\pgfsetstrokecolor{textcolor}%
\pgfsetfillcolor{textcolor}%
\pgftext[x=0.367283in, y=0.557573in, left, base]{\color{textcolor}\rmfamily\fontsize{10.000000}{12.000000}\selectfont \(\displaystyle {\ensuremath{-}0.6}\)}%
\end{pgfscope}%
\begin{pgfscope}%
\pgfpathrectangle{\pgfqpoint{0.750000in}{0.500000in}}{\pgfqpoint{4.650000in}{3.020000in}}%
\pgfusepath{clip}%
\pgfsetrectcap%
\pgfsetroundjoin%
\pgfsetlinewidth{0.803000pt}%
\definecolor{currentstroke}{rgb}{0.690196,0.690196,0.690196}%
\pgfsetstrokecolor{currentstroke}%
\pgfsetdash{}{0pt}%
\pgfpathmoveto{\pgfqpoint{0.750000in}{0.952915in}}%
\pgfpathlineto{\pgfqpoint{5.400000in}{0.952915in}}%
\pgfusepath{stroke}%
\end{pgfscope}%
\begin{pgfscope}%
\pgfsetbuttcap%
\pgfsetroundjoin%
\definecolor{currentfill}{rgb}{0.000000,0.000000,0.000000}%
\pgfsetfillcolor{currentfill}%
\pgfsetlinewidth{0.803000pt}%
\definecolor{currentstroke}{rgb}{0.000000,0.000000,0.000000}%
\pgfsetstrokecolor{currentstroke}%
\pgfsetdash{}{0pt}%
\pgfsys@defobject{currentmarker}{\pgfqpoint{-0.048611in}{0.000000in}}{\pgfqpoint{-0.000000in}{0.000000in}}{%
\pgfpathmoveto{\pgfqpoint{-0.000000in}{0.000000in}}%
\pgfpathlineto{\pgfqpoint{-0.048611in}{0.000000in}}%
\pgfusepath{stroke,fill}%
}%
\begin{pgfscope}%
\pgfsys@transformshift{0.750000in}{0.952915in}%
\pgfsys@useobject{currentmarker}{}%
\end{pgfscope}%
\end{pgfscope}%
\begin{pgfscope}%
\definecolor{textcolor}{rgb}{0.000000,0.000000,0.000000}%
\pgfsetstrokecolor{textcolor}%
\pgfsetfillcolor{textcolor}%
\pgftext[x=0.367283in, y=0.904689in, left, base]{\color{textcolor}\rmfamily\fontsize{10.000000}{12.000000}\selectfont \(\displaystyle {\ensuremath{-}0.4}\)}%
\end{pgfscope}%
\begin{pgfscope}%
\pgfpathrectangle{\pgfqpoint{0.750000in}{0.500000in}}{\pgfqpoint{4.650000in}{3.020000in}}%
\pgfusepath{clip}%
\pgfsetrectcap%
\pgfsetroundjoin%
\pgfsetlinewidth{0.803000pt}%
\definecolor{currentstroke}{rgb}{0.690196,0.690196,0.690196}%
\pgfsetstrokecolor{currentstroke}%
\pgfsetdash{}{0pt}%
\pgfpathmoveto{\pgfqpoint{0.750000in}{1.300031in}}%
\pgfpathlineto{\pgfqpoint{5.400000in}{1.300031in}}%
\pgfusepath{stroke}%
\end{pgfscope}%
\begin{pgfscope}%
\pgfsetbuttcap%
\pgfsetroundjoin%
\definecolor{currentfill}{rgb}{0.000000,0.000000,0.000000}%
\pgfsetfillcolor{currentfill}%
\pgfsetlinewidth{0.803000pt}%
\definecolor{currentstroke}{rgb}{0.000000,0.000000,0.000000}%
\pgfsetstrokecolor{currentstroke}%
\pgfsetdash{}{0pt}%
\pgfsys@defobject{currentmarker}{\pgfqpoint{-0.048611in}{0.000000in}}{\pgfqpoint{-0.000000in}{0.000000in}}{%
\pgfpathmoveto{\pgfqpoint{-0.000000in}{0.000000in}}%
\pgfpathlineto{\pgfqpoint{-0.048611in}{0.000000in}}%
\pgfusepath{stroke,fill}%
}%
\begin{pgfscope}%
\pgfsys@transformshift{0.750000in}{1.300031in}%
\pgfsys@useobject{currentmarker}{}%
\end{pgfscope}%
\end{pgfscope}%
\begin{pgfscope}%
\definecolor{textcolor}{rgb}{0.000000,0.000000,0.000000}%
\pgfsetstrokecolor{textcolor}%
\pgfsetfillcolor{textcolor}%
\pgftext[x=0.367283in, y=1.251806in, left, base]{\color{textcolor}\rmfamily\fontsize{10.000000}{12.000000}\selectfont \(\displaystyle {\ensuremath{-}0.2}\)}%
\end{pgfscope}%
\begin{pgfscope}%
\pgfpathrectangle{\pgfqpoint{0.750000in}{0.500000in}}{\pgfqpoint{4.650000in}{3.020000in}}%
\pgfusepath{clip}%
\pgfsetrectcap%
\pgfsetroundjoin%
\pgfsetlinewidth{0.803000pt}%
\definecolor{currentstroke}{rgb}{0.690196,0.690196,0.690196}%
\pgfsetstrokecolor{currentstroke}%
\pgfsetdash{}{0pt}%
\pgfpathmoveto{\pgfqpoint{0.750000in}{1.647147in}}%
\pgfpathlineto{\pgfqpoint{5.400000in}{1.647147in}}%
\pgfusepath{stroke}%
\end{pgfscope}%
\begin{pgfscope}%
\pgfsetbuttcap%
\pgfsetroundjoin%
\definecolor{currentfill}{rgb}{0.000000,0.000000,0.000000}%
\pgfsetfillcolor{currentfill}%
\pgfsetlinewidth{0.803000pt}%
\definecolor{currentstroke}{rgb}{0.000000,0.000000,0.000000}%
\pgfsetstrokecolor{currentstroke}%
\pgfsetdash{}{0pt}%
\pgfsys@defobject{currentmarker}{\pgfqpoint{-0.048611in}{0.000000in}}{\pgfqpoint{-0.000000in}{0.000000in}}{%
\pgfpathmoveto{\pgfqpoint{-0.000000in}{0.000000in}}%
\pgfpathlineto{\pgfqpoint{-0.048611in}{0.000000in}}%
\pgfusepath{stroke,fill}%
}%
\begin{pgfscope}%
\pgfsys@transformshift{0.750000in}{1.647147in}%
\pgfsys@useobject{currentmarker}{}%
\end{pgfscope}%
\end{pgfscope}%
\begin{pgfscope}%
\definecolor{textcolor}{rgb}{0.000000,0.000000,0.000000}%
\pgfsetstrokecolor{textcolor}%
\pgfsetfillcolor{textcolor}%
\pgftext[x=0.475308in, y=1.598922in, left, base]{\color{textcolor}\rmfamily\fontsize{10.000000}{12.000000}\selectfont \(\displaystyle {0.0}\)}%
\end{pgfscope}%
\begin{pgfscope}%
\pgfpathrectangle{\pgfqpoint{0.750000in}{0.500000in}}{\pgfqpoint{4.650000in}{3.020000in}}%
\pgfusepath{clip}%
\pgfsetrectcap%
\pgfsetroundjoin%
\pgfsetlinewidth{0.803000pt}%
\definecolor{currentstroke}{rgb}{0.690196,0.690196,0.690196}%
\pgfsetstrokecolor{currentstroke}%
\pgfsetdash{}{0pt}%
\pgfpathmoveto{\pgfqpoint{0.750000in}{1.994263in}}%
\pgfpathlineto{\pgfqpoint{5.400000in}{1.994263in}}%
\pgfusepath{stroke}%
\end{pgfscope}%
\begin{pgfscope}%
\pgfsetbuttcap%
\pgfsetroundjoin%
\definecolor{currentfill}{rgb}{0.000000,0.000000,0.000000}%
\pgfsetfillcolor{currentfill}%
\pgfsetlinewidth{0.803000pt}%
\definecolor{currentstroke}{rgb}{0.000000,0.000000,0.000000}%
\pgfsetstrokecolor{currentstroke}%
\pgfsetdash{}{0pt}%
\pgfsys@defobject{currentmarker}{\pgfqpoint{-0.048611in}{0.000000in}}{\pgfqpoint{-0.000000in}{0.000000in}}{%
\pgfpathmoveto{\pgfqpoint{-0.000000in}{0.000000in}}%
\pgfpathlineto{\pgfqpoint{-0.048611in}{0.000000in}}%
\pgfusepath{stroke,fill}%
}%
\begin{pgfscope}%
\pgfsys@transformshift{0.750000in}{1.994263in}%
\pgfsys@useobject{currentmarker}{}%
\end{pgfscope}%
\end{pgfscope}%
\begin{pgfscope}%
\definecolor{textcolor}{rgb}{0.000000,0.000000,0.000000}%
\pgfsetstrokecolor{textcolor}%
\pgfsetfillcolor{textcolor}%
\pgftext[x=0.475308in, y=1.946038in, left, base]{\color{textcolor}\rmfamily\fontsize{10.000000}{12.000000}\selectfont \(\displaystyle {0.2}\)}%
\end{pgfscope}%
\begin{pgfscope}%
\pgfpathrectangle{\pgfqpoint{0.750000in}{0.500000in}}{\pgfqpoint{4.650000in}{3.020000in}}%
\pgfusepath{clip}%
\pgfsetrectcap%
\pgfsetroundjoin%
\pgfsetlinewidth{0.803000pt}%
\definecolor{currentstroke}{rgb}{0.690196,0.690196,0.690196}%
\pgfsetstrokecolor{currentstroke}%
\pgfsetdash{}{0pt}%
\pgfpathmoveto{\pgfqpoint{0.750000in}{2.341380in}}%
\pgfpathlineto{\pgfqpoint{5.400000in}{2.341380in}}%
\pgfusepath{stroke}%
\end{pgfscope}%
\begin{pgfscope}%
\pgfsetbuttcap%
\pgfsetroundjoin%
\definecolor{currentfill}{rgb}{0.000000,0.000000,0.000000}%
\pgfsetfillcolor{currentfill}%
\pgfsetlinewidth{0.803000pt}%
\definecolor{currentstroke}{rgb}{0.000000,0.000000,0.000000}%
\pgfsetstrokecolor{currentstroke}%
\pgfsetdash{}{0pt}%
\pgfsys@defobject{currentmarker}{\pgfqpoint{-0.048611in}{0.000000in}}{\pgfqpoint{-0.000000in}{0.000000in}}{%
\pgfpathmoveto{\pgfqpoint{-0.000000in}{0.000000in}}%
\pgfpathlineto{\pgfqpoint{-0.048611in}{0.000000in}}%
\pgfusepath{stroke,fill}%
}%
\begin{pgfscope}%
\pgfsys@transformshift{0.750000in}{2.341380in}%
\pgfsys@useobject{currentmarker}{}%
\end{pgfscope}%
\end{pgfscope}%
\begin{pgfscope}%
\definecolor{textcolor}{rgb}{0.000000,0.000000,0.000000}%
\pgfsetstrokecolor{textcolor}%
\pgfsetfillcolor{textcolor}%
\pgftext[x=0.475308in, y=2.293154in, left, base]{\color{textcolor}\rmfamily\fontsize{10.000000}{12.000000}\selectfont \(\displaystyle {0.4}\)}%
\end{pgfscope}%
\begin{pgfscope}%
\pgfpathrectangle{\pgfqpoint{0.750000in}{0.500000in}}{\pgfqpoint{4.650000in}{3.020000in}}%
\pgfusepath{clip}%
\pgfsetrectcap%
\pgfsetroundjoin%
\pgfsetlinewidth{0.803000pt}%
\definecolor{currentstroke}{rgb}{0.690196,0.690196,0.690196}%
\pgfsetstrokecolor{currentstroke}%
\pgfsetdash{}{0pt}%
\pgfpathmoveto{\pgfqpoint{0.750000in}{2.688496in}}%
\pgfpathlineto{\pgfqpoint{5.400000in}{2.688496in}}%
\pgfusepath{stroke}%
\end{pgfscope}%
\begin{pgfscope}%
\pgfsetbuttcap%
\pgfsetroundjoin%
\definecolor{currentfill}{rgb}{0.000000,0.000000,0.000000}%
\pgfsetfillcolor{currentfill}%
\pgfsetlinewidth{0.803000pt}%
\definecolor{currentstroke}{rgb}{0.000000,0.000000,0.000000}%
\pgfsetstrokecolor{currentstroke}%
\pgfsetdash{}{0pt}%
\pgfsys@defobject{currentmarker}{\pgfqpoint{-0.048611in}{0.000000in}}{\pgfqpoint{-0.000000in}{0.000000in}}{%
\pgfpathmoveto{\pgfqpoint{-0.000000in}{0.000000in}}%
\pgfpathlineto{\pgfqpoint{-0.048611in}{0.000000in}}%
\pgfusepath{stroke,fill}%
}%
\begin{pgfscope}%
\pgfsys@transformshift{0.750000in}{2.688496in}%
\pgfsys@useobject{currentmarker}{}%
\end{pgfscope}%
\end{pgfscope}%
\begin{pgfscope}%
\definecolor{textcolor}{rgb}{0.000000,0.000000,0.000000}%
\pgfsetstrokecolor{textcolor}%
\pgfsetfillcolor{textcolor}%
\pgftext[x=0.475308in, y=2.640271in, left, base]{\color{textcolor}\rmfamily\fontsize{10.000000}{12.000000}\selectfont \(\displaystyle {0.6}\)}%
\end{pgfscope}%
\begin{pgfscope}%
\pgfpathrectangle{\pgfqpoint{0.750000in}{0.500000in}}{\pgfqpoint{4.650000in}{3.020000in}}%
\pgfusepath{clip}%
\pgfsetrectcap%
\pgfsetroundjoin%
\pgfsetlinewidth{0.803000pt}%
\definecolor{currentstroke}{rgb}{0.690196,0.690196,0.690196}%
\pgfsetstrokecolor{currentstroke}%
\pgfsetdash{}{0pt}%
\pgfpathmoveto{\pgfqpoint{0.750000in}{3.035612in}}%
\pgfpathlineto{\pgfqpoint{5.400000in}{3.035612in}}%
\pgfusepath{stroke}%
\end{pgfscope}%
\begin{pgfscope}%
\pgfsetbuttcap%
\pgfsetroundjoin%
\definecolor{currentfill}{rgb}{0.000000,0.000000,0.000000}%
\pgfsetfillcolor{currentfill}%
\pgfsetlinewidth{0.803000pt}%
\definecolor{currentstroke}{rgb}{0.000000,0.000000,0.000000}%
\pgfsetstrokecolor{currentstroke}%
\pgfsetdash{}{0pt}%
\pgfsys@defobject{currentmarker}{\pgfqpoint{-0.048611in}{0.000000in}}{\pgfqpoint{-0.000000in}{0.000000in}}{%
\pgfpathmoveto{\pgfqpoint{-0.000000in}{0.000000in}}%
\pgfpathlineto{\pgfqpoint{-0.048611in}{0.000000in}}%
\pgfusepath{stroke,fill}%
}%
\begin{pgfscope}%
\pgfsys@transformshift{0.750000in}{3.035612in}%
\pgfsys@useobject{currentmarker}{}%
\end{pgfscope}%
\end{pgfscope}%
\begin{pgfscope}%
\definecolor{textcolor}{rgb}{0.000000,0.000000,0.000000}%
\pgfsetstrokecolor{textcolor}%
\pgfsetfillcolor{textcolor}%
\pgftext[x=0.475308in, y=2.987387in, left, base]{\color{textcolor}\rmfamily\fontsize{10.000000}{12.000000}\selectfont \(\displaystyle {0.8}\)}%
\end{pgfscope}%
\begin{pgfscope}%
\pgfpathrectangle{\pgfqpoint{0.750000in}{0.500000in}}{\pgfqpoint{4.650000in}{3.020000in}}%
\pgfusepath{clip}%
\pgfsetrectcap%
\pgfsetroundjoin%
\pgfsetlinewidth{0.803000pt}%
\definecolor{currentstroke}{rgb}{0.690196,0.690196,0.690196}%
\pgfsetstrokecolor{currentstroke}%
\pgfsetdash{}{0pt}%
\pgfpathmoveto{\pgfqpoint{0.750000in}{3.382728in}}%
\pgfpathlineto{\pgfqpoint{5.400000in}{3.382728in}}%
\pgfusepath{stroke}%
\end{pgfscope}%
\begin{pgfscope}%
\pgfsetbuttcap%
\pgfsetroundjoin%
\definecolor{currentfill}{rgb}{0.000000,0.000000,0.000000}%
\pgfsetfillcolor{currentfill}%
\pgfsetlinewidth{0.803000pt}%
\definecolor{currentstroke}{rgb}{0.000000,0.000000,0.000000}%
\pgfsetstrokecolor{currentstroke}%
\pgfsetdash{}{0pt}%
\pgfsys@defobject{currentmarker}{\pgfqpoint{-0.048611in}{0.000000in}}{\pgfqpoint{-0.000000in}{0.000000in}}{%
\pgfpathmoveto{\pgfqpoint{-0.000000in}{0.000000in}}%
\pgfpathlineto{\pgfqpoint{-0.048611in}{0.000000in}}%
\pgfusepath{stroke,fill}%
}%
\begin{pgfscope}%
\pgfsys@transformshift{0.750000in}{3.382728in}%
\pgfsys@useobject{currentmarker}{}%
\end{pgfscope}%
\end{pgfscope}%
\begin{pgfscope}%
\definecolor{textcolor}{rgb}{0.000000,0.000000,0.000000}%
\pgfsetstrokecolor{textcolor}%
\pgfsetfillcolor{textcolor}%
\pgftext[x=0.475308in, y=3.334503in, left, base]{\color{textcolor}\rmfamily\fontsize{10.000000}{12.000000}\selectfont \(\displaystyle {1.0}\)}%
\end{pgfscope}%
\begin{pgfscope}%
\definecolor{textcolor}{rgb}{0.000000,0.000000,0.000000}%
\pgfsetstrokecolor{textcolor}%
\pgfsetfillcolor{textcolor}%
\pgftext[x=0.311727in,y=2.010000in,,bottom,rotate=90.000000]{\color{textcolor}\rmfamily\fontsize{10.000000}{12.000000}\selectfont \(\displaystyle J_n(\beta)\)}%
\end{pgfscope}%
\begin{pgfscope}%
\pgfpathrectangle{\pgfqpoint{0.750000in}{0.500000in}}{\pgfqpoint{4.650000in}{3.020000in}}%
\pgfusepath{clip}%
\pgfsetrectcap%
\pgfsetroundjoin%
\pgfsetlinewidth{1.505625pt}%
\definecolor{currentstroke}{rgb}{0.000000,0.000000,0.000000}%
\pgfsetstrokecolor{currentstroke}%
\pgfsetdash{}{0pt}%
\pgfpathmoveto{\pgfqpoint{2.145000in}{2.975210in}}%
\pgfpathlineto{\pgfqpoint{2.145000in}{0.883404in}}%
\pgfusepath{stroke}%
\end{pgfscope}%
\begin{pgfscope}%
\pgfpathrectangle{\pgfqpoint{0.750000in}{0.500000in}}{\pgfqpoint{4.650000in}{3.020000in}}%
\pgfusepath{clip}%
\pgfsetrectcap%
\pgfsetroundjoin%
\pgfsetlinewidth{1.505625pt}%
\definecolor{currentstroke}{rgb}{0.121569,0.466667,0.705882}%
\pgfsetstrokecolor{currentstroke}%
\pgfsetdash{}{0pt}%
\pgfpathmoveto{\pgfqpoint{0.749767in}{1.078413in}}%
\pgfpathlineto{\pgfqpoint{0.769208in}{1.096666in}}%
\pgfpathlineto{\pgfqpoint{0.789672in}{1.120234in}}%
\pgfpathlineto{\pgfqpoint{0.811160in}{1.149687in}}%
\pgfpathlineto{\pgfqpoint{0.834693in}{1.187305in}}%
\pgfpathlineto{\pgfqpoint{0.860274in}{1.234274in}}%
\pgfpathlineto{\pgfqpoint{0.887900in}{1.291687in}}%
\pgfpathlineto{\pgfqpoint{0.918596in}{1.362946in}}%
\pgfpathlineto{\pgfqpoint{0.952362in}{1.449362in}}%
\pgfpathlineto{\pgfqpoint{0.991244in}{1.557556in}}%
\pgfpathlineto{\pgfqpoint{1.040358in}{1.704000in}}%
\pgfpathlineto{\pgfqpoint{1.220442in}{2.250681in}}%
\pgfpathlineto{\pgfqpoint{1.257277in}{2.347717in}}%
\pgfpathlineto{\pgfqpoint{1.290020in}{2.425499in}}%
\pgfpathlineto{\pgfqpoint{1.318669in}{2.485878in}}%
\pgfpathlineto{\pgfqpoint{1.344249in}{2.533008in}}%
\pgfpathlineto{\pgfqpoint{1.367783in}{2.570238in}}%
\pgfpathlineto{\pgfqpoint{1.389270in}{2.598777in}}%
\pgfpathlineto{\pgfqpoint{1.409734in}{2.620884in}}%
\pgfpathlineto{\pgfqpoint{1.428152in}{2.636398in}}%
\pgfpathlineto{\pgfqpoint{1.445547in}{2.647135in}}%
\pgfpathlineto{\pgfqpoint{1.461918in}{2.653697in}}%
\pgfpathlineto{\pgfqpoint{1.477266in}{2.656684in}}%
\pgfpathlineto{\pgfqpoint{1.492614in}{2.656579in}}%
\pgfpathlineto{\pgfqpoint{1.507962in}{2.653365in}}%
\pgfpathlineto{\pgfqpoint{1.523310in}{2.647033in}}%
\pgfpathlineto{\pgfqpoint{1.539681in}{2.636850in}}%
\pgfpathlineto{\pgfqpoint{1.557076in}{2.622171in}}%
\pgfpathlineto{\pgfqpoint{1.575494in}{2.602340in}}%
\pgfpathlineto{\pgfqpoint{1.594934in}{2.576698in}}%
\pgfpathlineto{\pgfqpoint{1.615399in}{2.544600in}}%
\pgfpathlineto{\pgfqpoint{1.637909in}{2.503432in}}%
\pgfpathlineto{\pgfqpoint{1.662466in}{2.451816in}}%
\pgfpathlineto{\pgfqpoint{1.689069in}{2.388452in}}%
\pgfpathlineto{\pgfqpoint{1.718742in}{2.309344in}}%
\pgfpathlineto{\pgfqpoint{1.751485in}{2.212809in}}%
\pgfpathlineto{\pgfqpoint{1.788320in}{2.094308in}}%
\pgfpathlineto{\pgfqpoint{1.832318in}{1.941994in}}%
\pgfpathlineto{\pgfqpoint{1.893710in}{1.717221in}}%
\pgfpathlineto{\pgfqpoint{2.005240in}{1.307842in}}%
\pgfpathlineto{\pgfqpoint{2.050261in}{1.155202in}}%
\pgfpathlineto{\pgfqpoint{2.088119in}{1.037311in}}%
\pgfpathlineto{\pgfqpoint{2.120862in}{0.944956in}}%
\pgfpathlineto{\pgfqpoint{2.150535in}{0.870131in}}%
\pgfpathlineto{\pgfqpoint{2.177138in}{0.810959in}}%
\pgfpathlineto{\pgfqpoint{2.201695in}{0.763466in}}%
\pgfpathlineto{\pgfqpoint{2.224205in}{0.726259in}}%
\pgfpathlineto{\pgfqpoint{2.245693in}{0.696606in}}%
\pgfpathlineto{\pgfqpoint{2.265134in}{0.674846in}}%
\pgfpathlineto{\pgfqpoint{2.283551in}{0.658755in}}%
\pgfpathlineto{\pgfqpoint{2.300946in}{0.647647in}}%
\pgfpathlineto{\pgfqpoint{2.317317in}{0.640843in}}%
\pgfpathlineto{\pgfqpoint{2.332665in}{0.637685in}}%
\pgfpathlineto{\pgfqpoint{2.348013in}{0.637637in}}%
\pgfpathlineto{\pgfqpoint{2.363361in}{0.640680in}}%
\pgfpathlineto{\pgfqpoint{2.378709in}{0.646786in}}%
\pgfpathlineto{\pgfqpoint{2.395081in}{0.656630in}}%
\pgfpathlineto{\pgfqpoint{2.412475in}{0.670789in}}%
\pgfpathlineto{\pgfqpoint{2.430893in}{0.689838in}}%
\pgfpathlineto{\pgfqpoint{2.451357in}{0.715745in}}%
\pgfpathlineto{\pgfqpoint{2.472844in}{0.748102in}}%
\pgfpathlineto{\pgfqpoint{2.496378in}{0.789285in}}%
\pgfpathlineto{\pgfqpoint{2.521958in}{0.840404in}}%
\pgfpathlineto{\pgfqpoint{2.550608in}{0.904826in}}%
\pgfpathlineto{\pgfqpoint{2.582327in}{0.983948in}}%
\pgfpathlineto{\pgfqpoint{2.619162in}{1.084432in}}%
\pgfpathlineto{\pgfqpoint{2.664183in}{1.216714in}}%
\pgfpathlineto{\pgfqpoint{2.733761in}{1.432234in}}%
\pgfpathlineto{\pgfqpoint{2.816641in}{1.686727in}}%
\pgfpathlineto{\pgfqpoint{2.862685in}{1.818042in}}%
\pgfpathlineto{\pgfqpoint{2.900544in}{1.916920in}}%
\pgfpathlineto{\pgfqpoint{2.933286in}{1.994166in}}%
\pgfpathlineto{\pgfqpoint{2.962959in}{2.056550in}}%
\pgfpathlineto{\pgfqpoint{2.989562in}{2.105711in}}%
\pgfpathlineto{\pgfqpoint{3.014119in}{2.145020in}}%
\pgfpathlineto{\pgfqpoint{3.037653in}{2.176960in}}%
\pgfpathlineto{\pgfqpoint{3.059140in}{2.201062in}}%
\pgfpathlineto{\pgfqpoint{3.079604in}{2.219425in}}%
\pgfpathlineto{\pgfqpoint{3.099045in}{2.232668in}}%
\pgfpathlineto{\pgfqpoint{3.117463in}{2.241415in}}%
\pgfpathlineto{\pgfqpoint{3.134857in}{2.246285in}}%
\pgfpathlineto{\pgfqpoint{3.152252in}{2.247877in}}%
\pgfpathlineto{\pgfqpoint{3.169646in}{2.246227in}}%
\pgfpathlineto{\pgfqpoint{3.187041in}{2.241385in}}%
\pgfpathlineto{\pgfqpoint{3.205459in}{2.232852in}}%
\pgfpathlineto{\pgfqpoint{3.223876in}{2.220910in}}%
\pgfpathlineto{\pgfqpoint{3.244340in}{2.203783in}}%
\pgfpathlineto{\pgfqpoint{3.265828in}{2.181621in}}%
\pgfpathlineto{\pgfqpoint{3.289361in}{2.152718in}}%
\pgfpathlineto{\pgfqpoint{3.314941in}{2.116217in}}%
\pgfpathlineto{\pgfqpoint{3.343591in}{2.069653in}}%
\pgfpathlineto{\pgfqpoint{3.375311in}{2.012008in}}%
\pgfpathlineto{\pgfqpoint{3.412146in}{1.938491in}}%
\pgfpathlineto{\pgfqpoint{3.459213in}{1.837175in}}%
\pgfpathlineto{\pgfqpoint{3.628042in}{1.466702in}}%
\pgfpathlineto{\pgfqpoint{3.665901in}{1.394777in}}%
\pgfpathlineto{\pgfqpoint{3.698643in}{1.338988in}}%
\pgfpathlineto{\pgfqpoint{3.728316in}{1.294417in}}%
\pgfpathlineto{\pgfqpoint{3.754919in}{1.259815in}}%
\pgfpathlineto{\pgfqpoint{3.779476in}{1.232700in}}%
\pgfpathlineto{\pgfqpoint{3.803010in}{1.211285in}}%
\pgfpathlineto{\pgfqpoint{3.824497in}{1.195774in}}%
\pgfpathlineto{\pgfqpoint{3.844961in}{1.184672in}}%
\pgfpathlineto{\pgfqpoint{3.864402in}{1.177483in}}%
\pgfpathlineto{\pgfqpoint{3.883843in}{1.173586in}}%
\pgfpathlineto{\pgfqpoint{3.902261in}{1.172924in}}%
\pgfpathlineto{\pgfqpoint{3.921702in}{1.175399in}}%
\pgfpathlineto{\pgfqpoint{3.941143in}{1.181088in}}%
\pgfpathlineto{\pgfqpoint{3.961607in}{1.190471in}}%
\pgfpathlineto{\pgfqpoint{3.983094in}{1.203958in}}%
\pgfpathlineto{\pgfqpoint{4.005605in}{1.221921in}}%
\pgfpathlineto{\pgfqpoint{4.029138in}{1.244674in}}%
\pgfpathlineto{\pgfqpoint{4.054718in}{1.273703in}}%
\pgfpathlineto{\pgfqpoint{4.083368in}{1.311052in}}%
\pgfpathlineto{\pgfqpoint{4.115088in}{1.357642in}}%
\pgfpathlineto{\pgfqpoint{4.151923in}{1.417475in}}%
\pgfpathlineto{\pgfqpoint{4.196944in}{1.496779in}}%
\pgfpathlineto{\pgfqpoint{4.274707in}{1.641320in}}%
\pgfpathlineto{\pgfqpoint{4.342239in}{1.764103in}}%
\pgfpathlineto{\pgfqpoint{4.385214in}{1.836076in}}%
\pgfpathlineto{\pgfqpoint{4.421026in}{1.890363in}}%
\pgfpathlineto{\pgfqpoint{4.452745in}{1.933103in}}%
\pgfpathlineto{\pgfqpoint{4.481395in}{1.966767in}}%
\pgfpathlineto{\pgfqpoint{4.507998in}{1.993420in}}%
\pgfpathlineto{\pgfqpoint{4.532555in}{2.013820in}}%
\pgfpathlineto{\pgfqpoint{4.556089in}{2.029407in}}%
\pgfpathlineto{\pgfqpoint{4.578599in}{2.040571in}}%
\pgfpathlineto{\pgfqpoint{4.600087in}{2.047743in}}%
\pgfpathlineto{\pgfqpoint{4.620551in}{2.051378in}}%
\pgfpathlineto{\pgfqpoint{4.641015in}{2.051891in}}%
\pgfpathlineto{\pgfqpoint{4.661479in}{2.049301in}}%
\pgfpathlineto{\pgfqpoint{4.681943in}{2.043650in}}%
\pgfpathlineto{\pgfqpoint{4.703430in}{2.034494in}}%
\pgfpathlineto{\pgfqpoint{4.725941in}{2.021472in}}%
\pgfpathlineto{\pgfqpoint{4.749474in}{2.004264in}}%
\pgfpathlineto{\pgfqpoint{4.775055in}{1.981626in}}%
\pgfpathlineto{\pgfqpoint{4.802681in}{1.952920in}}%
\pgfpathlineto{\pgfqpoint{4.833377in}{1.916366in}}%
\pgfpathlineto{\pgfqpoint{4.868166in}{1.869838in}}%
\pgfpathlineto{\pgfqpoint{4.909094in}{1.809589in}}%
\pgfpathlineto{\pgfqpoint{4.963324in}{1.723717in}}%
\pgfpathlineto{\pgfqpoint{5.088155in}{1.524064in}}%
\pgfpathlineto{\pgfqpoint{5.130107in}{1.463808in}}%
\pgfpathlineto{\pgfqpoint{5.165919in}{1.417576in}}%
\pgfpathlineto{\pgfqpoint{5.197638in}{1.381514in}}%
\pgfpathlineto{\pgfqpoint{5.226288in}{1.353425in}}%
\pgfpathlineto{\pgfqpoint{5.252891in}{1.331504in}}%
\pgfpathlineto{\pgfqpoint{5.277448in}{1.315051in}}%
\pgfpathlineto{\pgfqpoint{5.300982in}{1.302838in}}%
\pgfpathlineto{\pgfqpoint{5.323492in}{1.294505in}}%
\pgfpathlineto{\pgfqpoint{5.346003in}{1.289505in}}%
\pgfpathlineto{\pgfqpoint{5.367490in}{1.287862in}}%
\pgfpathlineto{\pgfqpoint{5.388977in}{1.289273in}}%
\pgfpathlineto{\pgfqpoint{5.400233in}{1.291221in}}%
\pgfpathlineto{\pgfqpoint{5.400233in}{1.291221in}}%
\pgfusepath{stroke}%
\end{pgfscope}%
\begin{pgfscope}%
\pgfpathrectangle{\pgfqpoint{0.750000in}{0.500000in}}{\pgfqpoint{4.650000in}{3.020000in}}%
\pgfusepath{clip}%
\pgfsetrectcap%
\pgfsetroundjoin%
\pgfsetlinewidth{1.505625pt}%
\definecolor{currentstroke}{rgb}{1.000000,0.498039,0.054902}%
\pgfsetstrokecolor{currentstroke}%
\pgfsetdash{}{0pt}%
\pgfpathmoveto{\pgfqpoint{0.749767in}{1.339482in}}%
\pgfpathlineto{\pgfqpoint{0.793765in}{1.235984in}}%
\pgfpathlineto{\pgfqpoint{0.828554in}{1.161626in}}%
\pgfpathlineto{\pgfqpoint{0.859250in}{1.103102in}}%
\pgfpathlineto{\pgfqpoint{0.885854in}{1.058715in}}%
\pgfpathlineto{\pgfqpoint{0.910411in}{1.023613in}}%
\pgfpathlineto{\pgfqpoint{0.931898in}{0.997940in}}%
\pgfpathlineto{\pgfqpoint{0.952362in}{0.978165in}}%
\pgfpathlineto{\pgfqpoint{0.970780in}{0.964473in}}%
\pgfpathlineto{\pgfqpoint{0.988174in}{0.955261in}}%
\pgfpathlineto{\pgfqpoint{1.004545in}{0.950000in}}%
\pgfpathlineto{\pgfqpoint{1.019893in}{0.948145in}}%
\pgfpathlineto{\pgfqpoint{1.035242in}{0.949329in}}%
\pgfpathlineto{\pgfqpoint{1.050590in}{0.953602in}}%
\pgfpathlineto{\pgfqpoint{1.065938in}{0.960999in}}%
\pgfpathlineto{\pgfqpoint{1.082309in}{0.972364in}}%
\pgfpathlineto{\pgfqpoint{1.099703in}{0.988388in}}%
\pgfpathlineto{\pgfqpoint{1.118121in}{1.009794in}}%
\pgfpathlineto{\pgfqpoint{1.137562in}{1.037320in}}%
\pgfpathlineto{\pgfqpoint{1.158026in}{1.071712in}}%
\pgfpathlineto{\pgfqpoint{1.180537in}{1.115847in}}%
\pgfpathlineto{\pgfqpoint{1.204070in}{1.168863in}}%
\pgfpathlineto{\pgfqpoint{1.229650in}{1.234159in}}%
\pgfpathlineto{\pgfqpoint{1.257277in}{1.313181in}}%
\pgfpathlineto{\pgfqpoint{1.287973in}{1.410590in}}%
\pgfpathlineto{\pgfqpoint{1.321739in}{1.528245in}}%
\pgfpathlineto{\pgfqpoint{1.360621in}{1.675394in}}%
\pgfpathlineto{\pgfqpoint{1.406665in}{1.862272in}}%
\pgfpathlineto{\pgfqpoint{1.472150in}{2.142340in}}%
\pgfpathlineto{\pgfqpoint{1.575494in}{2.584052in}}%
\pgfpathlineto{\pgfqpoint{1.622561in}{2.770783in}}%
\pgfpathlineto{\pgfqpoint{1.660420in}{2.908946in}}%
\pgfpathlineto{\pgfqpoint{1.694185in}{3.020736in}}%
\pgfpathlineto{\pgfqpoint{1.723858in}{3.108631in}}%
\pgfpathlineto{\pgfqpoint{1.750462in}{3.178287in}}%
\pgfpathlineto{\pgfqpoint{1.775019in}{3.234297in}}%
\pgfpathlineto{\pgfqpoint{1.797529in}{3.278239in}}%
\pgfpathlineto{\pgfqpoint{1.817993in}{3.311774in}}%
\pgfpathlineto{\pgfqpoint{1.837434in}{3.337792in}}%
\pgfpathlineto{\pgfqpoint{1.854828in}{3.356134in}}%
\pgfpathlineto{\pgfqpoint{1.871200in}{3.369064in}}%
\pgfpathlineto{\pgfqpoint{1.886548in}{3.377326in}}%
\pgfpathlineto{\pgfqpoint{1.900873in}{3.381643in}}%
\pgfpathlineto{\pgfqpoint{1.915198in}{3.382670in}}%
\pgfpathlineto{\pgfqpoint{1.929522in}{3.380403in}}%
\pgfpathlineto{\pgfqpoint{1.943847in}{3.374850in}}%
\pgfpathlineto{\pgfqpoint{1.958172in}{3.366025in}}%
\pgfpathlineto{\pgfqpoint{1.973520in}{3.352970in}}%
\pgfpathlineto{\pgfqpoint{1.989891in}{3.334985in}}%
\pgfpathlineto{\pgfqpoint{2.008309in}{3.309826in}}%
\pgfpathlineto{\pgfqpoint{2.027750in}{3.277739in}}%
\pgfpathlineto{\pgfqpoint{2.049237in}{3.235865in}}%
\pgfpathlineto{\pgfqpoint{2.071748in}{3.185063in}}%
\pgfpathlineto{\pgfqpoint{2.096305in}{3.121962in}}%
\pgfpathlineto{\pgfqpoint{2.123931in}{3.042033in}}%
\pgfpathlineto{\pgfqpoint{2.154627in}{2.943108in}}%
\pgfpathlineto{\pgfqpoint{2.188393in}{2.823469in}}%
\pgfpathlineto{\pgfqpoint{2.227275in}{2.674016in}}%
\pgfpathlineto{\pgfqpoint{2.275366in}{2.476254in}}%
\pgfpathlineto{\pgfqpoint{2.353129in}{2.141129in}}%
\pgfpathlineto{\pgfqpoint{2.431916in}{1.805985in}}%
\pgfpathlineto{\pgfqpoint{2.478983in}{1.618810in}}%
\pgfpathlineto{\pgfqpoint{2.517865in}{1.476097in}}%
\pgfpathlineto{\pgfqpoint{2.552654in}{1.359792in}}%
\pgfpathlineto{\pgfqpoint{2.583350in}{1.267430in}}%
\pgfpathlineto{\pgfqpoint{2.612000in}{1.190768in}}%
\pgfpathlineto{\pgfqpoint{2.637580in}{1.130636in}}%
\pgfpathlineto{\pgfqpoint{2.662137in}{1.080643in}}%
\pgfpathlineto{\pgfqpoint{2.684647in}{1.041684in}}%
\pgfpathlineto{\pgfqpoint{2.705112in}{1.012087in}}%
\pgfpathlineto{\pgfqpoint{2.724552in}{0.989165in}}%
\pgfpathlineto{\pgfqpoint{2.742970in}{0.972140in}}%
\pgfpathlineto{\pgfqpoint{2.760365in}{0.960249in}}%
\pgfpathlineto{\pgfqpoint{2.776736in}{0.952754in}}%
\pgfpathlineto{\pgfqpoint{2.792084in}{0.948953in}}%
\pgfpathlineto{\pgfqpoint{2.807432in}{0.948232in}}%
\pgfpathlineto{\pgfqpoint{2.822780in}{0.950542in}}%
\pgfpathlineto{\pgfqpoint{2.839151in}{0.956279in}}%
\pgfpathlineto{\pgfqpoint{2.855523in}{0.965310in}}%
\pgfpathlineto{\pgfqpoint{2.872917in}{0.978403in}}%
\pgfpathlineto{\pgfqpoint{2.892358in}{0.997139in}}%
\pgfpathlineto{\pgfqpoint{2.912822in}{1.021319in}}%
\pgfpathlineto{\pgfqpoint{2.935333in}{1.052879in}}%
\pgfpathlineto{\pgfqpoint{2.959889in}{1.092791in}}%
\pgfpathlineto{\pgfqpoint{2.987516in}{1.143856in}}%
\pgfpathlineto{\pgfqpoint{3.018212in}{1.207256in}}%
\pgfpathlineto{\pgfqpoint{3.054024in}{1.288484in}}%
\pgfpathlineto{\pgfqpoint{3.099045in}{1.398645in}}%
\pgfpathlineto{\pgfqpoint{3.187041in}{1.624526in}}%
\pgfpathlineto{\pgfqpoint{3.245364in}{1.769454in}}%
\pgfpathlineto{\pgfqpoint{3.287315in}{1.865673in}}%
\pgfpathlineto{\pgfqpoint{3.322104in}{1.938077in}}%
\pgfpathlineto{\pgfqpoint{3.352800in}{1.995204in}}%
\pgfpathlineto{\pgfqpoint{3.380427in}{2.040484in}}%
\pgfpathlineto{\pgfqpoint{3.406007in}{2.076766in}}%
\pgfpathlineto{\pgfqpoint{3.429540in}{2.105059in}}%
\pgfpathlineto{\pgfqpoint{3.452051in}{2.127364in}}%
\pgfpathlineto{\pgfqpoint{3.472515in}{2.143488in}}%
\pgfpathlineto{\pgfqpoint{3.491956in}{2.155075in}}%
\pgfpathlineto{\pgfqpoint{3.510374in}{2.162668in}}%
\pgfpathlineto{\pgfqpoint{3.528791in}{2.166960in}}%
\pgfpathlineto{\pgfqpoint{3.547209in}{2.167961in}}%
\pgfpathlineto{\pgfqpoint{3.565627in}{2.165702in}}%
\pgfpathlineto{\pgfqpoint{3.584044in}{2.160234in}}%
\pgfpathlineto{\pgfqpoint{3.603485in}{2.151057in}}%
\pgfpathlineto{\pgfqpoint{3.623949in}{2.137731in}}%
\pgfpathlineto{\pgfqpoint{3.645437in}{2.119848in}}%
\pgfpathlineto{\pgfqpoint{3.667947in}{2.097053in}}%
\pgfpathlineto{\pgfqpoint{3.692504in}{2.067751in}}%
\pgfpathlineto{\pgfqpoint{3.720131in}{2.029721in}}%
\pgfpathlineto{\pgfqpoint{3.750827in}{1.981892in}}%
\pgfpathlineto{\pgfqpoint{3.785616in}{1.921709in}}%
\pgfpathlineto{\pgfqpoint{3.827567in}{1.842659in}}%
\pgfpathlineto{\pgfqpoint{3.888959in}{1.719509in}}%
\pgfpathlineto{\pgfqpoint{3.981048in}{1.535237in}}%
\pgfpathlineto{\pgfqpoint{4.025046in}{1.454459in}}%
\pgfpathlineto{\pgfqpoint{4.060858in}{1.394852in}}%
\pgfpathlineto{\pgfqpoint{4.092577in}{1.347790in}}%
\pgfpathlineto{\pgfqpoint{4.121227in}{1.310596in}}%
\pgfpathlineto{\pgfqpoint{4.147830in}{1.281019in}}%
\pgfpathlineto{\pgfqpoint{4.172387in}{1.258247in}}%
\pgfpathlineto{\pgfqpoint{4.194897in}{1.241371in}}%
\pgfpathlineto{\pgfqpoint{4.216385in}{1.228944in}}%
\pgfpathlineto{\pgfqpoint{4.236849in}{1.220519in}}%
\pgfpathlineto{\pgfqpoint{4.257313in}{1.215455in}}%
\pgfpathlineto{\pgfqpoint{4.276754in}{1.213769in}}%
\pgfpathlineto{\pgfqpoint{4.296195in}{1.215112in}}%
\pgfpathlineto{\pgfqpoint{4.315636in}{1.219450in}}%
\pgfpathlineto{\pgfqpoint{4.336100in}{1.227194in}}%
\pgfpathlineto{\pgfqpoint{4.357587in}{1.238737in}}%
\pgfpathlineto{\pgfqpoint{4.380098in}{1.254442in}}%
\pgfpathlineto{\pgfqpoint{4.403631in}{1.274620in}}%
\pgfpathlineto{\pgfqpoint{4.429211in}{1.300634in}}%
\pgfpathlineto{\pgfqpoint{4.457861in}{1.334388in}}%
\pgfpathlineto{\pgfqpoint{4.489580in}{1.376794in}}%
\pgfpathlineto{\pgfqpoint{4.525393in}{1.430003in}}%
\pgfpathlineto{\pgfqpoint{4.569390in}{1.501229in}}%
\pgfpathlineto{\pgfqpoint{4.635899in}{1.615609in}}%
\pgfpathlineto{\pgfqpoint{4.717755in}{1.755096in}}%
\pgfpathlineto{\pgfqpoint{4.761753in}{1.824005in}}%
\pgfpathlineto{\pgfqpoint{4.798588in}{1.876155in}}%
\pgfpathlineto{\pgfqpoint{4.831331in}{1.917204in}}%
\pgfpathlineto{\pgfqpoint{4.859980in}{1.948407in}}%
\pgfpathlineto{\pgfqpoint{4.886584in}{1.973048in}}%
\pgfpathlineto{\pgfqpoint{4.911141in}{1.991838in}}%
\pgfpathlineto{\pgfqpoint{4.934674in}{2.006113in}}%
\pgfpathlineto{\pgfqpoint{4.957185in}{2.016240in}}%
\pgfpathlineto{\pgfqpoint{4.978672in}{2.022624in}}%
\pgfpathlineto{\pgfqpoint{5.000160in}{2.025770in}}%
\pgfpathlineto{\pgfqpoint{5.021647in}{2.025674in}}%
\pgfpathlineto{\pgfqpoint{5.043134in}{2.022358in}}%
\pgfpathlineto{\pgfqpoint{5.064621in}{2.015874in}}%
\pgfpathlineto{\pgfqpoint{5.087132in}{2.005767in}}%
\pgfpathlineto{\pgfqpoint{5.110666in}{1.991699in}}%
\pgfpathlineto{\pgfqpoint{5.135223in}{1.973386in}}%
\pgfpathlineto{\pgfqpoint{5.161826in}{1.949624in}}%
\pgfpathlineto{\pgfqpoint{5.190476in}{1.919856in}}%
\pgfpathlineto{\pgfqpoint{5.222195in}{1.882420in}}%
\pgfpathlineto{\pgfqpoint{5.259030in}{1.833986in}}%
\pgfpathlineto{\pgfqpoint{5.304051in}{1.769322in}}%
\pgfpathlineto{\pgfqpoint{5.373629in}{1.663061in}}%
\pgfpathlineto{\pgfqpoint{5.400233in}{1.622103in}}%
\pgfpathlineto{\pgfqpoint{5.400233in}{1.622103in}}%
\pgfusepath{stroke}%
\end{pgfscope}%
\begin{pgfscope}%
\pgfpathrectangle{\pgfqpoint{0.750000in}{0.500000in}}{\pgfqpoint{4.650000in}{3.020000in}}%
\pgfusepath{clip}%
\pgfsetrectcap%
\pgfsetroundjoin%
\pgfsetlinewidth{1.505625pt}%
\definecolor{currentstroke}{rgb}{0.172549,0.627451,0.172549}%
\pgfsetstrokecolor{currentstroke}%
\pgfsetdash{}{0pt}%
\pgfpathmoveto{\pgfqpoint{0.749767in}{2.215882in}}%
\pgfpathlineto{\pgfqpoint{0.769208in}{2.197629in}}%
\pgfpathlineto{\pgfqpoint{0.789672in}{2.174060in}}%
\pgfpathlineto{\pgfqpoint{0.811160in}{2.144607in}}%
\pgfpathlineto{\pgfqpoint{0.834693in}{2.106989in}}%
\pgfpathlineto{\pgfqpoint{0.860274in}{2.060020in}}%
\pgfpathlineto{\pgfqpoint{0.887900in}{2.002607in}}%
\pgfpathlineto{\pgfqpoint{0.918596in}{1.931348in}}%
\pgfpathlineto{\pgfqpoint{0.952362in}{1.844932in}}%
\pgfpathlineto{\pgfqpoint{0.991244in}{1.736738in}}%
\pgfpathlineto{\pgfqpoint{1.040358in}{1.590294in}}%
\pgfpathlineto{\pgfqpoint{1.220442in}{1.043614in}}%
\pgfpathlineto{\pgfqpoint{1.257277in}{0.946577in}}%
\pgfpathlineto{\pgfqpoint{1.290020in}{0.868795in}}%
\pgfpathlineto{\pgfqpoint{1.318669in}{0.808416in}}%
\pgfpathlineto{\pgfqpoint{1.344249in}{0.761286in}}%
\pgfpathlineto{\pgfqpoint{1.367783in}{0.724056in}}%
\pgfpathlineto{\pgfqpoint{1.389270in}{0.695518in}}%
\pgfpathlineto{\pgfqpoint{1.409734in}{0.673410in}}%
\pgfpathlineto{\pgfqpoint{1.428152in}{0.657896in}}%
\pgfpathlineto{\pgfqpoint{1.445547in}{0.647159in}}%
\pgfpathlineto{\pgfqpoint{1.461918in}{0.640597in}}%
\pgfpathlineto{\pgfqpoint{1.477266in}{0.637610in}}%
\pgfpathlineto{\pgfqpoint{1.492614in}{0.637715in}}%
\pgfpathlineto{\pgfqpoint{1.507962in}{0.640930in}}%
\pgfpathlineto{\pgfqpoint{1.523310in}{0.647261in}}%
\pgfpathlineto{\pgfqpoint{1.539681in}{0.657445in}}%
\pgfpathlineto{\pgfqpoint{1.557076in}{0.672123in}}%
\pgfpathlineto{\pgfqpoint{1.575494in}{0.691954in}}%
\pgfpathlineto{\pgfqpoint{1.594934in}{0.717596in}}%
\pgfpathlineto{\pgfqpoint{1.615399in}{0.749694in}}%
\pgfpathlineto{\pgfqpoint{1.637909in}{0.790862in}}%
\pgfpathlineto{\pgfqpoint{1.662466in}{0.842478in}}%
\pgfpathlineto{\pgfqpoint{1.689069in}{0.905842in}}%
\pgfpathlineto{\pgfqpoint{1.718742in}{0.984950in}}%
\pgfpathlineto{\pgfqpoint{1.751485in}{1.081485in}}%
\pgfpathlineto{\pgfqpoint{1.788320in}{1.199987in}}%
\pgfpathlineto{\pgfqpoint{1.832318in}{1.352300in}}%
\pgfpathlineto{\pgfqpoint{1.893710in}{1.577073in}}%
\pgfpathlineto{\pgfqpoint{2.005240in}{1.986453in}}%
\pgfpathlineto{\pgfqpoint{2.050261in}{2.139092in}}%
\pgfpathlineto{\pgfqpoint{2.088119in}{2.256983in}}%
\pgfpathlineto{\pgfqpoint{2.120862in}{2.349338in}}%
\pgfpathlineto{\pgfqpoint{2.150535in}{2.424163in}}%
\pgfpathlineto{\pgfqpoint{2.177138in}{2.483336in}}%
\pgfpathlineto{\pgfqpoint{2.201695in}{2.530828in}}%
\pgfpathlineto{\pgfqpoint{2.224205in}{2.568035in}}%
\pgfpathlineto{\pgfqpoint{2.245693in}{2.597688in}}%
\pgfpathlineto{\pgfqpoint{2.265134in}{2.619448in}}%
\pgfpathlineto{\pgfqpoint{2.283551in}{2.635539in}}%
\pgfpathlineto{\pgfqpoint{2.300946in}{2.646647in}}%
\pgfpathlineto{\pgfqpoint{2.317317in}{2.653451in}}%
\pgfpathlineto{\pgfqpoint{2.332665in}{2.656609in}}%
\pgfpathlineto{\pgfqpoint{2.348013in}{2.656658in}}%
\pgfpathlineto{\pgfqpoint{2.363361in}{2.653614in}}%
\pgfpathlineto{\pgfqpoint{2.378709in}{2.647509in}}%
\pgfpathlineto{\pgfqpoint{2.395081in}{2.637664in}}%
\pgfpathlineto{\pgfqpoint{2.412475in}{2.623505in}}%
\pgfpathlineto{\pgfqpoint{2.430893in}{2.604456in}}%
\pgfpathlineto{\pgfqpoint{2.451357in}{2.578549in}}%
\pgfpathlineto{\pgfqpoint{2.472844in}{2.546192in}}%
\pgfpathlineto{\pgfqpoint{2.496378in}{2.505010in}}%
\pgfpathlineto{\pgfqpoint{2.521958in}{2.453890in}}%
\pgfpathlineto{\pgfqpoint{2.550608in}{2.389468in}}%
\pgfpathlineto{\pgfqpoint{2.582327in}{2.310346in}}%
\pgfpathlineto{\pgfqpoint{2.619162in}{2.209862in}}%
\pgfpathlineto{\pgfqpoint{2.664183in}{2.077580in}}%
\pgfpathlineto{\pgfqpoint{2.733761in}{1.862060in}}%
\pgfpathlineto{\pgfqpoint{2.816641in}{1.607567in}}%
\pgfpathlineto{\pgfqpoint{2.862685in}{1.476252in}}%
\pgfpathlineto{\pgfqpoint{2.900544in}{1.377374in}}%
\pgfpathlineto{\pgfqpoint{2.933286in}{1.300129in}}%
\pgfpathlineto{\pgfqpoint{2.962959in}{1.237745in}}%
\pgfpathlineto{\pgfqpoint{2.989562in}{1.188584in}}%
\pgfpathlineto{\pgfqpoint{3.014119in}{1.149274in}}%
\pgfpathlineto{\pgfqpoint{3.037653in}{1.117334in}}%
\pgfpathlineto{\pgfqpoint{3.059140in}{1.093232in}}%
\pgfpathlineto{\pgfqpoint{3.079604in}{1.074869in}}%
\pgfpathlineto{\pgfqpoint{3.099045in}{1.061626in}}%
\pgfpathlineto{\pgfqpoint{3.117463in}{1.052879in}}%
\pgfpathlineto{\pgfqpoint{3.134857in}{1.048010in}}%
\pgfpathlineto{\pgfqpoint{3.152252in}{1.046417in}}%
\pgfpathlineto{\pgfqpoint{3.169646in}{1.048067in}}%
\pgfpathlineto{\pgfqpoint{3.187041in}{1.052909in}}%
\pgfpathlineto{\pgfqpoint{3.205459in}{1.061442in}}%
\pgfpathlineto{\pgfqpoint{3.223876in}{1.073384in}}%
\pgfpathlineto{\pgfqpoint{3.244340in}{1.090512in}}%
\pgfpathlineto{\pgfqpoint{3.265828in}{1.112674in}}%
\pgfpathlineto{\pgfqpoint{3.289361in}{1.141576in}}%
\pgfpathlineto{\pgfqpoint{3.314941in}{1.178078in}}%
\pgfpathlineto{\pgfqpoint{3.343591in}{1.224641in}}%
\pgfpathlineto{\pgfqpoint{3.375311in}{1.282287in}}%
\pgfpathlineto{\pgfqpoint{3.412146in}{1.355803in}}%
\pgfpathlineto{\pgfqpoint{3.459213in}{1.457119in}}%
\pgfpathlineto{\pgfqpoint{3.628042in}{1.827592in}}%
\pgfpathlineto{\pgfqpoint{3.665901in}{1.899517in}}%
\pgfpathlineto{\pgfqpoint{3.698643in}{1.955306in}}%
\pgfpathlineto{\pgfqpoint{3.728316in}{1.999877in}}%
\pgfpathlineto{\pgfqpoint{3.754919in}{2.034480in}}%
\pgfpathlineto{\pgfqpoint{3.779476in}{2.061594in}}%
\pgfpathlineto{\pgfqpoint{3.803010in}{2.083009in}}%
\pgfpathlineto{\pgfqpoint{3.824497in}{2.098520in}}%
\pgfpathlineto{\pgfqpoint{3.844961in}{2.109623in}}%
\pgfpathlineto{\pgfqpoint{3.864402in}{2.116811in}}%
\pgfpathlineto{\pgfqpoint{3.883843in}{2.120709in}}%
\pgfpathlineto{\pgfqpoint{3.902261in}{2.121370in}}%
\pgfpathlineto{\pgfqpoint{3.921702in}{2.118895in}}%
\pgfpathlineto{\pgfqpoint{3.941143in}{2.113207in}}%
\pgfpathlineto{\pgfqpoint{3.961607in}{2.103824in}}%
\pgfpathlineto{\pgfqpoint{3.983094in}{2.090336in}}%
\pgfpathlineto{\pgfqpoint{4.005605in}{2.072374in}}%
\pgfpathlineto{\pgfqpoint{4.029138in}{2.049620in}}%
\pgfpathlineto{\pgfqpoint{4.054718in}{2.020592in}}%
\pgfpathlineto{\pgfqpoint{4.083368in}{1.983243in}}%
\pgfpathlineto{\pgfqpoint{4.115088in}{1.936652in}}%
\pgfpathlineto{\pgfqpoint{4.151923in}{1.876819in}}%
\pgfpathlineto{\pgfqpoint{4.196944in}{1.797515in}}%
\pgfpathlineto{\pgfqpoint{4.274707in}{1.652974in}}%
\pgfpathlineto{\pgfqpoint{4.342239in}{1.530191in}}%
\pgfpathlineto{\pgfqpoint{4.385214in}{1.458219in}}%
\pgfpathlineto{\pgfqpoint{4.421026in}{1.403931in}}%
\pgfpathlineto{\pgfqpoint{4.452745in}{1.361191in}}%
\pgfpathlineto{\pgfqpoint{4.481395in}{1.327527in}}%
\pgfpathlineto{\pgfqpoint{4.507998in}{1.300874in}}%
\pgfpathlineto{\pgfqpoint{4.532555in}{1.280474in}}%
\pgfpathlineto{\pgfqpoint{4.556089in}{1.264887in}}%
\pgfpathlineto{\pgfqpoint{4.578599in}{1.253724in}}%
\pgfpathlineto{\pgfqpoint{4.600087in}{1.246552in}}%
\pgfpathlineto{\pgfqpoint{4.620551in}{1.242917in}}%
\pgfpathlineto{\pgfqpoint{4.641015in}{1.242404in}}%
\pgfpathlineto{\pgfqpoint{4.661479in}{1.244994in}}%
\pgfpathlineto{\pgfqpoint{4.681943in}{1.250644in}}%
\pgfpathlineto{\pgfqpoint{4.703430in}{1.259800in}}%
\pgfpathlineto{\pgfqpoint{4.725941in}{1.272822in}}%
\pgfpathlineto{\pgfqpoint{4.749474in}{1.290031in}}%
\pgfpathlineto{\pgfqpoint{4.775055in}{1.312668in}}%
\pgfpathlineto{\pgfqpoint{4.802681in}{1.341374in}}%
\pgfpathlineto{\pgfqpoint{4.833377in}{1.377929in}}%
\pgfpathlineto{\pgfqpoint{4.868166in}{1.424456in}}%
\pgfpathlineto{\pgfqpoint{4.909094in}{1.484705in}}%
\pgfpathlineto{\pgfqpoint{4.963324in}{1.570577in}}%
\pgfpathlineto{\pgfqpoint{5.088155in}{1.770230in}}%
\pgfpathlineto{\pgfqpoint{5.130107in}{1.830486in}}%
\pgfpathlineto{\pgfqpoint{5.165919in}{1.876718in}}%
\pgfpathlineto{\pgfqpoint{5.197638in}{1.912780in}}%
\pgfpathlineto{\pgfqpoint{5.226288in}{1.940869in}}%
\pgfpathlineto{\pgfqpoint{5.252891in}{1.962790in}}%
\pgfpathlineto{\pgfqpoint{5.277448in}{1.979243in}}%
\pgfpathlineto{\pgfqpoint{5.300982in}{1.991456in}}%
\pgfpathlineto{\pgfqpoint{5.323492in}{1.999789in}}%
\pgfpathlineto{\pgfqpoint{5.346003in}{2.004790in}}%
\pgfpathlineto{\pgfqpoint{5.367490in}{2.006432in}}%
\pgfpathlineto{\pgfqpoint{5.388977in}{2.005022in}}%
\pgfpathlineto{\pgfqpoint{5.400233in}{2.003073in}}%
\pgfpathlineto{\pgfqpoint{5.400233in}{2.003073in}}%
\pgfusepath{stroke}%
\end{pgfscope}%
\begin{pgfscope}%
\pgfpathrectangle{\pgfqpoint{0.750000in}{0.500000in}}{\pgfqpoint{4.650000in}{3.020000in}}%
\pgfusepath{clip}%
\pgfsetrectcap%
\pgfsetroundjoin%
\pgfsetlinewidth{1.505625pt}%
\definecolor{currentstroke}{rgb}{0.839216,0.152941,0.156863}%
\pgfsetstrokecolor{currentstroke}%
\pgfsetdash{}{0pt}%
\pgfpathmoveto{\pgfqpoint{0.749767in}{1.727364in}}%
\pgfpathlineto{\pgfqpoint{0.872552in}{2.040600in}}%
\pgfpathlineto{\pgfqpoint{0.917573in}{2.145787in}}%
\pgfpathlineto{\pgfqpoint{0.955432in}{2.226589in}}%
\pgfpathlineto{\pgfqpoint{0.988174in}{2.289520in}}%
\pgfpathlineto{\pgfqpoint{1.017847in}{2.340180in}}%
\pgfpathlineto{\pgfqpoint{1.045474in}{2.381372in}}%
\pgfpathlineto{\pgfqpoint{1.071054in}{2.414026in}}%
\pgfpathlineto{\pgfqpoint{1.094587in}{2.439185in}}%
\pgfpathlineto{\pgfqpoint{1.117098in}{2.458729in}}%
\pgfpathlineto{\pgfqpoint{1.137562in}{2.472580in}}%
\pgfpathlineto{\pgfqpoint{1.157003in}{2.482243in}}%
\pgfpathlineto{\pgfqpoint{1.176444in}{2.488486in}}%
\pgfpathlineto{\pgfqpoint{1.194861in}{2.491252in}}%
\pgfpathlineto{\pgfqpoint{1.213279in}{2.490978in}}%
\pgfpathlineto{\pgfqpoint{1.231697in}{2.487704in}}%
\pgfpathlineto{\pgfqpoint{1.251138in}{2.481060in}}%
\pgfpathlineto{\pgfqpoint{1.271602in}{2.470622in}}%
\pgfpathlineto{\pgfqpoint{1.293089in}{2.455998in}}%
\pgfpathlineto{\pgfqpoint{1.315600in}{2.436837in}}%
\pgfpathlineto{\pgfqpoint{1.340157in}{2.411721in}}%
\pgfpathlineto{\pgfqpoint{1.366760in}{2.379939in}}%
\pgfpathlineto{\pgfqpoint{1.396433in}{2.339460in}}%
\pgfpathlineto{\pgfqpoint{1.430199in}{2.287869in}}%
\pgfpathlineto{\pgfqpoint{1.470104in}{2.220888in}}%
\pgfpathlineto{\pgfqpoint{1.523310in}{2.124965in}}%
\pgfpathlineto{\pgfqpoint{1.642002in}{1.909048in}}%
\pgfpathlineto{\pgfqpoint{1.682930in}{1.841992in}}%
\pgfpathlineto{\pgfqpoint{1.717719in}{1.790700in}}%
\pgfpathlineto{\pgfqpoint{1.748415in}{1.750787in}}%
\pgfpathlineto{\pgfqpoint{1.776042in}{1.719757in}}%
\pgfpathlineto{\pgfqpoint{1.801622in}{1.695559in}}%
\pgfpathlineto{\pgfqpoint{1.825156in}{1.677407in}}%
\pgfpathlineto{\pgfqpoint{1.847666in}{1.663908in}}%
\pgfpathlineto{\pgfqpoint{1.869153in}{1.654666in}}%
\pgfpathlineto{\pgfqpoint{1.889617in}{1.649247in}}%
\pgfpathlineto{\pgfqpoint{1.909058in}{1.647195in}}%
\pgfpathlineto{\pgfqpoint{1.928499in}{1.648174in}}%
\pgfpathlineto{\pgfqpoint{1.947940in}{1.652178in}}%
\pgfpathlineto{\pgfqpoint{1.968404in}{1.659630in}}%
\pgfpathlineto{\pgfqpoint{1.988868in}{1.670344in}}%
\pgfpathlineto{\pgfqpoint{2.010356in}{1.685014in}}%
\pgfpathlineto{\pgfqpoint{2.033889in}{1.704954in}}%
\pgfpathlineto{\pgfqpoint{2.058446in}{1.729860in}}%
\pgfpathlineto{\pgfqpoint{2.085050in}{1.761248in}}%
\pgfpathlineto{\pgfqpoint{2.114722in}{1.801164in}}%
\pgfpathlineto{\pgfqpoint{2.148488in}{1.852072in}}%
\pgfpathlineto{\pgfqpoint{2.187370in}{1.916561in}}%
\pgfpathlineto{\pgfqpoint{2.237507in}{2.006126in}}%
\pgfpathlineto{\pgfqpoint{2.375640in}{2.256404in}}%
\pgfpathlineto{\pgfqpoint{2.414521in}{2.318654in}}%
\pgfpathlineto{\pgfqpoint{2.447264in}{2.365548in}}%
\pgfpathlineto{\pgfqpoint{2.476937in}{2.402752in}}%
\pgfpathlineto{\pgfqpoint{2.503540in}{2.431231in}}%
\pgfpathlineto{\pgfqpoint{2.528097in}{2.453022in}}%
\pgfpathlineto{\pgfqpoint{2.550608in}{2.468932in}}%
\pgfpathlineto{\pgfqpoint{2.572095in}{2.480296in}}%
\pgfpathlineto{\pgfqpoint{2.591536in}{2.487235in}}%
\pgfpathlineto{\pgfqpoint{2.610977in}{2.490903in}}%
\pgfpathlineto{\pgfqpoint{2.629394in}{2.491299in}}%
\pgfpathlineto{\pgfqpoint{2.647812in}{2.488657in}}%
\pgfpathlineto{\pgfqpoint{2.666230in}{2.482952in}}%
\pgfpathlineto{\pgfqpoint{2.685671in}{2.473600in}}%
\pgfpathlineto{\pgfqpoint{2.705112in}{2.460839in}}%
\pgfpathlineto{\pgfqpoint{2.725576in}{2.443761in}}%
\pgfpathlineto{\pgfqpoint{2.748086in}{2.420735in}}%
\pgfpathlineto{\pgfqpoint{2.771620in}{2.392042in}}%
\pgfpathlineto{\pgfqpoint{2.797200in}{2.355708in}}%
\pgfpathlineto{\pgfqpoint{2.824826in}{2.310774in}}%
\pgfpathlineto{\pgfqpoint{2.854499in}{2.256416in}}%
\pgfpathlineto{\pgfqpoint{2.887242in}{2.189857in}}%
\pgfpathlineto{\pgfqpoint{2.925101in}{2.105532in}}%
\pgfpathlineto{\pgfqpoint{2.970122in}{1.997206in}}%
\pgfpathlineto{\pgfqpoint{3.032537in}{1.838104in}}%
\pgfpathlineto{\pgfqpoint{3.139973in}{1.563765in}}%
\pgfpathlineto{\pgfqpoint{3.186018in}{1.455274in}}%
\pgfpathlineto{\pgfqpoint{3.223876in}{1.373614in}}%
\pgfpathlineto{\pgfqpoint{3.257642in}{1.307963in}}%
\pgfpathlineto{\pgfqpoint{3.287315in}{1.256711in}}%
\pgfpathlineto{\pgfqpoint{3.314941in}{1.214971in}}%
\pgfpathlineto{\pgfqpoint{3.340522in}{1.181831in}}%
\pgfpathlineto{\pgfqpoint{3.364055in}{1.156260in}}%
\pgfpathlineto{\pgfqpoint{3.386566in}{1.136371in}}%
\pgfpathlineto{\pgfqpoint{3.407030in}{1.122259in}}%
\pgfpathlineto{\pgfqpoint{3.426471in}{1.112408in}}%
\pgfpathlineto{\pgfqpoint{3.445912in}{1.106044in}}%
\pgfpathlineto{\pgfqpoint{3.464329in}{1.103237in}}%
\pgfpathlineto{\pgfqpoint{3.482747in}{1.103550in}}%
\pgfpathlineto{\pgfqpoint{3.501165in}{1.106952in}}%
\pgfpathlineto{\pgfqpoint{3.520606in}{1.113841in}}%
\pgfpathlineto{\pgfqpoint{3.540047in}{1.124041in}}%
\pgfpathlineto{\pgfqpoint{3.560511in}{1.138251in}}%
\pgfpathlineto{\pgfqpoint{3.583021in}{1.157836in}}%
\pgfpathlineto{\pgfqpoint{3.606555in}{1.182517in}}%
\pgfpathlineto{\pgfqpoint{3.632135in}{1.213903in}}%
\pgfpathlineto{\pgfqpoint{3.660785in}{1.254207in}}%
\pgfpathlineto{\pgfqpoint{3.692504in}{1.304444in}}%
\pgfpathlineto{\pgfqpoint{3.728316in}{1.367115in}}%
\pgfpathlineto{\pgfqpoint{3.772314in}{1.450676in}}%
\pgfpathlineto{\pgfqpoint{3.837799in}{1.582570in}}%
\pgfpathlineto{\pgfqpoint{3.923748in}{1.754605in}}%
\pgfpathlineto{\pgfqpoint{3.968769in}{1.837708in}}%
\pgfpathlineto{\pgfqpoint{4.005605in}{1.899502in}}%
\pgfpathlineto{\pgfqpoint{4.038347in}{1.948562in}}%
\pgfpathlineto{\pgfqpoint{4.068020in}{1.987524in}}%
\pgfpathlineto{\pgfqpoint{4.094623in}{2.017557in}}%
\pgfpathlineto{\pgfqpoint{4.119180in}{2.040884in}}%
\pgfpathlineto{\pgfqpoint{4.142714in}{2.059085in}}%
\pgfpathlineto{\pgfqpoint{4.165225in}{2.072562in}}%
\pgfpathlineto{\pgfqpoint{4.186712in}{2.081762in}}%
\pgfpathlineto{\pgfqpoint{4.207176in}{2.087154in}}%
\pgfpathlineto{\pgfqpoint{4.227640in}{2.089246in}}%
\pgfpathlineto{\pgfqpoint{4.247081in}{2.088187in}}%
\pgfpathlineto{\pgfqpoint{4.267545in}{2.083898in}}%
\pgfpathlineto{\pgfqpoint{4.288009in}{2.076411in}}%
\pgfpathlineto{\pgfqpoint{4.309496in}{2.065195in}}%
\pgfpathlineto{\pgfqpoint{4.332007in}{2.049888in}}%
\pgfpathlineto{\pgfqpoint{4.355541in}{2.030174in}}%
\pgfpathlineto{\pgfqpoint{4.381121in}{2.004702in}}%
\pgfpathlineto{\pgfqpoint{4.409770in}{1.971577in}}%
\pgfpathlineto{\pgfqpoint{4.441490in}{1.929859in}}%
\pgfpathlineto{\pgfqpoint{4.477302in}{1.877364in}}%
\pgfpathlineto{\pgfqpoint{4.520277in}{1.808536in}}%
\pgfpathlineto{\pgfqpoint{4.580646in}{1.705386in}}%
\pgfpathlineto{\pgfqpoint{4.679896in}{1.535826in}}%
\pgfpathlineto{\pgfqpoint{4.723894in}{1.467096in}}%
\pgfpathlineto{\pgfqpoint{4.760730in}{1.415061in}}%
\pgfpathlineto{\pgfqpoint{4.793472in}{1.374056in}}%
\pgfpathlineto{\pgfqpoint{4.823145in}{1.341801in}}%
\pgfpathlineto{\pgfqpoint{4.849748in}{1.317243in}}%
\pgfpathlineto{\pgfqpoint{4.875329in}{1.297786in}}%
\pgfpathlineto{\pgfqpoint{4.898862in}{1.283652in}}%
\pgfpathlineto{\pgfqpoint{4.921373in}{1.273621in}}%
\pgfpathlineto{\pgfqpoint{4.942860in}{1.267288in}}%
\pgfpathlineto{\pgfqpoint{4.964347in}{1.264154in}}%
\pgfpathlineto{\pgfqpoint{4.985835in}{1.264225in}}%
\pgfpathlineto{\pgfqpoint{5.007322in}{1.267477in}}%
\pgfpathlineto{\pgfqpoint{5.028809in}{1.273864in}}%
\pgfpathlineto{\pgfqpoint{5.051320in}{1.283835in}}%
\pgfpathlineto{\pgfqpoint{5.074853in}{1.297728in}}%
\pgfpathlineto{\pgfqpoint{5.099410in}{1.315829in}}%
\pgfpathlineto{\pgfqpoint{5.126014in}{1.339337in}}%
\pgfpathlineto{\pgfqpoint{5.154663in}{1.368815in}}%
\pgfpathlineto{\pgfqpoint{5.186383in}{1.405929in}}%
\pgfpathlineto{\pgfqpoint{5.223218in}{1.454015in}}%
\pgfpathlineto{\pgfqpoint{5.267216in}{1.516823in}}%
\pgfpathlineto{\pgfqpoint{5.331678in}{1.614897in}}%
\pgfpathlineto{\pgfqpoint{5.400233in}{1.719645in}}%
\pgfpathlineto{\pgfqpoint{5.400233in}{1.719645in}}%
\pgfusepath{stroke}%
\end{pgfscope}%
\begin{pgfscope}%
\pgfpathrectangle{\pgfqpoint{0.750000in}{0.500000in}}{\pgfqpoint{4.650000in}{3.020000in}}%
\pgfusepath{clip}%
\pgfsetrectcap%
\pgfsetroundjoin%
\pgfsetlinewidth{1.505625pt}%
\definecolor{currentstroke}{rgb}{0.580392,0.403922,0.741176}%
\pgfsetstrokecolor{currentstroke}%
\pgfsetdash{}{0pt}%
\pgfpathmoveto{\pgfqpoint{0.749767in}{1.014252in}}%
\pgfpathlineto{\pgfqpoint{0.777394in}{0.981150in}}%
\pgfpathlineto{\pgfqpoint{0.802974in}{0.955046in}}%
\pgfpathlineto{\pgfqpoint{0.827531in}{0.934230in}}%
\pgfpathlineto{\pgfqpoint{0.851065in}{0.918259in}}%
\pgfpathlineto{\pgfqpoint{0.873575in}{0.906660in}}%
\pgfpathlineto{\pgfqpoint{0.895063in}{0.898944in}}%
\pgfpathlineto{\pgfqpoint{0.916550in}{0.894485in}}%
\pgfpathlineto{\pgfqpoint{0.938037in}{0.893239in}}%
\pgfpathlineto{\pgfqpoint{0.959524in}{0.895138in}}%
\pgfpathlineto{\pgfqpoint{0.981012in}{0.900096in}}%
\pgfpathlineto{\pgfqpoint{1.003522in}{0.908452in}}%
\pgfpathlineto{\pgfqpoint{1.027056in}{0.920490in}}%
\pgfpathlineto{\pgfqpoint{1.052636in}{0.937182in}}%
\pgfpathlineto{\pgfqpoint{1.080263in}{0.959117in}}%
\pgfpathlineto{\pgfqpoint{1.109935in}{0.986764in}}%
\pgfpathlineto{\pgfqpoint{1.142678in}{1.021570in}}%
\pgfpathlineto{\pgfqpoint{1.180537in}{1.066483in}}%
\pgfpathlineto{\pgfqpoint{1.226581in}{1.126169in}}%
\pgfpathlineto{\pgfqpoint{1.295136in}{1.220764in}}%
\pgfpathlineto{\pgfqpoint{1.390294in}{1.351427in}}%
\pgfpathlineto{\pgfqpoint{1.440431in}{1.414878in}}%
\pgfpathlineto{\pgfqpoint{1.482382in}{1.463191in}}%
\pgfpathlineto{\pgfqpoint{1.520241in}{1.502283in}}%
\pgfpathlineto{\pgfqpoint{1.556053in}{1.534906in}}%
\pgfpathlineto{\pgfqpoint{1.589818in}{1.561576in}}%
\pgfpathlineto{\pgfqpoint{1.622561in}{1.583581in}}%
\pgfpathlineto{\pgfqpoint{1.655304in}{1.601831in}}%
\pgfpathlineto{\pgfqpoint{1.688046in}{1.616466in}}%
\pgfpathlineto{\pgfqpoint{1.720789in}{1.627723in}}%
\pgfpathlineto{\pgfqpoint{1.755578in}{1.636343in}}%
\pgfpathlineto{\pgfqpoint{1.792413in}{1.642247in}}%
\pgfpathlineto{\pgfqpoint{1.834364in}{1.645784in}}%
\pgfpathlineto{\pgfqpoint{1.889617in}{1.647113in}}%
\pgfpathlineto{\pgfqpoint{2.001147in}{1.649133in}}%
\pgfpathlineto{\pgfqpoint{2.044121in}{1.653577in}}%
\pgfpathlineto{\pgfqpoint{2.080957in}{1.660455in}}%
\pgfpathlineto{\pgfqpoint{2.114722in}{1.669835in}}%
\pgfpathlineto{\pgfqpoint{2.147465in}{1.682145in}}%
\pgfpathlineto{\pgfqpoint{2.179184in}{1.697372in}}%
\pgfpathlineto{\pgfqpoint{2.210904in}{1.716038in}}%
\pgfpathlineto{\pgfqpoint{2.242623in}{1.738249in}}%
\pgfpathlineto{\pgfqpoint{2.275366in}{1.764911in}}%
\pgfpathlineto{\pgfqpoint{2.310155in}{1.797301in}}%
\pgfpathlineto{\pgfqpoint{2.346990in}{1.835905in}}%
\pgfpathlineto{\pgfqpoint{2.387918in}{1.883476in}}%
\pgfpathlineto{\pgfqpoint{2.433962in}{1.941888in}}%
\pgfpathlineto{\pgfqpoint{2.494331in}{2.023883in}}%
\pgfpathlineto{\pgfqpoint{2.627348in}{2.206171in}}%
\pgfpathlineto{\pgfqpoint{2.671346in}{2.260171in}}%
\pgfpathlineto{\pgfqpoint{2.708181in}{2.300565in}}%
\pgfpathlineto{\pgfqpoint{2.739900in}{2.330947in}}%
\pgfpathlineto{\pgfqpoint{2.768550in}{2.354318in}}%
\pgfpathlineto{\pgfqpoint{2.795154in}{2.372162in}}%
\pgfpathlineto{\pgfqpoint{2.819710in}{2.385054in}}%
\pgfpathlineto{\pgfqpoint{2.843244in}{2.393973in}}%
\pgfpathlineto{\pgfqpoint{2.865755in}{2.399201in}}%
\pgfpathlineto{\pgfqpoint{2.887242in}{2.401060in}}%
\pgfpathlineto{\pgfqpoint{2.908729in}{2.399772in}}%
\pgfpathlineto{\pgfqpoint{2.930217in}{2.395271in}}%
\pgfpathlineto{\pgfqpoint{2.951704in}{2.387513in}}%
\pgfpathlineto{\pgfqpoint{2.973191in}{2.376476in}}%
\pgfpathlineto{\pgfqpoint{2.995702in}{2.361399in}}%
\pgfpathlineto{\pgfqpoint{3.019235in}{2.341825in}}%
\pgfpathlineto{\pgfqpoint{3.043792in}{2.317318in}}%
\pgfpathlineto{\pgfqpoint{3.070396in}{2.286197in}}%
\pgfpathlineto{\pgfqpoint{3.098022in}{2.249058in}}%
\pgfpathlineto{\pgfqpoint{3.128718in}{2.202375in}}%
\pgfpathlineto{\pgfqpoint{3.161461in}{2.146816in}}%
\pgfpathlineto{\pgfqpoint{3.198296in}{2.078037in}}%
\pgfpathlineto{\pgfqpoint{3.241271in}{1.990846in}}%
\pgfpathlineto{\pgfqpoint{3.294477in}{1.875417in}}%
\pgfpathlineto{\pgfqpoint{3.471492in}{1.485234in}}%
\pgfpathlineto{\pgfqpoint{3.512420in}{1.405601in}}%
\pgfpathlineto{\pgfqpoint{3.548232in}{1.342418in}}%
\pgfpathlineto{\pgfqpoint{3.579951in}{1.292446in}}%
\pgfpathlineto{\pgfqpoint{3.608601in}{1.252736in}}%
\pgfpathlineto{\pgfqpoint{3.635205in}{1.220864in}}%
\pgfpathlineto{\pgfqpoint{3.659761in}{1.195977in}}%
\pgfpathlineto{\pgfqpoint{3.683295in}{1.176392in}}%
\pgfpathlineto{\pgfqpoint{3.705806in}{1.161683in}}%
\pgfpathlineto{\pgfqpoint{3.727293in}{1.151392in}}%
\pgfpathlineto{\pgfqpoint{3.747757in}{1.145036in}}%
\pgfpathlineto{\pgfqpoint{3.767198in}{1.142128in}}%
\pgfpathlineto{\pgfqpoint{3.786639in}{1.142264in}}%
\pgfpathlineto{\pgfqpoint{3.806080in}{1.145422in}}%
\pgfpathlineto{\pgfqpoint{3.826544in}{1.151968in}}%
\pgfpathlineto{\pgfqpoint{3.847008in}{1.161752in}}%
\pgfpathlineto{\pgfqpoint{3.868495in}{1.175412in}}%
\pgfpathlineto{\pgfqpoint{3.892029in}{1.194208in}}%
\pgfpathlineto{\pgfqpoint{3.916586in}{1.217885in}}%
\pgfpathlineto{\pgfqpoint{3.943189in}{1.247917in}}%
\pgfpathlineto{\pgfqpoint{3.972862in}{1.286332in}}%
\pgfpathlineto{\pgfqpoint{4.005605in}{1.334046in}}%
\pgfpathlineto{\pgfqpoint{4.042440in}{1.393329in}}%
\pgfpathlineto{\pgfqpoint{4.087461in}{1.471902in}}%
\pgfpathlineto{\pgfqpoint{4.156016in}{1.598697in}}%
\pgfpathlineto{\pgfqpoint{4.240942in}{1.754546in}}%
\pgfpathlineto{\pgfqpoint{4.286986in}{1.832477in}}%
\pgfpathlineto{\pgfqpoint{4.324844in}{1.890626in}}%
\pgfpathlineto{\pgfqpoint{4.357587in}{1.935505in}}%
\pgfpathlineto{\pgfqpoint{4.387260in}{1.971179in}}%
\pgfpathlineto{\pgfqpoint{4.414886in}{1.999682in}}%
\pgfpathlineto{\pgfqpoint{4.440467in}{2.021734in}}%
\pgfpathlineto{\pgfqpoint{4.464000in}{2.038152in}}%
\pgfpathlineto{\pgfqpoint{4.486511in}{2.050267in}}%
\pgfpathlineto{\pgfqpoint{4.507998in}{2.058484in}}%
\pgfpathlineto{\pgfqpoint{4.529485in}{2.063389in}}%
\pgfpathlineto{\pgfqpoint{4.549949in}{2.064967in}}%
\pgfpathlineto{\pgfqpoint{4.570414in}{2.063536in}}%
\pgfpathlineto{\pgfqpoint{4.591901in}{2.058827in}}%
\pgfpathlineto{\pgfqpoint{4.613388in}{2.050894in}}%
\pgfpathlineto{\pgfqpoint{4.635899in}{2.039216in}}%
\pgfpathlineto{\pgfqpoint{4.659432in}{2.023458in}}%
\pgfpathlineto{\pgfqpoint{4.685013in}{2.002419in}}%
\pgfpathlineto{\pgfqpoint{4.711616in}{1.976497in}}%
\pgfpathlineto{\pgfqpoint{4.741289in}{1.943148in}}%
\pgfpathlineto{\pgfqpoint{4.774031in}{1.901555in}}%
\pgfpathlineto{\pgfqpoint{4.811890in}{1.848237in}}%
\pgfpathlineto{\pgfqpoint{4.857934in}{1.777753in}}%
\pgfpathlineto{\pgfqpoint{4.932628in}{1.656779in}}%
\pgfpathlineto{\pgfqpoint{5.006299in}{1.539408in}}%
\pgfpathlineto{\pgfqpoint{5.051320in}{1.473353in}}%
\pgfpathlineto{\pgfqpoint{5.089178in}{1.423198in}}%
\pgfpathlineto{\pgfqpoint{5.121921in}{1.384781in}}%
\pgfpathlineto{\pgfqpoint{5.151594in}{1.354541in}}%
\pgfpathlineto{\pgfqpoint{5.179220in}{1.330688in}}%
\pgfpathlineto{\pgfqpoint{5.204800in}{1.312558in}}%
\pgfpathlineto{\pgfqpoint{5.229357in}{1.298905in}}%
\pgfpathlineto{\pgfqpoint{5.252891in}{1.289383in}}%
\pgfpathlineto{\pgfqpoint{5.275402in}{1.283603in}}%
\pgfpathlineto{\pgfqpoint{5.296889in}{1.281151in}}%
\pgfpathlineto{\pgfqpoint{5.318376in}{1.281696in}}%
\pgfpathlineto{\pgfqpoint{5.339863in}{1.285218in}}%
\pgfpathlineto{\pgfqpoint{5.362374in}{1.292048in}}%
\pgfpathlineto{\pgfqpoint{5.385908in}{1.302540in}}%
\pgfpathlineto{\pgfqpoint{5.400233in}{1.310552in}}%
\pgfpathlineto{\pgfqpoint{5.400233in}{1.310552in}}%
\pgfusepath{stroke}%
\end{pgfscope}%
\begin{pgfscope}%
\pgfpathrectangle{\pgfqpoint{0.750000in}{0.500000in}}{\pgfqpoint{4.650000in}{3.020000in}}%
\pgfusepath{clip}%
\pgfsetrectcap%
\pgfsetroundjoin%
\pgfsetlinewidth{1.505625pt}%
\definecolor{currentstroke}{rgb}{0.549020,0.337255,0.294118}%
\pgfsetstrokecolor{currentstroke}%
\pgfsetdash{}{0pt}%
\pgfpathmoveto{\pgfqpoint{0.749767in}{2.326253in}}%
\pgfpathlineto{\pgfqpoint{0.776371in}{2.314306in}}%
\pgfpathlineto{\pgfqpoint{0.803997in}{2.298519in}}%
\pgfpathlineto{\pgfqpoint{0.834693in}{2.277290in}}%
\pgfpathlineto{\pgfqpoint{0.868459in}{2.249959in}}%
\pgfpathlineto{\pgfqpoint{0.906318in}{2.215103in}}%
\pgfpathlineto{\pgfqpoint{0.950316in}{2.170164in}}%
\pgfpathlineto{\pgfqpoint{1.006592in}{2.107957in}}%
\pgfpathlineto{\pgfqpoint{1.189745in}{1.901955in}}%
\pgfpathlineto{\pgfqpoint{1.239882in}{1.852196in}}%
\pgfpathlineto{\pgfqpoint{1.283880in}{1.812735in}}%
\pgfpathlineto{\pgfqpoint{1.325832in}{1.779223in}}%
\pgfpathlineto{\pgfqpoint{1.365737in}{1.751284in}}%
\pgfpathlineto{\pgfqpoint{1.404618in}{1.727814in}}%
\pgfpathlineto{\pgfqpoint{1.443500in}{1.707988in}}%
\pgfpathlineto{\pgfqpoint{1.482382in}{1.691647in}}%
\pgfpathlineto{\pgfqpoint{1.522287in}{1.678245in}}%
\pgfpathlineto{\pgfqpoint{1.564238in}{1.667464in}}%
\pgfpathlineto{\pgfqpoint{1.609259in}{1.659152in}}%
\pgfpathlineto{\pgfqpoint{1.659396in}{1.653128in}}%
\pgfpathlineto{\pgfqpoint{1.718742in}{1.649253in}}%
\pgfpathlineto{\pgfqpoint{1.798552in}{1.647405in}}%
\pgfpathlineto{\pgfqpoint{1.992961in}{1.647212in}}%
\pgfpathlineto{\pgfqpoint{2.102444in}{1.649094in}}%
\pgfpathlineto{\pgfqpoint{2.166906in}{1.653248in}}%
\pgfpathlineto{\pgfqpoint{2.219089in}{1.659667in}}%
\pgfpathlineto{\pgfqpoint{2.265134in}{1.668442in}}%
\pgfpathlineto{\pgfqpoint{2.307085in}{1.679556in}}%
\pgfpathlineto{\pgfqpoint{2.346990in}{1.693318in}}%
\pgfpathlineto{\pgfqpoint{2.385872in}{1.710039in}}%
\pgfpathlineto{\pgfqpoint{2.423730in}{1.729690in}}%
\pgfpathlineto{\pgfqpoint{2.461589in}{1.752805in}}%
\pgfpathlineto{\pgfqpoint{2.500471in}{1.780200in}}%
\pgfpathlineto{\pgfqpoint{2.540376in}{1.812105in}}%
\pgfpathlineto{\pgfqpoint{2.583350in}{1.850530in}}%
\pgfpathlineto{\pgfqpoint{2.630418in}{1.896956in}}%
\pgfpathlineto{\pgfqpoint{2.684647in}{1.955024in}}%
\pgfpathlineto{\pgfqpoint{2.760365in}{2.041173in}}%
\pgfpathlineto{\pgfqpoint{2.864731in}{2.159477in}}%
\pgfpathlineto{\pgfqpoint{2.913845in}{2.210373in}}%
\pgfpathlineto{\pgfqpoint{2.953750in}{2.247532in}}%
\pgfpathlineto{\pgfqpoint{2.989562in}{2.276730in}}%
\pgfpathlineto{\pgfqpoint{3.021282in}{2.298694in}}%
\pgfpathlineto{\pgfqpoint{3.049931in}{2.314967in}}%
\pgfpathlineto{\pgfqpoint{3.077558in}{2.327137in}}%
\pgfpathlineto{\pgfqpoint{3.103138in}{2.335088in}}%
\pgfpathlineto{\pgfqpoint{3.127695in}{2.339543in}}%
\pgfpathlineto{\pgfqpoint{3.151229in}{2.340760in}}%
\pgfpathlineto{\pgfqpoint{3.174762in}{2.338883in}}%
\pgfpathlineto{\pgfqpoint{3.198296in}{2.333832in}}%
\pgfpathlineto{\pgfqpoint{3.221830in}{2.325551in}}%
\pgfpathlineto{\pgfqpoint{3.245364in}{2.314006in}}%
\pgfpathlineto{\pgfqpoint{3.269920in}{2.298472in}}%
\pgfpathlineto{\pgfqpoint{3.295501in}{2.278527in}}%
\pgfpathlineto{\pgfqpoint{3.322104in}{2.253776in}}%
\pgfpathlineto{\pgfqpoint{3.349730in}{2.223866in}}%
\pgfpathlineto{\pgfqpoint{3.379403in}{2.187164in}}%
\pgfpathlineto{\pgfqpoint{3.411123in}{2.143000in}}%
\pgfpathlineto{\pgfqpoint{3.445912in}{2.089196in}}%
\pgfpathlineto{\pgfqpoint{3.484793in}{2.023217in}}%
\pgfpathlineto{\pgfqpoint{3.529814in}{1.940495in}}%
\pgfpathlineto{\pgfqpoint{3.587114in}{1.828304in}}%
\pgfpathlineto{\pgfqpoint{3.750827in}{1.503651in}}%
\pgfpathlineto{\pgfqpoint{3.794824in}{1.425466in}}%
\pgfpathlineto{\pgfqpoint{3.832683in}{1.364309in}}%
\pgfpathlineto{\pgfqpoint{3.866449in}{1.315529in}}%
\pgfpathlineto{\pgfqpoint{3.897145in}{1.276540in}}%
\pgfpathlineto{\pgfqpoint{3.924771in}{1.246222in}}%
\pgfpathlineto{\pgfqpoint{3.950352in}{1.222452in}}%
\pgfpathlineto{\pgfqpoint{3.974908in}{1.203713in}}%
\pgfpathlineto{\pgfqpoint{3.998442in}{1.189634in}}%
\pgfpathlineto{\pgfqpoint{4.020953in}{1.179807in}}%
\pgfpathlineto{\pgfqpoint{4.042440in}{1.173795in}}%
\pgfpathlineto{\pgfqpoint{4.062904in}{1.171151in}}%
\pgfpathlineto{\pgfqpoint{4.083368in}{1.171514in}}%
\pgfpathlineto{\pgfqpoint{4.103832in}{1.174867in}}%
\pgfpathlineto{\pgfqpoint{4.125320in}{1.181564in}}%
\pgfpathlineto{\pgfqpoint{4.146807in}{1.191453in}}%
\pgfpathlineto{\pgfqpoint{4.169317in}{1.205143in}}%
\pgfpathlineto{\pgfqpoint{4.193874in}{1.223818in}}%
\pgfpathlineto{\pgfqpoint{4.219454in}{1.247217in}}%
\pgfpathlineto{\pgfqpoint{4.247081in}{1.276714in}}%
\pgfpathlineto{\pgfqpoint{4.277777in}{1.314189in}}%
\pgfpathlineto{\pgfqpoint{4.311543in}{1.360452in}}%
\pgfpathlineto{\pgfqpoint{4.350425in}{1.419176in}}%
\pgfpathlineto{\pgfqpoint{4.398515in}{1.497817in}}%
\pgfpathlineto{\pgfqpoint{4.477302in}{1.633727in}}%
\pgfpathlineto{\pgfqpoint{4.550973in}{1.758512in}}%
\pgfpathlineto{\pgfqpoint{4.597017in}{1.830512in}}%
\pgfpathlineto{\pgfqpoint{4.634875in}{1.884203in}}%
\pgfpathlineto{\pgfqpoint{4.668641in}{1.926853in}}%
\pgfpathlineto{\pgfqpoint{4.699337in}{1.960690in}}%
\pgfpathlineto{\pgfqpoint{4.726964in}{1.986715in}}%
\pgfpathlineto{\pgfqpoint{4.752544in}{2.006804in}}%
\pgfpathlineto{\pgfqpoint{4.777101in}{2.022286in}}%
\pgfpathlineto{\pgfqpoint{4.800635in}{2.033504in}}%
\pgfpathlineto{\pgfqpoint{4.823145in}{2.040846in}}%
\pgfpathlineto{\pgfqpoint{4.844632in}{2.044725in}}%
\pgfpathlineto{\pgfqpoint{4.866120in}{2.045535in}}%
\pgfpathlineto{\pgfqpoint{4.887607in}{2.043286in}}%
\pgfpathlineto{\pgfqpoint{4.909094in}{2.038013in}}%
\pgfpathlineto{\pgfqpoint{4.931605in}{2.029309in}}%
\pgfpathlineto{\pgfqpoint{4.955139in}{2.016829in}}%
\pgfpathlineto{\pgfqpoint{4.979695in}{2.000267in}}%
\pgfpathlineto{\pgfqpoint{5.006299in}{1.978463in}}%
\pgfpathlineto{\pgfqpoint{5.034948in}{1.950811in}}%
\pgfpathlineto{\pgfqpoint{5.065645in}{1.916840in}}%
\pgfpathlineto{\pgfqpoint{5.100434in}{1.873613in}}%
\pgfpathlineto{\pgfqpoint{5.141362in}{1.817537in}}%
\pgfpathlineto{\pgfqpoint{5.193545in}{1.740332in}}%
\pgfpathlineto{\pgfqpoint{5.348049in}{1.508002in}}%
\pgfpathlineto{\pgfqpoint{5.388977in}{1.453884in}}%
\pgfpathlineto{\pgfqpoint{5.400233in}{1.440021in}}%
\pgfpathlineto{\pgfqpoint{5.400233in}{1.440021in}}%
\pgfusepath{stroke}%
\end{pgfscope}%
\begin{pgfscope}%
\pgfpathrectangle{\pgfqpoint{0.750000in}{0.500000in}}{\pgfqpoint{4.650000in}{3.020000in}}%
\pgfusepath{clip}%
\pgfsetrectcap%
\pgfsetroundjoin%
\pgfsetlinewidth{1.505625pt}%
\definecolor{currentstroke}{rgb}{0.890196,0.466667,0.760784}%
\pgfsetstrokecolor{currentstroke}%
\pgfsetdash{}{0pt}%
\pgfpathmoveto{\pgfqpoint{0.749767in}{1.193691in}}%
\pgfpathlineto{\pgfqpoint{0.853111in}{1.296209in}}%
\pgfpathlineto{\pgfqpoint{0.933944in}{1.374252in}}%
\pgfpathlineto{\pgfqpoint{0.991244in}{1.425417in}}%
\pgfpathlineto{\pgfqpoint{1.041381in}{1.466197in}}%
\pgfpathlineto{\pgfqpoint{1.088448in}{1.500553in}}%
\pgfpathlineto{\pgfqpoint{1.133469in}{1.529594in}}%
\pgfpathlineto{\pgfqpoint{1.177467in}{1.554275in}}%
\pgfpathlineto{\pgfqpoint{1.220442in}{1.574891in}}%
\pgfpathlineto{\pgfqpoint{1.264439in}{1.592566in}}%
\pgfpathlineto{\pgfqpoint{1.309460in}{1.607297in}}%
\pgfpathlineto{\pgfqpoint{1.356528in}{1.619417in}}%
\pgfpathlineto{\pgfqpoint{1.406665in}{1.629111in}}%
\pgfpathlineto{\pgfqpoint{1.461918in}{1.636600in}}%
\pgfpathlineto{\pgfqpoint{1.525357in}{1.641998in}}%
\pgfpathlineto{\pgfqpoint{1.604143in}{1.645424in}}%
\pgfpathlineto{\pgfqpoint{1.718742in}{1.646971in}}%
\pgfpathlineto{\pgfqpoint{2.276389in}{1.650977in}}%
\pgfpathlineto{\pgfqpoint{2.348013in}{1.656139in}}%
\pgfpathlineto{\pgfqpoint{2.407359in}{1.663451in}}%
\pgfpathlineto{\pgfqpoint{2.459542in}{1.672923in}}%
\pgfpathlineto{\pgfqpoint{2.507633in}{1.684738in}}%
\pgfpathlineto{\pgfqpoint{2.552654in}{1.698904in}}%
\pgfpathlineto{\pgfqpoint{2.596652in}{1.715976in}}%
\pgfpathlineto{\pgfqpoint{2.639626in}{1.735970in}}%
\pgfpathlineto{\pgfqpoint{2.682601in}{1.759395in}}%
\pgfpathlineto{\pgfqpoint{2.726599in}{1.786994in}}%
\pgfpathlineto{\pgfqpoint{2.771620in}{1.818963in}}%
\pgfpathlineto{\pgfqpoint{2.818687in}{1.856195in}}%
\pgfpathlineto{\pgfqpoint{2.870871in}{1.901528in}}%
\pgfpathlineto{\pgfqpoint{2.932263in}{1.959191in}}%
\pgfpathlineto{\pgfqpoint{3.030491in}{2.056514in}}%
\pgfpathlineto{\pgfqpoint{3.113370in}{2.136901in}}%
\pgfpathlineto{\pgfqpoint{3.164530in}{2.182387in}}%
\pgfpathlineto{\pgfqpoint{3.207505in}{2.216548in}}%
\pgfpathlineto{\pgfqpoint{3.244340in}{2.242021in}}%
\pgfpathlineto{\pgfqpoint{3.278106in}{2.261696in}}%
\pgfpathlineto{\pgfqpoint{3.308802in}{2.276119in}}%
\pgfpathlineto{\pgfqpoint{3.337452in}{2.286302in}}%
\pgfpathlineto{\pgfqpoint{3.365079in}{2.292891in}}%
\pgfpathlineto{\pgfqpoint{3.391682in}{2.296053in}}%
\pgfpathlineto{\pgfqpoint{3.417262in}{2.296007in}}%
\pgfpathlineto{\pgfqpoint{3.442842in}{2.292824in}}%
\pgfpathlineto{\pgfqpoint{3.468422in}{2.286417in}}%
\pgfpathlineto{\pgfqpoint{3.494002in}{2.276726in}}%
\pgfpathlineto{\pgfqpoint{3.519582in}{2.263717in}}%
\pgfpathlineto{\pgfqpoint{3.546186in}{2.246665in}}%
\pgfpathlineto{\pgfqpoint{3.573812in}{2.225184in}}%
\pgfpathlineto{\pgfqpoint{3.602462in}{2.198923in}}%
\pgfpathlineto{\pgfqpoint{3.633158in}{2.166429in}}%
\pgfpathlineto{\pgfqpoint{3.665901in}{2.127040in}}%
\pgfpathlineto{\pgfqpoint{3.700690in}{2.080214in}}%
\pgfpathlineto{\pgfqpoint{3.738548in}{2.024013in}}%
\pgfpathlineto{\pgfqpoint{3.781523in}{1.954548in}}%
\pgfpathlineto{\pgfqpoint{3.832683in}{1.865754in}}%
\pgfpathlineto{\pgfqpoint{3.908400in}{1.727338in}}%
\pgfpathlineto{\pgfqpoint{4.006628in}{1.548771in}}%
\pgfpathlineto{\pgfqpoint{4.056765in}{1.464108in}}%
\pgfpathlineto{\pgfqpoint{4.097693in}{1.400730in}}%
\pgfpathlineto{\pgfqpoint{4.133505in}{1.350641in}}%
\pgfpathlineto{\pgfqpoint{4.166248in}{1.309956in}}%
\pgfpathlineto{\pgfqpoint{4.195921in}{1.277784in}}%
\pgfpathlineto{\pgfqpoint{4.223547in}{1.252174in}}%
\pgfpathlineto{\pgfqpoint{4.250151in}{1.231708in}}%
\pgfpathlineto{\pgfqpoint{4.274707in}{1.216630in}}%
\pgfpathlineto{\pgfqpoint{4.298241in}{1.205720in}}%
\pgfpathlineto{\pgfqpoint{4.320752in}{1.198593in}}%
\pgfpathlineto{\pgfqpoint{4.343262in}{1.194741in}}%
\pgfpathlineto{\pgfqpoint{4.364749in}{1.194132in}}%
\pgfpathlineto{\pgfqpoint{4.386237in}{1.196514in}}%
\pgfpathlineto{\pgfqpoint{4.408747in}{1.202187in}}%
\pgfpathlineto{\pgfqpoint{4.431258in}{1.211059in}}%
\pgfpathlineto{\pgfqpoint{4.454791in}{1.223670in}}%
\pgfpathlineto{\pgfqpoint{4.479348in}{1.240343in}}%
\pgfpathlineto{\pgfqpoint{4.505952in}{1.262269in}}%
\pgfpathlineto{\pgfqpoint{4.534601in}{1.290100in}}%
\pgfpathlineto{\pgfqpoint{4.565298in}{1.324374in}}%
\pgfpathlineto{\pgfqpoint{4.599063in}{1.366813in}}%
\pgfpathlineto{\pgfqpoint{4.637945in}{1.420881in}}%
\pgfpathlineto{\pgfqpoint{4.683989in}{1.490446in}}%
\pgfpathlineto{\pgfqpoint{4.751521in}{1.598793in}}%
\pgfpathlineto{\pgfqpoint{4.847702in}{1.752596in}}%
\pgfpathlineto{\pgfqpoint{4.894769in}{1.821894in}}%
\pgfpathlineto{\pgfqpoint{4.933651in}{1.873882in}}%
\pgfpathlineto{\pgfqpoint{4.968440in}{1.915324in}}%
\pgfpathlineto{\pgfqpoint{4.999136in}{1.947289in}}%
\pgfpathlineto{\pgfqpoint{5.027786in}{1.972819in}}%
\pgfpathlineto{\pgfqpoint{5.054389in}{1.992532in}}%
\pgfpathlineto{\pgfqpoint{5.079969in}{2.007671in}}%
\pgfpathlineto{\pgfqpoint{5.104526in}{2.018555in}}%
\pgfpathlineto{\pgfqpoint{5.128060in}{2.025552in}}%
\pgfpathlineto{\pgfqpoint{5.150571in}{2.029057in}}%
\pgfpathlineto{\pgfqpoint{5.173081in}{2.029432in}}%
\pgfpathlineto{\pgfqpoint{5.195592in}{2.026689in}}%
\pgfpathlineto{\pgfqpoint{5.218102in}{2.020866in}}%
\pgfpathlineto{\pgfqpoint{5.241636in}{2.011553in}}%
\pgfpathlineto{\pgfqpoint{5.266193in}{1.998427in}}%
\pgfpathlineto{\pgfqpoint{5.291773in}{1.981208in}}%
\pgfpathlineto{\pgfqpoint{5.319399in}{1.958778in}}%
\pgfpathlineto{\pgfqpoint{5.349072in}{1.930590in}}%
\pgfpathlineto{\pgfqpoint{5.381815in}{1.895068in}}%
\pgfpathlineto{\pgfqpoint{5.400233in}{1.873282in}}%
\pgfpathlineto{\pgfqpoint{5.400233in}{1.873282in}}%
\pgfusepath{stroke}%
\end{pgfscope}%
\begin{pgfscope}%
\pgfpathrectangle{\pgfqpoint{0.750000in}{0.500000in}}{\pgfqpoint{4.650000in}{3.020000in}}%
\pgfusepath{clip}%
\pgfsetrectcap%
\pgfsetroundjoin%
\pgfsetlinewidth{1.505625pt}%
\definecolor{currentstroke}{rgb}{0.498039,0.498039,0.498039}%
\pgfsetstrokecolor{currentstroke}%
\pgfsetdash{}{0pt}%
\pgfpathmoveto{\pgfqpoint{0.749767in}{1.874773in}}%
\pgfpathlineto{\pgfqpoint{0.805021in}{1.833834in}}%
\pgfpathlineto{\pgfqpoint{0.856181in}{1.799682in}}%
\pgfpathlineto{\pgfqpoint{0.905295in}{1.770581in}}%
\pgfpathlineto{\pgfqpoint{0.953385in}{1.745688in}}%
\pgfpathlineto{\pgfqpoint{1.000453in}{1.724749in}}%
\pgfpathlineto{\pgfqpoint{1.048543in}{1.706719in}}%
\pgfpathlineto{\pgfqpoint{1.097657in}{1.691583in}}%
\pgfpathlineto{\pgfqpoint{1.149840in}{1.678787in}}%
\pgfpathlineto{\pgfqpoint{1.205094in}{1.668463in}}%
\pgfpathlineto{\pgfqpoint{1.265463in}{1.660350in}}%
\pgfpathlineto{\pgfqpoint{1.335041in}{1.654218in}}%
\pgfpathlineto{\pgfqpoint{1.418943in}{1.650077in}}%
\pgfpathlineto{\pgfqpoint{1.532519in}{1.647799in}}%
\pgfpathlineto{\pgfqpoint{1.753531in}{1.647151in}}%
\pgfpathlineto{\pgfqpoint{2.349036in}{1.648601in}}%
\pgfpathlineto{\pgfqpoint{2.455450in}{1.652163in}}%
\pgfpathlineto{\pgfqpoint{2.534236in}{1.657771in}}%
\pgfpathlineto{\pgfqpoint{2.600745in}{1.665561in}}%
\pgfpathlineto{\pgfqpoint{2.659067in}{1.675460in}}%
\pgfpathlineto{\pgfqpoint{2.712274in}{1.687557in}}%
\pgfpathlineto{\pgfqpoint{2.762411in}{1.702064in}}%
\pgfpathlineto{\pgfqpoint{2.810502in}{1.719140in}}%
\pgfpathlineto{\pgfqpoint{2.857569in}{1.739089in}}%
\pgfpathlineto{\pgfqpoint{2.904636in}{1.762396in}}%
\pgfpathlineto{\pgfqpoint{2.952727in}{1.789741in}}%
\pgfpathlineto{\pgfqpoint{3.001841in}{1.821292in}}%
\pgfpathlineto{\pgfqpoint{3.054024in}{1.858594in}}%
\pgfpathlineto{\pgfqpoint{3.111324in}{1.903503in}}%
\pgfpathlineto{\pgfqpoint{3.179878in}{1.961410in}}%
\pgfpathlineto{\pgfqpoint{3.392705in}{2.144395in}}%
\pgfpathlineto{\pgfqpoint{3.439772in}{2.178988in}}%
\pgfpathlineto{\pgfqpoint{3.480701in}{2.205361in}}%
\pgfpathlineto{\pgfqpoint{3.517536in}{2.225489in}}%
\pgfpathlineto{\pgfqpoint{3.551302in}{2.240469in}}%
\pgfpathlineto{\pgfqpoint{3.583021in}{2.251163in}}%
\pgfpathlineto{\pgfqpoint{3.612694in}{2.257940in}}%
\pgfpathlineto{\pgfqpoint{3.641344in}{2.261317in}}%
\pgfpathlineto{\pgfqpoint{3.668970in}{2.261469in}}%
\pgfpathlineto{\pgfqpoint{3.695574in}{2.258612in}}%
\pgfpathlineto{\pgfqpoint{3.722177in}{2.252717in}}%
\pgfpathlineto{\pgfqpoint{3.748780in}{2.243716in}}%
\pgfpathlineto{\pgfqpoint{3.776407in}{2.231038in}}%
\pgfpathlineto{\pgfqpoint{3.804033in}{2.214949in}}%
\pgfpathlineto{\pgfqpoint{3.832683in}{2.194679in}}%
\pgfpathlineto{\pgfqpoint{3.862356in}{2.169900in}}%
\pgfpathlineto{\pgfqpoint{3.894075in}{2.139271in}}%
\pgfpathlineto{\pgfqpoint{3.927841in}{2.102176in}}%
\pgfpathlineto{\pgfqpoint{3.963653in}{2.058105in}}%
\pgfpathlineto{\pgfqpoint{4.002535in}{2.005265in}}%
\pgfpathlineto{\pgfqpoint{4.046533in}{1.940067in}}%
\pgfpathlineto{\pgfqpoint{4.098716in}{1.856883in}}%
\pgfpathlineto{\pgfqpoint{4.171364in}{1.734629in}}%
\pgfpathlineto{\pgfqpoint{4.283916in}{1.545379in}}%
\pgfpathlineto{\pgfqpoint{4.335077in}{1.465709in}}%
\pgfpathlineto{\pgfqpoint{4.377028in}{1.405848in}}%
\pgfpathlineto{\pgfqpoint{4.413863in}{1.358439in}}%
\pgfpathlineto{\pgfqpoint{4.447629in}{1.319921in}}%
\pgfpathlineto{\pgfqpoint{4.478325in}{1.289478in}}%
\pgfpathlineto{\pgfqpoint{4.506975in}{1.265320in}}%
\pgfpathlineto{\pgfqpoint{4.533578in}{1.246788in}}%
\pgfpathlineto{\pgfqpoint{4.559158in}{1.232666in}}%
\pgfpathlineto{\pgfqpoint{4.583715in}{1.222625in}}%
\pgfpathlineto{\pgfqpoint{4.607249in}{1.216304in}}%
\pgfpathlineto{\pgfqpoint{4.630783in}{1.213253in}}%
\pgfpathlineto{\pgfqpoint{4.653293in}{1.213408in}}%
\pgfpathlineto{\pgfqpoint{4.675804in}{1.216561in}}%
\pgfpathlineto{\pgfqpoint{4.699337in}{1.223029in}}%
\pgfpathlineto{\pgfqpoint{4.722871in}{1.232684in}}%
\pgfpathlineto{\pgfqpoint{4.747428in}{1.246067in}}%
\pgfpathlineto{\pgfqpoint{4.774031in}{1.264237in}}%
\pgfpathlineto{\pgfqpoint{4.801658in}{1.286948in}}%
\pgfpathlineto{\pgfqpoint{4.831331in}{1.315412in}}%
\pgfpathlineto{\pgfqpoint{4.864073in}{1.351279in}}%
\pgfpathlineto{\pgfqpoint{4.899885in}{1.395215in}}%
\pgfpathlineto{\pgfqpoint{4.940814in}{1.450413in}}%
\pgfpathlineto{\pgfqpoint{4.991974in}{1.524894in}}%
\pgfpathlineto{\pgfqpoint{5.092248in}{1.678024in}}%
\pgfpathlineto{\pgfqpoint{5.155687in}{1.771482in}}%
\pgfpathlineto{\pgfqpoint{5.200708in}{1.832498in}}%
\pgfpathlineto{\pgfqpoint{5.238566in}{1.878862in}}%
\pgfpathlineto{\pgfqpoint{5.272332in}{1.915557in}}%
\pgfpathlineto{\pgfqpoint{5.303028in}{1.944572in}}%
\pgfpathlineto{\pgfqpoint{5.331678in}{1.967561in}}%
\pgfpathlineto{\pgfqpoint{5.358281in}{1.985127in}}%
\pgfpathlineto{\pgfqpoint{5.383861in}{1.998420in}}%
\pgfpathlineto{\pgfqpoint{5.400233in}{2.005020in}}%
\pgfpathlineto{\pgfqpoint{5.400233in}{2.005020in}}%
\pgfusepath{stroke}%
\end{pgfscope}%
\begin{pgfscope}%
\pgfpathrectangle{\pgfqpoint{0.750000in}{0.500000in}}{\pgfqpoint{4.650000in}{3.020000in}}%
\pgfusepath{clip}%
\pgfsetrectcap%
\pgfsetroundjoin%
\pgfsetlinewidth{1.505625pt}%
\definecolor{currentstroke}{rgb}{0.737255,0.741176,0.133333}%
\pgfsetstrokecolor{currentstroke}%
\pgfsetdash{}{0pt}%
\pgfpathmoveto{\pgfqpoint{0.749767in}{1.874773in}}%
\pgfpathlineto{\pgfqpoint{0.805021in}{1.833834in}}%
\pgfpathlineto{\pgfqpoint{0.856181in}{1.799682in}}%
\pgfpathlineto{\pgfqpoint{0.905295in}{1.770581in}}%
\pgfpathlineto{\pgfqpoint{0.953385in}{1.745688in}}%
\pgfpathlineto{\pgfqpoint{1.000453in}{1.724749in}}%
\pgfpathlineto{\pgfqpoint{1.048543in}{1.706719in}}%
\pgfpathlineto{\pgfqpoint{1.097657in}{1.691583in}}%
\pgfpathlineto{\pgfqpoint{1.149840in}{1.678787in}}%
\pgfpathlineto{\pgfqpoint{1.205094in}{1.668463in}}%
\pgfpathlineto{\pgfqpoint{1.265463in}{1.660350in}}%
\pgfpathlineto{\pgfqpoint{1.335041in}{1.654218in}}%
\pgfpathlineto{\pgfqpoint{1.418943in}{1.650077in}}%
\pgfpathlineto{\pgfqpoint{1.532519in}{1.647799in}}%
\pgfpathlineto{\pgfqpoint{1.753531in}{1.647151in}}%
\pgfpathlineto{\pgfqpoint{2.349036in}{1.648601in}}%
\pgfpathlineto{\pgfqpoint{2.455450in}{1.652163in}}%
\pgfpathlineto{\pgfqpoint{2.534236in}{1.657771in}}%
\pgfpathlineto{\pgfqpoint{2.600745in}{1.665561in}}%
\pgfpathlineto{\pgfqpoint{2.659067in}{1.675460in}}%
\pgfpathlineto{\pgfqpoint{2.712274in}{1.687557in}}%
\pgfpathlineto{\pgfqpoint{2.762411in}{1.702064in}}%
\pgfpathlineto{\pgfqpoint{2.810502in}{1.719140in}}%
\pgfpathlineto{\pgfqpoint{2.857569in}{1.739089in}}%
\pgfpathlineto{\pgfqpoint{2.904636in}{1.762396in}}%
\pgfpathlineto{\pgfqpoint{2.952727in}{1.789741in}}%
\pgfpathlineto{\pgfqpoint{3.001841in}{1.821292in}}%
\pgfpathlineto{\pgfqpoint{3.054024in}{1.858594in}}%
\pgfpathlineto{\pgfqpoint{3.111324in}{1.903503in}}%
\pgfpathlineto{\pgfqpoint{3.179878in}{1.961410in}}%
\pgfpathlineto{\pgfqpoint{3.392705in}{2.144395in}}%
\pgfpathlineto{\pgfqpoint{3.439772in}{2.178988in}}%
\pgfpathlineto{\pgfqpoint{3.480701in}{2.205361in}}%
\pgfpathlineto{\pgfqpoint{3.517536in}{2.225489in}}%
\pgfpathlineto{\pgfqpoint{3.551302in}{2.240469in}}%
\pgfpathlineto{\pgfqpoint{3.583021in}{2.251163in}}%
\pgfpathlineto{\pgfqpoint{3.612694in}{2.257940in}}%
\pgfpathlineto{\pgfqpoint{3.641344in}{2.261317in}}%
\pgfpathlineto{\pgfqpoint{3.668970in}{2.261469in}}%
\pgfpathlineto{\pgfqpoint{3.695574in}{2.258612in}}%
\pgfpathlineto{\pgfqpoint{3.722177in}{2.252717in}}%
\pgfpathlineto{\pgfqpoint{3.748780in}{2.243716in}}%
\pgfpathlineto{\pgfqpoint{3.776407in}{2.231038in}}%
\pgfpathlineto{\pgfqpoint{3.804033in}{2.214949in}}%
\pgfpathlineto{\pgfqpoint{3.832683in}{2.194679in}}%
\pgfpathlineto{\pgfqpoint{3.862356in}{2.169900in}}%
\pgfpathlineto{\pgfqpoint{3.894075in}{2.139271in}}%
\pgfpathlineto{\pgfqpoint{3.927841in}{2.102176in}}%
\pgfpathlineto{\pgfqpoint{3.963653in}{2.058105in}}%
\pgfpathlineto{\pgfqpoint{4.002535in}{2.005265in}}%
\pgfpathlineto{\pgfqpoint{4.046533in}{1.940067in}}%
\pgfpathlineto{\pgfqpoint{4.098716in}{1.856883in}}%
\pgfpathlineto{\pgfqpoint{4.171364in}{1.734629in}}%
\pgfpathlineto{\pgfqpoint{4.283916in}{1.545379in}}%
\pgfpathlineto{\pgfqpoint{4.335077in}{1.465709in}}%
\pgfpathlineto{\pgfqpoint{4.377028in}{1.405848in}}%
\pgfpathlineto{\pgfqpoint{4.413863in}{1.358439in}}%
\pgfpathlineto{\pgfqpoint{4.447629in}{1.319921in}}%
\pgfpathlineto{\pgfqpoint{4.478325in}{1.289478in}}%
\pgfpathlineto{\pgfqpoint{4.506975in}{1.265320in}}%
\pgfpathlineto{\pgfqpoint{4.533578in}{1.246788in}}%
\pgfpathlineto{\pgfqpoint{4.559158in}{1.232666in}}%
\pgfpathlineto{\pgfqpoint{4.583715in}{1.222625in}}%
\pgfpathlineto{\pgfqpoint{4.607249in}{1.216304in}}%
\pgfpathlineto{\pgfqpoint{4.630783in}{1.213253in}}%
\pgfpathlineto{\pgfqpoint{4.653293in}{1.213408in}}%
\pgfpathlineto{\pgfqpoint{4.675804in}{1.216561in}}%
\pgfpathlineto{\pgfqpoint{4.699337in}{1.223029in}}%
\pgfpathlineto{\pgfqpoint{4.722871in}{1.232684in}}%
\pgfpathlineto{\pgfqpoint{4.747428in}{1.246067in}}%
\pgfpathlineto{\pgfqpoint{4.774031in}{1.264237in}}%
\pgfpathlineto{\pgfqpoint{4.801658in}{1.286948in}}%
\pgfpathlineto{\pgfqpoint{4.831331in}{1.315412in}}%
\pgfpathlineto{\pgfqpoint{4.864073in}{1.351279in}}%
\pgfpathlineto{\pgfqpoint{4.899885in}{1.395215in}}%
\pgfpathlineto{\pgfqpoint{4.940814in}{1.450413in}}%
\pgfpathlineto{\pgfqpoint{4.991974in}{1.524894in}}%
\pgfpathlineto{\pgfqpoint{5.092248in}{1.678024in}}%
\pgfpathlineto{\pgfqpoint{5.155687in}{1.771482in}}%
\pgfpathlineto{\pgfqpoint{5.200708in}{1.832498in}}%
\pgfpathlineto{\pgfqpoint{5.238566in}{1.878862in}}%
\pgfpathlineto{\pgfqpoint{5.272332in}{1.915557in}}%
\pgfpathlineto{\pgfqpoint{5.303028in}{1.944572in}}%
\pgfpathlineto{\pgfqpoint{5.331678in}{1.967561in}}%
\pgfpathlineto{\pgfqpoint{5.358281in}{1.985127in}}%
\pgfpathlineto{\pgfqpoint{5.383861in}{1.998420in}}%
\pgfpathlineto{\pgfqpoint{5.400233in}{2.005020in}}%
\pgfpathlineto{\pgfqpoint{5.400233in}{2.005020in}}%
\pgfusepath{stroke}%
\end{pgfscope}%
\begin{pgfscope}%
\pgfsetrectcap%
\pgfsetmiterjoin%
\pgfsetlinewidth{0.803000pt}%
\definecolor{currentstroke}{rgb}{0.000000,0.000000,0.000000}%
\pgfsetstrokecolor{currentstroke}%
\pgfsetdash{}{0pt}%
\pgfpathmoveto{\pgfqpoint{0.750000in}{0.500000in}}%
\pgfpathlineto{\pgfqpoint{0.750000in}{3.520000in}}%
\pgfusepath{stroke}%
\end{pgfscope}%
\begin{pgfscope}%
\pgfsetrectcap%
\pgfsetmiterjoin%
\pgfsetlinewidth{0.803000pt}%
\definecolor{currentstroke}{rgb}{0.000000,0.000000,0.000000}%
\pgfsetstrokecolor{currentstroke}%
\pgfsetdash{}{0pt}%
\pgfpathmoveto{\pgfqpoint{5.400000in}{0.500000in}}%
\pgfpathlineto{\pgfqpoint{5.400000in}{3.520000in}}%
\pgfusepath{stroke}%
\end{pgfscope}%
\begin{pgfscope}%
\pgfsetrectcap%
\pgfsetmiterjoin%
\pgfsetlinewidth{0.803000pt}%
\definecolor{currentstroke}{rgb}{0.000000,0.000000,0.000000}%
\pgfsetstrokecolor{currentstroke}%
\pgfsetdash{}{0pt}%
\pgfpathmoveto{\pgfqpoint{0.750000in}{0.500000in}}%
\pgfpathlineto{\pgfqpoint{5.400000in}{0.500000in}}%
\pgfusepath{stroke}%
\end{pgfscope}%
\begin{pgfscope}%
\pgfsetrectcap%
\pgfsetmiterjoin%
\pgfsetlinewidth{0.803000pt}%
\definecolor{currentstroke}{rgb}{0.000000,0.000000,0.000000}%
\pgfsetstrokecolor{currentstroke}%
\pgfsetdash{}{0pt}%
\pgfpathmoveto{\pgfqpoint{0.750000in}{3.520000in}}%
\pgfpathlineto{\pgfqpoint{5.400000in}{3.520000in}}%
\pgfusepath{stroke}%
\end{pgfscope}%
\begin{pgfscope}%
\pgfsetbuttcap%
\pgfsetmiterjoin%
\definecolor{currentfill}{rgb}{1.000000,1.000000,1.000000}%
\pgfsetfillcolor{currentfill}%
\pgfsetfillopacity{0.800000}%
\pgfsetlinewidth{1.003750pt}%
\definecolor{currentstroke}{rgb}{0.800000,0.800000,0.800000}%
\pgfsetstrokecolor{currentstroke}%
\pgfsetstrokeopacity{0.800000}%
\pgfsetdash{}{0pt}%
\pgfpathmoveto{\pgfqpoint{4.511110in}{1.859507in}}%
\pgfpathlineto{\pgfqpoint{5.302778in}{1.859507in}}%
\pgfpathquadraticcurveto{\pgfqpoint{5.330556in}{1.859507in}}{\pgfqpoint{5.330556in}{1.887284in}}%
\pgfpathlineto{\pgfqpoint{5.330556in}{3.422778in}}%
\pgfpathquadraticcurveto{\pgfqpoint{5.330556in}{3.450556in}}{\pgfqpoint{5.302778in}{3.450556in}}%
\pgfpathlineto{\pgfqpoint{4.511110in}{3.450556in}}%
\pgfpathquadraticcurveto{\pgfqpoint{4.483333in}{3.450556in}}{\pgfqpoint{4.483333in}{3.422778in}}%
\pgfpathlineto{\pgfqpoint{4.483333in}{1.887284in}}%
\pgfpathquadraticcurveto{\pgfqpoint{4.483333in}{1.859507in}}{\pgfqpoint{4.511110in}{1.859507in}}%
\pgfpathlineto{\pgfqpoint{4.511110in}{1.859507in}}%
\pgfpathclose%
\pgfusepath{stroke,fill}%
\end{pgfscope}%
\begin{pgfscope}%
\pgfsetrectcap%
\pgfsetroundjoin%
\pgfsetlinewidth{1.505625pt}%
\definecolor{currentstroke}{rgb}{0.121569,0.466667,0.705882}%
\pgfsetstrokecolor{currentstroke}%
\pgfsetdash{}{0pt}%
\pgfpathmoveto{\pgfqpoint{4.538888in}{3.346389in}}%
\pgfpathlineto{\pgfqpoint{4.677777in}{3.346389in}}%
\pgfpathlineto{\pgfqpoint{4.816666in}{3.346389in}}%
\pgfusepath{stroke}%
\end{pgfscope}%
\begin{pgfscope}%
\definecolor{textcolor}{rgb}{0.000000,0.000000,0.000000}%
\pgfsetstrokecolor{textcolor}%
\pgfsetfillcolor{textcolor}%
\pgftext[x=4.927777in,y=3.297778in,left,base]{\color{textcolor}\rmfamily\fontsize{10.000000}{12.000000}\selectfont n= -1}%
\end{pgfscope}%
\begin{pgfscope}%
\pgfsetrectcap%
\pgfsetroundjoin%
\pgfsetlinewidth{1.505625pt}%
\definecolor{currentstroke}{rgb}{1.000000,0.498039,0.054902}%
\pgfsetstrokecolor{currentstroke}%
\pgfsetdash{}{0pt}%
\pgfpathmoveto{\pgfqpoint{4.538888in}{3.152716in}}%
\pgfpathlineto{\pgfqpoint{4.677777in}{3.152716in}}%
\pgfpathlineto{\pgfqpoint{4.816666in}{3.152716in}}%
\pgfusepath{stroke}%
\end{pgfscope}%
\begin{pgfscope}%
\definecolor{textcolor}{rgb}{0.000000,0.000000,0.000000}%
\pgfsetstrokecolor{textcolor}%
\pgfsetfillcolor{textcolor}%
\pgftext[x=4.927777in,y=3.104105in,left,base]{\color{textcolor}\rmfamily\fontsize{10.000000}{12.000000}\selectfont n= 0}%
\end{pgfscope}%
\begin{pgfscope}%
\pgfsetrectcap%
\pgfsetroundjoin%
\pgfsetlinewidth{1.505625pt}%
\definecolor{currentstroke}{rgb}{0.172549,0.627451,0.172549}%
\pgfsetstrokecolor{currentstroke}%
\pgfsetdash{}{0pt}%
\pgfpathmoveto{\pgfqpoint{4.538888in}{2.959043in}}%
\pgfpathlineto{\pgfqpoint{4.677777in}{2.959043in}}%
\pgfpathlineto{\pgfqpoint{4.816666in}{2.959043in}}%
\pgfusepath{stroke}%
\end{pgfscope}%
\begin{pgfscope}%
\definecolor{textcolor}{rgb}{0.000000,0.000000,0.000000}%
\pgfsetstrokecolor{textcolor}%
\pgfsetfillcolor{textcolor}%
\pgftext[x=4.927777in,y=2.910432in,left,base]{\color{textcolor}\rmfamily\fontsize{10.000000}{12.000000}\selectfont n= 1}%
\end{pgfscope}%
\begin{pgfscope}%
\pgfsetrectcap%
\pgfsetroundjoin%
\pgfsetlinewidth{1.505625pt}%
\definecolor{currentstroke}{rgb}{0.839216,0.152941,0.156863}%
\pgfsetstrokecolor{currentstroke}%
\pgfsetdash{}{0pt}%
\pgfpathmoveto{\pgfqpoint{4.538888in}{2.765371in}}%
\pgfpathlineto{\pgfqpoint{4.677777in}{2.765371in}}%
\pgfpathlineto{\pgfqpoint{4.816666in}{2.765371in}}%
\pgfusepath{stroke}%
\end{pgfscope}%
\begin{pgfscope}%
\definecolor{textcolor}{rgb}{0.000000,0.000000,0.000000}%
\pgfsetstrokecolor{textcolor}%
\pgfsetfillcolor{textcolor}%
\pgftext[x=4.927777in,y=2.716759in,left,base]{\color{textcolor}\rmfamily\fontsize{10.000000}{12.000000}\selectfont n= 2}%
\end{pgfscope}%
\begin{pgfscope}%
\pgfsetrectcap%
\pgfsetroundjoin%
\pgfsetlinewidth{1.505625pt}%
\definecolor{currentstroke}{rgb}{0.580392,0.403922,0.741176}%
\pgfsetstrokecolor{currentstroke}%
\pgfsetdash{}{0pt}%
\pgfpathmoveto{\pgfqpoint{4.538888in}{2.571698in}}%
\pgfpathlineto{\pgfqpoint{4.677777in}{2.571698in}}%
\pgfpathlineto{\pgfqpoint{4.816666in}{2.571698in}}%
\pgfusepath{stroke}%
\end{pgfscope}%
\begin{pgfscope}%
\definecolor{textcolor}{rgb}{0.000000,0.000000,0.000000}%
\pgfsetstrokecolor{textcolor}%
\pgfsetfillcolor{textcolor}%
\pgftext[x=4.927777in,y=2.523087in,left,base]{\color{textcolor}\rmfamily\fontsize{10.000000}{12.000000}\selectfont n= 3}%
\end{pgfscope}%
\begin{pgfscope}%
\pgfsetrectcap%
\pgfsetroundjoin%
\pgfsetlinewidth{1.505625pt}%
\definecolor{currentstroke}{rgb}{0.549020,0.337255,0.294118}%
\pgfsetstrokecolor{currentstroke}%
\pgfsetdash{}{0pt}%
\pgfpathmoveto{\pgfqpoint{4.538888in}{2.378025in}}%
\pgfpathlineto{\pgfqpoint{4.677777in}{2.378025in}}%
\pgfpathlineto{\pgfqpoint{4.816666in}{2.378025in}}%
\pgfusepath{stroke}%
\end{pgfscope}%
\begin{pgfscope}%
\definecolor{textcolor}{rgb}{0.000000,0.000000,0.000000}%
\pgfsetstrokecolor{textcolor}%
\pgfsetfillcolor{textcolor}%
\pgftext[x=4.927777in,y=2.329414in,left,base]{\color{textcolor}\rmfamily\fontsize{10.000000}{12.000000}\selectfont n= 4}%
\end{pgfscope}%
\begin{pgfscope}%
\pgfsetrectcap%
\pgfsetroundjoin%
\pgfsetlinewidth{1.505625pt}%
\definecolor{currentstroke}{rgb}{0.890196,0.466667,0.760784}%
\pgfsetstrokecolor{currentstroke}%
\pgfsetdash{}{0pt}%
\pgfpathmoveto{\pgfqpoint{4.538888in}{2.184352in}}%
\pgfpathlineto{\pgfqpoint{4.677777in}{2.184352in}}%
\pgfpathlineto{\pgfqpoint{4.816666in}{2.184352in}}%
\pgfusepath{stroke}%
\end{pgfscope}%
\begin{pgfscope}%
\definecolor{textcolor}{rgb}{0.000000,0.000000,0.000000}%
\pgfsetstrokecolor{textcolor}%
\pgfsetfillcolor{textcolor}%
\pgftext[x=4.927777in,y=2.135741in,left,base]{\color{textcolor}\rmfamily\fontsize{10.000000}{12.000000}\selectfont n= 5}%
\end{pgfscope}%
\begin{pgfscope}%
\pgfsetrectcap%
\pgfsetroundjoin%
\pgfsetlinewidth{1.505625pt}%
\definecolor{currentstroke}{rgb}{0.498039,0.498039,0.498039}%
\pgfsetstrokecolor{currentstroke}%
\pgfsetdash{}{0pt}%
\pgfpathmoveto{\pgfqpoint{4.538888in}{1.990679in}}%
\pgfpathlineto{\pgfqpoint{4.677777in}{1.990679in}}%
\pgfpathlineto{\pgfqpoint{4.816666in}{1.990679in}}%
\pgfusepath{stroke}%
\end{pgfscope}%
\begin{pgfscope}%
\definecolor{textcolor}{rgb}{0.000000,0.000000,0.000000}%
\pgfsetstrokecolor{textcolor}%
\pgfsetfillcolor{textcolor}%
\pgftext[x=4.927777in,y=1.942068in,left,base]{\color{textcolor}\rmfamily\fontsize{10.000000}{12.000000}\selectfont n= 6}%
\end{pgfscope}%
\end{pgfpicture}%
\makeatother%
\endgroup%

	\caption{Bessle Funktion \(J_{k}(\beta)\)}
	\label{fig:bessel}
\end{figure}
TODO Grafik einfügen,
\newline
Nun einmal das Modulierte FM signal im Frequenzspektrum mit den einzelen Summen dargestellt

TODO
Hier wird beschrieben wie die Bessel Funktion der FM im Frequenzspektrum hilft, wieso diese gebrauch wird und ihre Vorteile.
\begin{itemize}
    \item Zuerest einmal die Herleitung von FM zu der Besselfunktion
    \item Im Frequenzspektrum darstellen mit Farben, ersichtlich machen. 
    \item Parameter tuing der Trägerfrequenz, Modulierende frequenz und Beta. 
\end{itemize}


%\subsection{De finibus bonorum et malorum
%\label{fm:subsection:bonorum}}




%
% teil3.tex -- Beispiel-File für Teil 3
%
% (c) 2020 Prof Dr Andreas Müller, Hochschule Rapperswil
%
\section{Fazit
\label{fm:section:fazit}}
\rhead{Zusamenfassend}
Ohne die Bessel-Funktion könnte man die einzelen Peaks der Frequenz
nicht so unabängig von einander berechnen.
Da die Bessel-Funktion schnell abklingt, brauchte es auch nur wenige
Bessel-Koeffizenten um das Nachrichtensignal wieder zurückzugewinnen.




\printbibliography[heading=subbibliography]
\end{refsection}


