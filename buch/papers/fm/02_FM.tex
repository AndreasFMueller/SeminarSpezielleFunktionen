%
% teil1.tex -- Beispiel-File für das Paper
%
% (c) 2020 Prof Dr Andreas Müller, Hochschule Rapperswil
%
\subsection{FM --- Frequenzmodulation
\label{fm:section:teil1}}
Als alternative Parameter, mit denen ein sinusförmiges Trägersignal
\(x_c = A_c \cdot \cos(\omega_c t + \varphi)\) moduliert werden kann,
bietet sich der Phasenwinkel
\(\varphi\)
oder die momentane Frequenzabweichung \(\frac{d\varphi}{dt}\) an.

\subsubsection{Modulation der Phase}
Bei der Phasenmodulation erzeugt das
Nachrichtensignal \(m(t)\) eine Phasenabweichung \(\varphi(t)\)
des modulierten Trägersignals im Vergleich zum nicht-modulierten Träger.
Sie ist proportional zum Nachrichtensignal \(m(t)\) durch eine Skalierung
mit der Phasenhubkonstanten 
\(k_p [\text{rad}]\),
welche die Amplitude des Nachrichtensignals auf die Phasenabweichung des
modulierten Trägersignals abbildet: \(\varphi(t) = k_p \cdot m(t)\).
Damit ergibt sich für das phasenmodulierte Trägersignal:
\[
x_{\text{PM}} (t) = A_c \cdot \cos (\omega_c t + k_p \cdot m(t))
\]
Die modulierte Phase \(\varphi(t)\) verändert dabei auch die
Momentanfrequenz \(f_i\), welche gemäss
\[
f_i = 2\pi \omega_i (t) = \omega_c + \frac{d\varphi(t)}{dt}
\]
berechnet wird.

\subsubsection{Modulation der momentanen Kreisfrequenz}
Bei der Frequenzmodulation ist die Abweichung der momentanen
Kreisfrequenz \(\omega_i\) von der Trägerkreisfrequenz \(\omega_c\)
proportional zum Nachrichtensignal \(m(t)\).
Sie ergibt sich, indem \(m(t)\) mit der Kreisfrequenzhubkonstanten
\(k_f [\text{rad/s}] \)skaliert wird: 
\[
\omega_i (t) = \omega_c + k_f \cdot m(t).
\]
Diese sich zeitlich verändernde Abweichung von der Kreisfrequenz \(\omega_c\)
verursacht gleichzeitig auch Schwankungen der Phase
\[
\varphi (t)
=
\int_{-\infty}^t \omega_i (\tau ) - \omega_c\, d\tau
=
\int_{-\infty}^t k_f \cdot m(t)\,d\tau
\]
der Phase.
Das frequenzmodulierte Signal ist daher von der Form
\[
x_{\text{FM}} (t)
=
A_c \cdot
\cos \biggl(
\omega_c t +  \int_{-\infty}^t k_f  \cdot m ( \tau) \,d\tau
\biggr) 
\]
Die Phase \(\varphi(t)\) hat dabei einen kontinuierlichen Verlauf,
d.~h.~das FM-modulierte Signal \(x_{\text{FM}}(t)\) weist keine Stellen
auf, wo sich die Phase sprunghaft ändert.
Aus diesem Grund spricht man bei frequenzmodulierten Signalen 
von einer Modulation mit kontinuierlicher Phase.
Wie aus diesen Ausführungen hervorgeht, sind Phasenmodulation und
Frequenzmodulation äquivalente Modulationsverfahren.

Beide Verfahren variieren sowohl die Phase \(\varphi\) wie auch die
Momentanfrequenz \(\omega_i.\)
Dadurch kann man leider nicht --- wie vielleicht erhofft ---
je mit einem eigenen Nachrichtensignal ein gemeinsames Trägersignal
unabhängig PM- und FM-modulieren, ohne dass sich diese Modulationen
für den Empfänger untrennbar vermischen würden.
Um die mathematische Behandlung der nicht-linearen Winkelmodulation
etwas zu verkürzen, ist es aufgrund dieser Äquivalenzen gerechtfertigt,
dass PM und FM gemeinsam behandelt werden. 

\subsubsection{Gemeinsamkeiten der Spektren von PM und FM}
Da sich beide Modulationsarten nur durch die Operation des Differenzierenens
unterscheiden, sind sie nahe genug miteinander verwandt um zu versuchen,
%Daher soll im folgenden nur auf die Frequenzmodulation eingegangen werden.
das Spektrum für PM und FM gemeinsam zu berechnen.
Dazu werden die modulierenden Nachrichtensignale \(m_{\text{PM}}(t)\)
bzw.~\(m_{\text{FM}}(t)\) als
\begin{align}
m_{\text{PM}}(t)
&=
A_m \cdot \sin(\omega_m t)
\\
m_{\text{FM}}(t) 
&= 
A_m \cdot \cos(\omega_m t)
\end{align}
angesetzt.

Mit diesen Nachrichtensignalen kann die Phase \(\varphi(t)\) des
modulierten Trägersignals sofort bestimmt werden,
bei PM durch eine einfach Skalierung mit der Phasenhubkonstanten \(k_p\), 
bei FM mit der Frequenzhubkonstanten \(k_f\) und anschliessender Integration, 
wodurch aus dem frequenzmodulierenden Cosinus ebenfalls ein
phasenmodulierender Sinus wird:
\begin{align}
\varphi_{\text{PM}}(t)
&=
k_p \cdot A_m \cdot \sin(\omega_m t)
\\
\varphi_{\text{FM}}(t)
&=
k_f \cdot A_m \cdot \frac{1}{\omega_m} \sin(\omega_m t)
\end{align}

Die Phase verändert sich für PM und FM sinusförmig mit einer maximale
Phasenabweichung \(\beta\),
welche sich --- zur Unterscheidung nachfolgend einmalig als
\(\beta_{\text{PM}}\) bzw. \(\beta_{\text{FM}}\) bezeichnet ---
wie folgt berechnet:
 \begin{align}
\beta_{\text{PM}} &= k_p \cdot A_m
\\
\beta_{\text{FM}} &= k_f \cdot A_m \frac{1}{\omega_m} 
\end{align}
Diese maximale Phasenabweichung \(\beta\) wird auch als Modulationsgrad
der Winkelmodulation bezeichnet, ist aber nur für Eintonsignale,
d.~h.~sinusförmig modulierte PM- und FM-Signale definiert. 
Die Unterscheidung von \(\beta_{\text{PM}}\) und \(\beta_{\text{FM}}(t)\)
in den vorhergehenden Formeln soll nur unterstreichen, dass
der Modulationsgrad für PM und FM unterschiedlich berechnet wird.
In allen nachfolgenden Formeln wird zusammengefasst nur noch ein einziges
\(\beta\) verwendet. 
Für diesen allgemeinen Modulationsgrad \(\beta\) muss dann aber jeweils
die Formel für \(\beta_{\text{PM}}\) bzw. \(\beta_{\text{FM}}(t)\)
verwendet werden, je nachdem ob eine Phasen- oder Frequenzmodulation
vorliegt.


\subsection{Frequenzspektrum}
Um das Frequenzspektrum zu berechnen, müsste man die Fourier-Transformation
des Signals
\[
x(t)
=
\cos(\omega_c t + \beta\sin\omega_m t)
\]
berechnen können, eine auf den ersten Blick recht herausfordernde
Aufgabe.
Der Satz~\ref{fm:satz:spektrum} zeigt jedoch, dass es unter Ausnützung
von Eigenschaften der Bessel-Funktionen möglich ist, das Signal
$x(t)$ direkt, also ohne Berechnung eines Fourier-Integrals,
als Summe von harmonischen Schwingungen zu schreiben.
Die Koeffizienten dieser Summe liefern dann ebenfalls die Information
des Frequenzspektrums.


%Um die Foriertransformation zu berechnen, muss man das Integral
%lösen,
%wen \( m(t) = \beta\sin(\omega_mt) \) ist ind \(A_c = 1\)
%\[
%\textrm{X}_{\text{FM}}(t)
%=
%\int^\infty_{-\infty} \cos (\omega_c \tau +\beta\sin(\omega_m\tau)) \exp^{-2\pi i s \tau}d\tau
%\]
%jedoch einfacher ist es wenn man mit Hilfe der Besselfunktion den einen
%Term \(\cos(\omega_c t+\beta\sin(\omega_mt)),\) wandelt, erhält man
%\[
%\int^\infty_{-\infty}
%\biggl(
%\sum_{k= -\infty}^\infty J_{k}(\beta) \cos((\omega_c+k\omega_m)t)
%\biggr) \exp^{-2\pi i s \tau}d\tau
%\]
%Dieses zu transformien ist einfacher da es wieder Summen sind.
%Nochmals zur erinnerung ergibt ein \( \cos() \) immer zwei
%\(e\)-Funktionen oder zwei Dirac Impulse im Frequenzbereich
%\[
%\cos((\omega_c+k\omega_m)t)
%=
%\frac{1}{2} \cdot e^{+j(\omega_c+k\omega_m)t}
%+
%\frac{1}{2} \cdot e^{-j(\omega_c+k\omega_m)t}
%\]
%Somit entsteht eine Reihe von Sumanden mit verschiedenen Dirac Impulsen
%die von einenander immer den Abstand von \(\omega_m\) haben.
%Wieviele Sumanden es benötigt, ist von der Grösse des \(\beta\) abhängig,
%bei kleinem \(\beta\) sind es nur wenige Summanden, und somit auch wenige
%Diracimpulse wie in Abb.\ref{fig:fm:bessel_fm}.
%Eine äusserst vorsichtige Schätzung der Bandbreite des winkelmodulierten
%Signals kann mit diesen Aussagen zusätzlich noch erfolgen.
%Da die Multiplikation im Zeitbereich einer Faltung im Frequenzbereich
%entspricht, ergibt sich für jeden modulierenden \(\varphi^n (t)\)-Terms
%gerade eine maximale Bandbreite \(B_{n\varphi} = n \cdot B_{n\varphi}\)
%Da jeder Term einer AM-Modulation entspricht, wird für jeden Term jeweils
%die Bandbreite des modulierenden \(\varphi^n (t)\) noch verdoppelt.
%In Bezug auf \(\varphi(t)\) wird insgesamt für das winkelmodulierte Signal
%also nicht nur einfach eine Frequenzverschiebung zur Trägerfrequenz
%\(\pm f_c = \pm \frac{\omega_c}{2\pi n}\) durchgeführt.
%Stattdessen wird \(\varphi(t)\) vor dieser Verschiebung bei der jeweiligen
%Potenzierung \((\varphi(t)\) ausgeprägt nicht-linear verformt.
%Daher spricht man bei PM oder FM auch von einer nicht-linearen oder
%exponentiellen Modulation.
%Diese starken Nicht-Linearitäten verunmöglichen es aber, eine analytische
%Lösung für das PM- bzw. FM-Spektrum von beliebigen modulierenden
%Nachrichtensignalen \(m(t)\) zu finden.
%Für die Berechnung solcher Spektren muss man sich auf numerische Lösungen
%beschränken.
%Zwei Spezialfälle werden denn noch oft analytisch betrachtet einerseits
%die Kleinhubwinkelmodulation 
%und,
%als einziger Vertreter einer Grosshubwinkelmodulation,
%ein Eintonsignal, d.h. ein sinusförmig moduliertes FM- oder PM-Signal.
%Doch diese zwei Unterscheidungen sind nur zur Vollständigkeit hier erwähnt
%worden und können im Skript von Nachrichtentechnik \cite{fm:NAT}
%nachgelesen werden.
%Doch befor wir uns das Frequenzspektrum der Frequenzmodulation ansehen,
%zuerset einmal der Zusammenhang mit der Besselfunktion, 
%die sonst eigentlich bekannt für ihre Differenzialgleichungen
%2.~Ordnung ist.

Die Amplitudenmodulation zeichnet sich dadurch aus, dass eine
Überlagerung von Frequenzen als modulierendes Signal eine Linearkombination
der Spektren ergibt.
Die Frequenzmodulation dagegen ist nicht linear, so dass sich nur
begrenzt von der im folgenden Abschnitt durchgeführten Analyse
auf des Frequenzspektrum eines beliebigen Signals geschlossen
werden kann.
In geschlossener Form lässt sich das Spektrum nur für ein Einton-Signal
durchführen.
Für sehr kleines $\beta$ wird sich zeigen, dass im Satz~\ref{fm:satz:spektrum}
nur wenige Terme berücksichtig werden müssen.


