%
% teil1.tex -- Beispiel-File für das Paper
%
% (c) 2020 Prof Dr Andreas Müller, Hochschule Rapperswil
%
\section{FM- Frequenzmodulation
\label{fm:section:teil1}}
\rhead{FM}
(skript Nat ab Seite 60)
Als weiterer Parameter, um ein sinusförmiges Trägersignal \(x_c = A_c \cdot \cos(\omega_c t + \varphi)\) zu modulieren,
bietet sich neben der Amplitude \(A_c\) auch der Phasenwinkel \(\varphi\) oder die momentane Frequenzabweichung \(\frac{d\varphi}{dt}\) an.
Bei der Phasenmodulation (Englisch: phase modulation, PM) erzeugt das Nachrichtensignal \(m(t)\) eine Phasenabweichung \(\varphi(t)\)
des modulierten Trägersignals im Vergleich zum nicht-modulierten Träger.
Sie ist proportional zum Nachrichtensignal \(m(t)\) durch eine Skalierung mit der Phasenhubkonstanten (Englisch: phase deviation constant) 
\[
    k_p [rad],
\]
welche die Amplitude des Nachrichtensignals auf die Phasenabweichung des
modulierten Trägersignals abbildet: \(\varphi(t) = k_p \cdot m(t)\).
Damit ergibt sich für das phasenmodulierte Trägersignal:
\[
    x_{PM} (t) = A_c \cdot \cos (\omega_c t + k_p \cdot m(t))
\]
Die modulierte Phase \(\varphi(t)\) verändert dabei auch die Momentanfrequenz (Englisch: instantaneous frequency) \(\omega_i\)
, welche wie folgt berechnet wird:
\[
    f_i = 2\pi \omega_i (t) = \omega_c + \frac{d\varphi(t)}{dt}
\]
Bei der Frequenzmodulation (Englisch: frequency modulation, FM) ist die Abweichung der momentanen
Kreisfrequenz \(\omega_i\) von der Trägerkreisfrequenz \(\omega_c\) proportional zum Nachrichtensignal \(m(t)\).
Sie ergibt sich, indem \(m(t)\) mit der (Kreis-)Frequenzhubkonstanten (Englisch: frequency deviation constant) \(k_f [rad/s] \)skaliert wird: 
\[
    \omega_i (t) = \omega_c + k_f \cdot m(t).
\]
Diese sich zeitlich verändernde Abweichung von der Kreisfrequenz \(\omega_c\)
verursacht gleichzeitig auch Schwankungen der Phase \(\varphi(t)\),
welche wie folgt berechnet wird:
\[
    \varphi (t) =
    \int_{-\infty}^t \omega_i (\tau ) - \omega_c\, d\tau =
    \int_{-\infty}^t k_f \cdot m(t)\,d\tau
\]
%\intertext{Somit ergibt sich für das frequenzmodulierte Trägersignal: }
\[
    x_{FM} (t) = A_c \cdot \cos \left( \omega_c t +  \int_{-\infty}^t k_f  \cdot m ( \tau) \,d\tau \right) 
\]
Die Phase \(\varphi(t)\) hat dabei einen kontinuierlichen Verlauf, d.h. das FM-modulierte Signal \(x_{FM}(t)\) weist keine Stellen auf,
 wo sich die Phase sprunghaft ändert. Aus diesem Grund spricht man bei frequenzmodulierten
 Signalen - speziell auch bei digitalen FM-Signalen - von einer Modulation mit kontinuierlicher Phase (Englisch: continuous phase modulation).
Wie aus diesen Ausführungen hervorgeht, sind Phasenmodulation und Frequenzmodulation äquivalente Modulationsverfahren.
Beide variieren sowohl die Phase \(\varphi\) wie auch die Momentanfrequenz \(\omega_i.\)
Dadurch kannman leider nicht - wie vielleicht erhofft - je mit einem eigenen Nachrichtensignal ein gemeinsames Trägersignal unabhängig PM- und FM-modulieren,
 ohne dass sich diese Modulationen für den Empfänger untrennbar vermischen würden.
Um die mathematische Behandlung der nicht-linearen Winkelmodulation etwas zu verkürzen, ist es aufgrund dieser Äquivalenzen gerechtfertigt,
dass PM und FM gemeinsam behandelt werden. 
Da beide nur durch die Operation differenzieren getrennt wird, sind diese zwei Modulationen so miteinenader Verwandt das ich nur auf die Frequenzmodulation eingehe.
Jeweils vor der Modulation bzw. nach der Demodulation kann dann noch eine Differentiation oder 
Integration durchgeführt wird, um von der einen Modulationsart zur anderen zu gelangen.
\cite{fm:NAT}

Um das Spektrum für PM und FM gemeinsam berechnen zu können, werden die modulierenden Nachrich-
tensignale \(m_{PM}(t)\) bzw. \(m_{FM}(t)\) unterschiedlich angesetzt:
\begin{align}
    m_{PM}(t)
    &=
    A_m \cdot \sin(\omega_m t)
    \\
    m_{FM}(t) 
    &= 
    A_m \cdot \cos(\omega_m t)
\end{align}

Mit diesen Nachrichtensignalen kann die Phase \(\varphi(t)\) des modulierten Trägersignals sofort bestimmt werden,
bei PM durch eine einfach Skalierung mit der Phasenhubkonstanten \(k_p\), 
bei FM mit der Frequenzhubkonstanten \(k_f\)und anschliessender Integration, 
wodurch aus dem frequenzmodulierenden Cosinus ebenfalls ein phasenmodulierender Sinus wird:
\begin{align}
    \varphi_{PM}(t) &= k_p \cdot A_m \cdot \sin(\omega_m t)
    \\
    \varphi_{FM}(t) &= k_f \cdot A_m \cdot \frac{1}{\omega_m} \sin(\omega_m t)
\end{align}

Die Phase verändert sich für PM und FM sinusförmig mit einer maximale Phasenabweichung \(\beta\),
 welche sich -- zur Unterscheidung nachfolgend einmalig als \(\beta_{PM}\) bzw. \(\beta_{FM}\) bezeichnet -- wie folgt berechnet:
 \begin{align}
    \beta_{PM} &= k_p \cdot A_m
    \\
    \beta_{FM} &= k_f \cdot A_m \frac{1}{\omega_m} 
\end{align}
Diese maximale Phasenabweichung \(\beta\) wird auch als Modulationsgrad (Englisch: modulation index) der Winkelmodulation bezeichnet, ist aber nur für Eintonsignale,
 d.h. sinusförmig modulierte PM- und FM-Signale definiert. 
Die Unterscheidung von \(\beta_{PM}\) und \(\beta_{FM}(t)\) in den vorhergehenden Formeln soll nur unterstreichen, dass
der Modulationsgrad für PM und FM unterschiedlich berechnet wird.
In allen nachfolgenden Formeln wird zusammengefasst nur noch ein einziges \(\beta\) verwendet. 
Für diesen allgemeinen Modulationsgrad \(\beta\) muss dann aber jeweils die Formel für \(\beta_{PM}\) bzw. \(\beta_{FM}(t)\) verwendet werden,
 je nachdem ob eine Phasen- oder Frequenzmodulation vorliegt.


\subsection{Frequenzspektrum}

Im die Foriertransformation zu berechnen muss man dieses Integral lösen, wen \( m(t) = \beta\sin(\omega_mt) \) ist ind \(A_c = 1\)
\[
    \textrm{X}_{FM} = \int^\infty_{-\infty} \cos (\omega_c \tau +\beta\sin(\omega_m\tau)) \exp^{-2\pi i s \tau}d\tau
\]
jedoch einfacher ist es wenn man mit Hilfe der Besselfunktion den einen Term \(\cos(\omega_c t+\beta\sin(\omega_mt)),\) wandelt,  erhält man
\[
    \int^\infty_{-\infty} \bigl(\sum_{k= -\infty}^\infty J_{k}(\beta) \cos((\omega_c+k\omega_m)t).\bigr) \exp^{-2\pi i s \tau}d\tau
\]
Dieses zu transformien ist einfacher da es wieder Summen sind.
Nochmals zur erinnerung ergibt ein \( \cos() \) immer zwei \(e\)-Funktionen oder zwei Dirac Impulse im Frequenzbereich
\[
    \cos((\omega_c+k\omega_m)t) = \frac{1}{2} \cdot e^{+j(\omega_c+k\omega_m)t}\;+\;\frac{1}{2} \cdot e^{-j(\omega_c+k\omega_m)t}.
\]
Somit entsteht eine Reihe von Sumanden mit verschiedenen Dirac Impulsen die von einenander immer den Abstand von \(\omega_m\) haben.
Wieviele Sumanden es benötigt ist von der grösse des \(\beta\) abhängig, bei kleinem \(\beta\) sind es nur wenige Summanden, und somit auch wenige Diracimpulse wie in Abb.\ref{fig:fm:bessel_fm}.
Eine äusserst vorsichtige Schätzung der Bandbreite des winkelmodulierten Signals kann mit diesen Aussagen zusätzlich noch erfolgen.
Da die Multiplikation im Zeitbereich einer Faltung im Frequenzbereich entspricht, ergibt sich für jeden modulierenden \(\varphi^n (t)\)-Terms gerade 
eine maximale Bandbreite \(B_{n\varphi} = n \cdot B_{n\varphi}\)
Da jeder Term einer AM-Modulation entspricht, wird für jeden Term jeweils die Bandbreite des modulierenden \(\varphi^n (t)\) noch verdoppelt.
In Bezug auf \(\varphi(t)\)wird insgesamt für das winkelmodulierte Signal also nicht nur einfach eine 
Frequenzverschiebung zur Trägerfrequenz \(\pm f_c = \pm \frac{\omega_c}{2\pi n}\) durchgeführt.
Stattdessen wird \(\varphi(t)\) vor dieser Verschiebung bei der jeweiligen Potenzierung \((\varphi(t)\) ausgeprägt nicht-linear verformt.
Daher spricht man bei PM oder FM auch von einer nicht-linearen oder exponentiellen Modulation.
Diese starken Nicht-Linearitäten verunmöglichen es aber, eine analytische Lösung für das PM- bzw. FM-Spektrum von beliebigen modulierenden Nachrichtensignalen \(m(t)\) zu finden.
Für die Berechnung solcher Spektren muss man sich auf numerische Lösungen beschränken.
Zwei Spezialfälle werden denn noch oft analytisch betrachtet einerseits die Kleinhubwinkelmodulation (Englisch: narrowband angle modulation) und, als
einziger Vertreter einer Grosshubwinkelmodulation (Englisch: wideband angle modulation), ein Eintonsignal, d.h. ein sinusförmig moduliertes FM- oder PM-Signal.
Doch diese zwei unterscheidungen sind nur zur Vollständigkeit hier erwähnt worden und können im Skript von Nachrichtentechnik \cite{fm:NAT} nachgelesen werden.
Doch befor wir uns das Frequenzspektrum der Frequenzmodulation ansehen, zuerset einmal der Zusammenhang mit der Besselfunktion, 
die sonst eigentlich bekannt für ihre Differenzialgleichungen 2.Ordnung ist.

\
%Nun
%TODO
%Hier Beschreiben ich FM und FM im Frequenzspektrum.
%Sed ut perspiciatis unde omnis iste natus error sit voluptatem
%accusantium doloremque laudantium, totam rem aperiam, eaque ipsa
%quae ab illo inventore veritatis et quasi architecto beatae vitae
%dicta sunt explicabo.
%Nemo enim ipsam voluptatem quia voluptas sit aspernatur aut odit
%aut fugit, sed quia consequuntur magni dolores eos qui ratione
%voluptatem sequi nesciunt
%\begin{equation}
%\int_a^b x^2\, dx
%=
%\left[ \frac13 x^3 \right]_a^b
%=
%\frac{b^3-a^3}3.
%\label{fm:equation1}
%\end{equation}
%Neque porro quisquam est, qui dolorem ipsum quia dolor sit amet,
%consectetur, adipisci velit, sed quia non numquam eius modi tempora
%incidunt ut labore et dolore magnam aliquam quaerat voluptatem.
%
%Ut enim ad minima veniam, quis nostrum exercitationem ullam corporis
%suscipit laboriosam, nisi ut aliquid ex ea commodi consequatur?
%Quis autem vel eum iure reprehenderit qui in ea voluptate velit
%esse quam nihil molestiae consequatur, vel illum qui dolorem eum
%fugiat quo voluptas nulla pariatur?
%
%\subsection{De finibus bonorum et malorum
%\label{fm:subsection:finibus}}
%At vero eos et accusamus et iusto odio dignissimos ducimus qui
%blanditiis praesentium voluptatum deleniti atque corrupti quos
%dolores et quas molestias excepturi sint occaecati cupiditate non
%provident, similique sunt in culpa qui officia deserunt mollitia
%animi, id est laborum et dolorum fuga \eqref{000tempmlate:equation1}.
%
%Et harum quidem rerum facilis est et expedita distinctio
%\ref{fm:section:loesung}.
%Nam libero tempore, cum soluta nobis est eligendi optio cumque nihil
%impedit quo minus id quod maxime placeat facere possimus, omnis
%voluptas assumenda est, omnis dolor repellendus
%\ref{fm:section:folgerung}.
%Temporibus autem quibusdam et aut officiis debitis aut rerum
%necessitatibus saepe eveniet ut et voluptates repudiandae sint et
%molestiae non recusandae.
%Itaque earum rerum hic tenetur a sapiente delectus, ut aut reiciendis
%voluptatibus maiores alias consequatur aut perferendis doloribus
%asperiores repellat.
