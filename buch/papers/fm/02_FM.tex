%
% teil1.tex -- Beispiel-File für das Paper
%
% (c) 2020 Prof Dr Andreas Müller, Hochschule Rapperswil
%
\section{FM
\label{fm:section:teil1}}
\rhead{FM}
\subsection{Frequenzmodulation}
(skript Nat ab Seite 60)
Als weiterer Parameter, um ein sinusförmiges Trägersignal \(x_c = A_c \cdot \cos(\omega_c t + \varphi)\) zu modulieren,
bietet sich neben der Amplitude \(A_c\) auch der Phasenwinkel \(\varphi\) oder die momentane Frequenzabweichung \(\frac{d\varphi}{dt}\) an.
Bei der Phasenmodulation (Englisch: phase modulation, PM) erzeugt das Nachrichtensignal \(m(t)\) eine Phasenabweichung \(\varphi(t)\) des modulierten Trägersignals im Vergleich zum nicht-modulierten Träger. Sie ist pro-
%portional zum Nachrichtensignal \(m(t)\) durch eine Skalierung mit der Phasenhubkonstanten (Englisch: phase deviation constant) 
%k p [rad],
%welche die Amplitude des Nachrichtensignals auf die Phasenabweichung des
%modulierten Trägersignals abbildet: φ(t) = k p · m(t). Damit ergibt sich für das phasenmodulierte Trägersi-
%gnal:
%x PM (t) = A c · cos (ω c t + k p · m(t))
%(5.16)
%Die modulierte Phase φ(t) verändert dabei auch die Momentanfrequenz (Englisch: instantaneous frequency)
%ω i
%, welche wie folgt berechnet wird:
%f i = 2π
%ω i (t) = ω c +
%d φ(t)
%dt
%(5.17)
%Bei der Frequenzmodulation (Englisch: frequency modulation, FM) ist die Abweichung der momentanen
%Kreisfrequenz ω i von der Trägerkreisfrequenz ω c proportional zum Nachrichtensignal m(t). Sie ergibt sich,
%indem m(t) mit der (Kreis-)Frequenzhubkonstanten (Englisch: frequency deviation constant) k f [rad/s] ska-
%liert wird: ω i (t) = ω c + k f · m(t). Diese sich zeitlich verändernde Abweichung von der Kreisfrequenz ω c
%verursacht gleichzeitig auch Schwankungen der Phase φ(t), welche wie folgt berechnet wird:
%φ(t) =
%Z t
%−∞
%ω i (τ ) − ω c dτ =
%Somit ergibt sich für das frequenzmodulierte Trägersignal:
%
%Z t
%−∞
%x FM (t) = A c · cos  ω c t + k f
%k f · m(t) dτ
%Z t
%−∞
%
%m(τ ) dτ 
%(5.18)
%(5.19)
%Die Phase φ(t) hat dabei einen kontinuierlichen Verlauf, d.h. das FM-modulierte Signal x FM (t) weist keine
%Stellen auf, wo sich die Phase sprunghaft ändert. Aus diesem Grund spricht man bei frequenzmodulierten
%Signalen – speziell auch bei digitalen FM-Signalen – von einer Modulation mit kontinuierlicher Phase (Eng-
%lisch: continuous phase modulation).
%Wie aus diesen Ausführungen hervorgeht, sind Phasenmodulation und Frequenzmodulation äquivalente Mo-
%dulationsverfahren. Beide variieren sowohl die Phase φ wie auch die Momentanfrequenz ω i . Dadurch kann
%man leider nicht – wie vielleicht erhofft – je mit einem eigenen Nachrichtensignal ein gemeinsames Trägersi-
%gnal unabhängig PM- und FM-modulieren, ohne dass sich diese Modulationen für den Empfänger untrennbar
%vermischen würden.
%
%Um die mathematische Behandlung der nicht-linearen Winkelmodulation etwas zu verkürzen, ist es aufgrund
%dieser Äquivalenzen gerechtfertigt, dass PM und FM gemeinsam behandelt werden. Jeweils vor der Modu-
%lation bzw. nach der Demodulation kann dann noch eine Differentiation oder Integration durchgeführt wird,
%um von der einen Modulationsart zur anderen zu gelangen.
%\subsection{Frequenzbereich}
%Nun
%TODO
%Hier Beschreiben ich FM und FM im Frequenzspektrum.
%Sed ut perspiciatis unde omnis iste natus error sit voluptatem
%accusantium doloremque laudantium, totam rem aperiam, eaque ipsa
%quae ab illo inventore veritatis et quasi architecto beatae vitae
%dicta sunt explicabo.
%Nemo enim ipsam voluptatem quia voluptas sit aspernatur aut odit
%aut fugit, sed quia consequuntur magni dolores eos qui ratione
%voluptatem sequi nesciunt
%\begin{equation}
%\int_a^b x^2\, dx
%=
%\left[ \frac13 x^3 \right]_a^b
%=
%\frac{b^3-a^3}3.
%\label{fm:equation1}
%\end{equation}
%Neque porro quisquam est, qui dolorem ipsum quia dolor sit amet,
%consectetur, adipisci velit, sed quia non numquam eius modi tempora
%incidunt ut labore et dolore magnam aliquam quaerat voluptatem.
%
%Ut enim ad minima veniam, quis nostrum exercitationem ullam corporis
%suscipit laboriosam, nisi ut aliquid ex ea commodi consequatur?
%Quis autem vel eum iure reprehenderit qui in ea voluptate velit
%esse quam nihil molestiae consequatur, vel illum qui dolorem eum
%fugiat quo voluptas nulla pariatur?
%
%\subsection{De finibus bonorum et malorum
%\label{fm:subsection:finibus}}
%At vero eos et accusamus et iusto odio dignissimos ducimus qui
%blanditiis praesentium voluptatum deleniti atque corrupti quos
%dolores et quas molestias excepturi sint occaecati cupiditate non
%provident, similique sunt in culpa qui officia deserunt mollitia
%animi, id est laborum et dolorum fuga \eqref{000tempmlate:equation1}.
%
%Et harum quidem rerum facilis est et expedita distinctio
%\ref{fm:section:loesung}.
%Nam libero tempore, cum soluta nobis est eligendi optio cumque nihil
%impedit quo minus id quod maxime placeat facere possimus, omnis
%voluptas assumenda est, omnis dolor repellendus
%\ref{fm:section:folgerung}.
%Temporibus autem quibusdam et aut officiis debitis aut rerum
%necessitatibus saepe eveniet ut et voluptates repudiandae sint et
%molestiae non recusandae.
%Itaque earum rerum hic tenetur a sapiente delectus, ut aut reiciendis
%voluptatibus maiores alias consequatur aut perferendis doloribus
%asperiores repellat.
