%
% teil3.tex -- Beispiel-File für Teil 3
%
% (c) 2020 Prof Dr Andreas Müller, Hochschule Rapperswil
%

Durch die Modulation wird ein Nachrichtensignal \(m(t)\) auf ein Trägersignal (z.B. ein Sinus- oder Rechtecksignal) abgebildet (kombiniert).
Durch dieses Auftragen vom Nachrichtensignal \(m(t)\) kann das modulierte Signal in einem gewünschten Frequenzbereich übertragen werden.
Der ursprünglich Frequenzbereich des Nachrichtensignal \(m(t)\) erstreckt sich typischerweise von 0 Hz bis zur Bandbreite \(B_m\).
Beim Empfänger wird dann durch Demodulation das ursprüngliche Nachrichtensignal \(m(t)\) so originalgetreu wie möglich zurückgewonnen.
Beim Trägersignal \(x_c(t)\) handelt es sich um ein informationsloses Hilfssignal.
Durch die Modulation mit dem Nachrichtensignal \(m(t)\) wird es zum modulierten zu übertragenden Signal.
Für alle Erklärungen wird ein sinusförmiges Trägersignal benutzt, jedoch kann auch ein Rechtecksignal,
welches Digital einfach umzusetzten ist, 
genauso als Trägersignal genutzt werden kann.\cite{fm:NAT}

\subsection{Modulationsarten\label{fm:section:modulation}}

Das sinusförmige Trägersignal hat die übliche Form: 
\(x_c(t) = A_c \cdot \cos(\omega_c(t)+\varphi)\).
Wobei die konstanten Amplitude \(A_c\) und Phase \(\varphi\) vom Nachrichtensignal \(m(t)\) verändert werden können.
Der Parameter \(\omega_c\), die Trägerkreisfrequenz bzw. die Trägerfrequenz \(f_c = \frac{\omega_c}{2\pi}\),
steht nicht für die modulation zur verfügung, statt dessen kann durch ihn die Frequenzachse frei gewählt werden.
\newblockpunct
Jedoch ist das für die Vielfalt der Modulationsarten keine Einschrenkung.
Ein Nachrichtensignal kann auch über die Momentanfrequenz (instantenous frequency) \(\omega_i\) eines trägers verändert werden.
Mathematisch wird dann daraus
\[
    \omega_i = \omega_c + \frac{d \varphi(t)}{dt}
\]
mit der Ableitung der Phase\cite{fm:NAT}.
Mit diesen drei Parameter ergeben sich auch drei Modulationsarten, die Amplitudenmodulation, welche \(A_c\) benutzt, 
die Phasenmodulation \(\varphi\) und dann noch die Momentankreisfrequenz \(\omega_i\): 
\begin{itemize}
    \item AM
    \item PM
    \item FM
\end{itemize}
Um modulation zu Verstehen ist es am Anschaulichst mit der AM Amplitudenmodulation, 
da Phasenmodulation und Frequenzmodulation den gleichen Parameter verändert vernachlässige ich die Phasenmodulation ganz.

To do: Bilder jeder Modulationsart



