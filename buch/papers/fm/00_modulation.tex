%
% teil3.tex -- Beispiel-File für Teil 3
%
% (c) 2020 Prof Dr Andreas Müller, Hochschule Rapperswil
%
\subsection{Modulationsarten\label{fm:section:modulation}}

Das sinusförmige Trägersignal hat die übliche Form: 
\(x_c(t) = A_c \cdot \cos(\omega_c(t)+\varphi)\).
Wobei die konstanten Amplitude \(A_c\) und Phase \(\varphi\) vom Nachrichtensignal \(m(t)\) verändert wird.
Der Parameter \(\omega_c\), die Trägerkreisfrequenz bzw. die Trägerfrequenz \(f_c = \frac{\omega_c}{2\pi}\),
steht nicht für die modulation zur verfügung, statt dessen kann durch ihn die Frequenzachse frei gewählt werden.
\newblockpunct
Jedoch ist das für die Vielfalt der Modulationsarten keine Einschrenkung.
Ein Nachrichtensignal kann auch über die Momentanfrequenz (instantenous frequency) \(\omega_i\) eines trägers verändert werden.
Mathematisch wird dann daraus
\[
    \omega_i = \omega_c + \frac{d \varphi(t)}{dt}
\]
mit der Ableitung der Phase\cite{fm:NAT}.
Mit diesen drei parameter ergeben sich auch drei modulationsarten, die Amplitudenmodulation welche \(A_c\) benutzt, 
die Phasenmodulation \(\varphi\) und dann noch die Momentankreisfrequenz \(\omega_i\):
\newline
\newline
To do: Bilder jeder Modulationsart



