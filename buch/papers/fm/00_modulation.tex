%
% teil3.tex -- Beispiel-File für Teil 3
%
% (c) 2020 Prof Dr Andreas Müller, Hochschule Rapperswil
%

Durch die Modulation wird ein Nachrichtensignal \(m(t)\) auf ein Trägersignal (z.B. ein Sinus- oder Rechtecksignal) abgebildet (kombiniert).
Durch dieses Auftragen vom Nachrichtensignal \(m(t)\) kann das modulierte Signal in einem gewünschten Frequenzbereich übertragen werden.
Der ursprünglich Frequenzbereich des Nachrichtensignal \(m(t)\) erstreckt sich typischerweise von 0 Hz bis zur Bandbreite \(B_m\).
Beim Empfänger wird dann durch Demodulation das ursprüngliche Nachrichtensignal \(m(t)\) so originalgetreu wie möglich zurückgewonnen.
Beim Trägersignal \(x_c(t)\) handelt es sich um ein informationsloses Hilfssignal.
Durch die Modulation mit dem Nachrichtensignal \(m(t)\) wird es zum modulierten zu übertragenden Signal.
Für alle Erklärungen wird ein sinusförmiges Trägersignal benutzt, jedoch kann auch ein Rechtecksignal,
welches Digital einfach umzusetzten ist, 
genauso als Trägersignal genutzt werden kann.\cite{fm:NAT}

\subsection{Modulationsarten\label{fm:section:modulation}}

Das sinusförmige Trägersignal hat die übliche Form: 
\(x_c(t) = A_c \cdot \cos(\omega_c(t)+\varphi)\).
Wobei die konstanten Amplitude \(A_c\) und Phase \(\varphi\) vom Nachrichtensignal \(m(t)\) verändert werden können.
Der Parameter \(\omega_c\), die Trägerkreisfrequenz bzw. die Trägerfrequenz \(f_c = \frac{\omega_c}{2\pi}\),
steht nicht für die modulation zur verfügung, statt dessen kann durch ihn die Frequenzachse frei gewählt werden.
\newblockpunct
Jedoch ist das für die Vielfalt der Modulationsarten keine Einschrenkung.
Ein Nachrichtensignal kann auch über die Momentanfrequenz (instantenous frequency) \(\omega_i\) eines trägers verändert werden.
Mathematisch wird dann daraus
\[
    \omega_i = \omega_c + \frac{d \varphi(t)}{dt}
\]
mit der Ableitung der Phase\cite{fm:NAT}.
Mit diesen drei Parameter ergeben sich auch drei Modulationsarten, die Amplitudenmodulation, welche \(A_c\) Abb.\ref{fig:fm:AM} benutzt, 
die Phasenmodulation \(\varphi\) Abb.\ref{fig:fm:PM} und dann noch die Momentankreisfrequenz \(\omega_i\) in Abb.\ref{fig:fm:FM}: 
\begin{figure} [h]%shot b resized
    \centering
	%% Creator: Matplotlib, PGF backend
%%
%% To include the figure in your LaTeX document, write
%%   \input{<filename>.pgf}
%%
%% Make sure the required packages are loaded in your preamble
%%   \usepackage{pgf}
%%
%% Also ensure that all the required font packages are loaded; for instance,
%% the lmodern package is sometimes necessary when using math font.
%%   \usepackage{lmodern}
%%
%% Figures using additional raster images can only be included by \input if
%% they are in the same directory as the main LaTeX file. For loading figures
%% from other directories you can use the `import` package
%%   \usepackage{import}
%%
%% and then include the figures with
%%   \import{<path to file>}{<filename>.pgf}
%%
%% Matplotlib used the following preamble
%%
\begingroup%
\makeatletter%
\begin{pgfpicture}%
\pgfpathrectangle{\pgfpointorigin}{\pgfqpoint{6.000000in}{4.000000in}}%
\pgfusepath{use as bounding box, clip}%
\begin{pgfscope}%
\pgfsetbuttcap%
\pgfsetmiterjoin%
\pgfsetlinewidth{0.000000pt}%
\definecolor{currentstroke}{rgb}{1.000000,1.000000,1.000000}%
\pgfsetstrokecolor{currentstroke}%
\pgfsetstrokeopacity{0.000000}%
\pgfsetdash{}{0pt}%
\pgfpathmoveto{\pgfqpoint{0.000000in}{0.000000in}}%
\pgfpathlineto{\pgfqpoint{6.000000in}{0.000000in}}%
\pgfpathlineto{\pgfqpoint{6.000000in}{4.000000in}}%
\pgfpathlineto{\pgfqpoint{0.000000in}{4.000000in}}%
\pgfpathlineto{\pgfqpoint{0.000000in}{0.000000in}}%
\pgfpathclose%
\pgfusepath{}%
\end{pgfscope}%
\begin{pgfscope}%
\pgfsetbuttcap%
\pgfsetmiterjoin%
\definecolor{currentfill}{rgb}{1.000000,1.000000,1.000000}%
\pgfsetfillcolor{currentfill}%
\pgfsetlinewidth{0.000000pt}%
\definecolor{currentstroke}{rgb}{0.000000,0.000000,0.000000}%
\pgfsetstrokecolor{currentstroke}%
\pgfsetstrokeopacity{0.000000}%
\pgfsetdash{}{0pt}%
\pgfpathmoveto{\pgfqpoint{0.750000in}{0.500000in}}%
\pgfpathlineto{\pgfqpoint{5.400000in}{0.500000in}}%
\pgfpathlineto{\pgfqpoint{5.400000in}{3.520000in}}%
\pgfpathlineto{\pgfqpoint{0.750000in}{3.520000in}}%
\pgfpathlineto{\pgfqpoint{0.750000in}{0.500000in}}%
\pgfpathclose%
\pgfusepath{fill}%
\end{pgfscope}%
\begin{pgfscope}%
\pgfpathrectangle{\pgfqpoint{0.750000in}{0.500000in}}{\pgfqpoint{4.650000in}{3.020000in}}%
\pgfusepath{clip}%
\pgfsetrectcap%
\pgfsetroundjoin%
\pgfsetlinewidth{1.505625pt}%
\definecolor{currentstroke}{rgb}{0.121569,0.466667,0.705882}%
\pgfsetstrokecolor{currentstroke}%
\pgfsetdash{}{0pt}%
\pgfpathmoveto{\pgfqpoint{0.749993in}{2.358925in}}%
\pgfpathlineto{\pgfqpoint{0.833225in}{2.298790in}}%
\pgfpathlineto{\pgfqpoint{0.888132in}{2.262885in}}%
\pgfpathlineto{\pgfqpoint{0.935239in}{2.235690in}}%
\pgfpathlineto{\pgfqpoint{0.978205in}{2.214389in}}%
\pgfpathlineto{\pgfqpoint{1.018660in}{2.197735in}}%
\pgfpathlineto{\pgfqpoint{1.057586in}{2.185018in}}%
\pgfpathlineto{\pgfqpoint{1.095720in}{2.175789in}}%
\pgfpathlineto{\pgfqpoint{1.133686in}{2.169771in}}%
\pgfpathlineto{\pgfqpoint{1.172110in}{2.166812in}}%
\pgfpathlineto{\pgfqpoint{1.211684in}{2.166881in}}%
\pgfpathlineto{\pgfqpoint{1.253333in}{2.170076in}}%
\pgfpathlineto{\pgfqpoint{1.298442in}{2.176691in}}%
\pgfpathlineto{\pgfqpoint{1.349644in}{2.187414in}}%
\pgfpathlineto{\pgfqpoint{1.414405in}{2.204330in}}%
\pgfpathlineto{\pgfqpoint{1.591917in}{2.252411in}}%
\pgfpathlineto{\pgfqpoint{1.644782in}{2.262440in}}%
\pgfpathlineto{\pgfqpoint{1.691788in}{2.268260in}}%
\pgfpathlineto{\pgfqpoint{1.735736in}{2.270616in}}%
\pgfpathlineto{\pgfqpoint{1.778122in}{2.269809in}}%
\pgfpathlineto{\pgfqpoint{1.819972in}{2.265922in}}%
\pgfpathlineto{\pgfqpoint{1.862201in}{2.258884in}}%
\pgfpathlineto{\pgfqpoint{1.905770in}{2.248462in}}%
\pgfpathlineto{\pgfqpoint{1.951984in}{2.234182in}}%
\pgfpathlineto{\pgfqpoint{2.003030in}{2.215095in}}%
\pgfpathlineto{\pgfqpoint{2.064287in}{2.188732in}}%
\pgfpathlineto{\pgfqpoint{2.277399in}{2.093939in}}%
\pgfpathlineto{\pgfqpoint{2.323445in}{2.078746in}}%
\pgfpathlineto{\pgfqpoint{2.364313in}{2.068454in}}%
\pgfpathlineto{\pgfqpoint{2.401967in}{2.062111in}}%
\pgfpathlineto{\pgfqpoint{2.437500in}{2.059225in}}%
\pgfpathlineto{\pgfqpoint{2.471650in}{2.059530in}}%
\pgfpathlineto{\pgfqpoint{2.504963in}{2.062909in}}%
\pgfpathlineto{\pgfqpoint{2.537907in}{2.069358in}}%
\pgfpathlineto{\pgfqpoint{2.570896in}{2.078977in}}%
\pgfpathlineto{\pgfqpoint{2.604332in}{2.091968in}}%
\pgfpathlineto{\pgfqpoint{2.638649in}{2.108652in}}%
\pgfpathlineto{\pgfqpoint{2.674316in}{2.129475in}}%
\pgfpathlineto{\pgfqpoint{2.711937in}{2.155078in}}%
\pgfpathlineto{\pgfqpoint{2.752358in}{2.186401in}}%
\pgfpathlineto{\pgfqpoint{2.797009in}{2.225017in}}%
\pgfpathlineto{\pgfqpoint{2.848848in}{2.274093in}}%
\pgfpathlineto{\pgfqpoint{2.918096in}{2.344267in}}%
\pgfpathlineto{\pgfqpoint{3.049985in}{2.478530in}}%
\pgfpathlineto{\pgfqpoint{3.100618in}{2.525097in}}%
\pgfpathlineto{\pgfqpoint{3.142903in}{2.559914in}}%
\pgfpathlineto{\pgfqpoint{3.180345in}{2.586844in}}%
\pgfpathlineto{\pgfqpoint{3.214484in}{2.607676in}}%
\pgfpathlineto{\pgfqpoint{3.246189in}{2.623476in}}%
\pgfpathlineto{\pgfqpoint{3.276076in}{2.634975in}}%
\pgfpathlineto{\pgfqpoint{3.304601in}{2.642684in}}%
\pgfpathlineto{\pgfqpoint{3.332144in}{2.646959in}}%
\pgfpathlineto{\pgfqpoint{3.359050in}{2.648025in}}%
\pgfpathlineto{\pgfqpoint{3.385634in}{2.645991in}}%
\pgfpathlineto{\pgfqpoint{3.412194in}{2.640851in}}%
\pgfpathlineto{\pgfqpoint{3.439023in}{2.632491in}}%
\pgfpathlineto{\pgfqpoint{3.466410in}{2.620689in}}%
\pgfpathlineto{\pgfqpoint{3.494645in}{2.605111in}}%
\pgfpathlineto{\pgfqpoint{3.524029in}{2.585316in}}%
\pgfpathlineto{\pgfqpoint{3.554920in}{2.560713in}}%
\pgfpathlineto{\pgfqpoint{3.587742in}{2.530543in}}%
\pgfpathlineto{\pgfqpoint{3.623063in}{2.493781in}}%
\pgfpathlineto{\pgfqpoint{3.661744in}{2.448948in}}%
\pgfpathlineto{\pgfqpoint{3.705312in}{2.393566in}}%
\pgfpathlineto{\pgfqpoint{3.757151in}{2.322447in}}%
\pgfpathlineto{\pgfqpoint{3.831164in}{2.215080in}}%
\pgfpathlineto{\pgfqpoint{3.940086in}{2.057523in}}%
\pgfpathlineto{\pgfqpoint{3.991813in}{1.988382in}}%
\pgfpathlineto{\pgfqpoint{4.034589in}{1.936198in}}%
\pgfpathlineto{\pgfqpoint{4.072243in}{1.894999in}}%
\pgfpathlineto{\pgfqpoint{4.106359in}{1.862131in}}%
\pgfpathlineto{\pgfqpoint{4.137830in}{1.835999in}}%
\pgfpathlineto{\pgfqpoint{4.167226in}{1.815519in}}%
\pgfpathlineto{\pgfqpoint{4.194981in}{1.799874in}}%
\pgfpathlineto{\pgfqpoint{4.221441in}{1.788446in}}%
\pgfpathlineto{\pgfqpoint{4.246920in}{1.780762in}}%
\pgfpathlineto{\pgfqpoint{4.271717in}{1.776480in}}%
\pgfpathlineto{\pgfqpoint{4.296124in}{1.775385in}}%
\pgfpathlineto{\pgfqpoint{4.320419in}{1.777386in}}%
\pgfpathlineto{\pgfqpoint{4.344882in}{1.782514in}}%
\pgfpathlineto{\pgfqpoint{4.369803in}{1.790921in}}%
\pgfpathlineto{\pgfqpoint{4.395448in}{1.802872in}}%
\pgfpathlineto{\pgfqpoint{4.422110in}{1.818756in}}%
\pgfpathlineto{\pgfqpoint{4.450088in}{1.839085in}}%
\pgfpathlineto{\pgfqpoint{4.479740in}{1.864528in}}%
\pgfpathlineto{\pgfqpoint{4.511479in}{1.895924in}}%
\pgfpathlineto{\pgfqpoint{4.545885in}{1.934413in}}%
\pgfpathlineto{\pgfqpoint{4.583841in}{1.981637in}}%
\pgfpathlineto{\pgfqpoint{4.626941in}{2.040364in}}%
\pgfpathlineto{\pgfqpoint{4.678801in}{2.116502in}}%
\pgfpathlineto{\pgfqpoint{4.755448in}{2.235204in}}%
\pgfpathlineto{\pgfqpoint{4.856915in}{2.391416in}}%
\pgfpathlineto{\pgfqpoint{4.909289in}{2.466238in}}%
\pgfpathlineto{\pgfqpoint{4.952724in}{2.523070in}}%
\pgfpathlineto{\pgfqpoint{4.991070in}{2.568299in}}%
\pgfpathlineto{\pgfqpoint{5.025923in}{2.604754in}}%
\pgfpathlineto{\pgfqpoint{5.058164in}{2.634112in}}%
\pgfpathlineto{\pgfqpoint{5.088352in}{2.657514in}}%
\pgfpathlineto{\pgfqpoint{5.116910in}{2.675822in}}%
\pgfpathlineto{\pgfqpoint{5.144185in}{2.689702in}}%
\pgfpathlineto{\pgfqpoint{5.170478in}{2.699664in}}%
\pgfpathlineto{\pgfqpoint{5.196079in}{2.706096in}}%
\pgfpathlineto{\pgfqpoint{5.221268in}{2.709258in}}%
\pgfpathlineto{\pgfqpoint{5.246322in}{2.709297in}}%
\pgfpathlineto{\pgfqpoint{5.271532in}{2.706239in}}%
\pgfpathlineto{\pgfqpoint{5.297189in}{2.699996in}}%
\pgfpathlineto{\pgfqpoint{5.323605in}{2.690354in}}%
\pgfpathlineto{\pgfqpoint{5.351081in}{2.676991in}}%
\pgfpathlineto{\pgfqpoint{5.379974in}{2.659444in}}%
\pgfpathlineto{\pgfqpoint{5.400007in}{2.645289in}}%
\pgfpathlineto{\pgfqpoint{5.400007in}{2.645289in}}%
\pgfusepath{stroke}%
\end{pgfscope}%
\begin{pgfscope}%
\pgfpathrectangle{\pgfqpoint{0.750000in}{0.500000in}}{\pgfqpoint{4.650000in}{3.020000in}}%
\pgfusepath{clip}%
\pgfsetrectcap%
\pgfsetroundjoin%
\pgfsetlinewidth{1.505625pt}%
\definecolor{currentstroke}{rgb}{1.000000,0.498039,0.054902}%
\pgfsetstrokecolor{currentstroke}%
\pgfsetdash{}{0pt}%
\pgfpathmoveto{\pgfqpoint{0.749993in}{2.702951in}}%
\pgfpathlineto{\pgfqpoint{0.753330in}{2.695213in}}%
\pgfpathlineto{\pgfqpoint{0.757928in}{2.675674in}}%
\pgfpathlineto{\pgfqpoint{0.764022in}{2.634765in}}%
\pgfpathlineto{\pgfqpoint{0.771934in}{2.558573in}}%
\pgfpathlineto{\pgfqpoint{0.782391in}{2.425056in}}%
\pgfpathlineto{\pgfqpoint{0.798294in}{2.176912in}}%
\pgfpathlineto{\pgfqpoint{0.825859in}{1.748383in}}%
\pgfpathlineto{\pgfqpoint{0.837711in}{1.612187in}}%
\pgfpathlineto{\pgfqpoint{0.846829in}{1.539560in}}%
\pgfpathlineto{\pgfqpoint{0.853949in}{1.504693in}}%
\pgfpathlineto{\pgfqpoint{0.859295in}{1.491584in}}%
\pgfpathlineto{\pgfqpoint{0.862978in}{1.489084in}}%
\pgfpathlineto{\pgfqpoint{0.865812in}{1.490743in}}%
\pgfpathlineto{\pgfqpoint{0.869328in}{1.497042in}}%
\pgfpathlineto{\pgfqpoint{0.874071in}{1.512743in}}%
\pgfpathlineto{\pgfqpoint{0.880309in}{1.545264in}}%
\pgfpathlineto{\pgfqpoint{0.888411in}{1.605532in}}%
\pgfpathlineto{\pgfqpoint{0.899158in}{1.710906in}}%
\pgfpathlineto{\pgfqpoint{0.916021in}{1.911290in}}%
\pgfpathlineto{\pgfqpoint{0.941466in}{2.208734in}}%
\pgfpathlineto{\pgfqpoint{0.953530in}{2.314451in}}%
\pgfpathlineto{\pgfqpoint{0.962815in}{2.371183in}}%
\pgfpathlineto{\pgfqpoint{0.970091in}{2.398731in}}%
\pgfpathlineto{\pgfqpoint{0.975616in}{2.409315in}}%
\pgfpathlineto{\pgfqpoint{0.979577in}{2.411398in}}%
\pgfpathlineto{\pgfqpoint{0.982859in}{2.409683in}}%
\pgfpathlineto{\pgfqpoint{0.986798in}{2.403597in}}%
\pgfpathlineto{\pgfqpoint{0.991954in}{2.389234in}}%
\pgfpathlineto{\pgfqpoint{0.998639in}{2.360505in}}%
\pgfpathlineto{\pgfqpoint{1.007265in}{2.308497in}}%
\pgfpathlineto{\pgfqpoint{1.018794in}{2.218283in}}%
\pgfpathlineto{\pgfqpoint{1.038581in}{2.034329in}}%
\pgfpathlineto{\pgfqpoint{1.059751in}{1.847370in}}%
\pgfpathlineto{\pgfqpoint{1.071804in}{1.766653in}}%
\pgfpathlineto{\pgfqpoint{1.081056in}{1.723586in}}%
\pgfpathlineto{\pgfqpoint{1.088287in}{1.702927in}}%
\pgfpathlineto{\pgfqpoint{1.093789in}{1.695235in}}%
\pgfpathlineto{\pgfqpoint{1.097874in}{1.694062in}}%
\pgfpathlineto{\pgfqpoint{1.101579in}{1.696322in}}%
\pgfpathlineto{\pgfqpoint{1.106021in}{1.703124in}}%
\pgfpathlineto{\pgfqpoint{1.111712in}{1.718143in}}%
\pgfpathlineto{\pgfqpoint{1.118955in}{1.746840in}}%
\pgfpathlineto{\pgfqpoint{1.128207in}{1.797279in}}%
\pgfpathlineto{\pgfqpoint{1.140594in}{1.883622in}}%
\pgfpathlineto{\pgfqpoint{1.165972in}{2.088974in}}%
\pgfpathlineto{\pgfqpoint{1.182779in}{2.210275in}}%
\pgfpathlineto{\pgfqpoint{1.193950in}{2.270308in}}%
\pgfpathlineto{\pgfqpoint{1.202510in}{2.301174in}}%
\pgfpathlineto{\pgfqpoint{1.209094in}{2.314810in}}%
\pgfpathlineto{\pgfqpoint{1.214016in}{2.318950in}}%
\pgfpathlineto{\pgfqpoint{1.217855in}{2.318516in}}%
\pgfpathlineto{\pgfqpoint{1.221862in}{2.314632in}}%
\pgfpathlineto{\pgfqpoint{1.226839in}{2.304977in}}%
\pgfpathlineto{\pgfqpoint{1.233144in}{2.285287in}}%
\pgfpathlineto{\pgfqpoint{1.241068in}{2.249467in}}%
\pgfpathlineto{\pgfqpoint{1.251145in}{2.188324in}}%
\pgfpathlineto{\pgfqpoint{1.264917in}{2.083798in}}%
\pgfpathlineto{\pgfqpoint{1.305160in}{1.766595in}}%
\pgfpathlineto{\pgfqpoint{1.315338in}{1.714952in}}%
\pgfpathlineto{\pgfqpoint{1.323083in}{1.689820in}}%
\pgfpathlineto{\pgfqpoint{1.328908in}{1.679935in}}%
\pgfpathlineto{\pgfqpoint{1.333149in}{1.677859in}}%
\pgfpathlineto{\pgfqpoint{1.336698in}{1.679506in}}%
\pgfpathlineto{\pgfqpoint{1.340805in}{1.685284in}}%
\pgfpathlineto{\pgfqpoint{1.346039in}{1.698624in}}%
\pgfpathlineto{\pgfqpoint{1.352657in}{1.724814in}}%
\pgfpathlineto{\pgfqpoint{1.360960in}{1.771433in}}%
\pgfpathlineto{\pgfqpoint{1.371562in}{1.850150in}}%
\pgfpathlineto{\pgfqpoint{1.386595in}{1.987601in}}%
\pgfpathlineto{\pgfqpoint{1.418970in}{2.288459in}}%
\pgfpathlineto{\pgfqpoint{1.429862in}{2.359292in}}%
\pgfpathlineto{\pgfqpoint{1.438054in}{2.394468in}}%
\pgfpathlineto{\pgfqpoint{1.444214in}{2.409162in}}%
\pgfpathlineto{\pgfqpoint{1.448633in}{2.413082in}}%
\pgfpathlineto{\pgfqpoint{1.451925in}{2.412309in}}%
\pgfpathlineto{\pgfqpoint{1.455519in}{2.407834in}}%
\pgfpathlineto{\pgfqpoint{1.460150in}{2.396482in}}%
\pgfpathlineto{\pgfqpoint{1.466110in}{2.372789in}}%
\pgfpathlineto{\pgfqpoint{1.473632in}{2.328980in}}%
\pgfpathlineto{\pgfqpoint{1.483140in}{2.253710in}}%
\pgfpathlineto{\pgfqpoint{1.495851in}{2.125994in}}%
\pgfpathlineto{\pgfqpoint{1.541384in}{1.644196in}}%
\pgfpathlineto{\pgfqpoint{1.550547in}{1.588305in}}%
\pgfpathlineto{\pgfqpoint{1.557488in}{1.562667in}}%
\pgfpathlineto{\pgfqpoint{1.562566in}{1.553825in}}%
\pgfpathlineto{\pgfqpoint{1.566059in}{1.552778in}}%
\pgfpathlineto{\pgfqpoint{1.569128in}{1.555286in}}%
\pgfpathlineto{\pgfqpoint{1.573023in}{1.563092in}}%
\pgfpathlineto{\pgfqpoint{1.578145in}{1.581142in}}%
\pgfpathlineto{\pgfqpoint{1.584707in}{1.616764in}}%
\pgfpathlineto{\pgfqpoint{1.593010in}{1.680473in}}%
\pgfpathlineto{\pgfqpoint{1.603724in}{1.788750in}}%
\pgfpathlineto{\pgfqpoint{1.619460in}{1.983180in}}%
\pgfpathlineto{\pgfqpoint{1.648085in}{2.337642in}}%
\pgfpathlineto{\pgfqpoint{1.659357in}{2.438556in}}%
\pgfpathlineto{\pgfqpoint{1.667838in}{2.489643in}}%
\pgfpathlineto{\pgfqpoint{1.674267in}{2.512009in}}%
\pgfpathlineto{\pgfqpoint{1.678876in}{2.518836in}}%
\pgfpathlineto{\pgfqpoint{1.682012in}{2.518989in}}%
\pgfpathlineto{\pgfqpoint{1.685081in}{2.515601in}}%
\pgfpathlineto{\pgfqpoint{1.689121in}{2.505835in}}%
\pgfpathlineto{\pgfqpoint{1.694455in}{2.483874in}}%
\pgfpathlineto{\pgfqpoint{1.701307in}{2.441210in}}%
\pgfpathlineto{\pgfqpoint{1.710023in}{2.365653in}}%
\pgfpathlineto{\pgfqpoint{1.721507in}{2.236576in}}%
\pgfpathlineto{\pgfqpoint{1.740591in}{1.980839in}}%
\pgfpathlineto{\pgfqpoint{1.761750in}{1.710260in}}%
\pgfpathlineto{\pgfqpoint{1.773301in}{1.599621in}}%
\pgfpathlineto{\pgfqpoint{1.782017in}{1.542762in}}%
\pgfpathlineto{\pgfqpoint{1.788679in}{1.517063in}}%
\pgfpathlineto{\pgfqpoint{1.793534in}{1.508591in}}%
\pgfpathlineto{\pgfqpoint{1.796815in}{1.507847in}}%
\pgfpathlineto{\pgfqpoint{1.799794in}{1.510656in}}%
\pgfpathlineto{\pgfqpoint{1.803678in}{1.519249in}}%
\pgfpathlineto{\pgfqpoint{1.808834in}{1.539087in}}%
\pgfpathlineto{\pgfqpoint{1.815508in}{1.578347in}}%
\pgfpathlineto{\pgfqpoint{1.824045in}{1.648746in}}%
\pgfpathlineto{\pgfqpoint{1.835328in}{1.769907in}}%
\pgfpathlineto{\pgfqpoint{1.853876in}{2.008275in}}%
\pgfpathlineto{\pgfqpoint{1.875738in}{2.278060in}}%
\pgfpathlineto{\pgfqpoint{1.887423in}{2.386227in}}%
\pgfpathlineto{\pgfqpoint{1.896295in}{2.442524in}}%
\pgfpathlineto{\pgfqpoint{1.903125in}{2.468508in}}%
\pgfpathlineto{\pgfqpoint{1.908169in}{2.477513in}}%
\pgfpathlineto{\pgfqpoint{1.911651in}{2.478622in}}%
\pgfpathlineto{\pgfqpoint{1.914720in}{2.476158in}}%
\pgfpathlineto{\pgfqpoint{1.918626in}{2.468427in}}%
\pgfpathlineto{\pgfqpoint{1.923805in}{2.450480in}}%
\pgfpathlineto{\pgfqpoint{1.930523in}{2.414872in}}%
\pgfpathlineto{\pgfqpoint{1.939183in}{2.350705in}}%
\pgfpathlineto{\pgfqpoint{1.950790in}{2.239307in}}%
\pgfpathlineto{\pgfqpoint{1.972116in}{1.997805in}}%
\pgfpathlineto{\pgfqpoint{1.990709in}{1.803050in}}%
\pgfpathlineto{\pgfqpoint{2.002204in}{1.712696in}}%
\pgfpathlineto{\pgfqpoint{2.011031in}{1.665325in}}%
\pgfpathlineto{\pgfqpoint{2.017895in}{1.643301in}}%
\pgfpathlineto{\pgfqpoint{2.023028in}{1.635611in}}%
\pgfpathlineto{\pgfqpoint{2.026722in}{1.634736in}}%
\pgfpathlineto{\pgfqpoint{2.030104in}{1.637307in}}%
\pgfpathlineto{\pgfqpoint{2.034345in}{1.644973in}}%
\pgfpathlineto{\pgfqpoint{2.039891in}{1.662138in}}%
\pgfpathlineto{\pgfqpoint{2.047078in}{1.695467in}}%
\pgfpathlineto{\pgfqpoint{2.056442in}{1.754947in}}%
\pgfpathlineto{\pgfqpoint{2.069610in}{1.860687in}}%
\pgfpathlineto{\pgfqpoint{2.108135in}{2.181795in}}%
\pgfpathlineto{\pgfqpoint{2.118793in}{2.240538in}}%
\pgfpathlineto{\pgfqpoint{2.127163in}{2.271144in}}%
\pgfpathlineto{\pgfqpoint{2.133725in}{2.284877in}}%
\pgfpathlineto{\pgfqpoint{2.138724in}{2.289212in}}%
\pgfpathlineto{\pgfqpoint{2.142697in}{2.288935in}}%
\pgfpathlineto{\pgfqpoint{2.146827in}{2.285255in}}%
\pgfpathlineto{\pgfqpoint{2.151982in}{2.276028in}}%
\pgfpathlineto{\pgfqpoint{2.158623in}{2.257137in}}%
\pgfpathlineto{\pgfqpoint{2.167249in}{2.222372in}}%
\pgfpathlineto{\pgfqpoint{2.179068in}{2.160584in}}%
\pgfpathlineto{\pgfqpoint{2.226464in}{1.897788in}}%
\pgfpathlineto{\pgfqpoint{2.236274in}{1.866306in}}%
\pgfpathlineto{\pgfqpoint{2.244164in}{1.850321in}}%
\pgfpathlineto{\pgfqpoint{2.250503in}{1.843676in}}%
\pgfpathlineto{\pgfqpoint{2.255715in}{1.842282in}}%
\pgfpathlineto{\pgfqpoint{2.260636in}{1.844205in}}%
\pgfpathlineto{\pgfqpoint{2.266250in}{1.850014in}}%
\pgfpathlineto{\pgfqpoint{2.273225in}{1.862110in}}%
\pgfpathlineto{\pgfqpoint{2.282220in}{1.884479in}}%
\pgfpathlineto{\pgfqpoint{2.294697in}{1.924671in}}%
\pgfpathlineto{\pgfqpoint{2.339248in}{2.075757in}}%
\pgfpathlineto{\pgfqpoint{2.350106in}{2.098699in}}%
\pgfpathlineto{\pgfqpoint{2.359034in}{2.110948in}}%
\pgfpathlineto{\pgfqpoint{2.366579in}{2.116475in}}%
\pgfpathlineto{\pgfqpoint{2.373263in}{2.117738in}}%
\pgfpathlineto{\pgfqpoint{2.379837in}{2.115800in}}%
\pgfpathlineto{\pgfqpoint{2.387124in}{2.110234in}}%
\pgfpathlineto{\pgfqpoint{2.395829in}{2.099402in}}%
\pgfpathlineto{\pgfqpoint{2.406822in}{2.080385in}}%
\pgfpathlineto{\pgfqpoint{2.422245in}{2.046864in}}%
\pgfpathlineto{\pgfqpoint{2.463559in}{1.954223in}}%
\pgfpathlineto{\pgfqpoint{2.475813in}{1.935474in}}%
\pgfpathlineto{\pgfqpoint{2.485511in}{1.925836in}}%
\pgfpathlineto{\pgfqpoint{2.493546in}{1.921900in}}%
\pgfpathlineto{\pgfqpoint{2.500633in}{1.921747in}}%
\pgfpathlineto{\pgfqpoint{2.507552in}{1.924749in}}%
\pgfpathlineto{\pgfqpoint{2.514996in}{1.931535in}}%
\pgfpathlineto{\pgfqpoint{2.523477in}{1.943737in}}%
\pgfpathlineto{\pgfqpoint{2.533454in}{1.963899in}}%
\pgfpathlineto{\pgfqpoint{2.545708in}{1.996172in}}%
\pgfpathlineto{\pgfqpoint{2.563263in}{2.052200in}}%
\pgfpathlineto{\pgfqpoint{2.590147in}{2.137224in}}%
\pgfpathlineto{\pgfqpoint{2.601151in}{2.161978in}}%
\pgfpathlineto{\pgfqpoint{2.609242in}{2.173366in}}%
\pgfpathlineto{\pgfqpoint{2.615436in}{2.177330in}}%
\pgfpathlineto{\pgfqpoint{2.620536in}{2.177151in}}%
\pgfpathlineto{\pgfqpoint{2.625569in}{2.173722in}}%
\pgfpathlineto{\pgfqpoint{2.631294in}{2.165734in}}%
\pgfpathlineto{\pgfqpoint{2.638113in}{2.150429in}}%
\pgfpathlineto{\pgfqpoint{2.646316in}{2.123773in}}%
\pgfpathlineto{\pgfqpoint{2.656326in}{2.079877in}}%
\pgfpathlineto{\pgfqpoint{2.669305in}{2.007791in}}%
\pgfpathlineto{\pgfqpoint{2.713767in}{1.747824in}}%
\pgfpathlineto{\pgfqpoint{2.722159in}{1.721217in}}%
\pgfpathlineto{\pgfqpoint{2.728342in}{1.710509in}}%
\pgfpathlineto{\pgfqpoint{2.732806in}{1.708012in}}%
\pgfpathlineto{\pgfqpoint{2.736411in}{1.709391in}}%
\pgfpathlineto{\pgfqpoint{2.740450in}{1.714671in}}%
\pgfpathlineto{\pgfqpoint{2.745517in}{1.726997in}}%
\pgfpathlineto{\pgfqpoint{2.751834in}{1.751324in}}%
\pgfpathlineto{\pgfqpoint{2.759623in}{1.794736in}}%
\pgfpathlineto{\pgfqpoint{2.769266in}{1.867455in}}%
\pgfpathlineto{\pgfqpoint{2.781843in}{1.988019in}}%
\pgfpathlineto{\pgfqpoint{2.828291in}{2.458589in}}%
\pgfpathlineto{\pgfqpoint{2.836561in}{2.502980in}}%
\pgfpathlineto{\pgfqpoint{2.842632in}{2.521120in}}%
\pgfpathlineto{\pgfqpoint{2.846839in}{2.525813in}}%
\pgfpathlineto{\pgfqpoint{2.849741in}{2.525116in}}%
\pgfpathlineto{\pgfqpoint{2.852921in}{2.520569in}}%
\pgfpathlineto{\pgfqpoint{2.857128in}{2.508387in}}%
\pgfpathlineto{\pgfqpoint{2.862597in}{2.481997in}}%
\pgfpathlineto{\pgfqpoint{2.869494in}{2.431946in}}%
\pgfpathlineto{\pgfqpoint{2.878076in}{2.344982in}}%
\pgfpathlineto{\pgfqpoint{2.888990in}{2.200113in}}%
\pgfpathlineto{\pgfqpoint{2.904759in}{1.944521in}}%
\pgfpathlineto{\pgfqpoint{2.933061in}{1.484778in}}%
\pgfpathlineto{\pgfqpoint{2.943886in}{1.359333in}}%
\pgfpathlineto{\pgfqpoint{2.951877in}{1.298577in}}%
\pgfpathlineto{\pgfqpoint{2.957747in}{1.274126in}}%
\pgfpathlineto{\pgfqpoint{2.961731in}{1.268020in}}%
\pgfpathlineto{\pgfqpoint{2.964187in}{1.268628in}}%
\pgfpathlineto{\pgfqpoint{2.966921in}{1.273291in}}%
\pgfpathlineto{\pgfqpoint{2.970737in}{1.286886in}}%
\pgfpathlineto{\pgfqpoint{2.975826in}{1.317860in}}%
\pgfpathlineto{\pgfqpoint{2.982344in}{1.378590in}}%
\pgfpathlineto{\pgfqpoint{2.990524in}{1.486467in}}%
\pgfpathlineto{\pgfqpoint{3.000925in}{1.668151in}}%
\pgfpathlineto{\pgfqpoint{3.015489in}{1.982723in}}%
\pgfpathlineto{\pgfqpoint{3.048791in}{2.717383in}}%
\pgfpathlineto{\pgfqpoint{3.059281in}{2.875928in}}%
\pgfpathlineto{\pgfqpoint{3.067115in}{2.952967in}}%
\pgfpathlineto{\pgfqpoint{3.072896in}{2.983861in}}%
\pgfpathlineto{\pgfqpoint{3.076780in}{2.991459in}}%
\pgfpathlineto{\pgfqpoint{3.078990in}{2.990940in}}%
\pgfpathlineto{\pgfqpoint{3.081467in}{2.986136in}}%
\pgfpathlineto{\pgfqpoint{3.085038in}{2.971329in}}%
\pgfpathlineto{\pgfqpoint{3.089893in}{2.936335in}}%
\pgfpathlineto{\pgfqpoint{3.096176in}{2.866172in}}%
\pgfpathlineto{\pgfqpoint{3.104122in}{2.739561in}}%
\pgfpathlineto{\pgfqpoint{3.114267in}{2.524125in}}%
\pgfpathlineto{\pgfqpoint{3.128362in}{2.151879in}}%
\pgfpathlineto{\pgfqpoint{3.163962in}{1.184915in}}%
\pgfpathlineto{\pgfqpoint{3.174330in}{0.995800in}}%
\pgfpathlineto{\pgfqpoint{3.182142in}{0.903183in}}%
\pgfpathlineto{\pgfqpoint{3.187945in}{0.865688in}}%
\pgfpathlineto{\pgfqpoint{3.191851in}{0.856283in}}%
\pgfpathlineto{\pgfqpoint{3.193971in}{0.856618in}}%
\pgfpathlineto{\pgfqpoint{3.196270in}{0.861336in}}%
\pgfpathlineto{\pgfqpoint{3.199663in}{0.876573in}}%
\pgfpathlineto{\pgfqpoint{3.204350in}{0.913726in}}%
\pgfpathlineto{\pgfqpoint{3.210477in}{0.989720in}}%
\pgfpathlineto{\pgfqpoint{3.218278in}{1.128690in}}%
\pgfpathlineto{\pgfqpoint{3.228266in}{1.366792in}}%
\pgfpathlineto{\pgfqpoint{3.242160in}{1.779791in}}%
\pgfpathlineto{\pgfqpoint{3.278643in}{2.897839in}}%
\pgfpathlineto{\pgfqpoint{3.289044in}{3.110831in}}%
\pgfpathlineto{\pgfqpoint{3.296934in}{3.216271in}}%
\pgfpathlineto{\pgfqpoint{3.302826in}{3.259706in}}%
\pgfpathlineto{\pgfqpoint{3.306844in}{3.271204in}}%
\pgfpathlineto{\pgfqpoint{3.309009in}{3.271217in}}%
\pgfpathlineto{\pgfqpoint{3.309020in}{3.271206in}}%
\pgfpathlineto{\pgfqpoint{3.311196in}{3.266820in}}%
\pgfpathlineto{\pgfqpoint{3.314466in}{3.252009in}}%
\pgfpathlineto{\pgfqpoint{3.319031in}{3.215019in}}%
\pgfpathlineto{\pgfqpoint{3.325035in}{3.138304in}}%
\pgfpathlineto{\pgfqpoint{3.332724in}{2.996606in}}%
\pgfpathlineto{\pgfqpoint{3.342634in}{2.751666in}}%
\pgfpathlineto{\pgfqpoint{3.356506in}{2.323817in}}%
\pgfpathlineto{\pgfqpoint{3.392776in}{1.171115in}}%
\pgfpathlineto{\pgfqpoint{3.403322in}{0.945449in}}%
\pgfpathlineto{\pgfqpoint{3.411380in}{0.831915in}}%
\pgfpathlineto{\pgfqpoint{3.417451in}{0.783826in}}%
\pgfpathlineto{\pgfqpoint{3.421658in}{0.770159in}}%
\pgfpathlineto{\pgfqpoint{3.423924in}{0.769537in}}%
\pgfpathlineto{\pgfqpoint{3.424013in}{0.769609in}}%
\pgfpathlineto{\pgfqpoint{3.426122in}{0.773437in}}%
\pgfpathlineto{\pgfqpoint{3.429292in}{0.786817in}}%
\pgfpathlineto{\pgfqpoint{3.433756in}{0.820949in}}%
\pgfpathlineto{\pgfqpoint{3.439670in}{0.892722in}}%
\pgfpathlineto{\pgfqpoint{3.447293in}{1.026566in}}%
\pgfpathlineto{\pgfqpoint{3.457181in}{1.259664in}}%
\pgfpathlineto{\pgfqpoint{3.471186in}{1.671140in}}%
\pgfpathlineto{\pgfqpoint{3.506184in}{2.726813in}}%
\pgfpathlineto{\pgfqpoint{3.516998in}{2.950562in}}%
\pgfpathlineto{\pgfqpoint{3.525290in}{3.065059in}}%
\pgfpathlineto{\pgfqpoint{3.531618in}{3.115498in}}%
\pgfpathlineto{\pgfqpoint{3.536115in}{3.131207in}}%
\pgfpathlineto{\pgfqpoint{3.538771in}{3.132581in}}%
\pgfpathlineto{\pgfqpoint{3.540858in}{3.129566in}}%
\pgfpathlineto{\pgfqpoint{3.543939in}{3.118605in}}%
\pgfpathlineto{\pgfqpoint{3.548302in}{3.090064in}}%
\pgfpathlineto{\pgfqpoint{3.554139in}{3.029046in}}%
\pgfpathlineto{\pgfqpoint{3.561705in}{2.914216in}}%
\pgfpathlineto{\pgfqpoint{3.571604in}{2.712507in}}%
\pgfpathlineto{\pgfqpoint{3.585911in}{2.350303in}}%
\pgfpathlineto{\pgfqpoint{3.618633in}{1.505763in}}%
\pgfpathlineto{\pgfqpoint{3.629793in}{1.302982in}}%
\pgfpathlineto{\pgfqpoint{3.638408in}{1.196891in}}%
\pgfpathlineto{\pgfqpoint{3.645060in}{1.148233in}}%
\pgfpathlineto{\pgfqpoint{3.649914in}{1.131549in}}%
\pgfpathlineto{\pgfqpoint{3.652994in}{1.129188in}}%
\pgfpathlineto{\pgfqpoint{3.655193in}{1.131359in}}%
\pgfpathlineto{\pgfqpoint{3.658262in}{1.139668in}}%
\pgfpathlineto{\pgfqpoint{3.662603in}{1.161626in}}%
\pgfpathlineto{\pgfqpoint{3.668429in}{1.208888in}}%
\pgfpathlineto{\pgfqpoint{3.676051in}{1.298540in}}%
\pgfpathlineto{\pgfqpoint{3.686184in}{1.457789in}}%
\pgfpathlineto{\pgfqpoint{3.701652in}{1.756350in}}%
\pgfpathlineto{\pgfqpoint{3.728793in}{2.278557in}}%
\pgfpathlineto{\pgfqpoint{3.740500in}{2.444276in}}%
\pgfpathlineto{\pgfqpoint{3.749551in}{2.532874in}}%
\pgfpathlineto{\pgfqpoint{3.756671in}{2.575804in}}%
\pgfpathlineto{\pgfqpoint{3.762050in}{2.592287in}}%
\pgfpathlineto{\pgfqpoint{3.765755in}{2.595769in}}%
\pgfpathlineto{\pgfqpoint{3.768378in}{2.594452in}}%
\pgfpathlineto{\pgfqpoint{3.771614in}{2.588641in}}%
\pgfpathlineto{\pgfqpoint{3.776078in}{2.573370in}}%
\pgfpathlineto{\pgfqpoint{3.782093in}{2.540485in}}%
\pgfpathlineto{\pgfqpoint{3.790084in}{2.477839in}}%
\pgfpathlineto{\pgfqpoint{3.801166in}{2.363907in}}%
\pgfpathlineto{\pgfqpoint{3.846799in}{1.864910in}}%
\pgfpathlineto{\pgfqpoint{3.856185in}{1.806754in}}%
\pgfpathlineto{\pgfqpoint{3.863696in}{1.777466in}}%
\pgfpathlineto{\pgfqpoint{3.869599in}{1.765207in}}%
\pgfpathlineto{\pgfqpoint{3.874041in}{1.761973in}}%
\pgfpathlineto{\pgfqpoint{3.877657in}{1.762918in}}%
\pgfpathlineto{\pgfqpoint{3.881752in}{1.767597in}}%
\pgfpathlineto{\pgfqpoint{3.887154in}{1.779054in}}%
\pgfpathlineto{\pgfqpoint{3.894486in}{1.802683in}}%
\pgfpathlineto{\pgfqpoint{3.905244in}{1.849093in}}%
\pgfpathlineto{\pgfqpoint{3.935778in}{1.986176in}}%
\pgfpathlineto{\pgfqpoint{3.944773in}{2.010796in}}%
\pgfpathlineto{\pgfqpoint{3.951726in}{2.021851in}}%
\pgfpathlineto{\pgfqpoint{3.957205in}{2.025442in}}%
\pgfpathlineto{\pgfqpoint{3.961881in}{2.025006in}}%
\pgfpathlineto{\pgfqpoint{3.966859in}{2.021203in}}%
\pgfpathlineto{\pgfqpoint{3.973041in}{2.012181in}}%
\pgfpathlineto{\pgfqpoint{3.981378in}{1.993951in}}%
\pgfpathlineto{\pgfqpoint{3.996198in}{1.952506in}}%
\pgfpathlineto{\pgfqpoint{4.010528in}{1.916040in}}%
\pgfpathlineto{\pgfqpoint{4.018485in}{1.903392in}}%
\pgfpathlineto{\pgfqpoint{4.024199in}{1.899378in}}%
\pgfpathlineto{\pgfqpoint{4.028708in}{1.899707in}}%
\pgfpathlineto{\pgfqpoint{4.033183in}{1.903343in}}%
\pgfpathlineto{\pgfqpoint{4.038439in}{1.912027in}}%
\pgfpathlineto{\pgfqpoint{4.044845in}{1.929174in}}%
\pgfpathlineto{\pgfqpoint{4.052724in}{1.959893in}}%
\pgfpathlineto{\pgfqpoint{4.062690in}{2.012315in}}%
\pgfpathlineto{\pgfqpoint{4.076952in}{2.105943in}}%
\pgfpathlineto{\pgfqpoint{4.103603in}{2.281927in}}%
\pgfpathlineto{\pgfqpoint{4.113323in}{2.325456in}}%
\pgfpathlineto{\pgfqpoint{4.120387in}{2.344680in}}%
\pgfpathlineto{\pgfqpoint{4.125487in}{2.350992in}}%
\pgfpathlineto{\pgfqpoint{4.129137in}{2.351320in}}%
\pgfpathlineto{\pgfqpoint{4.132619in}{2.348255in}}%
\pgfpathlineto{\pgfqpoint{4.136949in}{2.339753in}}%
\pgfpathlineto{\pgfqpoint{4.142473in}{2.321317in}}%
\pgfpathlineto{\pgfqpoint{4.149414in}{2.286333in}}%
\pgfpathlineto{\pgfqpoint{4.158108in}{2.225220in}}%
\pgfpathlineto{\pgfqpoint{4.169380in}{2.122025in}}%
\pgfpathlineto{\pgfqpoint{4.187548in}{1.922359in}}%
\pgfpathlineto{\pgfqpoint{4.208629in}{1.700239in}}%
\pgfpathlineto{\pgfqpoint{4.219622in}{1.615027in}}%
\pgfpathlineto{\pgfqpoint{4.227780in}{1.573242in}}%
\pgfpathlineto{\pgfqpoint{4.233885in}{1.555863in}}%
\pgfpathlineto{\pgfqpoint{4.238192in}{1.551227in}}%
\pgfpathlineto{\pgfqpoint{4.241239in}{1.551853in}}%
\pgfpathlineto{\pgfqpoint{4.244542in}{1.556214in}}%
\pgfpathlineto{\pgfqpoint{4.248906in}{1.567819in}}%
\pgfpathlineto{\pgfqpoint{4.254598in}{1.592747in}}%
\pgfpathlineto{\pgfqpoint{4.261863in}{1.639820in}}%
\pgfpathlineto{\pgfqpoint{4.271159in}{1.722173in}}%
\pgfpathlineto{\pgfqpoint{4.283815in}{1.864693in}}%
\pgfpathlineto{\pgfqpoint{4.325587in}{2.357662in}}%
\pgfpathlineto{\pgfqpoint{4.335173in}{2.426542in}}%
\pgfpathlineto{\pgfqpoint{4.342539in}{2.459696in}}%
\pgfpathlineto{\pgfqpoint{4.348052in}{2.472408in}}%
\pgfpathlineto{\pgfqpoint{4.351902in}{2.475007in}}%
\pgfpathlineto{\pgfqpoint{4.354882in}{2.473477in}}%
\pgfpathlineto{\pgfqpoint{4.358486in}{2.467550in}}%
\pgfpathlineto{\pgfqpoint{4.363285in}{2.452915in}}%
\pgfpathlineto{\pgfqpoint{4.369557in}{2.422782in}}%
\pgfpathlineto{\pgfqpoint{4.377682in}{2.367097in}}%
\pgfpathlineto{\pgfqpoint{4.388529in}{2.269230in}}%
\pgfpathlineto{\pgfqpoint{4.406809in}{2.071041in}}%
\pgfpathlineto{\pgfqpoint{4.427522in}{1.856910in}}%
\pgfpathlineto{\pgfqpoint{4.439028in}{1.768066in}}%
\pgfpathlineto{\pgfqpoint{4.447878in}{1.721385in}}%
\pgfpathlineto{\pgfqpoint{4.454808in}{1.699436in}}%
\pgfpathlineto{\pgfqpoint{4.460054in}{1.691544in}}%
\pgfpathlineto{\pgfqpoint{4.463893in}{1.690460in}}%
\pgfpathlineto{\pgfqpoint{4.467397in}{1.692828in}}%
\pgfpathlineto{\pgfqpoint{4.471749in}{1.700032in}}%
\pgfpathlineto{\pgfqpoint{4.477474in}{1.716194in}}%
\pgfpathlineto{\pgfqpoint{4.485052in}{1.747813in}}%
\pgfpathlineto{\pgfqpoint{4.495531in}{1.806243in}}%
\pgfpathlineto{\pgfqpoint{4.539413in}{2.067793in}}%
\pgfpathlineto{\pgfqpoint{4.548017in}{2.094739in}}%
\pgfpathlineto{\pgfqpoint{4.554836in}{2.107110in}}%
\pgfpathlineto{\pgfqpoint{4.560237in}{2.111302in}}%
\pgfpathlineto{\pgfqpoint{4.564779in}{2.111178in}}%
\pgfpathlineto{\pgfqpoint{4.569567in}{2.107721in}}%
\pgfpathlineto{\pgfqpoint{4.575571in}{2.099147in}}%
\pgfpathlineto{\pgfqpoint{4.583796in}{2.081394in}}%
\pgfpathlineto{\pgfqpoint{4.599587in}{2.038063in}}%
\pgfpathlineto{\pgfqpoint{4.612533in}{2.006864in}}%
\pgfpathlineto{\pgfqpoint{4.620267in}{1.995420in}}%
\pgfpathlineto{\pgfqpoint{4.625903in}{1.991925in}}%
\pgfpathlineto{\pgfqpoint{4.630456in}{1.992523in}}%
\pgfpathlineto{\pgfqpoint{4.635087in}{1.996482in}}%
\pgfpathlineto{\pgfqpoint{4.640567in}{2.005660in}}%
\pgfpathlineto{\pgfqpoint{4.647308in}{2.023571in}}%
\pgfpathlineto{\pgfqpoint{4.655789in}{2.055709in}}%
\pgfpathlineto{\pgfqpoint{4.667262in}{2.112599in}}%
\pgfpathlineto{\pgfqpoint{4.699938in}{2.281961in}}%
\pgfpathlineto{\pgfqpoint{4.707516in}{2.303297in}}%
\pgfpathlineto{\pgfqpoint{4.712862in}{2.310531in}}%
\pgfpathlineto{\pgfqpoint{4.716600in}{2.311215in}}%
\pgfpathlineto{\pgfqpoint{4.719993in}{2.308489in}}%
\pgfpathlineto{\pgfqpoint{4.724133in}{2.300636in}}%
\pgfpathlineto{\pgfqpoint{4.729378in}{2.283283in}}%
\pgfpathlineto{\pgfqpoint{4.735896in}{2.249969in}}%
\pgfpathlineto{\pgfqpoint{4.743909in}{2.191453in}}%
\pgfpathlineto{\pgfqpoint{4.753886in}{2.093814in}}%
\pgfpathlineto{\pgfqpoint{4.767367in}{1.928291in}}%
\pgfpathlineto{\pgfqpoint{4.802365in}{1.484329in}}%
\pgfpathlineto{\pgfqpoint{4.811583in}{1.410702in}}%
\pgfpathlineto{\pgfqpoint{4.818313in}{1.378524in}}%
\pgfpathlineto{\pgfqpoint{4.823022in}{1.368375in}}%
\pgfpathlineto{\pgfqpoint{4.825991in}{1.367562in}}%
\pgfpathlineto{\pgfqpoint{4.828580in}{1.370496in}}%
\pgfpathlineto{\pgfqpoint{4.832084in}{1.379986in}}%
\pgfpathlineto{\pgfqpoint{4.836783in}{1.402805in}}%
\pgfpathlineto{\pgfqpoint{4.842820in}{1.449133in}}%
\pgfpathlineto{\pgfqpoint{4.850375in}{1.533319in}}%
\pgfpathlineto{\pgfqpoint{4.859828in}{1.676299in}}%
\pgfpathlineto{\pgfqpoint{4.872271in}{1.915864in}}%
\pgfpathlineto{\pgfqpoint{4.917938in}{2.844338in}}%
\pgfpathlineto{\pgfqpoint{4.926420in}{2.938300in}}%
\pgfpathlineto{\pgfqpoint{4.932703in}{2.978341in}}%
\pgfpathlineto{\pgfqpoint{4.937044in}{2.989984in}}%
\pgfpathlineto{\pgfqpoint{4.939566in}{2.990496in}}%
\pgfpathlineto{\pgfqpoint{4.941843in}{2.986943in}}%
\pgfpathlineto{\pgfqpoint{4.945135in}{2.975008in}}%
\pgfpathlineto{\pgfqpoint{4.949677in}{2.945317in}}%
\pgfpathlineto{\pgfqpoint{4.955592in}{2.883925in}}%
\pgfpathlineto{\pgfqpoint{4.963080in}{2.770820in}}%
\pgfpathlineto{\pgfqpoint{4.972533in}{2.577087in}}%
\pgfpathlineto{\pgfqpoint{4.985133in}{2.249243in}}%
\pgfpathlineto{\pgfqpoint{5.030208in}{1.013845in}}%
\pgfpathlineto{\pgfqpoint{5.039170in}{0.873652in}}%
\pgfpathlineto{\pgfqpoint{5.045899in}{0.810481in}}%
\pgfpathlineto{\pgfqpoint{5.050687in}{0.789450in}}%
\pgfpathlineto{\pgfqpoint{5.053577in}{0.786722in}}%
\pgfpathlineto{\pgfqpoint{5.055474in}{0.789066in}}%
\pgfpathlineto{\pgfqpoint{5.058275in}{0.798543in}}%
\pgfpathlineto{\pgfqpoint{5.062293in}{0.824639in}}%
\pgfpathlineto{\pgfqpoint{5.067672in}{0.882341in}}%
\pgfpathlineto{\pgfqpoint{5.074580in}{0.993199in}}%
\pgfpathlineto{\pgfqpoint{5.083341in}{1.188099in}}%
\pgfpathlineto{\pgfqpoint{5.094813in}{1.518658in}}%
\pgfpathlineto{\pgfqpoint{5.113540in}{2.162924in}}%
\pgfpathlineto{\pgfqpoint{5.135090in}{2.873677in}}%
\pgfpathlineto{\pgfqpoint{5.146518in}{3.155009in}}%
\pgfpathlineto{\pgfqpoint{5.155078in}{3.297732in}}%
\pgfpathlineto{\pgfqpoint{5.161528in}{3.360688in}}%
\pgfpathlineto{\pgfqpoint{5.166092in}{3.380588in}}%
\pgfpathlineto{\pgfqpoint{5.168626in}{3.382614in}}%
\pgfpathlineto{\pgfqpoint{5.168749in}{3.382548in}}%
\pgfpathlineto{\pgfqpoint{5.170635in}{3.379616in}}%
\pgfpathlineto{\pgfqpoint{5.173514in}{3.368221in}}%
\pgfpathlineto{\pgfqpoint{5.177643in}{3.337393in}}%
\pgfpathlineto{\pgfqpoint{5.183167in}{3.270047in}}%
\pgfpathlineto{\pgfqpoint{5.190276in}{3.141674in}}%
\pgfpathlineto{\pgfqpoint{5.199360in}{2.916533in}}%
\pgfpathlineto{\pgfqpoint{5.211514in}{2.531109in}}%
\pgfpathlineto{\pgfqpoint{5.239224in}{1.515702in}}%
\pgfpathlineto{\pgfqpoint{5.254022in}{1.051528in}}%
\pgfpathlineto{\pgfqpoint{5.264323in}{0.816380in}}%
\pgfpathlineto{\pgfqpoint{5.272191in}{0.699366in}}%
\pgfpathlineto{\pgfqpoint{5.278094in}{0.650905in}}%
\pgfpathlineto{\pgfqpoint{5.282134in}{0.637935in}}%
\pgfpathlineto{\pgfqpoint{5.284132in}{0.637645in}}%
\pgfpathlineto{\pgfqpoint{5.284288in}{0.637793in}}%
\pgfpathlineto{\pgfqpoint{5.286386in}{0.642179in}}%
\pgfpathlineto{\pgfqpoint{5.289578in}{0.657366in}}%
\pgfpathlineto{\pgfqpoint{5.294064in}{0.695810in}}%
\pgfpathlineto{\pgfqpoint{5.300013in}{0.776505in}}%
\pgfpathlineto{\pgfqpoint{5.307668in}{0.926613in}}%
\pgfpathlineto{\pgfqpoint{5.317579in}{1.187241in}}%
\pgfpathlineto{\pgfqpoint{5.331607in}{1.646766in}}%
\pgfpathlineto{\pgfqpoint{5.366672in}{2.826091in}}%
\pgfpathlineto{\pgfqpoint{5.377463in}{3.074931in}}%
\pgfpathlineto{\pgfqpoint{5.385744in}{3.202244in}}%
\pgfpathlineto{\pgfqpoint{5.392049in}{3.258044in}}%
\pgfpathlineto{\pgfqpoint{5.396502in}{3.275206in}}%
\pgfpathlineto{\pgfqpoint{5.399002in}{3.276714in}}%
\pgfpathlineto{\pgfqpoint{5.399091in}{3.276660in}}%
\pgfpathlineto{\pgfqpoint{5.400007in}{3.275678in}}%
\pgfpathlineto{\pgfqpoint{5.400007in}{3.275678in}}%
\pgfusepath{stroke}%
\end{pgfscope}%
\begin{pgfscope}%
\pgfsetbuttcap%
\pgfsetroundjoin%
\pgfsetlinewidth{0.803000pt}%
\definecolor{currentstroke}{rgb}{0.000000,0.000000,0.000000}%
\pgfsetstrokecolor{currentstroke}%
\pgfsetdash{}{0pt}%
\pgfsys@defobject{currentmarker}{\pgfqpoint{0.000000in}{0.000000in}}{\pgfqpoint{0.048611in}{0.000000in}}{%
\pgfpathmoveto{\pgfqpoint{0.000000in}{0.000000in}}%
\pgfpathlineto{\pgfqpoint{0.048611in}{0.000000in}}%
\pgfusepath{stroke}%
}%
\begin{pgfscope}%
\pgfsys@transformshift{0.750000in}{2.014900in}%
\pgfsys@useobject{currentmarker}{}%
\end{pgfscope}%
\end{pgfscope}%
\begin{pgfscope}%
\definecolor{textcolor}{rgb}{0.000000,0.000000,0.000000}%
\pgfsetstrokecolor{textcolor}%
\pgfsetfillcolor{textcolor}%
\pgftext[x=0.701389in,y=2.014900in,right,]{\color{textcolor}\rmfamily\fontsize{10.000000}{12.000000}\selectfont \(\displaystyle {0}\)}%
\end{pgfscope}%
\begin{pgfscope}%
\pgfsetrectcap%
\pgfsetroundjoin%
\pgfsetlinewidth{0.803000pt}%
\definecolor{currentstroke}{rgb}{0.000000,0.000000,0.000000}%
\pgfsetstrokecolor{currentstroke}%
\pgfsetdash{}{0pt}%
\pgfpathmoveto{\pgfqpoint{0.750000in}{0.500000in}}%
\pgfpathlineto{\pgfqpoint{0.750000in}{3.520000in}}%
\pgfusepath{stroke}%
\end{pgfscope}%
\begin{pgfscope}%
\pgfsetroundcap%
\pgfsetroundjoin%
\pgfsetlinewidth{1.003750pt}%
\definecolor{currentstroke}{rgb}{0.000000,0.000000,0.000000}%
\pgfsetstrokecolor{currentstroke}%
\pgfsetdash{}{0pt}%
\pgfpathmoveto{\pgfqpoint{0.750000in}{2.014900in}}%
\pgfpathlineto{\pgfqpoint{5.400000in}{2.014900in}}%
\pgfpathlineto{\pgfqpoint{5.523361in}{2.014900in}}%
\pgfusepath{stroke}%
\end{pgfscope}%
\begin{pgfscope}%
\pgfsetroundcap%
\pgfsetroundjoin%
\definecolor{currentfill}{rgb}{0.121569,0.466667,0.705882}%
\pgfsetfillcolor{currentfill}%
\pgfsetlinewidth{1.003750pt}%
\definecolor{currentstroke}{rgb}{0.000000,0.000000,0.000000}%
\pgfsetstrokecolor{currentstroke}%
\pgfsetdash{}{0pt}%
\pgfpathmoveto{\pgfqpoint{5.467805in}{2.042677in}}%
\pgfpathlineto{\pgfqpoint{5.523361in}{2.014900in}}%
\pgfpathlineto{\pgfqpoint{5.467805in}{1.987122in}}%
\pgfpathlineto{\pgfqpoint{5.467805in}{2.042677in}}%
\pgfpathlineto{\pgfqpoint{5.467805in}{2.042677in}}%
\pgfpathclose%
\pgfusepath{stroke,fill}%
\end{pgfscope}%
\begin{pgfscope}%
\definecolor{textcolor}{rgb}{0.000000,0.000000,0.000000}%
\pgfsetstrokecolor{textcolor}%
\pgfsetfillcolor{textcolor}%
\pgftext[x=5.632500in,y=1.917458in,left,base]{\color{textcolor}\rmfamily\fontsize{10.000000}{12.000000}\selectfont \(\displaystyle t\)}%
\end{pgfscope}%
\end{pgfpicture}%
\makeatother%
\endgroup%

	\caption{Modulationsart Amplitudenmodulation mit dem Parameter \(A_c\)}
	\label{fig:fm:AM}

\end{figure}

\begin{figure}[h]
	\centering
	%% Creator: Matplotlib, PGF backend
%%
%% To include the figure in your LaTeX document, write
%%   \input{<filename>.pgf}
%%
%% Make sure the required packages are loaded in your preamble
%%   \usepackage{pgf}
%%
%% Also ensure that all the required font packages are loaded; for instance,
%% the lmodern package is sometimes necessary when using math font.
%%   \usepackage{lmodern}
%%
%% Figures using additional raster images can only be included by \input if
%% they are in the same directory as the main LaTeX file. For loading figures
%% from other directories you can use the `import` package
%%   \usepackage{import}
%%
%% and then include the figures with
%%   \import{<path to file>}{<filename>.pgf}
%%
%% Matplotlib used the following preamble
%%
\begingroup%
\makeatletter%
\begin{pgfpicture}%
\pgfpathrectangle{\pgfpointorigin}{\pgfqpoint{6.000000in}{4.000000in}}%
\pgfusepath{use as bounding box, clip}%
\begin{pgfscope}%
\pgfsetbuttcap%
\pgfsetmiterjoin%
\pgfsetlinewidth{0.000000pt}%
\definecolor{currentstroke}{rgb}{1.000000,1.000000,1.000000}%
\pgfsetstrokecolor{currentstroke}%
\pgfsetstrokeopacity{0.000000}%
\pgfsetdash{}{0pt}%
\pgfpathmoveto{\pgfqpoint{0.000000in}{0.000000in}}%
\pgfpathlineto{\pgfqpoint{6.000000in}{0.000000in}}%
\pgfpathlineto{\pgfqpoint{6.000000in}{4.000000in}}%
\pgfpathlineto{\pgfqpoint{0.000000in}{4.000000in}}%
\pgfpathlineto{\pgfqpoint{0.000000in}{0.000000in}}%
\pgfpathclose%
\pgfusepath{}%
\end{pgfscope}%
\begin{pgfscope}%
\pgfsetbuttcap%
\pgfsetmiterjoin%
\definecolor{currentfill}{rgb}{1.000000,1.000000,1.000000}%
\pgfsetfillcolor{currentfill}%
\pgfsetlinewidth{0.000000pt}%
\definecolor{currentstroke}{rgb}{0.000000,0.000000,0.000000}%
\pgfsetstrokecolor{currentstroke}%
\pgfsetstrokeopacity{0.000000}%
\pgfsetdash{}{0pt}%
\pgfpathmoveto{\pgfqpoint{0.750000in}{0.500000in}}%
\pgfpathlineto{\pgfqpoint{5.400000in}{0.500000in}}%
\pgfpathlineto{\pgfqpoint{5.400000in}{3.520000in}}%
\pgfpathlineto{\pgfqpoint{0.750000in}{3.520000in}}%
\pgfpathlineto{\pgfqpoint{0.750000in}{0.500000in}}%
\pgfpathclose%
\pgfusepath{fill}%
\end{pgfscope}%
\begin{pgfscope}%
\pgfpathrectangle{\pgfqpoint{0.750000in}{0.500000in}}{\pgfqpoint{4.650000in}{3.020000in}}%
\pgfusepath{clip}%
\pgfsetrectcap%
\pgfsetroundjoin%
\pgfsetlinewidth{1.505625pt}%
\definecolor{currentstroke}{rgb}{0.121569,0.466667,0.705882}%
\pgfsetstrokecolor{currentstroke}%
\pgfsetdash{}{0pt}%
\pgfpathmoveto{\pgfqpoint{0.749993in}{2.494653in}}%
\pgfpathlineto{\pgfqpoint{0.827031in}{2.415968in}}%
\pgfpathlineto{\pgfqpoint{0.877999in}{2.368254in}}%
\pgfpathlineto{\pgfqpoint{0.921400in}{2.331745in}}%
\pgfpathlineto{\pgfqpoint{0.960583in}{2.302732in}}%
\pgfpathlineto{\pgfqpoint{0.997032in}{2.279517in}}%
\pgfpathlineto{\pgfqpoint{1.031617in}{2.261094in}}%
\pgfpathlineto{\pgfqpoint{1.064952in}{2.246794in}}%
\pgfpathlineto{\pgfqpoint{1.097528in}{2.236152in}}%
\pgfpathlineto{\pgfqpoint{1.129780in}{2.228847in}}%
\pgfpathlineto{\pgfqpoint{1.162167in}{2.224674in}}%
\pgfpathlineto{\pgfqpoint{1.195156in}{2.223546in}}%
\pgfpathlineto{\pgfqpoint{1.229316in}{2.225491in}}%
\pgfpathlineto{\pgfqpoint{1.265374in}{2.230675in}}%
\pgfpathlineto{\pgfqpoint{1.304435in}{2.239469in}}%
\pgfpathlineto{\pgfqpoint{1.348450in}{2.252636in}}%
\pgfpathlineto{\pgfqpoint{1.402297in}{2.272134in}}%
\pgfpathlineto{\pgfqpoint{1.601369in}{2.347457in}}%
\pgfpathlineto{\pgfqpoint{1.645529in}{2.358893in}}%
\pgfpathlineto{\pgfqpoint{1.685460in}{2.366090in}}%
\pgfpathlineto{\pgfqpoint{1.722935in}{2.369739in}}%
\pgfpathlineto{\pgfqpoint{1.759005in}{2.370165in}}%
\pgfpathlineto{\pgfqpoint{1.794393in}{2.367498in}}%
\pgfpathlineto{\pgfqpoint{1.829737in}{2.361730in}}%
\pgfpathlineto{\pgfqpoint{1.865639in}{2.352721in}}%
\pgfpathlineto{\pgfqpoint{1.902779in}{2.340182in}}%
\pgfpathlineto{\pgfqpoint{1.942018in}{2.323625in}}%
\pgfpathlineto{\pgfqpoint{1.984694in}{2.302189in}}%
\pgfpathlineto{\pgfqpoint{2.033430in}{2.274134in}}%
\pgfpathlineto{\pgfqpoint{2.096406in}{2.234057in}}%
\pgfpathlineto{\pgfqpoint{2.234901in}{2.144868in}}%
\pgfpathlineto{\pgfqpoint{2.281640in}{2.119182in}}%
\pgfpathlineto{\pgfqpoint{2.321213in}{2.100867in}}%
\pgfpathlineto{\pgfqpoint{2.356624in}{2.087812in}}%
\pgfpathlineto{\pgfqpoint{2.389267in}{2.079014in}}%
\pgfpathlineto{\pgfqpoint{2.419990in}{2.073892in}}%
\pgfpathlineto{\pgfqpoint{2.449375in}{2.072097in}}%
\pgfpathlineto{\pgfqpoint{2.477900in}{2.073435in}}%
\pgfpathlineto{\pgfqpoint{2.505945in}{2.077842in}}%
\pgfpathlineto{\pgfqpoint{2.533867in}{2.085368in}}%
\pgfpathlineto{\pgfqpoint{2.561979in}{2.096166in}}%
\pgfpathlineto{\pgfqpoint{2.590616in}{2.110511in}}%
\pgfpathlineto{\pgfqpoint{2.620078in}{2.128770in}}%
\pgfpathlineto{\pgfqpoint{2.650713in}{2.151449in}}%
\pgfpathlineto{\pgfqpoint{2.682909in}{2.179199in}}%
\pgfpathlineto{\pgfqpoint{2.717171in}{2.212891in}}%
\pgfpathlineto{\pgfqpoint{2.754222in}{2.253765in}}%
\pgfpathlineto{\pgfqpoint{2.795224in}{2.303731in}}%
\pgfpathlineto{\pgfqpoint{2.842531in}{2.366438in}}%
\pgfpathlineto{\pgfqpoint{2.903175in}{2.452312in}}%
\pgfpathlineto{\pgfqpoint{3.058176in}{2.674180in}}%
\pgfpathlineto{\pgfqpoint{3.103810in}{2.732656in}}%
\pgfpathlineto{\pgfqpoint{3.142434in}{2.777291in}}%
\pgfpathlineto{\pgfqpoint{3.176673in}{2.812272in}}%
\pgfpathlineto{\pgfqpoint{3.207754in}{2.839726in}}%
\pgfpathlineto{\pgfqpoint{3.236413in}{2.861021in}}%
\pgfpathlineto{\pgfqpoint{3.263164in}{2.877135in}}%
\pgfpathlineto{\pgfqpoint{3.288408in}{2.888804in}}%
\pgfpathlineto{\pgfqpoint{3.312480in}{2.896582in}}%
\pgfpathlineto{\pgfqpoint{3.335704in}{2.900876in}}%
\pgfpathlineto{\pgfqpoint{3.358370in}{2.901945in}}%
\pgfpathlineto{\pgfqpoint{3.380779in}{2.899912in}}%
\pgfpathlineto{\pgfqpoint{3.403211in}{2.894763in}}%
\pgfpathlineto{\pgfqpoint{3.425944in}{2.886347in}}%
\pgfpathlineto{\pgfqpoint{3.449235in}{2.874387in}}%
\pgfpathlineto{\pgfqpoint{3.473340in}{2.858479in}}%
\pgfpathlineto{\pgfqpoint{3.498506in}{2.838099in}}%
\pgfpathlineto{\pgfqpoint{3.524978in}{2.812602in}}%
\pgfpathlineto{\pgfqpoint{3.553045in}{2.781181in}}%
\pgfpathlineto{\pgfqpoint{3.583043in}{2.742851in}}%
\pgfpathlineto{\pgfqpoint{3.615430in}{2.696328in}}%
\pgfpathlineto{\pgfqpoint{3.650863in}{2.639871in}}%
\pgfpathlineto{\pgfqpoint{3.690447in}{2.570805in}}%
\pgfpathlineto{\pgfqpoint{3.736404in}{2.484156in}}%
\pgfpathlineto{\pgfqpoint{3.795117in}{2.366416in}}%
\pgfpathlineto{\pgfqpoint{3.962060in}{2.027705in}}%
\pgfpathlineto{\pgfqpoint{4.007347in}{1.945106in}}%
\pgfpathlineto{\pgfqpoint{4.045994in}{1.880834in}}%
\pgfpathlineto{\pgfqpoint{4.080345in}{1.829521in}}%
\pgfpathlineto{\pgfqpoint{4.111515in}{1.788351in}}%
\pgfpathlineto{\pgfqpoint{4.140185in}{1.755457in}}%
\pgfpathlineto{\pgfqpoint{4.166802in}{1.729495in}}%
\pgfpathlineto{\pgfqpoint{4.191711in}{1.709404in}}%
\pgfpathlineto{\pgfqpoint{4.215203in}{1.694334in}}%
\pgfpathlineto{\pgfqpoint{4.237545in}{1.683597in}}%
\pgfpathlineto{\pgfqpoint{4.259006in}{1.676655in}}%
\pgfpathlineto{\pgfqpoint{4.279864in}{1.673120in}}%
\pgfpathlineto{\pgfqpoint{4.300387in}{1.672759in}}%
\pgfpathlineto{\pgfqpoint{4.320877in}{1.675493in}}%
\pgfpathlineto{\pgfqpoint{4.341601in}{1.681399in}}%
\pgfpathlineto{\pgfqpoint{4.362839in}{1.690704in}}%
\pgfpathlineto{\pgfqpoint{4.384846in}{1.703776in}}%
\pgfpathlineto{\pgfqpoint{4.407869in}{1.721118in}}%
\pgfpathlineto{\pgfqpoint{4.432154in}{1.743364in}}%
\pgfpathlineto{\pgfqpoint{4.457956in}{1.771290in}}%
\pgfpathlineto{\pgfqpoint{4.485577in}{1.805852in}}%
\pgfpathlineto{\pgfqpoint{4.515385in}{1.848230in}}%
\pgfpathlineto{\pgfqpoint{4.547894in}{1.899966in}}%
\pgfpathlineto{\pgfqpoint{4.583896in}{1.963244in}}%
\pgfpathlineto{\pgfqpoint{4.624787in}{2.041582in}}%
\pgfpathlineto{\pgfqpoint{4.673656in}{2.142202in}}%
\pgfpathlineto{\pgfqpoint{4.742279in}{2.291288in}}%
\pgfpathlineto{\pgfqpoint{4.857094in}{2.540797in}}%
\pgfpathlineto{\pgfqpoint{4.907403in}{2.642189in}}%
\pgfpathlineto{\pgfqpoint{4.949153in}{2.719610in}}%
\pgfpathlineto{\pgfqpoint{4.985892in}{2.781431in}}%
\pgfpathlineto{\pgfqpoint{5.019115in}{2.831463in}}%
\pgfpathlineto{\pgfqpoint{5.049615in}{2.871965in}}%
\pgfpathlineto{\pgfqpoint{5.077917in}{2.904551in}}%
\pgfpathlineto{\pgfqpoint{5.104389in}{2.930440in}}%
\pgfpathlineto{\pgfqpoint{5.129331in}{2.950617in}}%
\pgfpathlineto{\pgfqpoint{5.153002in}{2.965878in}}%
\pgfpathlineto{\pgfqpoint{5.175645in}{2.976872in}}%
\pgfpathlineto{\pgfqpoint{5.197519in}{2.984111in}}%
\pgfpathlineto{\pgfqpoint{5.218891in}{2.987961in}}%
\pgfpathlineto{\pgfqpoint{5.240028in}{2.988642in}}%
\pgfpathlineto{\pgfqpoint{5.261209in}{2.986228in}}%
\pgfpathlineto{\pgfqpoint{5.282726in}{2.980643in}}%
\pgfpathlineto{\pgfqpoint{5.304856in}{2.971668in}}%
\pgfpathlineto{\pgfqpoint{5.327879in}{2.958941in}}%
\pgfpathlineto{\pgfqpoint{5.352063in}{2.941975in}}%
\pgfpathlineto{\pgfqpoint{5.377720in}{2.920119in}}%
\pgfpathlineto{\pgfqpoint{5.400007in}{2.898074in}}%
\pgfpathlineto{\pgfqpoint{5.400007in}{2.898074in}}%
\pgfusepath{stroke}%
\end{pgfscope}%
\begin{pgfscope}%
\pgfpathrectangle{\pgfqpoint{0.750000in}{0.500000in}}{\pgfqpoint{4.650000in}{3.020000in}}%
\pgfusepath{clip}%
\pgfsetrectcap%
\pgfsetroundjoin%
\pgfsetlinewidth{1.505625pt}%
\definecolor{currentstroke}{rgb}{1.000000,0.498039,0.054902}%
\pgfsetstrokecolor{currentstroke}%
\pgfsetdash{}{0pt}%
\pgfpathmoveto{\pgfqpoint{0.749993in}{0.660882in}}%
\pgfpathlineto{\pgfqpoint{0.755473in}{0.696307in}}%
\pgfpathlineto{\pgfqpoint{0.762415in}{0.763418in}}%
\pgfpathlineto{\pgfqpoint{0.770997in}{0.879051in}}%
\pgfpathlineto{\pgfqpoint{0.781554in}{1.066354in}}%
\pgfpathlineto{\pgfqpoint{0.794890in}{1.362282in}}%
\pgfpathlineto{\pgfqpoint{0.813974in}{1.862418in}}%
\pgfpathlineto{\pgfqpoint{0.850389in}{2.822969in}}%
\pgfpathlineto{\pgfqpoint{0.864082in}{3.099268in}}%
\pgfpathlineto{\pgfqpoint{0.874595in}{3.254571in}}%
\pgfpathlineto{\pgfqpoint{0.882842in}{3.335559in}}%
\pgfpathlineto{\pgfqpoint{0.889137in}{3.371056in}}%
\pgfpathlineto{\pgfqpoint{0.893589in}{3.381815in}}%
\pgfpathlineto{\pgfqpoint{0.896290in}{3.382433in}}%
\pgfpathlineto{\pgfqpoint{0.898712in}{3.379163in}}%
\pgfpathlineto{\pgfqpoint{0.902116in}{3.368436in}}%
\pgfpathlineto{\pgfqpoint{0.906769in}{3.342187in}}%
\pgfpathlineto{\pgfqpoint{0.912796in}{3.288524in}}%
\pgfpathlineto{\pgfqpoint{0.920329in}{3.191125in}}%
\pgfpathlineto{\pgfqpoint{0.929614in}{3.027606in}}%
\pgfpathlineto{\pgfqpoint{0.941187in}{2.764898in}}%
\pgfpathlineto{\pgfqpoint{0.956733in}{2.334941in}}%
\pgfpathlineto{\pgfqpoint{1.002121in}{1.035151in}}%
\pgfpathlineto{\pgfqpoint{1.013337in}{0.819083in}}%
\pgfpathlineto{\pgfqpoint{1.021963in}{0.706254in}}%
\pgfpathlineto{\pgfqpoint{1.028525in}{0.655590in}}%
\pgfpathlineto{\pgfqpoint{1.033201in}{0.639176in}}%
\pgfpathlineto{\pgfqpoint{1.036014in}{0.637412in}}%
\pgfpathlineto{\pgfqpoint{1.038101in}{0.640082in}}%
\pgfpathlineto{\pgfqpoint{1.041125in}{0.649995in}}%
\pgfpathlineto{\pgfqpoint{1.045377in}{0.676027in}}%
\pgfpathlineto{\pgfqpoint{1.050979in}{0.731698in}}%
\pgfpathlineto{\pgfqpoint{1.058088in}{0.836276in}}%
\pgfpathlineto{\pgfqpoint{1.066960in}{1.016426in}}%
\pgfpathlineto{\pgfqpoint{1.078221in}{1.313300in}}%
\pgfpathlineto{\pgfqpoint{1.094169in}{1.824672in}}%
\pgfpathlineto{\pgfqpoint{1.127258in}{2.899885in}}%
\pgfpathlineto{\pgfqpoint{1.138675in}{3.165620in}}%
\pgfpathlineto{\pgfqpoint{1.147268in}{3.301375in}}%
\pgfpathlineto{\pgfqpoint{1.153763in}{3.361614in}}%
\pgfpathlineto{\pgfqpoint{1.158339in}{3.380646in}}%
\pgfpathlineto{\pgfqpoint{1.160950in}{3.382594in}}%
\pgfpathlineto{\pgfqpoint{1.161017in}{3.382558in}}%
\pgfpathlineto{\pgfqpoint{1.162937in}{3.379694in}}%
\pgfpathlineto{\pgfqpoint{1.165838in}{3.368635in}}%
\pgfpathlineto{\pgfqpoint{1.169979in}{3.338866in}}%
\pgfpathlineto{\pgfqpoint{1.175481in}{3.274238in}}%
\pgfpathlineto{\pgfqpoint{1.182522in}{3.151457in}}%
\pgfpathlineto{\pgfqpoint{1.191450in}{2.936857in}}%
\pgfpathlineto{\pgfqpoint{1.203157in}{2.574060in}}%
\pgfpathlineto{\pgfqpoint{1.222966in}{1.845738in}}%
\pgfpathlineto{\pgfqpoint{1.242921in}{1.153032in}}%
\pgfpathlineto{\pgfqpoint{1.254114in}{0.863873in}}%
\pgfpathlineto{\pgfqpoint{1.262462in}{0.719326in}}%
\pgfpathlineto{\pgfqpoint{1.268711in}{0.657214in}}%
\pgfpathlineto{\pgfqpoint{1.273053in}{0.638819in}}%
\pgfpathlineto{\pgfqpoint{1.275206in}{0.637402in}}%
\pgfpathlineto{\pgfqpoint{1.275474in}{0.637586in}}%
\pgfpathlineto{\pgfqpoint{1.277383in}{0.641201in}}%
\pgfpathlineto{\pgfqpoint{1.280329in}{0.654721in}}%
\pgfpathlineto{\pgfqpoint{1.284536in}{0.690615in}}%
\pgfpathlineto{\pgfqpoint{1.290150in}{0.768185in}}%
\pgfpathlineto{\pgfqpoint{1.297370in}{0.915085in}}%
\pgfpathlineto{\pgfqpoint{1.306655in}{1.172770in}}%
\pgfpathlineto{\pgfqpoint{1.319422in}{1.622002in}}%
\pgfpathlineto{\pgfqpoint{1.357612in}{3.021817in}}%
\pgfpathlineto{\pgfqpoint{1.367087in}{3.238030in}}%
\pgfpathlineto{\pgfqpoint{1.374218in}{3.339025in}}%
\pgfpathlineto{\pgfqpoint{1.379408in}{3.375902in}}%
\pgfpathlineto{\pgfqpoint{1.382677in}{3.382721in}}%
\pgfpathlineto{\pgfqpoint{1.384285in}{3.381373in}}%
\pgfpathlineto{\pgfqpoint{1.386595in}{3.374009in}}%
\pgfpathlineto{\pgfqpoint{1.390077in}{3.350874in}}%
\pgfpathlineto{\pgfqpoint{1.394898in}{3.295365in}}%
\pgfpathlineto{\pgfqpoint{1.401225in}{3.182891in}}%
\pgfpathlineto{\pgfqpoint{1.409383in}{2.977561in}}%
\pgfpathlineto{\pgfqpoint{1.420231in}{2.619025in}}%
\pgfpathlineto{\pgfqpoint{1.438634in}{1.888904in}}%
\pgfpathlineto{\pgfqpoint{1.458052in}{1.160528in}}%
\pgfpathlineto{\pgfqpoint{1.468855in}{0.862835in}}%
\pgfpathlineto{\pgfqpoint{1.476913in}{0.716403in}}%
\pgfpathlineto{\pgfqpoint{1.482928in}{0.655226in}}%
\pgfpathlineto{\pgfqpoint{1.487046in}{0.638328in}}%
\pgfpathlineto{\pgfqpoint{1.488876in}{0.637434in}}%
\pgfpathlineto{\pgfqpoint{1.489233in}{0.637735in}}%
\pgfpathlineto{\pgfqpoint{1.491164in}{0.642050in}}%
\pgfpathlineto{\pgfqpoint{1.494177in}{0.657808in}}%
\pgfpathlineto{\pgfqpoint{1.498485in}{0.699162in}}%
\pgfpathlineto{\pgfqpoint{1.504232in}{0.787631in}}%
\pgfpathlineto{\pgfqpoint{1.511687in}{0.954842in}}%
\pgfpathlineto{\pgfqpoint{1.521452in}{1.249946in}}%
\pgfpathlineto{\pgfqpoint{1.535771in}{1.787957in}}%
\pgfpathlineto{\pgfqpoint{1.564564in}{2.880494in}}%
\pgfpathlineto{\pgfqpoint{1.575221in}{3.164879in}}%
\pgfpathlineto{\pgfqpoint{1.583257in}{3.305996in}}%
\pgfpathlineto{\pgfqpoint{1.589272in}{3.365135in}}%
\pgfpathlineto{\pgfqpoint{1.593423in}{3.381671in}}%
\pgfpathlineto{\pgfqpoint{1.595321in}{3.382551in}}%
\pgfpathlineto{\pgfqpoint{1.595644in}{3.382283in}}%
\pgfpathlineto{\pgfqpoint{1.597597in}{3.378084in}}%
\pgfpathlineto{\pgfqpoint{1.600633in}{3.362814in}}%
\pgfpathlineto{\pgfqpoint{1.604963in}{3.322919in}}%
\pgfpathlineto{\pgfqpoint{1.610744in}{3.237697in}}%
\pgfpathlineto{\pgfqpoint{1.618243in}{3.076848in}}%
\pgfpathlineto{\pgfqpoint{1.628042in}{2.793985in}}%
\pgfpathlineto{\pgfqpoint{1.642237in}{2.283993in}}%
\pgfpathlineto{\pgfqpoint{1.673619in}{1.136981in}}%
\pgfpathlineto{\pgfqpoint{1.684467in}{0.859523in}}%
\pgfpathlineto{\pgfqpoint{1.692725in}{0.718784in}}%
\pgfpathlineto{\pgfqpoint{1.698986in}{0.657564in}}%
\pgfpathlineto{\pgfqpoint{1.703394in}{0.638979in}}%
\pgfpathlineto{\pgfqpoint{1.705693in}{0.637403in}}%
\pgfpathlineto{\pgfqpoint{1.705894in}{0.637530in}}%
\pgfpathlineto{\pgfqpoint{1.707814in}{0.640872in}}%
\pgfpathlineto{\pgfqpoint{1.710760in}{0.653476in}}%
\pgfpathlineto{\pgfqpoint{1.714978in}{0.687026in}}%
\pgfpathlineto{\pgfqpoint{1.720636in}{0.759705in}}%
\pgfpathlineto{\pgfqpoint{1.727957in}{0.897559in}}%
\pgfpathlineto{\pgfqpoint{1.737432in}{1.139708in}}%
\pgfpathlineto{\pgfqpoint{1.750501in}{1.561145in}}%
\pgfpathlineto{\pgfqpoint{1.792875in}{2.988700in}}%
\pgfpathlineto{\pgfqpoint{1.803142in}{3.208938in}}%
\pgfpathlineto{\pgfqpoint{1.811100in}{3.320656in}}%
\pgfpathlineto{\pgfqpoint{1.817148in}{3.368243in}}%
\pgfpathlineto{\pgfqpoint{1.821356in}{3.381791in}}%
\pgfpathlineto{\pgfqpoint{1.823643in}{3.382397in}}%
\pgfpathlineto{\pgfqpoint{1.823710in}{3.382343in}}%
\pgfpathlineto{\pgfqpoint{1.825820in}{3.378572in}}%
\pgfpathlineto{\pgfqpoint{1.828989in}{3.365389in}}%
\pgfpathlineto{\pgfqpoint{1.833453in}{3.331785in}}%
\pgfpathlineto{\pgfqpoint{1.839368in}{3.261143in}}%
\pgfpathlineto{\pgfqpoint{1.846968in}{3.129690in}}%
\pgfpathlineto{\pgfqpoint{1.856711in}{2.902768in}}%
\pgfpathlineto{\pgfqpoint{1.869902in}{2.516268in}}%
\pgfpathlineto{\pgfqpoint{1.919408in}{0.990221in}}%
\pgfpathlineto{\pgfqpoint{1.929708in}{0.795739in}}%
\pgfpathlineto{\pgfqpoint{1.937788in}{0.695232in}}%
\pgfpathlineto{\pgfqpoint{1.943982in}{0.651443in}}%
\pgfpathlineto{\pgfqpoint{1.948346in}{0.638370in}}%
\pgfpathlineto{\pgfqpoint{1.950901in}{0.637567in}}%
\pgfpathlineto{\pgfqpoint{1.953100in}{0.640913in}}%
\pgfpathlineto{\pgfqpoint{1.956325in}{0.652529in}}%
\pgfpathlineto{\pgfqpoint{1.960834in}{0.681930in}}%
\pgfpathlineto{\pgfqpoint{1.966771in}{0.743342in}}%
\pgfpathlineto{\pgfqpoint{1.974337in}{0.856809in}}%
\pgfpathlineto{\pgfqpoint{1.983913in}{1.050932in}}%
\pgfpathlineto{\pgfqpoint{1.996412in}{1.372700in}}%
\pgfpathlineto{\pgfqpoint{2.016109in}{1.971692in}}%
\pgfpathlineto{\pgfqpoint{2.042860in}{2.767219in}}%
\pgfpathlineto{\pgfqpoint{2.056151in}{3.073672in}}%
\pgfpathlineto{\pgfqpoint{2.066363in}{3.244342in}}%
\pgfpathlineto{\pgfqpoint{2.074398in}{3.332562in}}%
\pgfpathlineto{\pgfqpoint{2.080547in}{3.370726in}}%
\pgfpathlineto{\pgfqpoint{2.084866in}{3.381891in}}%
\pgfpathlineto{\pgfqpoint{2.087411in}{3.382381in}}%
\pgfpathlineto{\pgfqpoint{2.089732in}{3.378891in}}%
\pgfpathlineto{\pgfqpoint{2.093080in}{3.367273in}}%
\pgfpathlineto{\pgfqpoint{2.097711in}{3.338521in}}%
\pgfpathlineto{\pgfqpoint{2.103760in}{3.279351in}}%
\pgfpathlineto{\pgfqpoint{2.111405in}{3.171251in}}%
\pgfpathlineto{\pgfqpoint{2.120969in}{2.988397in}}%
\pgfpathlineto{\pgfqpoint{2.133211in}{2.690109in}}%
\pgfpathlineto{\pgfqpoint{2.151045in}{2.170945in}}%
\pgfpathlineto{\pgfqpoint{2.184246in}{1.200316in}}%
\pgfpathlineto{\pgfqpoint{2.197203in}{0.914326in}}%
\pgfpathlineto{\pgfqpoint{2.207124in}{0.757232in}}%
\pgfpathlineto{\pgfqpoint{2.214858in}{0.678264in}}%
\pgfpathlineto{\pgfqpoint{2.220672in}{0.645909in}}%
\pgfpathlineto{\pgfqpoint{2.224645in}{0.637586in}}%
\pgfpathlineto{\pgfqpoint{2.226911in}{0.637913in}}%
\pgfpathlineto{\pgfqpoint{2.229355in}{0.642412in}}%
\pgfpathlineto{\pgfqpoint{2.232881in}{0.656484in}}%
\pgfpathlineto{\pgfqpoint{2.237703in}{0.690108in}}%
\pgfpathlineto{\pgfqpoint{2.243941in}{0.757774in}}%
\pgfpathlineto{\pgfqpoint{2.251775in}{0.879594in}}%
\pgfpathlineto{\pgfqpoint{2.261540in}{1.083653in}}%
\pgfpathlineto{\pgfqpoint{2.274118in}{1.416755in}}%
\pgfpathlineto{\pgfqpoint{2.293558in}{2.025796in}}%
\pgfpathlineto{\pgfqpoint{2.319662in}{2.825453in}}%
\pgfpathlineto{\pgfqpoint{2.332217in}{3.119013in}}%
\pgfpathlineto{\pgfqpoint{2.341669in}{3.274850in}}%
\pgfpathlineto{\pgfqpoint{2.348912in}{3.349388in}}%
\pgfpathlineto{\pgfqpoint{2.354213in}{3.377204in}}%
\pgfpathlineto{\pgfqpoint{2.357639in}{3.382712in}}%
\pgfpathlineto{\pgfqpoint{2.359559in}{3.381456in}}%
\pgfpathlineto{\pgfqpoint{2.362103in}{3.374957in}}%
\pgfpathlineto{\pgfqpoint{2.365808in}{3.355632in}}%
\pgfpathlineto{\pgfqpoint{2.370819in}{3.311018in}}%
\pgfpathlineto{\pgfqpoint{2.377281in}{3.222845in}}%
\pgfpathlineto{\pgfqpoint{2.385394in}{3.065962in}}%
\pgfpathlineto{\pgfqpoint{2.395650in}{2.802635in}}%
\pgfpathlineto{\pgfqpoint{2.409545in}{2.358472in}}%
\pgfpathlineto{\pgfqpoint{2.450457in}{0.999680in}}%
\pgfpathlineto{\pgfqpoint{2.460445in}{0.787780in}}%
\pgfpathlineto{\pgfqpoint{2.467978in}{0.685753in}}%
\pgfpathlineto{\pgfqpoint{2.473514in}{0.646207in}}%
\pgfpathlineto{\pgfqpoint{2.477130in}{0.637410in}}%
\pgfpathlineto{\pgfqpoint{2.478949in}{0.638176in}}%
\pgfpathlineto{\pgfqpoint{2.481181in}{0.643898in}}%
\pgfpathlineto{\pgfqpoint{2.484540in}{0.662462in}}%
\pgfpathlineto{\pgfqpoint{2.489194in}{0.707827in}}%
\pgfpathlineto{\pgfqpoint{2.495276in}{0.800730in}}%
\pgfpathlineto{\pgfqpoint{2.503021in}{0.970917in}}%
\pgfpathlineto{\pgfqpoint{2.512965in}{1.263688in}}%
\pgfpathlineto{\pgfqpoint{2.527015in}{1.779016in}}%
\pgfpathlineto{\pgfqpoint{2.558386in}{2.952536in}}%
\pgfpathlineto{\pgfqpoint{2.568452in}{3.207383in}}%
\pgfpathlineto{\pgfqpoint{2.575918in}{3.327584in}}%
\pgfpathlineto{\pgfqpoint{2.581342in}{3.373049in}}%
\pgfpathlineto{\pgfqpoint{2.584835in}{3.382633in}}%
\pgfpathlineto{\pgfqpoint{2.586464in}{3.381719in}}%
\pgfpathlineto{\pgfqpoint{2.588596in}{3.375319in}}%
\pgfpathlineto{\pgfqpoint{2.591866in}{3.354028in}}%
\pgfpathlineto{\pgfqpoint{2.596442in}{3.301127in}}%
\pgfpathlineto{\pgfqpoint{2.602468in}{3.191581in}}%
\pgfpathlineto{\pgfqpoint{2.610235in}{2.988561in}}%
\pgfpathlineto{\pgfqpoint{2.620480in}{2.631999in}}%
\pgfpathlineto{\pgfqpoint{2.637064in}{1.928744in}}%
\pgfpathlineto{\pgfqpoint{2.656851in}{1.124924in}}%
\pgfpathlineto{\pgfqpoint{2.666939in}{0.831913in}}%
\pgfpathlineto{\pgfqpoint{2.674294in}{0.696979in}}%
\pgfpathlineto{\pgfqpoint{2.679595in}{0.647145in}}%
\pgfpathlineto{\pgfqpoint{2.682954in}{0.637318in}}%
\pgfpathlineto{\pgfqpoint{2.684416in}{0.638410in}}%
\pgfpathlineto{\pgfqpoint{2.686481in}{0.645524in}}%
\pgfpathlineto{\pgfqpoint{2.689706in}{0.669673in}}%
\pgfpathlineto{\pgfqpoint{2.694248in}{0.730283in}}%
\pgfpathlineto{\pgfqpoint{2.700274in}{0.856817in}}%
\pgfpathlineto{\pgfqpoint{2.708131in}{1.093313in}}%
\pgfpathlineto{\pgfqpoint{2.718845in}{1.518847in}}%
\pgfpathlineto{\pgfqpoint{2.757380in}{3.141864in}}%
\pgfpathlineto{\pgfqpoint{2.764969in}{3.306517in}}%
\pgfpathlineto{\pgfqpoint{2.770449in}{3.368998in}}%
\pgfpathlineto{\pgfqpoint{2.773986in}{3.382541in}}%
\pgfpathlineto{\pgfqpoint{2.775192in}{3.382257in}}%
\pgfpathlineto{\pgfqpoint{2.775504in}{3.381776in}}%
\pgfpathlineto{\pgfqpoint{2.777435in}{3.375075in}}%
\pgfpathlineto{\pgfqpoint{2.780504in}{3.351264in}}%
\pgfpathlineto{\pgfqpoint{2.784890in}{3.289630in}}%
\pgfpathlineto{\pgfqpoint{2.790771in}{3.158245in}}%
\pgfpathlineto{\pgfqpoint{2.798516in}{2.908847in}}%
\pgfpathlineto{\pgfqpoint{2.809285in}{2.451220in}}%
\pgfpathlineto{\pgfqpoint{2.843402in}{0.927079in}}%
\pgfpathlineto{\pgfqpoint{2.851281in}{0.732252in}}%
\pgfpathlineto{\pgfqpoint{2.856983in}{0.656042in}}%
\pgfpathlineto{\pgfqpoint{2.860733in}{0.637818in}}%
\pgfpathlineto{\pgfqpoint{2.861883in}{0.637406in}}%
\pgfpathlineto{\pgfqpoint{2.862429in}{0.638066in}}%
\pgfpathlineto{\pgfqpoint{2.864260in}{0.644285in}}%
\pgfpathlineto{\pgfqpoint{2.867217in}{0.667333in}}%
\pgfpathlineto{\pgfqpoint{2.871480in}{0.728320in}}%
\pgfpathlineto{\pgfqpoint{2.877239in}{0.860294in}}%
\pgfpathlineto{\pgfqpoint{2.884883in}{1.113967in}}%
\pgfpathlineto{\pgfqpoint{2.895631in}{1.585146in}}%
\pgfpathlineto{\pgfqpoint{2.927961in}{3.069544in}}%
\pgfpathlineto{\pgfqpoint{2.935952in}{3.278144in}}%
\pgfpathlineto{\pgfqpoint{2.941755in}{3.361103in}}%
\pgfpathlineto{\pgfqpoint{2.945616in}{3.381907in}}%
\pgfpathlineto{\pgfqpoint{2.946855in}{3.382639in}}%
\pgfpathlineto{\pgfqpoint{2.947435in}{3.381986in}}%
\pgfpathlineto{\pgfqpoint{2.949221in}{3.375998in}}%
\pgfpathlineto{\pgfqpoint{2.952134in}{3.353413in}}%
\pgfpathlineto{\pgfqpoint{2.956352in}{3.293082in}}%
\pgfpathlineto{\pgfqpoint{2.962077in}{3.161572in}}%
\pgfpathlineto{\pgfqpoint{2.969700in}{2.907821in}}%
\pgfpathlineto{\pgfqpoint{2.980480in}{2.433928in}}%
\pgfpathlineto{\pgfqpoint{3.012420in}{0.966005in}}%
\pgfpathlineto{\pgfqpoint{3.020567in}{0.749675in}}%
\pgfpathlineto{\pgfqpoint{3.026515in}{0.661839in}}%
\pgfpathlineto{\pgfqpoint{3.030533in}{0.638508in}}%
\pgfpathlineto{\pgfqpoint{3.031950in}{0.637333in}}%
\pgfpathlineto{\pgfqpoint{3.032519in}{0.637899in}}%
\pgfpathlineto{\pgfqpoint{3.034271in}{0.643363in}}%
\pgfpathlineto{\pgfqpoint{3.037140in}{0.664355in}}%
\pgfpathlineto{\pgfqpoint{3.041313in}{0.721060in}}%
\pgfpathlineto{\pgfqpoint{3.046994in}{0.845501in}}%
\pgfpathlineto{\pgfqpoint{3.054560in}{1.086427in}}%
\pgfpathlineto{\pgfqpoint{3.065174in}{1.534191in}}%
\pgfpathlineto{\pgfqpoint{3.099848in}{3.074429in}}%
\pgfpathlineto{\pgfqpoint{3.107950in}{3.276120in}}%
\pgfpathlineto{\pgfqpoint{3.113921in}{3.358972in}}%
\pgfpathlineto{\pgfqpoint{3.117983in}{3.381422in}}%
\pgfpathlineto{\pgfqpoint{3.119512in}{3.382659in}}%
\pgfpathlineto{\pgfqpoint{3.120036in}{3.382174in}}%
\pgfpathlineto{\pgfqpoint{3.121800in}{3.377155in}}%
\pgfpathlineto{\pgfqpoint{3.124679in}{3.357815in}}%
\pgfpathlineto{\pgfqpoint{3.128864in}{3.305613in}}%
\pgfpathlineto{\pgfqpoint{3.134544in}{3.191443in}}%
\pgfpathlineto{\pgfqpoint{3.142088in}{2.971034in}}%
\pgfpathlineto{\pgfqpoint{3.152523in}{2.565831in}}%
\pgfpathlineto{\pgfqpoint{3.192744in}{0.903598in}}%
\pgfpathlineto{\pgfqpoint{3.200746in}{0.730058in}}%
\pgfpathlineto{\pgfqpoint{3.206705in}{0.658171in}}%
\pgfpathlineto{\pgfqpoint{3.210790in}{0.638465in}}%
\pgfpathlineto{\pgfqpoint{3.212441in}{0.637371in}}%
\pgfpathlineto{\pgfqpoint{3.212910in}{0.637779in}}%
\pgfpathlineto{\pgfqpoint{3.214751in}{0.642445in}}%
\pgfpathlineto{\pgfqpoint{3.217698in}{0.659963in}}%
\pgfpathlineto{\pgfqpoint{3.221961in}{0.706685in}}%
\pgfpathlineto{\pgfqpoint{3.227719in}{0.807858in}}%
\pgfpathlineto{\pgfqpoint{3.235319in}{1.001463in}}%
\pgfpathlineto{\pgfqpoint{3.245620in}{1.350913in}}%
\pgfpathlineto{\pgfqpoint{3.263554in}{2.085375in}}%
\pgfpathlineto{\pgfqpoint{3.282615in}{2.821527in}}%
\pgfpathlineto{\pgfqpoint{3.293530in}{3.132151in}}%
\pgfpathlineto{\pgfqpoint{3.301822in}{3.289626in}}%
\pgfpathlineto{\pgfqpoint{3.308138in}{3.358840in}}%
\pgfpathlineto{\pgfqpoint{3.312625in}{3.380466in}}%
\pgfpathlineto{\pgfqpoint{3.315013in}{3.382642in}}%
\pgfpathlineto{\pgfqpoint{3.315214in}{3.382531in}}%
\pgfpathlineto{\pgfqpoint{3.317033in}{3.379462in}}%
\pgfpathlineto{\pgfqpoint{3.319868in}{3.367346in}}%
\pgfpathlineto{\pgfqpoint{3.323986in}{3.334189in}}%
\pgfpathlineto{\pgfqpoint{3.329566in}{3.261177in}}%
\pgfpathlineto{\pgfqpoint{3.336876in}{3.120744in}}%
\pgfpathlineto{\pgfqpoint{3.346473in}{2.871113in}}%
\pgfpathlineto{\pgfqpoint{3.360144in}{2.425858in}}%
\pgfpathlineto{\pgfqpoint{3.398244in}{1.144107in}}%
\pgfpathlineto{\pgfqpoint{3.409594in}{0.881074in}}%
\pgfpathlineto{\pgfqpoint{3.418556in}{0.738064in}}%
\pgfpathlineto{\pgfqpoint{3.425609in}{0.668930in}}%
\pgfpathlineto{\pgfqpoint{3.430899in}{0.642677in}}%
\pgfpathlineto{\pgfqpoint{3.434392in}{0.637297in}}%
\pgfpathlineto{\pgfqpoint{3.436423in}{0.638482in}}%
\pgfpathlineto{\pgfqpoint{3.439057in}{0.644668in}}%
\pgfpathlineto{\pgfqpoint{3.442885in}{0.662824in}}%
\pgfpathlineto{\pgfqpoint{3.448107in}{0.704484in}}%
\pgfpathlineto{\pgfqpoint{3.454915in}{0.786217in}}%
\pgfpathlineto{\pgfqpoint{3.463642in}{0.931641in}}%
\pgfpathlineto{\pgfqpoint{3.475003in}{1.177172in}}%
\pgfpathlineto{\pgfqpoint{3.491263in}{1.602999in}}%
\pgfpathlineto{\pgfqpoint{3.534073in}{2.752285in}}%
\pgfpathlineto{\pgfqpoint{3.548057in}{3.032747in}}%
\pgfpathlineto{\pgfqpoint{3.559418in}{3.203411in}}%
\pgfpathlineto{\pgfqpoint{3.568825in}{3.302380in}}%
\pgfpathlineto{\pgfqpoint{3.576492in}{3.353928in}}%
\pgfpathlineto{\pgfqpoint{3.582485in}{3.376101in}}%
\pgfpathlineto{\pgfqpoint{3.586782in}{3.382382in}}%
\pgfpathlineto{\pgfqpoint{3.589661in}{3.382187in}}%
\pgfpathlineto{\pgfqpoint{3.592652in}{3.378316in}}%
\pgfpathlineto{\pgfqpoint{3.596703in}{3.367247in}}%
\pgfpathlineto{\pgfqpoint{3.602105in}{3.342408in}}%
\pgfpathlineto{\pgfqpoint{3.609035in}{3.294528in}}%
\pgfpathlineto{\pgfqpoint{3.617729in}{3.211012in}}%
\pgfpathlineto{\pgfqpoint{3.628565in}{3.074712in}}%
\pgfpathlineto{\pgfqpoint{3.642325in}{2.859613in}}%
\pgfpathlineto{\pgfqpoint{3.661197in}{2.511588in}}%
\pgfpathlineto{\pgfqpoint{3.726706in}{1.262334in}}%
\pgfpathlineto{\pgfqpoint{3.743457in}{1.020447in}}%
\pgfpathlineto{\pgfqpoint{3.757474in}{0.859818in}}%
\pgfpathlineto{\pgfqpoint{3.769393in}{0.756200in}}%
\pgfpathlineto{\pgfqpoint{3.779460in}{0.693454in}}%
\pgfpathlineto{\pgfqpoint{3.787796in}{0.659043in}}%
\pgfpathlineto{\pgfqpoint{3.794459in}{0.643059in}}%
\pgfpathlineto{\pgfqpoint{3.799525in}{0.637748in}}%
\pgfpathlineto{\pgfqpoint{3.803387in}{0.637653in}}%
\pgfpathlineto{\pgfqpoint{3.807226in}{0.640928in}}%
\pgfpathlineto{\pgfqpoint{3.811958in}{0.649550in}}%
\pgfpathlineto{\pgfqpoint{3.817917in}{0.667528in}}%
\pgfpathlineto{\pgfqpoint{3.825260in}{0.700390in}}%
\pgfpathlineto{\pgfqpoint{3.834121in}{0.755337in}}%
\pgfpathlineto{\pgfqpoint{3.844645in}{0.841378in}}%
\pgfpathlineto{\pgfqpoint{3.857055in}{0.969933in}}%
\pgfpathlineto{\pgfqpoint{3.871742in}{1.156190in}}%
\pgfpathlineto{\pgfqpoint{3.889564in}{1.424034in}}%
\pgfpathlineto{\pgfqpoint{3.913157in}{1.829183in}}%
\pgfpathlineto{\pgfqpoint{3.973644in}{2.889575in}}%
\pgfpathlineto{\pgfqpoint{3.989904in}{3.110404in}}%
\pgfpathlineto{\pgfqpoint{4.002604in}{3.244137in}}%
\pgfpathlineto{\pgfqpoint{4.012749in}{3.321255in}}%
\pgfpathlineto{\pgfqpoint{4.020773in}{3.361154in}}%
\pgfpathlineto{\pgfqpoint{4.026889in}{3.378043in}}%
\pgfpathlineto{\pgfqpoint{4.031230in}{3.382581in}}%
\pgfpathlineto{\pgfqpoint{4.034310in}{3.381934in}}%
\pgfpathlineto{\pgfqpoint{4.037636in}{3.377560in}}%
\pgfpathlineto{\pgfqpoint{4.041966in}{3.366046in}}%
\pgfpathlineto{\pgfqpoint{4.047512in}{3.341519in}}%
\pgfpathlineto{\pgfqpoint{4.054353in}{3.295929in}}%
\pgfpathlineto{\pgfqpoint{4.062578in}{3.218563in}}%
\pgfpathlineto{\pgfqpoint{4.072299in}{3.095752in}}%
\pgfpathlineto{\pgfqpoint{4.083782in}{2.908779in}}%
\pgfpathlineto{\pgfqpoint{4.097598in}{2.629918in}}%
\pgfpathlineto{\pgfqpoint{4.115633in}{2.197882in}}%
\pgfpathlineto{\pgfqpoint{4.163465in}{1.019180in}}%
\pgfpathlineto{\pgfqpoint{4.175395in}{0.814272in}}%
\pgfpathlineto{\pgfqpoint{4.184457in}{0.706296in}}%
\pgfpathlineto{\pgfqpoint{4.191331in}{0.656601in}}%
\pgfpathlineto{\pgfqpoint{4.196253in}{0.639657in}}%
\pgfpathlineto{\pgfqpoint{4.199311in}{0.637333in}}%
\pgfpathlineto{\pgfqpoint{4.201465in}{0.639565in}}%
\pgfpathlineto{\pgfqpoint{4.204467in}{0.648092in}}%
\pgfpathlineto{\pgfqpoint{4.208652in}{0.670633in}}%
\pgfpathlineto{\pgfqpoint{4.214131in}{0.719060in}}%
\pgfpathlineto{\pgfqpoint{4.221006in}{0.810012in}}%
\pgfpathlineto{\pgfqpoint{4.229454in}{0.966330in}}%
\pgfpathlineto{\pgfqpoint{4.239889in}{1.221401in}}%
\pgfpathlineto{\pgfqpoint{4.253504in}{1.636804in}}%
\pgfpathlineto{\pgfqpoint{4.299706in}{3.115493in}}%
\pgfpathlineto{\pgfqpoint{4.308713in}{3.281798in}}%
\pgfpathlineto{\pgfqpoint{4.315397in}{3.355745in}}%
\pgfpathlineto{\pgfqpoint{4.320107in}{3.379800in}}%
\pgfpathlineto{\pgfqpoint{4.322752in}{3.382658in}}%
\pgfpathlineto{\pgfqpoint{4.322863in}{3.382607in}}%
\pgfpathlineto{\pgfqpoint{4.324627in}{3.379947in}}%
\pgfpathlineto{\pgfqpoint{4.327339in}{3.369002in}}%
\pgfpathlineto{\pgfqpoint{4.331267in}{3.338341in}}%
\pgfpathlineto{\pgfqpoint{4.336523in}{3.270032in}}%
\pgfpathlineto{\pgfqpoint{4.343264in}{3.138043in}}%
\pgfpathlineto{\pgfqpoint{4.351824in}{2.904170in}}%
\pgfpathlineto{\pgfqpoint{4.363129in}{2.502456in}}%
\pgfpathlineto{\pgfqpoint{4.385761in}{1.555215in}}%
\pgfpathlineto{\pgfqpoint{4.400292in}{1.022226in}}%
\pgfpathlineto{\pgfqpoint{4.409599in}{0.782488in}}%
\pgfpathlineto{\pgfqpoint{4.416396in}{0.676594in}}%
\pgfpathlineto{\pgfqpoint{4.421161in}{0.641822in}}%
\pgfpathlineto{\pgfqpoint{4.423772in}{0.637296in}}%
\pgfpathlineto{\pgfqpoint{4.423951in}{0.637369in}}%
\pgfpathlineto{\pgfqpoint{4.425458in}{0.639941in}}%
\pgfpathlineto{\pgfqpoint{4.427924in}{0.651736in}}%
\pgfpathlineto{\pgfqpoint{4.431607in}{0.686876in}}%
\pgfpathlineto{\pgfqpoint{4.436640in}{0.768373in}}%
\pgfpathlineto{\pgfqpoint{4.443224in}{0.930600in}}%
\pgfpathlineto{\pgfqpoint{4.451829in}{1.226543in}}%
\pgfpathlineto{\pgfqpoint{4.464138in}{1.769334in}}%
\pgfpathlineto{\pgfqpoint{4.491402in}{2.996312in}}%
\pgfpathlineto{\pgfqpoint{4.500129in}{3.246242in}}%
\pgfpathlineto{\pgfqpoint{4.506412in}{3.350275in}}%
\pgfpathlineto{\pgfqpoint{4.510698in}{3.380349in}}%
\pgfpathlineto{\pgfqpoint{4.512450in}{3.382700in}}%
\pgfpathlineto{\pgfqpoint{4.512963in}{3.382282in}}%
\pgfpathlineto{\pgfqpoint{4.514593in}{3.377625in}}%
\pgfpathlineto{\pgfqpoint{4.517305in}{3.358637in}}%
\pgfpathlineto{\pgfqpoint{4.521278in}{3.305678in}}%
\pgfpathlineto{\pgfqpoint{4.526679in}{3.187286in}}%
\pgfpathlineto{\pgfqpoint{4.533821in}{2.955593in}}%
\pgfpathlineto{\pgfqpoint{4.543620in}{2.526232in}}%
\pgfpathlineto{\pgfqpoint{4.579700in}{0.840209in}}%
\pgfpathlineto{\pgfqpoint{4.586486in}{0.691457in}}%
\pgfpathlineto{\pgfqpoint{4.591206in}{0.643351in}}%
\pgfpathlineto{\pgfqpoint{4.593661in}{0.637282in}}%
\pgfpathlineto{\pgfqpoint{4.593929in}{0.637416in}}%
\pgfpathlineto{\pgfqpoint{4.595280in}{0.640480in}}%
\pgfpathlineto{\pgfqpoint{4.597623in}{0.655264in}}%
\pgfpathlineto{\pgfqpoint{4.601194in}{0.700688in}}%
\pgfpathlineto{\pgfqpoint{4.606161in}{0.808388in}}%
\pgfpathlineto{\pgfqpoint{4.612823in}{1.027626in}}%
\pgfpathlineto{\pgfqpoint{4.621997in}{1.442830in}}%
\pgfpathlineto{\pgfqpoint{4.658345in}{3.210271in}}%
\pgfpathlineto{\pgfqpoint{4.664661in}{3.341663in}}%
\pgfpathlineto{\pgfqpoint{4.668958in}{3.379647in}}%
\pgfpathlineto{\pgfqpoint{4.670688in}{3.382711in}}%
\pgfpathlineto{\pgfqpoint{4.671224in}{3.382224in}}%
\pgfpathlineto{\pgfqpoint{4.672775in}{3.376981in}}%
\pgfpathlineto{\pgfqpoint{4.675420in}{3.354949in}}%
\pgfpathlineto{\pgfqpoint{4.679348in}{3.292269in}}%
\pgfpathlineto{\pgfqpoint{4.684750in}{3.150454in}}%
\pgfpathlineto{\pgfqpoint{4.692048in}{2.868509in}}%
\pgfpathlineto{\pgfqpoint{4.702717in}{2.321495in}}%
\pgfpathlineto{\pgfqpoint{4.727827in}{1.000352in}}%
\pgfpathlineto{\pgfqpoint{4.735695in}{0.752643in}}%
\pgfpathlineto{\pgfqpoint{4.741275in}{0.658885in}}%
\pgfpathlineto{\pgfqpoint{4.744891in}{0.637709in}}%
\pgfpathlineto{\pgfqpoint{4.745773in}{0.637371in}}%
\pgfpathlineto{\pgfqpoint{4.746386in}{0.638257in}}%
\pgfpathlineto{\pgfqpoint{4.748138in}{0.645839in}}%
\pgfpathlineto{\pgfqpoint{4.751029in}{0.674590in}}%
\pgfpathlineto{\pgfqpoint{4.755247in}{0.751959in}}%
\pgfpathlineto{\pgfqpoint{4.761017in}{0.921435in}}%
\pgfpathlineto{\pgfqpoint{4.768896in}{1.253883in}}%
\pgfpathlineto{\pgfqpoint{4.781384in}{1.932050in}}%
\pgfpathlineto{\pgfqpoint{4.800267in}{2.933595in}}%
\pgfpathlineto{\pgfqpoint{4.808749in}{3.229782in}}%
\pgfpathlineto{\pgfqpoint{4.814797in}{3.348774in}}%
\pgfpathlineto{\pgfqpoint{4.818860in}{3.380821in}}%
\pgfpathlineto{\pgfqpoint{4.820255in}{3.382705in}}%
\pgfpathlineto{\pgfqpoint{4.820879in}{3.382031in}}%
\pgfpathlineto{\pgfqpoint{4.822520in}{3.375803in}}%
\pgfpathlineto{\pgfqpoint{4.825277in}{3.350904in}}%
\pgfpathlineto{\pgfqpoint{4.829339in}{3.282014in}}%
\pgfpathlineto{\pgfqpoint{4.834930in}{3.128285in}}%
\pgfpathlineto{\pgfqpoint{4.842552in}{2.824067in}}%
\pgfpathlineto{\pgfqpoint{4.854147in}{2.220230in}}%
\pgfpathlineto{\pgfqpoint{4.876133in}{1.072124in}}%
\pgfpathlineto{\pgfqpoint{4.884603in}{0.787401in}}%
\pgfpathlineto{\pgfqpoint{4.890697in}{0.671467in}}%
\pgfpathlineto{\pgfqpoint{4.894837in}{0.639376in}}%
\pgfpathlineto{\pgfqpoint{4.896344in}{0.637296in}}%
\pgfpathlineto{\pgfqpoint{4.896935in}{0.637882in}}%
\pgfpathlineto{\pgfqpoint{4.898564in}{0.643572in}}%
\pgfpathlineto{\pgfqpoint{4.901310in}{0.666575in}}%
\pgfpathlineto{\pgfqpoint{4.905361in}{0.730509in}}%
\pgfpathlineto{\pgfqpoint{4.910941in}{0.873567in}}%
\pgfpathlineto{\pgfqpoint{4.918507in}{1.155748in}}%
\pgfpathlineto{\pgfqpoint{4.929734in}{1.705124in}}%
\pgfpathlineto{\pgfqpoint{4.954967in}{2.964881in}}%
\pgfpathlineto{\pgfqpoint{4.963516in}{3.234243in}}%
\pgfpathlineto{\pgfqpoint{4.969743in}{3.346904in}}%
\pgfpathlineto{\pgfqpoint{4.974040in}{3.379935in}}%
\pgfpathlineto{\pgfqpoint{4.975848in}{3.382706in}}%
\pgfpathlineto{\pgfqpoint{4.976350in}{3.382307in}}%
\pgfpathlineto{\pgfqpoint{4.977946in}{3.377675in}}%
\pgfpathlineto{\pgfqpoint{4.980624in}{3.358511in}}%
\pgfpathlineto{\pgfqpoint{4.984597in}{3.304423in}}%
\pgfpathlineto{\pgfqpoint{4.990077in}{3.182330in}}%
\pgfpathlineto{\pgfqpoint{4.997453in}{2.941335in}}%
\pgfpathlineto{\pgfqpoint{5.007933in}{2.486038in}}%
\pgfpathlineto{\pgfqpoint{5.041134in}{0.970882in}}%
\pgfpathlineto{\pgfqpoint{5.049370in}{0.755495in}}%
\pgfpathlineto{\pgfqpoint{5.055474in}{0.665159in}}%
\pgfpathlineto{\pgfqpoint{5.059693in}{0.639233in}}%
\pgfpathlineto{\pgfqpoint{5.061445in}{0.637317in}}%
\pgfpathlineto{\pgfqpoint{5.061936in}{0.637708in}}%
\pgfpathlineto{\pgfqpoint{5.063643in}{0.642218in}}%
\pgfpathlineto{\pgfqpoint{5.066445in}{0.660096in}}%
\pgfpathlineto{\pgfqpoint{5.070551in}{0.709239in}}%
\pgfpathlineto{\pgfqpoint{5.076165in}{0.818054in}}%
\pgfpathlineto{\pgfqpoint{5.083653in}{1.029673in}}%
\pgfpathlineto{\pgfqpoint{5.094021in}{1.419684in}}%
\pgfpathlineto{\pgfqpoint{5.136206in}{3.112912in}}%
\pgfpathlineto{\pgfqpoint{5.144386in}{3.284744in}}%
\pgfpathlineto{\pgfqpoint{5.150569in}{3.358670in}}%
\pgfpathlineto{\pgfqpoint{5.154910in}{3.380777in}}%
\pgfpathlineto{\pgfqpoint{5.156986in}{3.382651in}}%
\pgfpathlineto{\pgfqpoint{5.157321in}{3.382430in}}%
\pgfpathlineto{\pgfqpoint{5.159106in}{3.378814in}}%
\pgfpathlineto{\pgfqpoint{5.161952in}{3.364657in}}%
\pgfpathlineto{\pgfqpoint{5.166104in}{3.325984in}}%
\pgfpathlineto{\pgfqpoint{5.171739in}{3.241016in}}%
\pgfpathlineto{\pgfqpoint{5.179161in}{3.077603in}}%
\pgfpathlineto{\pgfqpoint{5.189082in}{2.784205in}}%
\pgfpathlineto{\pgfqpoint{5.204282in}{2.230399in}}%
\pgfpathlineto{\pgfqpoint{5.231568in}{1.238071in}}%
\pgfpathlineto{\pgfqpoint{5.243286in}{0.924524in}}%
\pgfpathlineto{\pgfqpoint{5.252359in}{0.756289in}}%
\pgfpathlineto{\pgfqpoint{5.259468in}{0.674909in}}%
\pgfpathlineto{\pgfqpoint{5.264792in}{0.643845in}}%
\pgfpathlineto{\pgfqpoint{5.268296in}{0.637323in}}%
\pgfpathlineto{\pgfqpoint{5.270204in}{0.638363in}}%
\pgfpathlineto{\pgfqpoint{5.272671in}{0.644431in}}%
\pgfpathlineto{\pgfqpoint{5.276320in}{0.662986in}}%
\pgfpathlineto{\pgfqpoint{5.281353in}{0.706658in}}%
\pgfpathlineto{\pgfqpoint{5.287971in}{0.793969in}}%
\pgfpathlineto{\pgfqpoint{5.296520in}{0.951537in}}%
\pgfpathlineto{\pgfqpoint{5.307769in}{1.221264in}}%
\pgfpathlineto{\pgfqpoint{5.324498in}{1.705978in}}%
\pgfpathlineto{\pgfqpoint{5.359205in}{2.721367in}}%
\pgfpathlineto{\pgfqpoint{5.372910in}{3.026948in}}%
\pgfpathlineto{\pgfqpoint{5.383847in}{3.207505in}}%
\pgfpathlineto{\pgfqpoint{5.392775in}{3.308666in}}%
\pgfpathlineto{\pgfqpoint{5.399940in}{3.358781in}}%
\pgfpathlineto{\pgfqpoint{5.400007in}{3.359118in}}%
\pgfpathlineto{\pgfqpoint{5.400007in}{3.359118in}}%
\pgfusepath{stroke}%
\end{pgfscope}%
\begin{pgfscope}%
\pgfsetbuttcap%
\pgfsetroundjoin%
\pgfsetlinewidth{0.803000pt}%
\definecolor{currentstroke}{rgb}{0.000000,0.000000,0.000000}%
\pgfsetstrokecolor{currentstroke}%
\pgfsetdash{}{0pt}%
\pgfsys@defobject{currentmarker}{\pgfqpoint{0.000000in}{0.000000in}}{\pgfqpoint{0.048611in}{0.000000in}}{%
\pgfpathmoveto{\pgfqpoint{0.000000in}{0.000000in}}%
\pgfpathlineto{\pgfqpoint{0.048611in}{0.000000in}}%
\pgfusepath{stroke}%
}%
\begin{pgfscope}%
\pgfsys@transformshift{0.750000in}{2.010000in}%
\pgfsys@useobject{currentmarker}{}%
\end{pgfscope}%
\end{pgfscope}%
\begin{pgfscope}%
\definecolor{textcolor}{rgb}{0.000000,0.000000,0.000000}%
\pgfsetstrokecolor{textcolor}%
\pgfsetfillcolor{textcolor}%
\pgftext[x=0.701389in,y=2.010000in,right,]{\color{textcolor}\rmfamily\fontsize{10.000000}{12.000000}\selectfont \(\displaystyle {0}\)}%
\end{pgfscope}%
\begin{pgfscope}%
\pgfsetrectcap%
\pgfsetroundjoin%
\pgfsetlinewidth{0.803000pt}%
\definecolor{currentstroke}{rgb}{0.000000,0.000000,0.000000}%
\pgfsetstrokecolor{currentstroke}%
\pgfsetdash{}{0pt}%
\pgfpathmoveto{\pgfqpoint{0.750000in}{0.500000in}}%
\pgfpathlineto{\pgfqpoint{0.750000in}{3.520000in}}%
\pgfusepath{stroke}%
\end{pgfscope}%
\begin{pgfscope}%
\pgfsetroundcap%
\pgfsetroundjoin%
\pgfsetlinewidth{1.003750pt}%
\definecolor{currentstroke}{rgb}{0.000000,0.000000,0.000000}%
\pgfsetstrokecolor{currentstroke}%
\pgfsetdash{}{0pt}%
\pgfpathmoveto{\pgfqpoint{0.750000in}{2.010000in}}%
\pgfpathlineto{\pgfqpoint{5.400000in}{2.010000in}}%
\pgfpathlineto{\pgfqpoint{5.523361in}{2.010000in}}%
\pgfusepath{stroke}%
\end{pgfscope}%
\begin{pgfscope}%
\pgfsetroundcap%
\pgfsetroundjoin%
\definecolor{currentfill}{rgb}{0.121569,0.466667,0.705882}%
\pgfsetfillcolor{currentfill}%
\pgfsetlinewidth{1.003750pt}%
\definecolor{currentstroke}{rgb}{0.000000,0.000000,0.000000}%
\pgfsetstrokecolor{currentstroke}%
\pgfsetdash{}{0pt}%
\pgfpathmoveto{\pgfqpoint{5.467805in}{2.037778in}}%
\pgfpathlineto{\pgfqpoint{5.523361in}{2.010000in}}%
\pgfpathlineto{\pgfqpoint{5.467805in}{1.982222in}}%
\pgfpathlineto{\pgfqpoint{5.467805in}{2.037778in}}%
\pgfpathlineto{\pgfqpoint{5.467805in}{2.037778in}}%
\pgfpathclose%
\pgfusepath{stroke,fill}%
\end{pgfscope}%
\begin{pgfscope}%
\definecolor{textcolor}{rgb}{0.000000,0.000000,0.000000}%
\pgfsetstrokecolor{textcolor}%
\pgfsetfillcolor{textcolor}%
\pgftext[x=5.632500in,y=1.872727in,left,base]{\color{textcolor}\rmfamily\fontsize{10.000000}{12.000000}\selectfont \(\displaystyle t\)}%
\end{pgfscope}%
\end{pgfpicture}%
\makeatother%
\endgroup%

	\caption{Modulationsart Phasenmodulation mit dem Parameter \(\varphi\)}
	\label{fig:fm:PM}
\end{figure}

\begin{figure}[h]
	\centering
	%% Creator: Matplotlib, PGF backend
%%
%% To include the figure in your LaTeX document, write
%%   \input{<filename>.pgf}
%%
%% Make sure the required packages are loaded in your preamble
%%   \usepackage{pgf}
%%
%% Also ensure that all the required font packages are loaded; for instance,
%% the lmodern package is sometimes necessary when using math font.
%%   \usepackage{lmodern}
%%
%% Figures using additional raster images can only be included by \input if
%% they are in the same directory as the main LaTeX file. For loading figures
%% from other directories you can use the `import` package
%%   \usepackage{import}
%%
%% and then include the figures with
%%   \import{<path to file>}{<filename>.pgf}
%%
%% Matplotlib used the following preamble
%%
\begingroup%
\makeatletter%
\begin{pgfpicture}%
\pgfpathrectangle{\pgfpointorigin}{\pgfqpoint{6.000000in}{4.000000in}}%
\pgfusepath{use as bounding box, clip}%
\begin{pgfscope}%
\pgfsetbuttcap%
\pgfsetmiterjoin%
\pgfsetlinewidth{0.000000pt}%
\definecolor{currentstroke}{rgb}{1.000000,1.000000,1.000000}%
\pgfsetstrokecolor{currentstroke}%
\pgfsetstrokeopacity{0.000000}%
\pgfsetdash{}{0pt}%
\pgfpathmoveto{\pgfqpoint{0.000000in}{0.000000in}}%
\pgfpathlineto{\pgfqpoint{6.000000in}{0.000000in}}%
\pgfpathlineto{\pgfqpoint{6.000000in}{4.000000in}}%
\pgfpathlineto{\pgfqpoint{0.000000in}{4.000000in}}%
\pgfpathlineto{\pgfqpoint{0.000000in}{0.000000in}}%
\pgfpathclose%
\pgfusepath{}%
\end{pgfscope}%
\begin{pgfscope}%
\pgfsetbuttcap%
\pgfsetmiterjoin%
\definecolor{currentfill}{rgb}{1.000000,1.000000,1.000000}%
\pgfsetfillcolor{currentfill}%
\pgfsetlinewidth{0.000000pt}%
\definecolor{currentstroke}{rgb}{0.000000,0.000000,0.000000}%
\pgfsetstrokecolor{currentstroke}%
\pgfsetstrokeopacity{0.000000}%
\pgfsetdash{}{0pt}%
\pgfpathmoveto{\pgfqpoint{0.750000in}{0.500000in}}%
\pgfpathlineto{\pgfqpoint{5.400000in}{0.500000in}}%
\pgfpathlineto{\pgfqpoint{5.400000in}{3.520000in}}%
\pgfpathlineto{\pgfqpoint{0.750000in}{3.520000in}}%
\pgfpathlineto{\pgfqpoint{0.750000in}{0.500000in}}%
\pgfpathclose%
\pgfusepath{fill}%
\end{pgfscope}%
\begin{pgfscope}%
\pgfpathrectangle{\pgfqpoint{0.750000in}{0.500000in}}{\pgfqpoint{4.650000in}{3.020000in}}%
\pgfusepath{clip}%
\pgfsetrectcap%
\pgfsetroundjoin%
\pgfsetlinewidth{1.505625pt}%
\definecolor{currentstroke}{rgb}{0.121569,0.466667,0.705882}%
\pgfsetstrokecolor{currentstroke}%
\pgfsetdash{}{0pt}%
\pgfpathmoveto{\pgfqpoint{0.749993in}{2.494653in}}%
\pgfpathlineto{\pgfqpoint{0.827031in}{2.415968in}}%
\pgfpathlineto{\pgfqpoint{0.877999in}{2.368254in}}%
\pgfpathlineto{\pgfqpoint{0.921400in}{2.331745in}}%
\pgfpathlineto{\pgfqpoint{0.960583in}{2.302732in}}%
\pgfpathlineto{\pgfqpoint{0.997032in}{2.279517in}}%
\pgfpathlineto{\pgfqpoint{1.031617in}{2.261094in}}%
\pgfpathlineto{\pgfqpoint{1.064952in}{2.246794in}}%
\pgfpathlineto{\pgfqpoint{1.097528in}{2.236152in}}%
\pgfpathlineto{\pgfqpoint{1.129780in}{2.228847in}}%
\pgfpathlineto{\pgfqpoint{1.162167in}{2.224674in}}%
\pgfpathlineto{\pgfqpoint{1.195156in}{2.223546in}}%
\pgfpathlineto{\pgfqpoint{1.229316in}{2.225491in}}%
\pgfpathlineto{\pgfqpoint{1.265374in}{2.230675in}}%
\pgfpathlineto{\pgfqpoint{1.304435in}{2.239469in}}%
\pgfpathlineto{\pgfqpoint{1.348450in}{2.252636in}}%
\pgfpathlineto{\pgfqpoint{1.402297in}{2.272134in}}%
\pgfpathlineto{\pgfqpoint{1.601369in}{2.347457in}}%
\pgfpathlineto{\pgfqpoint{1.645529in}{2.358893in}}%
\pgfpathlineto{\pgfqpoint{1.685460in}{2.366090in}}%
\pgfpathlineto{\pgfqpoint{1.722935in}{2.369739in}}%
\pgfpathlineto{\pgfqpoint{1.759005in}{2.370165in}}%
\pgfpathlineto{\pgfqpoint{1.794393in}{2.367498in}}%
\pgfpathlineto{\pgfqpoint{1.829737in}{2.361730in}}%
\pgfpathlineto{\pgfqpoint{1.865639in}{2.352721in}}%
\pgfpathlineto{\pgfqpoint{1.902779in}{2.340182in}}%
\pgfpathlineto{\pgfqpoint{1.942018in}{2.323625in}}%
\pgfpathlineto{\pgfqpoint{1.984694in}{2.302189in}}%
\pgfpathlineto{\pgfqpoint{2.033430in}{2.274134in}}%
\pgfpathlineto{\pgfqpoint{2.096406in}{2.234057in}}%
\pgfpathlineto{\pgfqpoint{2.234901in}{2.144868in}}%
\pgfpathlineto{\pgfqpoint{2.281640in}{2.119182in}}%
\pgfpathlineto{\pgfqpoint{2.321213in}{2.100867in}}%
\pgfpathlineto{\pgfqpoint{2.356624in}{2.087812in}}%
\pgfpathlineto{\pgfqpoint{2.389267in}{2.079014in}}%
\pgfpathlineto{\pgfqpoint{2.419990in}{2.073892in}}%
\pgfpathlineto{\pgfqpoint{2.449375in}{2.072097in}}%
\pgfpathlineto{\pgfqpoint{2.477900in}{2.073435in}}%
\pgfpathlineto{\pgfqpoint{2.505945in}{2.077842in}}%
\pgfpathlineto{\pgfqpoint{2.533867in}{2.085368in}}%
\pgfpathlineto{\pgfqpoint{2.561979in}{2.096166in}}%
\pgfpathlineto{\pgfqpoint{2.590616in}{2.110511in}}%
\pgfpathlineto{\pgfqpoint{2.620078in}{2.128770in}}%
\pgfpathlineto{\pgfqpoint{2.650713in}{2.151449in}}%
\pgfpathlineto{\pgfqpoint{2.682909in}{2.179199in}}%
\pgfpathlineto{\pgfqpoint{2.717171in}{2.212891in}}%
\pgfpathlineto{\pgfqpoint{2.754222in}{2.253765in}}%
\pgfpathlineto{\pgfqpoint{2.795224in}{2.303731in}}%
\pgfpathlineto{\pgfqpoint{2.842531in}{2.366438in}}%
\pgfpathlineto{\pgfqpoint{2.903175in}{2.452312in}}%
\pgfpathlineto{\pgfqpoint{3.058176in}{2.674180in}}%
\pgfpathlineto{\pgfqpoint{3.103810in}{2.732656in}}%
\pgfpathlineto{\pgfqpoint{3.142434in}{2.777291in}}%
\pgfpathlineto{\pgfqpoint{3.176673in}{2.812272in}}%
\pgfpathlineto{\pgfqpoint{3.207754in}{2.839726in}}%
\pgfpathlineto{\pgfqpoint{3.236413in}{2.861021in}}%
\pgfpathlineto{\pgfqpoint{3.263164in}{2.877135in}}%
\pgfpathlineto{\pgfqpoint{3.288408in}{2.888804in}}%
\pgfpathlineto{\pgfqpoint{3.312480in}{2.896582in}}%
\pgfpathlineto{\pgfqpoint{3.335704in}{2.900876in}}%
\pgfpathlineto{\pgfqpoint{3.358370in}{2.901945in}}%
\pgfpathlineto{\pgfqpoint{3.380779in}{2.899912in}}%
\pgfpathlineto{\pgfqpoint{3.403211in}{2.894763in}}%
\pgfpathlineto{\pgfqpoint{3.425944in}{2.886347in}}%
\pgfpathlineto{\pgfqpoint{3.449235in}{2.874387in}}%
\pgfpathlineto{\pgfqpoint{3.473340in}{2.858479in}}%
\pgfpathlineto{\pgfqpoint{3.498506in}{2.838099in}}%
\pgfpathlineto{\pgfqpoint{3.524978in}{2.812602in}}%
\pgfpathlineto{\pgfqpoint{3.553045in}{2.781181in}}%
\pgfpathlineto{\pgfqpoint{3.583043in}{2.742851in}}%
\pgfpathlineto{\pgfqpoint{3.615430in}{2.696328in}}%
\pgfpathlineto{\pgfqpoint{3.650863in}{2.639871in}}%
\pgfpathlineto{\pgfqpoint{3.690447in}{2.570805in}}%
\pgfpathlineto{\pgfqpoint{3.736404in}{2.484156in}}%
\pgfpathlineto{\pgfqpoint{3.795117in}{2.366416in}}%
\pgfpathlineto{\pgfqpoint{3.962060in}{2.027705in}}%
\pgfpathlineto{\pgfqpoint{4.007347in}{1.945106in}}%
\pgfpathlineto{\pgfqpoint{4.045994in}{1.880834in}}%
\pgfpathlineto{\pgfqpoint{4.080345in}{1.829521in}}%
\pgfpathlineto{\pgfqpoint{4.111515in}{1.788351in}}%
\pgfpathlineto{\pgfqpoint{4.140185in}{1.755457in}}%
\pgfpathlineto{\pgfqpoint{4.166802in}{1.729495in}}%
\pgfpathlineto{\pgfqpoint{4.191711in}{1.709404in}}%
\pgfpathlineto{\pgfqpoint{4.215203in}{1.694334in}}%
\pgfpathlineto{\pgfqpoint{4.237545in}{1.683597in}}%
\pgfpathlineto{\pgfqpoint{4.259006in}{1.676655in}}%
\pgfpathlineto{\pgfqpoint{4.279864in}{1.673120in}}%
\pgfpathlineto{\pgfqpoint{4.300387in}{1.672759in}}%
\pgfpathlineto{\pgfqpoint{4.320877in}{1.675493in}}%
\pgfpathlineto{\pgfqpoint{4.341601in}{1.681399in}}%
\pgfpathlineto{\pgfqpoint{4.362839in}{1.690704in}}%
\pgfpathlineto{\pgfqpoint{4.384846in}{1.703776in}}%
\pgfpathlineto{\pgfqpoint{4.407869in}{1.721118in}}%
\pgfpathlineto{\pgfqpoint{4.432154in}{1.743364in}}%
\pgfpathlineto{\pgfqpoint{4.457956in}{1.771290in}}%
\pgfpathlineto{\pgfqpoint{4.485577in}{1.805852in}}%
\pgfpathlineto{\pgfqpoint{4.515385in}{1.848230in}}%
\pgfpathlineto{\pgfqpoint{4.547894in}{1.899966in}}%
\pgfpathlineto{\pgfqpoint{4.583896in}{1.963244in}}%
\pgfpathlineto{\pgfqpoint{4.624787in}{2.041582in}}%
\pgfpathlineto{\pgfqpoint{4.673656in}{2.142202in}}%
\pgfpathlineto{\pgfqpoint{4.742279in}{2.291288in}}%
\pgfpathlineto{\pgfqpoint{4.857094in}{2.540797in}}%
\pgfpathlineto{\pgfqpoint{4.907403in}{2.642189in}}%
\pgfpathlineto{\pgfqpoint{4.949153in}{2.719610in}}%
\pgfpathlineto{\pgfqpoint{4.985892in}{2.781431in}}%
\pgfpathlineto{\pgfqpoint{5.019115in}{2.831463in}}%
\pgfpathlineto{\pgfqpoint{5.049615in}{2.871965in}}%
\pgfpathlineto{\pgfqpoint{5.077917in}{2.904551in}}%
\pgfpathlineto{\pgfqpoint{5.104389in}{2.930440in}}%
\pgfpathlineto{\pgfqpoint{5.129331in}{2.950617in}}%
\pgfpathlineto{\pgfqpoint{5.153002in}{2.965878in}}%
\pgfpathlineto{\pgfqpoint{5.175645in}{2.976872in}}%
\pgfpathlineto{\pgfqpoint{5.197519in}{2.984111in}}%
\pgfpathlineto{\pgfqpoint{5.218891in}{2.987961in}}%
\pgfpathlineto{\pgfqpoint{5.240028in}{2.988642in}}%
\pgfpathlineto{\pgfqpoint{5.261209in}{2.986228in}}%
\pgfpathlineto{\pgfqpoint{5.282726in}{2.980643in}}%
\pgfpathlineto{\pgfqpoint{5.304856in}{2.971668in}}%
\pgfpathlineto{\pgfqpoint{5.327879in}{2.958941in}}%
\pgfpathlineto{\pgfqpoint{5.352063in}{2.941975in}}%
\pgfpathlineto{\pgfqpoint{5.377720in}{2.920119in}}%
\pgfpathlineto{\pgfqpoint{5.400007in}{2.898074in}}%
\pgfpathlineto{\pgfqpoint{5.400007in}{2.898074in}}%
\pgfusepath{stroke}%
\end{pgfscope}%
\begin{pgfscope}%
\pgfpathrectangle{\pgfqpoint{0.750000in}{0.500000in}}{\pgfqpoint{4.650000in}{3.020000in}}%
\pgfusepath{clip}%
\pgfsetrectcap%
\pgfsetroundjoin%
\pgfsetlinewidth{1.505625pt}%
\definecolor{currentstroke}{rgb}{1.000000,0.498039,0.054902}%
\pgfsetstrokecolor{currentstroke}%
\pgfsetdash{}{0pt}%
\pgfpathmoveto{\pgfqpoint{0.749993in}{1.618911in}}%
\pgfpathlineto{\pgfqpoint{0.775516in}{3.041961in}}%
\pgfpathlineto{\pgfqpoint{0.782983in}{3.278414in}}%
\pgfpathlineto{\pgfqpoint{0.788306in}{3.365042in}}%
\pgfpathlineto{\pgfqpoint{0.791698in}{3.382593in}}%
\pgfpathlineto{\pgfqpoint{0.792502in}{3.382435in}}%
\pgfpathlineto{\pgfqpoint{0.792948in}{3.381637in}}%
\pgfpathlineto{\pgfqpoint{0.794723in}{3.373481in}}%
\pgfpathlineto{\pgfqpoint{0.797658in}{3.342793in}}%
\pgfpathlineto{\pgfqpoint{0.801966in}{3.260561in}}%
\pgfpathlineto{\pgfqpoint{0.807936in}{3.080213in}}%
\pgfpathlineto{\pgfqpoint{0.816318in}{2.722643in}}%
\pgfpathlineto{\pgfqpoint{0.831785in}{1.900661in}}%
\pgfpathlineto{\pgfqpoint{0.847130in}{1.151345in}}%
\pgfpathlineto{\pgfqpoint{0.856315in}{0.840591in}}%
\pgfpathlineto{\pgfqpoint{0.863223in}{0.698386in}}%
\pgfpathlineto{\pgfqpoint{0.868301in}{0.646887in}}%
\pgfpathlineto{\pgfqpoint{0.871537in}{0.637288in}}%
\pgfpathlineto{\pgfqpoint{0.872921in}{0.638563in}}%
\pgfpathlineto{\pgfqpoint{0.875008in}{0.646417in}}%
\pgfpathlineto{\pgfqpoint{0.878311in}{0.672927in}}%
\pgfpathlineto{\pgfqpoint{0.883077in}{0.739626in}}%
\pgfpathlineto{\pgfqpoint{0.889639in}{0.880011in}}%
\pgfpathlineto{\pgfqpoint{0.898790in}{1.149426in}}%
\pgfpathlineto{\pgfqpoint{0.914102in}{1.706914in}}%
\pgfpathlineto{\pgfqpoint{0.937649in}{2.545912in}}%
\pgfpathlineto{\pgfqpoint{0.950427in}{2.892988in}}%
\pgfpathlineto{\pgfqpoint{0.961163in}{3.106787in}}%
\pgfpathlineto{\pgfqpoint{0.970571in}{3.237429in}}%
\pgfpathlineto{\pgfqpoint{0.978886in}{3.313662in}}%
\pgfpathlineto{\pgfqpoint{0.986173in}{3.354819in}}%
\pgfpathlineto{\pgfqpoint{0.992389in}{3.374296in}}%
\pgfpathlineto{\pgfqpoint{0.997478in}{3.381512in}}%
\pgfpathlineto{\pgfqpoint{1.001585in}{3.382664in}}%
\pgfpathlineto{\pgfqpoint{1.005670in}{3.380445in}}%
\pgfpathlineto{\pgfqpoint{1.011060in}{3.373530in}}%
\pgfpathlineto{\pgfqpoint{1.019296in}{3.357209in}}%
\pgfpathlineto{\pgfqpoint{1.045678in}{3.301693in}}%
\pgfpathlineto{\pgfqpoint{1.053234in}{3.293845in}}%
\pgfpathlineto{\pgfqpoint{1.059249in}{3.291663in}}%
\pgfpathlineto{\pgfqpoint{1.064684in}{3.292936in}}%
\pgfpathlineto{\pgfqpoint{1.070565in}{3.297714in}}%
\pgfpathlineto{\pgfqpoint{1.077730in}{3.307891in}}%
\pgfpathlineto{\pgfqpoint{1.087506in}{3.327763in}}%
\pgfpathlineto{\pgfqpoint{1.111154in}{3.377862in}}%
\pgfpathlineto{\pgfqpoint{1.116756in}{3.382465in}}%
\pgfpathlineto{\pgfqpoint{1.120685in}{3.382121in}}%
\pgfpathlineto{\pgfqpoint{1.124401in}{3.378413in}}%
\pgfpathlineto{\pgfqpoint{1.128720in}{3.369189in}}%
\pgfpathlineto{\pgfqpoint{1.133842in}{3.350167in}}%
\pgfpathlineto{\pgfqpoint{1.139802in}{3.314997in}}%
\pgfpathlineto{\pgfqpoint{1.146598in}{3.254771in}}%
\pgfpathlineto{\pgfqpoint{1.154254in}{3.157208in}}%
\pgfpathlineto{\pgfqpoint{1.162870in}{3.005022in}}%
\pgfpathlineto{\pgfqpoint{1.172679in}{2.773139in}}%
\pgfpathlineto{\pgfqpoint{1.184275in}{2.420397in}}%
\pgfpathlineto{\pgfqpoint{1.200267in}{1.828337in}}%
\pgfpathlineto{\pgfqpoint{1.223547in}{0.977137in}}%
\pgfpathlineto{\pgfqpoint{1.232631in}{0.752508in}}%
\pgfpathlineto{\pgfqpoint{1.238903in}{0.662753in}}%
\pgfpathlineto{\pgfqpoint{1.243032in}{0.638649in}}%
\pgfpathlineto{\pgfqpoint{1.244516in}{0.637330in}}%
\pgfpathlineto{\pgfqpoint{1.245074in}{0.637872in}}%
\pgfpathlineto{\pgfqpoint{1.246804in}{0.643208in}}%
\pgfpathlineto{\pgfqpoint{1.249605in}{0.663782in}}%
\pgfpathlineto{\pgfqpoint{1.253623in}{0.719522in}}%
\pgfpathlineto{\pgfqpoint{1.258980in}{0.842083in}}%
\pgfpathlineto{\pgfqpoint{1.265910in}{1.079039in}}%
\pgfpathlineto{\pgfqpoint{1.275151in}{1.512054in}}%
\pgfpathlineto{\pgfqpoint{1.309791in}{3.254951in}}%
\pgfpathlineto{\pgfqpoint{1.315327in}{3.360556in}}%
\pgfpathlineto{\pgfqpoint{1.318798in}{3.382497in}}%
\pgfpathlineto{\pgfqpoint{1.319478in}{3.382601in}}%
\pgfpathlineto{\pgfqpoint{1.320047in}{3.381616in}}%
\pgfpathlineto{\pgfqpoint{1.321733in}{3.372944in}}%
\pgfpathlineto{\pgfqpoint{1.324545in}{3.339244in}}%
\pgfpathlineto{\pgfqpoint{1.328641in}{3.247573in}}%
\pgfpathlineto{\pgfqpoint{1.334254in}{3.044048in}}%
\pgfpathlineto{\pgfqpoint{1.341988in}{2.638307in}}%
\pgfpathlineto{\pgfqpoint{1.356295in}{1.686307in}}%
\pgfpathlineto{\pgfqpoint{1.368437in}{0.972126in}}%
\pgfpathlineto{\pgfqpoint{1.375368in}{0.722862in}}%
\pgfpathlineto{\pgfqpoint{1.380010in}{0.646016in}}%
\pgfpathlineto{\pgfqpoint{1.382198in}{0.637275in}}%
\pgfpathlineto{\pgfqpoint{1.382599in}{0.637626in}}%
\pgfpathlineto{\pgfqpoint{1.383860in}{0.642695in}}%
\pgfpathlineto{\pgfqpoint{1.386171in}{0.667565in}}%
\pgfpathlineto{\pgfqpoint{1.389731in}{0.744797in}}%
\pgfpathlineto{\pgfqpoint{1.394764in}{0.930135in}}%
\pgfpathlineto{\pgfqpoint{1.401839in}{1.318713in}}%
\pgfpathlineto{\pgfqpoint{1.414550in}{2.224069in}}%
\pgfpathlineto{\pgfqpoint{1.427061in}{3.028623in}}%
\pgfpathlineto{\pgfqpoint{1.433891in}{3.293877in}}%
\pgfpathlineto{\pgfqpoint{1.438455in}{3.374135in}}%
\pgfpathlineto{\pgfqpoint{1.440531in}{3.382725in}}%
\pgfpathlineto{\pgfqpoint{1.440977in}{3.382254in}}%
\pgfpathlineto{\pgfqpoint{1.442272in}{3.376242in}}%
\pgfpathlineto{\pgfqpoint{1.444638in}{3.347502in}}%
\pgfpathlineto{\pgfqpoint{1.448276in}{3.259617in}}%
\pgfpathlineto{\pgfqpoint{1.453454in}{3.049918in}}%
\pgfpathlineto{\pgfqpoint{1.460909in}{2.606772in}}%
\pgfpathlineto{\pgfqpoint{1.489300in}{0.781643in}}%
\pgfpathlineto{\pgfqpoint{1.494501in}{0.659175in}}%
\pgfpathlineto{\pgfqpoint{1.497693in}{0.637290in}}%
\pgfpathlineto{\pgfqpoint{1.498485in}{0.638269in}}%
\pgfpathlineto{\pgfqpoint{1.498585in}{0.638576in}}%
\pgfpathlineto{\pgfqpoint{1.500204in}{0.649135in}}%
\pgfpathlineto{\pgfqpoint{1.502971in}{0.691439in}}%
\pgfpathlineto{\pgfqpoint{1.507101in}{0.809022in}}%
\pgfpathlineto{\pgfqpoint{1.512960in}{1.075473in}}%
\pgfpathlineto{\pgfqpoint{1.521787in}{1.637334in}}%
\pgfpathlineto{\pgfqpoint{1.543482in}{3.061475in}}%
\pgfpathlineto{\pgfqpoint{1.550201in}{3.300092in}}%
\pgfpathlineto{\pgfqpoint{1.554798in}{3.374224in}}%
\pgfpathlineto{\pgfqpoint{1.556997in}{3.382726in}}%
\pgfpathlineto{\pgfqpoint{1.557388in}{3.382406in}}%
\pgfpathlineto{\pgfqpoint{1.558649in}{3.377627in}}%
\pgfpathlineto{\pgfqpoint{1.560959in}{3.354193in}}%
\pgfpathlineto{\pgfqpoint{1.564552in}{3.281205in}}%
\pgfpathlineto{\pgfqpoint{1.569697in}{3.105421in}}%
\pgfpathlineto{\pgfqpoint{1.577040in}{2.735606in}}%
\pgfpathlineto{\pgfqpoint{1.590667in}{1.857621in}}%
\pgfpathlineto{\pgfqpoint{1.604126in}{1.070893in}}%
\pgfpathlineto{\pgfqpoint{1.611927in}{0.776280in}}%
\pgfpathlineto{\pgfqpoint{1.617507in}{0.664104in}}%
\pgfpathlineto{\pgfqpoint{1.621178in}{0.637958in}}%
\pgfpathlineto{\pgfqpoint{1.622060in}{0.637318in}}%
\pgfpathlineto{\pgfqpoint{1.622730in}{0.638276in}}%
\pgfpathlineto{\pgfqpoint{1.624426in}{0.646230in}}%
\pgfpathlineto{\pgfqpoint{1.627283in}{0.677149in}}%
\pgfpathlineto{\pgfqpoint{1.631524in}{0.761558in}}%
\pgfpathlineto{\pgfqpoint{1.637472in}{0.949244in}}%
\pgfpathlineto{\pgfqpoint{1.645976in}{1.326941in}}%
\pgfpathlineto{\pgfqpoint{1.665216in}{2.367919in}}%
\pgfpathlineto{\pgfqpoint{1.677771in}{2.943918in}}%
\pgfpathlineto{\pgfqpoint{1.686420in}{3.213324in}}%
\pgfpathlineto{\pgfqpoint{1.693004in}{3.334641in}}%
\pgfpathlineto{\pgfqpoint{1.697814in}{3.376307in}}%
\pgfpathlineto{\pgfqpoint{1.700749in}{3.382714in}}%
\pgfpathlineto{\pgfqpoint{1.700783in}{3.382706in}}%
\pgfpathlineto{\pgfqpoint{1.702200in}{3.380726in}}%
\pgfpathlineto{\pgfqpoint{1.704521in}{3.370676in}}%
\pgfpathlineto{\pgfqpoint{1.708115in}{3.339153in}}%
\pgfpathlineto{\pgfqpoint{1.713237in}{3.263282in}}%
\pgfpathlineto{\pgfqpoint{1.720279in}{3.107834in}}%
\pgfpathlineto{\pgfqpoint{1.730256in}{2.811550in}}%
\pgfpathlineto{\pgfqpoint{1.749407in}{2.128885in}}%
\pgfpathlineto{\pgfqpoint{1.768970in}{1.477020in}}%
\pgfpathlineto{\pgfqpoint{1.782028in}{1.139604in}}%
\pgfpathlineto{\pgfqpoint{1.793110in}{0.927496in}}%
\pgfpathlineto{\pgfqpoint{1.802897in}{0.794911in}}%
\pgfpathlineto{\pgfqpoint{1.811602in}{0.715436in}}%
\pgfpathlineto{\pgfqpoint{1.819291in}{0.670925in}}%
\pgfpathlineto{\pgfqpoint{1.825931in}{0.648639in}}%
\pgfpathlineto{\pgfqpoint{1.831455in}{0.639504in}}%
\pgfpathlineto{\pgfqpoint{1.835931in}{0.637278in}}%
\pgfpathlineto{\pgfqpoint{1.840082in}{0.638643in}}%
\pgfpathlineto{\pgfqpoint{1.845160in}{0.643955in}}%
\pgfpathlineto{\pgfqpoint{1.852302in}{0.656458in}}%
\pgfpathlineto{\pgfqpoint{1.864355in}{0.684864in}}%
\pgfpathlineto{\pgfqpoint{1.884733in}{0.732028in}}%
\pgfpathlineto{\pgfqpoint{1.894588in}{0.747253in}}%
\pgfpathlineto{\pgfqpoint{1.902333in}{0.754042in}}%
\pgfpathlineto{\pgfqpoint{1.908828in}{0.755912in}}%
\pgfpathlineto{\pgfqpoint{1.914921in}{0.754462in}}%
\pgfpathlineto{\pgfqpoint{1.921506in}{0.749530in}}%
\pgfpathlineto{\pgfqpoint{1.929362in}{0.739466in}}%
\pgfpathlineto{\pgfqpoint{1.939473in}{0.721023in}}%
\pgfpathlineto{\pgfqpoint{1.957006in}{0.681239in}}%
\pgfpathlineto{\pgfqpoint{1.971904in}{0.650713in}}%
\pgfpathlineto{\pgfqpoint{1.980085in}{0.640280in}}%
\pgfpathlineto{\pgfqpoint{1.985854in}{0.637334in}}%
\pgfpathlineto{\pgfqpoint{1.990452in}{0.638284in}}%
\pgfpathlineto{\pgfqpoint{1.995039in}{0.642635in}}%
\pgfpathlineto{\pgfqpoint{2.000262in}{0.652295in}}%
\pgfpathlineto{\pgfqpoint{2.006333in}{0.670636in}}%
\pgfpathlineto{\pgfqpoint{2.013330in}{0.702433in}}%
\pgfpathlineto{\pgfqpoint{2.021299in}{0.754120in}}%
\pgfpathlineto{\pgfqpoint{2.030305in}{0.834253in}}%
\pgfpathlineto{\pgfqpoint{2.040483in}{0.954324in}}%
\pgfpathlineto{\pgfqpoint{2.052100in}{1.130325in}}%
\pgfpathlineto{\pgfqpoint{2.065760in}{1.387324in}}%
\pgfpathlineto{\pgfqpoint{2.083270in}{1.780023in}}%
\pgfpathlineto{\pgfqpoint{2.133914in}{2.953645in}}%
\pgfpathlineto{\pgfqpoint{2.146581in}{3.159231in}}%
\pgfpathlineto{\pgfqpoint{2.156737in}{3.278215in}}%
\pgfpathlineto{\pgfqpoint{2.164973in}{3.342115in}}%
\pgfpathlineto{\pgfqpoint{2.171479in}{3.371529in}}%
\pgfpathlineto{\pgfqpoint{2.176311in}{3.381476in}}%
\pgfpathlineto{\pgfqpoint{2.179559in}{3.382594in}}%
\pgfpathlineto{\pgfqpoint{2.182382in}{3.380024in}}%
\pgfpathlineto{\pgfqpoint{2.186110in}{3.371736in}}%
\pgfpathlineto{\pgfqpoint{2.191154in}{3.352032in}}%
\pgfpathlineto{\pgfqpoint{2.197783in}{3.312332in}}%
\pgfpathlineto{\pgfqpoint{2.206343in}{3.240535in}}%
\pgfpathlineto{\pgfqpoint{2.217603in}{3.117580in}}%
\pgfpathlineto{\pgfqpoint{2.234064in}{2.899904in}}%
\pgfpathlineto{\pgfqpoint{2.275100in}{2.345892in}}%
\pgfpathlineto{\pgfqpoint{2.289664in}{2.196007in}}%
\pgfpathlineto{\pgfqpoint{2.301471in}{2.103976in}}%
\pgfpathlineto{\pgfqpoint{2.311169in}{2.050485in}}%
\pgfpathlineto{\pgfqpoint{2.318981in}{2.022836in}}%
\pgfpathlineto{\pgfqpoint{2.325019in}{2.011300in}}%
\pgfpathlineto{\pgfqpoint{2.329438in}{2.008449in}}%
\pgfpathlineto{\pgfqpoint{2.332965in}{2.009636in}}%
\pgfpathlineto{\pgfqpoint{2.336904in}{2.014659in}}%
\pgfpathlineto{\pgfqpoint{2.341848in}{2.026581in}}%
\pgfpathlineto{\pgfqpoint{2.347997in}{2.050311in}}%
\pgfpathlineto{\pgfqpoint{2.355407in}{2.092311in}}%
\pgfpathlineto{\pgfqpoint{2.364134in}{2.160939in}}%
\pgfpathlineto{\pgfqpoint{2.374268in}{2.266819in}}%
\pgfpathlineto{\pgfqpoint{2.386131in}{2.425399in}}%
\pgfpathlineto{\pgfqpoint{2.400940in}{2.668258in}}%
\pgfpathlineto{\pgfqpoint{2.437400in}{3.286428in}}%
\pgfpathlineto{\pgfqpoint{2.444911in}{3.357169in}}%
\pgfpathlineto{\pgfqpoint{2.449866in}{3.379886in}}%
\pgfpathlineto{\pgfqpoint{2.452667in}{3.382639in}}%
\pgfpathlineto{\pgfqpoint{2.452700in}{3.382625in}}%
\pgfpathlineto{\pgfqpoint{2.454497in}{3.380102in}}%
\pgfpathlineto{\pgfqpoint{2.457176in}{3.369797in}}%
\pgfpathlineto{\pgfqpoint{2.460948in}{3.341108in}}%
\pgfpathlineto{\pgfqpoint{2.465847in}{3.277200in}}%
\pgfpathlineto{\pgfqpoint{2.471907in}{3.153535in}}%
\pgfpathlineto{\pgfqpoint{2.479272in}{2.934199in}}%
\pgfpathlineto{\pgfqpoint{2.488390in}{2.562143in}}%
\pgfpathlineto{\pgfqpoint{2.501459in}{1.885571in}}%
\pgfpathlineto{\pgfqpoint{2.520062in}{0.942970in}}%
\pgfpathlineto{\pgfqpoint{2.527383in}{0.715562in}}%
\pgfpathlineto{\pgfqpoint{2.532149in}{0.645463in}}%
\pgfpathlineto{\pgfqpoint{2.534403in}{0.637275in}}%
\pgfpathlineto{\pgfqpoint{2.534782in}{0.637564in}}%
\pgfpathlineto{\pgfqpoint{2.536032in}{0.641990in}}%
\pgfpathlineto{\pgfqpoint{2.538298in}{0.663828in}}%
\pgfpathlineto{\pgfqpoint{2.541746in}{0.731888in}}%
\pgfpathlineto{\pgfqpoint{2.546523in}{0.895459in}}%
\pgfpathlineto{\pgfqpoint{2.552962in}{1.234549in}}%
\pgfpathlineto{\pgfqpoint{2.562582in}{1.927821in}}%
\pgfpathlineto{\pgfqpoint{2.579132in}{3.111286in}}%
\pgfpathlineto{\pgfqpoint{2.585047in}{3.334161in}}%
\pgfpathlineto{\pgfqpoint{2.588652in}{3.381902in}}%
\pgfpathlineto{\pgfqpoint{2.589243in}{3.382718in}}%
\pgfpathlineto{\pgfqpoint{2.589958in}{3.380991in}}%
\pgfpathlineto{\pgfqpoint{2.591520in}{3.366773in}}%
\pgfpathlineto{\pgfqpoint{2.594198in}{3.308834in}}%
\pgfpathlineto{\pgfqpoint{2.598182in}{3.145581in}}%
\pgfpathlineto{\pgfqpoint{2.603863in}{2.768943in}}%
\pgfpathlineto{\pgfqpoint{2.613360in}{1.895326in}}%
\pgfpathlineto{\pgfqpoint{2.625011in}{0.902208in}}%
\pgfpathlineto{\pgfqpoint{2.630312in}{0.673478in}}%
\pgfpathlineto{\pgfqpoint{2.633314in}{0.637273in}}%
\pgfpathlineto{\pgfqpoint{2.633370in}{0.637290in}}%
\pgfpathlineto{\pgfqpoint{2.634028in}{0.639418in}}%
\pgfpathlineto{\pgfqpoint{2.635535in}{0.657756in}}%
\pgfpathlineto{\pgfqpoint{2.638169in}{0.734728in}}%
\pgfpathlineto{\pgfqpoint{2.642164in}{0.955952in}}%
\pgfpathlineto{\pgfqpoint{2.648157in}{1.484217in}}%
\pgfpathlineto{\pgfqpoint{2.667230in}{3.292239in}}%
\pgfpathlineto{\pgfqpoint{2.670868in}{3.381056in}}%
\pgfpathlineto{\pgfqpoint{2.671448in}{3.382726in}}%
\pgfpathlineto{\pgfqpoint{2.672196in}{3.379668in}}%
\pgfpathlineto{\pgfqpoint{2.673736in}{3.354815in}}%
\pgfpathlineto{\pgfqpoint{2.676425in}{3.252202in}}%
\pgfpathlineto{\pgfqpoint{2.680566in}{2.957482in}}%
\pgfpathlineto{\pgfqpoint{2.687239in}{2.225003in}}%
\pgfpathlineto{\pgfqpoint{2.700007in}{0.833144in}}%
\pgfpathlineto{\pgfqpoint{2.704303in}{0.649261in}}%
\pgfpathlineto{\pgfqpoint{2.705687in}{0.637273in}}%
\pgfpathlineto{\pgfqpoint{2.706368in}{0.640227in}}%
\pgfpathlineto{\pgfqpoint{2.707830in}{0.666396in}}%
\pgfpathlineto{\pgfqpoint{2.710441in}{0.779358in}}%
\pgfpathlineto{\pgfqpoint{2.714526in}{1.111676in}}%
\pgfpathlineto{\pgfqpoint{2.721545in}{1.983015in}}%
\pgfpathlineto{\pgfqpoint{2.731813in}{3.176298in}}%
\pgfpathlineto{\pgfqpoint{2.735998in}{3.371682in}}%
\pgfpathlineto{\pgfqpoint{2.737236in}{3.382727in}}%
\pgfpathlineto{\pgfqpoint{2.737939in}{3.379048in}}%
\pgfpathlineto{\pgfqpoint{2.739435in}{3.347220in}}%
\pgfpathlineto{\pgfqpoint{2.742091in}{3.212094in}}%
\pgfpathlineto{\pgfqpoint{2.746298in}{2.815001in}}%
\pgfpathlineto{\pgfqpoint{2.754691in}{1.653653in}}%
\pgfpathlineto{\pgfqpoint{2.762246in}{0.802259in}}%
\pgfpathlineto{\pgfqpoint{2.766063in}{0.641704in}}%
\pgfpathlineto{\pgfqpoint{2.766799in}{0.637273in}}%
\pgfpathlineto{\pgfqpoint{2.767569in}{0.642137in}}%
\pgfpathlineto{\pgfqpoint{2.769121in}{0.681360in}}%
\pgfpathlineto{\pgfqpoint{2.771866in}{0.844026in}}%
\pgfpathlineto{\pgfqpoint{2.776274in}{1.319417in}}%
\pgfpathlineto{\pgfqpoint{2.793304in}{3.360385in}}%
\pgfpathlineto{\pgfqpoint{2.794889in}{3.382727in}}%
\pgfpathlineto{\pgfqpoint{2.795514in}{3.379188in}}%
\pgfpathlineto{\pgfqpoint{2.796920in}{3.345655in}}%
\pgfpathlineto{\pgfqpoint{2.799487in}{3.195877in}}%
\pgfpathlineto{\pgfqpoint{2.803661in}{2.740376in}}%
\pgfpathlineto{\pgfqpoint{2.820724in}{0.649463in}}%
\pgfpathlineto{\pgfqpoint{2.821852in}{0.637273in}}%
\pgfpathlineto{\pgfqpoint{2.822588in}{0.642528in}}%
\pgfpathlineto{\pgfqpoint{2.824106in}{0.686199in}}%
\pgfpathlineto{\pgfqpoint{2.826818in}{0.869858in}}%
\pgfpathlineto{\pgfqpoint{2.831293in}{1.418968in}}%
\pgfpathlineto{\pgfqpoint{2.845701in}{3.330824in}}%
\pgfpathlineto{\pgfqpoint{2.847966in}{3.382727in}}%
\pgfpathlineto{\pgfqpoint{2.848346in}{3.381256in}}%
\pgfpathlineto{\pgfqpoint{2.849439in}{3.360608in}}%
\pgfpathlineto{\pgfqpoint{2.851604in}{3.249427in}}%
\pgfpathlineto{\pgfqpoint{2.855220in}{2.876813in}}%
\pgfpathlineto{\pgfqpoint{2.862396in}{1.737454in}}%
\pgfpathlineto{\pgfqpoint{2.869661in}{0.787035in}}%
\pgfpathlineto{\pgfqpoint{2.873121in}{0.638452in}}%
\pgfpathlineto{\pgfqpoint{2.873456in}{0.637273in}}%
\pgfpathlineto{\pgfqpoint{2.874136in}{0.642225in}}%
\pgfpathlineto{\pgfqpoint{2.875598in}{0.685957in}}%
\pgfpathlineto{\pgfqpoint{2.878254in}{0.875886in}}%
\pgfpathlineto{\pgfqpoint{2.882696in}{1.454192in}}%
\pgfpathlineto{\pgfqpoint{2.896110in}{3.320736in}}%
\pgfpathlineto{\pgfqpoint{2.898499in}{3.382727in}}%
\pgfpathlineto{\pgfqpoint{2.898833in}{3.381519in}}%
\pgfpathlineto{\pgfqpoint{2.899860in}{3.362557in}}%
\pgfpathlineto{\pgfqpoint{2.901947in}{3.254658in}}%
\pgfpathlineto{\pgfqpoint{2.905463in}{2.884435in}}%
\pgfpathlineto{\pgfqpoint{2.912527in}{1.730634in}}%
\pgfpathlineto{\pgfqpoint{2.919625in}{0.782480in}}%
\pgfpathlineto{\pgfqpoint{2.923017in}{0.637999in}}%
\pgfpathlineto{\pgfqpoint{2.923285in}{0.637274in}}%
\pgfpathlineto{\pgfqpoint{2.923899in}{0.641628in}}%
\pgfpathlineto{\pgfqpoint{2.925305in}{0.682943in}}%
\pgfpathlineto{\pgfqpoint{2.927894in}{0.868105in}}%
\pgfpathlineto{\pgfqpoint{2.932247in}{1.440370in}}%
\pgfpathlineto{\pgfqpoint{2.945616in}{3.323781in}}%
\pgfpathlineto{\pgfqpoint{2.947926in}{3.382727in}}%
\pgfpathlineto{\pgfqpoint{2.948284in}{3.381277in}}%
\pgfpathlineto{\pgfqpoint{2.949355in}{3.359912in}}%
\pgfpathlineto{\pgfqpoint{2.951498in}{3.242623in}}%
\pgfpathlineto{\pgfqpoint{2.955113in}{2.845356in}}%
\pgfpathlineto{\pgfqpoint{2.963126in}{1.518424in}}%
\pgfpathlineto{\pgfqpoint{2.969376in}{0.750923in}}%
\pgfpathlineto{\pgfqpoint{2.972557in}{0.637296in}}%
\pgfpathlineto{\pgfqpoint{2.972657in}{0.637306in}}%
\pgfpathlineto{\pgfqpoint{2.972992in}{0.638952in}}%
\pgfpathlineto{\pgfqpoint{2.974108in}{0.662267in}}%
\pgfpathlineto{\pgfqpoint{2.976306in}{0.786129in}}%
\pgfpathlineto{\pgfqpoint{2.980011in}{1.199886in}}%
\pgfpathlineto{\pgfqpoint{2.989576in}{2.759521in}}%
\pgfpathlineto{\pgfqpoint{2.994944in}{3.314298in}}%
\pgfpathlineto{\pgfqpoint{2.997466in}{3.382727in}}%
\pgfpathlineto{\pgfqpoint{2.997745in}{3.381887in}}%
\pgfpathlineto{\pgfqpoint{2.998693in}{3.366464in}}%
\pgfpathlineto{\pgfqpoint{3.000669in}{3.273357in}}%
\pgfpathlineto{\pgfqpoint{3.004039in}{2.942486in}}%
\pgfpathlineto{\pgfqpoint{3.010378in}{1.948764in}}%
\pgfpathlineto{\pgfqpoint{3.018581in}{0.810184in}}%
\pgfpathlineto{\pgfqpoint{3.022185in}{0.639882in}}%
\pgfpathlineto{\pgfqpoint{3.022687in}{0.637273in}}%
\pgfpathlineto{\pgfqpoint{3.023424in}{0.642975in}}%
\pgfpathlineto{\pgfqpoint{3.024953in}{0.690586in}}%
\pgfpathlineto{\pgfqpoint{3.027698in}{0.890643in}}%
\pgfpathlineto{\pgfqpoint{3.032330in}{1.493648in}}%
\pgfpathlineto{\pgfqpoint{3.045432in}{3.292742in}}%
\pgfpathlineto{\pgfqpoint{3.048456in}{3.382727in}}%
\pgfpathlineto{\pgfqpoint{3.048501in}{3.382705in}}%
\pgfpathlineto{\pgfqpoint{3.048980in}{3.379970in}}%
\pgfpathlineto{\pgfqpoint{3.050275in}{3.349969in}}%
\pgfpathlineto{\pgfqpoint{3.052708in}{3.207699in}}%
\pgfpathlineto{\pgfqpoint{3.056759in}{2.758394in}}%
\pgfpathlineto{\pgfqpoint{3.073644in}{0.654517in}}%
\pgfpathlineto{\pgfqpoint{3.075006in}{0.637273in}}%
\pgfpathlineto{\pgfqpoint{3.075698in}{0.641662in}}%
\pgfpathlineto{\pgfqpoint{3.077171in}{0.680207in}}%
\pgfpathlineto{\pgfqpoint{3.079849in}{0.847151in}}%
\pgfpathlineto{\pgfqpoint{3.084268in}{1.348714in}}%
\pgfpathlineto{\pgfqpoint{3.100194in}{3.331599in}}%
\pgfpathlineto{\pgfqpoint{3.102660in}{3.382727in}}%
\pgfpathlineto{\pgfqpoint{3.102950in}{3.381997in}}%
\pgfpathlineto{\pgfqpoint{3.103921in}{3.369266in}}%
\pgfpathlineto{\pgfqpoint{3.105930in}{3.293740in}}%
\pgfpathlineto{\pgfqpoint{3.109311in}{3.029597in}}%
\pgfpathlineto{\pgfqpoint{3.115115in}{2.289752in}}%
\pgfpathlineto{\pgfqpoint{3.126665in}{0.833098in}}%
\pgfpathlineto{\pgfqpoint{3.130705in}{0.646494in}}%
\pgfpathlineto{\pgfqpoint{3.131821in}{0.637273in}}%
\pgfpathlineto{\pgfqpoint{3.132558in}{0.641391in}}%
\pgfpathlineto{\pgfqpoint{3.134087in}{0.675488in}}%
\pgfpathlineto{\pgfqpoint{3.136821in}{0.818370in}}%
\pgfpathlineto{\pgfqpoint{3.141251in}{1.238616in}}%
\pgfpathlineto{\pgfqpoint{3.161061in}{3.356127in}}%
\pgfpathlineto{\pgfqpoint{3.163103in}{3.382727in}}%
\pgfpathlineto{\pgfqpoint{3.163572in}{3.381335in}}%
\pgfpathlineto{\pgfqpoint{3.164799in}{3.364570in}}%
\pgfpathlineto{\pgfqpoint{3.167143in}{3.281458in}}%
\pgfpathlineto{\pgfqpoint{3.170937in}{3.018710in}}%
\pgfpathlineto{\pgfqpoint{3.177310in}{2.325624in}}%
\pgfpathlineto{\pgfqpoint{3.190568in}{0.882183in}}%
\pgfpathlineto{\pgfqpoint{3.195288in}{0.662911in}}%
\pgfpathlineto{\pgfqpoint{3.197531in}{0.637273in}}%
\pgfpathlineto{\pgfqpoint{3.197911in}{0.638016in}}%
\pgfpathlineto{\pgfqpoint{3.199027in}{0.648611in}}%
\pgfpathlineto{\pgfqpoint{3.201225in}{0.705255in}}%
\pgfpathlineto{\pgfqpoint{3.204797in}{0.890526in}}%
\pgfpathlineto{\pgfqpoint{3.210444in}{1.369087in}}%
\pgfpathlineto{\pgfqpoint{3.230811in}{3.241995in}}%
\pgfpathlineto{\pgfqpoint{3.235185in}{3.370869in}}%
\pgfpathlineto{\pgfqpoint{3.236993in}{3.382727in}}%
\pgfpathlineto{\pgfqpoint{3.237540in}{3.381586in}}%
\pgfpathlineto{\pgfqpoint{3.238891in}{3.369543in}}%
\pgfpathlineto{\pgfqpoint{3.241379in}{3.314107in}}%
\pgfpathlineto{\pgfqpoint{3.245308in}{3.146580in}}%
\pgfpathlineto{\pgfqpoint{3.251367in}{2.736576in}}%
\pgfpathlineto{\pgfqpoint{3.277493in}{0.793589in}}%
\pgfpathlineto{\pgfqpoint{3.282593in}{0.662065in}}%
\pgfpathlineto{\pgfqpoint{3.285830in}{0.637349in}}%
\pgfpathlineto{\pgfqpoint{3.286332in}{0.637472in}}%
\pgfpathlineto{\pgfqpoint{3.286800in}{0.638520in}}%
\pgfpathlineto{\pgfqpoint{3.288408in}{0.648825in}}%
\pgfpathlineto{\pgfqpoint{3.291220in}{0.690727in}}%
\pgfpathlineto{\pgfqpoint{3.295539in}{0.808349in}}%
\pgfpathlineto{\pgfqpoint{3.301978in}{1.079385in}}%
\pgfpathlineto{\pgfqpoint{3.313328in}{1.712090in}}%
\pgfpathlineto{\pgfqpoint{3.330972in}{2.663437in}}%
\pgfpathlineto{\pgfqpoint{3.340703in}{3.028425in}}%
\pgfpathlineto{\pgfqpoint{3.348750in}{3.225330in}}%
\pgfpathlineto{\pgfqpoint{3.355557in}{3.324610in}}%
\pgfpathlineto{\pgfqpoint{3.361182in}{3.367635in}}%
\pgfpathlineto{\pgfqpoint{3.365501in}{3.381199in}}%
\pgfpathlineto{\pgfqpoint{3.368313in}{3.382583in}}%
\pgfpathlineto{\pgfqpoint{3.370702in}{3.379895in}}%
\pgfpathlineto{\pgfqpoint{3.374317in}{3.370240in}}%
\pgfpathlineto{\pgfqpoint{3.380054in}{3.344735in}}%
\pgfpathlineto{\pgfqpoint{3.392129in}{3.273122in}}%
\pgfpathlineto{\pgfqpoint{3.403914in}{3.211511in}}%
\pgfpathlineto{\pgfqpoint{3.411012in}{3.188916in}}%
\pgfpathlineto{\pgfqpoint{3.416034in}{3.181570in}}%
\pgfpathlineto{\pgfqpoint{3.419549in}{3.180979in}}%
\pgfpathlineto{\pgfqpoint{3.422841in}{3.183852in}}%
\pgfpathlineto{\pgfqpoint{3.427060in}{3.192261in}}%
\pgfpathlineto{\pgfqpoint{3.432718in}{3.211296in}}%
\pgfpathlineto{\pgfqpoint{3.440597in}{3.249967in}}%
\pgfpathlineto{\pgfqpoint{3.464959in}{3.377685in}}%
\pgfpathlineto{\pgfqpoint{3.468787in}{3.382696in}}%
\pgfpathlineto{\pgfqpoint{3.470974in}{3.381565in}}%
\pgfpathlineto{\pgfqpoint{3.473563in}{3.375807in}}%
\pgfpathlineto{\pgfqpoint{3.477057in}{3.359333in}}%
\pgfpathlineto{\pgfqpoint{3.481498in}{3.321582in}}%
\pgfpathlineto{\pgfqpoint{3.486866in}{3.246572in}}%
\pgfpathlineto{\pgfqpoint{3.493172in}{3.111024in}}%
\pgfpathlineto{\pgfqpoint{3.500571in}{2.879492in}}%
\pgfpathlineto{\pgfqpoint{3.509510in}{2.494166in}}%
\pgfpathlineto{\pgfqpoint{3.522232in}{1.793750in}}%
\pgfpathlineto{\pgfqpoint{3.538894in}{0.906228in}}%
\pgfpathlineto{\pgfqpoint{3.545613in}{0.695933in}}%
\pgfpathlineto{\pgfqpoint{3.549820in}{0.640601in}}%
\pgfpathlineto{\pgfqpoint{3.551148in}{0.637277in}}%
\pgfpathlineto{\pgfqpoint{3.551829in}{0.638345in}}%
\pgfpathlineto{\pgfqpoint{3.553358in}{0.647754in}}%
\pgfpathlineto{\pgfqpoint{3.555969in}{0.686772in}}%
\pgfpathlineto{\pgfqpoint{3.559797in}{0.797207in}}%
\pgfpathlineto{\pgfqpoint{3.565053in}{1.049626in}}%
\pgfpathlineto{\pgfqpoint{3.572419in}{1.570760in}}%
\pgfpathlineto{\pgfqpoint{3.594962in}{3.273265in}}%
\pgfpathlineto{\pgfqpoint{3.599225in}{3.375907in}}%
\pgfpathlineto{\pgfqpoint{3.600631in}{3.382726in}}%
\pgfpathlineto{\pgfqpoint{3.601301in}{3.381032in}}%
\pgfpathlineto{\pgfqpoint{3.602774in}{3.365927in}}%
\pgfpathlineto{\pgfqpoint{3.605352in}{3.301636in}}%
\pgfpathlineto{\pgfqpoint{3.609247in}{3.115249in}}%
\pgfpathlineto{\pgfqpoint{3.614916in}{2.675128in}}%
\pgfpathlineto{\pgfqpoint{3.627092in}{1.410018in}}%
\pgfpathlineto{\pgfqpoint{3.634770in}{0.799270in}}%
\pgfpathlineto{\pgfqpoint{3.639122in}{0.648339in}}%
\pgfpathlineto{\pgfqpoint{3.640629in}{0.637273in}}%
\pgfpathlineto{\pgfqpoint{3.641276in}{0.639346in}}%
\pgfpathlineto{\pgfqpoint{3.642705in}{0.658583in}}%
\pgfpathlineto{\pgfqpoint{3.645249in}{0.742584in}}%
\pgfpathlineto{\pgfqpoint{3.649178in}{0.990906in}}%
\pgfpathlineto{\pgfqpoint{3.655260in}{1.601980in}}%
\pgfpathlineto{\pgfqpoint{3.670940in}{3.247429in}}%
\pgfpathlineto{\pgfqpoint{3.674846in}{3.378020in}}%
\pgfpathlineto{\pgfqpoint{3.675738in}{3.382726in}}%
\pgfpathlineto{\pgfqpoint{3.676497in}{3.379111in}}%
\pgfpathlineto{\pgfqpoint{3.678049in}{3.349878in}}%
\pgfpathlineto{\pgfqpoint{3.680760in}{3.229551in}}%
\pgfpathlineto{\pgfqpoint{3.684979in}{2.883232in}}%
\pgfpathlineto{\pgfqpoint{3.692333in}{1.974902in}}%
\pgfpathlineto{\pgfqpoint{3.702243in}{0.848721in}}%
\pgfpathlineto{\pgfqpoint{3.706495in}{0.649626in}}%
\pgfpathlineto{\pgfqpoint{3.707823in}{0.637273in}}%
\pgfpathlineto{\pgfqpoint{3.708515in}{0.640680in}}%
\pgfpathlineto{\pgfqpoint{3.709989in}{0.670449in}}%
\pgfpathlineto{\pgfqpoint{3.712622in}{0.798327in}}%
\pgfpathlineto{\pgfqpoint{3.716785in}{1.175297in}}%
\pgfpathlineto{\pgfqpoint{3.724642in}{2.232810in}}%
\pgfpathlineto{\pgfqpoint{3.732956in}{3.193446in}}%
\pgfpathlineto{\pgfqpoint{3.736962in}{3.375192in}}%
\pgfpathlineto{\pgfqpoint{3.737944in}{3.382727in}}%
\pgfpathlineto{\pgfqpoint{3.738703in}{3.378188in}}%
\pgfpathlineto{\pgfqpoint{3.740243in}{3.341250in}}%
\pgfpathlineto{\pgfqpoint{3.742966in}{3.187907in}}%
\pgfpathlineto{\pgfqpoint{3.747330in}{2.739044in}}%
\pgfpathlineto{\pgfqpoint{3.765409in}{0.652284in}}%
\pgfpathlineto{\pgfqpoint{3.766748in}{0.637273in}}%
\pgfpathlineto{\pgfqpoint{3.767440in}{0.641377in}}%
\pgfpathlineto{\pgfqpoint{3.768925in}{0.677386in}}%
\pgfpathlineto{\pgfqpoint{3.771581in}{0.831355in}}%
\pgfpathlineto{\pgfqpoint{3.775877in}{1.289833in}}%
\pgfpathlineto{\pgfqpoint{3.793265in}{3.364942in}}%
\pgfpathlineto{\pgfqpoint{3.794682in}{3.382727in}}%
\pgfpathlineto{\pgfqpoint{3.795352in}{3.378759in}}%
\pgfpathlineto{\pgfqpoint{3.796802in}{3.343009in}}%
\pgfpathlineto{\pgfqpoint{3.799425in}{3.187440in}}%
\pgfpathlineto{\pgfqpoint{3.803699in}{2.719178in}}%
\pgfpathlineto{\pgfqpoint{3.820573in}{0.658937in}}%
\pgfpathlineto{\pgfqpoint{3.822124in}{0.637273in}}%
\pgfpathlineto{\pgfqpoint{3.822761in}{0.641018in}}%
\pgfpathlineto{\pgfqpoint{3.824178in}{0.675677in}}%
\pgfpathlineto{\pgfqpoint{3.826767in}{0.829361in}}%
\pgfpathlineto{\pgfqpoint{3.831008in}{1.296586in}}%
\pgfpathlineto{\pgfqpoint{3.847804in}{3.360668in}}%
\pgfpathlineto{\pgfqpoint{3.849366in}{3.382727in}}%
\pgfpathlineto{\pgfqpoint{3.849991in}{3.379095in}}%
\pgfpathlineto{\pgfqpoint{3.851397in}{3.345030in}}%
\pgfpathlineto{\pgfqpoint{3.853975in}{3.192972in}}%
\pgfpathlineto{\pgfqpoint{3.858205in}{2.729308in}}%
\pgfpathlineto{\pgfqpoint{3.875112in}{0.659776in}}%
\pgfpathlineto{\pgfqpoint{3.876697in}{0.637273in}}%
\pgfpathlineto{\pgfqpoint{3.877322in}{0.640762in}}%
\pgfpathlineto{\pgfqpoint{3.878728in}{0.674029in}}%
\pgfpathlineto{\pgfqpoint{3.881306in}{0.822880in}}%
\pgfpathlineto{\pgfqpoint{3.885525in}{1.275705in}}%
\pgfpathlineto{\pgfqpoint{3.902878in}{3.361857in}}%
\pgfpathlineto{\pgfqpoint{3.904441in}{3.382727in}}%
\pgfpathlineto{\pgfqpoint{3.905066in}{3.379317in}}%
\pgfpathlineto{\pgfqpoint{3.906472in}{3.347266in}}%
\pgfpathlineto{\pgfqpoint{3.909050in}{3.204268in}}%
\pgfpathlineto{\pgfqpoint{3.913268in}{2.769019in}}%
\pgfpathlineto{\pgfqpoint{3.931448in}{0.654836in}}%
\pgfpathlineto{\pgfqpoint{3.932932in}{0.637273in}}%
\pgfpathlineto{\pgfqpoint{3.933580in}{0.640722in}}%
\pgfpathlineto{\pgfqpoint{3.935019in}{0.672351in}}%
\pgfpathlineto{\pgfqpoint{3.937631in}{0.810667in}}%
\pgfpathlineto{\pgfqpoint{3.941883in}{1.227320in}}%
\pgfpathlineto{\pgfqpoint{3.961167in}{3.367553in}}%
\pgfpathlineto{\pgfqpoint{3.962607in}{3.382727in}}%
\pgfpathlineto{\pgfqpoint{3.963276in}{3.379478in}}%
\pgfpathlineto{\pgfqpoint{3.964727in}{3.350236in}}%
\pgfpathlineto{\pgfqpoint{3.967361in}{3.222657in}}%
\pgfpathlineto{\pgfqpoint{3.971602in}{2.842667in}}%
\pgfpathlineto{\pgfqpoint{3.980251in}{1.715963in}}%
\pgfpathlineto{\pgfqpoint{3.988297in}{0.849101in}}%
\pgfpathlineto{\pgfqpoint{3.992605in}{0.651726in}}%
\pgfpathlineto{\pgfqpoint{3.994123in}{0.637273in}}%
\pgfpathlineto{\pgfqpoint{3.994759in}{0.639903in}}%
\pgfpathlineto{\pgfqpoint{3.996187in}{0.664273in}}%
\pgfpathlineto{\pgfqpoint{3.998788in}{0.771720in}}%
\pgfpathlineto{\pgfqpoint{4.002950in}{1.092829in}}%
\pgfpathlineto{\pgfqpoint{4.010394in}{1.954293in}}%
\pgfpathlineto{\pgfqpoint{4.021085in}{3.106782in}}%
\pgfpathlineto{\pgfqpoint{4.025951in}{3.350767in}}%
\pgfpathlineto{\pgfqpoint{4.028440in}{3.382727in}}%
\pgfpathlineto{\pgfqpoint{4.028719in}{3.382308in}}%
\pgfpathlineto{\pgfqpoint{4.029690in}{3.374626in}}%
\pgfpathlineto{\pgfqpoint{4.031687in}{3.329040in}}%
\pgfpathlineto{\pgfqpoint{4.035013in}{3.170047in}}%
\pgfpathlineto{\pgfqpoint{4.040225in}{2.749871in}}%
\pgfpathlineto{\pgfqpoint{4.062947in}{0.713374in}}%
\pgfpathlineto{\pgfqpoint{4.066652in}{0.639464in}}%
\pgfpathlineto{\pgfqpoint{4.067433in}{0.637274in}}%
\pgfpathlineto{\pgfqpoint{4.068192in}{0.639554in}}%
\pgfpathlineto{\pgfqpoint{4.069776in}{0.658067in}}%
\pgfpathlineto{\pgfqpoint{4.072566in}{0.733897in}}%
\pgfpathlineto{\pgfqpoint{4.076897in}{0.948905in}}%
\pgfpathlineto{\pgfqpoint{4.083749in}{1.466514in}}%
\pgfpathlineto{\pgfqpoint{4.104205in}{3.094649in}}%
\pgfpathlineto{\pgfqpoint{4.110142in}{3.318360in}}%
\pgfpathlineto{\pgfqpoint{4.114182in}{3.378794in}}%
\pgfpathlineto{\pgfqpoint{4.115566in}{3.382724in}}%
\pgfpathlineto{\pgfqpoint{4.116236in}{3.381654in}}%
\pgfpathlineto{\pgfqpoint{4.117776in}{3.372036in}}%
\pgfpathlineto{\pgfqpoint{4.120499in}{3.331738in}}%
\pgfpathlineto{\pgfqpoint{4.124717in}{3.216318in}}%
\pgfpathlineto{\pgfqpoint{4.131034in}{2.947209in}}%
\pgfpathlineto{\pgfqpoint{4.142205in}{2.314255in}}%
\pgfpathlineto{\pgfqpoint{4.159570in}{1.363567in}}%
\pgfpathlineto{\pgfqpoint{4.169212in}{0.997903in}}%
\pgfpathlineto{\pgfqpoint{4.177214in}{0.800464in}}%
\pgfpathlineto{\pgfqpoint{4.184055in}{0.699729in}}%
\pgfpathlineto{\pgfqpoint{4.189791in}{0.654896in}}%
\pgfpathlineto{\pgfqpoint{4.194322in}{0.639645in}}%
\pgfpathlineto{\pgfqpoint{4.197469in}{0.637298in}}%
\pgfpathlineto{\pgfqpoint{4.199958in}{0.639282in}}%
\pgfpathlineto{\pgfqpoint{4.203596in}{0.646982in}}%
\pgfpathlineto{\pgfqpoint{4.209634in}{0.668232in}}%
\pgfpathlineto{\pgfqpoint{4.230224in}{0.745860in}}%
\pgfpathlineto{\pgfqpoint{4.235648in}{0.754051in}}%
\pgfpathlineto{\pgfqpoint{4.239453in}{0.755114in}}%
\pgfpathlineto{\pgfqpoint{4.242868in}{0.752696in}}%
\pgfpathlineto{\pgfqpoint{4.247143in}{0.745319in}}%
\pgfpathlineto{\pgfqpoint{4.252979in}{0.728286in}}%
\pgfpathlineto{\pgfqpoint{4.262332in}{0.689807in}}%
\pgfpathlineto{\pgfqpoint{4.274027in}{0.644838in}}%
\pgfpathlineto{\pgfqpoint{4.278547in}{0.637546in}}%
\pgfpathlineto{\pgfqpoint{4.281036in}{0.637945in}}%
\pgfpathlineto{\pgfqpoint{4.283524in}{0.642309in}}%
\pgfpathlineto{\pgfqpoint{4.286828in}{0.655398in}}%
\pgfpathlineto{\pgfqpoint{4.291002in}{0.686224in}}%
\pgfpathlineto{\pgfqpoint{4.296024in}{0.748905in}}%
\pgfpathlineto{\pgfqpoint{4.301905in}{0.864904in}}%
\pgfpathlineto{\pgfqpoint{4.308724in}{1.066028in}}%
\pgfpathlineto{\pgfqpoint{4.316792in}{1.403476in}}%
\pgfpathlineto{\pgfqpoint{4.327283in}{1.985814in}}%
\pgfpathlineto{\pgfqpoint{4.348710in}{3.203007in}}%
\pgfpathlineto{\pgfqpoint{4.354312in}{3.355495in}}%
\pgfpathlineto{\pgfqpoint{4.357571in}{3.382690in}}%
\pgfpathlineto{\pgfqpoint{4.358040in}{3.382434in}}%
\pgfpathlineto{\pgfqpoint{4.358419in}{3.381429in}}%
\pgfpathlineto{\pgfqpoint{4.359937in}{3.370160in}}%
\pgfpathlineto{\pgfqpoint{4.362537in}{3.323202in}}%
\pgfpathlineto{\pgfqpoint{4.366354in}{3.189720in}}%
\pgfpathlineto{\pgfqpoint{4.371622in}{2.883243in}}%
\pgfpathlineto{\pgfqpoint{4.379333in}{2.229213in}}%
\pgfpathlineto{\pgfqpoint{4.395683in}{0.814378in}}%
\pgfpathlineto{\pgfqpoint{4.400292in}{0.652087in}}%
\pgfpathlineto{\pgfqpoint{4.402111in}{0.637273in}}%
\pgfpathlineto{\pgfqpoint{4.402658in}{0.638666in}}%
\pgfpathlineto{\pgfqpoint{4.403975in}{0.653336in}}%
\pgfpathlineto{\pgfqpoint{4.406374in}{0.721525in}}%
\pgfpathlineto{\pgfqpoint{4.410068in}{0.928884in}}%
\pgfpathlineto{\pgfqpoint{4.415570in}{1.437038in}}%
\pgfpathlineto{\pgfqpoint{4.434430in}{3.335948in}}%
\pgfpathlineto{\pgfqpoint{4.437131in}{3.382727in}}%
\pgfpathlineto{\pgfqpoint{4.437332in}{3.382445in}}%
\pgfpathlineto{\pgfqpoint{4.438169in}{3.375540in}}%
\pgfpathlineto{\pgfqpoint{4.439966in}{3.329397in}}%
\pgfpathlineto{\pgfqpoint{4.442979in}{3.158072in}}%
\pgfpathlineto{\pgfqpoint{4.447644in}{2.688806in}}%
\pgfpathlineto{\pgfqpoint{4.465823in}{0.653106in}}%
\pgfpathlineto{\pgfqpoint{4.467196in}{0.637273in}}%
\pgfpathlineto{\pgfqpoint{4.467877in}{0.641198in}}%
\pgfpathlineto{\pgfqpoint{4.469339in}{0.676237in}}%
\pgfpathlineto{\pgfqpoint{4.471950in}{0.827396in}}%
\pgfpathlineto{\pgfqpoint{4.476135in}{1.279122in}}%
\pgfpathlineto{\pgfqpoint{4.493400in}{3.377303in}}%
\pgfpathlineto{\pgfqpoint{4.494136in}{3.382727in}}%
\pgfpathlineto{\pgfqpoint{4.494895in}{3.376698in}}%
\pgfpathlineto{\pgfqpoint{4.496446in}{3.327661in}}%
\pgfpathlineto{\pgfqpoint{4.499192in}{3.124368in}}%
\pgfpathlineto{\pgfqpoint{4.503734in}{2.518164in}}%
\pgfpathlineto{\pgfqpoint{4.516613in}{0.697062in}}%
\pgfpathlineto{\pgfqpoint{4.518867in}{0.637273in}}%
\pgfpathlineto{\pgfqpoint{4.519258in}{0.639070in}}%
\pgfpathlineto{\pgfqpoint{4.520362in}{0.663841in}}%
\pgfpathlineto{\pgfqpoint{4.522539in}{0.796053in}}%
\pgfpathlineto{\pgfqpoint{4.526199in}{1.239905in}}%
\pgfpathlineto{\pgfqpoint{4.541756in}{3.381995in}}%
\pgfpathlineto{\pgfqpoint{4.541991in}{3.382727in}}%
\pgfpathlineto{\pgfqpoint{4.542604in}{3.377643in}}%
\pgfpathlineto{\pgfqpoint{4.543988in}{3.329121in}}%
\pgfpathlineto{\pgfqpoint{4.546555in}{3.109016in}}%
\pgfpathlineto{\pgfqpoint{4.550985in}{2.413151in}}%
\pgfpathlineto{\pgfqpoint{4.561576in}{0.714150in}}%
\pgfpathlineto{\pgfqpoint{4.563864in}{0.637273in}}%
\pgfpathlineto{\pgfqpoint{4.564244in}{0.639378in}}%
\pgfpathlineto{\pgfqpoint{4.565337in}{0.669355in}}%
\pgfpathlineto{\pgfqpoint{4.567513in}{0.831582in}}%
\pgfpathlineto{\pgfqpoint{4.571274in}{1.385328in}}%
\pgfpathlineto{\pgfqpoint{4.583629in}{3.361862in}}%
\pgfpathlineto{\pgfqpoint{4.584767in}{3.382727in}}%
\pgfpathlineto{\pgfqpoint{4.585503in}{3.373989in}}%
\pgfpathlineto{\pgfqpoint{4.587021in}{3.301219in}}%
\pgfpathlineto{\pgfqpoint{4.589800in}{2.990311in}}%
\pgfpathlineto{\pgfqpoint{4.595079in}{1.974938in}}%
\pgfpathlineto{\pgfqpoint{4.602344in}{0.747340in}}%
\pgfpathlineto{\pgfqpoint{4.604888in}{0.637273in}}%
\pgfpathlineto{\pgfqpoint{4.605156in}{0.638518in}}%
\pgfpathlineto{\pgfqpoint{4.606083in}{0.661942in}}%
\pgfpathlineto{\pgfqpoint{4.608036in}{0.806028in}}%
\pgfpathlineto{\pgfqpoint{4.611495in}{1.332561in}}%
\pgfpathlineto{\pgfqpoint{4.623682in}{3.373831in}}%
\pgfpathlineto{\pgfqpoint{4.624385in}{3.382727in}}%
\pgfpathlineto{\pgfqpoint{4.625144in}{3.372003in}}%
\pgfpathlineto{\pgfqpoint{4.626695in}{3.285216in}}%
\pgfpathlineto{\pgfqpoint{4.629541in}{2.919344in}}%
\pgfpathlineto{\pgfqpoint{4.635657in}{1.621985in}}%
\pgfpathlineto{\pgfqpoint{4.641359in}{0.713525in}}%
\pgfpathlineto{\pgfqpoint{4.643357in}{0.637273in}}%
\pgfpathlineto{\pgfqpoint{4.643848in}{0.641824in}}%
\pgfpathlineto{\pgfqpoint{4.645087in}{0.694252in}}%
\pgfpathlineto{\pgfqpoint{4.647486in}{0.953234in}}%
\pgfpathlineto{\pgfqpoint{4.651961in}{1.840357in}}%
\pgfpathlineto{\pgfqpoint{4.659807in}{3.293252in}}%
\pgfpathlineto{\pgfqpoint{4.661938in}{3.382727in}}%
\pgfpathlineto{\pgfqpoint{4.662374in}{3.378878in}}%
\pgfpathlineto{\pgfqpoint{4.663545in}{3.331075in}}%
\pgfpathlineto{\pgfqpoint{4.665867in}{3.083744in}}%
\pgfpathlineto{\pgfqpoint{4.670186in}{2.223983in}}%
\pgfpathlineto{\pgfqpoint{4.678176in}{0.719494in}}%
\pgfpathlineto{\pgfqpoint{4.680185in}{0.637273in}}%
\pgfpathlineto{\pgfqpoint{4.680665in}{0.642066in}}%
\pgfpathlineto{\pgfqpoint{4.681904in}{0.698042in}}%
\pgfpathlineto{\pgfqpoint{4.684314in}{0.975937in}}%
\pgfpathlineto{\pgfqpoint{4.688935in}{1.944437in}}%
\pgfpathlineto{\pgfqpoint{4.696110in}{3.293314in}}%
\pgfpathlineto{\pgfqpoint{4.698175in}{3.382727in}}%
\pgfpathlineto{\pgfqpoint{4.698644in}{3.378147in}}%
\pgfpathlineto{\pgfqpoint{4.699860in}{3.323251in}}%
\pgfpathlineto{\pgfqpoint{4.702237in}{3.047857in}}%
\pgfpathlineto{\pgfqpoint{4.706813in}{2.079960in}}%
\pgfpathlineto{\pgfqpoint{4.713955in}{0.725065in}}%
\pgfpathlineto{\pgfqpoint{4.715986in}{0.637273in}}%
\pgfpathlineto{\pgfqpoint{4.716466in}{0.642242in}}%
\pgfpathlineto{\pgfqpoint{4.717694in}{0.699679in}}%
\pgfpathlineto{\pgfqpoint{4.720104in}{0.987282in}}%
\pgfpathlineto{\pgfqpoint{4.724814in}{2.004197in}}%
\pgfpathlineto{\pgfqpoint{4.731644in}{3.294770in}}%
\pgfpathlineto{\pgfqpoint{4.733664in}{3.382727in}}%
\pgfpathlineto{\pgfqpoint{4.734144in}{3.377727in}}%
\pgfpathlineto{\pgfqpoint{4.735371in}{3.319720in}}%
\pgfpathlineto{\pgfqpoint{4.737782in}{3.029337in}}%
\pgfpathlineto{\pgfqpoint{4.742514in}{2.000026in}}%
\pgfpathlineto{\pgfqpoint{4.749254in}{0.725282in}}%
\pgfpathlineto{\pgfqpoint{4.751274in}{0.637273in}}%
\pgfpathlineto{\pgfqpoint{4.751754in}{0.642406in}}%
\pgfpathlineto{\pgfqpoint{4.752993in}{0.701790in}}%
\pgfpathlineto{\pgfqpoint{4.755415in}{0.996718in}}%
\pgfpathlineto{\pgfqpoint{4.760214in}{2.047161in}}%
\pgfpathlineto{\pgfqpoint{4.766843in}{3.294969in}}%
\pgfpathlineto{\pgfqpoint{4.768851in}{3.382727in}}%
\pgfpathlineto{\pgfqpoint{4.769342in}{3.377546in}}%
\pgfpathlineto{\pgfqpoint{4.770581in}{3.318092in}}%
\pgfpathlineto{\pgfqpoint{4.773003in}{3.023369in}}%
\pgfpathlineto{\pgfqpoint{4.777802in}{1.974723in}}%
\pgfpathlineto{\pgfqpoint{4.784442in}{0.726145in}}%
\pgfpathlineto{\pgfqpoint{4.786473in}{0.637273in}}%
\pgfpathlineto{\pgfqpoint{4.786953in}{0.642247in}}%
\pgfpathlineto{\pgfqpoint{4.788181in}{0.700034in}}%
\pgfpathlineto{\pgfqpoint{4.790591in}{0.989080in}}%
\pgfpathlineto{\pgfqpoint{4.795334in}{2.015180in}}%
\pgfpathlineto{\pgfqpoint{4.802119in}{3.292549in}}%
\pgfpathlineto{\pgfqpoint{4.804184in}{3.382727in}}%
\pgfpathlineto{\pgfqpoint{4.804642in}{3.378154in}}%
\pgfpathlineto{\pgfqpoint{4.805847in}{3.323563in}}%
\pgfpathlineto{\pgfqpoint{4.808224in}{3.047054in}}%
\pgfpathlineto{\pgfqpoint{4.812833in}{2.070214in}}%
\pgfpathlineto{\pgfqpoint{4.819931in}{0.728906in}}%
\pgfpathlineto{\pgfqpoint{4.822029in}{0.637273in}}%
\pgfpathlineto{\pgfqpoint{4.822475in}{0.641489in}}%
\pgfpathlineto{\pgfqpoint{4.823669in}{0.693572in}}%
\pgfpathlineto{\pgfqpoint{4.826024in}{0.959231in}}%
\pgfpathlineto{\pgfqpoint{4.830544in}{1.895576in}}%
\pgfpathlineto{\pgfqpoint{4.837932in}{3.289220in}}%
\pgfpathlineto{\pgfqpoint{4.840075in}{3.382727in}}%
\pgfpathlineto{\pgfqpoint{4.840510in}{3.378893in}}%
\pgfpathlineto{\pgfqpoint{4.841682in}{3.330335in}}%
\pgfpathlineto{\pgfqpoint{4.844003in}{3.079494in}}%
\pgfpathlineto{\pgfqpoint{4.848389in}{2.200036in}}%
\pgfpathlineto{\pgfqpoint{4.856223in}{0.729991in}}%
\pgfpathlineto{\pgfqpoint{4.858399in}{0.637273in}}%
\pgfpathlineto{\pgfqpoint{4.858812in}{0.640733in}}%
\pgfpathlineto{\pgfqpoint{4.859962in}{0.685658in}}%
\pgfpathlineto{\pgfqpoint{4.862250in}{0.920751in}}%
\pgfpathlineto{\pgfqpoint{4.866513in}{1.742620in}}%
\pgfpathlineto{\pgfqpoint{4.874883in}{3.293761in}}%
\pgfpathlineto{\pgfqpoint{4.877048in}{3.382727in}}%
\pgfpathlineto{\pgfqpoint{4.877472in}{3.379382in}}%
\pgfpathlineto{\pgfqpoint{4.878621in}{3.336194in}}%
\pgfpathlineto{\pgfqpoint{4.880909in}{3.110198in}}%
\pgfpathlineto{\pgfqpoint{4.885128in}{2.325648in}}%
\pgfpathlineto{\pgfqpoint{4.893933in}{0.724640in}}%
\pgfpathlineto{\pgfqpoint{4.896132in}{0.637273in}}%
\pgfpathlineto{\pgfqpoint{4.896544in}{0.640288in}}%
\pgfpathlineto{\pgfqpoint{4.897683in}{0.680308in}}%
\pgfpathlineto{\pgfqpoint{4.899948in}{0.891085in}}%
\pgfpathlineto{\pgfqpoint{4.904077in}{1.620884in}}%
\pgfpathlineto{\pgfqpoint{4.913563in}{3.301842in}}%
\pgfpathlineto{\pgfqpoint{4.915751in}{3.382727in}}%
\pgfpathlineto{\pgfqpoint{4.916164in}{3.379751in}}%
\pgfpathlineto{\pgfqpoint{4.917313in}{3.341087in}}%
\pgfpathlineto{\pgfqpoint{4.919590in}{3.139423in}}%
\pgfpathlineto{\pgfqpoint{4.923697in}{2.447794in}}%
\pgfpathlineto{\pgfqpoint{4.933785in}{0.715198in}}%
\pgfpathlineto{\pgfqpoint{4.936006in}{0.637273in}}%
\pgfpathlineto{\pgfqpoint{4.936408in}{0.639868in}}%
\pgfpathlineto{\pgfqpoint{4.937535in}{0.674313in}}%
\pgfpathlineto{\pgfqpoint{4.939790in}{0.857576in}}%
\pgfpathlineto{\pgfqpoint{4.943785in}{1.482195in}}%
\pgfpathlineto{\pgfqpoint{4.954989in}{3.320030in}}%
\pgfpathlineto{\pgfqpoint{4.957065in}{3.382727in}}%
\pgfpathlineto{\pgfqpoint{4.957523in}{3.379643in}}%
\pgfpathlineto{\pgfqpoint{4.958728in}{3.342534in}}%
\pgfpathlineto{\pgfqpoint{4.961083in}{3.155056in}}%
\pgfpathlineto{\pgfqpoint{4.965223in}{2.532142in}}%
\pgfpathlineto{\pgfqpoint{4.976774in}{0.709937in}}%
\pgfpathlineto{\pgfqpoint{4.979129in}{0.637273in}}%
\pgfpathlineto{\pgfqpoint{4.979475in}{0.638909in}}%
\pgfpathlineto{\pgfqpoint{4.980535in}{0.663456in}}%
\pgfpathlineto{\pgfqpoint{4.982677in}{0.799522in}}%
\pgfpathlineto{\pgfqpoint{4.986394in}{1.268148in}}%
\pgfpathlineto{\pgfqpoint{5.001259in}{3.366139in}}%
\pgfpathlineto{\pgfqpoint{5.002453in}{3.382727in}}%
\pgfpathlineto{\pgfqpoint{5.003167in}{3.376722in}}%
\pgfpathlineto{\pgfqpoint{5.004674in}{3.325660in}}%
\pgfpathlineto{\pgfqpoint{5.007419in}{3.107807in}}%
\pgfpathlineto{\pgfqpoint{5.012196in}{2.437386in}}%
\pgfpathlineto{\pgfqpoint{5.023836in}{0.766056in}}%
\pgfpathlineto{\pgfqpoint{5.027184in}{0.637852in}}%
\pgfpathlineto{\pgfqpoint{5.027429in}{0.637273in}}%
\pgfpathlineto{\pgfqpoint{5.028043in}{0.641118in}}%
\pgfpathlineto{\pgfqpoint{5.029449in}{0.678021in}}%
\pgfpathlineto{\pgfqpoint{5.032049in}{0.843648in}}%
\pgfpathlineto{\pgfqpoint{5.036424in}{1.354454in}}%
\pgfpathlineto{\pgfqpoint{5.051825in}{3.317883in}}%
\pgfpathlineto{\pgfqpoint{5.054615in}{3.382727in}}%
\pgfpathlineto{\pgfqpoint{5.054760in}{3.382557in}}%
\pgfpathlineto{\pgfqpoint{5.055463in}{3.376804in}}%
\pgfpathlineto{\pgfqpoint{5.057104in}{3.332194in}}%
\pgfpathlineto{\pgfqpoint{5.060005in}{3.153364in}}%
\pgfpathlineto{\pgfqpoint{5.064815in}{2.633195in}}%
\pgfpathlineto{\pgfqpoint{5.081008in}{0.739017in}}%
\pgfpathlineto{\pgfqpoint{5.084513in}{0.638624in}}%
\pgfpathlineto{\pgfqpoint{5.084981in}{0.637274in}}%
\pgfpathlineto{\pgfqpoint{5.085707in}{0.640760in}}%
\pgfpathlineto{\pgfqpoint{5.087236in}{0.669906in}}%
\pgfpathlineto{\pgfqpoint{5.089981in}{0.792508in}}%
\pgfpathlineto{\pgfqpoint{5.094400in}{1.150701in}}%
\pgfpathlineto{\pgfqpoint{5.103217in}{2.179650in}}%
\pgfpathlineto{\pgfqpoint{5.112368in}{3.100036in}}%
\pgfpathlineto{\pgfqpoint{5.117379in}{3.345007in}}%
\pgfpathlineto{\pgfqpoint{5.120258in}{3.382727in}}%
\pgfpathlineto{\pgfqpoint{5.120348in}{3.382689in}}%
\pgfpathlineto{\pgfqpoint{5.121028in}{3.380050in}}%
\pgfpathlineto{\pgfqpoint{5.122602in}{3.358326in}}%
\pgfpathlineto{\pgfqpoint{5.125392in}{3.268919in}}%
\pgfpathlineto{\pgfqpoint{5.129789in}{3.013074in}}%
\pgfpathlineto{\pgfqpoint{5.137143in}{2.374170in}}%
\pgfpathlineto{\pgfqpoint{5.152812in}{1.005713in}}%
\pgfpathlineto{\pgfqpoint{5.158939in}{0.723206in}}%
\pgfpathlineto{\pgfqpoint{5.163124in}{0.643662in}}%
\pgfpathlineto{\pgfqpoint{5.164742in}{0.637274in}}%
\pgfpathlineto{\pgfqpoint{5.165345in}{0.638245in}}%
\pgfpathlineto{\pgfqpoint{5.166796in}{0.647817in}}%
\pgfpathlineto{\pgfqpoint{5.169418in}{0.689839in}}%
\pgfpathlineto{\pgfqpoint{5.173525in}{0.813030in}}%
\pgfpathlineto{\pgfqpoint{5.179775in}{1.106281in}}%
\pgfpathlineto{\pgfqpoint{5.191604in}{1.838825in}}%
\pgfpathlineto{\pgfqpoint{5.206536in}{2.702283in}}%
\pgfpathlineto{\pgfqpoint{5.215688in}{3.064840in}}%
\pgfpathlineto{\pgfqpoint{5.223198in}{3.252617in}}%
\pgfpathlineto{\pgfqpoint{5.229459in}{3.341232in}}%
\pgfpathlineto{\pgfqpoint{5.234459in}{3.374983in}}%
\pgfpathlineto{\pgfqpoint{5.237996in}{3.382588in}}%
\pgfpathlineto{\pgfqpoint{5.240005in}{3.381868in}}%
\pgfpathlineto{\pgfqpoint{5.242539in}{3.376508in}}%
\pgfpathlineto{\pgfqpoint{5.246579in}{3.359464in}}%
\pgfpathlineto{\pgfqpoint{5.253096in}{3.316494in}}%
\pgfpathlineto{\pgfqpoint{5.280851in}{3.117455in}}%
\pgfpathlineto{\pgfqpoint{5.286911in}{3.098986in}}%
\pgfpathlineto{\pgfqpoint{5.291107in}{3.094226in}}%
\pgfpathlineto{\pgfqpoint{5.294020in}{3.094882in}}%
\pgfpathlineto{\pgfqpoint{5.297234in}{3.099351in}}%
\pgfpathlineto{\pgfqpoint{5.301575in}{3.111443in}}%
\pgfpathlineto{\pgfqpoint{5.307434in}{3.138002in}}%
\pgfpathlineto{\pgfqpoint{5.315559in}{3.190884in}}%
\pgfpathlineto{\pgfqpoint{5.342053in}{3.375611in}}%
\pgfpathlineto{\pgfqpoint{5.345947in}{3.382614in}}%
\pgfpathlineto{\pgfqpoint{5.348001in}{3.381805in}}%
\pgfpathlineto{\pgfqpoint{5.350367in}{3.376421in}}%
\pgfpathlineto{\pgfqpoint{5.353670in}{3.359935in}}%
\pgfpathlineto{\pgfqpoint{5.357967in}{3.320664in}}%
\pgfpathlineto{\pgfqpoint{5.363234in}{3.240942in}}%
\pgfpathlineto{\pgfqpoint{5.369517in}{3.094380in}}%
\pgfpathlineto{\pgfqpoint{5.376983in}{2.841523in}}%
\pgfpathlineto{\pgfqpoint{5.386213in}{2.414412in}}%
\pgfpathlineto{\pgfqpoint{5.400007in}{1.621029in}}%
\pgfpathlineto{\pgfqpoint{5.400007in}{1.621029in}}%
\pgfusepath{stroke}%
\end{pgfscope}%
\begin{pgfscope}%
\pgfsetbuttcap%
\pgfsetroundjoin%
\pgfsetlinewidth{0.803000pt}%
\definecolor{currentstroke}{rgb}{0.000000,0.000000,0.000000}%
\pgfsetstrokecolor{currentstroke}%
\pgfsetdash{}{0pt}%
\pgfsys@defobject{currentmarker}{\pgfqpoint{0.000000in}{0.000000in}}{\pgfqpoint{0.048611in}{0.000000in}}{%
\pgfpathmoveto{\pgfqpoint{0.000000in}{0.000000in}}%
\pgfpathlineto{\pgfqpoint{0.048611in}{0.000000in}}%
\pgfusepath{stroke}%
}%
\begin{pgfscope}%
\pgfsys@transformshift{0.750000in}{2.010000in}%
\pgfsys@useobject{currentmarker}{}%
\end{pgfscope}%
\end{pgfscope}%
\begin{pgfscope}%
\definecolor{textcolor}{rgb}{0.000000,0.000000,0.000000}%
\pgfsetstrokecolor{textcolor}%
\pgfsetfillcolor{textcolor}%
\pgftext[x=0.701389in,y=2.010000in,right,]{\color{textcolor}\rmfamily\fontsize{10.000000}{12.000000}\selectfont \(\displaystyle {0}\)}%
\end{pgfscope}%
\begin{pgfscope}%
\pgfsetrectcap%
\pgfsetroundjoin%
\pgfsetlinewidth{0.803000pt}%
\definecolor{currentstroke}{rgb}{0.000000,0.000000,0.000000}%
\pgfsetstrokecolor{currentstroke}%
\pgfsetdash{}{0pt}%
\pgfpathmoveto{\pgfqpoint{0.750000in}{0.500000in}}%
\pgfpathlineto{\pgfqpoint{0.750000in}{3.520000in}}%
\pgfusepath{stroke}%
\end{pgfscope}%
\begin{pgfscope}%
\pgfsetroundcap%
\pgfsetroundjoin%
\pgfsetlinewidth{1.003750pt}%
\definecolor{currentstroke}{rgb}{0.000000,0.000000,0.000000}%
\pgfsetstrokecolor{currentstroke}%
\pgfsetdash{}{0pt}%
\pgfpathmoveto{\pgfqpoint{0.750000in}{2.010000in}}%
\pgfpathlineto{\pgfqpoint{5.400000in}{2.010000in}}%
\pgfpathlineto{\pgfqpoint{5.523361in}{2.010000in}}%
\pgfusepath{stroke}%
\end{pgfscope}%
\begin{pgfscope}%
\pgfsetroundcap%
\pgfsetroundjoin%
\definecolor{currentfill}{rgb}{0.121569,0.466667,0.705882}%
\pgfsetfillcolor{currentfill}%
\pgfsetlinewidth{1.003750pt}%
\definecolor{currentstroke}{rgb}{0.000000,0.000000,0.000000}%
\pgfsetstrokecolor{currentstroke}%
\pgfsetdash{}{0pt}%
\pgfpathmoveto{\pgfqpoint{5.467805in}{2.037778in}}%
\pgfpathlineto{\pgfqpoint{5.523361in}{2.010000in}}%
\pgfpathlineto{\pgfqpoint{5.467805in}{1.982222in}}%
\pgfpathlineto{\pgfqpoint{5.467805in}{2.037778in}}%
\pgfpathlineto{\pgfqpoint{5.467805in}{2.037778in}}%
\pgfpathclose%
\pgfusepath{stroke,fill}%
\end{pgfscope}%
\begin{pgfscope}%
\definecolor{textcolor}{rgb}{0.000000,0.000000,0.000000}%
\pgfsetstrokecolor{textcolor}%
\pgfsetfillcolor{textcolor}%
\pgftext[x=5.632500in,y=1.872727in,left,base]{\color{textcolor}\rmfamily\fontsize{10.000000}{12.000000}\selectfont \(\displaystyle t\)}%
\end{pgfscope}%
\end{pgfpicture}%
\makeatother%
\endgroup%

	\caption{Modulationsart Freguenzmodulation mit dem Parameter \(\omega_i\)}
	\label{fig:fm:FM}
\end{figure}

Um modulation zu Verstehen ist es am Anschaulichst mit der AM Amplitudenmodulation, 
da Phasenmodulation und Frequenzmodulation den gleichen Parameter verändert vernachlässige ich die Phasenmodulation ganz.



