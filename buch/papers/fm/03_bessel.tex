%
% teil2.tex -- Beispiel-File für teil2 
%
% (c) 2020 Prof Dr Andreas Müller, Hochschule Rapperswil
%
\section{FM und Besselfunktion 
\label{fm:section:teil2}}
\rhead{Teil 2}


TODO
Hier wird beschrieben wie die Bessel Funktion der FM im Frequenzspektrum hilft, wieso diese gebrauch wird und ihre Vorteile.
\begin{itemize}
    \item Zuerest einmal die Herleitung von FM zu der Besselfunktion
    \item Im Frequenzspektrum darstellen mit Farben, ersichtlich machen. 
    \item Parameter tuing der Trägerfrequenz, Modulierende frequenz und Beta. 
\end{itemize}


%\subsection{De finibus bonorum et malorum
%\label{fm:subsection:bonorum}}



