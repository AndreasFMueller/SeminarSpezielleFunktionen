%
% einleitung.tex -- Beispiel-File für die Einleitung
%
% (c) 2020 Prof Dr Andreas Müller, Hochschule Rapperswil
%
\section{AM - FM\label{fm:section:teil0}}
\rhead{AM- FM}

Das sinusförmige Trägersignal hat die übliche Form: 
\(x_c(t) = A_c \cdot cos(\omega_c(t)+\varphi)\).
Wobei die konstanten Amplitude \(A_c\) und Phase \(\varphi\) vom Nachrichtensignal \(m(t)\) verändert wird.
Der Parameter \(\omega_c\), die Trägerkreisfrequenz bzw. die Trägerfrequenz \(f_c = \frac{\omega_c}{2\pi}\),
steht nicht für die modulation zur verfügung, statt dessen kann durch ihn die Frequenzachse frei gewählt werden.
\newblockpunct
Jedoch ist das für die Vilfalt der Modulationsarten keine Einschrenkung.
Ein Nachrichtensignal kann auch über die Momentanfrequenz (instantenous frequency) \(\omega_i\) eines trägers verändert werden.
Mathematisch wird dann daraus
\[
    \omega_i = \omega_c + \frac{d \varphi(t)}{dt}
\]
mit der Ableitung der Phase.
\newline
\newline
TODO:
Hier beschrieib ich was AmplitudenModulation ist und mache dan den link zu Frequenzmodulation inkl Formel \[cos( cos x)\]



%Lorem ipsum dolor sit amet, consetetur sadipscing elitr, sed diam
%nonumy eirmod tempor invidunt ut labore et dolore magna aliquyam
erat, sed diam voluptua \cite{fm:bibtex}.
%At vero eos et accusam et justo duo dolores et ea rebum.
%Stet clita kasd gubergren, no sea takimata sanctus est Lorem ipsum
%dolor sit amet.


