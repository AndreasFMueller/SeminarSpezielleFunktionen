\documentclass[11pt,aspectratio=169]{beamer}
\usepackage[utf8]{inputenc}
\usepackage[T1]{fontenc}
\usepackage{lmodern}
\usepackage[ngerman]{babel}
\usepackage{tikz}
\usetheme{Hannover}

\begin{document}
	\author{Joshua Bär}
	\title{FM - Bessel}
	\subtitle{}
	\logo{}
	\institute{OST Ostschweizer Fachhochschule}
	\date{16.5.2022}
	\subject{Mathematisches Seminar}
	%\setbeamercovered{transparent}
	\setbeamercovered{invisible}
	\setbeamertemplate{navigation symbols}{}
	\begin{frame}[plain]
		\maketitle
	\end{frame}
%-------------------------------------------------------------------------------	
\section{Einführung}
	\begin{frame}
		\frametitle{Frequenzmodulation}
		\begin{itemize}
		\visible<1->{\item Für Übertragung von Daten}
		\visible<2->{\item Amplituden unabhängig}
		\end{itemize}
	\end{frame}
%-------------------------------------------------------------------------------	
	\begin{frame}
	\frametitle{Parameter}
	\begin{center}
		\begin{tabular}{ c c c } 
			\hline
			Nutzlas & Fehler & Versenden \\
			\hline 
			3 & 2 & 7 Werte eines Polynoms vom Grad 2 \\ 
			4 & 2 & 8 Werte eines Polynoms vom Grad 3 \\
\visible<1->{3}& 
\visible<1->{3}& 
\visible<1->{9 Werte eines Polynoms vom Grad 2} \\ 
			&&\\
\visible<1->{$k$} & 
\visible<1->{$t$} & 
\visible<1->{$k+2t$ Werte eines Polynoms vom Grad $k-1$} \\ 
			\hline
			&&\\
			&&\\
			\multicolumn{3}{l} {
				\visible<1>{Ausserdem können bis zu $2t$ Fehler erkannt werden!}
			}
		\end{tabular}
	\end{center}	
	\end{frame}

%-------------------------------------------------------------------------------	

\section{Diskrete Fourier Transformation}
	\begin{frame}
		\frametitle{Idee}
		\begin{itemize}
			\item Fourier-transformieren
			\item Übertragung
			\item Rücktransformieren
		\end{itemize}
	\end{frame}
%-------------------------------------------------------------------------------	
	\begin{frame}
		\begin{figure}
			\only<1>{
			\includegraphics[width=0.9\linewidth]{images/fig1.pdf}
			}
			\only<2>{
			\includegraphics[width=0.9\linewidth]{images/fig2.pdf}
			}
			\only<3>{
			\includegraphics[width=0.9\linewidth]{images/fig3.pdf}
			}
			\only<4>{
			\includegraphics[width=0.9\linewidth]{images/fig4.pdf}
			}
			\only<5>{
			\includegraphics[width=0.9\linewidth]{images/fig5.pdf}
			}
			\only<6>{
			\includegraphics[width=0.9\linewidth]{images/fig6.pdf}
			}
			\only<7>{
			\includegraphics[width=0.9\linewidth]{images/fig7.pdf}
			}
	\end{figure}
	\end{frame}
%-------------------------------------------------------------------------------		
	\begin{frame}
	\frametitle{Diskrete Fourier Transformation}
	\begin{itemize}
	\item Diskrete Fourier-Transformation gegeben durch:
	\visible<1->{
	\[
	\label{ft_discrete}
	\hat{c}_{k} 
	= \frac{1}{N} \sum_{n=0}^{N-1}
	{f}_n \cdot e^{-\frac{2\pi j}{N} \cdot kn}
	\]}
	\visible<2->{
	\item Ersetzte
	\[
	w = e^{-\frac{2\pi j}{N} k}
	\]}
	\visible<3->{
	\item Wenn $N$ konstant:
	\[
	\hat{c}_{k}=\frac{1}{N}( {f}_0 w^0 + {f}_1 w^1 + {f}_2 w^2 + \dots + {f}_{N-1} w^N)
	\]}
	\end{itemize}
	\end{frame}	

%-------------------------------------------------------------------------------	

%-------------------------------------------------------------------------------
	\begin{frame}
		\frametitle{Ein Beispiel}
		
		\begin{itemize}
			
			\onslide<1->{\item endlicher Körper $q = 11$}
			
			\onslide<2->{ist eine Primzahl}

			\onslide<3->{beinhaltet die Zahlen $\mathbb{F}_{11} = \{0,1,2,3,4,5,6,7,8,9,10\}$}
			
			\vspace{10pt}
			
			\onslide<4->{\item Nachrichtenblock $=$ Nutzlast $+$ Fehlerkorrekturstellen}
			
			\onslide<5->{$n = q - 1 = 10$ Zahlen}
			
			\vspace{10pt}
			
			\onslide<6->{\item Max.~Fehler $t = 2$}
			
			\onslide<7->{maximale Anzahl von Fehler, die wir noch korrigieren können}
			
			\vspace{10pt}
			
			\onslide<8->{\item Nutzlast $k = n -2t = 6$ Zahlen}
			
			\onslide<9->{Fehlerkorrkturstellen $2t = 4$ Zahlen}
			
			\onslide<10->{Nachricht $m = [0,0,0,0,4,7,2,5,8,1]$}
			
			\onslide<11->{als Polynom $m(X) = 4X^5 + 7X^4 + 2X^3 + 5X^2 + 8X + 1$}
			
		\end{itemize}
		
	\end{frame}


\end{document}
