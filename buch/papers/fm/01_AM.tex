%
% einleitung.tex -- Beispiel-File für die Einleitung
%
% (c) 2020 Prof Dr Andreas Müller, Hochschule Rapperswil
%
\section{Amplitudenmodulation\label{fm:section:teil0}}
\rhead{AM}

Das Ziel ist FM zu verstehen doch dazu wird zuerst AM erklärt welches einwenig einfacher zu verstehen ist und erst dann übertragen wir die Ideen in FM.
Nun zur Amplitudenmodulation verwenden wir das bevorzugte Trägersignal
\[
    x_c(t) = A_c \cdot \cos(\omega_ct).
\]
Dies bringt den grossen Vorteil das, dass modulierend Signal sämtliche Anteile im Frequenzspektrum inanspruch nimmt 
und das Trägersignal nur  zwei komplexe Schwingungen besitzt. 
Dies sieht man besonders in der Eulerischen Formel
\[
    x_c(t) = \frac{A_c}{2} \cdot e^{j\omega_ct}\;+\;\frac{A_c}{2} \cdot e^{-j\omega_ct}.
\]
Dabei ist die negative Frequenz der zweiten komplexen Schwingung zwingend erforderlich, damit in der Summe immer ein reellwertiges Trägersignal ergibt.
Nun wird der Parameter \(A_c\) durch das  Modulierende Signal \(m(t)\) ersetzt, wobei so \(m(t) \leqslant |1|\) normiert wurde.
\newline
\newline
TODO:
Hier beschrieib ich was AmplitudenModulation ist und mache dan den link zu Frequenzmodulation inkl Formel \[\cos( \cos x)\]
so wird beschrieben das daraus eigentlich \(x_c(t) = A_c \cdot \cos(\omega_i)\) wird und somit \(x_c(t) = A_c \cdot \cos(\omega_c + \frac{d \varphi(t)}{dt})\).
Da \(\sin \) abgeleitet \(\cos \) ergibt, so wird aus dem \(m(t)\) ein \( \frac{d \varphi(t)}{dt}\)  in der momentan frequenz. \[ \Rightarrow \cos( \cos x) \]

\subsection{Frequenzspektrum}