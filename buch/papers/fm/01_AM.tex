%
% einleitung.tex -- Beispiel-File für die Einleitung
%
% (c) 2020 Prof Dr Andreas Müller, Hochschule Rapperswil
%
\section{Amplitudenmodulation\label{fm:section:teil0}}
\rhead{AM}

Das Ziel ist FM zu verstehen doch dazu wird zuerst AM erklärt, welches einwenig einfacher zu verstehen ist und erst dann übertragen wir die Ideen in FM.
Nun zur Amplitudenmodulation verwenden wir das bevorzugte Trägersignal
\[
    x_c(t) = A_c \cdot \cos(\omega_ct).
\]
Dabei entseht wine Umhüllende kurve die unserem ursprünglichen signal \(m(t)\) entspricht.
\[
    x_c(t) = m(t) \cdot \cos(\omega_ct).
\]

TODO: Bild Umhüllende


Dies bringt den grossen Vorteil das, dass modulierend Signal sämtliche Anteile im Frequenzspektrum in Anspruch nimmt 
und das Trägersignal nur zwei komplexe Schwingungen besitzt. 
Dies sieht man besonders in der Eulerischen Formel
\[
    x_c(t) = \frac{A_c}{2} \cdot e^{j\omega_ct}\;+\;\frac{A_c}{2} \cdot e^{-j\omega_ct}.
    \label{fm:eq:AM:euler}
\]
Dabei ist die negative Frequenz der zweiten komplexen Schwingung zwingend erforderlich, damit in der Summe immer ein reellwertiges Trägersignal ergibt.
Nun wird der Parameter \(A_c\) durch das  Modulierende Signal \(m(t)\) ersetzt, wobei so \(m(t) \leqslant |1|\) normiert wurde.

\subsection{Frequenzspektrum}
Das Frequenzspektrum ist eine Darstellung von einem Signal im Frequenzbereich, das heisst man erkennt welche Frequenzen in einem Signal vorhanden sind.

Das Frequenzspektrum zeigt uns wo welche Frequenzen sich befinden, dies ist gerade wichtig fürs Frequenzmultiplexen.
Frequenzmultiplexen braucht man um mehrere Nachrichten verteilt zu übertragen.
Dafür müsse wir \(x_c(t)\) Fourietransformieren da es nur zwei Summanden, wie \eqref{fm:eq:AM:euler} mit \(e^{+-j\omega_ct}\)gezeigt, verschiebt sich das Signal \(m(t)\) um die träger Frequenz \(\omega_ct\).
Ist zudem \(m(t) 0 \sin(\omega_m t)\) so vereinfacht sich die Darstellung auf Dirac Impulse die jeweils rechts und links vom Träger signal sind.
TODO Bild 
Fürs Frequenzmultiplexen wird dann das Signal \(m(t)\) in die verschiedenen Frequenzen unterteilt um mehr Nachrichten zu übertragen.
Doch aufs Frequenzmultiplexen und weitere formen wie SSB und DSB möchte ich nicht weiter eingehen.
Diese können gerne im Nat Skript nachgelesen werden.

Das sich unser Signal \(m(t)\)  so mit dem Träger Signal verhält findet es man wieder gut beim empfangen, mit Hilfe der bandbreite weiss man was alles dazugehört.
Bei der Frequenzmodulation ist dies einwenig anders, dies sieht man auch (ref modulationsarten)

\begin{figure}
	\centering
	%% Creator: Matplotlib, PGF backend
%%
%% To include the figure in your LaTeX document, write
%%   \input{<filename>.pgf}
%%
%% Make sure the required packages are loaded in your preamble
%%   \usepackage{pgf}
%%
%% Also ensure that all the required font packages are loaded; for instance,
%% the lmodern package is sometimes necessary when using math font.
%%   \usepackage{lmodern}
%%
%% Figures using additional raster images can only be included by \input if
%% they are in the same directory as the main LaTeX file. For loading figures
%% from other directories you can use the `import` package
%%   \usepackage{import}
%%
%% and then include the figures with
%%   \import{<path to file>}{<filename>.pgf}
%%
%% Matplotlib used the following preamble
%%   \usepackage{fontspec}
%%   \setmainfont{DejaVuSerif.ttf}[Path=\detokenize{/home/joshua/.local/lib/python3.8/site-packages/matplotlib/mpl-data/fonts/ttf/}]
%%   \setsansfont{DejaVuSans.ttf}[Path=\detokenize{/home/joshua/.local/lib/python3.8/site-packages/matplotlib/mpl-data/fonts/ttf/}]
%%   \setmonofont{DejaVuSansMono.ttf}[Path=\detokenize{/home/joshua/.local/lib/python3.8/site-packages/matplotlib/mpl-data/fonts/ttf/}]
%%
\begingroup%
\makeatletter%
\begin{pgfpicture}%
\pgfpathrectangle{\pgfpointorigin}{\pgfqpoint{6.000000in}{4.000000in}}%
\pgfusepath{use as bounding box, clip}%
\begin{pgfscope}%
\pgfsetbuttcap%
\pgfsetmiterjoin%
\pgfsetlinewidth{0.000000pt}%
\definecolor{currentstroke}{rgb}{1.000000,1.000000,1.000000}%
\pgfsetstrokecolor{currentstroke}%
\pgfsetstrokeopacity{0.000000}%
\pgfsetdash{}{0pt}%
\pgfpathmoveto{\pgfqpoint{0.000000in}{0.000000in}}%
\pgfpathlineto{\pgfqpoint{6.000000in}{0.000000in}}%
\pgfpathlineto{\pgfqpoint{6.000000in}{4.000000in}}%
\pgfpathlineto{\pgfqpoint{0.000000in}{4.000000in}}%
\pgfpathlineto{\pgfqpoint{0.000000in}{0.000000in}}%
\pgfpathclose%
\pgfusepath{}%
\end{pgfscope}%
\begin{pgfscope}%
\pgfsetbuttcap%
\pgfsetmiterjoin%
\definecolor{currentfill}{rgb}{1.000000,1.000000,1.000000}%
\pgfsetfillcolor{currentfill}%
\pgfsetlinewidth{0.000000pt}%
\definecolor{currentstroke}{rgb}{0.000000,0.000000,0.000000}%
\pgfsetstrokecolor{currentstroke}%
\pgfsetstrokeopacity{0.000000}%
\pgfsetdash{}{0pt}%
\pgfpathmoveto{\pgfqpoint{0.750000in}{0.500000in}}%
\pgfpathlineto{\pgfqpoint{5.400000in}{0.500000in}}%
\pgfpathlineto{\pgfqpoint{5.400000in}{3.520000in}}%
\pgfpathlineto{\pgfqpoint{0.750000in}{3.520000in}}%
\pgfpathlineto{\pgfqpoint{0.750000in}{0.500000in}}%
\pgfpathclose%
\pgfusepath{fill}%
\end{pgfscope}%
\begin{pgfscope}%
\pgfsetbuttcap%
\pgfsetroundjoin%
\definecolor{currentfill}{rgb}{0.000000,0.000000,0.000000}%
\pgfsetfillcolor{currentfill}%
\pgfsetlinewidth{0.803000pt}%
\definecolor{currentstroke}{rgb}{0.000000,0.000000,0.000000}%
\pgfsetstrokecolor{currentstroke}%
\pgfsetdash{}{0pt}%
\pgfsys@defobject{currentmarker}{\pgfqpoint{0.000000in}{-0.048611in}}{\pgfqpoint{0.000000in}{0.000000in}}{%
\pgfpathmoveto{\pgfqpoint{0.000000in}{0.000000in}}%
\pgfpathlineto{\pgfqpoint{0.000000in}{-0.048611in}}%
\pgfusepath{stroke,fill}%
}%
\begin{pgfscope}%
\pgfsys@transformshift{0.961364in}{0.500000in}%
\pgfsys@useobject{currentmarker}{}%
\end{pgfscope}%
\end{pgfscope}%
\begin{pgfscope}%
\definecolor{textcolor}{rgb}{0.000000,0.000000,0.000000}%
\pgfsetstrokecolor{textcolor}%
\pgfsetfillcolor{textcolor}%
\pgftext[x=0.961364in,y=0.402778in,,top]{\color{textcolor}\sffamily\fontsize{10.000000}{12.000000}\selectfont 0.00}%
\end{pgfscope}%
\begin{pgfscope}%
\pgfsetbuttcap%
\pgfsetroundjoin%
\definecolor{currentfill}{rgb}{0.000000,0.000000,0.000000}%
\pgfsetfillcolor{currentfill}%
\pgfsetlinewidth{0.803000pt}%
\definecolor{currentstroke}{rgb}{0.000000,0.000000,0.000000}%
\pgfsetstrokecolor{currentstroke}%
\pgfsetdash{}{0pt}%
\pgfsys@defobject{currentmarker}{\pgfqpoint{0.000000in}{-0.048611in}}{\pgfqpoint{0.000000in}{0.000000in}}{%
\pgfpathmoveto{\pgfqpoint{0.000000in}{0.000000in}}%
\pgfpathlineto{\pgfqpoint{0.000000in}{-0.048611in}}%
\pgfusepath{stroke,fill}%
}%
\begin{pgfscope}%
\pgfsys@transformshift{1.806818in}{0.500000in}%
\pgfsys@useobject{currentmarker}{}%
\end{pgfscope}%
\end{pgfscope}%
\begin{pgfscope}%
\definecolor{textcolor}{rgb}{0.000000,0.000000,0.000000}%
\pgfsetstrokecolor{textcolor}%
\pgfsetfillcolor{textcolor}%
\pgftext[x=1.806818in,y=0.402778in,,top]{\color{textcolor}\sffamily\fontsize{10.000000}{12.000000}\selectfont 0.02}%
\end{pgfscope}%
\begin{pgfscope}%
\pgfsetbuttcap%
\pgfsetroundjoin%
\definecolor{currentfill}{rgb}{0.000000,0.000000,0.000000}%
\pgfsetfillcolor{currentfill}%
\pgfsetlinewidth{0.803000pt}%
\definecolor{currentstroke}{rgb}{0.000000,0.000000,0.000000}%
\pgfsetstrokecolor{currentstroke}%
\pgfsetdash{}{0pt}%
\pgfsys@defobject{currentmarker}{\pgfqpoint{0.000000in}{-0.048611in}}{\pgfqpoint{0.000000in}{0.000000in}}{%
\pgfpathmoveto{\pgfqpoint{0.000000in}{0.000000in}}%
\pgfpathlineto{\pgfqpoint{0.000000in}{-0.048611in}}%
\pgfusepath{stroke,fill}%
}%
\begin{pgfscope}%
\pgfsys@transformshift{2.652273in}{0.500000in}%
\pgfsys@useobject{currentmarker}{}%
\end{pgfscope}%
\end{pgfscope}%
\begin{pgfscope}%
\definecolor{textcolor}{rgb}{0.000000,0.000000,0.000000}%
\pgfsetstrokecolor{textcolor}%
\pgfsetfillcolor{textcolor}%
\pgftext[x=2.652273in,y=0.402778in,,top]{\color{textcolor}\sffamily\fontsize{10.000000}{12.000000}\selectfont 0.04}%
\end{pgfscope}%
\begin{pgfscope}%
\pgfsetbuttcap%
\pgfsetroundjoin%
\definecolor{currentfill}{rgb}{0.000000,0.000000,0.000000}%
\pgfsetfillcolor{currentfill}%
\pgfsetlinewidth{0.803000pt}%
\definecolor{currentstroke}{rgb}{0.000000,0.000000,0.000000}%
\pgfsetstrokecolor{currentstroke}%
\pgfsetdash{}{0pt}%
\pgfsys@defobject{currentmarker}{\pgfqpoint{0.000000in}{-0.048611in}}{\pgfqpoint{0.000000in}{0.000000in}}{%
\pgfpathmoveto{\pgfqpoint{0.000000in}{0.000000in}}%
\pgfpathlineto{\pgfqpoint{0.000000in}{-0.048611in}}%
\pgfusepath{stroke,fill}%
}%
\begin{pgfscope}%
\pgfsys@transformshift{3.497727in}{0.500000in}%
\pgfsys@useobject{currentmarker}{}%
\end{pgfscope}%
\end{pgfscope}%
\begin{pgfscope}%
\definecolor{textcolor}{rgb}{0.000000,0.000000,0.000000}%
\pgfsetstrokecolor{textcolor}%
\pgfsetfillcolor{textcolor}%
\pgftext[x=3.497727in,y=0.402778in,,top]{\color{textcolor}\sffamily\fontsize{10.000000}{12.000000}\selectfont 0.06}%
\end{pgfscope}%
\begin{pgfscope}%
\pgfsetbuttcap%
\pgfsetroundjoin%
\definecolor{currentfill}{rgb}{0.000000,0.000000,0.000000}%
\pgfsetfillcolor{currentfill}%
\pgfsetlinewidth{0.803000pt}%
\definecolor{currentstroke}{rgb}{0.000000,0.000000,0.000000}%
\pgfsetstrokecolor{currentstroke}%
\pgfsetdash{}{0pt}%
\pgfsys@defobject{currentmarker}{\pgfqpoint{0.000000in}{-0.048611in}}{\pgfqpoint{0.000000in}{0.000000in}}{%
\pgfpathmoveto{\pgfqpoint{0.000000in}{0.000000in}}%
\pgfpathlineto{\pgfqpoint{0.000000in}{-0.048611in}}%
\pgfusepath{stroke,fill}%
}%
\begin{pgfscope}%
\pgfsys@transformshift{4.343182in}{0.500000in}%
\pgfsys@useobject{currentmarker}{}%
\end{pgfscope}%
\end{pgfscope}%
\begin{pgfscope}%
\definecolor{textcolor}{rgb}{0.000000,0.000000,0.000000}%
\pgfsetstrokecolor{textcolor}%
\pgfsetfillcolor{textcolor}%
\pgftext[x=4.343182in,y=0.402778in,,top]{\color{textcolor}\sffamily\fontsize{10.000000}{12.000000}\selectfont 0.08}%
\end{pgfscope}%
\begin{pgfscope}%
\pgfsetbuttcap%
\pgfsetroundjoin%
\definecolor{currentfill}{rgb}{0.000000,0.000000,0.000000}%
\pgfsetfillcolor{currentfill}%
\pgfsetlinewidth{0.803000pt}%
\definecolor{currentstroke}{rgb}{0.000000,0.000000,0.000000}%
\pgfsetstrokecolor{currentstroke}%
\pgfsetdash{}{0pt}%
\pgfsys@defobject{currentmarker}{\pgfqpoint{0.000000in}{-0.048611in}}{\pgfqpoint{0.000000in}{0.000000in}}{%
\pgfpathmoveto{\pgfqpoint{0.000000in}{0.000000in}}%
\pgfpathlineto{\pgfqpoint{0.000000in}{-0.048611in}}%
\pgfusepath{stroke,fill}%
}%
\begin{pgfscope}%
\pgfsys@transformshift{5.188636in}{0.500000in}%
\pgfsys@useobject{currentmarker}{}%
\end{pgfscope}%
\end{pgfscope}%
\begin{pgfscope}%
\definecolor{textcolor}{rgb}{0.000000,0.000000,0.000000}%
\pgfsetstrokecolor{textcolor}%
\pgfsetfillcolor{textcolor}%
\pgftext[x=5.188636in,y=0.402778in,,top]{\color{textcolor}\sffamily\fontsize{10.000000}{12.000000}\selectfont 0.10}%
\end{pgfscope}%
\begin{pgfscope}%
\pgfsetbuttcap%
\pgfsetroundjoin%
\definecolor{currentfill}{rgb}{0.000000,0.000000,0.000000}%
\pgfsetfillcolor{currentfill}%
\pgfsetlinewidth{0.803000pt}%
\definecolor{currentstroke}{rgb}{0.000000,0.000000,0.000000}%
\pgfsetstrokecolor{currentstroke}%
\pgfsetdash{}{0pt}%
\pgfsys@defobject{currentmarker}{\pgfqpoint{-0.048611in}{0.000000in}}{\pgfqpoint{-0.000000in}{0.000000in}}{%
\pgfpathmoveto{\pgfqpoint{-0.000000in}{0.000000in}}%
\pgfpathlineto{\pgfqpoint{-0.048611in}{0.000000in}}%
\pgfusepath{stroke,fill}%
}%
\begin{pgfscope}%
\pgfsys@transformshift{0.750000in}{0.637273in}%
\pgfsys@useobject{currentmarker}{}%
\end{pgfscope}%
\end{pgfscope}%
\begin{pgfscope}%
\definecolor{textcolor}{rgb}{0.000000,0.000000,0.000000}%
\pgfsetstrokecolor{textcolor}%
\pgfsetfillcolor{textcolor}%
\pgftext[x=0.431898in, y=0.584511in, left, base]{\color{textcolor}\sffamily\fontsize{10.000000}{12.000000}\selectfont 0.0}%
\end{pgfscope}%
\begin{pgfscope}%
\pgfsetbuttcap%
\pgfsetroundjoin%
\definecolor{currentfill}{rgb}{0.000000,0.000000,0.000000}%
\pgfsetfillcolor{currentfill}%
\pgfsetlinewidth{0.803000pt}%
\definecolor{currentstroke}{rgb}{0.000000,0.000000,0.000000}%
\pgfsetstrokecolor{currentstroke}%
\pgfsetdash{}{0pt}%
\pgfsys@defobject{currentmarker}{\pgfqpoint{-0.048611in}{0.000000in}}{\pgfqpoint{-0.000000in}{0.000000in}}{%
\pgfpathmoveto{\pgfqpoint{-0.000000in}{0.000000in}}%
\pgfpathlineto{\pgfqpoint{-0.048611in}{0.000000in}}%
\pgfusepath{stroke,fill}%
}%
\begin{pgfscope}%
\pgfsys@transformshift{0.750000in}{1.186364in}%
\pgfsys@useobject{currentmarker}{}%
\end{pgfscope}%
\end{pgfscope}%
\begin{pgfscope}%
\definecolor{textcolor}{rgb}{0.000000,0.000000,0.000000}%
\pgfsetstrokecolor{textcolor}%
\pgfsetfillcolor{textcolor}%
\pgftext[x=0.431898in, y=1.133602in, left, base]{\color{textcolor}\sffamily\fontsize{10.000000}{12.000000}\selectfont 0.2}%
\end{pgfscope}%
\begin{pgfscope}%
\pgfsetbuttcap%
\pgfsetroundjoin%
\definecolor{currentfill}{rgb}{0.000000,0.000000,0.000000}%
\pgfsetfillcolor{currentfill}%
\pgfsetlinewidth{0.803000pt}%
\definecolor{currentstroke}{rgb}{0.000000,0.000000,0.000000}%
\pgfsetstrokecolor{currentstroke}%
\pgfsetdash{}{0pt}%
\pgfsys@defobject{currentmarker}{\pgfqpoint{-0.048611in}{0.000000in}}{\pgfqpoint{-0.000000in}{0.000000in}}{%
\pgfpathmoveto{\pgfqpoint{-0.000000in}{0.000000in}}%
\pgfpathlineto{\pgfqpoint{-0.048611in}{0.000000in}}%
\pgfusepath{stroke,fill}%
}%
\begin{pgfscope}%
\pgfsys@transformshift{0.750000in}{1.735455in}%
\pgfsys@useobject{currentmarker}{}%
\end{pgfscope}%
\end{pgfscope}%
\begin{pgfscope}%
\definecolor{textcolor}{rgb}{0.000000,0.000000,0.000000}%
\pgfsetstrokecolor{textcolor}%
\pgfsetfillcolor{textcolor}%
\pgftext[x=0.431898in, y=1.682693in, left, base]{\color{textcolor}\sffamily\fontsize{10.000000}{12.000000}\selectfont 0.4}%
\end{pgfscope}%
\begin{pgfscope}%
\pgfsetbuttcap%
\pgfsetroundjoin%
\definecolor{currentfill}{rgb}{0.000000,0.000000,0.000000}%
\pgfsetfillcolor{currentfill}%
\pgfsetlinewidth{0.803000pt}%
\definecolor{currentstroke}{rgb}{0.000000,0.000000,0.000000}%
\pgfsetstrokecolor{currentstroke}%
\pgfsetdash{}{0pt}%
\pgfsys@defobject{currentmarker}{\pgfqpoint{-0.048611in}{0.000000in}}{\pgfqpoint{-0.000000in}{0.000000in}}{%
\pgfpathmoveto{\pgfqpoint{-0.000000in}{0.000000in}}%
\pgfpathlineto{\pgfqpoint{-0.048611in}{0.000000in}}%
\pgfusepath{stroke,fill}%
}%
\begin{pgfscope}%
\pgfsys@transformshift{0.750000in}{2.284545in}%
\pgfsys@useobject{currentmarker}{}%
\end{pgfscope}%
\end{pgfscope}%
\begin{pgfscope}%
\definecolor{textcolor}{rgb}{0.000000,0.000000,0.000000}%
\pgfsetstrokecolor{textcolor}%
\pgfsetfillcolor{textcolor}%
\pgftext[x=0.431898in, y=2.231784in, left, base]{\color{textcolor}\sffamily\fontsize{10.000000}{12.000000}\selectfont 0.6}%
\end{pgfscope}%
\begin{pgfscope}%
\pgfsetbuttcap%
\pgfsetroundjoin%
\definecolor{currentfill}{rgb}{0.000000,0.000000,0.000000}%
\pgfsetfillcolor{currentfill}%
\pgfsetlinewidth{0.803000pt}%
\definecolor{currentstroke}{rgb}{0.000000,0.000000,0.000000}%
\pgfsetstrokecolor{currentstroke}%
\pgfsetdash{}{0pt}%
\pgfsys@defobject{currentmarker}{\pgfqpoint{-0.048611in}{0.000000in}}{\pgfqpoint{-0.000000in}{0.000000in}}{%
\pgfpathmoveto{\pgfqpoint{-0.000000in}{0.000000in}}%
\pgfpathlineto{\pgfqpoint{-0.048611in}{0.000000in}}%
\pgfusepath{stroke,fill}%
}%
\begin{pgfscope}%
\pgfsys@transformshift{0.750000in}{2.833636in}%
\pgfsys@useobject{currentmarker}{}%
\end{pgfscope}%
\end{pgfscope}%
\begin{pgfscope}%
\definecolor{textcolor}{rgb}{0.000000,0.000000,0.000000}%
\pgfsetstrokecolor{textcolor}%
\pgfsetfillcolor{textcolor}%
\pgftext[x=0.431898in, y=2.780875in, left, base]{\color{textcolor}\sffamily\fontsize{10.000000}{12.000000}\selectfont 0.8}%
\end{pgfscope}%
\begin{pgfscope}%
\pgfsetbuttcap%
\pgfsetroundjoin%
\definecolor{currentfill}{rgb}{0.000000,0.000000,0.000000}%
\pgfsetfillcolor{currentfill}%
\pgfsetlinewidth{0.803000pt}%
\definecolor{currentstroke}{rgb}{0.000000,0.000000,0.000000}%
\pgfsetstrokecolor{currentstroke}%
\pgfsetdash{}{0pt}%
\pgfsys@defobject{currentmarker}{\pgfqpoint{-0.048611in}{0.000000in}}{\pgfqpoint{-0.000000in}{0.000000in}}{%
\pgfpathmoveto{\pgfqpoint{-0.000000in}{0.000000in}}%
\pgfpathlineto{\pgfqpoint{-0.048611in}{0.000000in}}%
\pgfusepath{stroke,fill}%
}%
\begin{pgfscope}%
\pgfsys@transformshift{0.750000in}{3.382727in}%
\pgfsys@useobject{currentmarker}{}%
\end{pgfscope}%
\end{pgfscope}%
\begin{pgfscope}%
\definecolor{textcolor}{rgb}{0.000000,0.000000,0.000000}%
\pgfsetstrokecolor{textcolor}%
\pgfsetfillcolor{textcolor}%
\pgftext[x=0.431898in, y=3.329966in, left, base]{\color{textcolor}\sffamily\fontsize{10.000000}{12.000000}\selectfont 1.0}%
\end{pgfscope}%
\begin{pgfscope}%
\pgfpathrectangle{\pgfqpoint{0.750000in}{0.500000in}}{\pgfqpoint{4.650000in}{3.020000in}}%
\pgfusepath{clip}%
\pgfsetrectcap%
\pgfsetroundjoin%
\pgfsetlinewidth{1.505625pt}%
\definecolor{currentstroke}{rgb}{0.121569,0.466667,0.705882}%
\pgfsetstrokecolor{currentstroke}%
\pgfsetdash{}{0pt}%
\pgfpathmoveto{\pgfqpoint{0.961364in}{2.010000in}}%
\pgfpathlineto{\pgfqpoint{2.371864in}{2.010000in}}%
\pgfpathlineto{\pgfqpoint{3.076057in}{3.382727in}}%
\pgfpathlineto{\pgfqpoint{5.188636in}{3.382727in}}%
\pgfpathlineto{\pgfqpoint{5.188636in}{3.382727in}}%
\pgfusepath{stroke}%
\end{pgfscope}%
\begin{pgfscope}%
\pgfpathrectangle{\pgfqpoint{0.750000in}{0.500000in}}{\pgfqpoint{4.650000in}{3.020000in}}%
\pgfusepath{clip}%
\pgfsetrectcap%
\pgfsetroundjoin%
\pgfsetlinewidth{1.505625pt}%
\definecolor{currentstroke}{rgb}{1.000000,0.498039,0.054902}%
\pgfsetstrokecolor{currentstroke}%
\pgfsetdash{}{0pt}%
\pgfpathmoveto{\pgfqpoint{0.961364in}{0.637273in}}%
\pgfpathlineto{\pgfqpoint{1.246847in}{0.926335in}}%
\pgfpathlineto{\pgfqpoint{1.399105in}{1.076010in}}%
\pgfpathlineto{\pgfqpoint{1.525987in}{1.196531in}}%
\pgfpathlineto{\pgfqpoint{1.640180in}{1.300781in}}%
\pgfpathlineto{\pgfqpoint{1.743800in}{1.391265in}}%
\pgfpathlineto{\pgfqpoint{1.841076in}{1.472151in}}%
\pgfpathlineto{\pgfqpoint{1.932008in}{1.543827in}}%
\pgfpathlineto{\pgfqpoint{2.018710in}{1.608319in}}%
\pgfpathlineto{\pgfqpoint{2.101184in}{1.665928in}}%
\pgfpathlineto{\pgfqpoint{2.181542in}{1.718347in}}%
\pgfpathlineto{\pgfqpoint{2.259786in}{1.765684in}}%
\pgfpathlineto{\pgfqpoint{2.335915in}{1.808081in}}%
\pgfpathlineto{\pgfqpoint{2.371864in}{1.826808in}}%
\pgfpathlineto{\pgfqpoint{2.477599in}{2.063051in}}%
\pgfpathlineto{\pgfqpoint{2.644660in}{2.445205in}}%
\pgfpathlineto{\pgfqpoint{2.811721in}{2.824886in}}%
\pgfpathlineto{\pgfqpoint{2.911111in}{3.043231in}}%
\pgfpathlineto{\pgfqpoint{2.993584in}{3.217228in}}%
\pgfpathlineto{\pgfqpoint{3.065484in}{3.362048in}}%
\pgfpathlineto{\pgfqpoint{3.076057in}{3.382726in}}%
\pgfpathlineto{\pgfqpoint{3.120466in}{3.381160in}}%
\pgfpathlineto{\pgfqpoint{3.164874in}{3.376606in}}%
\pgfpathlineto{\pgfqpoint{3.209283in}{3.369067in}}%
\pgfpathlineto{\pgfqpoint{3.255806in}{3.357979in}}%
\pgfpathlineto{\pgfqpoint{3.302330in}{3.343639in}}%
\pgfpathlineto{\pgfqpoint{3.348853in}{3.326064in}}%
\pgfpathlineto{\pgfqpoint{3.395376in}{3.305276in}}%
\pgfpathlineto{\pgfqpoint{3.444014in}{3.280132in}}%
\pgfpathlineto{\pgfqpoint{3.492652in}{3.251537in}}%
\pgfpathlineto{\pgfqpoint{3.543405in}{3.218057in}}%
\pgfpathlineto{\pgfqpoint{3.594157in}{3.180906in}}%
\pgfpathlineto{\pgfqpoint{3.647025in}{3.138360in}}%
\pgfpathlineto{\pgfqpoint{3.702007in}{3.090019in}}%
\pgfpathlineto{\pgfqpoint{3.756989in}{3.037583in}}%
\pgfpathlineto{\pgfqpoint{3.814085in}{2.978890in}}%
\pgfpathlineto{\pgfqpoint{3.873297in}{2.913572in}}%
\pgfpathlineto{\pgfqpoint{3.934623in}{2.841277in}}%
\pgfpathlineto{\pgfqpoint{3.998064in}{2.761675in}}%
\pgfpathlineto{\pgfqpoint{4.063619in}{2.674461in}}%
\pgfpathlineto{\pgfqpoint{4.133404in}{2.576313in}}%
\pgfpathlineto{\pgfqpoint{4.205304in}{2.469740in}}%
\pgfpathlineto{\pgfqpoint{4.281433in}{2.351203in}}%
\pgfpathlineto{\pgfqpoint{4.361791in}{2.220139in}}%
\pgfpathlineto{\pgfqpoint{4.446379in}{2.076089in}}%
\pgfpathlineto{\pgfqpoint{4.537311in}{1.914912in}}%
\pgfpathlineto{\pgfqpoint{4.634587in}{1.736053in}}%
\pgfpathlineto{\pgfqpoint{4.742436in}{1.531083in}}%
\pgfpathlineto{\pgfqpoint{4.865088in}{1.291080in}}%
\pgfpathlineto{\pgfqpoint{5.015231in}{0.990100in}}%
\pgfpathlineto{\pgfqpoint{5.188636in}{0.637273in}}%
\pgfpathlineto{\pgfqpoint{5.188636in}{0.637273in}}%
\pgfusepath{stroke}%
\end{pgfscope}%
\begin{pgfscope}%
\pgfsetrectcap%
\pgfsetmiterjoin%
\pgfsetlinewidth{0.803000pt}%
\definecolor{currentstroke}{rgb}{0.000000,0.000000,0.000000}%
\pgfsetstrokecolor{currentstroke}%
\pgfsetdash{}{0pt}%
\pgfpathmoveto{\pgfqpoint{0.750000in}{0.500000in}}%
\pgfpathlineto{\pgfqpoint{0.750000in}{3.520000in}}%
\pgfusepath{stroke}%
\end{pgfscope}%
\begin{pgfscope}%
\pgfsetrectcap%
\pgfsetmiterjoin%
\pgfsetlinewidth{0.803000pt}%
\definecolor{currentstroke}{rgb}{0.000000,0.000000,0.000000}%
\pgfsetstrokecolor{currentstroke}%
\pgfsetdash{}{0pt}%
\pgfpathmoveto{\pgfqpoint{5.400000in}{0.500000in}}%
\pgfpathlineto{\pgfqpoint{5.400000in}{3.520000in}}%
\pgfusepath{stroke}%
\end{pgfscope}%
\begin{pgfscope}%
\pgfsetrectcap%
\pgfsetmiterjoin%
\pgfsetlinewidth{0.803000pt}%
\definecolor{currentstroke}{rgb}{0.000000,0.000000,0.000000}%
\pgfsetstrokecolor{currentstroke}%
\pgfsetdash{}{0pt}%
\pgfpathmoveto{\pgfqpoint{0.750000in}{0.500000in}}%
\pgfpathlineto{\pgfqpoint{5.400000in}{0.500000in}}%
\pgfusepath{stroke}%
\end{pgfscope}%
\begin{pgfscope}%
\pgfsetrectcap%
\pgfsetmiterjoin%
\pgfsetlinewidth{0.803000pt}%
\definecolor{currentstroke}{rgb}{0.000000,0.000000,0.000000}%
\pgfsetstrokecolor{currentstroke}%
\pgfsetdash{}{0pt}%
\pgfpathmoveto{\pgfqpoint{0.750000in}{3.520000in}}%
\pgfpathlineto{\pgfqpoint{5.400000in}{3.520000in}}%
\pgfusepath{stroke}%
\end{pgfscope}%
\end{pgfpicture}%
\makeatother%
\endgroup%

	\caption{modulierende Signal \(m(t)\)}
	\label{fig:bessel}
\end{figure}
%
TODO:
Bilder 
%Hier beschrieib ich was AmplitudenModulation ist und mache dan den link zu Frequenzmodulation inkl Formel \[\cos( \cos x)\]
%so wird beschrieben das daraus eigentlich \(x_c(t) = A_c \cdot \cos(\omega_i)\) wird und somit \(x_c(t) = A_c \cdot \cos(\omega_c + \frac{d \varphi(t)}{dt})\).
%Da \(\sin \) abgeleitet \(\cos \) ergibt, so wird aus dem \(m(t)\) ein \( \frac{d \varphi(t)}{dt}\)  in der momentan frequenz. \[ \Rightarrow \cos( \cos x) \]


%
%Ein Ziel der Modulation besteht darin, mehrere Nachrichtensignale von verschiedenen Sendern gleichzeitig
%in verschiedenen Frequenzbereichen über den gleichen Kanal zu senden. Um dieses Frequenzmultiplexing
%störungsfrei und mit eine Vielzahl von Teilnehmern durchführen zu können, muss die spektrale Beschaffen-
%heit der modulierten Signale möglichst gut bekannt sein.
%Dank des Modulationssatzes der Fouriertransformation lässt sich das Spektrum eines gewöhnlichen AM Si-
%gnals sofort bestimmen:
%A c μ
%F
%·(M n (ω−ω c ) + M n (ω+ω c )) (5.5)
%A c ·(1+μm n (t))·cos(ω c t) ❝ s A c π (δ(ω−ω c ) + δ(ω+ω c )) +
%2
%Das zweiseitige Spektrum des Nachrichtensignals M (ω) wird mit dem Faktor A 2 c μ gewichtet und einmal
%nach +ω c und einmal nach −ω c verschoben. Dies führt im Vergleich zum Basisbandsignal zu einer Verdop-
%pelung der Bandbreite mit je einem Seitenband links und rechts der Trägerfrequenz. Weiter beinhaltet das
%Amplitudendichtespektrum je eine Deltafunktion mit Gewicht A c π an den Stellen ±ω c , d.h. ein fester, nicht-
%modulierter Amplitudenanteil bei der eigentlichen Trägerfrequenz.
%Das Amplitudendichtespektrum ist im nachfolgenden Graphen für A c = 1 und μ = 100% dargestellt.5.3. Gewöhnliche Amplitudenmodulation
%47
%Abbildung 5.12: Amplitudendichtespektrum von gewöhnlicher AM
%Für das Nachrichtensignal wurde in diesem Graph mit einem Keil symbolhaft ein Amplitudendichtespektrum
%|M (ω)| gewählt, bei welchem der Anteil auf der positiven und jener auf der negativen Frequenzachse visuell
%gut auseinandergehalten werden können. Ein solch geformtes Spektrum wird aber in der Praxis kaum je
%auftreten: bei periodischen Testsignalen besteht das Nachrichtensignal aus einem Linienspektrum, bei einem
%Energiesignal mit zufälligem Verlauf aus einem kontinuierlichen Spektrum, welches jedoch nicht auf diese
%einfache Art geformt sein wird