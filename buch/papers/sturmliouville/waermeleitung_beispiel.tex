%
% waermeleitung_beispiel.tex -- Beispiel Wärmeleitung in homogenem Stab.
% Author: Erik Löffler
%
% (c) 2020 Prof Dr Andreas Müller, Hochschule Rapperswil
%

\subsection{Wärmeleitung in einem Homogenen Stab}

In diesem Abschnitt wird das Problem der Wärmeleitung in einem homogenen Stab
betrachtet und wie das Sturm-Liouville-Problem bei der Beschreibung dieses
physikalischen Phänomenes auftritt.

Zunächst wird ein eindimensionaler homogener Stab der Länge $l$ und
Wärmeleitkoeffizient $\kappa$ betrachtet.
Es ergibt sich für das Wärmeleitungsproblem
die partielle Differentialgleichung
\begin{equation}
    \label{sturmliouville:eq:example-fourier-heat-equation}
    \frac{\partial u}{\partial t} =
    \kappa \frac{\partial^{2}u}{{\partial x}^{2}},
\end{equation}
wobei der Stab in diesem Fall auf der $X$-Achse im Intervall $[0,l]$ liegt.

Da diese Differentialgleichung das Problem allgemein für einen homogenen
Stab beschreibt, werden zusätzliche Bedingungen benötigt, um beispielsweise
die Lösung für einen Stab zu finden, bei dem die Enden auf konstanter 
Tempreatur gehalten werden.

%
% Randbedingungen für Stab mit konstanten Endtemperaturen
%
\subsubsection{Randbedingungen für Stab mit Enden auf konstanter Temperatur}

Die Enden des Stabes auf konstanter Temperatur zu halten bedeutet, dass die
Lösungsfunktion $u(t,x)$ bei $x = 0$ und $x = l$ nur die vorgegebene
Temperatur zurückgeben darf. Diese wird einfachheitshalber als $0$ angenomen.
Es folgt nun
\begin{equation}
    \label{sturmliouville:eq:example-fourier-boundary-condition-ends-constant}
    u(t,0)
    =
    u(t,l)
    =
    0
\end{equation}
als Randbedingungen.

%
% Randbedingungen für Stab mit isolierten Enden
%

\subsubsection{Randbedingungen für Stab mit isolierten Enden}

Bei isolierten Enden des Stabes können beliebige Temperaturen für $x = 0$ und
$x = l$ auftreten. In diesem Fall ist es nicht erlaubt, dass Wärme vom Stab
an die Umgebung oder von der Umgebung an den Stab abgegeben wird.

Aus der Physik ist bekannt, dass Wärme immer von der höheren zur tieferen
Temperatur fliesst. Um Wärmefluss zu unterdrücken, muss also dafür gesorgt
werden, dass am Rand des Stabes keine Temperaturdifferenz existiert oder 
dass die partiellen Ableitungen von $u(t,x)$ nach $x$ bei $x = 0$ und $x = l$
verschwinden.
Somit folgen
\begin{equation}
    \label{sturmliouville:eq:example-fourier-boundary-condition-ends-isolated}
    \frac{\partial}{\partial x} u(t, 0)
    =
    \frac{\partial}{\partial x} u(t, l)
    =
    0
\end{equation}
als Randbedingungen.

%
% Lösung der Differenzialgleichung mittels Separation
%

\subsubsection{Lösung der Differenzialgleichung}

Da die Lösungsfunktion von zwei Variablen abhängig ist, wird als Lösungsansatz
die Separationsmethode verwendet.
Dazu wird 
\[
    u(t,x)
    =
    T(t)X(x)
\]
in die partielle 
Differenzialgleichung~\eqref{sturmliouville:eq:example-fourier-heat-equation}
eingesetzt.
Daraus ergibt sich 
\[
    T^{\prime}(t)X(x)
    =
    \kappa T(t)X^{\prime \prime}(x)
\]
als neue Form.

Nun können alle von $t$ abhängigen Ausdrücke auf die linke Seite, sowie alle
von $x$ abhängigen Ausdrücke auf die rechte Seite gebracht werden und mittels
der neuen Variablen $\mu$ gekoppelt werden:
\[
    \frac{T^{\prime}(t)}{\kappa T(t)}
    =
    \frac{X^{\prime \prime}(x)}{X(x)}
    =
    \mu
\]
Durch die Einführung von $\mu$ kann das Problem nun in zwei separate
Differenzialgleichungen aufgeteilt werden:
\begin{equation}
    \label{sturmliouville:eq:example-fourier-separated-x}
    X^{\prime \prime}(x) - \mu X(x)
    =
    0
\end{equation}
\begin{equation}
    \label{sturmliouville:eq:example-fourier-separated-t}
    T^{\prime}(t) - \kappa \mu T(t)
    =
    0
\end{equation}

%
% Überprüfung Orthogonalität der Lösungen
%

Es ist an dieser Stelle zu bemerken, dass die Gleichung in $x$ in 
Sturm-Liouville-Form ist.
Erfüllen die Randbedingungen des Stab-Problems auch die Randbedingungen des
Sturm-Liouville-Problems, kann bereits die Aussage getroffen werden, dass alle
Lösungen für die Gleichung in $x$ orthogonal sein werden.

Da die Bedingungen des Stab-Problem nur Anforderungen an $x$ stellen, können
diese direkt für $X(x)$ übernomen werden. Es gilt also $X(0) = X(l) = 0$.
Damit die Lösungen von $X$ orthogonal sind, müssen also die Gleichungen
\begin{equation}
\begin{aligned}
	\label{sturmliouville:eq:example-fourier-randbedingungen}
	k_a X(a) + h_a p(a) X'(a) &= 0 \\
	k_b X(b) + h_b p(b) X'(b) &= 0
\end{aligned}
\end{equation}
erfüllt sein und es muss ausserdem
\begin{equation}
\begin{aligned}
    \label{sturmliouville:eq:example-fourier-coefficient-constraints}
    |k_a|^2 + |h_a|^2 &\neq 0\\
    |k_b|^2 + |h_b|^2 &\neq 0\\
\end{aligned}
\end{equation}
gelten.

Um zu verifizieren, ob die Randbedingungen erfüllt sind, wird zunächst
$p(x)$
benötigt.
Dazu wird die Gleichung~\eqref{sturmliouville:eq:example-fourier-separated-x}
mit der
Sturm-Liouville-Form~\eqref{eq:sturm-liouville-equation} verglichen, was zu
$p(x) = 1$ führt.

Werden nun $p(x)$ und die 
Randbedingungen~\eqref{sturmliouville:eq:example-fourier-boundary-condition-ends-constant}
in \eqref{sturmliouville:eq:example-fourier-randbedingungen} eigesetzt, erhält
man
\[
\begin{aligned}
	k_a y(0) + h_a y'(0) &= h_a y'(0) = 0 \\
	k_b y(l) + h_b y'(l) &= h_b y'(l) = 0.
\end{aligned}
\]
Damit die Gleichungen erfüllt sind, müssen $h_a = 0$ und $h_b = 0$ sein.
Zusätzlich müssen aber die 
Bedingungen~\eqref{sturmliouville:eq:example-fourier-coefficient-constraints}
erfüllt sein und da $y(0) = 0$ und $y(l) = 0$ sind, können belibige $k_a \neq 0$
und $k_b \neq 0$ gewählt werden.

Somit ist gezeigt, dass die Randbedingungen des Stab-Problems für Enden auf
konstanter Temperatur auch die Sturm-Liouville-Randbedingungen erfüllen und
alle daraus reultierenden Lösungen orthogonal sind.
Analog dazu kann gezeit werden, dass die Randbedingungen für einen Stab mit
isolierten Enden ebenfalls die Sturm-Liouville-Randbedingungen erfüllen und
somit auch zu orthogonalen Lösungen führen.

%
%   Lösung von X(x), Teil mu
%

\subsubsection{Lösund der Differentialgleichung in $x$}
Als erstes wird auf die
Gleichung~\eqref{sturmliouville:eq:example-fourier-separated-x} eingegangen.
Aufgrund der Struktur der Gleichung
\[
    X^{\prime \prime}(x) - \mu X(x)
    =
    0
\]
wird ein trigonometrischer Ansatz gewählt.
Die Lösungen für $X(x)$ sind also von der Form
\[
    X(x)
    =
    A \cos \left( \alpha x\right) + B \sin \left( \beta x\right).
\]

Dieser Ansatz wird nun solange differenziert, bis alle in
Gleichung~\eqref{sturmliouville:eq:example-fourier-separated-x} enthaltenen
Ableitungen vorhanden sind.
Man erhält also
\[
    X^{\prime}(x)
    =
    - \alpha A \sin \left( \alpha x \right) +
    \beta B \cos \left( \beta x \right)
\]
und
\[
    X^{\prime \prime}(x)
    =
    -\alpha^{2} A \cos \left( \alpha x \right) -
    \beta^{2} B \sin \left( \beta x \right).
\]

Eingesetzt in Gleichung~\eqref{sturmliouville:eq:example-fourier-separated-x}
ergibt dies
\[
    -\alpha^{2}A\cos(\alpha x) - \beta^{2}B\sin(\beta x) -
    \mu\left(A\cos(\alpha x) + B\sin(\beta x)\right)
    =
    0
\]
und durch umformen somit
\[
    -\alpha^{2}A\cos(\alpha x) - \beta^{2}B\sin(\beta x)
    =
    \mu A\cos(\alpha x) + \mu B\sin(\beta x).
\]

Mittels Koeffizientenvergleich von
\[
\begin{aligned}
    -\alpha^{2}A\cos(\alpha x)
    &=
    \mu A\cos(\alpha x)
    \\
    -\beta^{2}B\sin(\beta x)
    &=
    \mu B\sin(\beta x)
\end{aligned}
\]
ist schnell ersichtlich, dass $ \mu = -\alpha^{2} = -\beta^{2} $ gelten muss für
$ A \neq 0 $ oder $ B \neq 0 $.
Zur Berechnung von $ \mu $ bleiben also noch  $ \alpha $ und $ \beta $ zu
bestimmen.
Dazu werden nochmals die
Randbedingungen~\eqref{sturmliouville:eq:example-fourier-boundary-condition-ends-constant} 
und \eqref{sturmliouville:eq:example-fourier-boundary-condition-ends-isolated}
benötigt.

Da die Koeffizienten $A$ und $B$, sowie die Parameter $\alpha$ uns $\beta$ im
allgemeninen ungleich $0$ sind, müssen die Randbedingungen durch die
trigonometrischen Funktionen erfüllt werden.

Es werden nun die 
Randbedingungen~\eqref{sturmliouville:eq:example-fourier-boundary-condition-ends-constant}
für einen Stab mit Enden auf konstanter Temperatur in die
Gleichung~\eqref{sturmliouville:eq:example-fourier-separated-x} eingesetzt.
Betrachten wir zunächst die Bedingung für $x = 0$.
Dies fürht zu
\[
    X(0)
    =
    A \cos(0 \alpha) + B \sin(0 \beta)
    =
    0.
\]
Da $\cos(0) \neq 0$ ist, muss in diesem Fall $A = 0$ gelten.
Für den zweiten Summanden ist wegen $\sin(0) = 0$ die Randbedingung erfüllt.

Wird nun die zweite Randbedingung für $x = l$ mit $A = 0$ eingesetzt, ergibt
sich
\[
    X(l)
    =
    0 \cos(\alpha l) + B \sin(\beta l)
    =
    B \sin(\beta l)
    = 0.
\]

$\beta$ muss also so gewählt werden, dass $\sin(\beta l) = 0$ gilt.
Es bleibt noch nach $\beta$ aufzulösen:
\[
\begin{aligned}
    \sin(\beta l) &= 0 \\
    \beta l &= n \pi \qquad n \in \mathbb{N} \\
    \beta &= \frac{n \pi}{l} \qquad n \in \mathbb{N}
\end{aligned}
\]

Es folgt nun wegen $\mu = -\beta^{2}$, dass
\[
    \mu_1 = -\beta^{2} = -\frac{n^{2}\pi^{2}}{l^{2}}
\]
sein muss.
Ausserdem ist zu bemerken, dass dies auch gleich $-\alpha^{2}$ ist.
Da aber $A = 0$ gilt und der Summand mit $\alpha$ verschwindet, ist dies keine
Verletzung der Randbedingungen.

Durch alanoges Vorgehen kann nun auch das Problem mit isolierten Enden gelöst
werden.
Setzt man nun die 
Randbedingungen~\eqref{sturmliouville:eq:example-fourier-boundary-condition-ends-isolated}
in $X^{\prime}$ ein, beginnend für $x = 0$. Es ergibt sich
\[
    X^{\prime}(0)
    =
    -\alpha A \sin(0 \alpha) + \beta B \cos(0 \beta)
    = 0.
\]
In diesem Fall muss $B = 0$ gelten.
Zusammen mit der Bedignung für $x = l$
folgt nun
\[
    X^{\prime}(l)
    =
    - \alpha A \sin(\alpha l) + 0 \beta \cos(\beta l)
    =
    - \alpha A \sin(\alpha l)
    = 0.
\]

Wiedrum muss über die $\sin$-Funktion sicher gestellt werden, dass der
Ausdruck den Randbedingungen entspricht.
Es folgt nun
\[
\begin{aligned}
    \sin(\alpha l) &= 0 \\
    \alpha l &= n \pi \qquad n \in \mathbb{N} \\
    \alpha &= \frac{n \pi}{l} \qquad n \in \mathbb{N}
\end{aligned}
\]
und somit
\[
    \mu_2 = -\alpha^{2} = -\frac{n^{2}\pi^{2}}{l^{2}}.
\]

Es ergibt sich also sowohl für einen Stab mit Enden auf konstanter Temperatur
wie auch mit isolierten Enden
\begin{equation}
    \label{sturmliouville:eq:example-fourier-mu-solution}
    \mu
    =
    -\frac{n^{2}\pi^{2}}{l^{2}}.
\end{equation}

%
% Lösung von X(x), Teil: Koeffizienten a_n und b_n mittels skalarprodukt.
%

Bisher wurde über die Koeffizienten $A$ und $B$ noch nicht viel ausgesagt.
Zunächst ist wegen vorhergehender Rechnung ersichtlich, dass es sich bei
$A$ und $B$ nicht um einzelne Koeffizienten handelt.
Stattdessen können die Koeffizienten für jedes $n \in \mathbb{N}$
unterschiedlich sein.
Die Lösung $X(x)$ wird nun umgeschrieben zu
\[
    X(x)
    =
    a_0
    +
    \sum_{n = 1}^{\infty} a_n\cos\left(\frac{n\pi}{l}x\right)
    +
    \sum_{n = 1}^{\infty} b_n\sin\left(\frac{n\pi}{l}x\right).
\]

Um eine eindeutige Lösung für $X(x)$ zu erhalten werden noch weitere
Bedingungen benötigt.
Diese sind die Startbedingungen oder $u(0, x) = X(x)$ für $t = 0$.
Es gilt also nun die Gleichung
\begin{equation}
    \label{sturmliouville:eq:example-fourier-initial-conditions}
    u(0, x)
    =
    a_0
    +
    \sum_{n = 1}^{\infty} a_n\cos\left(\frac{n\pi}{l}x\right)
    +
    \sum_{n = 1}^{\infty} b_n\sin\left(\frac{n\pi}{l}x\right)
\end{equation}
nach allen $a_n$ und $b_n$ aufzulösen.
Da aber $a_n$ und $b_n$ jeweils als Faktor zu einer trigonometrischen Funktion
gehört, von der wir wissen, dass sie orthogonal zu allen anderen
trigonometrischen Funktionen der Lösung ist, kann direkt das Skalarprodukt
verwendet werden um die Koeffizienten $a_n$ und $b_n$ zu bestimmen.
Es wird also die Tatsache ausgenutzt, dass die Gleichheit in
\eqref{sturmliouville:eq:example-fourier-initial-conditions} nach Anwendung des
Skalarproduktes immernoch gelten muss und dass das Skalaprodukt mit einer
Basisfunktion sämtliche Summanden auf der rechten Seite auslöscht.

Zur Berechnung von $a_m$ mit $ m \in \mathbb{N} $ wird beidseitig das
Skalarprodukt mit der Basisfunktion $ \cos\left(\frac{m \pi}{l}x\right)$
gebildet:
\begin{equation}
    \label{sturmliouville:eq:dot-product-cosine}
    \langle u(0, x), \cos\left(\frac{m \pi}{l}x\right) \rangle
    =
    \langle a_0
    +
    \sum_{n = 1}^{\infty} a_n\cos\left(\frac{n\pi}{l}x\right)
    +
    \sum_{n = 1}^{\infty} b_n\sin\left(\frac{n\pi}{l}x\right),
    \cos\left(\frac{m \pi}{l}x\right)\rangle
\end{equation}

Bevor diese Form in die Integralform umgeschrieben werden kann, muss überlegt
sein, welche Integralgrenzen zu verwenden sind.
In diesem Fall haben die $\sin$ und $\cos$ Terme beispielsweise keine ganze
Periode im Intervall $x \in [0, l]$ für ungerade $n$ und $m$.
Um die Skalarprodukte aber korrekt zu berechnen, muss über ein ganzzahliges
Vielfaches der Periode der trigonometrischen Funktionen integriert werden.
Dazu werden die Integralgrenzen $-l$ und $l$ verwendet und es werden ausserdem
neue Funktionen $\hat{u}_c(0, x)$ für die Berechnung mit Cosinus und
$\hat{u}_s(0, x)$ für die Berechnung mit Sinus angenomen, welche $u(0, t)$
gerade, respektive ungerade auf $[-l, l]$ fortsetzen:
\[
\begin{aligned}
    \hat{u}_c(0, x)
    &=
    \begin{cases}
        u(0, -x) & -l \leq x < 0
        \\
        u(0, x) & 0 \leq x \leq l
    \end{cases}
    \\
    \hat{u}_s(0, x)
    &=
    \begin{cases}
        -u(0, -x) & -l \leq x < 0
        \\
        u(0, x) & 0 \leq x \leq l
    \end{cases}.
\end{aligned}
\]

Die Konsequenz davon ist, dass nun das Resultat der Integrale um den Faktor zwei
skalliert wurde, also gilt nun
\[
\begin{aligned}
    \int_{-l}^{l}\hat{u}_c(0, x)\cos\left(\frac{m \pi}{l}x\right)dx
    &=
    2\int_{0}^{l}u(0, x)\cos\left(\frac{m \pi}{l}x\right)dx
    \\
    \int_{-l}^{l}\hat{u}_s(0, x)\sin\left(\frac{m \pi}{l}x\right)dx
    &=
    2\int_{0}^{l}u(0, x)\sin\left(\frac{m \pi}{l}x\right)dx.
\end{aligned}
\]

Zunächst wird nun das Skalaprodukt~\eqref{sturmliouville:eq:dot-product-cosine}
berechnet:
\[
\begin{aligned}
    \int_{-l}^{l}\hat{u}_c(0, x)\cos\left(\frac{m \pi}{l}x\right)dx
    =&
    \int_{-l}^{l} \left[a_0
    +
    \sum_{n = 1}^{\infty} a_n\cos\left(\frac{n\pi}{l}x\right)
    +
    \sum_{n = 1}^{\infty} b_n\sin\left(\frac{n\pi}{l}x\right)\right]
    \cos\left(\frac{m \pi}{l}x\right) dx
    \\
    2\int_{0}^{l}u(0, x)\cos\left(\frac{m \pi}{l}x\right)dx
    =&
    a_0 \int_{-l}^{l}\cos\left(\frac{m \pi}{l}x\right) dx
    +
    \sum_{n = 1}^{\infty}\left[a_n\int_{-l}^{l}\cos\left(\frac{n\pi}{l}x\right)
        \cos\left(\frac{m \pi}{l}x\right)dx\right]
    \\
    &+
    \sum_{n = 1}^{\infty}\left[b_n\int_{-l}^{l}\sin\left(\frac{n\pi}{l}x\right)
        \cos\left(\frac{m \pi}{l}x\right)dx\right].
\end{aligned}
\]

Betrachtet man nun die Summanden auf der rechten Seite stellt man fest, dass
nahezu alle Terme verschwinden, denn
\[
    \int_{-l}^{l}\cos\left(\frac{m \pi}{l}x\right) dx
    =
    0,
\]
da hier über ein ganzzahliges Vielfaches der Periode integriert wird,
\[
    \int_{-l}^{l}\cos\left(\frac{n\pi}{l}x\right)
    \cos\left(\frac{m \pi}{l}x\right)dx
    =
    0
\]
für $m\neq n$, da Cosinus-Funktionen mit verschiedenen Kreisfrequenzen
orthogonal zueinander stehen und
\[
    \int_{-l}^{l}\sin\left(\frac{n\pi}{l}x\right)
        \cos\left(\frac{m \pi}{l}x\right)dx
    =
    0
\]
da Sinus- und Cosinus-Funktionen ebenfalls orthogonal zueinander sind.

Es bleibt also lediglich der Summand für $a_m$ stehen, was die Gleichung zu
\[
    2\int_{0}^{l}u(0, x)\cos\left(\frac{m \pi}{l}x\right)dx
    =
    a_m\int_{-l}^{l}\cos^2\left(\frac{m\pi}{l}x\right)dx
\]
vereinfacht. Im nächsten Schritt wird nun das Integral auf der rechten Seite
berechnet und dann nach $a_m$ aufgelöst. Am einnfachsten geht dies, wenn zuerst
mit $u = \frac{m \pi}{l}x$ substituiert wird:
\[
    \begin{aligned}
    2\int_{0}^{l}u(0, x)\cos\left(\frac{m \pi}{l}x\right)dx
    &=
    a_m\frac{l}{m\pi}\int_{-m\pi}^{m\pi}\cos^2\left(u\right)du
    \\
    &=
    a_m\frac{l}{m\pi}\left[\frac{u}{2} + 
    \frac{\sin\left(2u\right)}{4}\right]_{u=-m\pi}^{m\pi}
    \\
    &=
    a_m\frac{l}{m\pi}\biggl(\frac{m\pi}{2} + 
    \underbrace{\frac{\sin\left(2m\pi\right)}{4}}_{\displaystyle = 0} - 
    \frac{-m\pi}{2} -
    \underbrace{\frac{\sin\left(-2m\pi\right)}{4}}_{\displaystyle = 0}\biggr)
    \\
    &=
    a_m l
    \\
    a_m
    &=
    \frac{2}{l} \int_{0}^{l}u(0, x)\cos\left(\frac{m \pi}{l}x\right)dx
    \end{aligned}
\]

Analog dazu kann durch das Bilden des Skalarproduktes mit 
$ \sin\left(\frac{m \pi}{l}x\right) $ gezeigt werden, dass
\[
    b_m
    =
    \frac{2}{l} \int_{0}^{l}u(0, x)\sin\left(\frac{m \pi}{l}x\right)dx
\]
gilt.

Etwas anders ist es allerdings bei $a_0$.
Wie der Name bereits suggeriert, handelt es sich hierbei um den Koeffizienten
zur Basisfunktion $\cos\left(\frac{0 \pi}{l}x\right)$ beziehungsweise der
konstanten Funktion $1$.
Um einen Ausdruck für $a_0$ zu erhalten, wird wiederum auf beiden Seiten
der Gleichung~\eqref{sturmliouville:eq:example-fourier-initial-conditions} das
Skalarprodukt mit der konstanten Basisfunktion $1$ gebildet:
\[
\begin{aligned}
    \int_{-l}^{l}\hat{u}_c(0, x)dx
    &=
    \int_{-l}^{l} a_0
    +
    \sum_{n = 1}^{\infty} a_n\cos\left(\frac{n\pi}{l}x\right)
    +
    \sum_{n = 1}^{\infty} b_n\sin\left(\frac{n\pi}{l}x\right)dx
    \\
    2\int_{0}^{l}u(0, x)dx
    &=
    a_0 \int_{-l}^{l}dx
    +
    \sum_{n = 1}^{\infty}\left[a_n\int_{-l}^{l}\cos\left(\frac{n\pi}{l}x\right)
        dx\right] +
    \sum_{n = 1}^{\infty}\left[b_n\int_{-l}^{l}\sin\left(\frac{n\pi}{l}x\right)
        dx\right].
\end{aligned}
\]

Hier fallen nun alle Terme, die $\sin$ oder $\cos$ beinhalten weg, da jeweils
über ein Vielfaches der Periode integriert wird.
Es bleibt also noch
\[
    2\int_{0}^{l}u(0, x)dx
    =
    a_0 \int_{-l}^{l}dx
\]
, was sich wie folgt nach $a_0$ auflösen lässt:
\[
\begin{aligned}
    2\int_{0}^{l}u(0, x)dx
    &=
    a_0 \int_{-l}^{l}dx
    \\
    &=
    a_0 \left[x\right]_{x=-l}^{l}
    \\
    &=
    a_0(l - (-l))
    \\
    &=
    a_0 \cdot 2l
    \\
    a_0
    &=
    \frac{1}{l} \int_{0}^{l}u(0, x)dx
\end{aligned}
\]

%
% Lösung von T(t) 
%

\subsubsection{Lösung der Differentialgleichung in $t$}
Zuletzt wird die zweite Gleichung der 
Separation~\eqref{sturmliouville:eq:example-fourier-separated-t} betrachtet.
Diese wird über das charakteristische Polynom
\[
    \lambda - \kappa \mu
    =
    0
\]
gelöst.

Es ist direkt ersichtlich, dass $\lambda = \kappa \mu$ gelten muss, was zur
Lösung
\[
    T(t)
    =
    e^{\kappa \mu t}
\]
führt und mit dem Resultat~\eqref{sturmliouville:eq:example-fourier-mu-solution}
\[
    T(t)
    =
    e^{-\frac{n^{2}\pi^{2}\kappa}{l^{2}}t}
\]
ergibt.

Dieses Resultat kann nun mit allen vorhergehenden Resultaten zusammengesetzt
werden um die vollständige Lösung für das Stab-Problem zu erhalten.

\subsubsection{Lösung für einen Stab mit Enden auf konstanter Temperatur}
\[
\begin{aligned}
    u(t,x)
    &=
    \sum_{n=1}^{\infty}b_{n}e^{-\frac{n^{2}\pi^{2}\kappa}{l^{2}}t}
    \sin\left(\frac{n\pi}{l}x\right)
    \\
    b_{n}
    &=
    \frac{2}{l}\int_{0}^{l}u(0,x)sin\left(\frac{n\pi}{l}x\right) dx
\end{aligned}
\]

\subsubsection{Lösung für einen Stab mit isolierten Enden}
\[
\begin{aligned}
    u(t,x)
    &=
    a_{0} + \sum_{n=1}^{\infty}a_{n}e^{-\frac{n^{2}\pi^{2}\kappa}{l^{2}}t}
    \cos\left(\frac{n\pi}{l}x\right)
    \\
    a_{0}
    &=
    \frac{1}{l}\int_{0}^{l}u(0,x) dx
    \\
    a_{n}
    &=
    \frac{2}{l}\int_{0}^{l}u(0,x)sin\left(\frac{n\pi}{l}x\right) dx
\end{aligned}
\]
