%
% waermeleitung_beispiel.tex -- Beispiel Wärmeleitung in homogenem Stab. 
%%%%%%%%%%%%%%%%%%%%%%%%%%% Erster Entwurf %%%%%%%%%%%%%%%%%%%%%%%%%%%%%%%%%%%%
%
% (c) 2020 Prof Dr Andreas Müller, Hochschule Rapperswil
%

\subsection{Wärmeleitung in einem Homogenen Stab}

In diesem Abschnitt betrachten wir das Problem der Wärmeleitung in einem
homogenen Stab und wie das Sturm-Liouville-Problem bei der Beschreibung dieses
physikalischen Phänomenes auftritt.

Zunächst wird ein eindimensionaler homogener Stab der Länge $l$ und
Wärmeleitkoeffizient $\kappa$. Somit ergibt sich für das Wärmeleitungsproblem
die partielle Differentialgleichung
\[
    \frac{\partial u}{\partial t} =
    \kappa \frac{\partial^{2}u}{{\partial x}^{2}}
\]
wobei der Stab in diesem Fall auf der X-Achse im Intervall $[0,l]$ liegt.

Da diese Differentialgleichung das Problem allgemein für einen homogenen
Stab beschreibt, werden zusätzliche Bedingungen benötigt, um beispielsweise
die Lösung für einen Stab zu finden, bei dem die Enden auf konstanter 
Tempreatur gehalten werden.

%%%%%%%%%%%%% Randbedingungen für Stab mit konstanten Endtemperaturen %%%%%%%%%

\subsubsection{Stab mit Enden auf konstanter Temperatur}

Die Enden des Stabes auf konstanter Temperatur zu halten bedeutet, dass die
Lösungsfunktion $u(t,x)$ bei $x = 0$ und $x = l$ nur die vorgegebene
Temperatur zurückgeben darf. Es folgen nun
\[
    u(t,0)
    =
    u(t,l)
    =
    0
\]
als Randbedingungen.

%%%%%%%%%%%%% Randbedingungen für Stab mit isolierten Enden %%%%%%%%%%%%%%%%%%%

\subsubsection{Stab mit isolierten Enden}

Bei isolierten Enden des Stabes können belibige Temperaturen für $x = 0$ und
$x = l$ auftreten. In diesem Fall nicht erlaubt ist es, dass Wärme vom Stab
an die Umgebung oder von der Umgebung an den Stab abgegeben wird.

Aus der Physik ist bekannt, dass Wärme immer von der höheren zur tieferen
Temperatur fliesst. Um Wärmefluss zu unterdrücken, muss also dafür gesorgt
werden, dass am Rand des Stabes keine Temperaturdifferenz existiert oder 
dass die partiellen Ableitungen von $u(t,x)$ nach $x$ bei $x = 0$ und $x = l$
verschwinden. Somit folgen
\[
    \frac{\partial}{\partial x} u(t, 0)
    =
    \frac{\partial}{\partial x} u(t, l)
    =
    0
\]
als Randbedingungen.

%%%%%%%%%%% Lösung der Differenzialgleichung %%%%%%%%%%%%%%%%%%%%%%%%%%%%%%%%%%

\subsubsection{Lösung der Differenzialgleichung}

% TODO: Referenz Separationsmethode
% TODO: Formeln sauber in Text einbinden.

Da die Lösungsfunktion von zwei Variablen abhängig ist, wird als Lösungsansatz
die Separationsmethode verwendet.

\[
    u(t,x)
    =
    T(t)X(x)
\]
Dieser Ausdruck wird in die ursprüngliche Differenzialgleichung eingesetzt:
\[
    T^{\prime}(t)X(x)
    =
    \kappa T(t)X^{\prime \prime}(x)
\]
Nun können alle von $t$ abhängigen Ausdrücke auf die linke Seite, sowie alle
von $x$ abhängigen Ausdrücke auf die rechte Seite gebracht werden und mittels
der neuen Variablen $\mu$ gekoppelt werden:
\[
    \frac{T^{\prime}(t)}{\kappa T(t)}
    =
    \frac{X^{\prime \prime}(x)}{X(x)}
    =
    \mu
\]
Durch die Einführung von $\mu$ kann das Problem nun in zwei separate
Differenzialgleichungen aufgeteilt werden:
\[
    T^{\prime}(t) - \kappa \mu T(t)
    =
    0
\]
\[
    X^{\prime \prime}(x) - \mu X(x)
    =
    0
\]
