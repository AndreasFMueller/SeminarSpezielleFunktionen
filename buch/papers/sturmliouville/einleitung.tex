%
% einleitung.tex -- Beispiel-File für die Einleitung
% Author: Réda Haddouche
%
% (c) 2020 Prof Dr Andreas Müller, Hochschule Rapperswil
%

\section{Was ist das Sturm-Liouville-Problem\label{sturmliouville:section:teil0}}
\rhead{Was ist das Sturm-Liouville-Problem}
Das Sturm-Liouville-Problem wurde benannt nach dem schweizerisch-französischen
Mathematiker und Physiker Jacques Charles Fran\c{c}ois Sturm und dem
französischen Mathematiker Joseph Liouville.
Gemeinsam haben sie in der mathematischen Physik die Sturm-Liouville-Theorie
entwickelt.
Diese gilt für die Lösung von gewöhnlichen Differentialgleichungen.
Handelt es sich um eine partielle
Differentialgleichung, kann man sie mittels Separation in
mehrere gewöhnliche Differentialgleichungen umwandeln.

\begin{definition}
	\index{Sturm-Liouville-Gleichung}%
Wenn die lineare homogene Differentialgleichung
\[
	\frac{d^2y}{dx^2} + a(x)\frac{dy}{dx} + b(x)y = 0
\]
als
\begin{equation}
	\label{sturmliouville:eq:sturm-liouville-equation}
	\frac{d}{dx} (p(x) \frac{dy}{dx}) + (q(x) +
	\lambda w(x)) y
	=
	0 
\end{equation}
geschrieben werden kann, dann wird die
Gleichung~\eqref{sturmliouville:eq:sturm-liouville-equation} als 
Sturm-Liouville-Gleichung bezeichnet.
\end{definition}
Alle homogenen linearen gewöhnlichen Differentialgleichungen 2. Ordnung können
in die Form der Gleichung \eqref{sturmliouville:eq:sturm-liouville-equation} 
umgewandelt werden.

Damit es sich um ein Sturm-Liouville-Problem handelt, benötigt es noch die
Randbedingungen, die im nächsten Unterkapitel behandelt wird.

\subsection{Randbedingungen
\label{sturmliouville:sub:was-ist-das-slp-randbedingungen}}
Geeignete Randbedingungen sind erforderlich, um die Lösungen einer
Differentialgleichung genau zu bestimmen.
Die Sturm-Liouville-Gleichung mit homogenen Randbedingungen des dritten Typs
\begin{equation}
	\begin{aligned}
		\label{sturmliouville:eq:randbedingungen}
		k_a y(a) + h_a p(a) y'(a) &= 0 \\
		k_b y(b) + h_b p(b) y'(b) &= 0
	\end{aligned}
\end{equation}
ist das klassische Sturm-Liouville-Problem.


\subsection{Koeffizientenfunktionen
\label{sturmliouville:sub:koeffizientenfunktionen}}
Die Funktionen $p(x)$, $q(x)$ und $w(x)$ werden als Koeffizientenfunktionen 
bezeichnet.
Diese Funktionen erhält man, indem man eine Differentialgleichung in die
Sturm-Liouville-Form bringt und dann die Koeffizientenfunktionen vergleicht.
Die Funktion $w(x)$ (manchmal auch $r(x)$ genannt) wird als Gewichtsfunktion
oder Dichtefunktion bezeichnet.
Die Eigenschaften der Koeffizientenfunktionen haben
einen großen Einfluss auf die Lösbarkeit des Sturm-Liouville-Problems und werden
im nächsten Abschnitt diskutiert.

%
%Kapitel mit "Das reguläre Sturm-Liouville-Problem"
%

\subsection{Das reguläre und singuläre Sturm-Liouville-Problem
\label{sturmliouville:sub:reguläre_sturm_liouville_problem}}
Damit es sich um ein reguläres Sturm-Liouville-Problem handelt, müssen einige
Bedingungen beachtet werden.
\begin{definition}
	\label{sturmliouville:def:reguläres_sturm-liouville-problem}
	\index{regläres Sturm-Liouville-Problem}
	Die Bedingungen für ein reguläres Sturm-Liouville-Problem sind:
	\begin{itemize}
		\item Die Funktionen $p(x), p'(x), q(x)$ und $w(x)$ müssen stetig und
		reell sein
		\item sowie in einem endlichen Intervall $[a,b]$ integrierbar
		sein.
		\item $p(x)$ und $w(x)$ sind $>0$.
		\item Es gelten die Randbedingungen 
		\eqref{sturmliouville:eq:randbedingungen}, wobei
		$|k_i|^2 + |h_i|^2\ne 0$ mit $i=a,b$.
	\end{itemize}
\end{definition}
Wird eine oder mehrere dieser Bedingungen nicht erfüllt, so handelt es sich um
ein singuläres Sturm-Liouville-Problem.

\begin{beispiel}
	Das Randwertproblem
	\begin{equation}
		\begin{aligned}
		x^2y'' + xy' + (\lambda^2x^2 - m^2)y &= 0 \qquad 0<x<a,\\
		y(a) &= 0
		\end{aligned}
	\end{equation}
	ist kein reguläres Sturm-Liouville-Problem.
	Wenn man die Gleichung in die Sturm-Liouville Form umformt, dann
	erhält man
	die Koeffizientenfunktionen $p(x) = w(x) = x$ und $q(x) = -m^2/x$.
	Schaut man jetzt die Bedingungen in
	Definition~\ref{sturmliouville:def:reguläres_sturm-liouville-problem} an und 
	vergleicht diese mit unseren Koeffizientenfunktionen, so erkennt man einige
	Probleme:
	\begin{itemize}
		\item $p(x)$ und $w(x)$ sind nicht positiv, wenn $x = 0$ ist.
		\item $q(x)$ ist nicht kontinuierlich, wenn $x = 0$ ist.
		\item Die Randbedingung bei $x = 0$ und $x = a$ fehlt.
	\end{itemize}
\end{beispiel}

Bei einem regulärem Problem, besteht die Lösung nur aus Eigenvektoren.
Handelt es sich um ein singuläres Problem, so besteht die Lösung im Allgemeinen
nicht mehr nur aus Eigenvektoren.
 
