%
% einleitung.tex -- Beispiel-File für die Einleitung
%
% (c) 2020 Prof Dr Andreas Müller, Hochschule Rapperswil
%

% TODO:
% order:
%   1. State goal of showing examples in intro
%   2. Show Sturm-Liouville form 
%   3. Explain boundary conditions as necessary in regards to examples
%      (make singular property brief)
%
% Remove Eigenvaluedecomposition -> is discussed in properties of solutions
% Check for readability 

\section{Was ist das Sturm-Liouville-Problem\label{sturmliouville:section:teil0}}
\rhead{Einleitung}
Das Sturm-Liouville-Problem wurde benannt nach dem schweizerisch-französischen
Mathematiker und Physiker Jacques Charles Fran\c{c}ois Sturm und dem
französischen Mathematiker Joseph Liouville.
Gemeinsam haben sie in der mathematischen Physik die Sturm-Liouville-Theorie
entwickelt und gilt für die Lösung von gewöhnlichen Differentialgleichungen,
jedoch verwendet man die Theorie öfters bei der Lösung von partiellen
Differentialgleichungen.
Normalerweise betrachtet man für das Strum-Liouville-Problem eine gewöhnliche
Differentialgleichung 2. Ordnung, und wenn es sich um eine partielle
Differentialgleichung handelt, kann man sie in mehrere gewöhnliche
Differentialgleichungen umwandeln. Wie z. B. den Separationsansatz, die
partielle Differentialgleichung mit mehreren Variablen.

\begin{definition}
	\index{Sturm-Liouville-Gleichung}%
Wenn die lineare homogene Differentialgleichung
\begin{equation}
	\frac{d^2y}{dx^2} + a(x)\frac{dy}{dx} + b(x)y = 0
\end{equation}
als
\begin{equation}
	\label{eq:sturm-liouville-equation}
	\frac{d}{dx} (p(x) \frac{dy}{dx}) + (q(x) +
	\lambda w(x)) y
	=
	0 
\end{equation}
geschrieben werden kann, dann wird diese Gleichung als Sturm-Liouville-Gleichung
bezeichnet.
\end{definition}
Alle homogenen linearen gewöhnlichen Differentialgleichungen 2. Ordnung können
in die Form der Gleichung \eqref{eq:sturm-liouville-equation} umgewandelt
werden.

Damit es sich um ein Sturm-Liouville-Problem handelt, benötigt es noch die Randbedingung, die im nächsten Unterkapitel behandelt wird. 

\subsection{Randbedingungen\label{sub:was-ist-das-slp-randbedingungen}}
Geeignete Randbedingungen sind erforderlich, um die Lösungen einer
Differentialgleichung genau zu bestimmen.
Die Sturm-Liouville-Gleichung mit homogenen Randbedingungen des dritten Typs
\begin{equation}
	\begin{aligned}
		\label{eq:randbedingungen}
		k_a y(a) + h_a p(a) y'(a) &= 0 \\
		k_b y(b) + h_b p(b) y'(b) &= 0.
	\end{aligned}
\end{equation}
ist das klassische Sturm-Liouville-Problem.


\subsection{Koeffizientenfunktionen\label{sub:koeffizientenfunktionen}}
Die Funktionen $p(x)$, $q(x)$ und $w(x)$ werden als Koeffizientenfunktionen mit
ihren freien Variablen $x$ bezeichnet.
Diese Funktionen erhält man, indem man eine Differentialgleichung in die Sturm-Liouville-Form bringt.
Die Funktion $w(x)$ (manchmal auch $r(x)$ genannt) wird als Gewichtsfunktion
oder Dichtefunktion bezeichnet.
Die Eigenschaften der Koeffizientenfunktionen sowie andere Bedingungen haben einen großen Einfluss auf die Lösbarkeit des Sturm-Liouville-Problems und werden im nächsten Kapitel diskutiert. 



%
%Kapitel mit "Das reguläre Sturm-Liouville-Problem"
%

\subsection{Das reguläre oder singuläre Sturm-Liouville-Problem
\label{sub:reguläre_sturm_liouville_problem}}
Damit es sich um ein reguläres Sturm-Liouville-Problem handelt, müssen einige
Bedingungen beachtet werden.
\begin{definition}
	\label{def:reguläres_sturm-liouville-problem}
	\index{regläres Sturm-Liouville-Problem}
	Die Bedingungen für ein reguläres Sturm-Liouville-Problem sind:
	\begin{itemize}
		\item Die Funktionen $p(x), p'(x), q(x)$ und $w(x)$ müssen stetig und
		reell sein.
		\item sowie müssen in einem endlichen Intervall $[a,b]$ integrierbar
		sein.
		\item $p(x)$ und $w(x)$ sind $>0$.
		\item Es gelten die Randbedingungen \eqref{eq:randbedingungen}, wobei
		$|k_i|^2 + |h_i|^2\ne 0$ mit $i=a,b$.
	\end{itemize}
\end{definition}
Werden diese Bedingungen nicht erfüllt, so handelt es sich um ein singuläres Sturm-Liouville-Problem.

\begin{beispiel}
	Das Randwertproblem
	\begin{equation}
		\begin{aligned}
		x^2y'' + xy' + (\lambda^2x^2 - m^2)y &= 0, 0<x<a,\\
		y(a) &= 0
		\end{aligned}
	\end{equation}
	ist kein reguläres Sturm-Liouville-Problem.
	Wenn man die Gleichung in die Sturm-Liouville Form umformen, dann ergeben
	die Koeffizientenfunktionen $p(x) = w(x) = x$ und $q(x) = -m^2/x$.
	Schaut man jetzt die Bedingungen im
	Kapitel~\ref{sub:reguläre_sturm_liouville_problem} an und vergleicht diese mit
	unseren Koeffizientenfunktionen, so erkennt man einige Probleme:
	\begin{itemize}
		\item $p(x)$ und $w(x)$ sind nicht positiv, wenn $x = 0$ ist.
		\item $q(x)$ ist nicht kontinuierlich, wenn $x = 0$ ist.
		\item Die Randbedingung bei $x = 0$ fehlt.
	\end{itemize}
\end{beispiel}

Verwendet man das reguläre Sturm-Liouville-Problem, obwohl eine oder beide
Bedingungen nicht erfüllt sind, dann ist es schwierig zu sagen, ob die Lösung
eindeutige Ergebnisse hat.
Es ist schwierig, Kriterien anzuwenden, da die Formulierungen z. B. in der
Lösungsfunktion liegen.
Ähnlich wie bei der Fourier-Reihe gegenüber der Fourier-Transformation gibt es
immer noch eine zugehörige Eigenfunktionsentwicklung, und zwar die
Integraltransformation sowie gibt es weiterhin verallgemeinerte Eigenfunktionen.
