%
% einleitung.tex -- Beispiel-File für die Einleitung
%
% (c) 2020 Prof Dr Andreas Müller, Hochschule Rapperswil
%
\section{Was ist das Sturm-Liouville-Problem\label{sturmliouville:section:teil0}}
\rhead{Einleitung}
Das Sturm-Liouville-Problem wurde benannt nach dem schweizerisch-französischen Mathematiker und Physiker Jacques Charles Fran\c{c}ois Sturm und dem französischen Mathematiker Joseph Liouville.
Gemeinsam haben sie in der mathematischen Physik die Sturm-Liouville-Theorie entwickelt und gilt für die Lösung von gewöhnlichen Differentialgleichungen, jedoch verwendet man die Theorie öfters bei der Lösung von partiellen Differentialgleichungen.
Normalerweise betrachtet man für das Strum-Liouville-Problem eine gewöhnliche Differentialgleichung 2. Ordnung, und wenn es sich um eine partielle Differentialgleichung handelt, kann man sie in mehrere gewöhnliche Differentialgleichungen umwandeln. Wie z. B. den Separationsansatz, die partielle Differentialgleichung mit mehreren Variablen.

\begin{definition}
	\index{Sturm-Liouville-Gleichung}%
Wenn die lineare homogene Differentialgleichung
\begin{equation}
	\frac{d^2y}{dx^2} + a(x)\frac{dy}{dx} + b(x)y = 0
\end{equation}
als
\begin{equation}
	\label{eq:sturm-liouville-equation}
	\frac{d}{dx}\lbrack p(x) \frac{dy}{dx} \rbrack + \lbrack q(x) + \lambda w(x) \rbrack y = 0 
\end{equation}
geschrieben werden kann, dann wird diese Gleichung als Sturm-Liouville-Gleichung bezeichnet.
\end{definition}
Alle homogene 2. Ordnung lineare gewöhnliche Differentialgleichungen können in die Form der Gleichung \ref{eq:sturm-liouville-equation} umgeformt werden.

\subsection{Randbedingungen\label{sub:was-ist-das-slp-randbedingungen}}
Wenn von der Funktion $y(x)$ die Werte $x$ des jeweiligen Randes des Definitionsbereiches anzunehmen sind, also
\begin{equation}
	y(a) = y(b) = 0,
\end{equation}
so spricht man von einer Dirichlet-Randbedingung\footnote{Die Dirichlet-Randbedingung oder auch Randbedingung des ersten Typs genannt ist nach dem deutschen Mathematiker Peter Gstav Lejeune Dirichlet benannt. Sie findet Anwendung auf gewöhnliche oder patielle Differentialgleichungen und gibt mit der Bedingung die Werte an, die für die abgeleitete Lösung innerhalb der Domänengrenze gelten.}, und von einer Neumann-Randbedingung\footnote{Die Neumann-Randbedingung oder auch Randbedingung des zweiten Typs genannt, ist nach dem deutschen Mathematiker Carl Neumann benannt. Sie legt die Werte fest, die eine Lösung entlang der Domänengrenze annehmen muss, wenn eine gewöhnliche oder partielle Differentialgleichung gestellt wird.} spricht man, wenn
\begin{equation}
	y'(a) = y'(b) = 0
\end{equation}
ergibt.

Die Sturm-Liouville-Theorie besagt, dass, wenn man die Sturm-Liouville-Gleichung mit den homogenen Randbedingungen des dritten Typs\footnote{Die Randbedingung des dritten Typs, oder Robin-Randbedingungen (benannt nach dem französischen mathematischen Analytiker und angewandten Mathematiker Victor Gustave Robin), wird genannt, wenn sie einer gewöhnlichen oder partiellen Differentialgleichung auferlegt wird, so sind die Spezifikationen einer Linearkombination der Werte einer Funktion sowie die Werte ihrer Ableitung am Rande des Bereichs}
\begin{equation}
\begin{aligned}
	\label{eq:randbedingungen}
	k_a y(a) + h_a p(a) y'(a) &= 0 \\
	k_b y(b) + h_b p(b) y'(b) &= 0
\end{aligned}
\end{equation}
kombiniert, dann bekommt man das klassische Sturm-Liouville-Problem.

\subsection{Eigenwertproblem}
Die Gleichungen \ref{eq:sturm-liouville-equation} hat die Form eines Eigenwertproblems
Wenn bei der Sturm-Liouville-Gleichung \ref{eq:sturm-liouville-equation} alles  konstant bleibt, aber der Wert von $\lambda$ sich ändert, erhält man eine andere Eigenfunktion, weil man eine andere gewöhnliche Differentialgleichung löst;
der Parameter $\lambda$ wird als Eigenwert bezeichnet.
Es ist genau das gleiche Prinzip wie bei den Matrizen, andere Eigenwerte ergeben andere Eigenvektoren.
Es besteht eine Korrespondenz zwischen den Eigenwerten und den Eigenvektoren.
Das gleiche gilt auch beim Sturm-Liouville-Problem, und zwar
\begin{equation}
	\lambda \overset{Korrespondenz}\leftrightarrow y.
\end{equation}

Die Theorie besagt, wenn $y_m$, $y_n$ Eigenfuktionen des Sturm-Liouville-Problems sind, die verschiedene Eigenwerte $\lambda_m$, $\lambda_n$ ($\lambda_m \neq \lambda_n$) entsprechen, so sind $y_m$, $y_n$ orthogonal zu y -
dies gilt für das Intervall (a,b).
Somit ergibt die Gleichung
\begin{equation}
	\label{eq:skalar-sturm-liouville}
	\int_{a}^{b} w(x)y_m y_n = 0.
\end{equation}

\subsection{Koeffizientenfunktionen}
Die Funktionen $p(x)$, $q(x)$ und $w(x)$ werden als Koeffizientenfunktionen mit ihren freien Variablen $x$ bezeichnet.
Die Funktion $w(x)$ (manchmal auch $r(x)$ genannt) wird als Gewichtsfunktion oder Dichtefunktion bezeichnet.
Es gibt zwei verschiedene Sturm-Liouville-Probleme: das reguläre Sturm-Liouville-Problem und das singuläre Sturm-Liouville-Problem. 
Die Funktionen für das reguläre und das singuläre Sturm-Liouville-Problem sind nicht dieselben.

%
%Kapitel mit "Das reguläre Sturm-Liouville-Problem"
%

\subsection{Das reguläre Sturm-Liouville-Problem\label{sub:reguläre_sturm_liouville_problem}}
Damit es sich um ein reguläres Sturm-Liouville-Problem handelt, müssen einige Bedingungen beachtet werden.
\begin{definition}
	\label{def:reguläres_sturm-liouville-problem}
	\index{regläres Sturm-Liouville-Problem}
	Die Bedingungen für ein reguläres Sturm-Liouville-Problem sind:
	\begin{itemize}
		\item Die Funktionen $p(x), p'(x), q(x)$ und $w(x)$ müssen stetig und reell sein.
		\item sowie müssen in einem endlichen Intervall $[a,b]$ integrierbar sein.
		\item $p(x)$ und $w(x)$ sind $>0$.
		\item Es gelten die Randbedingungen \ref{eq:randbedingungen}, wobei $|k_i|^2 + |h_i|^2\ne 0$ mit $i=a,b$.
	\end{itemize}
\end{definition}
Bei einem regulären Sturm-Liouville-Problem geht es darum, wichtige Eigenschaften der Eigenfunktionen beschreiben zu können, ohne sie genau zu kennen.


%
%Kapitel mit "Das singuläre Sturm-Liouville-Problem"
%


\subsection{Das singuläre Sturm-Liouville-Problem\label{sub:singuläre_sturm_liouville_problem}}
Von einem singulären Sturm-Liouville-Problem spricht man, wenn die Bedingungen des regulärem Problem nicht erfüllt sind.
\begin{definition}
	\label{def:singulär_sturm-liouville-problem}
	\index{singuläres Sturm-Liouville-Problem}
Es handelt sich um ein singuläres Sturm-Liouville-Problem, wenn:
	\begin{itemize}
		\item wenn sein Definitionsbereich auf dem Intervall $[ \ a,b] \ $ unbeschränkt ist oder
		\item wenn die Koeffizienten an den Randpunkten Singularitäten haben.
	\end{itemize}
\end{definition}
Allerdings kann nur eine der Bedingungen nicht erfüllt sein, so dass es sich bereits um ein singuläres Sturm-Liouville-Problem handelt.

\begin{beispiel}
	Das Randwertproblem
	\begin{equation}
		\begin{aligned}
		x^2y'' + xy' + (\lambda^2x^2 - m^2)y &= 0, 0<x<a,\\
		y(a) &= 0
		\end{aligned}
	\end{equation}
	ist kein reguläres Sturm-Liouville-Problem.
	Wenn man die Gleichung in die Sturm-Liouville Form umformen, dann ergeben die Koeffizientenfunktionen $p(x) = w(x) = x$ und $q(x) = -m^2/x$.
	Schaut man jetzt die Bedingungen im Kapitel \ref{sub:reguläre_sturm_liouville_problem} an und vergleicht diese unseren Koeffizientenfunktionen, so erkennt man einige Probleme:
	\begin{itemize}
		\item $p(x)$ und $w(x)$ sind nicht positiv, wenn $x = 0$ ist.
		\item $q(x)$ ist nicht kontinuierlich, wenn $x = 0$ ist.
		\item Die Randbedingung bei $x = 0$ fehlt.
	\end{itemize}
\end{beispiel}

Verwendet man das reguläre Sturm-Liouville-Problem, obwohl eine oder beide Bedingungen nicht erfüllt sind, dann ist es schwierig zu sagen, ob die Lösung eindeutige Ergebnisse hat.
Es ist schwierig, Kriterien anzuwenden, da die Formulierungen z. B. in der Lösungsfunktion liegen.
Ähnlich wie bei der Fourier-Reihe gegenüber der Fourier-Transformation gibt es immer noch eine zugehörige Eigenfunktionsentwicklung, und zwar die Integraltransformation sowie gibt es weiterhin verallgemeinerte Eigenfunktionen.

 



