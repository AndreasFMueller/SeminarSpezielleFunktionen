%
% eigenschaften.tex -- Eigenschaften der Lösungen
% Author: Erik Löffler
%
% (c) 2020 Prof Dr Andreas Müller, Hochschule Rapperswil
%

\section{Eigenschaften von Lösungen
\label{sturmliouville:sec:solution-properties}}
\rhead{Eigenschaften von Lösungen}

Im Weiteren werden nun die Eigenschaften der Lösung eines
Sturm-Liouville-Problems diskutiert.
Im Wesentlichen wird darauf eingegangen, wie die Orthogonalität der Lösungen
zustande kommt, damit diese später in den Beispielen verwendet werden kann.
Dazu wird zunächst das Eigenwertproblem für Matrizen wiederholt und angeschaut
unter welchen Voraussetzungen die Lösungen dieses Problems orthogonal sind.
Dann wird gezeigt, dass das Sturm-Liouville-Problem auch ein Eigenwertproblem
dieser Art ist und es wird auf au die Orthogonalität der Lösungsfunktionen
geschlossen.

\subsection{Eigenwertprobleme mit symmetrischen Matrizen
\label{sturmliouville:sec:eigenvalue-problem-matrix}}

% TODO: intro

Angenommen es sei eine reelle, symmetrische $n \times n$-Matrix $A$ gegeben.
Dass $A$ symmetrisch ist, bedeutet, dass
\[
    \langle Av, w \rangle
    =
    \langle v, Aw \rangle
    \qquad
    v, w \in \mathbb{R}^n
\]
erfüllt ist.

Für reelle, symmetrische Matrizen zeigt dies auch direkt, dass die Matrix
selbstadjungiert ist.
Das ist wichtig, da der Spektralsatz~\cite{sturmliouville:spektralsatz-wiki}
für selbstadjungierte Matrizen formuliert ist. Dieser sagt nun aus, dass die
Matrix $A$ diagonalisierbar ist.
In anderen Worten bilden die Eigenvektoren $v_i \in \mathbb{R}^n$ des 
Eigenwertproblems
\[
    A v_i
    =
    \lambda_i v_i
    \qquad \lambda_i \in \mathbb{R}
\]
eine Orthogonalbasis.

\subsection{Das Sturm-Liouville-Problem als Eigenwertproblem}

In Kapitel~\ref{buch:integrale:subsection:sturm-liouville-problem} wurde bereits
der Operator
\[
    L
    =
    \frac{1}{w(x)}\left( -\frac{d}{dx}p(x) \frac{d}{dx} + q(x)\right)
\]
eingeführt.
Dieser wird nun verwendet um die Differenzialgleichung 
\[
    (p(x)y'(x))' + q(x)y(x)
    =
    \lambda w(x) y(x)
\]
in das Eigenwertproblem
\begin{equation}
    \label{sturmliouville:eq:eigenvalue-problem}
    L y
    =
    \lambda y.
\end{equation}
umzuschreiben.

\subsection{Orthogonalität der Lösungsfunktionen}

Nun wird das Eigenwertproblem~\eqref{sturmliouville:eq:eigenvalue-problem} näher
angeschaut.
Um auf die Orthogonalität der Lösungsfunktion zu schliessen, wird dafür der
Operator $L$ genauer betrachtet.
Analog zur Matrix $A$ aus 
Abschnitt~\ref{sturmliouville:sec:eigenvalue-problem-matrix} kann auch für
$L$ gezeigt werden, dass dieser Operator selbstadjungiert ist.

Dazu wird das modifizierte Skalarprodukt
\begin{equation}
    \label{sturmliouville:eq:modified-dot-product}
    \langle f, g \rangle_w
    =
    \int_a^b f(x)g(x)w(x)\,dx
\end{equation}
aus Kapitel~\ref{buch:integrale:subsection:sturm-liouville-problem} verwendet,
welches auch die Gewichtsfunktion $w(x)$ berücksichtigt.
Damit $L$ bezüglich dieses Skalarproduktes selbstadjungiert ist, muss also
\[
    \langle L u, v\rangle_w
    =
    \langle u, L v\rangle_w
\]
gelten.

Wie in Kapitel~\ref{buch:integrale:subsection:sturm-liouville-problem} bereits
gezeigt, ist dies durch die
Randbedingungen~\eqref{sturmliouville:eq:randbedingungen} des
Sturm-Liouville-Problems sicher gestellt.

Um nun über den Spektralsatz~\cite{sturmliouville:spektralsatz-wiki} auf die
Orthogonalität der Lösungsfunktion $y$ zu schliessen, muss der Operator $L$ ein
sogenannter ''kompakter Operator'' sein.
Bei einem regulären Sturm-Liouville-Problem ist diese Eigenschaft für $L$
gegeben und wird im Weiteren nicht näher diskutiert.

Es kann nun also dank dem Spektralsatz darauf geschlossen werden, dass die
Lösungsfunktion $y$ eines regulären Sturm-Liouville-Problems eine
Linearkombination aus orthogonalen Basisfunktionen sein muss.
