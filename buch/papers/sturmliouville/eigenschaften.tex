%
% eigenschaften.tex -- Eigenschaften der Lösungen
% Author: Erik Löffler
%
% (c) 2020 Prof Dr Andreas Müller, Hochschule Rapperswil
%
\section{Eigenschaften von Lösungen
\label{sturmliouville:section:solution-properties}}
\rhead{Eigenschaften von Lösungen}

Im weiteren werden nun die Eigenschaften der Lösungen eines
Sturm-Liouville-Problems diskutiert und aufgezeigt, wie diese Eigenschaften
zustande kommen.

Dazu wird der Operator $L_0$ welcher bereits in 
Kapitel~\ref{buch:integrale:subsection:sturm-liouville-problem} betrachtet
wurde, noch etwas genauer angeschaut.
Es wird also im Folgenden
\[
    L_0
    =
    -\frac{d}{dx}p(x)\frac{d}{dx}
\]
zusammen mit den Randbedingungen
\[
    \begin{aligned}
        k_a y(a) + h_a p(a) y'(a) &= 0 \\
        k_b y(b) + h_b p(b) y'(b) &= 0
    \end{aligned}
\]
verwendet.
Wie im Kapitel~\ref{buch:integrale:subsection:sturm-liouville-problem} bereits 
gezeigt, resultieren die Randbedingungen aus der Anforderung den Operator $L_0$
selbsadjungiert zu machen.
Es wurde allerdings noch nicht darauf eingegangen, welche Eigenschaften dies
für die Lösungen des Sturm-Liouville-Problems zur Folge hat.

\subsubsection{Exkurs zum Spektralsatz}

Um zu verstehen welche Eigenschaften der selbstadjungierte Operator $L_0$ in 
den Lösungen hervorbringt, wird der Spektralsatz benötigt.

Dieser wird in der linearen Algebra oft verwendet um zu zeigen, dass eine Matrix
diagonalisierbar ist, beziehungsweise dass eine Orthonormalbasis existiert.

Im Fall einer gegebenen $n\times n$-Matrix $A$ mit reellen Einträgen wird dazu 
zunächst gezeigt, dass $A$ selbstadjungiert ist, also dass
\[
    \langle Av, w \rangle
    =
    \langle v, Aw \rangle
\]
für $ v, w \in \mathbb{R}^n$ gilt.
Ist dies der Fall, kann die Aussage des Spektralsatzes
\cite{sturmliouville:spektralsatz-wiki} verwended werden.
Daraus folgt dann, dass eine Orthonormalbasis aus Eigenvektoren existiert,
wenn $A$ nur Eigenwerte aus $\mathbb{R}$ besitzt.

Dies ist allerdings nicht die Einzige Version des Spektralsatzes.
Unter anderen gibt es den Spektralsatz für kompakte Operatoren
\cite{sturmliouville:spektralsatz-wiki}, welcher für das
Sturm-Liouville-Problem von Bedeutung ist.
Welche Voraussetzungen erfüllt sein müssen, um diese Version des
Satzes verwenden zu können, wird hier aber nicht diskutiert und kann bei den
Beispielen in diesem Kapitel als gegeben betrachtet werden.
Grundsätzlich ist die Aussage in dieser Version dieselbe, wie bei den Matrizen,
also dass für ein Operator eine Orthonormalbasis aus Eigenvektoren existiert,
falls er selbstadjungiert ist.

\subsubsection{Anwendung des Spektralsatzes auf $L_0$}

Der Spektralsatz besagt also, dass, weil $L_0$ selbstadjungiert ist, eine
Orthonormalbasis aus Eigenvektoren existiert.
Genauer bedeutet dies, dass alle Eigenvektoren, beziehungsweise alle Lösungen
des Sturm-Liouville-Problems orthogonal zueinander sind bezüglich des
Skalarprodukts, in dem $L_0$ selbstadjungiert ist.

Erfüllt also eine Differenzialgleichung die in
Abschnitt~\ref{sturmliouville:section:teil0} präsentierten Eigenschaften und 
erfüllen die Randbedingungen der Differentialgleichung die Randbedingungen
des Sturm-Liouville-Problems, kann bereits geschlossen werden, dass die
Lösungsfunktion des Problems eine Linearkombination aus orthogonalen
Basisfunktionen ist.