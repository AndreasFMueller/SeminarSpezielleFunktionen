% !TeX root = ../../buch.tex
%
% main.tex -- Paper zum Thema <sturmliouville>
%
% (c) 2020 Hochschule Rapperswil
%
\chapter{Sturm-Liouville-Problem\label{chapter:sturmliouville}}
\lhead{Sturm-Liouville-Problem}
\begin{refsection}
\chapterauthor{Réda Haddouche und Erik Löffler}



\section{Einleitung}
\rhead{Einleitung}
Heutzutage ist die Navigation ein Teil des Lebens. 
\index{Navigation}%
Man sendet dem Kollegen seinen eigenen Standort, um sich das ewige Erklären zu sparen oder gibt die Adresse des Ziels ein, damit man seinen Aufenthaltsort zum Beispiel auf einer riesigen Wiese am See findet. 
Dies wird durch Technologien wie Funknavigation, welches ein auf Laufzeitmessung beruhendes Hyperbelverfahren mit Langwellen ist, oder die verbreitete Satellitennavigation, welche vier Satelliten für eine Messung zur Standortbestimmung nutzt.
\index{Funknavigation}%
\index{GPS}%
Vor all diesen technologischen Fortschritten gab es lediglich die Astronavigation, welche heute noch auf Schiffen verwendet wird im Falle eines Stromausfalls. 
Aber wie funktioniert die Navigation mit den Sternen? Welche Hilfsmittel benötigt man, welche Rolle spielt die Mathematik und weshalb kann die Erde nicht flach sein? 
In diesem Kapitel werden genau diese Fragen mithilfe des nautischen Dreiecks, der sphärischen Trigonometrie und einigen Hilfsmitteln und Messgeräten beantwortet.
\index{sphärische Trigonometrie}%

%einleitung "was ist das sturm-liouville-problem"
%
% eigenschaften.tex -- Eigenschaften der Lösungen
% Author: Erik Löffler
%
% (c) 2020 Prof Dr Andreas Müller, Hochschule Rapperswil
%

% TODO:
%  state goal
%  use only what is necessary
%  make sure it is easy enough to understand (sentences as shot as possible)
%    -> Eigenvalue problem with matrices only
%    -> prepare reader for following examples
%
% order:
%  1. Eigenvalue problems with matrices
%  2. Sturm-Liouville is an Eigenvalue problem
%  3. Sturm-Liouville operator (self-adjacent)
%  4. Spectral theorem (brief)
%  5. Base of orthonormal functions

\section{Eigenschaften von Lösungen
\label{sturmliouville:sec:solution-properties}}
\rhead{Eigenschaften von Lösungen}

Im weiteren werden nun die Eigenschaften der Lösung eines
Sturm-Liouville-Problems diskutiert.
Im wesentlichen wird darauf eingegangen, wie die Orthogonalität der Lösungen
zustande kommt, damit diese später in den Beispielen verwendet werden kann.
Dazu wird zunächst das Eigenwertproblem für Matrizen wiederholt und angeschaut
unter welchen Voraussetzungen die Lösungen dieses Problems orthogonal sind.
Dann wird gezeigt, dass das Sturm-Liouville-Problem auch ein Eigenwertproblem
dieser Art ist und es wird auf au die Orthogonalität der Lösungsfunktionen
geschlossen.

\subsection{Eigenwertprobleme mit symmetrischen Matrizen
\label{sturmliouville:sec:eigenvalue-problem-matrix}}

% TODO: intro

Angenomen es sei eine reelle, symmetrische $n \times n$-Matrix $A$ gegeben.
Dass $A$ symmetrisch ist, bedeutet, dass
\[
    \langle Av, w \rangle
    =
    \langle v, Aw \rangle
    \qquad
    v, w \in \mathbb{R}^n
\]
erfüllt ist.

Für reelle, symmetrische Matrizen zeigt dies auch direkt, dass die Matrix
selbstadjungiert ist.
Das ist wichtig, da der Spektralsatz~\cite{sturmliouville:spektralsatz-wiki}
für selbstadjungierte Matrizen formuliert ist. Dieser sagt nun aus, dass die
Matrix $A$ diagonalisierbar ist.
In anderen Worten bilden die Eigenvektoren $v_i \in \mathbb{R}^n$ des 
Eigenwertproblems
\[
    A v_i
    =
    \lambda_i v_i
    \qquad \lambda_i \in \mathbb{R}
\]
eine Orthogonalbasis.

\subsection{Das Sturm-Liouville-Problem als Eigenwertproblem}

In Kapitel~\ref{buch:integrale:subsection:sturm-liouville-problem} wurde bereits
der Operator
\[
    L
    =
    \frac{1}{w(x)}\left( -\frac{d}{dx}p(x) \frac{d}{dx} + q(x)\right)
\]
eingeführt.
Dieser wird nun verwendet um die Differenzialgleichung 
\[
    (p(x)y'(x))' + q(x)y(x)
    =
    \lambda w(x) y(x)
\]
in das Eigenwertproblem
\begin{equation}
    \label{sturmliouville:eq:eigenvalue-problem}
    L y
    =
    \lambda y.
\end{equation}
umzuschreiben.

\subsection{Orthogonalität der Lösungsfunktionen}

Nun wird das Eigenwertproblem~\eqref{sturmliouville:eq:eigenvalue-problem} näher
angeschaut.
Um auf die Orthogonalität der Lösungsfunktion zu schliessen, wird dafür der
Operator $L$ genauer betrachtet.
Analog zur Matrix $A$ aus 
Abschnitt~\ref{sturmliouville:sec:eigenvalue-problem-matrix} kann auch für
$L$ gezeigt werden, dass dieser Operator selbstadjungiert ist, also dass
\[
    \langle L v, w\rangle
    =
    \langle v, L w\rangle
\]
gilt.
Wie in Kapitel~\ref{buch:integrale:subsection:sturm-liouville-problem} bereits
gezeigt, ist dies durch die
Randbedingungen~\eqref{sturmliouville:eq:randbedingungen} des
Sturm-Liouville-Problems sicher gestellt.

Um nun über den Spektralsatz~\cite{sturmliouville:spektralsatz-wiki} auf die
Orthogonalität der Lösungsfunktion $y$ zu schliessen, muss der Operator $L$ ein
sogenannter ''kompakter Operator'' sein.
Bei einem regulären Sturm-Liouville-Problem ist diese Eigenschaft für $L$
gegeben und wird im Weiteren nicht näher diskutiert.

Es kann nun also dank dem Spektralsatz darauf geschlossen werden, dass die
Lösungsfunktion $y$ eises regulären Sturm-Liouville-Problems eine
Linearkombination aus orthogonalen Basisfunktionen sein muss.

%%%%%%%%%%%%%%%%%%%%%%%%%%%%%% OLD section %%%%%%%%%%%%%%%%%%%%%%%%%%%%%%%%%%%%

\iffalse

\section{OLD: Eigenschaften von Lösungen
%\label{sturmliouville:section:solution-properties}
}
\rhead{Eigenschaften von Lösungen}

Im weiteren werden nun die Eigenschaften der Lösungen eines
Sturm-Liouville-Problems diskutiert und aufgezeigt, wie diese Eigenschaften
zustande kommen.

Dazu wird der Operator $L_0$ welcher bereits in
Kapitel~\ref{buch:integrale:subsection:sturm-liouville-problem} betrachtet
wurde, noch etwas genauer angeschaut.
Es wird also im Folgenden
\[
    L_0
    =
    -\frac{d}{dx}p(x)\frac{d}{dx}
\]
zusammen mit den Randbedingungen
\[
    \begin{aligned}
        k_a y(a) + h_a p(a) y'(a) &= 0 \\
        k_b y(b) + h_b p(b) y'(b) &= 0
    \end{aligned}
\]
verwendet.
Wie im Kapitel~\ref{buch:integrale:subsection:sturm-liouville-problem} bereits 
gezeigt, resultieren die Randbedingungen aus der Anforderung den Operator $L_0$
selbsadjungiert zu machen.
Es wurde allerdings noch nicht darauf eingegangen, welche Eigenschaften dies
für die Lösungen des Sturm-Liouville-Problems zur Folge hat.

\subsubsection{Exkurs zum Spektralsatz}

Um zu verstehen welche Eigenschaften der selbstadjungierte Operator $L_0$ in 
den Lösungen hervorbringt, wird der Spektralsatz benötigt.

Dieser wird in der linearen Algebra oft verwendet um zu zeigen, dass eine Matrix
diagonalisierbar ist, beziehungsweise dass eine Orthonormalbasis existiert.

Im Fall einer gegebenen $n\times n$-Matrix $A$ mit reellen Einträgen wird dazu 
zunächst gezeigt, dass $A$ selbstadjungiert ist, also dass
\[
    \langle Av, w \rangle
    =
    \langle v, Aw \rangle
\]
für $ v, w \in \mathbb{R}^n$ gilt.
Ist dies der Fall, kann die Aussage des Spektralsatzes
\cite{sturmliouville:spektralsatz-wiki} verwended werden.
Daraus folgt dann, dass eine Orthonormalbasis aus Eigenvektoren existiert,
wenn $A$ nur Eigenwerte aus $\mathbb{R}$ besitzt.

Dies ist allerdings nicht die Einzige Version des Spektralsatzes.
Unter anderen gibt es den Spektralsatz für kompakte Operatoren
\cite{sturmliouville:spektralsatz-wiki}, welcher für das
Sturm-Liouville-Problem von Bedeutung ist.
Welche Voraussetzungen erfüllt sein müssen, um diese Version des
Satzes verwenden zu können, wird hier aber nicht diskutiert und kann bei den
Beispielen in diesem Kapitel als gegeben betrachtet werden.
Grundsätzlich ist die Aussage in dieser Version dieselbe, wie bei den Matrizen,
also dass für ein Operator eine Orthonormalbasis aus Eigenvektoren existiert,
falls er selbstadjungiert ist.

\subsubsection{Anwendung des Spektralsatzes auf $L_0$}

Der Spektralsatz besagt also, dass, weil $L_0$ selbstadjungiert ist, eine
Orthonormalbasis aus Eigenvektoren existiert.
Genauer bedeutet dies, dass alle Eigenvektoren, beziehungsweise alle Lösungen
des Sturm-Liouville-Problems orthogonal zueinander sind bezüglich des
Skalarprodukts, in dem $L_0$ selbstadjungiert ist.

Erfüllt also eine Differenzialgleichung die in
Abschnitt~\ref{sturmliouville:section:teil0} präsentierten Eigenschaften und
erfüllen die Randbedingungen der Differentialgleichung die Randbedingungen
des Sturm-Liouville-Problems, kann bereits geschlossen werden, dass die
Lösungsfunktion des Problems eine Linearkombination aus orthogonalen
Basisfunktionen ist.

\fi

%Eigenschaften von Lösungen eines solchen Problems
%
% teil2.tex -- Beispiel-File für teil2 
%
% (c) 2020 Prof Dr Andreas Müller, Hochschule Rapperswil
%
\section{Beispiele 
\label{sturmliouville:section:teil2}}
\rhead{Beispiele}


%Beispiele sind: Wärmeleitung in einem Stab, Tschebyscheff-Polynome

\printbibliography[heading=subbibliography]
\end{refsection}
