% !TeX root = ../../buch.tex
%
% main.tex -- Paper zum Thema <sturmliouville>
%
% (c) 2020 Hochschule Rapperswil
%
\chapter{Sturm-Liouville-Problem\label{chapter:sturmliouville}}
\lhead{Sturm-Liouville-Problem}
\begin{refsection}
\chapterauthor{Réda Haddouche und Erik Löffler}

In diesem Kapitel wird zunächst nochmals ein Überblick über das
Sturm-Liouville-Problem und dessen Randbedingungen gegeben.
Dann wird ein Zusammenhang zwischen reellen symmetrischen Matrizen und
dem Sturm-Liouville-Operator $L$ hergestellt, um auf die Orthogonalität der
Lösungsfunktionen zu schliessen.
Zuletzt wird anhand von zwei Beispielen gezeigt, dass durch das
Sturm-Liouville-Problem die Eigenschaften der Lösungen bereits vor dem
vollständingen Lösen der Beispiele bekannt sind.

%einleitung "was ist das sturm-liouville-problem"


\section{Einleitung}
\rhead{Einleitung}
Heutzutage ist die Navigation ein Teil des Lebens. 
\index{Navigation}%
Man sendet dem Kollegen seinen eigenen Standort, um sich das ewige Erklären zu sparen oder gibt die Adresse des Ziels ein, damit man seinen Aufenthaltsort zum Beispiel auf einer riesigen Wiese am See findet. 
Dies wird durch Technologien wie Funknavigation, welches ein auf Laufzeitmessung beruhendes Hyperbelverfahren mit Langwellen ist, oder die verbreitete Satellitennavigation, welche vier Satelliten für eine Messung zur Standortbestimmung nutzt.
\index{Funknavigation}%
\index{GPS}%
Vor all diesen technologischen Fortschritten gab es lediglich die Astronavigation, welche heute noch auf Schiffen verwendet wird im Falle eines Stromausfalls. 
Aber wie funktioniert die Navigation mit den Sternen? Welche Hilfsmittel benötigt man, welche Rolle spielt die Mathematik und weshalb kann die Erde nicht flach sein? 
In diesem Kapitel werden genau diese Fragen mithilfe des nautischen Dreiecks, der sphärischen Trigonometrie und einigen Hilfsmitteln und Messgeräten beantwortet.
\index{sphärische Trigonometrie}%


%Eigenschaften von Lösungen eines solchen Problems
%
% eigenschaften.tex -- Eigenschaften der Lösungen
% Author: Erik Löffler
%
% (c) 2020 Prof Dr Andreas Müller, Hochschule Rapperswil
%

% TODO:
%  state goal
%  use only what is necessary
%  make sure it is easy enough to understand (sentences as shot as possible)
%    -> Eigenvalue problem with matrices only
%    -> prepare reader for following examples
%
% order:
%  1. Eigenvalue problems with matrices
%  2. Sturm-Liouville is an Eigenvalue problem
%  3. Sturm-Liouville operator (self-adjacent)
%  4. Spectral theorem (brief)
%  5. Base of orthonormal functions

\section{Eigenschaften von Lösungen
\label{sturmliouville:sec:solution-properties}}
\rhead{Eigenschaften von Lösungen}

Im weiteren werden nun die Eigenschaften der Lösung eines
Sturm-Liouville-Problems diskutiert.
Im wesentlichen wird darauf eingegangen, wie die Orthogonalität der Lösungen
zustande kommt, damit diese später in den Beispielen verwendet werden kann.
Dazu wird zunächst das Eigenwertproblem für Matrizen wiederholt und angeschaut
unter welchen Voraussetzungen die Lösungen dieses Problems orthogonal sind.
Dann wird gezeigt, dass das Sturm-Liouville-Problem auch ein Eigenwertproblem
dieser Art ist und es wird auf au die Orthogonalität der Lösungsfunktionen
geschlossen.

\subsection{Eigenwertprobleme mit symmetrischen Matrizen
\label{sturmliouville:sec:eigenvalue-problem-matrix}}

% TODO: intro

Angenomen es sei eine reelle, symmetrische $n \times n$-Matrix $A$ gegeben.
Dass $A$ symmetrisch ist, bedeutet, dass
\[
    \langle Av, w \rangle
    =
    \langle v, Aw \rangle
    \qquad
    v, w \in \mathbb{R}^n
\]
erfüllt ist.

Für reelle, symmetrische Matrizen zeigt dies auch direkt, dass die Matrix
selbstadjungiert ist.
Das ist wichtig, da der Spektralsatz~\cite{sturmliouville:spektralsatz-wiki}
für selbstadjungierte Matrizen formuliert ist. Dieser sagt nun aus, dass die
Matrix $A$ diagonalisierbar ist.
In anderen Worten bilden die Eigenvektoren $v_i \in \mathbb{R}^n$ des 
Eigenwertproblems
\[
    A v_i
    =
    \lambda_i v_i
    \qquad \lambda_i \in \mathbb{R}
\]
eine Orthogonalbasis.

\subsection{Das Sturm-Liouville-Problem als Eigenwertproblem}

In Kapitel~\ref{buch:integrale:subsection:sturm-liouville-problem} wurde bereits
der Operator
\[
    L
    =
    \frac{1}{w(x)}\left( -\frac{d}{dx}p(x) \frac{d}{dx} + q(x)\right)
\]
eingeführt.
Dieser wird nun verwendet um die Differenzialgleichung 
\[
    (p(x)y'(x))' + q(x)y(x)
    =
    \lambda w(x) y(x)
\]
in das Eigenwertproblem
\begin{equation}
    \label{sturmliouville:eq:eigenvalue-problem}
    L y
    =
    \lambda y.
\end{equation}
umzuschreiben.

\subsection{Orthogonalität der Lösungsfunktionen}

Nun wird das Eigenwertproblem~\eqref{sturmliouville:eq:eigenvalue-problem} näher
angeschaut.
Um auf die Orthogonalität der Lösungsfunktion zu schliessen, wird dafür der
Operator $L$ genauer betrachtet.
Analog zur Matrix $A$ aus 
Abschnitt~\ref{sturmliouville:sec:eigenvalue-problem-matrix} kann auch für
$L$ gezeigt werden, dass dieser Operator selbstadjungiert ist, also dass
\[
    \langle L v, w\rangle
    =
    \langle v, L w\rangle
\]
gilt.
Wie in Kapitel~\ref{buch:integrale:subsection:sturm-liouville-problem} bereits
gezeigt, ist dies durch die
Randbedingungen~\eqref{sturmliouville:eq:randbedingungen} des
Sturm-Liouville-Problems sicher gestellt.

Um nun über den Spektralsatz~\cite{sturmliouville:spektralsatz-wiki} auf die
Orthogonalität der Lösungsfunktion $y$ zu schliessen, muss der Operator $L$ ein
sogenannter ''kompakter Operator'' sein.
Bei einem regulären Sturm-Liouville-Problem ist diese Eigenschaft für $L$
gegeben und wird im Weiteren nicht näher diskutiert.

Es kann nun also dank dem Spektralsatz darauf geschlossen werden, dass die
Lösungsfunktion $y$ eises regulären Sturm-Liouville-Problems eine
Linearkombination aus orthogonalen Basisfunktionen sein muss.

%%%%%%%%%%%%%%%%%%%%%%%%%%%%%% OLD section %%%%%%%%%%%%%%%%%%%%%%%%%%%%%%%%%%%%

\iffalse

\section{OLD: Eigenschaften von Lösungen
%\label{sturmliouville:section:solution-properties}
}
\rhead{Eigenschaften von Lösungen}

Im weiteren werden nun die Eigenschaften der Lösungen eines
Sturm-Liouville-Problems diskutiert und aufgezeigt, wie diese Eigenschaften
zustande kommen.

Dazu wird der Operator $L_0$ welcher bereits in
Kapitel~\ref{buch:integrale:subsection:sturm-liouville-problem} betrachtet
wurde, noch etwas genauer angeschaut.
Es wird also im Folgenden
\[
    L_0
    =
    -\frac{d}{dx}p(x)\frac{d}{dx}
\]
zusammen mit den Randbedingungen
\[
    \begin{aligned}
        k_a y(a) + h_a p(a) y'(a) &= 0 \\
        k_b y(b) + h_b p(b) y'(b) &= 0
    \end{aligned}
\]
verwendet.
Wie im Kapitel~\ref{buch:integrale:subsection:sturm-liouville-problem} bereits 
gezeigt, resultieren die Randbedingungen aus der Anforderung den Operator $L_0$
selbsadjungiert zu machen.
Es wurde allerdings noch nicht darauf eingegangen, welche Eigenschaften dies
für die Lösungen des Sturm-Liouville-Problems zur Folge hat.

\subsubsection{Exkurs zum Spektralsatz}

Um zu verstehen welche Eigenschaften der selbstadjungierte Operator $L_0$ in 
den Lösungen hervorbringt, wird der Spektralsatz benötigt.

Dieser wird in der linearen Algebra oft verwendet um zu zeigen, dass eine Matrix
diagonalisierbar ist, beziehungsweise dass eine Orthonormalbasis existiert.

Im Fall einer gegebenen $n\times n$-Matrix $A$ mit reellen Einträgen wird dazu 
zunächst gezeigt, dass $A$ selbstadjungiert ist, also dass
\[
    \langle Av, w \rangle
    =
    \langle v, Aw \rangle
\]
für $ v, w \in \mathbb{R}^n$ gilt.
Ist dies der Fall, kann die Aussage des Spektralsatzes
\cite{sturmliouville:spektralsatz-wiki} verwended werden.
Daraus folgt dann, dass eine Orthonormalbasis aus Eigenvektoren existiert,
wenn $A$ nur Eigenwerte aus $\mathbb{R}$ besitzt.

Dies ist allerdings nicht die Einzige Version des Spektralsatzes.
Unter anderen gibt es den Spektralsatz für kompakte Operatoren
\cite{sturmliouville:spektralsatz-wiki}, welcher für das
Sturm-Liouville-Problem von Bedeutung ist.
Welche Voraussetzungen erfüllt sein müssen, um diese Version des
Satzes verwenden zu können, wird hier aber nicht diskutiert und kann bei den
Beispielen in diesem Kapitel als gegeben betrachtet werden.
Grundsätzlich ist die Aussage in dieser Version dieselbe, wie bei den Matrizen,
also dass für ein Operator eine Orthonormalbasis aus Eigenvektoren existiert,
falls er selbstadjungiert ist.

\subsubsection{Anwendung des Spektralsatzes auf $L_0$}

Der Spektralsatz besagt also, dass, weil $L_0$ selbstadjungiert ist, eine
Orthonormalbasis aus Eigenvektoren existiert.
Genauer bedeutet dies, dass alle Eigenvektoren, beziehungsweise alle Lösungen
des Sturm-Liouville-Problems orthogonal zueinander sind bezüglich des
Skalarprodukts, in dem $L_0$ selbstadjungiert ist.

Erfüllt also eine Differenzialgleichung die in
Abschnitt~\ref{sturmliouville:section:teil0} präsentierten Eigenschaften und
erfüllen die Randbedingungen der Differentialgleichung die Randbedingungen
des Sturm-Liouville-Problems, kann bereits geschlossen werden, dass die
Lösungsfunktion des Problems eine Linearkombination aus orthogonalen
Basisfunktionen ist.

\fi


% Fourier: Erik work
%
% waermeleitung_beispiel.tex -- Beispiel Wärmeleitung in homogenem Stab.
% Author: Erik Löffler
%
% (c) 2020 Prof Dr Andreas Müller, Hochschule Rapperswil
%

\subsection{Fourierreihe als Lösung des Sturm-Liouville-Problems
(Wärmeleitung)}

In diesem Abschnitt wird das Problem der Wärmeleitung in einem homogenen Stab
betrachtet und wie das Sturm-Liouville-Problem bei der Beschreibung dieses
physikalischen Phänomenes auftritt.

% TODO: u is dependent on 2 variables (t, x)
% TODO: mention initial conditions u(0, x)

Zunächst wird ein eindimensionaler homogener Stab der Länge $l$ und
Wärmeleitkoeffizient $\kappa$ betrachtet.
Es ergibt sich für das Wärmeleitungsproblem
die partielle Differentialgleichung
\begin{equation}
    \label{sturmliouville:eq:example-fourier-heat-equation}
    \frac{\partial u}{\partial t} =
    \kappa \frac{\partial^{2}u}{{\partial x}^{2}},
\end{equation}
wobei der Stab in diesem Fall auf der $X$-Achse im Intervall $[0,l]$ liegt.

Da diese Differentialgleichung das Problem allgemein für einen homogenen
Stab beschreibt, werden zusätzliche Bedingungen benötigt, um beispielsweise
die Lösung für einen Stab zu finden, bei dem die Enden auf konstanter 
Tempreatur gehalten werden.

%
% Randbedingungen für Stab mit konstanten Endtemperaturen
%
\subsubsection{Randbedingungen für Stab mit Enden auf konstanter Temperatur}

Die Enden des Stabes auf konstanter Temperatur zu halten bedeutet, dass die
Lösungsfunktion $u(t,x)$ bei $x = 0$ und $x = l$ nur die vorgegebene
Temperatur zurückgeben darf. Diese wird einfachheitshalber als $0$ angenomen.
Es folgt nun
\begin{equation}
    \label{sturmliouville:eq:example-fourier-boundary-condition-ends-constant}
    u(t,0)
    =
    u(t,l)
    =
    0
\end{equation}
als Randbedingungen.

%
% Randbedingungen für Stab mit isolierten Enden
%

\subsubsection{Randbedingungen für Stab mit isolierten Enden}

Bei isolierten Enden des Stabes können beliebige Temperaturen für $x = 0$ und
$x = l$ auftreten. In diesem Fall ist es nicht erlaubt, dass Wärme vom Stab
an die Umgebung oder von der Umgebung an den Stab abgegeben wird.

Aus der Physik ist bekannt, dass Wärme immer von der höheren zur tieferen
Temperatur fliesst. Um Wärmefluss zu unterdrücken, muss also dafür gesorgt
werden, dass am Rand des Stabes keine Temperaturdifferenz existiert oder 
dass die partiellen Ableitungen von $u(t,x)$ nach $x$ bei $x = 0$ und $x = l$
verschwinden.
Somit folgen
\begin{equation}
    \label{sturmliouville:eq:example-fourier-boundary-condition-ends-isolated}
    \frac{\partial}{\partial x} u(t, 0)
    =
    \frac{\partial}{\partial x} u(t, l)
    =
    0
\end{equation}
als Randbedingungen.

%
% Lösung der Differenzialgleichung mittels Separation
%

\subsubsection{Lösung der Differenzialgleichung}

Da die Lösungsfunktion von zwei Variablen abhängig ist, wird als Lösungsansatz
die Separationsmethode verwendet.
Dazu wird 
\[
    u(t,x)
    =
    T(t)X(x)
\]
in die partielle 
Differenzialgleichung~\eqref{sturmliouville:eq:example-fourier-heat-equation}
eingesetzt.
Daraus ergibt sich 
\[
    T^{\prime}(t)X(x)
    =
    \kappa T(t)X^{\prime \prime}(x)
\]
als neue Form.

Nun können alle von $t$ abhängigen Ausdrücke auf die linke Seite, sowie alle
von $x$ abhängigen Ausdrücke auf die rechte Seite gebracht werden und mittels
der neuen Variablen $\mu$ gekoppelt werden:
\[
    \frac{T^{\prime}(t)}{\kappa T(t)}
    =
    \frac{X^{\prime \prime}(x)}{X(x)}
    =
    \mu
\]
Durch die Einführung von $\mu$ kann das Problem nun in zwei separate
Differenzialgleichungen aufgeteilt werden:
\begin{equation}
    \label{sturmliouville:eq:example-fourier-separated-x}
    X^{\prime \prime}(x) - \mu X(x)
    =
    0
\end{equation}
\begin{equation}
    \label{sturmliouville:eq:example-fourier-separated-t}
    T^{\prime}(t) - \kappa \mu T(t)
    =
    0
\end{equation}

%
% Überprüfung Orthogonalität der Lösungen
%

Es ist an dieser Stelle zu bemerken, dass die Gleichung in $x$ in 
Sturm-Liouville-Form ist.
Erfüllen die Randbedingungen des Stab-Problems auch die Randbedingungen des
Sturm-Liouville-Problems, kann bereits die Aussage getroffen werden, dass alle
Lösungen für die Gleichung in $x$ orthogonal sein werden.

Da die Bedingungen des Stab-Problem nur Anforderungen an $x$ stellen, können
diese direkt für $X(x)$ übernomen werden. Es gilt also $X(0) = X(l) = 0$.
Damit die Lösungen von $X$ orthogonal sind, müssen also die Gleichungen
\begin{equation}
\begin{aligned}
	\label{sturmliouville:eq:example-fourier-randbedingungen}
	k_a X(a) + h_a p(a) X'(a) &= 0 \\
	k_b X(b) + h_b p(b) X'(b) &= 0
\end{aligned}
\end{equation}
erfüllt sein und es muss ausserdem
\begin{equation}
\begin{aligned}
    \label{sturmliouville:eq:example-fourier-coefficient-constraints}
    |k_a|^2 + |h_a|^2 &\neq 0\\
    |k_b|^2 + |h_b|^2 &\neq 0\\
\end{aligned}
\end{equation}
gelten.

Um zu verifizieren, ob die Randbedingungen erfüllt sind, wird zunächst
$p(x)$
benötigt.
Dazu wird die Gleichung~\eqref{sturmliouville:eq:example-fourier-separated-x}
mit der
Sturm-Liouville-Form~\eqref{eq:sturm-liouville-equation} verglichen, was zu
$p(x) = 1$ führt.

Werden nun $p(x)$ und die 
Randbedingungen~\eqref{sturmliouville:eq:example-fourier-boundary-condition-ends-constant}
in \eqref{sturmliouville:eq:example-fourier-randbedingungen} eigesetzt, erhält
man
\[
\begin{aligned}
	k_a y(0) + h_a y'(0) &= h_a y'(0) = 0 \\
	k_b y(l) + h_b y'(l) &= h_b y'(l) = 0.
\end{aligned}
\]
Damit die Gleichungen erfüllt sind, müssen $h_a = 0$ und $h_b = 0$ sein.
Zusätzlich müssen aber die 
Bedingungen~\eqref{sturmliouville:eq:example-fourier-coefficient-constraints}
erfüllt sein und da $y(0) = 0$ und $y(l) = 0$ sind, können belibige $k_a \neq 0$
und $k_b \neq 0$ gewählt werden.

Somit ist gezeigt, dass die Randbedingungen des Stab-Problems für Enden auf
konstanter Temperatur auch die Sturm-Liouville-Randbedingungen erfüllen und
alle daraus reultierenden Lösungen orthogonal sind.
Analog dazu kann gezeit werden, dass die Randbedingungen für einen Stab mit
isolierten Enden ebenfalls die Sturm-Liouville-Randbedingungen erfüllen und
somit auch zu orthogonalen Lösungen führen.

%
%   Lösung von X(x), Teil mu
%

\subsubsection{Lösund der Differentialgleichung in $x$}
Als erstes wird auf die
Gleichung~\eqref{sturmliouville:eq:example-fourier-separated-x} eingegangen.
Aufgrund der Struktur der Gleichung
\[
    X^{\prime \prime}(x) - \mu X(x)
    =
    0
\]
wird ein trigonometrischer Ansatz gewählt.
Die Lösungen für $X(x)$ sind also von der Form
\[
    X(x)
    =
    A \cos \left( \alpha x\right) + B \sin \left( \beta x\right).
\]

Dieser Ansatz wird nun solange differenziert, bis alle in
Gleichung~\eqref{sturmliouville:eq:example-fourier-separated-x} enthaltenen
Ableitungen vorhanden sind.
Man erhält also
\[
    X^{\prime}(x)
    =
    - \alpha A \sin \left( \alpha x \right) +
    \beta B \cos \left( \beta x \right)
\]
und
\[
    X^{\prime \prime}(x)
    =
    -\alpha^{2} A \cos \left( \alpha x \right) -
    \beta^{2} B \sin \left( \beta x \right).
\]

Eingesetzt in Gleichung~\eqref{sturmliouville:eq:example-fourier-separated-x}
ergibt dies
\[
    -\alpha^{2}A\cos(\alpha x) - \beta^{2}B\sin(\beta x) -
    \mu\left(A\cos(\alpha x) + B\sin(\beta x)\right)
    =
    0
\]
und durch umformen somit
\[
    -\alpha^{2}A\cos(\alpha x) - \beta^{2}B\sin(\beta x)
    =
    \mu A\cos(\alpha x) + \mu B\sin(\beta x).
\]

Mittels Koeffizientenvergleich von
\[
\begin{aligned}
    -\alpha^{2}A\cos(\alpha x)
    &=
    \mu A\cos(\alpha x)
    \\
    -\beta^{2}B\sin(\beta x)
    &=
    \mu B\sin(\beta x)
\end{aligned}
\]
ist schnell ersichtlich, dass $ \mu = -\alpha^{2} = -\beta^{2} $ gelten muss für
$ A \neq 0 $ oder $ B \neq 0 $.
Zur Berechnung von $ \mu $ bleiben also noch  $ \alpha $ und $ \beta $ zu
bestimmen.
Dazu werden nochmals die
Randbedingungen~\eqref{sturmliouville:eq:example-fourier-boundary-condition-ends-constant} 
und \eqref{sturmliouville:eq:example-fourier-boundary-condition-ends-isolated}
benötigt.

Da die Koeffizienten $A$ und $B$, sowie die Parameter $\alpha$ uns $\beta$ im
allgemeninen ungleich $0$ sind, müssen die Randbedingungen durch die
trigonometrischen Funktionen erfüllt werden.

Es werden nun die 
Randbedingungen~\eqref{sturmliouville:eq:example-fourier-boundary-condition-ends-constant}
für einen Stab mit Enden auf konstanter Temperatur in die
Gleichung~\eqref{sturmliouville:eq:example-fourier-separated-x} eingesetzt.
Betrachten wir zunächst die Bedingung für $x = 0$.
Dies fürht zu
\[
    X(0)
    =
    A \cos(0 \alpha) + B \sin(0 \beta)
    =
    0.
\]
Da $\cos(0) \neq 0$ ist, muss in diesem Fall $A = 0$ gelten.
Für den zweiten Summanden ist wegen $\sin(0) = 0$ die Randbedingung erfüllt.

Wird nun die zweite Randbedingung für $x = l$ mit $A = 0$ eingesetzt, ergibt
sich
\[
    X(l)
    =
    0 \cos(\alpha l) + B \sin(\beta l)
    =
    B \sin(\beta l)
    = 0.
\]

$\beta$ muss also so gewählt werden, dass $\sin(\beta l) = 0$ gilt.
Es bleibt noch nach $\beta$ aufzulösen:
\[
\begin{aligned}
    \sin(\beta l) &= 0 \\
    \beta l &= n \pi \qquad n \in \mathbb{N} \\
    \beta &= \frac{n \pi}{l} \qquad n \in \mathbb{N}
\end{aligned}
\]

Es folgt nun wegen $\mu = -\beta^{2}$, dass
\[
    \mu_1 = -\beta^{2} = -\frac{n^{2}\pi^{2}}{l^{2}}
\]
sein muss.
Ausserdem ist zu bemerken, dass dies auch gleich $-\alpha^{2}$ ist.
Da aber $A = 0$ gilt und der Summand mit $\alpha$ verschwindet, ist dies keine
Verletzung der Randbedingungen.

Durch alanoges Vorgehen kann nun auch das Problem mit isolierten Enden gelöst
werden.
Setzt man nun die 
Randbedingungen~\eqref{sturmliouville:eq:example-fourier-boundary-condition-ends-isolated}
in $X^{\prime}$ ein, beginnend für $x = 0$. Es ergibt sich
\[
    X^{\prime}(0)
    =
    -\alpha A \sin(0 \alpha) + \beta B \cos(0 \beta)
    = 0.
\]
In diesem Fall muss $B = 0$ gelten.
Zusammen mit der Bedignung für $x = l$
folgt nun
\[
    X^{\prime}(l)
    =
    - \alpha A \sin(\alpha l) + 0 \beta \cos(\beta l)
    =
    - \alpha A \sin(\alpha l)
    = 0.
\]

Wiedrum muss über die $\sin$-Funktion sicher gestellt werden, dass der
Ausdruck den Randbedingungen entspricht.
Es folgt nun
\[
\begin{aligned}
    \sin(\alpha l) &= 0 \\
    \alpha l &= n \pi \qquad n \in \mathbb{N} \\
    \alpha &= \frac{n \pi}{l} \qquad n \in \mathbb{N}
\end{aligned}
\]
und somit
\[
    \mu_2 = -\alpha^{2} = -\frac{n^{2}\pi^{2}}{l^{2}}.
\]

Es ergibt sich also sowohl für einen Stab mit Enden auf konstanter Temperatur
wie auch mit isolierten Enden
\begin{equation}
    \label{sturmliouville:eq:example-fourier-mu-solution}
    \mu
    =
    -\frac{n^{2}\pi^{2}}{l^{2}}.
\end{equation}

% TODO: infinite base vectors and fourier series
\subsubsection{TODO: Auf Anzahl Lösungen und Fourierreihe eingehen}

% TODO: check ease of reading
\subsubsection{Berechnung der Koeffizienten}

% TODO: move explanation A/B -> a_n/b_n to fourier subsection

%
% Lösung von X(x), Teil: Koeffizienten a_n und b_n mittels skalarprodukt.
%

Bisher wurde über die Koeffizienten $A$ und $B$ noch nicht viel ausgesagt.
Zunächst ist wegen vorhergehender Rechnung ersichtlich, dass es sich bei
$A$ und $B$ nicht um einzelne Koeffizienten handelt.
Stattdessen können die Koeffizienten für jedes $n \in \mathbb{N}$
unterschiedlich sein.
Die Lösung $X(x)$ wird nun umgeschrieben zu
\[
    X(x)
    =
    a_0
    +
    \sum_{n = 1}^{\infty} a_n\cos\left(\frac{n\pi}{l}x\right)
    +
    \sum_{n = 1}^{\infty} b_n\sin\left(\frac{n\pi}{l}x\right).
\]

Um eine eindeutige Lösung für $X(x)$ zu erhalten werden noch weitere
Bedingungen benötigt.
Diese sind die Startbedingungen oder $u(0, x) = X(x)$ für $t = 0$.
Es gilt also nun die Gleichung
\begin{equation}
    \label{sturmliouville:eq:example-fourier-initial-conditions}
    u(0, x)
    =
    a_0
    +
    \sum_{n = 1}^{\infty} a_n\cos\left(\frac{n\pi}{l}x\right)
    +
    \sum_{n = 1}^{\infty} b_n\sin\left(\frac{n\pi}{l}x\right)
\end{equation}
nach allen $a_n$ und $b_n$ aufzulösen.
Da aber $a_n$ und $b_n$ jeweils als Faktor zu einer trigonometrischen Funktion
gehört, von der wir wissen, dass sie orthogonal zu allen anderen
trigonometrischen Funktionen der Lösung ist, kann direkt das Skalarprodukt
verwendet werden um die Koeffizienten $a_n$ und $b_n$ zu bestimmen.
Es wird also die Tatsache ausgenutzt, dass die Gleichheit in
\eqref{sturmliouville:eq:example-fourier-initial-conditions} nach Anwendung des
Skalarproduktes immernoch gelten muss und dass das Skalaprodukt mit einer
Basisfunktion sämtliche Summanden auf der rechten Seite auslöscht.

Zur Berechnung von $a_m$ mit $ m \in \mathbb{N} $ wird beidseitig das
Skalarprodukt mit der Basisfunktion $ \cos\left(\frac{m \pi}{l}x\right)$
gebildet:
\begin{equation}
    \label{sturmliouville:eq:dot-product-cosine}
    \langle u(0, x), \cos\left(\frac{m \pi}{l}x\right) \rangle
    =
    \langle a_0
    +
    \sum_{n = 1}^{\infty} a_n\cos\left(\frac{n\pi}{l}x\right)
    +
    \sum_{n = 1}^{\infty} b_n\sin\left(\frac{n\pi}{l}x\right),
    \cos\left(\frac{m \pi}{l}x\right)\rangle
\end{equation}

Bevor diese Form in die Integralform umgeschrieben werden kann, muss überlegt
sein, welche Integralgrenzen zu verwenden sind.
In diesem Fall haben die $\sin$ und $\cos$ Terme beispielsweise keine ganze
Periode im Intervall $x \in [0, l]$ für ungerade $n$ und $m$.
Um die Skalarprodukte aber korrekt zu berechnen, muss über ein ganzzahliges
Vielfaches der Periode der trigonometrischen Funktionen integriert werden.
Dazu werden die Integralgrenzen $-l$ und $l$ verwendet und es werden ausserdem
neue Funktionen $\hat{u}_c(0, x)$ für die Berechnung mit Cosinus und
$\hat{u}_s(0, x)$ für die Berechnung mit Sinus angenomen, welche $u(0, t)$
gerade, respektive ungerade auf $[-l, l]$ fortsetzen:
\[
\begin{aligned}
    \hat{u}_c(0, x)
    &=
    \begin{cases}
        u(0, -x) & -l \leq x < 0
        \\
        u(0, x) & 0 \leq x \leq l
    \end{cases}
    \\
    \hat{u}_s(0, x)
    &=
    \begin{cases}
        -u(0, -x) & -l \leq x < 0
        \\
        u(0, x) & 0 \leq x \leq l
    \end{cases}.
\end{aligned}
\]

Die Konsequenz davon ist, dass nun das Resultat der Integrale um den Faktor zwei
skalliert wurde, also gilt nun
\[
\begin{aligned}
    \int_{-l}^{l}\hat{u}_c(0, x)\cos\left(\frac{m \pi}{l}x\right)dx
    &=
    2\int_{0}^{l}u(0, x)\cos\left(\frac{m \pi}{l}x\right)dx
    \\
    \int_{-l}^{l}\hat{u}_s(0, x)\sin\left(\frac{m \pi}{l}x\right)dx
    &=
    2\int_{0}^{l}u(0, x)\sin\left(\frac{m \pi}{l}x\right)dx.
\end{aligned}
\]

Zunächst wird nun das Skalaprodukt~\eqref{sturmliouville:eq:dot-product-cosine}
berechnet:
\[
\begin{aligned}
    \int_{-l}^{l}\hat{u}_c(0, x)\cos\left(\frac{m \pi}{l}x\right)dx
    =&
    \int_{-l}^{l} \left[a_0
    +
    \sum_{n = 1}^{\infty} a_n\cos\left(\frac{n\pi}{l}x\right)
    +
    \sum_{n = 1}^{\infty} b_n\sin\left(\frac{n\pi}{l}x\right)\right]
    \cos\left(\frac{m \pi}{l}x\right) dx
    \\
    2\int_{0}^{l}u(0, x)\cos\left(\frac{m \pi}{l}x\right)dx
    =&
    a_0 \int_{-l}^{l}\cos\left(\frac{m \pi}{l}x\right) dx
    +
    \sum_{n = 1}^{\infty}\left[a_n\int_{-l}^{l}\cos\left(\frac{n\pi}{l}x\right)
        \cos\left(\frac{m \pi}{l}x\right)dx\right]
    \\
    &+
    \sum_{n = 1}^{\infty}\left[b_n\int_{-l}^{l}\sin\left(\frac{n\pi}{l}x\right)
        \cos\left(\frac{m \pi}{l}x\right)dx\right].
\end{aligned}
\]

Betrachtet man nun die Summanden auf der rechten Seite stellt man fest, dass
nahezu alle Terme verschwinden, denn
\[
    \int_{-l}^{l}\cos\left(\frac{m \pi}{l}x\right) dx
    =
    0,
\]
da hier über ein ganzzahliges Vielfaches der Periode integriert wird,
\[
    \int_{-l}^{l}\cos\left(\frac{n\pi}{l}x\right)
    \cos\left(\frac{m \pi}{l}x\right)dx
    =
    0
\]
für $m\neq n$, da Cosinus-Funktionen mit verschiedenen Kreisfrequenzen
orthogonal zueinander stehen und
\[
    \int_{-l}^{l}\sin\left(\frac{n\pi}{l}x\right)
        \cos\left(\frac{m \pi}{l}x\right)dx
    =
    0
\]
da Sinus- und Cosinus-Funktionen ebenfalls orthogonal zueinander sind.

Es bleibt also lediglich der Summand für $a_m$ stehen, was die Gleichung zu
\[
    2\int_{0}^{l}u(0, x)\cos\left(\frac{m \pi}{l}x\right)dx
    =
    a_m\int_{-l}^{l}\cos^2\left(\frac{m\pi}{l}x\right)dx
\]
vereinfacht. Im nächsten Schritt wird nun das Integral auf der rechten Seite
berechnet und dann nach $a_m$ aufgelöst. Am einnfachsten geht dies, wenn zuerst
mit $u = \frac{m \pi}{l}x$ substituiert wird:
\[
    \begin{aligned}
    2\int_{0}^{l}u(0, x)\cos\left(\frac{m \pi}{l}x\right)dx
    &=
    a_m\frac{l}{m\pi}\int_{-m\pi}^{m\pi}\cos^2\left(u\right)du
    \\
    &=
    a_m\frac{l}{m\pi}\left[\frac{u}{2} + 
    \frac{\sin\left(2u\right)}{4}\right]_{u=-m\pi}^{m\pi}
    \\
    &=
    a_m\frac{l}{m\pi}\biggl(\frac{m\pi}{2} + 
    \underbrace{\frac{\sin\left(2m\pi\right)}{4}}_{\displaystyle = 0} - 
    \frac{-m\pi}{2} -
    \underbrace{\frac{\sin\left(-2m\pi\right)}{4}}_{\displaystyle = 0}\biggr)
    \\
    &=
    a_m l
    \\
    a_m
    &=
    \frac{2}{l} \int_{0}^{l}u(0, x)\cos\left(\frac{m \pi}{l}x\right)dx
    \end{aligned}
\]

Analog dazu kann durch das Bilden des Skalarproduktes mit 
$ \sin\left(\frac{m \pi}{l}x\right) $ gezeigt werden, dass
\[
    b_m
    =
    \frac{2}{l} \int_{0}^{l}u(0, x)\sin\left(\frac{m \pi}{l}x\right)dx
\]
gilt.

Etwas anders ist es allerdings bei $a_0$.
Wie der Name bereits suggeriert, handelt es sich hierbei um den Koeffizienten
zur Basisfunktion $\cos\left(\frac{0 \pi}{l}x\right)$ beziehungsweise der
konstanten Funktion $1$.
Um einen Ausdruck für $a_0$ zu erhalten, wird wiederum auf beiden Seiten
der Gleichung~\eqref{sturmliouville:eq:example-fourier-initial-conditions} das
Skalarprodukt mit der konstanten Basisfunktion $1$ gebildet:
\[
\begin{aligned}
    \int_{-l}^{l}\hat{u}_c(0, x)dx
    &=
    \int_{-l}^{l} a_0
    +
    \sum_{n = 1}^{\infty} a_n\cos\left(\frac{n\pi}{l}x\right)
    +
    \sum_{n = 1}^{\infty} b_n\sin\left(\frac{n\pi}{l}x\right)dx
    \\
    2\int_{0}^{l}u(0, x)dx
    &=
    a_0 \int_{-l}^{l}dx
    +
    \sum_{n = 1}^{\infty}\left[a_n\int_{-l}^{l}\cos\left(\frac{n\pi}{l}x\right)
        dx\right] +
    \sum_{n = 1}^{\infty}\left[b_n\int_{-l}^{l}\sin\left(\frac{n\pi}{l}x\right)
        dx\right].
\end{aligned}
\]

Hier fallen nun alle Terme, die $\sin$ oder $\cos$ beinhalten weg, da jeweils
über ein Vielfaches der Periode integriert wird.
Es bleibt also noch
\[
    2\int_{0}^{l}u(0, x)dx
    =
    a_0 \int_{-l}^{l}dx
\]
, was sich wie folgt nach $a_0$ auflösen lässt:
\[
\begin{aligned}
    2\int_{0}^{l}u(0, x)dx
    &=
    a_0 \int_{-l}^{l}dx
    \\
    &=
    a_0 \left[x\right]_{x=-l}^{l}
    \\
    &=
    a_0(l - (-l))
    \\
    &=
    a_0 \cdot 2l
    \\
    a_0
    &=
    \frac{1}{l} \int_{0}^{l}u(0, x)dx
\end{aligned}
\]

%
% Lösung von T(t) 
%

\subsubsection{Lösung der Differentialgleichung in $t$}
Zuletzt wird die zweite Gleichung der 
Separation~\eqref{sturmliouville:eq:example-fourier-separated-t} betrachtet.
Diese wird über das charakteristische Polynom
\[
    \lambda - \kappa \mu
    =
    0
\]
gelöst.

Es ist direkt ersichtlich, dass $\lambda = \kappa \mu$ gelten muss, was zur
Lösung
\[
    T(t)
    =
    e^{\kappa \mu t}
\]
führt und mit dem Resultat~\eqref{sturmliouville:eq:example-fourier-mu-solution}
\[
    T(t)
    =
    e^{-\frac{n^{2}\pi^{2}\kappa}{l^{2}}t}
\]
ergibt.

Dieses Resultat kann nun mit allen vorhergehenden Resultaten zusammengesetzt
werden um die vollständige Lösung für das Stab-Problem zu erhalten.

% TODO: elaborate

\subsubsection{Lösung für einen Stab mit Enden auf konstanter Temperatur}
\[
\begin{aligned}
    u(t,x)
    &=
    \sum_{n=1}^{\infty}b_{n}e^{-\frac{n^{2}\pi^{2}\kappa}{l^{2}}t}
    \sin\left(\frac{n\pi}{l}x\right)
    \\
    b_{n}
    &=
    \frac{2}{l}\int_{0}^{l}u(0,x)sin\left(\frac{n\pi}{l}x\right) dx
\end{aligned}
\]

\subsubsection{Lösung für einen Stab mit isolierten Enden}
\[
\begin{aligned}
    u(t,x)
    &=
    a_{0} + \sum_{n=1}^{\infty}a_{n}e^{-\frac{n^{2}\pi^{2}\kappa}{l^{2}}t}
    \cos\left(\frac{n\pi}{l}x\right)
    \\
    a_{0}
    &=
    \frac{1}{l}\int_{0}^{l}u(0,x) dx
    \\
    a_{n}
    &=
    \frac{2}{l}\int_{0}^{l}u(0,x)sin\left(\frac{n\pi}{l}x\right) dx
\end{aligned}
\]


% Tschebyscheff
%
% tschebyscheff_beispiel.tex
% Author: Réda Haddouche
%
% (c) 2020 Prof Dr Andreas Müller, Hochschule Rapperswil
%

\subsection{Tschebyscheff-Polynome
\label{sturmliouville:sub:tschebyscheff-polynome}}
\rhead{Tschebyscheff-Polynome}
In diesem Unterkapitel wird anhand der
Tschebyscheff-Differentialgleichung~\eqref{buch:potenzen:tschebyscheff:dgl} gezeigt, dass die Tschebyscheff-Polynome orthogonal zueinander sind.
Zu diesem Zweck werden die Koeffizientenfunktionen nochmals dargestellt, so dass
überprüft werden kann, ob die Randbedingungen erfüllt werden können.
Sobald feststeht, ob das Problem regulär oder singulär ist, zeigt eine
kleine Rechnung, dass die Lösungen orthogonal sind.

\subsubsection*{Definition der Koeffizientenfunktion}
Im Kapitel \ref{sub:beispiele_sturm_liouville_problem} sind die
Koeffizientenfunktionen, die man braucht, schon aufgelistet:
\begin{align*}
	w(x) &= \frac{1}{\sqrt{1-x^2}}, \\
	p(x) &= \sqrt{1-x^2}, \\
	q(x) &= 0.
\end{align*}
Da die Sturm-Liouville-Gleichung
\begin{equation}
	\label{eq:sturm-liouville-equation-tscheby}
	\frac{d}{dx} (\sqrt{1-x^2} \frac{dy}{dx}) +
	(0 + \lambda \frac{1}{\sqrt{1-x^2}}) y
	=
	0 
\end{equation}
nun mit den Koeffizientenfunktionen aufgestellt werden kann, bleibt die Frage,
ob es sich um ein reguläres oder singuläres Sturm-Liouville-Problem handelt.
Zunächst werden jedoch die Randbedingungen betrachtet.

\subsubsection*{Randwertproblem}
Für die Verifizierung der Randbedingungen benötigt man erneut $p(x)$.
Die Randwerte setzt man $a = -1$ und $b = 1$.
Beim Einsetzen in die Randbedingung \eqref{sturmliouville:eq:randbedingungen},
erhält man
\begin{equation}
	\begin{aligned}
		k_a y(-1) + h_a p(-1) y'(-1) &= 0\\
		k_b y(1) + h_b p(1) y'(-1) &= 0.
	\end{aligned} 
\end{equation}
Die Funktion $y(x)$ und $y'(x)$ sind in diesem Fall die Tschebyscheff Polynome
(siehe \ref{sub:definiton_der_tschebyscheff-Polynome}).
Die Funktion $y(x)$ wird nun mit der Funktion $T_n(x)$ ersetzt und für die
Verifizierung der Randbedingung wählt man $n=0$.
Somit erhält man
\begin{equation}
	\begin{aligned}
		k_a T_0(-1) + h_a p(-1) T_{0}'(-1) &= k_a = 0\\
		k_b T_0(1) + h_b p(1) T_{0}'(1) &= k_b = 0.
	\end{aligned}
\end{equation}
Ähnlich wie beim Beispiel der Wärmeleitung in einem homogenen Stab können,
damit die Bedingung $|k_i|^2 + |h_i|^2\ne 0$ erfüllt ist, beliebige
$h_a \ne 0$ und $h_b \ne 0$ gewählt werden.
Es wurde somit gezeigt, dass die Sturm-Liouville-Randbedingungen erfüllt sind.

\subsubsection*{Handelt es sich um ein reguläres oder Singuläres Problem?}
Für das reguläre Problem muss laut der
Definition~\ref{sturmliouville:def:reguläres_sturm-liouville-problem} die funktion
$p(x) = \sqrt{1-x^2}$, $p'(x) = -2x$, $q(x) = 0$ und
$w(x) = \frac{1}{\sqrt{1-x^2}}$ stetig und reell sein.
Auf dem Intervall $(-1,1)$ sind die Tschebyscheff-Polynome erster Art
\begin{equation}
	T_n(x)
	=
	\cos n (\arccos x).
\end{equation}
Die nächste Bedingung, laut der Definition \ref{sturmliouville:def:reguläres_sturm-liouville-problem}, beinhaltet, dass die Funktion $p(x)$ und $w(x)>0$ sein
müssen.
Die Funktion
\begin{equation*}
	p(x)^{-1} = \frac{1}{\sqrt{1-x^2}}
\end{equation*}
ist die gleiche wie $w(x)$ und erfüllt die Bedingung.
Es zeigt sich also, dass $p(x)$, $p'(x)$, $q(x)$ und $w(x)$
die Bedingungen erfüllen.
Da auch die Randbedingungen erfüllt sind, handelt es sich um ein reguläres Sturm-Liouville-Problem.


\begin{beispiel}
	In diesem Beispiel wird zuletzt die Orthogonalität der Lösungsfunktion
	illustriert.
	Dazu verwendet man das Skalarprodukt
	\[
		\int_{a}^{b} w(x) y_m y_n = 0.
	\]
	Eigesetzt ergibt dies $y_m(x) = T_1(x)$ und $y_n(x) = T_2(x)$, sowie $a=-1$ und $b = 1$
	ergibt
	\[
	\begin{aligned}
	\int_{-1}^{1} \frac{1}{\sqrt{1-x^2}} x (2x^2-1) dx &=
	\lbrack - \frac{\sqrt{1-x^2}(2x^2+1)}{3}\rbrack_{-1}^{1}\\
	&= 0.
	\end{aligned}
	\]
	Somit ist gezeigt, dass $T_1(x)$ und $T_2(x)$ orthogonal sind.
	Analog kann Orthogonalität für alle $y_n(x)$ und $y_m(x)$ mit $n \ne m$ gezeigt werden.
\end{beispiel}


\printbibliography[heading=subbibliography]
\end{refsection}
