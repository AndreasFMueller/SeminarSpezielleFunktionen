%
% tschebyscheff_beispiel.tex
%
% (c) 2020 Prof Dr Andreas Müller, Hochschule Rapperswil
%

\subsection{Tschebyscheff-Polynome\label{sub:tschebyscheff-polynome}}
Im Kapitel \ref{sub:beispiele_sturm_liouville_problem} sind die Koeffizientenfunktionen die man braucht schon aufgeliste, und zwar mit
\begin{align*}
	w(x) &= \frac{1}{\sqrt{1-x^2}} \\
	p(x) &= \sqrt{1-x^2} \\
	q(x) &= 0
\end{align*}.
Da die Sturm-Liouville-Gleichung
\begin{equation}
	\label{eq:sturm-liouville-equation-tscheby}
	\frac{d}{dx}\lbrack \sqrt{1-x^2} \frac{dy}{dx} \rbrack + \lbrack 0 + \lambda \frac{1}{\sqrt{1-x^2}} \rbrack y = 0 
\end{equation}
nun mit den Koeffizientenfunktionen aufgestellt werden kann, bleibt die Frage, ob es sich um ein reguläres oder singuläres Sturm-Liouville-Problem handelt.
Für das reguläre Problem laut der Definition \ref{def:reguläres_sturm-liouville-problem} muss die funktion $p(x) = \sqrt{1-x^2}$, $p'(x) = -2x$, $q(x) = 0$ und $w(x) = \frac{1}{\sqrt{1-x^2}}$ stetig und reell sein - und sie sind es auch.
Auf dem Intervall $(-1,1)$ sind die Tschebyscheff-Polynome erster Art mit Hilfe von Hyperbelfunktionen
\begin{equation}
	T_n(x) = \cos n (\arccos x)
\end{equation}.
Für $x>1$ und $x<-1$ sehen die Polynome wie folgt aus:
\begin{equation}
	T_n(x) = \left\{\begin{array}{ll} \cosh (n \arccos x), & x > 1\\
		(-1)^n \cosh (n \arccos (-x)), & x<-1 \end{array}\right.
\end{equation},
jedoch ist die Orthogonalität nur auf dem Intervall $[ -1, 1]$ sichergestellt.
Die nächste Bedingung beinhaltet, dass die Funktion $p(x)^-1$ und $w(x)>0$ sein müssen.
Die Funktion
\begin{equation*}
	p(x)^-1 = \frac{1}{\sqrt{1-x^2}}
\end{equation*}
ist die gleiche wie $w(x)$.

Für die Verifizierung der Randbedingungen benötigt man erneut $p(x)$.
Da sich die Polynome nur auf dem Intervall $[ -1,1 ]$ orthogonal verhalten, sind $a = -1$ und $b = 1$ gesetzt.
Beim einsetzen in die Randbedingung \ref{eq:randbedingungen}, erhält man
\begin{equation}
\begin{aligned}
	k_a y(-1) + h_a y'(-1) &= h_a 
\end{aligned} 
\end{equation}











