%
% tschebyscheff_beispiel.tex
% Author: Réda Haddouche
%
% (c) 2020 Prof Dr Andreas Müller, Hochschule Rapperswil
%

\section{Beispiel: Tschebyscheff-Polynome
\label{sturmliouville:sub:tschebyscheff-polynome}}
\rhead{Tschebyscheff-Polynome}
In diesem Unterkapitel wird anhand der
Tschebyscheff-Differentialgleichung~\eqref{buch:potenzen:tschebyscheff:dgl}
gezeigt, dass die Tschebyscheff-Polynome orthogonal zueinander sind.
Zu diesem Zweck werden die Koeffizientenfunktionen nochmals dargestellt, so dass
überprüft werden kann, ob die Randbedingungen erfüllt werden.
Sobald feststeht, ob das Problem regulär oder singulär ist, zeigt eine
kleine Rechnung, dass die Lösungen orthogonal sind.

\subsection*{Definition der Koeffizientenfunktion}
Im Kapitel \ref{sub:beispiele_sturm_liouville_problem} sind die
Koeffizientenfunktionen, die man braucht, schon aufgelistet:
\begin{align*}
	w(x) &= \frac{1}{\sqrt{1-x^2}}, \\
	p(x) &= \sqrt{1-x^2}, \\
	q(x) &= 0.
\end{align*}
Da die Sturm-Liouville-Gleichung
\begin{equation}
	\label{eq:sturm-liouville-equation-tscheby}
	\frac{d}{dx} \biggl (\sqrt{1-x^2} \frac{dy}{dx}\biggr ) +
	\biggl (0 + \lambda \frac{1}{\sqrt{1-x^2}}\biggr ) y
	=
	0 
\end{equation}
nun mit den Koeffizientenfunktionen aufgestellt werden kann, bleibt die Frage,
ob es sich um ein reguläres oder singuläres Sturm-Liouville-Problem handelt.
Zunächst werden jedoch die Randbedingungen betrachtet.

\subsection*{Randwertproblem}
Für die Verifizierung der Randbedingungen benötigt man erneut $p(x)$.
Die Randwerte setzt man $a = -1$ und $b = 1$.
Beim Einsetzen in die Randbedingung \eqref{sturmliouville:eq:randbedingungen},
erhält man
\begin{equation}
	\begin{aligned}
		k_a y(-1) + h_a p(-1) y'(-1) &= 0\\
		k_b y(1) + h_b p(1) y'(1) &= 0.
	\end{aligned} 
\end{equation}
Die Funktion $y(x)$ und $y'(x)$ sind in diesem Fall die Tschebyscheff Polynome
(siehe \ref{sub:definiton_der_tschebyscheff-Polynome}).
Die Funktion $y(x)$ wird nun mit der Funktion $T_n(x)$ ersetzt und für die
Verifizierung der Randbedingung wählt man $n=0$.
Somit erhält man
\begin{equation}
	\begin{aligned}
		k_a T_0(-1) + h_a p(-1) T_{0}'(-1) &= k_a = 0\\
		k_b T_0(1) + h_b p(1) T_{0}'(1) &= k_b = 0.
	\end{aligned}
\end{equation}
Ähnlich wie beim Beispiel der Wärmeleitung in einem homogenen Stab können,
damit die Bedingung $|k_i|^2 + |h_i|^2\ne 0$ erfüllt ist, beliebige
$h_a \ne 0$ und $h_b \ne 0$ gewählt werden.
Es wurde somit gezeigt, dass die Sturm-Liouville-Randbedingungen erfüllt sind.

\subsection*{Handelt es sich um ein reguläres oder singuläres Problem?}
Für das reguläre Problem muss laut der
Definition~\ref{sturmliouville:def:reguläres_sturm-liouville-problem} die Funktion
$p(x) = \sqrt{1-x^2}$, $p'(x) = -2x$, $q(x) = 0$ und
$w(x) = \frac{1}{\sqrt{1-x^2}}$ stetig und reell sein.
Auf dem Intervall $(-1,1)$ sind die Tschebyscheff-Polynome erster Art
\begin{equation}
	T_n(x)
	=
	\cos n (\arccos x).
\end{equation}
Die nächste Bedingung, laut der Definition \ref{sturmliouville:def:reguläres_sturm-liouville-problem}, beinhaltet, dass die Funktion $p(x)$ und $w(x)>0$ sein
müssen.
Die Funktion
\begin{equation*}
	p(x)^{-1} = \frac{1}{\sqrt{1-x^2}}
\end{equation*}
ist die gleiche wie $w(x)$ und erfüllt die Bedingung.
Es zeigt sich also, dass $p(x)$, $p'(x)$, $q(x)$ und $w(x)$
die Bedingungen erfüllen.
Da auch die Randbedingungen erfüllt sind, handelt es sich um ein reguläres Sturm-Liouville-Problem.


\begin{beispiel}
	In diesem Beispiel wird zuletzt die Orthogonalität der Lösungsfunktion
	illustriert.
	Dazu verwendet man das Skalarprodukt
	\[
		\int_{a}^{b} w(x) y_m(x) y_n(x) = 0.
	\]
	mit $y_m(x) = T_1(x)$ und $y_n(x) = T_2(x)$, sowie $a=-1$ und $b = 1$.
	Eigesetzt ergibt dies
	\[
	\begin{aligned}
	\int_{-1}^{1} \frac{1}{\sqrt{1-x^2}} x (2x^2-1) dx &=
	\biggl [ - \frac{\sqrt{1-x^2}(2x^2+1)}{3} \biggr ]_{-1}^{1}\\
	&= 0.
	\end{aligned}
	\]
	Somit ist gezeigt, dass $T_1(x)$ und $T_2(x)$ orthogonal sind.
	Analog kann Orthogonalität für alle $y_n(x)$ und $y_m(x)$ mit $n \ne m$ gezeigt werden.
\end{beispiel}
