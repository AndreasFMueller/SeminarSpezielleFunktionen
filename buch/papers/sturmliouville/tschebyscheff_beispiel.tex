%
% tschebyscheff_beispiel.tex
%
% (c) 2020 Prof Dr Andreas Müller, Hochschule Rapperswil
%

\subsection{Sind Tschebyscheff-Polynome orthogonal zueinander?\label{sub:tschebyscheff-polynome}}
\subsubsection*{Definition der Koeffizientenfunktion}
Im Kapitel \ref{sub:beispiele_sturm_liouville_problem} sind die Koeffizientenfunktionen, die man braucht, schon aufgeliste, und zwar mit
\begin{align*}
	w(x) &= \frac{1}{\sqrt{1-x^2}} \\
	p(x) &= \sqrt{1-x^2} \\
	q(x) &= 0
\end{align*}.
Da die Sturm-Liouville-Gleichung
\begin{equation}
	\label{eq:sturm-liouville-equation-tscheby}
	\frac{d}{dx} (\sqrt{1-x^2} \frac{dy}{dx}) + (0 + \lambda \frac{1}{\sqrt{1-x^2}}) y = 0 
\end{equation}
nun mit den Koeffizientenfunktionen aufgestellt werden kann, bleibt die Frage, ob es sich um ein reguläres oder singuläres Sturm-Liouville-Problem handelt.

\subsubsection*{regulär oder singulär?}
Für das reguläre Problem laut der Definition \ref{def:reguläres_sturm-liouville-problem} muss die funktion $p(x) = \sqrt{1-x^2}$, $p'(x) = -2x$, $q(x) = 0$ und $w(x) = \frac{1}{\sqrt{1-x^2}}$ stetig und reell sein --- und sie sind es auch.
Auf dem Intervall $(-1,1)$ sind die Tschebyscheff-Polynome erster Art mit Hilfe von Hyperbelfunktionen
\begin{equation}
	T_n(x) = \cos n (\arccos x)
\end{equation}.
Für $x>1$ und $x<-1$ sehen die Polynome wie folgt aus:
\begin{equation}
	T_n(x) = \left\{\begin{array}{ll} \cosh (n \arccos x), & x > 1\\
		(-1)^n \cosh (n \arccos (-x)), & x<-1 \end{array}\right.
\end{equation},
jedoch ist die Orthogonalität nur auf dem Intervall $[ -1, 1]$ sichergestellt.
Die nächste Bedingung beinhaltet, dass die Funktion $p(x)$ und $w(x)>0$ sein müssen.
Die Funktion
\begin{equation*}
	p(x)^{-1} = \frac{1}{\sqrt{1-x^2}}
\end{equation*}
ist die gleiche wie $w(x)$ und erfüllt die Bedingung.

\subsubsection*{Randwertproblem}
Für die Verifizierung der Randbedingungen benötigt man erneut $p(x)$.
Da sich die Polynome nur auf dem Intervall $[ -1,1 ]$ orthogonal verhalten, sind $a = -1$ und $b = 1$ gesetzt.
Beim einsetzen in die Randbedingung \ref{eq:randbedingungen}, erhält man
\begin{equation}
\begin{aligned}
	k_a y(-1) + h_a y'(-1) &= 0
	k_b y(-1) + h_b y'(-1) &= 0.
\end{aligned} 
\end{equation}
Die Funktion $y(x)$ und $y'(x)$ sind in diesem Fall die Tschebyscheff Polynome (siehe \label{sub:definiton_der_tschebyscheff-Polynome}).
Es gibt zwei Arten von Tschebyscheff Polynome: die erste Art $T_n(x)$ und die zweite Art $U_n(x)$.
Jedoch beachtet man in diesem Kapitel nur die Tschebyscheff Polynome erster Art (\ref{eq:tschebyscheff-polynome}).
Die Funktion $y(x)$ wird nun mit der Funktion $T_n(x)$ ersetzt und für die Verifizierung der Randbedingung wählt man $n=2$.
Somit erhält man
\begin{equation}
	\begin{aligned}
	k_a T_2(-1) + h_a T_{2}'(-1) &= k_a = 0\\
	k_b T_2(1) + h_b T_{2}'(1) &= k_b = 0.
\end{aligned}
\end{equation}
Ähnlich wie beim Beispiel der Wärmeleitung in einem homogenen Stab kann man, damit die Bedingung $|k_i|^2 + |h_i|^2\ne 0$ erfüllt ist, können beliebige $h_a \ne 0$ und $h_b \ne 0$ gewählt werden.
Somit ist erneut gezeigt, dass die Randbedingungen der Tschebyscheff-Polynome auf die Sturm-Liouville-Randbedingungen erfüllt und alle daraus resultierenden Lösungen orthogonal sind.

\begin{beispiel}
	Die Gleichung \ref{eq:skalar-sturm-liouville} mit $y_m = T_1(x)$ und $y_n(x) = T_2(x)$ eingesetzt sowie $a=-1$ und $b = 1$ ergibt
	\[
	\int_{-1}^{1} w(x) x (2x^2-1) dx = 0.
	\]
\end{beispiel}

 










