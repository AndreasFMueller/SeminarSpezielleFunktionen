%
% tschebyscheff_beispiel.tex
% Author: Réda Haddouche
%
% (c) 2020 Prof Dr Andreas Müller, Hochschule Rapperswil
%

\subsection{Tschebyscheff-Polynome
\label{sturmliouville:sub:tschebyscheff-polynome}}
\rhead{Tschebyscheff-Polynome}
\subsubsection*{Definition der Koeffizientenfunktion}
Im Kapitel \ref{sub:beispiele_sturm_liouville_problem} sind die
Koeffizientenfunktionen, die man braucht, schon aufgelistet:
\begin{align*}
	w(x) &= \frac{1}{\sqrt{1-x^2}}, \\
	p(x) &= \sqrt{1-x^2}, \\
	q(x) &= 0.
\end{align*}
Da die Sturm-Liouville-Gleichung
\begin{equation}
	\label{eq:sturm-liouville-equation-tscheby}
	\frac{d}{dx} (\sqrt{1-x^2} \frac{dy}{dx}) +
	(0 + \lambda \frac{1}{\sqrt{1-x^2}}) y
	=
	0 
\end{equation}
nun mit den Koeffizientenfunktionen aufgestellt werden kann, bleibt die Frage,
ob es sich um ein reguläres oder singuläres Sturm-Liouville-Problem handelt.
Zunächst werden jedoch die Randbedingungen betrachtet.

\subsubsection*{Randwertproblem}
Für die Verifizierung der Randbedingungen benötigt man erneut $p(x)$.
Die Randwerte setzt man $a = -1$ und $b = 1$.
Beim Einsetzen in die Randbedingung \eqref{sturmliouville:eq:randbedingungen},
erhält man
\begin{equation}
	\begin{aligned}
		k_a y(-1) + h_a y'(-1) &= 0\\
		k_b y(-1) + h_b y'(-1) &= 0.
	\end{aligned} 
\end{equation}
Die Funktion $y(x)$ und $y'(x)$ sind in diesem Fall die Tschebyscheff Polynome
(siehe \ref{sub:definiton_der_tschebyscheff-Polynome}).
Die Funktion $y(x)$ wird nun mit der Funktion $T_n(x)$ ersetzt und für die
Verifizierung der Randbedingung wählt man $n=0$.
Somit erhält man
\begin{equation}
	\begin{aligned}
		k_a T_0(-1) + h_a T_{0}'(-1) &= k_a = 0\\
		k_b T_0(1) + h_b T_{0}'(1) &= k_b = 0.
	\end{aligned}
\end{equation}
Ähnlich wie beim Beispiel der Wärmeleitung in einem homogenen Stab kann man,
damit die Bedingung $|k_i|^2 + |h_i|^2\ne 0$ erfüllt ist, beliebige
$h_a \ne 0$ und $h_b \ne 0$ gewählt werden.
Es wird also erneut gezeigt, dass die Randbedingungen $[-1,1]$,
die Sturm-Liouville-Randbedingungen erfüllen.

\subsubsection*{regulär oder singulär?}
Für das reguläre Problem muss laut der
Definition~\ref{sturmliouville:def:reguläres_sturm-liouville-problem} die funktion
$p(x) = \sqrt{1-x^2}$, $p'(x) = -2x$, $q(x) = 0$ und
$w(x) = \frac{1}{\sqrt{1-x^2}}$ stetig und reell sein.
Auf dem Intervall $(-1,1)$ sind die Tschebyscheff-Polynome erster Art
\begin{equation}
	T_n(x)
	=
	\cos n (\arccos x).
\end{equation}
Die nächste Bedingung, laut der Definition \ref{sturmliouville:def:reguläres_sturm-liouville-problem}, beinhaltet, dass die Funktion $p(x)$ und $w(x)>0$ sein
müssen.
Die Funktion
\begin{equation*}
	p(x)^{-1} = \frac{1}{\sqrt{1-x^2}}
\end{equation*}
ist die gleiche wie $w(x)$ und erfüllt die Bedingung.



\begin{beispiel}
	Die Gleichung 
	\[
		\int_{a}^{b} w(x) y_m y_n = 0
	\]
	
	mit
	$y_m(x) = T_1(x)$ und $y_n(x) = T_2(x)$ eingesetzt sowie $a=-1$ und $b = 1$
	ergibt
	\[
	\int_{-1}^{1} w(x) x (2x^2-1) dx = 0.
	\]
\end{beispiel}
