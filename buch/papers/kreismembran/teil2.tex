%
% (c) 2020 Prof Dr Andreas Müller, Hochschule Rapperswil
%

\section{Die Hankel Transformation \label{kreismembran:section:teil2}}
\rhead{Die Hankel Transformation}

Hermann Hankel (1839--1873) war ein deutscher Mathematiker, der für seinen Beitrag zur mathematischen Analysis und insbesondere für die nach ihm benannte Transformation bekannt ist.
Diese Transformation tritt bei der Untersuchung von Funktionen auf, die nur von der Entfernung des Ursprungs abhängen.
Er untersuchte auch Funktionen, jetzt Hankel- oder Bessel-Funktionen genannt, der dritten Art.
Die Hankel-Transformation, die die Bessel-Funktion enthält, taucht natürlich bei achsensymmetrischen Problemen auf, die in zylindrischen Polarkoordinaten formuliert sind.
In diesem Abschnitt werden die Theorie der Transformation und einige Eigenschaften der Grundoperationen erläutert.

\subsubsection{Definition der Hankel-Transformation \label{subsub:hankel_tansformation}}
Wir führen die Definition der Hankel-Transformation \cite{lokenath_debnath_integral_2015} aus der zweidimensionalen Fourier-Trans\-formation und ihrer Umkehrung ein, die durch:
\begin{align}
	\mathscr{F}\{f(x,y)\} & = F(k,l)=\frac{1}{2\pi}\int_{-\infty}^{\infty}\int_{-\infty}^{\infty}e^{-i( \bm{\kappa}\cdot \mathbf{r})}f(x,y) \; dx \; dy,\label{equation:fourier_transform}\\
	\mathscr{F}^{-1}\{F(x,y)\} & = f(x,y)=\frac{1}{2\pi}\int_{-\infty}^{\infty}\int_{-\infty}^{\infty}e^{i(\bm{\kappa}\cdot \mathbf{r})}F(k,l) \; dx \; dy \label{equation:inv_fourier_transform}
\end{align}
definiert ist, wobei $\mathbf{r}=(x,y)$ und $\bm{\kappa}=(k,l)$. Polarkoordinaten sind für diese Art von Problem am besten geeignet. Mit $(x,y)=r(\cos\theta,\sin\theta)$ und $(k,l)=\kappa(\cos\phi,\sin\phi)$ findet man $\bm{\kappa}\cdot\mathbf{r}=\kappa r(\cos(\theta-\phi))$ und danach:
\begin{align}
	F(k,\phi)=\frac{1}{2\pi}\int_{0}^{\infty}r \; dr \int_{0}^{2\pi}e^{-ikr\cos(\theta-\phi)}f(r,\theta) \; d\phi.
	\label{equation:F_ohne_variable_wechsel}
\end{align}
Dann wird angenommen, dass $f(r,\theta)=e^{in\theta}f(r)$, was keine strenge Einschränkung ist, weil die \textit{Fourier-Theorie} besagt, dass sich jede Funktion durch Überlagerung solcher Terme darstellen lässt. Es wird auch eine Änderung der Variabeln vorgenommen $\theta-\phi=\alpha-\frac{\pi}{2}$, um \eqref{equation:F_ohne_variable_wechsel} zu reduzieren:
\begin{align}
	F(k,\phi)=\frac{1}{2\pi}\int_{0}^{\infty}rf(r) \; dr \int_{\phi_{0}}^{2\pi+\phi_{0}}e^{in(\phi-\frac{\pi}{2})+i(n\alpha-kr\sin\alpha)} \; d\alpha,
	\label{equation:F_ohne_bessel}
\end{align}
wo $\phi_{0}=(\frac{\pi}{2}-\phi)$.

Unter Verwendung der Integraldarstellung
\begin{equation*}
	J_n(\kappa r)=\frac{1}{2\pi}\int_{\phi_{0}}^{2\pi + \phi_{0}}e^{i(n\alpha-\kappa r \sin \alpha)} \; d\alpha
	\label{equation:bessel_n_ordnung}
\end{equation*}
 der Bessel-Funktion vom Ordnung $n$ \eqref{buch:fourier:eqn:bessel-integraldarstellung} wird \eqref{equation:F_ohne_bessel} zu:
\begin{align}
	F(k,\phi)&=e^{in(\phi-\frac{\pi}{2})}\int_{0}^{\infty}rJ_n(\kappa r) f(r) \; dr  \nonumber \\ 
	&=e^{in(\phi-\frac{\pi}{2})}\tilde{f}_n(\kappa),
	\label{equation:F_mit_bessel_step_2}
\end{align}
wo $\tilde{f}_n(\kappa)$ ist die \textit{Hankel-Transformation} von $f(r)$ und ist formell definiert durch:
\begin{align}
	\mathscr{H}_n\{f(r)\}=\tilde{f}_n(\kappa)=\int_{0}^{\infty}rJ_n(\kappa r) f(r) \; dr.
	\label{equation:hankel}
\end{align}

\subsubsection{Inverse Hankel-Transformation \label{subsub:inverse_hankel_tansformation}}
Wie bei der Entwicklung der Hankel-Transformation können auch für die Umkehrformel Analogien zur Fourier-Transformation hergestellt werden. Vergleicht man die beiden Transformationen, so stellt man fest, dass sie sehr ähnlich sind, wenn man den Term $J_n(\kappa r)$ der Hankel-Transformation durch $e^{-i( \bm{\kappa}\cdot \mathbf{r})}$ der Fourier-Transformation ersetzt. Diese beide Funktionen sind orthogonal, und bei orthogonalen Matrizen genügt bekanntlich die Transponierung, um sie zu invertieren. Da das Skalarprodukt der Bessel-Funktionen jedoch nicht dasselbe ist wie das der Exponentialfunktionen, muss man durch $\kappa\; d\kappa$ statt nur durch $d\kappa$ integrieren, um die Umkehrfunktion zu erhalten.

Die inverse \textit{Hankel-Transformation} ist also als 
\begin{align}
	\mathscr{H}^{-1}_n\{\tilde{f}_n(\kappa)\}=f(r)=\int_{0}^{\infty}\kappa J_n(\kappa r) \tilde{f}_n(\kappa) \; d\kappa.
	\label{equation:inv_hankel}
\end{align}
definiert.

Die Integrale \eqref{equation:hankel} und \eqref{equation:inv_hankel} existieren für bestimmte grosse Klassen von Funktionen, die normalerweise in physikalischen Anwendungen vorkommen.

Alternativ dazu kann die berühmte Hankel-Integralformel 

\begin{align*}
	f(r) = \int_{0}^{\infty}\kappa J_n(\kappa r) \; d\kappa \int_{0}^{\infty} p J_n(\kappa p)f(p) \; dp,
	\label{equation:hankel_integral_formula}
\end{align*}
verwendet werden, um die Hankel-Transformation \eqref{equation:hankel} und ihre Umkehrung \eqref{equation:inv_hankel} zu definieren.

Insbesondere die Hankel-Transformation der nullten Ordnung ($n=0$) und der ersten Ordnung ($n=1$) sind häufig nützlich, um Lösungen für Probleme mit der Laplace Gleichung in einer achsensymmetrischen zylindrischen Geometrie zu finden.

\subsection{Operatoreigenschaften der Hankel-Transformation \label{sub:op_properties_hankel}}
In diesem Kapitel werden die operativen Eigenschaften der Hankel-Transformation aufgeführt. Die Beweise für ihre Gültigkeit werden jedoch nicht analysiert, diese sind im Buch \textit{Integral Tansforms and Their Applications} \cite{lokenath_debnath_integral_2015} zu finden.

\begin{satz}{Skalierung:}
	Wenn $\mathscr{H}_n\{f(r)\}=\tilde{f}_n(\kappa)$, dann gilt:
	
	\begin{equation*}
		\mathscr{H}_n\{f(ar)\}=\frac{1}{a^{2}}\tilde{f}_n \left(\frac{\kappa}{a}\right), \quad a>0.
	\end{equation*}
\end{satz}

\begin{satz}{Parsevalsche Relation:}
Wenn $\tilde{f}(\kappa)=\mathscr{H}_n\{f(r)\}$ und $\tilde{g}(\kappa)=\mathscr{H}_n\{g(r)\}$, dann gilt:

\begin{equation*}
	\int_{0}^{\infty}rf(r)g(r) \; dr = \int_{0}^{\infty}\kappa\tilde{f}(\kappa)\tilde{g}(\kappa) \; d\kappa.
\end{equation*}
\end{satz}

\begin{satz}{Hankel-Transformationen von Ableitungen:}
Wenn $\tilde{f}_n(\kappa)=\mathscr{H}_n\{f(r)\}$, dann gilt:

\begin{align*}
	&\mathscr{H}_n\{f'(r)\}=\frac{\kappa}{2n}\left[(n-1)\tilde{f}_{n+1}(\kappa)-(n+1)\tilde{f}_{n-1}(\kappa)\right], \quad n\geq1, \\
	&\mathscr{H}_1\{f'(r)\}=-\kappa \tilde{f}_0(\kappa),
\end{align*}
vorausgesetzt, dass $rf(r)$ verschwindet wenn $r\to0$ und $r\to\infty$.
\end{satz}

\begin{satz}
Wenn $\mathscr{H}_n\{f(r)\}=\tilde{f}_n(\kappa)$, dann gilt:

\begin{equation*}
	\mathscr{H}_n \left\{ \left( \nabla^2 - \frac{n^2}{r^2} f(r)\right)\right\}= \mathscr{H}_n\left\{\frac{1}{r}\frac{d}{dr}\left(r\frac{df}{dr}\right) - \frac{n^2}{r^2}f(r)\right\}=-\kappa^2\tilde{f}_{n}(\kappa),
\end{equation*}
bereitgestellt, dass $rf'(r)$ und $rf(r)$ verschwinden für $r\to0$ und $r\to\infty$.
\end{satz}
