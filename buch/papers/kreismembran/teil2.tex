%
% (c) 2020 Prof Dr Andreas Müller, Hochschule Rapperswil
%

\section{Lösung der partiellen Differentialgleichung
	\label{kreismembran:section:teil2}}
\rhead{Lösung der partiellen Differentialgleichung}

Wie im vorherigen Kapitel gezeigt, lautet die partielle Differentialgleichung, die die Schwingungen einer Membran beschreibt:
\begin{equation*}
	\frac{1}{c^2}\frac{\partial^2u}{\partial t^2} = \Delta u
\end{equation*}
Da es sich um eine Kreisscheibe handelt, werden Polarkoordinaten verwendet, so dass sich der Laplaceoperator ergibt:
\begin{equation*}
	\Delta
	=
	\frac{\partial^2}{\partial r^2}
	+
	\frac1r
	\frac{\partial}{\partial r}
	+
	\frac{1}{r 2}
	\frac{\partial^2}{\partial\varphi^2}.
	\label{buch:pde:kreis:laplace}
\end{equation*}

Es wird eine runde elastische Membran berücksichtigt, die den Gebietbereich $\Omega$ abdeckt und am Rand $\Gamma$ befestigt ist.
Es wird daher davon ausgegangen, dass die Membran aus einem homogenen Material von vernachlässigbarer Dicke gefertigt ist.
Die Membran kann verformt werden, aber innere elastische Kräfte wirken den Verformungen entgegen. Es wirken keine äusseren Kräfte. Es handelt sich somit von einer kreisförmligen eigespannten homogenen schwingenden Membran.

Daher ist die Membranabweichung im Punkt $(r,\theta)$ $\in$ $\overline{\rm \Omega}$ zum Zeitpunkt $t$:
\begin{align*}
	u: \overline{\rm \Omega} \times \mathbb{R}_{\geq 0} &\longrightarrow \mathbb{R}\\
	(r,\theta,t) &\longmapsto u(r,\theta,t)
\end{align*}
Da die Membran am Rand befestigt ist, kann es keine Schwingungen geben, so dass die \textit{Dirichlet-Randbedingung} gilt:
\begin{equation*}
	u\big|_{\Gamma} = 0
\end{equation*}


Um eine eindeutige Lösung bestimmen zu können, werden die folgenden Anfangsbedingungen festgelegt:

\begin{align*}
	u(r,\theta, 0) &:= f(x,y)\\
	\frac{\partial}{\partial t} u(r,\theta, 0) &:= g(x,y)
\end{align*}

An dieser Stelle könnte man zum Beispiel die bereits in Kapitel (TODO:refKAPITEL) vorgestellte Methode der Separation anwenden. Da es sich in diesem Fall jedoch um einem achsensymmetrischen Problem handelt, das in Polarkoordinaten formuliert ist, wird man die Transformationsmethode verwenden, insbesondere die Hankel Transformation.
