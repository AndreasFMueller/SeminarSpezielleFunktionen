%
% (c) 2020 Prof Dr Andreas Müller, Hochschule Rapperswil
%

\section{Die Hankel Transformation \label{kreismembran:section:teil2}}
\rhead{Die Hankel Transformation}

Hermann Hankel (1839-1873) war ein deutscher Mathematiker, der für seinen Beitrag zur mathematischen Analyse und insbesondere für seine namensgebende Transformation bekannt ist.
Diese Transformation tritt bei der Untersuchung von funktionen auf, die nur von der Enternung des Ursprungs abhängen.
Er studierte auch funktionen, jetzt Hankel- oder Bessel- Funktionen genannt, der dritten Art.
Die Hankel Transformation mit Bessel Funktionen al Kern taucht natürlich bei achsensymmetrischen Problemen auf, die in Zylindrischen Polarkoordinaten formuliert sind.
In diesem Kapitel werden die Theorie der Transformation und einige Eigenschaften der Grundoperationen erläutert.


Wir führen die Definition der Hankel Transformation aus der zweidimensionalen Fourier Transformation und ihrer Umkehrung ein, die durch:
\begin{align}
	\mathscr{F}\{f(x,y)\} & = F(k,l)=\frac{1}{2\pi}\int_{-\infty}^{\infty}\int_{-\infty}^{\infty}e^{-i( \bm{\kappa}\cdot \mathbf{r})}f(x,y) dx dy,\label{equation:fourier_transform}\\
	\mathscr{F}^{-1}\{F(x,y)\} & = f(x,y)=\frac{1}{2\pi}\int_{-\infty}^{\infty}\int_{-\infty}^{\infty}e^{i(\bm{\kappa}\cdot \mathbf{r}))}F(k,l) dx dy \label{equation:inv_fourier_transform}
\end{align}
wo $\mathbf{r}=(x,y)$ und $\bm{\kappa}=(k,l)$. Wie bereits erwähnt, sind Polarkoordinaten für diese Art von Problemen am besten geeignet, also mit, $(x,y)=r(\cos\theta,\sin\theta)$ und $(k,l)=\kappa(\cos\phi,\sin\phi)$, findet man $\bm{\kappa}\cdot\mathbf{r}=\kappa r(\cos(\theta-\phi))$ und danach:
\begin{align}
	F(k,\phi)=\frac{1}{2\pi}\int_{0}^{\infty}r dr \int_{0}^{2\pi}e^{-ikr\cos(\theta-\phi)}f(r,\theta) d\phi.
	\label{equation:F_ohne_variable_wechsel}
\end{align}
Dann wird angenommen dass, $f(r,\theta)=e^{in\theta}f(r)$, was keine strenge Einschränkung ist, und es wird eine Änderung der Variabeln vorgenommen $\theta-\phi=\alpha-\frac{\pi}{2}$, um \eqref{equation:F_ohne_variable_wechsel} zu reduzieren:
\begin{align}
	F(k,\phi)=\frac{1}{2\pi}\int_{0}^{\infty}rf(r) dr \int_{\phi_{0}}^{2\pi+\phi_{0}}e^{in(\phi-\frac{\pi}{2})+i(n\alpha-kr\sin\alpha)} d\alpha,
	\label{equation:F_ohne_bessel}
\end{align}
wo $\phi_{0}=(\frac{\pi}{2}-\phi)$.

Unter Verwendung der Integral Darstellung der Besselfunktion vom Ordnung n 
\begin{align}
	J_n(\kappa r)=\frac{1}{2\pi}\int_{\phi_{0}}^{2\pi + \phi_{0}}e^{i(n\alpha-\kappa r \sin \alpha)} d\alpha
	\label{equation:bessel_n_ordnung}
\end{align}
\eqref{equation:F_ohne_bessel} wird sie zu:
\begin{align}
	F(k,\phi)&=e^{in(\phi-\frac{\pi}{2})}\int_{0}^{\infty}rJ_n(\kappa r) f(r) dr \label{equation:F_mit_bessel_step_1} \\
	&=e^{in(\phi-\frac{\pi}{2})}\tilde{f}_n(\kappa),
	\label{equation:F_mit_bessel_step_2}
\end{align}
wo $\tilde{f}_n(\kappa)$ ist die \textit{Hankel Transformation} von $f(r)$ und ist formell definiert durch:
\begin{align}
	\mathscr{H}_n\{f(r)\}=\tilde{f}_n(\kappa)=\int_{0}^{\infty}rJ_n(\kappa r) f(r) dr.
	\label{equation:hankel}
\end{align}

Ähnlich verhält es sich mit der inversen Fourier Transformation in Form von polaren Koordinaten unter der Annahme $f(r,\theta)=e^{in\theta}f(r)$ mit \eqref{equation:F_mit_bessel_step_2}, wird die inverse Fourier Transformation \eqref{equation:inv_fourier_transform}:

\begin{align}
	e^{in\theta}f(r)&=\frac{1}{2\pi}\int_{0}^{\infty}\kappa d\kappa \int_{0}^{2\pi}e^{i\kappa r \cos (\theta - \phi)}F(\kappa,\phi) d\phi\\
	&= \frac{1}{2\pi}\int_{0}^{\infty}\kappa \tilde{f}_n(\kappa) d\kappa \int_{0}^{2\pi}e^{in(\phi - \frac{\pi}{2})- i\kappa r \cos (\theta - \phi)} d\phi,
\end{align}
was durch den Wechsel der Variablen $\theta-\phi=-(\alpha+\frac{\pi}{2})$ und $\theta_0=-(\theta+\frac{\pi}{2})$,

\begin{align}
	&= \frac{1}{2\pi}\int_{0}^{\infty}\kappa \tilde{f}_n(\kappa) d\kappa \int_{\theta_0}^{2\pi+\theta_0}e^{in(\theta + \alpha - i\kappa r \sin\alpha)} d\alpha \nonumber \\
	&= e^{in\theta}\int_{0}^{\infty}\kappa J_n(\kappa r) \tilde{f}_n(\kappa) d\kappa,\quad \text{von \eqref{equation:bessel_n_ordnung}}
\end{align}

Also, die inverse \textit{Hankel Transformation} ist so definiert:
\begin{align}
	\mathscr{H}^{-1}_n\{\tilde{f}_n(\kappa)\}=f(r)=\int_{0}^{\infty}\kappa J_n(\kappa r) \tilde{f}_n(\kappa) d\kappa.
	\label{equation:inv_hankel}
\end{align}

Anstelle von $\tilde{f}_n(\kappa)$, wird häufig für die Hankel Transformation verwendet, indem die Ordnung angegeben wird.
\eqref{equation:hankel} und \eqref{equation:inv_hankel} Integralen existieren für eine grosse Klasse von Funktionen, die normalerweise in physikalischen Anwendungen benötigt werden.
Alternativ kann auch die berühmte Hankel Transformationsformel verwendet werden, 

\begin{align}
	f(r) = \int_{0}^{\infty}\kappa J_n(\kappa r) d\kappa \int_{0}^{\infty} p J_n(\kappa p)f(p) dp,
	\label{equation:hankel_integral_formula}
\end{align}
um die Hankel Transformation \eqref{equation:hankel} und ihre Inverse \eqref{equation:inv_hankel} zu definieren.
Insbesondere die Hankel Transformation der nullten Ordnung ($n=0$) und der ersten Ordnung ($n=1$) sind häufig nützlich, um Lösungen für Probleme mit der Laplace Gleichung in einer achsensymmetrischen zylindrischen Geometrie zu finden.

\subsection{Operative Eigenschaften der Hankel Transformation\label{sub:op_properties_hankel}}
In diesem Kapitel werden die operativen Eigenschaften der Hankel Transformation aufgeführt. Der Beweis für ihre Gültigkeit wird jedoch nicht analysiert.

\subsubsection{Theorem 1: Skalierung \label{subsub:skalierung}}
Wenn $\mathscr{H}_n\{f(r)\}=\tilde{f}_n(\kappa)$, dann:

\begin{equation*}
	\mathscr{H}_n\{f(ar)\}=\frac{1}{a^{2}}\tilde{f}_n \left(\frac{\kappa}{a}\right), \quad a>0.
\end{equation*}

\subsubsection{Theorem 2: Persevalsche Relation \label{subsub:perseval}}
Wenn $\tilde{f}(\kappa)=\mathscr{H}_n\{f(r)\}$ und $\tilde{g}(\kappa)=\mathscr{H}_n\{g(r)\}$, dann:

\begin{equation*}
	\int_{0}^{\infty}rf(r) dr = \int_{0}^{\infty}\kappa\tilde{f}(\kappa)\tilde{g}(\kappa) d\kappa.
\end{equation*}

\subsubsection{Theorem 3: Hankel Transformationen von Ableitungen \label{subsub:ableitungen}}
Wenn $\tilde{f}_n(\kappa)=\mathscr{H}_n\{f(r)\}$, dann:

\begin{align*}
	&\mathscr{H}_n\{f'(r)\}=\frac{\kappa}{2n}\left[(n-1)\tilde{f}_{n+1}(\kappa)-(n+1)\tilde{f}_{n-1}(\kappa)\right], \quad n\geq1, \\
	&\mathscr{H}_1\{f'(r)\}=-\kappa \tilde{f}_0(\kappa),
\end{align*}
bereitgestellt dass $[rf(r)]$ verschwindet als $r\to0$ und $r\to\infty$.

\subsubsection{Theorem 4 \label{subsub:thorem4}}
Wenn $\mathscr{H}_n\{f(r)\}=\tilde{f}_n(\kappa)$, dann:

\begin{equation*}
	\mathscr{H}_n \left\{ \left( \nabla^2 - \frac{n^2}{r^2} f(r)\right)\right\}= \mathscr{H}_n\left\{\frac{1}{r}\frac{d}{dr}\left(r\frac{df}{dr}\right) - \frac{n^2}{r^2}f(r)\right\}=-\kappa^2\tilde{f}_{n}(\kappa),
\end{equation*}
bereitgestellt dass $rf'(r)$ und $rf(r)$ verschwinden als $r\to0$ und $r\to\infty$.


