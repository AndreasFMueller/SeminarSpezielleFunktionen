%
% teil3.tex -- Beispiel-File für Teil 3
%
% (c) 2020 Prof Dr Andreas Müller, Hochschule Rapperswil
%
\section{Lösungsmethode 2: Transformationsmethode
\label{kreismembran:section:teil3}}
\rhead{Lösungsmethode 2: Transformationsmethode}
Die Hankel-Transformation wird dann zur Lösung der Differentialgleichung verwendet. Es müssen jedoch einige Änderungen an dem Problem vorgenommen werden, damit es mit den Annahmen übereinstimmt, die für die Verwendung der Hankel-Transformation erforderlich sind. Das heisst, dass die Funktion $u$ nur von der Entfernung zum Ausgangspunkt abhängt. 

\subsubsection{Transformation und Reduktion auf eine algebraische Gleichung\label{subsub:transf_reduktion}}
Führt man also das Konzept einer unendlichen und achsensymmetrischen Membran ein:
\begin{equation*}
	\frac{\partial^2u}{\partial t^2}
	=
	c^2  \left(\frac{\partial^2 u}{\partial r^2}
	+
	\frac{1}{r}
	\frac{\partial u}{\partial r} \right), \quad 0<r<\infty, \quad t>0
	\label{eq:PDE_inf_membane}
\end{equation*}

\begin{align}
	u(r,0)=f(r), \quad u_t(r,0) = g(r), \quad \text{für} \quad 0<r<\infty
	\label{eq:PDE_inf_membane_RB}
\end{align}

Mit Anwendung der Hankel-Transformation nullter Ordnung in Abhängigkeit von $r$ auf die Gleichungen \eqref{eq:PDE_inf_membane} und \eqref{eq:PDE_inf_membane_RB}:

\begin{align}
	\tilde{u}(\kappa,t)=\int_{0}^{\infty}r J_0(\kappa r)u(r,t) \; dr,
\end{align}
bekommt man:

\begin{equation*}
	\frac{d^2 \tilde{u}}{dt^2} + c^2\kappa^2\tilde{u}=0,
\end{equation*}

\begin{equation*}
	\tilde{u}(\kappa,0)=\tilde{f}(\kappa), \quad 
	\tilde{u}_t(\kappa,0)=\tilde{g}(\kappa).
\end{equation*}
Die allgemeine Lösung für diese Transformation lautet, wie in Gleighung \eqref{eq:cos_sin_überlagerung} gesehen, wie folgt

\begin{equation*}
	\tilde{u}(\kappa,t)=\tilde{f}(\kappa)\cos(c\kappa t) + \frac{1}{c\kappa}\tilde{g}(\kappa)\sin(c\kappa t).
\end{equation*}
Wendet man an nun die inverse Hankel-Transformation an, so erhält man die formale Lösung

\begin{align}
	u(r,t)=\int_{0}^{\infty}\kappa\tilde{f}(\kappa)\cos(c\kappa t) J_0(\kappa r) \; d\kappa +\frac{1}{c}\int_{0}^{\infty}\tilde{g}(\kappa)\sin(c\kappa t)J_0(\kappa r) \; d\kappa.
	\label{eq:formale_lösung}
\end{align}

\subsubsection{Erfüllung der Anfangsbedingungen\label{subsub:erfüllung_AB}}
Es wird in Folgenden davon ausgegangen, dass sich die Membran verformt und zum Zeitpunkt $t=0$ freigegeben wird

\begin{equation*}
	u(r,0)=f(r)=Aa(r^2 + a^2)^{-\frac{1}{2}}, \quad u_t(r,0)=g(r)=0
\end{equation*}
so dass $\tilde{g}(\kappa)\equiv 0$ und
\begin{equation*}
	\tilde{f}(\kappa)=Aa\int_{0}^{\infty}r(a^2 + r^2)^{-\frac{1}{2}} J_0 (\kappa r) \; dr=\frac{Aa}{\kappa}e^{-a\kappa}
\end{equation*}
Die formale Lösung  \eqref{eq:formale_lösung} lautet also
\begin{align*}
	u(r,t)&=Aa\int_{0}^{\infty}e^{-a\kappa} J_0(\kappa r)\cos(c\kappa t) \; dk=AaRe\int_{0}^{\infty}e^{-\kappa(a+ict)} J_0(\kappa r) \; dk\\
	&=AaRe\left\{r^2+\left(a+ict\right)^2\right\}^{-\frac{1}{2}}
\end{align*}

Nimmt man jedoch die allgemeine Lösung mit Summationen, 

\begin{align}
	u(r, t) = \displaystyle\sum_{m=1}^{\infty} J_0 (k_{m}r)[a_{m}\cos(c \kappa_{m} t)+b_{m}\sin(c \kappa_{m} t)]
	\label{eq:lösung_unendliche_generelle}
\end{align}
kann man die Lösungsmethoden 1 und 2 vergleichen.

\subsection{Vergleich der Lösungen
\label{kreismembran:vergleich}}
Bei der Analyse der Gleichungen \eqref{eq:lösung_endliche_generelle} und \eqref{eq:lösung_unendliche_generelle} fällt sofort auf, dass die Gleichung \eqref{eq:lösung_unendliche_generelle} nicht mehr von $m$ und $n$ abhängt, sondern nur noch von $n$ \cite{nishanth_p_vibrations_2018}. Das macht Sinn, denn $n$ beschreibt die Anzahl der Knotenlinien, und in einer unendlichen Membran gibt es keine. Tatsächlich werden $a_{m0}$, $b_{m0}$ und $\kappa_{m0}$ in $a_m$, $b_m$ bzw. $\kappa_m$ umbenannt. Die beiden Termen $\cos(n\varphi)$ und $\sin(n\varphi)$ verschwinden ebenfalls, da für $n=0$ der $\cos(n\varphi)$ gleich 1 und der $\sin(n \varphi)$ gleich 0 ist.
Die Funktion hängt also nicht mehr von der Besselfunktionen $n$-ter Ordnung ab, sondern nur von der $0$-ter Ordnung. 


