%
% teil1.tex -- Beispiel-File für das Paper
%
% (c) 2020 Prof Dr Andreas Müller, Hochschule Rapperswil


\section{Lösungsmethode 1: Separationsmethode 
	\label{kreismembran:section:teil1}}
\rhead{Lösungsmethode 1: Separationsmethode}
An diesem Punkt bleibt also nur noch die Lösung der partiellen Differentialgleichung. In diesem Kapitel wird sie mit Hilfe der Separationsmethode gelöst.

\subsection{Aufgabestellung\label{sub:aufgabestellung}}
Wie im vorherigen Abschnitt gezeigt, lautet die partielle Differentialgleichung, die die Schwingungen einer Membran beschreibt:
\begin{equation*}
	\frac{1}{c^2}\frac{\partial^2u}{\partial t^2} = \Delta u.
\end{equation*}
Da es sich um eine Kreisscheibe handelt, werden Polarkoordinaten verwendet, so dass sich der Laplaceoperator
\begin{equation*}
	\Delta
	=
	\frac{\partial^2}{\partial r^2}
	+
	\frac1r
	\frac{\partial}{\partial r}
	+
	\frac{1}{r 2}
	\frac{\partial^2}{\partial\varphi^2}
	\label{buch:pde:kreis:laplace}
\end{equation*}
ergibt.

Es wird eine runde elastische Membran berücksichtigt, die das Gebiet $\Omega$ abdeckt und am Rand $\Gamma$ befestigt ist.
Es wird daher davon ausgegangen, dass die Membran aus einem homogenen Material von vernachlässigbarer Dicke gefertigt ist.
Die Membran kann verformt werden, aber innere elastische Kräfte wirken den Verformungen entgegen. Es wirken keine äusseren Kräfte. Es handelt sich somit von einer kreisförmligen eingespannten homogenen schwingenden Membran.

Daher ist die Membranabweichung im Punkt $(r,\varphi)$ $\in$ $\overline{\rm \Omega}$ zum Zeitpunkt $t$:
\begin{align*}
	u: \overline{\rm \Omega} \times \mathbb{R}_{\geq 0} &\longrightarrow \mathbb{R}\\
	(r,\varphi,t) &\longmapsto u(r,\varphi,t)
\end{align*}
Da die Membran am Rand befestigt ist, kann es keine Schwingungen geben, so dass die \textit{Dirichlet-Randbedingung} \cite{prof_dr_horst_knorrer_kreisformige_2013}
\begin{equation*}
	u\big|_{\Gamma} = 0 \quad \text{für} \quad 0 \leq \varphi \leq 2\pi,\quad t \geq 0
\end{equation*}
gilt.

Um eine eindeutige Lösung bestimmen zu können, werden die folgenden Anfangsbedingungen festgelegt:
\begin{align*}
	u(r,\varphi, 0) &= f(r,\varphi)\\
	u_t(r,\varphi, 0) &= g(r,\varphi).
\end{align*}

\subsection{Lösung\label{sub:lösung1}}
\subsubsection{Ansatz der Separation der Variablen\label{subsub:ansatz_separation}}
Daher muss an dieser Stelle von einer Separation der Variablen ausgegangen werden:
\begin{equation*}
	u(r,\varphi, t) = F(r)G(\varphi)T(t)
\end{equation*}
Dank der Randbedingungen kann also gefordert werden, dass $F(R)=0$ ist, und natürlich, dass $G(\varphi)$ $2\pi$ periodisch ist. Eingesetzt in der Differenzialgleichung ergibt sich:
\begin{equation*}
	\frac{1}{c^2}\frac{T''(t)}{T(t)}=\frac{F''(r)}{F(r)}+\frac{1}{r}\frac{F'(r)}{F(r)}+\frac{1}{r^2}\frac{G''(\varphi)}{G(\varphi)}.
\end{equation*}
Da die linke Seite nur von $t$ und die rechte Seite nur von $r$ und $\varphi$ abhängt, müssen sie gleich einer reellen Zahl sein. Aus physikalischen Gründen suchen wir nach Lösungen, die weder exponentiell in der Zeit wachsen noch exponentiell abklingen. Dies bedeutet, dass die Konstante negativ sein muss, also schreibt man $k=-k^2$. Daraus ergeben sich die folgenden zwei Gleichungen:
\begin{align*}
	T''(t) + c^2\kappa^2T(t) &= 0\\
	r^2\frac{F''(r)}{F(r)} + r \frac{F'(r)}{F(r)} +\kappa^2 r^2 &= - \frac{G''(\varphi)}{G(\varphi)}.
\end{align*}
In der zweiten Gleichung hängt die linke Seite nur von $r$ ab, während die rechte Seite nur von $\varphi$ abhängt. Sie müssen also wiederum gleich einer reellen Zahl $\nu$ sein. Also das:
\begin{align*}
	r^2F''(r) + rF'(r) + (\kappa^2 r^2 - \nu)F(r) &= 0 \\
	G''(\varphi) &= \nu G(\varphi).
\end{align*}

\subsubsection{Lösung für $G(\varphi)$\label{subsub:lösung_G}}
Da für die Zweite Gelichung Lösungen von Schwingungen erwartet werden, für die $G''(\varphi)=-\omega^2 G(\varphi)$ gilt, schreibt die gemeinsame Konstante als $-\nu^2$, was die Formeln später vereinfacht. Also:
\begin{equation*}
 G(\varphi) = C_n \cos(\varphi) + D_n \sin(\varphi)
 \label{eq:cos_sin_überlagerung}
\end{equation*}

\subsubsection{Lösung für $F(r)$\label{subsub:lösung_F}}
Die Gleichung für $F$ hat die Gestalt
\begin{align}
	r^2F''(r) + rF'(r) + (\kappa^2 r^2 - n^2)F(r) = 0 
	\label{eq:2nd_degree_PDE}
\end{align}
Wir bereits in Kapitel \ref{buch:differntialgleichungen:section:bessel} gezeigt, sind die Besselfunktionen
\begin{equation*}
	J_{\nu}(x) = r^\nu \displaystyle\sum_{m=0}^{\infty} \frac{(-1)^m x^{2m}}{2^{2m+\nu}m! \Gamma (\nu + m+1)}
\end{equation*}
Lösungen der Besselschen Differenzialgleichung
\begin{equation*}
	x^2 y'' + xy' + (x^2 - \nu^2)y = 0
\end{equation*}
Die Funktionen $F(r) = J_n(\kappa r)$ lösen also die Differentialgleichung \eqref{eq:2nd_degree_PDE}. Die
Randbedingung $F(R)=0$ impliziert, dass $\kappa R$ eine Nullstelle der Besselfunktion
$J_n$ sein muss. Man kann zeigen, dass die Besselfunktionen $J_n, n \geq 0$, alle unendlich
viele Nullstellen
\begin{equation*}
	\alpha_{1n} < \alpha_{2n} < ...
\end{equation*}
haben, und dass $\underset{\substack{m\to\infty}}{\text{lim}} \alpha_{mn}=\infty$. Somit ergibt sich, dass $\kappa = \frac{\alpha_{mn}}{R}$ für ein $m\geq 1$, und dass
\begin{equation*}
	F(r) = J_n (\kappa_{mn}r) \quad \text{mit} \quad \kappa_{mn}=\frac{\alpha_{mn}}{R}
\end{equation*}

\subsubsection{Lösung für $T(t)$\label{subsub:lösung_T}}
Die Differenzialgleichung $T''(t) + c^2\kappa^2T(t) = 0$, wird auf ähnliche Weise gelöst wie $G(\varphi)$. 

\subsubsection{Zusammenfassung der Lösungen\label{subsub:zusammenfassung_lösungen}}
Durch Überlagerung aller Ergebnisse erhält man die Lösung
\begin{align}
	u(r, \varphi, t) = \displaystyle\sum_{m=1}^{\infty}\displaystyle\sum_{n=0}^{\infty} J_n (k_{mn}r)[a_{mn}\cos(n\varphi) + b_{mn}\sin(n\varphi)](n\varphi)[c_{mn}\cos(c \kappa_{mn} t)+d_{mn}\sin(c \kappa_{mn} t)]
	\label{eq:lösung_endliche_generelle}
\end{align}

Dabei sind $m$ und $n$ ganze Zahlen, wobei $m$ für die Anzahl der Knotenkreise und $n$
für die Anzahl der Knotenlinien steht. Es gibt bestimmte Bereiche auf der Membran, in denen es keine Bewegung oder Vibration gibt. Wenn der nicht schwingende Bereich ein Kreis ist, nennt man ihn einen Knotenkreis, und wenn er eine Linie ist, nennt man ihn ebenfalls eine Knotenlinie. $Jn(\kappa_{mn}r)$ ist die Besselfunktion $n$-ter Ordnung, wobei $\kappa mn$ die Wellenzahl und $r$ der Radius ist. $a_{mn}$ und $b_{mn}$ sind die zu bestimmenden Konstanten.


An diesem Punkt stellte sich die Frage, ob es möglich wäre, die partielle Differentialgleichung mit einer anderen Methode als der der Trennung der Variablen zu lösen. Nach einer kurzen Recherche wurde festgestellt, dass die beste Methode die Transformationsmethode ist, genauer gesagt die Anwendung der Hankel-Transformation. Im nächsten Kapitel wird daher diese Integraltransformation vorgestellt und entwickelt, und es wird erläutert, warum sie für diese Art von Problem geeignet ist.
