%
% einleitung.tex -- Beispiel-File für die Einleitung
%
% (c) 2020 Prof Dr Andreas Müller, Hochschule Rapperswil
%
\section{Lösungsmethode 3: Simulation 
	\label{kreismembran:section:teil4}}
\paragraph{TODO Einleitung}

Um numerisch das Verhalten einer Membran zu ermitteln, muss eine numerische Darstellung definiert werden.
Die Membran wird hier in Form der Matrix $  A $ digitalisiert.
Jedes Element  $ A_{ij} $ steh für die Auslenkung der Membran $ u(x,y,t) $ an der Stelle $ \{x,y\}=\{i,j\} $.
Die zeitliche Dimension wird in Form des Array $  X[] $ aus $ v \times A $ Matrizen dargestellt.
Das Element auf Zeile $ i $, Spalte $ j $ der $ w $-ten Matrix von $ X[] $ also $ X[w]_{ij} $ entspricht der Auslenkung $ u(i,j,w) $.

\paragraph{title}