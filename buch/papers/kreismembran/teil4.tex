%
% einleitung.tex -- Beispiel-File für die Einleitung
%
% (c) 2020 Prof Dr Andreas Müller, Hochschule Rapperswil
%
\section{Lösungsmethode 3: Simulation 
	\label{kreismembran:section:teil4}}

Um numerisch das Verhalten einer Membran zu ermitteln, muss eine numerische Darstellung definiert werden.
Die Membran wird hier in Form der Matrix $  U $ digitalisiert.
Jedes Element  $ U_{ij} $ steh für die Auslenkung der Membran $ u(x,y,t) $ an der Stelle $ \{x,y\}=\{i,j\} $.
Die zeitliche Dimension wird in Form des Array $  U[] $ aus $ z \times U $ Matrizen dargestellt, wobei $ z $ der Anzahl Zeitschritten entspricht.
Das Element auf Zeile $ i $, Spalte $ j $ der $ w $-ten Matrix von $ U[] $ also $ U[w]_{ij} $ entspricht somit der Auslenkung $ u(i,j,w) $.
Da die DGL von Zweiter Ordnung ist, reicht eine Zustandsvariabel pro Membran-Element nicht aus. 
Es wird neben der Auslenkung auch die Geschwindigkeit jedes Membran-Elementes benötigt um den Zustand eindeutig zu beschreiben. 
Dazu existiert neben $ U[] $ ein analoger Array $ V[] $ welcher die Geschwindigkeiten aller Membran-Elementen repräsentiert. 
$ V[w]_{ij} $ entspricht also $ \dot{u}(i,j,w) $. 
Der Zustand einer Membran zum Zeitpunkt $ w $ wird mit $ X[w] $ beschrieben, was $ U[w] $ und $ V[w] $ beinhaltet.

\subsection{Propagation}
Um das Verhalten der Membran zu berechnen, muss aus einem gegebenen Zustand $ X[w] $ der Folgezustand $ X[w+1] $ gerechnet werden können, wobei dazwischen ein Zeitintervall $ dt $ vergeht. 
Die Berechnung von Folgezuständen kann anschliessend repetiert werden über das zu untersuchende Zeitfenster.
Da die Digitale Membran sich wie die analytisch untersuchte verhalten soll, muss auch sie
\begin{equation*}
	\frac{1}{c^2}\frac{\partial^2u}{\partial t^2} = \Delta u
\end{equation*}
erfüllen.
