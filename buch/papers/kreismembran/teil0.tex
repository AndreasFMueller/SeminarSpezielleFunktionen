%
% einleitung.tex -- Beispiel-File für die Einleitung
%
% (c) 2020 Prof Dr Andreas Müller, Hochschule Rapperswil
%
\section{Einleitung\label{kreismembran:section:teil0}}
\rhead{Einleitung}
Eine naheliegende kreisförmige Membrane ist eine Runde Trommel. 
Der Zusammenhang zwischen rund und kreisförmig wird hier nicht erläutert, was in diesem Kapitel als Membrane verstanden wird sollte jedoch erwähnt sein. 
Eine Membrane, Membran oder selten ein Schwingblatt ist laut Duden \cite{kreismembran:Duden:Membrane} ein "dünnes Blättchen aus Metall, Papier o. Ä., das durch seine Schwingungsfähigkeit geeignet ist, Schallwellen zu übertragen". 
Um zu verstehen wie sich eine Kreisförmige Membrane oder eben eine Trommel verhaltet, wird das Verhalten eines infinitesimal kleines Stück einer Membrane untersucht.     

\paragraph{Annahmen} Für die Herleitung einer Differentialgleichung mit überschaubarer Komplexität werden gebräuchliche Annahmen zur Modellierung einer Membrane \cite{kreismembran:wellengleichung_herleitung} getroffen: 
\begin{enumerate}[i]
	\item Die Membrane ist homogen. 
	Dies bedeutet, dass die Membrane über die ganze Fläche die selbe Dichte $ \rho $  und Elastizität hat. 
	Durch die konstante Elastizität ist die ganze Membrane unter gleichmässiger Spannung $ T $.
	\item Die Membrane ist perfekt flexibel. 
	Daraus folgt, dass die Membrane ohne Kraftaufwand verbogen werden kann. 
	Die Membrane ist dadurch nicht alleine schwing-fähig, hierzu muss sie gespannt werden mit der Kraft $ T $.
	\item Die Membrane kann sich nur in Richtung ihrer Normalen in kleinem Ausmass Auslenken.
	Auslenkungen in der ebene der Membrane sind nicht möglich.
	\item Die Membrane erfährt keine Art von Dämpfung. 
	Neben der perfekten Flexibilität wird die Membrane auch nicht durch ihr umliegendes Medium aus gebremst.
	Dadurch entsteht kein dämpfender Term abhängig von der Geschwindigkeit der Membrane in der Differenzialgleichung. 
\end{enumerate}

