%
% teil1.tex -- Mathematischer Hintergrund
%
% (c) 2022 Fabian Dünki, Hochschule Rapperswil
%
\section{Mathematischer Hintergrund
\label{0f1:section:mathHintergrund}}
\rhead{Mathematischer Hintergrund}
Basierend auf den Herleitungen des Abschnittes \ref{buch:rekursion:section:hypergeometrische-funktion}, werden im nachfolgenden Abschnitt nochmals die Resultate
beschrieben.

\subsection{Hypergeometrische Funktion
\label{0f1:subsection:hypergeometrisch}}
Als Grundlage der umgesetzten Algorithmen dient die hypergeometrische Funktion $\mathstrut_0F_1$. Diese ist ein Speziallfall der allgemein definierten Funktion $\mathstrut_pF_q$.

\begin{definition}
	\label{0f1:math:qFp:def}
	Die hypergeometrische Funktion
	$\mathstrut_pF_q$ ist definiert durch die Reihe
	\[
	\mathstrut_pF_q
	\biggl(
	\begin{matrix}
		a_1,\dots,a_p\\
		b_1,\dots,b_q
	\end{matrix}
	;
	x
	\biggr)
	=
	\mathstrut_pF_q(a_1,\dots,a_p;b_1,\dots,b_q;x)
	=
	\sum_{k=0}^\infty
	\frac{(a_1)_k\cdots(a_p)_k}{(b_1)_k\cdots(b_q)_k}\frac{x^k}{k!}.
	\]
\end{definition}

Angewendet auf die Funktion $\mathstrut_pF_q$ ergibt sich für $\mathstrut_0F_1$:

\begin{equation}
    \label{0f1:math:0f1:eq}
    \mathstrut_0F_1
    \biggl(
    \begin{matrix}
    \text{---}
    \\\
    b_1
    \end{matrix}
    ;
    x
    \biggr)
    =
    \mathstrut_0F_1(;b_1;x)
    =
    \sum_{k=0}^\infty
    \frac{x^k}{(b_1)_k \cdot k!}.
\end{equation}




\subsection{Airy Funktion
\label{0f1:subsection:airy}}
Die Funktion $\operatorname{Ai}(x)$ und die verwandte Funktion $\operatorname{Bi}(x)$ werden als Airy-Funktion bezeichnet. Sie werden zur Lösung verschiedener physikalischer Probleme benutzt, wie zum Beispiel zur Lösung der Schrödinger-Gleichung \cite{0f1:wiki-airyFunktion}.

\begin{definition}
    \label{0f1:airy:differentialgleichung:def}
    Die Differentialgleichung
    $y'' - xy = 0$
    heisst die {\em Airy-Differentialgleichung}.
\end{definition}

Die Airy Funktion lässt sich auf verschiedene Arten darstellen. 
Als hypergeometrische Funktion berechnet, ergibt sich wie in Abschnitt \ref{buch:differentialgleichungen:section:hypergeometrisch} hergeleitet, folgende Lösungen der Airy-Differentialgleichung zu den Anfangsbedingungen $\operatorname{Ai}(0)=1$ und $\operatorname{Ai}'(0)=0$, sowie $\operatorname{Bi}(0)=0$ und $\operatorname{Bi}'(0)=1$.

\begin{align}
\label{0f1:airy:hypergeometrisch:eq}
\operatorname{Ai}(x)
=&
\sum_{k=0}^\infty
\frac{1}{(\frac23)_k} \frac{1}{k!}\biggl(\frac{x^3}{9}\biggr)^k
=
\mathstrut_0F_1\biggl(
\begin{matrix}\text{---}\\\frac23\end{matrix};\frac{x^3}{9}
\biggr).
\\
\operatorname{Bi}(x)
=&
\sum_{k=0}^\infty
\frac{1}{(\frac43)_k} \frac{1}{k!}\biggl(\frac{x^3}{9}\biggr)^k
=
x\cdot\mathstrut_0F_1\biggl(
\begin{matrix}\text{---}\\\frac43\end{matrix};
\frac{x^3}{9}
\biggr).
\qedhere
\end{align}

Um die Stabilität der Algorithmen zu $\mathstrut_0F_1$ zu überprüfen, wird in dieser Arbeit die Airy Funktion $\operatorname{Ai}(x)$ \eqref{0f1:airy:hypergeometrisch:eq}
benutzt.


