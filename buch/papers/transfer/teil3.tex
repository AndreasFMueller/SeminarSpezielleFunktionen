%
% teil2.tex -- Beispiel-File für teil2 
%
% (c) 2020 Prof Dr Andreas Müller, Hochschule Rapperswil
%
\section{MiniMax-Polynom 
\label{transfer:section:teil3}}
\rhead{MiniMax-Polynom}



\subsection{Idee
\label{transfer:subsection:idee}}
Finde das Polynom eines bestimmten Grades, welches eine Funktion in einem Intervall am besten approximiert.


\subsection{Definition
	\label{transfer:subsection:definition}}
Das Polynom welches 
	    $$ \max _{a \leq x \leq b}|f(x)-P(x)| , a \in \mathbb{R}, b \in \mathbb{R}.$$
minimiert.
\subsection{Beispiel
	\label{transfer:subsection:beispiel}}
Um ein MiniMax-Polynom zu berechnen, kann der Remez-Algorithmus verwendet werden. Dieser basiert im wesentlichen auf dem Alternantensatz von Tschebyschow.


