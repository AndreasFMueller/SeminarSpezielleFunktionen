%
% teil3.tex -- Beispiel-File für Teil 3
%
% (c) 2020 Prof Dr Andreas Müller, Hochschule Rapperswil
%
\section{K-Tanh
\label{transfer:section:teil3}}
\rhead{K-Tanh}


\subsection{Algorithmus
\label{transfer:subsection:Ktanh-Algorithmus}}
\cite{transfer:DBLP:journals/corr/abs-1909-07729}
\subsubsection{Vereinfacht
\label{transfer:subsection:Ktanh-Algorithmus:Vereinfacht}}
Negative Werte werden nicht separat behandelt. Diese werden dank der Syymertrie um den Ursprung mit einem einfachen Vorzeichenwechsel aus den positiven berechnet.
Für $x < 0.25$ gilt $y = x$.
Ist $x > 3.75$ gitl $y = 1$.
Ist der Wert zwischen diesen Grenzen, werden über einen Lookuptable geeignete Werte gefunden um aus dem $x$ die Approximation des Tanh zu berechnen.
Dafür werden eine bestimmte Anzahl LSBs des Exponenten und MSBs der Mantisse zu einem Index $t$ zusammengestzt. Der dann die Stelle im Lookuptable zeigt.
Damit werden die richtigen Werte für $E_{t}, r_{t}, b_{t}$ aus der Tabelle, die im Vorhinein schon berechnet wurden, ausgelesen.
Damit hat man das $E$ bereits gefunden und mit der Formel
\[
    M_{o} \leftarrow\left(M_{i} \gg r\right)+b
\]

kann das neue $M$ berechnet werden.
