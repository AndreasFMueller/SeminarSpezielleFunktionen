% vim:ts=2 sw=2 et spell tw=80:
\section{Proofs}

\subsection{Legendre Functions} \label{kugel:sec:proofs:legendre}

\kugeltodo{Fix theorem numbers to match, review text.}

\begin{lemma}
    The polynomial function
    \begin{align*}
        y_n(x)&=\sum_{k=0}^{\lfloor \frac{n}{2} \rfloor} (-1)^k \frac{(2n-2k)!}{2^n k! (n-k)!(n-2k)!} x^{n-2k}\\
        &= \frac{1}{n!2^n}\frac{d^n}{dx^n}(1-x^2)^n =: P_n(x),
    \end{align*}
    is a solution to the second order differential equation
    \begin{equation}\label{kugel:eq:sol_leg}
        (1-x^2)\frac{d^2y}{dx^2}-2x\frac{dy}{dx} + n(n+1)y=0, \quad \forall n>0.
    \end{equation}
\end{lemma}
\begin{proof}
    In order to find a solution to Eq.\eqref{eq:legendre}, the following Ansatz can be performed:
    \begin{equation}\label{eq:ansatz}
    y(x) = \sum_{k=0}^\infty a_k x^k.
    \end{equation}
    Given Eq.\eqref{eq:ansatz}, then
    \begin{align*}
    \frac{dy}{dx} &= \sum_{k=0}^\infty k a_k x^{k-1}, \\
    \frac{d^2y}{dx^2} &= \sum_{k=0}^\infty k (k-1) a_k x^{k-2}.
    \end{align*}
    Eq.\eqref{eq:legendre} can be therefore written as
    \begin{align}
    &(1-x^2)\sum_{k=0}^\infty k (k-1) a_k x^{k-2} - 2x\sum_{k=0}^\infty k a_k x^{k-1} + n(n+1)\sum_{k=0}^\infty a_k x^k=0 \label{eq:ansatz_in_legendre} \\
    &=\sum_{k=0}^\infty k (k-1) a_k x^{k-2} - \sum_{k=0}^\infty k (k-1) a_k x^{k} - 2x\sum_{k=0}^\infty k a_k x^{k-1} + n(n+1)\sum_{k=0}^\infty a_k x^k=0. \nonumber
    \end{align}
    If one consider the term
    \begin{equation}\label{eq:term}
    \sum_{k=0}^\infty k (k-1) a_k x^{k-2},
    \end{equation}
    the substitution $\tilde{k}=k-2$ yields Eq.\eqref{eq:term} to
    \begin{equation*}
    \sum_{\tilde{k}=-2}^\infty (\tilde{k}+2) (\tilde{k}+1) a_{\tilde{k}+2} x^{\tilde{k}}=\sum_{\tilde{k}=0}^\infty (\tilde{k}+2) (\tilde{k}+1) a_{\tilde{k}} x^{\tilde{k}}.
    \end{equation*}
    This means that Eq.\eqref{eq:ansatz_in_legendre} becomes
    \begin{align}
    &\sum_{k=0}^\infty (k+1)(k+2) a_{k+2} x^{k} - \sum_{k=0}^\infty k (k-1) a_k x^{k} - 2\sum_{k=0}^\infty k a_k x^k + n(n+1)\sum_{k=0}^\infty a_k x^k \nonumber \\
    = &\sum_{k=0}^\infty \big[ (k+1)(k+2) a_{k+2} - k (k-1) a_k - 2 k a_k + n(n+1) a_k \big] x^k \stackrel{!}{=} 0. \label{eq:condition}
    \end{align}
    The condition in Eq.\eqref{eq:condition} is equivalent to 
    \begin{equation}\label{eq:condition_2}
    (k+1)(k+2) a_{k+2} - k (k-1) a_k - 2 k a_k + n(n+1) a_k = 0.
    \end{equation}
    We can derive a recursion formula for $a_{k+2}$ from Eq.\eqref{eq:condition_2}, which can be expressed as
    \begin{equation}\label{eq:recursion}
    a_{k+2}= \frac{k (k-1) - 2 k + n(n+1)}{(k+1)(k+2)}a_k = \frac{(k-n)(k+n+1)}{(k+2)(k+1)}a_k.
    \end{equation}
    All coefficients can be calculated using the latter. 
    
    Following Eq.\eqref{eq:recursion}, if we want to compute $a_6$ we would have
    \begin{align*}
    a_{6}= -\frac{(n-4)(n+5)}{6\cdot 5}a_4 &= -\frac{(n-4)(5+n)}{6 \cdot 5} -\frac{(n-2)(n+3)}{4 \cdot 3}  a_2 \\
    &= -\frac{(n-4)(n+5)}{6 \cdot 5} -\frac{(n-2)(n+3)}{4 \cdot 3} -\frac{n(n+1)}{2 \cdot 1}  a_0 \\
    &= -\frac{(n+5)(n+3)(n+1)n(n-2)(n-4)}{6!} a_0.
    \end{align*}
    One can generalize this relation for the $i^\text{th}$ even coefficient as
    \begin{equation*}
    a_{2k} = (-1)^k \frac{(n+(2k-1))(n+(2k-1)-2)\hdots (n-(2k-2)+2)(n-(2k-2))}{(2k)!}a_0
    \end{equation*}
    where $i=2k$.
    
    A similar expression can be written for the odd coefficients $a_{2k-1}$. In this case, the equation starts from $a_1$ and to find the pattern we can write the recursion for an odd coefficient, $a_7$ for example
    \begin{align*}
    a_{7}= -\frac{(n-5)(n+6)}{7\cdot 6}a_5 &= - \frac{(n-5)(n+6)}{7\cdot 6} -\frac{(n-3)(n+4)}{5 \cdot 4}  a_3 \\
    &= - \frac{(n-5)(n+6)}{7\cdot 6} -\frac{(n-3)(n+4)}{5 \cdot 4}  -\frac{(n-1)(n+2)}{3 \cdot 2}  a_1 \\
    &= -\frac{(n+6)(n+4)(n+2)(n-1)(n-3)(n-5)}{7!} a_1.
    \end{align*}
    As before, we can generalize this equation for the $i^\text{th}$ odd coefficient
    \begin{equation*}
    a_{2k+1} = (-1)^k \frac{(n + 2k)(n+2k-2)\hdots(n-(2k-1)+2)(n-(2k-1))}{(2k+1)!}a_1
    \end{equation*}
    where $i=2k+1$.
    
    Let be 
    \begin{align*}
    y_\text{e}^K(x) &:= \sum_{k=0}^K(-1)^k \frac{(n+(2k-1))(n+(2k-1)-2)\hdots \color{red}(n-(2k-2)+2)(n-(2k-2))}{(2k)!} x^{2k}, \\
    y_\text{o}^K(x) &:= \sum_{k=0}^K(-1)^k \frac{(n + 2k)(n+2k-2)\hdots \color{blue} (n-(2k-1)+2)(n-(2k-1))}{(2k+1)!} x^{2k+1}.
    \end{align*}
    The solution to the Eq.\eqref{eq:legendre} can be written as
    \begin{equation}\label{eq:solution}
    y(x) = \lim_{K \to \infty} \left[ a_0 y_\text{e}^K(x) + a_1 y_\text{o}^K(x) \right].
    \end{equation}
    
    The colored parts can be analyzed separately:
    \begin{itemize}
      \item[\textcolor{red}{\textbullet}] Suppose that $n=n_0$ is an even number. Then the red part, for a specific value of $k=k_0$, will follow the following relation:
    \begin{equation*}
    n_0-(2k_0-2)=0. 
    \end{equation*}
    From that point on, given the recursive nature of Eq.\eqref{eq:recursion}, all the subsequent coefficients will also be 0, making the sum finite.
    \begin{equation*}
    a_{2k}=0 \iff y_{\text{o}}^{2k}(x)=y_{\text{o}}^{2k_0}(x), \quad \forall k>k_0
    \end{equation*} 
      \item[\textcolor{blue}{\textbullet}] Suppose that $n=n_0$ is an odd number. Then the blue part, for a specific value of $k=k_0$, will follow the following relation 
    \begin{equation*}
    n_0-(2k_0-1)=0.  
    \end{equation*}
    From that point on, for the same reason as before, all the subsequent coefficients will also be 0, making the sum finite.
    \begin{equation*}
    a_{2k+1}=0 \iff y_{\text{o}}^{2k+1}(x)=y_{\text{o}}^{2k_0+1}(x), \quad \forall k>k_0
    \end{equation*}
    \end{itemize} 
    
    There is the possibility of expressing the solution in Eq.\eqref{eq:solution} in a more compact form, combining the two solutions $y_\text{o}^K(x)$ and $y_\text{e}^K(x)$. They are both a polynomial of maximum degree $n$, assuming $n \in \mathbb{N}$. In the case where $n$ is even, the polynomial solution
    \begin{equation*}
    \lim_{K\to \infty} y_\text{e}^K(x)
    \end{equation*}
    will be a finite sum. If instead $n$ is odd, will be 
    \begin{equation*}
    \lim_{K\to \infty} y_\text{o}^K(x)
    \end{equation*}
    to be a finite sum. 
    
    Depending on the coefficient we start with, $a_1$ or $a_0$, we will obtain the odd or even polynomial respectively. Starting with the last coefficient $a_n$ and, recursively, calculating all the others in descending order, we can express the two parts $y_\text{o}^K(x)$ and $y_\text{e}^K(x)$ with a single sum. Hence, because we start with the last coefficient, the choice concerning $a_1$ and $a_0$ will be at the end of the sum, and not at the beginning. To compact Eq.\eqref{eq:solution}, Eq.\eqref{eq:recursion} can be reconsidered to calculate the coefficient $a_{k-2}$, using $a_k$
    \begin{equation*}
    a_{k-2} = -\frac{(k+2)(k+1)}{(k-n)(k+n+1)}a_k
    \end{equation*}
    Now the game is to find a pattern, as before. Remember that $n$ is a fixed parameter of Eq.\eqref{eq:legendre}.  
    \begin{align*}
    a_{n-2} &= -\frac{n(n-1)}{2(2n-1)}a_n, \\
    a_{n-4} &= -\frac{(n-2)(n-3)}{4(2n-3)}a_{n-2} \\
    &= -\frac{(n-2)(n-3)}{4(2n-3)}-\frac{n(n-1)}{2(2n-1)}a_n.
    \end{align*}
    In general 
    \begin{equation}\label{eq:general_recursion}
    a_{n-2k} = (-1)^k \frac{n(n-1)(n-2)(n-3) \hdots (n-2k+1)}{2\cdot4\hdots 2k(2n-1)(2n-3)\hdots(2n-2k+1)}a_n
    \end{equation}
    The whole solution can now be written as
    \begin{align}
    y(x) &= a_n x^n + a_{n-2} x^{n-2} + a_{n-4} x^{n-4} + a_{n-6} x^{n-6} + \hdots + \begin{cases} 
    a_1 x, \quad &\text{if } n \text{ odd} \\ 
    a_0, \quad  &\text{if } n \text{ even} 
    \end{cases} \nonumber \\
    &= \sum_{k=0}^{\lfloor \frac{n}{2} \rfloor} a_{n-2k}x^{n-2k} \label{eq:solution_2}
    \end{align}
    By considering
    \begin{align}
    (2n-1)(2n-3)\hdots (2n-2k+1)&=\frac{2n(2n-1)(2n-2)(2n-3)\hdots(2n-2k+1)}
    {2n(2n-2)(2n-4)(2n-6)\hdots(2n-2k+2)} \nonumber \\ 
    &=\frac{\frac{(2n)!}{(2n-2k)!}}
    {2^kn(n-1)(n-2)(n-3)\hdots(n-k+1)} \nonumber \\
    &=\frac{\frac{(2n)!}{(2n-2k)!}}
    {2^k\frac{n!}{(n-k)!}}=\frac{(n-k)!(2n)!}{n!(2n-2k)!2^k} \label{eq:1_sub_recursion}, \\
    2 \cdot 4  \hdots 2k &= 2^r 1\cdot2 \hdots r = 2^r r!\label{eq:2_sub_recursion}, \\
    n(n-1)(n-2)(n-3) \hdots (n-2k+1) &= \frac{n!}{(n-2k)!}\label{eq:3_sub_recursion}.
    \end{align}
    Eq.\eqref{eq:solution_2} can be rewritten as
    \begin{equation}\label{eq:solution_3}
    y(x)=a_n \sum_{k=0}^{\lfloor \frac{n}{2} \rfloor} (-1)^k \frac{n!^2(2n-2k)!}{k!(n-2k)!(n-k)!(2n)!}  x^{n-2k}.
    \end{equation}
    Eq.\eqref{eq:solution_3} is defined for any $a_n$. By letting $a_n$ be declared as
    \begin{equation*}
    a_{n} := \frac{(2n)!}{2^n n!^2},
    \end{equation*}
    the so called \emph{Legendre polynomial} emerges
    \begin{equation}\label{eq:leg_poly}
    P_n(x):=\sum_{k=0}^{\lfloor \frac{n}{2} \rfloor} (-1)^k \frac{(2n-2k)!}{2^n k! (n-k)!(n-2k)!} x^{n-2k} 
    \end{equation}
\end{proof}


\begin{lemma}\label{kugel:lemma:sol_associated_leg_eq}
  If $Z_n(z)$ is a solution of the Legendre equation \eqref{kugel:eqn:legendre},
  then
  \begin{equation*}
    P^m_n(z) = (1 - z^2)^{m/2} \frac{d^m}{dz^m}Z_n(z)
  \end{equation*}
  solves the associated Legendre equation \eqref{kugel:eqn:associated-legendre}.
\end{lemma}
% \begin{proof} [TODO: modificare la $m$ (è già usata come costante di separazione) o forse è giusta (?)]
\begin{proof}
    To begin, we can start by differentiating $m$ times Eq.\eqref{kugel:eq:leg_eq} (which is staisfied by $y(x)$), obtaining
    \begin{equation}\label{eq:lagrange_mderiv}
    \frac{d^m}{dx^m}\left[ (1-x^2)\frac{d^2y}{dx^2} \right] -2 \frac{d^m}{dx^m}\left[ x\frac{dy}{dx} \right] + n(n+1)\frac{d^m}{dx^m}y=0.
    \end{equation}
    \emph{Leibniz's theorem} says, that if we want to differentiate $m$ times a multiplication of two functions, we can use the binomial coefficients to build up a sum. This allows us to be more compact, obtaining 
    \begin{equation}\label{eq:leibniz}
    \frac{d^m}{dx^m}[u(x)v(x)] = \sum_{i=0}^m \binom{n}{i} \frac{d^{m-i}u}{dx^{m-1}} \frac{d^{i}v}{dx^i}.
    \end{equation}
    Using Eq.\eqref{eq:leibniz} in Eq.\eqref{eq:lagrange_mderiv}, we have
    \begin{align}
    (1-x^2)\frac{d^{m+2}y}{dx^{m+2}} &+ m \frac{d}{dx}(1-x^2)\frac{d^{m+1}y}{dx^{m+1}} + \frac{m(m-1)}{2}\frac{d^{2}}{dx^{2}}(1-x^2)\frac{d^{m}y}{dx^{m}} + n(n+1)\frac{d^m{}y}{dx^{m}} \nonumber \\
    &-2\left(x\frac{d^{m+1}y}{dx^{m+1}} + m\frac{d}{dx}x\frac{d^{m}y}{dx^{m}} \right) \nonumber \\
    &= (1-x^2)\frac{d^{m+2}y}{dx^{m+2}} -2x(m+1)\frac{d^{m+1}y}{dx^{m+1}}+(n(n+1)-m(m-1)-2m)\frac{d^{m}y}{dx^{m}}=0. \label{eq:aux_3}
    \end{align}
    To make the notation easier to follow, a new function can be defined
    \begin{equation*}
    \frac{d^{m}y}{dx^{m}} := y_m.
    \end{equation*}
    Eq.\eqref{eq:aux_3} now becomes
    \begin{equation}\label{eq:1st_subs}
    (1-x^2)\frac{d^{2}y_m}{dx^{2}} -2x(m+1)\frac{dy_m}{dx}+(n(n+1)-m(m+1))y_m=0
    \end{equation}
    A second function can be further defined as
    \begin{equation*}
    (1-x^2)^{\frac{m}{2}}\frac{d^{m}y}{dx^{m}} = (1-x^2)^{\frac{m}{2}}y_m := \hat{y}_m,
    \end{equation*}
    allowing to write Eq.\eqref{eq:1st_subs} as
    \begin{equation}\label{eq:2st_subs}
    (1-x^2)\frac{d^2}{dx^2}[\hat{y}_m(1-x^2)^{-\frac{m}{2}}] -2(m+1)x\frac{d}{dx}[\hat{y}_m(1-x^2)^{-\frac{m}{2}}] + (n(n+1)-m(m+1))\hat{y}_m(1-x^2)^{-\frac{m}{2}}=0.
    \end{equation}
    The goal now is to compute the two terms 
    \begin{align*}
    \frac{d^2}{dx^2}[\hat{y}_m(1-x^2)^{-\frac{m}{2}}] &=  \frac{d^2\hat{y}_m}{dx^2} (1-x^2)^{-\frac{m}{2}} + \frac{d\hat{y}_m}{dx}\frac{m}{2}(1-x^2)^{-\frac{m}{2}-1}2x \\
    &+ m\left( \frac{d\hat{y}_m}{dx} x (1-x^2)^{-\frac{m}{2}-1} + \hat{y}_m (1-x^2)^{-\frac{m}{2}-1} - \hat{y}_m x (-\frac{m}{2}-1)(1-x^2)^{-\frac{m}{2}} 2x\right) \\
    &= \frac{d^2\hat{y}_m}{dx^2} (1-x^2)^{-\frac{m}{2}} + \frac{d\hat{y}_m}{dx}mx (1-x^2)^{-\frac{m}{2}-1} + m\frac{d\hat{y}_m}{dx}x (1-x^2)^{-\frac{m}{2}-1}\\
    &+ m\hat{y}_m  (1-x^2)^{-\frac{m}{2}-1} + m\hat{y}_m x^2(m+2)(1-x^2)^{-\frac{m}{2}-2}
    \end{align*}
    and
    \begin{align*}
    \frac{d}{dx}[\hat{y}_m(1-x^2)^{-\frac{m}{2}}] &= \frac{d\hat{y}_m}{dx}(1-x^2)^{-\frac{m}{2}} + \hat{y}_m\frac{m}{2}(1-x^2)^{-\frac{m}{2}-1}2x \\
    &= \frac{d\hat{y}_m}{dx}(1-x^2)^{-\frac{m}{2}} + \hat{y}_mm(1-x^2)^{-\frac{m}{2}-1}x,
    \end{align*}
    to use them in Eq.\eqref{eq:2st_subs}, obtaining
    \begin{align*}
    (1-x^2)\biggl[\frac{d^2\hat{y}_m}{dx^2} (1-x^2)^{-\frac{m}{2}} &+ \frac{d\hat{y}_m}{dx}mx (1-x^2)^{-\frac{m}{2}-1} + m\frac{d\hat{y}_m}{dx}x (1-x^2)^{-\frac{m}{2}-1} \\ 
    &+ m\hat{y}_m  (1-x^2)^{-\frac{m}{2}-1} + m\hat{y}_m x^2(m+2)(1-x^2)^{-\frac{m}{2}-2}\biggr] \\
    &-2(m+1)x\left[  \frac{d\hat{y}_m}{dx}(1-x^2)^{-\frac{m}{2}} + \hat{y}_mm(1-x^2)^{-\frac{m}{2}-1}x \right] \\
    &+ (n(n+1)-m(m+1))\hat{y}_m(1-x^2)^{-\frac{m}{2}}=0.\\
    \end{align*}
    We can now divide by $(1-x^2)^{-\frac{m}{2}}$, obtaining
    \begin{align*}
    (1-x^2)\biggl[\frac{d^2\hat{y}_m}{dx^2} &+ \frac{d\hat{y}_m}{dx}mx (1-x^2)^{-1} + m\frac{d\hat{y}_m}{dx}x (1-x^2)^{-1} + m\hat{y}_m  (1-x^2)^{-1} + m\hat{y}_m x^2(m+2)(1-x^2)^{-2}\biggr] \\
    &-2(m+1)x\left[  \frac{d\hat{y}_m}{dx} + \hat{y}_mm(1-x^2)^{-1}x \right] + (n(n+1)-m(m+1))\hat{y}_m\\
    &= \frac{d^2\hat{y}_m}{dx^2} + \frac{d\hat{y}_m}{dx}mx + m\frac{d\hat{y}_m}{dx}x + m\hat{y}_m + m\hat{y}_m x^2(m+2)(1-x^2)^{-1} \\
    &-2(m+1)x\left[  \frac{d\hat{y}_m}{dx} + \hat{y}_mm(1-x^2)^{-1}x \right] + (n(n+1)-m(m+1))\hat{y}_m\\
    \end{align*}
    and collecting some terms
    \begin{equation*}
    (1-x^2)\frac{d^2\hat{y}_m}{dx^2} - 2x\frac{d\hat{y}_m}{dx} + \left( -x^2 \frac{m^2}{1-x^2} + m+n(n+1)-m(m+1)\right)\hat{y}_m=0.
    \end{equation*}
    Showing that 
    \begin{align*}
    -x^2 \frac{m^2}{1-x^2} + m+n(n+1)-m(m+1) &= n(n+1)- m^2 -x^2 \frac{m^2}{1-x^2} \\
    &= n(n+1)- \frac{m}{1-x^2}
    \end{align*}
    implies $\hat{y}_m(x)$ being a solution of Eq.\eqref{kugel:eq:associated_leg_eq}
\end{proof}
