% vim:ts=2 sw=2 et spell tw=80:

\section{Construction of the Spherical Harmonics}

\if 0
\kugeltodo{Rewrite this section if the preliminaries become an addendum}
We finally arrived at the main section, which gives our chapter its name. The
idea is to discuss spherical harmonics, their mathematical derivation and some
of their properties and applications.

The subsection \ref{} \kugeltodo{Fix references} will be devoted to the
Eigenvalue problem of the Laplace operator. Through the latter we will derive
the set of Eigenfunctions that obey the equation presented in \ref{}
\kugeltodo{reference to eigenvalue equation}, which will be defined as
\emph{Spherical Harmonics}. In fact, this subsection will present their
mathematical derivation.

In the subsection \ref{}, on the other hand, some interesting properties
related to them will be discussed. Some of these will come back to help us
understand in more detail why they are useful in various real-world
applications, which will be presented in the section \ref{}.

One specific property will be studied in more detail in the subsection \ref{},
namely the recursive property.  The last subsection is devoted to one of the
most beautiful applications (In our humble opinion), namely the derivation of a
Fourier-style series expansion but defined on the sphere instead of a plane.
More importantly, this subsection will allow us to connect all the dots we have
created with the previous sections, concluding that Fourier is just a specific
case of the application of the concept of orthogonality. Our hope is that after
reading this section you will appreciate the beauty and power of generalization
that mathematics offers us.
\fi

\subsection{Eigenvalue Problem}

\begin{figure}
  \centering
  \includegraphics{papers/kugel/figures/tikz/spherical-coordinates}
  \caption{
    Spherical coordinate system. Space is described with the free variables $r
    \in \mathbb{R}_0^+$, $\vartheta \in [0; \pi]$ and $\varphi \in [0; 2\pi)$.
    \label{kugel:fig:spherical-coordinates}
  }
\end{figure}

From Section \ref{buch:pde:section:kugel}, we know that the spherical Laplacian
in the spherical coordinate system (shown in Figure
\ref{kugel:fig:spherical-coordinates}) is is defined as
\begin{equation*}
    \sphlaplacian :=
      \frac{1}{r^2} \frac{\partial}{\partial r} \left(
        r^2 \frac{\partial}{\partial r}
      \right)
      + \frac{1}{r^2} \left[
          \frac{1}{\sin\vartheta} \frac{\partial}{\partial \vartheta} \left(
            \sin\vartheta \frac{\partial}{\partial\vartheta}
          \right)
        + \frac{1}{\sin^2 \vartheta} \frac{\partial^2}{\partial\varphi^2}
      \right].
\end{equation*}
But we will not consider this algebraic monstrosity in its entirety. As the
title suggests, we will only care about the \emph{surface} of the sphere.  This
is for many reasons, but mainly to simplify reduce the already broad scope of
this text. Concretely, we will always work on the unit sphere, which just means
that we set $r = 1$ and keep only $\vartheta$ and $\varphi$ as free variables.
Now, since the variable $r$ became a constant, we can leave out all derivatives
with respect to $r$ and substitute all $r$'s with 1's to obtain a new operator
that deserves its own name.

\begin{definition}[Surface spherical Laplacian]
  \label{kugel:def:surface-laplacian}
  The operator
  \begin{equation*}
      \surflaplacian :=
        \frac{1}{\sin\vartheta} \frac{\partial}{\partial \vartheta} \left(
          \sin\vartheta \frac{\partial}{\partial\vartheta}
        \right)
        + \frac{1}{\sin^2 \vartheta} \frac{\partial^2}{\partial\varphi^2},
  \end{equation*}
  is called the surface spherical Laplacian.
\end{definition}

In the definition, the subscript ``$\partial S$'' was used to emphasize the
fact that we are on the spherical surface, which can be understood as being the
boundary of the sphere. But what does it actually do? To get an intuition,
first of all, notice the fact that $\surflaplacian$ have second derivatives,
which means that this a measure of \emph{curvature}; But curvature of what? To
get an even stronger intuition we will go into geometry, were curvature can be
grasped very well visually. Consider figure \ref{kugel:fig:curvature} where the
curvature is shown using colors. First we have the curvature of a curve in 1D,
then the curvature of a surface (2D), and finally the curvature of a function on
the surface of the unit sphere.

\begin{figure}
  \centering
  \includegraphics[width=.3\linewidth]{papers/kugel/figures/tikz/curvature-1d}
  \hskip 5mm
  \includegraphics[width=.3\linewidth]{papers/kugel/figures/povray/curvature}
  \hskip 5mm
  \includegraphics[width=.3\linewidth]{papers/kugel/figures/povray/spherecurve}
  \caption{
    \kugeltodo{Fix alignment / size, add caption. Would be nice to match colors.}
    \label{kugel:fig:curvature}
  }
\end{figure}

Now that we have defined an operator, we can go and study its eigenfunctions,
which means that we would like to find the functions $f(\vartheta, \varphi)$
that satisfy the equation
\begin{equation} \label{kuvel:eqn:eigen}
    \surflaplacian f = -\lambda f.
\end{equation}
Perhaps it may not be obvious at first glance, but we are in fact dealing with a
partial differential equation (PDE) \kugeltodo{Boundary conditions?}. If we unpack the notation of the operator
$\nabla^2_{\partial S}$ according to definition
\ref{kugel:def:surface-laplacian}, we get:
\begin{equation} \label{kugel:eqn:eigen-pde}
    \frac{1}{\sin\vartheta} \frac{\partial}{\partial \vartheta} \left(
      \sin\vartheta \frac{\partial f}{\partial\vartheta}
    \right)
    + \frac{1}{\sin^2 \vartheta} \frac{\partial^2 f}{\partial\varphi^2}
    + \lambda f = 0.
\end{equation}
Since all functions satisfying \eqref{kugel:eqn:eigen-pde} are the
\emph{eigenfunctions} of $\surflaplacian$, our new goal is to solve this PDE.
The task may seem very difficult but we can simplify it with a well-known
technique: \emph{the separation Ansatz}. It consists in assuming that the
function $f(\vartheta, \varphi)$ can be factorized in the following form:
\begin{equation}
    f(\vartheta, \varphi) = \Theta(\vartheta)\Phi(\varphi). 
\end{equation}
In other words, we are saying that the effect of the two independent variables
can be described using the multiplication of two functions that describe their
effect separately. This separation process was already presented in section
\ref{buch:pde:section:kugel}, but we will briefly rehearse it here for
convenience. If we substitute this assumption in
\eqref{kugel:eqn:eigen-pde}, we have:
\begin{equation*}
    \frac{1}{\sin\vartheta} \frac{\partial}{\partial \vartheta} \left(
      \sin\vartheta \frac{\partial  \Theta(\vartheta)}{\partial\vartheta}
    \right) \Phi(\varphi)
    + \frac{1}{\sin^2 \vartheta} \frac{\partial^2 \Phi(\varphi)}{\partial\varphi^2}
      \Theta(\vartheta)
    + \lambda \Theta(\vartheta)\Phi(\varphi) = 0.
\end{equation*}
Dividing by $\Theta(\vartheta)\Phi(\varphi)$ and introducing an auxiliary
variable $m^2$, the separation constant, yields:
\begin{equation*}
  \frac{1}{\Theta(\vartheta)}\sin \vartheta \frac{d}{d \vartheta} \left(
    \sin \vartheta \frac{d \Theta}{d \vartheta}
  \right)
  + \lambda \sin^2 \vartheta
  = -\frac{1}{\Phi(\varphi)} \frac{d^2\Phi(\varphi)}{d\varphi^2}
  = m^2,
\end{equation*}
which is equivalent to the following system of 2 first order differential
equations (ODEs):
\begin{subequations}
  \begin{gather}
    \frac{d^2\Phi(\varphi)}{d\varphi^2} = -m^2 \Phi(\varphi),
      \label{kugel:eqn:ode-phi} \\ 
    \sin \vartheta \frac{d}{d \vartheta} \left(
      \sin \vartheta \frac{d \Theta}{d \vartheta}
    \right)
    + \left( \lambda - \frac{m^2}{\sin^2 \vartheta} \right)
      \Theta(\vartheta) = 0
      \label{kugel:eqn:ode-theta}.
  \end{gather}
\end{subequations}
The solution of \eqref{kugel:eqn:ode-phi} is easy to find: The complex
exponential is obviously the function we are looking for. So we can directly
write the solutions
\begin{equation} \label{kugel:eqn:ode-phi-sol}
    \Phi(\varphi) = e^{i m \varphi}, \quad m \in \mathbb{Z}.
\end{equation}
The restriction that the separation constant $m$ needs to be an integer arises
from the fact that we require a $2\pi$-periodicity in $\varphi$ since the
coordinate systems requires that $\Phi(\varphi + 2\pi) = \Phi(\varphi)$.
Unfortunately, solving \eqref{kugel:eqn:ode-theta} is as straightforward,
actually, it is quite difficult, and the process is so involved that it will
require a dedicated section of its own.

\subsection{Legendre Functions}

To solve \eqref{kugel:eqn:ode-theta} we start with the substitution $z = \cos
\vartheta$ \kugeltodo{Explain geometric origin with picture}. The operator
$\frac{d}{d \vartheta}$ becomes
\begin{equation*}
    \frac{d}{d \vartheta}
    = \frac{dz}{d \vartheta}\frac{d}{dz}
    = -\sin \vartheta \frac{d}{dz}
    = -\sqrt{1-z^2} \frac{d}{dz},
\end{equation*} 
since $\sin \vartheta = \sqrt{1 - \cos^2 \vartheta} = \sqrt{1 - z^2}$, and
then \eqref{kugel:eqn:ode-theta} becomes 
\begin{align*}
  \frac{-\sqrt{1-z^2}}{\sqrt{1-z^2}} \frac{d}{dz} \left[
    \left(\sqrt{1-z^2}\right) \left(-\sqrt{1-z^2}\right) \frac{d \Theta}{dz}
  \right]
  + \left( \lambda - \frac{m^2}{1 - z^2} \right)\Theta(\vartheta) &= 0,
  \\
  \frac{d}{dz} \left[ (1-z^2) \frac{d \Theta}{dz} \right]
  + \left( \lambda - \frac{m^2}{1 - z^2} \right)\Theta(\vartheta) &= 0,
  \\
  (1-z^2)\frac{d^2 \Theta}{dz} - 2z\frac{d \Theta}{dz}
  + \left( \lambda - \frac{m^2}{1 - z^2} \right)\Theta(\vartheta) &= 0.
\end{align*}
By making two final cosmetic substitutions, namely $Z(z) = \Theta(\cos^{-1}z)$
and $\lambda = n(n+1)$, we obtain what is known in the literature as the
\emph{associated Legendre equation of order $m$}:
\nocite{olver_introduction_2013}
\begin{equation} \label{kugel:eqn:associated-legendre}
  (1 - z^2)\frac{d^2 Z}{dz}
  - 2z\frac{d Z}{dz}
  + \left( n(n + 1) - \frac{m^2}{1 - z^2} \right) Z(z) = 0,
  \quad
  z \in [-1; 1], m \in \mathbb{Z}.
\end{equation}

Our new goal has therefore become to solve
\eqref{kugel:eqn:associated-legendre}, since if we find a solution for $Z(z)$ we
can perform the substitution backwards and get back to our eigenvalue problem.
However, the associated Legendre equation is not any easier, so to attack the
problem we will look for the solutions in the easier special case when $m = 0$.
This reduces the problem because it removes the double pole, which is always
tricky to deal with. In fact, the reduced problem when $m = 0$ is known as the
\emph{Legendre equation}:
\begin{equation} \label{kugel:eqn:legendre}
  (1 - z^2)\frac{d^2 Z}{dz}
  - 2z\frac{d Z}{dz}
  + n(n + 1) Z(z) = 0,
  \quad
  z \in [-1; 1].
\end{equation}

The Legendre equation is a second order differential equation, and therefore it
has 2 independent solutions, which are known as \emph{Legendre functions} of the
first and second kind. For the scope of this text we will only derive a special
case of the former that is known known as the \emph{Legendre polynomials}, since
we only need a solution between $-1$ and $1$.

\begin{lemma}[Legendre polynomials]
  \label{kugel:lem:legendre-poly}
  The polynomial function
  \[
    P_n(z) = \sum^{\lfloor n/2 \rfloor}_{k=0}
      \frac{(-1)^k}{2^n s^k!} \frac{(2n - 2k)!}{(n - k)! (n-2k)!} z^{n - 2k}
  \]
  is the only finite solution of the Legendre equation
  \eqref{kugel:eqn:legendre} when $n \in \mathbb{Z}$ and $z \in [-1; 1]$.
\end{lemma}
\begin{proof}
  This results is derived in section \ref{kugel:sec:proofs:legendre}.
\end{proof}

Since the Legendre \emph{polynomials} are indeed polynomials, they can also be
expressed using the hypergeometric functions described in section
\ref{buch:rekursion:section:hypergeometrische-funktion}, so in fact
\begin{equation}
  P_n(z) = {}_2F_1 \left( \begin{matrix}
    n + 1, & -n \\ \multicolumn{2}{c}{1}
  \end{matrix} ; \frac{1 - z}{2} \right).
\end{equation}
Further, there are a few more interesting but not very relevant forms to write
$P_n(z)$ such as \emph{Rodrigues' formula} and \emph{Laplace's integral
representation} which are
\begin{equation*}
  P_n(z) = \frac{1}{2^n} \frac{d^n}{dz^n} (x^2 - 1)^n,
  \qquad \text{and} \qquad
  P_n(z) = \frac{1}{\pi} \int_0^\pi \left(
    z + \cos\vartheta \sqrt{z^2 - 1}
  \right) \, d\vartheta
\end{equation*}
respectively, both of which we will not prove (see chapter 3 of
\cite{bell_special_2004} for a proof). Now that we have a solution for the
Legendre equation, we can make use of the following lemma patch the solutions
such that they also become solutions of the associated Legendre equation
\eqref{kugel:eqn:associated-legendre}.

\begin{lemma} \label{kugel:lem:extend-legendre}
  If $Z_n(z)$ is a solution of the Legendre equation \eqref{kugel:eqn:legendre},
  then
  \begin{equation*}
    Z^m_n(z) = (1 - z^2)^{m/2} \frac{d^m}{dz^m}Z_n(z)
  \end{equation*}
  solves the associated Legendre equation \eqref{kugel:eqn:associated-legendre}.
  \nocite{bell_special_2004}
\end{lemma}
\begin{proof}
  See section \ref{kugel:sec:proofs:legendre}.
\end{proof}

What is happening in lemma \ref{kugel:lem:extend-legendre}, is that we are
essentially inserting a square root function in the solution in order to be able
to reach the parts of the domain near the poles at $\pm 1$ of the associated
Legendre equation, which is not possible only using power series
\kugeltodo{Reference book theory on extended power series method.}. Now, since
we have a solution in our domain, namely $P_n(z)$, we can insert it in the lemma 
obtain the \emph{associated Legendre functions}.

\begin{definition}[Ferrers or Associated Legendre functions]
  The functions
  \begin{equation}\label{kugel:eq:associated_leg_func}
    P^m_n (z) = \frac{1}{n!2^n}(1-z^2)^{\frac{m}{2}}\frac{d^{m}}{dz^{m}} P_n(z)
      = \frac{1}{n!2^n}(1-z^2)^{\frac{m}{2}}\frac{d^{m+n}}{dz^{m+n}}(1-z^2)^n 
  \end{equation}
  are known as Ferrers or associated Legendre functions.
\end{definition}

\subsection{Spherical Harmonics}

As you may recall, previously we performed the substitution $x=\cos \vartheta$. Now we need to return to the old domain, which can be done straightforwardly:
\begin{equation*}
    \Theta(\vartheta) = P_{m,n}(\cos \vartheta),
\end{equation*}
obtaining the much sought function $\Theta(\vartheta)$. \newline
So we finally reached the end of this tortuous path. Now we just need to put together all the information we have to construct $f(\vartheta, \varphi)$ in the following way:
\begin{equation}\label{kugel:eq:sph_harm_0}
    f(\vartheta, \varphi) = \Theta(\vartheta)\Phi(\varphi) = P_{m,n}(\cos \vartheta)e^{jm\varphi}, \quad |m|\leq n.
\end{equation}
The constraint $|m|<n$, can be justified by considering Eq.\eqref{kugel:eq:associated_leg_func}, in which the derivative of degree $m+n$ is present. A derivative to be well defined must have an order that is greater than zero. Furthermore, it can be seen that this derivative is applied on a polynomial of degree $2n$. As is known from Calculus 1, if you derive a polynomial of degree $2n$ more than $2n$ times, you get zero, which is a trivial solution in which we are not interested.\newline
We can thus summarize these two conditions by writing:
\begin{equation*}
    \begin{rcases}
        m+n \leq 2n &\implies m \leq n \\
        m+n \geq 0  &\implies  m \geq -n
    \end{rcases} |m| \leq n.
\end{equation*}
The set of functions in Eq.\eqref{kugel:eq:sph_harm_0} is named \emph{Spherical Harmonics}, which are the eigenfunctions of the Laplace operator on the \emph{spherical surface domain}, which is exactly what we were looking for at the beginning of this section.
\begin{definition}{Spherical Harmonics}
    \begin{equation}\label{kugel:eq:sph_harm_1}
    \tilde{Y}_{m,n}(\vartheta, \varphi) := P_{m,n}(\cos \vartheta)e^{jm\varphi}, \quad |m|\leq n.
    \end{equation}
\end{definition}

\subsection{Normalization}
As explained in the chapter \ref{}, the concept of orthogonality is very important and at the practical level it is very useful, because it allows us to develop very powerful techniques at the mathematical level.\newline 
Throughout this book we have been confronted with the Sturm-Liouville theory (see chapter \ref{}). The latter, among other things, carries with it the concept of orthogonality. Indeed, if we consider the solutions of the Sturm-Liouville equation, which can be expressed in this form
\begin{equation}\label{kugel:eq:sturm_liouville}
    \mathcal{S}f := \frac{d}{dx}\left[p(x)\frac{df}{dx}\right]+q(x)f(x)
\end{equation}
possiamo dire che formano una base ortogonale.\newline
Adesso possiamo dare un occhiata alle due equazioni che abbiamo ottenuto tramite la Separation Ansatz (Eqs.\eqref{kugel:eq:associated_leg_eq}\eqref{kugel:eq:ODE_1}), le quali possono essere riscritte come:
\begin{align*}
    \frac{d}{dx} \left[ (1-x^2) \cdot \frac{dP_{m,n}}{dx} \right] &+ \left(n(n+1)-\frac{m}{1-x^2} \right) \cdot P_{m,n}(x) = 0, \\
    \frac{d}{d\varphi} \left[ 1 \cdot \frac{ d\Phi }{d\varphi} \right] &+ 1 \cdot \Phi(\varphi) = 0. 
\end{align*}
Si può concludere in modo diretto che sono due casi dell'equazione di Sturm-Liouville. Questo significa che le loro soluzioni sono ortogonali sotto l'inner product con weight function $w(x)=1$, dunque:
\begin{align}
\int_{0}^{2\pi} \Phi_m(\varphi)\Phi_m'(\varphi) d\varphi &= \delta_{m'm}, \nonumber \\
\int_{-1}^1 P_{m,m'}(x)P_{n,n'}(x) dx &= \delta_{m'm}\delta_{n'n}. \label{kugel:eq:orthogonality_associated_func}
\end{align}
Inoltre, possiamo provare l'ortogonalità di $\Theta(\vartheta)$ utilizzando \eqref{kugel:eq:orthogonality_associated_func}:
\begin{align}
    x
\end{align}
Ora, visto che la soluzione dell'eigenfunction problem è formata dalla moltiplicazione di $\Phi_m(\varphi)$ e $P_{m,n}(x)$
\begin{lemma}

\end{lemma}
\subsection{Properties}

\subsection{Recurrence Relations}

\section{Series Expansions in $C(S^2)$}

\subsection{Orthogonality of $P_n$, $P^m_n$ and $Y^m_n$}

\subsection{Series Expansion}

\subsection{Fourier on $S^2$}
