% vim:ts=2 sw=2 et spell tw=78:

\section{Introduction}

This chapter of the book is devoted to the sef of functions called
\emph{spherical harmonics}. However, before we dive into the topic, we want to
make a few preliminary remarks to avoid ``upsetting'' a certain type of
reader. Specifically, we would like to specify that the authors of this
chapter not mathematicians but engineers, and therefore the text will not be
always complete with sound proofs after every claim. Instead we will go
through the topic in a more intuitive way including rigorous proofs only if
they are enlightening or when they are very short. Where no proofs are given
we will try to give an intuition for why it is true.

That being said, when talking about spherical harmonics one could start by
describing their name. The latter may be a cause of some confusion because of
the misleading translations in other languages. In German the name for this
set of functions is ``Kugelfunktionen'', which puts the emphasis only on the
spherical context, whereas the English name ``spherical harmonics'' also
contains the \emph{harmonic} part hinting at Fourier theories and harmonic
analysis in general.

The structure of this chapter is organized in the following way. First, we
will quickly go through some fundamental linear algebra and Fourier theory to
refresh a few important concepts. In principle, we could have written the
whole thing starting from a much more abstract level without much preparation,
but then we would have lost some of the beauty that comes from the
appreciation of the power of some surprisingly simple ideas. Then once the
basics are done, we can explore the main topic of spherical harmonics which as
we will see arises from the eigenfunctions of the Laplacian operator in
spherical coordinates. Finally, after studying what we think are the most
beautiful and interesting properties of the spherical harmonics, to conclude
this journey we will present a few real-world applications, which are of
course most of interest for engineers.

