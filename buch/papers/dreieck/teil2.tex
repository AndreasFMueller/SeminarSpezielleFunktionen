%
% teil2.tex -- Beispiel-File für teil2 
%
% (c) 2020 Prof Dr Andreas Müller, Hochschule Rapperswil
%
\section{Beliebige Polynome
\label{dreieck:section:beliebig}}
\rhead{Beliebige Polynome}
Im Abschnitt~\ref{dreieck:section:hermite-polynome} wurden die
Hermite-Polynome $H_n(t)$ mit $n>0$ als Lösungen des gestellten
Problems erkannt.
Eine Linearkombination von solchen Polynomen hat natürlich
ebenfalls eine elementare Stammfunktion.
Das Problem kann daher neu formuliert werden:

\begin{problem}
\label{dreieck:problem2}
Welche Polynome $P(t)$ lassen sich aus den Hermite-Polynomen
$H_n(t)$ mit $n>0$ linear kombinieren?
\end{problem}

Sei also
\[
P(t) = p_0 + p_1t + \ldots + p_{n-1}t^{n-1} + p_nt^n
\]
ein beliebiges Polynom vom Grad $n$.
Eine elementare Stammfunktion von $P(t)e^{-t^2}$ existiert sicher,
wenn sich $P(t)$ aus den Funktionen $H_n(t)$ mit $n>0$ linear
kombinieren lässt.
Gesucht ist also zunächst eine Darstellung von $P(t)$ als Linearkombination
von Hermite-Polynomen.

\begin{lemma}
Jedes Polynome $P(t)$ vom Grad $n$ lässt sich auf eindeutige Art und
Weise als Linearkombination
\begin{equation}
P(t) = a_0H_0(t) + a_1H_1(t) + \ldots + a_nH_n(t)
=
\sum_{k=0}^n a_nH_n(t)
\label{dreieck:lemma}
\end{equation}
von Hermite-Polynomen schreiben.
\end{lemma}

\begin{proof}[Beweis]
Zunächst halten wir fest, dass aus der
Rekursionsformel~\eqref{dreieck:eqn:rekursion}
folgt, dass der Leitkoeffizient bei jedem Rekursionsschnitt
mit $2$ multipliziert wird.
Der Leitkoeffizient von $H_n(t)$ ist also $2^n$.

Wir führen den Beweis mit vollständiger Induktion.
Für $n=0$ ist $P(t)=p_0 = p_0 H_0(t)$ als Linearkombination von
Hermite-Polynomen darstellbar, dies ist die Induktionsverankerung.

Wir nehmen jetzt im Sinne der Induktionsannahme an,
dass sich ein Polynom vom Grad $n-1$ als
Linearkombination der Polynome $H_0(t),\dots,H_{n-1}(t)$ schreiben
lässt und untersuchen ein Polynom $P(t)$ vom Grad $n$.
Da der Leitkoeffizient des Polynoms $H_n(t)$ ist $2^n$, ist zerlegen
wir
\[
P(t)
=
\underbrace{\biggl(P(t) - \frac{p_n}{2^n} H_n(t)\biggr)}_{\displaystyle = Q(t)}
+
\frac{p_n}{2^n} H_n(t).
\]
Das Polynom $Q(t)$ hat Grad $n-1$, besitzt also nach Induktionsannahme
eine Darstellung
\[
Q(t) = a_0H_0(t)+a_1H_1(t)+\ldots+a_{n-1}H_{n-1}(t)
\]
als Linearkombination der Polynome $H_0(t),\dots,H_{n-1}(t)$.
Somit ist
\[
P(t)
= a_0H_0(t)+a_1H_1(t)+\ldots+a_{n-1}H_{n-1}(t) +
\frac{p_n}{2^n} H_n(t)
\]
eine Darstellung von $P(t)$ als Linearkombination der Polynome
$H_0(t),\dots,H_n(t)$.
Damit ist der Induktionsschritt vollzogen und das Lemma für alle
$n$ bewiesen.
\end{proof}

\begin{satz}
\label{dreieck:satz1}
Die Funktion $P(t)e^{-t^2}$ hat genau dann eine elementare Stammfunktion,
wenn in der Darstellung~\eqref{dreieck:lemma}
von $P(t)$ als Linearkombination von Hermite-Polynomen $a_0=0$ gilt.
\end{satz}

\begin{proof}[Beweis]
Es ist
\begin{align*}
\int P(t)e^{-t^2}\,dt
&=
a_0\int e^{-t^2}\,dt
+
\int
\sum_{k=1} a_kH_k(t)\,dt
\\
&=
a_0
\frac{\sqrt{\pi}}2
\operatorname{erf}(t)
+
\sum_{k=1} a_k\int H_k(t)\,dt.
\end{align*}
Da die Integrale in der Summe alle elementar darstellbar sind,
ist das Integral genau dann elementar, wenn $a_0=0$ ist.
\end{proof}


