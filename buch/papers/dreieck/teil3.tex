%
% teil3.tex -- Beispiel-File für Teil 3
%
% (c) 2020 Prof Dr Andreas Müller, Hochschule Rapperswil
%
\section{Integralbedingung
\label{dreieck:section:integralbedingung}}
\rhead{Lösung}
Die Tatsache, dass die Hermite-Polynome orthogonal sind, erlaubt, das
Kriterium von Satz~\ref{dreieck:satz1} in einer besonders attraktiven
Integralform zu formulieren.

Aus den Polynomen $H_n(t)$ lassen sich durch Normierung die
\index{orthogonale Polynome}%
\index{Polynome, orthogonale}%
orthonormierten Polynome
\[
\tilde{H}_n(t)
=
\frac{1}{\| H_n\|_w} H_n(t)
\qquad\text{mit}\quad
\|H_n\|_w^2
=
\int_{-\infty}^\infty H_n(t)e^{-t^2}\,dt
\]
bilden.
Da diese Polynome eine orthonormierte Basis des Vektorraums der Polynome
bilden, kann die gesuchte Zerlegung eines Polynoms $P(t)$ auch mit
Hilfe des Skalarproduktes gefunden werden:
\begin{align*}
P(t)
&=
\sum_{k=1}^n
\langle \tilde{H}_k, P\rangle_w
\tilde{H}_k(t)
=
\sum_{k=1}^n
\biggl\langle \frac{H_k}{\|H_k\|_w}, P\biggr\rangle_w
\frac{H_k(t)}{\|H_k\|_w}
=
\sum_{k=1}^n
\underbrace{
\frac{ \langle H_k, P\rangle_w }{\|H_k\|_w^2}
}_{\displaystyle =a_k}
H_k(t).
\end{align*}
Die Darstellung von $P(t)$ als Linearkombination von Hermite-Polynomen
hat somit die Koeffizienten
\[
a_k = \frac{\langle H_k,P\rangle_w}{\|H_k\|_w^2}.
\]
Aus dem Kriterium $a_0=0$ dafür, dass eine elementare Stammfunktion
von $P(t)e^{-t^2}$ existiert, wird daher die Bedingung, dass
$\langle H_0,P\rangle_w=0$ ist.
Da $H_0(t)=1$ ist, folgt als Bedingung
\[
a_0
=
\langle H_0,P\rangle_w
=
\int_{-\infty}^\infty P(t) e^{-t^2}\,dt
=
0.
\]

\begin{satz}
Ein Integrand der Form $P(t)e^{-t^2}$ mit einem Polynom $P(t)$
hat genau dann eine elementare Stammfunktion, wenn
\[
\int_{-\infty}^\infty P(t)e^{-t^2}\,dt = 0
\]
ist.
\end{satz}




