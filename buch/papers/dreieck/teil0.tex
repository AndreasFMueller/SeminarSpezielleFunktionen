%
% einleitung.tex -- Beispiel-File für die Einleitung
%
% (c) 2020 Prof Dr Andreas Müller, Hochschule Rapperswil
%
\section{Problemstellung\label{dreieck:section:problemstellung}}
\rhead{Problemstellung}
Es ist bekannt, dass das Fehlerintegral
\index{Fehlerintegral}%
\index{Fehlerfunktion}%
\[
\frac{1}{\sqrt{2\pi}\sigma} \int_{-\infty}^x e^{-\frac{t^2}{2\sigma}}\,dt
\]
nicht in geschlossener Form dargestellt werden kann.
Mit der in Kapitel~\ref{buch:chapter:integral} skizzierten Theorie von
Liouville und dem Risch-Algorithmus kann dies strengt gezeigt werden.
\index{Prinzip von Liouville}%

Andererseits gibt es durchaus Integranden, die $e^{-t^2}$ enthalten,
für die eine Stammfunktion in geschlossener Form gefunden werden kann.
Zum Beispiel folgt aus der Ableitung
\[
\frac{d}{dt} e^{-t^2}
=
-2te^{-t^2}
\]
die Stammfunktion
\[
\int te^{-t^2}\,dt
=
-\frac12 e^{-t^2}.
\]
Leitet man $e^{-t^2}$ zweimal ab, erhält man
\[
\frac{d^2}{dt^2} e^{-t^2}
=
(4t^2-2) e^{-t^2}
\qquad\Rightarrow\qquad
\int (t^2-{\textstyle\frac12}) e^{-t^2}\,dt
=
{\textstyle\frac14}
e^{-t^2}.
\]
Es gibt also viele weitere Polynome $P(t)$, für die der Integrand
$P(t)e^{-t^2}$ eine Stammfunktion in geschlossener Form hat.
Damit stellt sich jetzt das folgende allgemeine Problem.

\begin{problem}
\label{dreieck:problem}
Für welche Polynome $P(t)$ hat der Integrand $P(t)e^{-t^2}$
eine elementare Stammfunktion?
\end{problem}

