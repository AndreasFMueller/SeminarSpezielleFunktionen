%
% teil1.tex -- Beispiel-File für das Paper
%
% (c) 2020 Prof Dr Andreas Müller, Hochschule Rapperswil
%
\section{Hermite-Polynome
\label{dreieck:section:hermite-polynome}}
\rhead{Hermite-Polyome}
In Abschnitt~\ref{dreieck:section:problemstellung} hat sich schon angedeutet,
dass die Polynome, die man durch Ableiten von $e^{-t^2}$ erhalten
kann, bezüglich des gestellten Problems besondere Eigenschaften
haben.
Zunächst halten wir fest, dass die Ableitung einer Funktion der Form
$P(t)e^{-t^2}$ mit einem Polynom $P(t)$ 
\begin{equation}
\frac{d}{dt} P(t)e^{-t^2}
=
P'(t)e^{-t^2} -2tP(t)e^{-t^2}
=
(P'(t)-2tP(t)) e^{-t^2}
\label{dreieck:eqn:ableitung}
\end{equation}
ist.
Insbesondere hat die Ableitung wieder die Form $Q(t)e^{-t^2}$
mit einem Polynome $Q(t)$, welches man auch als
\[
Q(t)
=
e^{t^2}\frac{d}{dt}P(t)e^{-t^2}
\]
erhalten kann.

Die Polynome, die man aus der Funktion $H_0(t)=e^{-t^2}$ durch
Ableiten erhalten kann, wurden bereits in
Abschnitt~\ref{buch:orthogonalitaet:section:rodrigues}
bis auf ein Vorzeichen hergeleitet, sie heissen die Hermite-Polynome 
\index{Hermite-Polynome}%
und es gilt
\[
H_n(t) 
=
(-1)^n
e^{t^2} \frac{d^n}{dt^n} e^{-t^2}.
\]
Das Vorzeichen dient dazu sicherzustellen, dass der Leitkoeffizient
immer $1$ ist.
Das Polynom $H_n(t)$ hat den Grad $n$.

In Abschnitt wurde auch gezeigt, dass die Polynome $H_n(t)$
bezüglich des Skalarproduktes
\[
\langle f,g\rangle_{w}
=
\int_{-\infty}^\infty f(t)g(t)e^{-t^2}\,dt,
\qquad
w(t)=e^{-t^2},
\]
orthogonal sind.
Ausserdem folgt aus \eqref{dreieck:eqn:ableitung}
die Rekursionsbeziehung
\begin{equation}
H_{n}(t)
=
2tH_{n-1}(t)
-
H_{n-1}'(t)
\label{dreieck:eqn:rekursion}
\end{equation}
für $n>0$.

Im Hinblick auf die Problemstellung ist jetzt die Frage interessant,
ob die Integranden $H_n(t)e^{-t^2}$ eine Stammfunktion in geschlossener
Form haben.
Mit Hilfe der Rekursionsbeziehung~\eqref{dreieck:eqn:rekursion}
kann man für $n>0$ unmittelbar verifizieren, dass
\begin{align*}
\int H_n(t)e^{-t^2}\,dt
&=
\int \bigl( 2tH_{n-1}(t) - H'_{n-1}(t)\bigr)e^{-t^2}\,dt
\\
&=
-\int \bigl( \exp'(-t^2) H_{n-1}(t) + H'_{n-1}(t)\bigr)e^{-t^2}\,dt
\\
&=
-\int \bigl( e^{-t^2}H_{n-1}(t)\bigr)' \,dt
=
-e^{-t^2}H_{n-1}(t)
\end{align*}
ist.
Für $n>0$ hat also $H_n(t)e^{-t^2}$ eine elementare Stammfunktion.
Die Hermite-Polynome sind also Lösungen für das
Problem~\ref{dreieck:problem}.


