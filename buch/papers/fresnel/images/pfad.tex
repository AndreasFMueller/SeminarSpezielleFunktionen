%
% pfad.tex -- template for standalon tikz images
%
% (c) 2021 Prof Dr Andreas Müller, OST Ostschweizer Fachhochschule
%
\documentclass[tikz]{standalone}
\usepackage{amsmath}
\usepackage{times}
\usepackage{txfonts}
\usepackage{pgfplots}
\usepackage{csvsimple}
\usetikzlibrary{arrows,intersections,math}
\begin{document}
\def\skala{1}
\definecolor{darkgreen}{rgb}{0,0.6,0}
\begin{tikzpicture}[>=latex,thick,scale=\skala]

\fill[color=gray!40] (0,0) -- (2,0) arc (0:45:2) -- cycle;
\node at (22.5:1.4) {$\displaystyle\frac{\pi}4$};

\draw[->] (-1,0) -- (9,0) coordinate[label={$\operatorname{Re}$}];
\draw[->] (0,-1) -- (0,6) coordinate[label={left:$\operatorname{Im}$}];

\draw[->,color=red,line width=1.4pt] (0,0) -- (7,0);
\draw[->,color=blue,line width=1.4pt] (7,0) arc (0:45:7);
\draw[->,color=darkgreen,line width=1.4pt] (45:7) -- (0,0);

\node[color=red] at (3.5,0) [below] {$\gamma_1(t) = tR$};
\node[color=blue] at (25:7) [right] {$\gamma_2(t) = Re^{it}$};
\node[color=darkgreen] at (45:3.5) [above left] {$\gamma_3(t) = te^{i\pi/4}$};

\node at (7,0) [below] {$R$};
\node at (45:7) [above] {$Re^{i\pi/4}$};

\end{tikzpicture}
\end{document}

