%
% teil3.tex -- Beispiel-File für Teil 3
%
% (c) 2020 Prof Dr Andreas Müller, Hochschule Rapperswil
%
\section{Numerische Berechnung der Fresnel-Integrale
\label{fresnel:section:numerik}}
\rhead{Numerische Berechnung}
Die Fresnel-Integrale können mit verschiedenen Methoden effizient berechnet
werden.

\subsection{Komplexe Fehlerfunktionen}
Es wurde schon darauf hingewiesen, dass der Integrand der Fresnel-Integrale
mit $e^{t^2}$ verwandt ist.
Tatsächlich kann gezeigt werden dass sich die Fresnel-Integrale mit 
Hilfe der komplexen Fehlerfunktion als
\[
\left.
\begin{matrix}
S_1(z)
\\
C_1(z)
\end{matrix}
\;
\right\}
=
\frac{1\pm i}4\biggl(
\operatorname{erf}\biggl(\frac{1+i}2\sqrt{\pi}z\biggr)
\mp
\operatorname{erf}\biggl(\frac{1-i}2\sqrt{\pi}z\biggr)
\biggr)
\]
ausdrücken lassen \cite{fresnel:fresnelC}.
Diese Darstellung ist jedoch für die numerische Berechnung nur
beschränkt nützlich, weil die meisten Bibliotheken für die Fehlerfunktion
diese nur für reelle Argument auszuwerten gestatten.

\subsection{Als Lösung einer Differentialgleichung}
Da die Fresnel-Integrale die sehr einfachen Differentialgleichungen
\[
C'(x) = \cos \biggl(\frac{\pi}2 x^2\biggr)
\qquad\text{und}\qquad
S'(x) = \sin \biggl(\frac{\pi}2 x^2\biggr)
\]
erfüllen, kann man eine Methode zur numerischen Lösung von
Differentialgleichung verwenden.
Die Abbildungen~\ref{fresnel:figure:plot} und \ref{fresnel:figure:eulerspirale}
wurden auf diese Weise erzeugt.

\subsection{Taylor-Reihe integrieren}
Die Taylorreihen
\begin{align*}
\cos x
&=
\sum_{k=0}^\infty \frac{(-1)^k}{(2k)!} x^{2k}
&&\text{und}&
\sin x
&= 
\sum_{k=0}^\infty \frac{(-1)^k}{(2k+1)!} x^{2k+1}
\intertext{%
der trigonometrischen Funktionen werden durch Einsetzen von $x=t^2$
zu}
\cos t^2
&=
\sum_{k=0}^\infty \frac{(-1)^k}{(2k)!} t^{4k}
&&\text{und}&
\sin t^2
&= 
\sum_{k=0}^\infty \frac{(-1)^k}{(2k+1)!} t^{4k+2}.
\intertext{%
Die Fresnel-Integrale $C_1(x)$ und $S_1(x)$ können daher durch
termweise Integration mit Hilfe der Reihen}
C_1(x)
&=
\sum_{k=0}^\infty \frac{(-1)^k}{(2k)!} \frac{x^{4k+1}}{4k+1}
&&\text{und}&
S_1(x)
&=
\sum_{k=0}^\infty \frac{(-1)^k}{(2k+1)!} \frac{x^{4k+3}}{4k+3}
\end{align*}
berechnet werden.
Diese Reihen sind insbesondere für kleine Werte von $x$ sehr
schnell konvergent.

\subsection{Hypergeometrische Reihen}
Aus der Reihenentwicklung kann jetzt auch eine Darstellung der
Fresnel-Integrale durch hypergeometrische Reihen gefunden werden
\cite{fresnel:fresnelC}.
Es ergibt sich
\begin{align*}
S(z)
&=
\frac{\pi z^3}{6}
\cdot
\mathstrut_1F_2\biggl(
\begin{matrix}\frac34\\\frac32,\frac74\end{matrix}
;
-\frac{\pi^2z^4}{16}
\biggr)
\\
C(z)
&=
z
\cdot
\mathstrut_1F_2\biggl(
\begin{matrix}\frac14\\\frac12,\frac54\end{matrix}
;
-\frac{\pi^2z^4}{16}
\biggr).
\end{align*}


