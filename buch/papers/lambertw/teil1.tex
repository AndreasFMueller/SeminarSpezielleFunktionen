%
% teil1.tex -- Beispiel-File für das Paper
%
% (c) 2020 Prof Dr Andreas Müller, Hochschule Rapperswil
%
\section{Wird das Ziel erreicht?
\label{lambertw:section:teil1}}
\rhead{Problemstellung}

Sehr oft kommt es vor, dass bei Verfolgungsproblemen die Frage auftaucht, ob das Ziel überhaupt erreicht wird.
Wenn zum Beispiel die Geschwindigkeit des Verfolgers kleiner ist als diejenige des Ziels, gibt es Anfangsbedingungen bei denen das Ziel nie erreicht wird.
Sobald diese Frage beantwortet wurde stellt sich meist die Frage, wie lange es dauert bis das Ziel erreicht wird.
Diese beiden Fragen werden in diesem Kapitel behandelt und an einem Beispiel betrachtet.

\subsection{Ziel erreichen (überarbeiten)
\label{lambertw:subsection:ZielErreichen}}
Für diese Betrachtung wird das Beispiel aus \eqref{lambertw:section:teil4} zur Hilfe genommen.
Wir verwenden die Hergeleiteten Gleichungen
\begin{align*}
    x\left(t\right)
    &=
    \sqrt{\frac{W\left(\chi\cdot e^{\chi-\frac{4t}{r_0-y_0}}\right)}{\chi}} \\
    y(x)
    &=
    \frac{1}{4}\left(\left(y_0+r_0\right)\eta+\left(r_0-y_0\right)ln\left(\eta\right)-r_0+3y_0\right) \\
    \chi
    &=
    \frac{r_0+y_0}{r_0-y_0}\\
    \eta
    &=
    \left(\frac{x}{x_0}\right)^2 
    \\
    r_0
    &=
    \sqrt{x_0^2+y_0^2} \\
\end{align*}
Wir definieren einen Treffer wenn die Koordinaten des Verfolgers mit denen des Ziels übereinstimmen bei einem diskreten Zeitpunkt $t_1$. Aus dem vorangegangenem Beispiel, sind die Gleichungen zu den x- und y-Koordinaten des Verfolgers bekannt. Die Des Ziels sind

\begin{equation}
    \vec{Z}(t)
    =
    \left( \begin{array}{c} 0 \\ v \cdot t \end{array} \right)
    =
    \left( \begin{array}{c} 0 \\ t \end{array} \right)
    ;\quad
    \vec{V}(t)
    =
    \left( \begin{array}{c} x(t) \\ y(t) \end{array} \right)
    \label{lambertw:Anfangspunkte}
\end{equation}

Somit gilt es

\begin{equation*}
    \vec{Z}(t_1)=\vec{V}(t_1)
\end{equation*}

zu lösen. Da $y(t)$ viel komplexer ist als $x(t)$ wird das Problem in zwei einzelne Teilprobleme zerlegt. Wobei die Bedingung der x- und y-Koordinaten einzeln überprüft werden.

\begin{align*}
    0
    &=
    x(t)
    =
    \sqrt{\frac{W\left(\chi\cdot e^{\chi-\frac{4t}{r_0-y_0}}\right)}{\chi}}
    \\
    v \cdot t
    &=
    y(t)
    =
    \frac{1}{4}\left(\left(y_0+r_0\right)\eta+\left(r_0-y_0\right)ln\left(\eta\right)-r_0+3y_0\right)
    \\
\end{align*}

Zuerst wird die Bedingung der x-Koordinate betrachtet.
Diese kann durch quadrieren und anschliessendes multiplizieren von $\chi$ vereinfacht werden.
Es ist zu beachten, dass $W(x)$ die Lambert W-Funktion ist, welche im Kapitel \eqref{buch:section:lambertw} behandelt wurde.
Die Gleichung

\begin{equation}
    0
    =
    W\left(\chi\cdot e^{\chi-\frac{4t}{r_0-y_0}}\right)
\end{equation}


entspricht genau den Nullstellen der Lambert W-Funktion. Da die Lambert W-Funktion genau eine Nullstelle bei

\begin{equation*}
    W(0)=0
\end{equation*}

besitzt. Kann die Bedingung weiter vereinfacht werden zu

\begin{equation}
    0
    =
    \chi\cdot e^{\chi-\frac{4t}{r_0-y_0}}
\end{equation}

Da $\chi\neq0$ und die Exponentialfunktion nie null sein kann, ist diese Bedingung unmöglich zu erfüllen.
Beim Grenzwert für $t\rightarrow\infty$ geht die Exponentialfunktion gegen null.
Dies nützt nicht viel, da unendlich viel Zeit vergehen müsste damit ein Treffer möglich wäre.
Somit kann nach den Gestellten Bedingungen das Ziel nie getroffen werden.
Dieses Resultat ist aber eher akademischer Natur, weil der Verfolger und das Ziel als Punkt betrachtet wurden.
Wobei aber in Realität nicht von Punkten sondern von Objekten mit einer räumlichen Ausdehnung gesprochen werden kann.
Somit wird in einer nächsten Betrachtung untersucht, ob der Verfolger dem Ziel näher kommt als ein definierter Trefferradius.
Falls dies stattfinden sollte, wird dies als Treffer interpretiert.
Mathematisch kann dies mit

\begin{equation}
    |\vec{V}-\vec{Z]|<a_min \quad a_min\in\mathbb{R}>0
\end{equation}

beschrieben werden, wobei $a_min$ dem Trefferradius entspricht.
Diese Gleichung wird noch quadriert, um die Wurzeln des Betrages loszuwerden.
Da sowohl der Betrag als auch $a_min$ grösser null sind, bleibt die Aussage unverändert.

\begin{equation}
    |\vec{V}-\vec{Z]|^2<a_min^2 \quad a_min\in\mathbb{R}>0
\end{equation}


