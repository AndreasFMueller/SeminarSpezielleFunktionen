%
% teil3.tex -- Beispiel-File für Teil 3
%
% (c) 2020 Prof Dr Andreas Müller, Hochschule Rapperswil
%
\section{Beispiel einer Verfolgungskurve
\label{lambertw:section:teil4}}
\rhead{Beispiel einer Verfolgungskurve}
In diesem Abschnitt wird rechnerisch das Beispiel einer Verfolgungskurve
mit der Verfolgungsstrategie ``Jagd'' beschrieben.
Dafür werden zuerst Bewegungsraum, Anfangspositionen und
\index{Bewegungsraum}%
Bewegungsverhalten definiert, in einem nächsten Schritt soll eine
Differentialgleichung dafür aufgestellt und anschliessend gelöst
werden.

\subsection{Anfangsbedingungen definieren und einsetzen
	\label{lambertw:subsection:Anfangsbedingungen}}
Das zu verfolgende Ziel \(Z\) bewegt sich entlang der \(y\)-Achse
mit konstanter Geschwindigkeit \(|\dot{z}| = 1\), beginnend beim
Ursprung des kartesischen Koordinatensystems.
Der Verfolger \(V\) startet auf einem beliebigen Punkt im ersten
Quadranten und bewegt sich auch mit konstanter Geschwindigkeit
\(|\dot{v}| = 1\) in Richtung Ziel.
Aus diesen Bedingungen ergibt sich der erste Quadrant als Bewegungsraum
für \(V\).
Diese Anfangspunkte oder Anfangsbedingungen können wie folgt formuliert werden:
\begin{equation}
	Z
	=
	\begin{pmatrix} 0 \\ |\dot{z}| \cdot t \end{pmatrix}
	=
	\begin{pmatrix} 0 \\ t \end{pmatrix}
	,\:
	V
	=
	\begin{pmatrix} x \\ y \end{pmatrix}
	\:\text{und}\:\:
	|\dot{v}|
	=
	1.
	\label{lambertw:Anfangsbed}
\end{equation}
Wir haben nun die Anfangsbedingungen definiert, jetzt fehlt nur
noch eine Differentialgleichung, welche die fortlaufende Änderung
der Position und Bewegungsrichtung des Verfolgers beschreibt.
Diese Differentialgleichung haben wir bereits in
Abschnitt~\ref{lambertw:subsection:Verfolger}
definiert, und zwar Gleichung \eqref{lambertw:pursuerDGL}.
Wenn man die Startpunkte einfügt, ergibt sich die Gleichung
\begin{equation}
	\frac{\begin{pmatrix} 0-x \\ t-y \end{pmatrix}}{\sqrt{x^2 + (t-y)^2}}
	\cdot
	\begin{pmatrix} \dot{x} \\ \dot{y} \end{pmatrix}
	=
	1.
	\label{lambertw:eqMitAnfangsbed}
\end{equation}

\subsection{Differentialgleichung vereinfachen
	\label{lambertw:subsection:DGLvereinfach}}
Nun haben wir eine Gleichung, es stellt sich aber die Frage, ob es
überhaupt eine geschlossene Lösung dafür gibt.
Eine Funktion welche die Beziehung \(y(x)\) beschreibt oder sogar
\(x(t)\) und \(y(t)\) liefert.
Zum jetzigen Zeitpunkt mag es nicht trivial scheinen, aber mit den
gewählten Anfangsbedingungen \eqref{lambertw:Anfangsbed} ist es
möglich, eine geschlossene Lösung für die Gleichung
\eqref{lambertw:eqMitAnfangsbed} zu finden.

Auf dem Weg dahin muss die definierte Differentialgleichung zuerst
wesentlich vereinfacht werden, sei es mittels algebraischer Umformungen
oder mit den Tools aus der Analysis.
Da die nächsten Schritte sehr algebralastig sind und sie das Lesen
dieses Papers träge machen würden, werden wir uns hier nur auf die
wesentlichsten Schritte konzentrieren, welche notwendig sind, um
den Lösungsweg nachvollziehen zu können.

\subsubsection{Skalarprodukt auflösen
	\label{lambertw:subsubsection:SkalProdAufl}}
Zuerst müssen wir den Bruch und das Skalarprodukt in \eqref{lambertw:eqMitAnfangsbed} wegbringen, damit wir eine viel handlichere Differentialgleichung erhalten.
Dies führt zu
\begin{equation}
		-x \cdot \dot{x} + (t-y) \cdot \dot{y}
		= \sqrt{x^2 + (t-y)^2}.
		\label{lambertw:eqOhneSkalarprod}
\end{equation}
Im letzten Schritt, fällt die Nützlichkeit des Skalarproduktes in der Verfolgungsgleichung \eqref{lambertw:pursuerDGL} markant auf.
Anstatt zwei gekoppelte Differentialgleichungen zu erhalten, eine für die \(x\)- und die andere für die \(y\)-Komponente, erhält man einen einzigen Ausdruck, was in der Regel mit weniger Lösungsaufwand verbunden ist.

\subsubsection{Quadrieren und Gruppieren
	\label{lambertw:subsubsection:QuadUndGrup}}
Mit der Quadratwurzel in \eqref{lambertw:eqOhneSkalarprod} kann man nichts anfangen, sie steht nur im Weg, also muss man sie loswerden.
Wenn man dies macht, kann \eqref{lambertw:eqOhneSkalarprod} auf die Form  
\begin{equation}
	(\dot{x}^2-1) \cdot x^2 -2x (t-y) \dot{x}\dot{y}
	+
	(\dot{y}^2-1) \cdot (t-y)^2
	=0
	\label{lambertw:eqOhneWurzel}
\end{equation}
gebracht werden.
Diese Form mag auf den ersten Blick nicht gerade nützlich sein,
aber man kann sie mit einer Substitution weiter vereinfachen.

\subsubsection{Wichtige Substitution
	\label{lambertw:subsubsection:WichtSubst}}
Wenn man beachtet, dass die Geschwindigkeit des Verfolgers konstant
und gleich 1 ist, dann ergibt sich die Beziehung
\begin{equation}
	\dot{x}^2 + \dot{y}^2 
	= 1.
	\label{lambertw:eqGeschwSubst}
\end{equation}
Umformungen der Gleichung \eqref{lambertw:eqGeschwSubst} können in
\eqref{lambertw:eqOhneWurzel} erkannt werden.
Wenn man sie ersetzt,
erhält man
\begin{equation}
	\dot{y}^2 \cdot x^2 +2x (t-y) \dot{x}\dot{y}
	+
	\dot{x}^2 \cdot (t-y)^2
	=0.
		\label{lambertw:eqGeschwSubstituiert}
\end{equation}
Diese unscheinbare Substitution führt dazu, dass weitere Vereinfachungen
durchgeführt werden können.

\subsubsection{Binom erkennen und vereinfachen
	\label{lambertw:subsubsection:BinomVereinfach}}
Versteckt im Ausdruck \eqref{lambertw:eqGeschwSubstituiert} befindet sich
die erste binomische Formel, wobei
\begin{equation}
	(x \dot{y} + (t-y) \dot{x})^2
	= 0
	\label{lambertw:eqAlgVerinfacht}
\end{equation}
die faktorisierte Darstellung davon ist.
Da der linke Term gleich Null ist, muss auch die Basis des Quadrates in \eqref{lambertw:eqAlgVerinfacht} gleich Null sein.
Es ergibt sich eine weitere Vereinfachung, welche zu der im Vergleich zu \eqref{lambertw:eqOhneSkalarprod} wesentlich einfacheren Differentialgleichung 
\begin{equation}
	x \dot{y} + (t-y) \dot{x}
	= 0
	\label{lambertw:eqGanzVerinfacht}
\end{equation}
führt.
Kompakt, ohne Wurzelterme und Quadrate, nur elementare Operationen
und Ableitungen.

Nun stellt sich die Frage wie es weiter gehen soll, bei der Gleichung
\eqref{lambertw:eqGanzVerinfacht} scheinen keine weiteren Vereinfachungen
möglich zu sein.
Wir brauchen einen neuen Ansatz, um unser Ziel einer möglichen Lösung
zu verfolgen.

\subsection{Zeitabhängigkeit loswerden
	\label{lambertw:subsection:ZeitabhLoswerden}}
Der nächste logische Schritt scheint irgendwie die Zeitabhängigkeit in der Gleichung \eqref{lambertw:eqGanzVerinfacht} loszuwerden, aber wieso? Nun, wie am Anfang von Abschnitt \ref{lambertw:subsection:DGLvereinfach} beschrieben, suchen wir eine Lösung der Art \(y(x)\), dies ist natürlich erst möglich wenn wir die Abhängigkeit nach \(t\) eliminieren können.

\subsubsection{Zeitliche Ableitungen loswerden
	\label{lambertw:subsubsection:ZeitAbleit}}
Der erste Schritt auf dem Weg zur Funktion \(y(x)\) ist, die zeitlichen Ableitungen los zu werden, dafür wird \eqref{lambertw:eqGanzVerinfacht} beidseitig durch \(\dot{x}\) dividiert, was erlaubt ist, weil diese Änderung ungleich Null ist:
\begin{equation}
	x \frac{\dot{y}}{\dot{x}} + (t-y) \frac{\dot{x}}{\dot{x}}
	= 0.
	\label{lambertw:eqVorKeineZeitAbleit}
\end{equation}
Der Grund dafür ist, dass
\begin{equation}
	\frac{\displaystyle\dot{y}}{\displaystyle\dot{x}} 
	= \frac{\displaystyle\frac{dy}{dt}}{\displaystyle\frac{dx}{dt}}  
	= \frac{dy}{dx}
	= y^{\prime},
	\label{lambertw:eqQuotZeitAbleit}
\end{equation}
und somit kann der Quotient dieser zeitlichen Ableitungen in eine Ableitung nach \(x\) umgewandelt werden.
Nachdem die Eigenschaft \eqref{lambertw:eqQuotZeitAbleit} in \eqref{lambertw:eqVorKeineZeitAbleit} eingesetzt wurde, entsteht beim Vereinfachen die neue Gleichung
\begin{equation}
	x y^{\prime} + t - y
	= 0.
	\label{lambertw:DGLmitT}
\end{equation}

\subsubsection{Variable \(t\) eliminieren
	\label{lambertw:subsubsection:VarTelimin}}
Hier wäre es natürlich passend, wenn man die Abhängigkeit nach \(t\) komplett wegbringen könnte, aber wie?
Wir wissen, dass sich der Verfolger mit Geschwindigkeit 1 bewegt, also legt er in der Zeit \(t\) die Strecke \(1\cdot t = t\) zurück. Längen und Strecken können auch mit der Bogenlänge repräsentiert werden, somit kann Zeit und zurückgelegte Strecke in der Gleichung  
\begin{equation}
	s
	= 
	|\dot{v}| \cdot t
	=
	1 \cdot t
	=
	t
	=
	\int_{\displaystyle x_0}^{\displaystyle x_{\text{end}}}\sqrt{1+y^{\prime\, 2}} \: dx
	\label{lambertw:eqZuBogenlaenge}
\end{equation}
verbunden werden.
  
Nicht gerade auffällig ist die Richtung, in welche hier integriert wird. Wenn der Verfolger sich wie vorgesehen am Anfang im ersten Quadranten befindet, dann muss sich dieser nach links bewegen, was nicht der üblichen Integrationsrichtung entspricht. Um eine Integration wie üblich von links nach rechts ausführen zu können, müssen die Integrationsgrenzen vertauscht werden, was in einem Vorzeichenwechsel resultiert. 

Wenn man nun \eqref{lambertw:eqZuBogenlaenge} in die Differentialgleichung \eqref{lambertw:DGLmitT} einfügt, dann ergibt sich der neue Ausdruck
\begin{equation}
	x y^{\prime} - \int\sqrt{1+y^{\prime\, 2}} \: dx - y
	= 0.
	\label{lambertw:DGLohneT}
\end{equation}
Um das Integral los zu werden, leitet man \eqref{lambertw:DGLohneT} nach \(x\) ab und erhält die Differentialgleichung zweiter Ordnung 
\begin{align}
	y^{\prime}+ xy^{\prime\prime} - \sqrt{1+y^{\prime\, 2}} - y^{\prime}
	&= 0, \\
	xy^{\prime\prime} - \sqrt{1+y^{\prime\, 2}}
	&= 0.
	\label{lambertw:DGLohneInt}
\end{align}
Nun sind wir unserem Ziel einen weiteren Schritt näher. Die Gleichung \eqref{lambertw:DGLohneInt} mag auf den ersten Blick nicht gerade einfach sein, aber im nächsten Abschnitt werden wir sehen, dass sie relativ einfach zu lösen ist.

\subsection{Differentialgleichung lösen
	\label{lambertw:subsection:DGLloes}}
Die Gleichung \eqref{lambertw:DGLohneInt} ist eine Differentialgleichung zweiter Ordnung, in der \(y\) nicht vorkommt. Sie kann mittels der Substitution \(y^{\prime} = u\) in die Differentialgleichung
\begin{equation}
	xu^{\prime} - \sqrt{1+u^2}
	= 0
	\label{lambertw:DGLmitU}
\end{equation}
erster Ordnung umgewandelt werden.
Diese Gleichung ist separierbar, was sie viel handlicher macht. In der separierten Form
\begin{equation}
	\int{\frac{1}{\sqrt{1+u^2}}\:du} 
	= 
	\int{\frac{1}{x}\:dx},
\end{equation}
lässt sich die Gleichung mittels einer Integrationstabelle sehr rasch lösen. 
Das Ergebnis ist 
\begin{align}
	\operatorname{arsinh}(u)
	&=
	\operatorname{ln}(x) + C, \\
	u
	&=
	\operatorname{sinh}(\operatorname{ln}(x) + C).
	\label{lambertw:loesDGLmitU}
\end{align}
Wenn man in \eqref{lambertw:loesDGLmitU} die Substitution rückgängig macht, erhält man die Differentialgleichung 
\begin{equation}
	y^{\prime}
	=
	\operatorname{sinh}(\operatorname{ln}(x) + C)
	\label{lambertw:loesDGLmitY}
\end{equation}
erster Ordnung, die bereits separiert ist.
Ersetzt man den \(\operatorname{sinh}\) durch seine exponentielle Definition \(\operatorname{sinh}(x)=\frac{1}{2}(e^x-e^{-x})\), so resultiert auf sehr einfache Art die Lösung 
\begin{equation}
	y
	=
	C_1 + C_2 x^2 - \frac{\operatorname{ln}(x)}{8 \cdot C_2}
\end{equation}
für \eqref{lambertw:loesDGLmitY}.

Nun haben wir eine Lösung, aber wie es immer mit Lösungen ist, stellt sich die Frage, ob sie überhaupt plausibel ist.

\subsection{Lösung analysieren
	\label{lambertw:subsection:LoesAnalys}}

\definecolor{applegreen}{rgb}{0.55, 0.71, 0.0}

\begin{figure}
	\centering
	\includegraphics{papers/lambertw/Bilder/VerfolgungskurveBsp.png}
	\caption[Graph der Verfolgungskurve]{Graph der Verfolgungskurve wobei, ({\color{red}rot}) die Funktion \ensuremath{y(x)} ist, ({\color{applegreen}grün}) der quadratische Teil und ({\color{blue}blau}) dem \ensuremath{\operatorname{ln}(x)}-Teil entspricht.
	\label{lambertw:BildFunkLoes}
	}
\end{figure}

Das Resultat, wie ersichtlich, ist die Funktion  
\begin{equation}
	{\color{red}{y(x)}}
	=
	C_1 + C_2 {\color{applegreen}{x^2}} {\color{blue}{-}} \frac{\color{blue}{\operatorname{ln}(x)}}{8 \cdot C_2},
	\label{lambertw:funkLoes}
\end{equation}
für welche die Koeffizienten \(C_1\) und \(C_2\) aus den Anfangsbedingungen bestimmt werden können. Zuerst soll aber eine qualitative Intuition oder Idee für das Aussehen der Funktion \(y(x)\) geschaffen werden:
\begin{itemize}
	\item
	Für grosse \(x\)-Werte, welche in der Regel in der Nähe von \(x_0\) sein sollten, ist der quadratisch Term in der Funktion \eqref{lambertw:funkLoes} dominant. 
	\item
	Für immer kleiner werdende \(x\) geht der Verfolger in Richtung \(y\)-Achse, wobei seine Steigung stetig sinkt, was Sinn macht wenn der Verfolgte entlang der \(y\)-Achse steigt. Irgendwann werden Verfolger und Ziel auf gleicher Höhe sein, also gleiche \(y\)- aber verschiedene \(x\)-Koordinate besitzen.
	In diesem Punkt findet ein Monotoniewechsel in der Kurve \eqref{lambertw:funkLoes} statt, was zu einem Minimum führt.
	\item
	Für \(x\)-Werte in der Nähe von \(0\) ist das asymptotische Verhalten des Logarithmus dominant, dies macht auch Sinn, da sich der Verfolgte auf der \(y\)-Achse bewegt und der Verfolger ihm nachgeht.
\end{itemize}
Alle diese Eigenschaften stimmen mit dem überein, was man von einer Kurve dieser Art erwarten würde, welche durch die Grafik \ref{lambertw:BildFunkLoes} repräsentiert wurde.

\subsection{Anfangswertproblem 
	\label{lambertw:subsection:AllgLoes}}
In diesem Abschnitt soll eine Parameterfunktion hergeleitet werden, bei der jeder beliebige Anfangspunkt im ersten Quadranten eingesetzt werden kann, ausser der Ursprung im Koordinatensystem. Diese Aufgabe ist ein Anfangswertproblem für \(y(x)\).

Das Lösen des Anfangswertproblems ist ein Problem aus der Analysis, auf welches hier nicht explizit eingegangen wird. Zur Vollständigkeit und Nachvollziehbarkeit, wird aber das Gleichungssystem präsentiert, welches notwendig ist, um das Anfangswertproblem zu lösen.

\subsubsection{Anfangswerte bestimmen
	\label{lambertw:subsubsection:Anfangswerte}}
Der erste Schritt auf dem Weg zur gesuchten Parameterfunktion ist, die Anfangswerte 
\begin{equation}
	y(x)\big \vert_{t=0}
	=
	y(x_0)
	= 
	y_0
	\label{lambertw:eq1Anfangswert}
\end{equation}
und
\begin{equation}
	\frac{dy}{dx}\bigg \vert_{t=0}
	=
	y^{\prime}(x_0)
	=
	\frac{y_0}{x_0}
	\label{lambertw:eq2Anfangswert}
\end{equation}
zu definieren.
Der zweite Anfangswert \eqref{lambertw:eq2Anfangswert} mag nicht grade offensichtlich sein. Die Erklärung dafür ist aber simpel: Der Verfolger wird sich zum Zeitpunkt \(t=0\) in Richtung Koordinatenursprung bewegen wollen, wo sich das Ziel befindet. Somit entsteht das Steigungsdreieck mit \(\Delta x = x_0\) und \(\Delta y = y_0\).

\subsubsection{Gleichungssystem aufstellen und lösen
	\label{lambertw:subsubsection:GlSys}}
Wenn man die Anfangswerte \eqref{lambertw:eq1Anfangswert} und \eqref{lambertw:eq2Anfangswert} in die Gleichung \eqref{lambertw:funkLoes} und deren Ableitung \(y^{\prime}(x)\) einsetzt, dann ergibt sich das Gleichungssystem
\begin{subequations}
	\label{lambertw:eqGleichungssystem}
	\begin{align}
		y_0
		&=
		C_1 + C_2 x^2_0 - \frac{\operatorname{ln}(x_0)}{8 \cdot C_2}, \\
		\frac{y_0}{x_0}
		&=
		2 \cdot  C_2 x_0 - \frac{1}{8 \cdot C_2 \cdot x_0}.
	\end{align}
\end{subequations}
Damit die gesuchte Funktion im ersten Quadranten bleibt, werden nur die positiven Lösungen  
\begin{subequations}
	\begin{align}
		\label{lambertw:eqKoeff1}
		C_1
		&=
		\frac{2\cdot\operatorname{ln}(x_0)\left(\sqrt{x_0^2 + y_0^2} - y_0 \right) - \sqrt{x_0^2 + y_0^2} + 3 y_0}{4}, \\
		\label{lambertw:eqKoeff2}
		C_2
		&=
		\frac{\sqrt{x_0^2 + y_0^2} + y_0}{4x_0^2}
	\end{align}
\end{subequations}
des Gleichungssystems gewählt.
\subsubsection{Gesuchte Parameterfunktion aufstellen
	\label{lambertw:subsubsection:ParamFunk}}
Wenn man die Koeffizienten \eqref{lambertw:eqKoeff1} und \eqref{lambertw:eqKoeff2} in die Funktion \eqref{lambertw:funkLoes} einsetzt, dann ergibt sich beim Vereinfachen die gesuchte Parameterfunktion
\begin{equation}
	y(x)
	=
	\frac{1}{4}\left((y_0+r_0)\eta+(y_0-r_0)
	\operatorname{ln}(\eta)-r_0+3y_0\right).
	\label{lambertw:eqAllgLoes}
\end{equation}
Damit die Funktion \eqref{lambertw:eqAllgLoes} trotzdem übersichtlich bleibt, wurden Anfangssteigung \(\eta\) und Anfangsentfernung \(r_0\) wie folgt definiert:
\begin{equation}
	\eta
	=
	\biggl(\frac{x}{x_0}\biggr)^2
	\:\:\text{und}\:\:
	r_0
	=
	\sqrt{x_0^2+y_0^2}.
\end{equation}
Diese neue allgemeine Funktion \eqref{lambertw:eqAllgLoes} weist immer noch die selbe Struktur wie die vorher hergeleitete Funktion \eqref{lambertw:funkLoes} auf. Sie enthält einerseits einen quadratischen Teil, der in \(\eta\) enthalten ist, anderseits den \(\operatorname{ln}\)-Teil. Aus dieser Ähnlichkeit kann geschlossen werden, dass sich \eqref{lambertw:eqAllgLoes} auf eine ähnliche Art verhalten wird.

Nun sind wir soweit, dass wir eine \(y(x)\)-Beziehung für beliebige Anfangswerte darstellen können, unser erstes Ziel wurde erreicht. Wir können aber einen Schritt weiter gehen und uns Fragen: Ist es analytisch möglich herauszufinden, wo sich Verfolger und Ziel zu jedem Zeitpunkt befinden? Dieser Frage werden wir im nächsten Abschnitt nachgehen.

\subsection{Funktion nach der Zeit 
	\label{lambertw:subsection:FunkNachT}}
In diesem Abschnitt werden algebraische Umformungen ein wenig detaillierter als zuvor beschrieben.
Dies hat auch einen bestimmten Grund: Den Einsatz einer speziellen
Funktion aufzeigen, sowie auch wann und wieso diese vorkommt.
Welche spezielle Funktion? Fragt man sich wahrscheinlich in diesem Moment.
Nun, um diese Frage kurz zu beantworten, es ist ``YouTube's favorite
special function'' laut dem Mathematiker Michael Penn, die
Lambert-\(W\)-Funktion \(W(x)\) welche im Kapitel
\ref{buch:section:lambertw} bereits beschrieben wurde.

\subsubsection{Zeitabhängigkeit wiederherstellen
	\label{lambertw:subsubsection:ZeitabhWiederherst}}
\index{Zeitabhängigkeit}%
Der erste Schritt ist es herauszufinden, wie die Zeitabhängigkeit
wieder hineingebracht werden kann.
Dafür greifen wir auf die letzte Gleichung zu, in welcher \(t\)
noch enthalten war, und zwar die Differentialgleichung
\begin{equation}
	x y^{\prime} + t - y
	= 0
	\label{lambertw:eqDGLmitTnochmals}
\end{equation}
aus dem Abschnitt \eqref{lambertw:subsection:ZeitabhLoswerden}, welche zur Übersichtlichkeit hier nochmals aufgeführt wurde.
Wie in \eqref{lambertw:eqDGLmitTnochmals} zu sehen ist, werden \(y\) und deren Ableitung \(y^{\prime}\) benötigt, diese sind:
\begin{subequations}
	\label{lambertw:eqFunkUndAbleit}
	\begin{align}
		\label{lambertw:eqFunkUndAbleit1}
		y
		&=
		\frac{1}{4}\bigl((y_0+r_0)\eta+(y_0-r_0)\operatorname{ln}(\eta)-r_0+3y_0\bigr), \\
		y^\prime
		&=
		\frac{1}{2}\biggl((y_0+r_0)\frac{x}{x_0^2}+(y_0-r_0)\frac{1}{x}\biggr).
	\end{align}
\end{subequations}

Wenn man diese Gleichungen \eqref{lambertw:eqFunkUndAbleit} in die Differentialgleichung \eqref{lambertw:eqDGLmitTnochmals} einfügt, vereinfacht und nach \(t\) auflöst, dann ergibt sich der Ausdruck
\begin{equation}
	-4t
	=
	(y_0+r_0)(\eta-1)+(r_0-y_0)\operatorname{ln}(\eta).
	\label{lambertw:eqFunkUndAbleitEingefuegt}
\end{equation}

\subsubsection{Umformungen die zur Funktion nach der Zeit führen
	\label{lambertw:subsubsection:UmformBisZumZiel}}
Mit dem Ausdruck \eqref{lambertw:eqFunkUndAbleitEingefuegt}, welcher Terme mit \(x\) und \(t\) verbindet, kann nun nach der gesuchten Variable \(x\) aufgelöst werden. 

In einem nächsten Schritt wird alles mit \(x\) auf die eine Seite gebracht, der Rest auf die andere Seite und anschliessend beidseitig exponenziert, sodass man 
\begin{equation}
	-4t+(y_0+r_0)
	=
	(y_0+r_0)\eta+(r_0-y_0)\operatorname{ln}(\eta)
\end{equation}
und anschliessend
\begin{equation}
	e^{\displaystyle -4t+(y_0+r_0)}
	=
	e^{\displaystyle (y_0+r_0)\eta}\cdot\eta^{\displaystyle (r_0-y_0)}
	\label{lambertw:eqMitExp}
\end{equation}
erhält.
Auf dem rechten Term von \eqref{lambertw:eqMitExp} beginnen wir langsam eine ähnliche Struktur wie \(\eta e^\eta\) zu erkennen, dies schreit nach der Struktur, die benötigt wird, um \(\eta\) mittels der Lambert-\(W\)-Funktion \(W(x)\) zu erhalten. Dies macht durchaus Sinn, wenn wir die Funktion \(x(t)\) finden wollen und \(W(x)\) die Umkehrfunktion von \(x e^x\) ist. 

Die erste Sache, die uns in \eqref{lambertw:eqMitExp} stört ist, dass \(\eta\) als Potenz da steht. Dieses Problem können wir loswerden, indem wir beidseitig mit \(\:1 / (r_0-y_0)\:\) potenzieren:
\begin{equation}
	\operatorname{exp}\biggl(\displaystyle \frac{-4t}{r_0-y_0}+\frac{y_0+r_0}{r_0-y_0}\biggr)
	=
	\eta\cdot \operatorname{exp}\biggl(\displaystyle \frac{y_0+r_0}{r_0-y_0}\eta\biggr).
	\label{lambertw:eqOhnePotenz}
\end{equation}

\subsubsection{Eine essenzielle Substitution
	\label{lambertw:subsubsection:SubstChi}}
Das nächste Problem, auf welches wir in \eqref{lambertw:eqOhnePotenz} treffen, ist, dass \(\eta\) nicht alleine im Exponent steht. Dies kann elegant mit der Substitution 
\begin{equation}
	\chi
	=
	\frac{y_0+r_0}{r_0-y_0}
	\label{lambertw:eqChiSubst}
\end{equation}
gelöst werden.
Es gäbe natürlich andere Substitutionen wie z.B. 
\[\displaystyle \chi=\frac{y_0+r_0}{r_0-y_0}\cdot\eta,\] 
die auf dasselbe Ergebnis führen würden, aber \eqref{lambertw:eqChiSubst} liefert in einem Schritt die kompakteste Lösung. Also fahren wir mit der Substitution \eqref{lambertw:eqChiSubst} weiter, setzen diese in die Gleichung \eqref{lambertw:eqOhnePotenz} ein und multiplizieren beidseitig mit \(\chi\). Daraus erhalten wir die Gleichung
\begin{equation}
	\chi\cdot \operatorname{exp}\biggl(\displaystyle \chi-\frac{4t}{r_0-y_0}\biggr)
	=
	\chi\eta\cdot e^{\displaystyle \chi\eta}.
	\label{lambertw:eqNachSubst}
\end{equation}

\subsubsection{Funktion nach der Zeit dank Lambert-\(W\)
	\label{lambertw:subsubsection:LambertWundFvonT}}
Nun sind wir endlich soweit, dass wir die angedeutete
Lambert-\(W\)-Funktion \(W(x)\) einsetzen können.
Wenn wir beidseitig \(W(x)\) anwenden, dann erhalten wir den Ausdruck
\begin{equation}
	W\biggl(\chi\cdot \operatorname{exp}\biggl(\displaystyle \chi-\frac{4t}{r_0-y_0}\biggr)\biggr)
	=
	\chi\eta.
\end{equation}
Nach dem Auflösen nach \(x\) welches in \(\eta\) enthalten ist,
erhalten wir die gesuchte \(x(t)\)-Funktion \eqref{lambertw:eqFunkXNachT}.
Dieses \(x(t)\) in Kombination mit \eqref{lambertw:eqFunkUndAbleit1}
liefert die Position des Verfolgers zu jedem Zeitpunkt.
Das Gleichungspaar besteht also aus den Gleichungen
\begin{subequations}
	\label{lambertw:eqFunktionenNachT}
	\begin{align}
		\label{lambertw:eqFunkXNachT}
		x(t)
		&=
		x_0\cdot\sqrt{\frac{W\biggl(\chi\cdot \operatorname{exp}\biggl(\displaystyle \chi-\frac{4t}{r_0-y_0}\biggr)\biggr)}{\chi}}, \\
		\label{lambertw:eqFunkYNachT}
		y(x(t))
		=
		y(t)
		&=
		\frac{1}{4}\biggl((y_0+r_0)\biggl(\frac{x(t)}{x_0}\biggr)^2+(y_0-r_0)\operatorname{ln}\biggl(\biggl(\frac{x(t)}{x_0}\biggr)^2\biggr)-r_0+3y_0\biggr).
	\end{align}
\end{subequations}
Nun haben wir unser letztes Ziel erreicht und sind in der Lage eine Verfolgung rechnerisch sowie graphisch zu repräsentieren.

\subsubsection{Hinweise zur Lambert-\(W\)-Funktion
	\label{lambertw:subsubsection:HinwLambertW}}
Wir sind aber noch nicht ganz fertig, eine Frage muss noch beantwortet werden.
Und zwar wieso man schon bei der Gleichung \eqref{lambertw:eqFunkUndAbleitEingefuegt} weiss, dass die Lambert-\(W\)-Funktion zum Einsatz kommen wird.
Nun, der Grund dafür ist die Struktur
\begin{equation}
	y
	=
	p(x) +\operatorname{ln}(x),
	\label{lambertw:eqEinsatzLambW}
\end{equation}
bei welcher \(p(x)\) eine beliebige Potenz von \(x\) darstellt. 

Jedes Mal wenn \(x\) gesucht ist und in einer Struktur der Art
\eqref{lambertw:eqEinsatzLambW} vorkommt, dann kann mit ein paar
Umformungen die Struktur \(f(x)e^{f(x)}\) erzielt werden.
Wie bereits in Abschnitt~\ref{lambertw:subsection:FunkNachT}
gezeigt wurde, kann \(x\) nun mittels der \(W(x)\)-Funktion aufgelöst
werden.
Erstaunlicherweise ist \eqref{lambertw:eqEinsatzLambW} eine Struktur,
die oft vorkommt, was die Lambert-\(W\)-Funktion so wichtig macht.
