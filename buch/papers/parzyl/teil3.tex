%
% teil3.tex -- Beispiel-File für Teil 3
%
% (c) 2020 Prof Dr Andreas Müller, Hochschule Rapperswil
%
\section{Eigenschaften
\label{parzyl:section:Eigenschaften}}
\rhead{Eigenschaften}

\subsection{Potenzreihenentwicklung
	\label{parzyl:potenz}}
%Die parabolischen Zylinderfunktionen, welche in Gleichung \ref{parzyl:eq:solution_dgl} gegeben sind, 
%können auch als Potenzreihen geschrieben werden
Die parabolischen Zylinderfunktionen können auch als Potenzreihen geschrieben werden.
Im folgenden Abschnitt werden die Terme welche nur von $n$ oder $a$ abhängig sind vernachlässigt.
Die parabolischen Zylinderfunktionen sind Linearkombinationen aus einem geraden Teil $w_1(\alpha, z)$ 
und einem ungeraden Teil $w_2(\alpha, z)$, welche als Potenzreihe
\begin{align}
	w_1(\alpha,z)
	&=  
	e^{-z^2/4} \,
	{}_{1} F_{1}
	(
	\alpha, {\textstyle \frac{1}{2}} ; {\textstyle \frac{1}{2}}z^2) 
	= 
	e^{-\frac{z^2}{4}}
	\sum^{\infty}_{n=0}
	\frac{\left ( \alpha \right )_{n}}{\left ( \frac{1}{2}\right )_{n}}
	\frac{\left ( \frac{1}{2} z^2\right )^n}{n!} \\
	&=
	e^{-\frac{z^2}{4}}
	\left ( 
	1 
	+
	\left ( 2\alpha \right )\frac{z^2}{2!}
	+
	\left ( 2\alpha \right )\left ( 2 + 2\alpha \right )\frac{z^4}{4!}  
	+
	\dots
	\right )
\end{align}
und
\begin{align}
	w_2(\alpha,z)
	&=  
	ze^{-z^2/4} \,
	{}_{1} F_{1}
	(
	{\textstyle \frac{1}{2}} 
	+ \alpha, {\textstyle \frac{3}{2}} ; {\textstyle \frac{1}{2}}z^2) 
	= 
	ze^{-\frac{z^2}{4}}
	\sum^{\infty}_{n=0}
	\frac{\left ( \frac{3}{4} - k \right )_{n}}{\left ( \frac{3}{2}\right )_{n}}
	\frac{\left ( \frac{1}{2} z^2\right )^n}{n!} \\
	&=
	e^{-\frac{z^2}{4}}
	\left ( 
	z 
	+
	\left ( 1 + 2\alpha \right )\frac{z^3}{3!}
	+
	\left ( 1 + 2\alpha \right )\left ( 3 + 2\alpha \right )\frac{z^5}{5!}  
	+
	\dots
	\right ).
\end{align}
sind.
Bei den Potenzreihen sieht man gut, dass die Ordnung des Polynoms im generellen ins unendliche geht. 
Es gibt allerdings die Möglichkeit für bestimmte $\alpha$ das die Terme in der Klammer gleich null werden 
und das Polynom somit eine endliche Ordnung $n$ hat. Dies geschieht bei $w_1(\alpha,z)$ falls
\begin{equation}
	\alpha =  -n \qquad n \in \mathbb{N}_0
\end{equation}
und bei $w_2(\alpha,z)$ falls
\begin{equation}
	\alpha = -\frac{1}{2} - n \qquad n \in \mathbb{N}_0.
\end{equation}

\subsection{Ableitung}
Es kann gezeigt werden, dass die Ableitungen $\frac{\partial w_1(z,k)}{\partial z}$ und $\frac{\partial w_2(z,k)}{\partial z}$ einen Zusammenhang zwischen $w_1(z,k)$ und $w_2(z,k)$ zeigen. Die Ableitung von $w_1(z,k)$ nach $z$ kann über die Produktregel berechnet werden und ist gegeben als
\begin{equation}
	\frac{\partial w_1(z,k)}{\partial z} = \left (\frac{1}{2} - 2k \right ) w_2(z, k -\frac{1}{2}) - \frac{1}{2} z w_1(z,k),
\end{equation} 
und die Ableitung von $w_2(z,k)$ als
\begin{equation}
	\frac{\partial w_2(z,k)}{\partial z} = w_1(z, k -\frac{1}{2}) - \frac{1}{2} z w_2(z,k).
\end{equation}
Über diese Eigenschaft können einfach weitere Ableitungen berechnet werden. 

