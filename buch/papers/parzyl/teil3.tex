%
% teil3.tex -- Beispiel-File für Teil 3
%
% (c) 2020 Prof Dr Andreas Müller, Hochschule Rapperswil
%
\section{Teil 3
\label{parzyl:section:teil3}}
\rhead{Teil 3}
\subsection{Helmholtz Gleichung im parabolischen Zylinderkoordinatensystem
\label{parzyl:subsection:malorum}}
Die Differentialgleichungen, welche zu den parabolischen Zylinderfunktionen führen tauchen, wie bereits erwähnt, dann auf, wenn die Helmholtz Gleichung
\begin{equation}
	\Delta f(x,y,z) = \lambda f(x,y,z) 
\end{equation}
im parabolischen Zylinderkoordinatensystem
\begin{equation}
	\Delta f(\sigma,\tau,z) = \lambda f(\sigma,\tau,z) 
\end{equation}
gelöst wird.
Wobei der Laplace Operator $\Delta$ im parabolischen Zylinderkoordinatensystem gegeben ist als
\begin{equation}
	\nabla 
	= 
	\frac{1}{\sigma^2 + \tau^2}
	\left ( 
	\frac{\partial^2}{\partial \sigma^2} 
	+ 
	\frac{\partial^2}{\partial \tau^2}
	\right )
	+ 
	\frac{\partial^2}{\partial z^2}.
\end{equation}
Die Helmholtz Gleichung würde also wie folgt lauten
\begin{equation}
	\nabla f(\sigma, \tau, z)
	=
	\frac{1}{\sigma^2 + \tau^2}
	\left ( 
	\frac{\partial^2 f(\sigma,\tau,z)}{\partial \sigma^2} 
	+ 
	\frac{\partial^2 f(\sigma,\tau,z)}{\partial \tau^2}
	\right )
	+ 
	\frac{\partial^2 f(\sigma,\tau,z)}{\partial z^2}
	= 
	\lambda f(\sigma,\tau,z)
\end{equation}
Diese partielle Differentialgleichung kann mit Hilfe von Separation gelöst werden, dazu wird 
\begin{equation}
	f(\sigma,\tau,z) = g(\sigma)h(\tau)i(z)
\end{equation}
gesetzt. 
Was dann schlussendlich zu den Differentialgleichungen 
\begin{equation}
	h''(\tau) 
	- 
	\left (
	\lambda\tau^2
	-
	\mu 
	\right )
	h(\tau)
	=
	0
\end{equation}
und 
\begin{equation}
	g''(\sigma) 
	- 
	\left (
	\lambda\sigma^2
	+
	\mu 
	\right )
	g(\sigma)
	=
	0
\end{equation}
führt.
