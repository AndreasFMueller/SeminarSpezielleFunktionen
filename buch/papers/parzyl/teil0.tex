%
% einleitung.tex -- Beispiel-File für die Einleitung
%
% (c) 2020 Prof Dr Andreas Müller, Hochschule Rapperswil
%
\section{Problem\label{parzyl:section:teil0}}
\rhead{Teil 0}

\subsection{Laplace Gleichung}

\subsection{Parabolische Zylinderkoordinaten
\label{parzyl:subsection:finibus}}
Im parabloischen Zylinderkoordinatensystem bilden parabolische Zylinder die Koordinatenflächen.
Die Koordinate $(\sigma, \tau, z)$ sind in kartesischen Koordinaten ausgedrückt mit
\begin{align}
    x & = \sigma \tau \\
    y & = \frac{1}{2}\left(\tau^2 - \sigma^2\right) \\
    z & = z.
\end{align}
Wird $\tau$ oder $\sigma$ konstant gesetzt reultieren die Parabeln
\begin{equation}
    y = \frac{1}{2} \left( \frac{x^2}{\sigma^2} - \sigma^2 \right)
\end{equation}
und 
\begin{equation}
    y = \frac{1}{2} \left( -\frac{x^2}{\tau^2} + \tau^2 \right).
\end{equation}

\subsection{Differnetialgleichung}
Lorem ipsum dolor sit amet, consetetur sadipscing elitr, sed diam
nonumy eirmod tempor invidunt ut labore et dolore magna aliquyam
erat, sed diam voluptua \cite{parzyl:bibtex}.
At vero eos et accusam et justo duo dolores et ea rebum.
Stet clita kasd gubergren, no sea takimata sanctus est Lorem ipsum
dolor sit amet.

Lorem ipsum dolor sit amet, consetetur sadipscing elitr, sed diam
nonumy eirmod tempor invidunt ut labore et dolore magna aliquyam
erat, sed diam voluptua.
At vero eos et accusam et justo duo dolores et ea rebum.  Stet clita
kasd gubergren, no sea takimata sanctus est Lorem ipsum dolor sit
amet.


