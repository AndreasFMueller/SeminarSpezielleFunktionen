%
% teil1.tex -- Beispiel-File für das Paper
%
% (c) 2020 Prof Dr Andreas Müller, Hochschule Rapperswil
%
\section{Lösung
\label{parzyl:section:teil1}}
\rhead{Problemstellung}
Die Differentialgleichungen \eqref{parzyl:sep_dgl_1} und \eqref{parzyl:sep_dgl_2} können mit einer Substitution
in die Whittaker Gleichung gelöst werden.
\begin{definition}
    Die Funktion 
    \begin{equation*}
        W_{k,m}(z) = 
    e^{-z/2} z^{m+1/2} \,
    {}_{1} F_{1}
    (
        {\textstyle \frac{1}{2}} 
        + m - k, 1 + 2m; z)
    \end{equation*}
    heisst Whittaker Funktion und ist eine Lösung
    von
    \begin{equation}
        \frac{d^2W}{d z^2} +
        \left(-\frac{1}{4}  + \frac{k}{z} + \frac{\frac{1}{4} - m^2}{z^2} \right) W = 0.
    \end{equation}
\end{definition}

Lösung Folgt\dots


