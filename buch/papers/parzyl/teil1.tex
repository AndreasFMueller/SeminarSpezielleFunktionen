%
% teil1.tex -- Beispiel-File für das Paper
%
% (c) 2020 Prof Dr Andreas Müller, Hochschule Rapperswil
%
\section{Lösung
\label{parzyl:section:teil1}}
\rhead{Problemstellung}
Die Differentialgleichungen \eqref{parzyl:sep_dgl_1} und \eqref{parzyl:sep_dgl_2} können mit
Hilfe der Whittaker Gleichung gelöst werden.
\begin{definition}
    Die Funktion 
    \begin{equation*}
        W_{k,m}(z) = 
    e^{-z/2} z^{m+1/2} \,
    {}_{1} F_{1}
    (
        {\textstyle \frac{1}{2}} 
        + m - k, 1 + 2m; z)
    \end{equation*}
    heisst Whittaker Funktion und ist eine Lösung
    von der Whittaker Differentialgleichung
    \begin{equation}
        \frac{d^2W}{d z^2} +
        \left(-\frac{1}{4}  + \frac{k}{z} + \frac{\frac{1}{4} - m^2}{z^2} \right) W = 0.
        \label{parzyl:eq:whitDiffEq}
    \end{equation}
\end{definition}
Es wird nun die Differentialgleichung bestimmt, welche
\begin{equation}
    w = z^{-1/2} W_{k,-1/4} \left({\textstyle \frac{1}{2}} z^2\right)
\end{equation}
als Lösung hat.
Dafür wird $w$ in \eqref{parzyl:eq:whitDiffEq} eingesetzt woraus
\begin{equation}
    \frac{d^2 w}{dz^2} - \left(\frac{1}{4} z^2 - 2k\right) w = 0
\label{parzyl:eq:weberDiffEq}
\end{equation}
resultiert. DIese Differentialgleichung ist dieselbe wie 
\eqref{parzyl:sep_dgl_2} und \eqref{parzyl:sep_dgl_2}, welche somit
$w$ als Lösung haben.
Da es sich um eine Differentialgleichung zweiter Ordnung handelt, hat sie nicht nur
eine sondern zwei Lösungen.
Die zweite Lösung der Whittaker-Gleichung ist $W_{k,-m} (z)$.
Somit hat \eqref{parzyl:eq:weberDiffEq}
\begin{align}
    w_1 & = z^{-1/2} W_{k,-1/4} \left({\textstyle \frac{1}{2}} z^2\right)\\
    w_2 & = z^{-1/2} W_{k,1/4} \left({\textstyle \frac{1}{2}} z^2\right)
\end{align}
als Lösungen.

Ausgeschrieben ergeben sich als Lösungen
\begin{align}
    w_1 &= e^{-z^2/4} \,
    {}_{1} F_{1}
    (
        {\textstyle \frac{1}{4}} 
         - k, {\textstyle \frac{1}{2}} ; {\textstyle \frac{1}{2}}z^2) \\
    w_2 & = z e^{-z^2/4} \,
         {}_{1} F_{1}
         ({\textstyle \frac{3}{4}} 
              - k, {\textstyle \frac{3}{2}} ; {\textstyle \frac{1}{2}}z^2)
\end{align}