%
% teil2.tex -- Beispiel-File für teil2 
%
% (c) 2020 Prof Dr Andreas Müller, Hochschule Rapperswil
%
\section{Anwendung in der Physik 
\label{parzyl:section:teil2}}
\rhead{Teil 2}


\subsection{Elektrisches Feld einer semi-infiniten Platte
\label{parzyl:subsection:bonorum}}
Die parabolischen Zylinderkoordinaten tauchen auf, wenn man das elektrische Feld einer semi-infiniten Platte finden will.
Das dies so ist kann im zwei Dimensionalen mit Hilfe von komplexen Funktionen gezeigt werden. Die Platte ist dann nur eine Linie, was man in Abbildung TODO sieht.
Jede komplexe Funktion $F(z)$ kann geschrieben werden als
\begin{equation}
	F(s) = U(x,y) + iV(x,y) \qquad s \in \mathbb{C}; x,y \in \mathbb{R}.
\end{equation}  
Dabei muss gelten, falls die Funktion differenzierbar ist, dass
\begin{equation}
	\frac{\partial U(x,y)}{\partial x} 
	=
	\frac{\partial V(x,y)}{\partial y} 
	\qquad
	\frac{\partial V(x,y)}{\partial x}
	=
	-\frac{\partial U(x,y)}{\partial y}.
\end{equation}
Aus dieser Bedingung folgt 
\begin{equation}
	\label{parzyl_e_feld_zweite_ab}
	\underbrace{
	\frac{\partial^2 U(x,y)}{\partial x^2}
	+ 
	\frac{\partial^2 U(x,y)}{\partial y^2}
	=
	0
	}_{\displaystyle{\nabla^2U(x,y)=0}}
	\qquad
	\underbrace{
	\frac{\partial^2 V(x,y)}{\partial x^2}
	+
	\frac{\partial^2 V(x,y)}{\partial y^2}
	=
	0
	}_{\displaystyle{\nabla^2V(x,y) = 0}}.
\end{equation}
Zusätzlich kann auch gezeigt werden, dass die Funktion $F(z)$ eine winkeltreue Abbildung ist. 
Der Zusammenhang zum elektrischen Feld ist jetzt, dass das Potential an einem quellenfreien Punkt gegeben ist als 
\begin{equation}
	\nabla^2\phi(x,y) = 0.
\end{equation}
Dies ist eine Bedingung welche differenzierbare Funktionen, wie in Gleichung \ref{parzyl_e_feld_zweite_ab} gezeigt wird, bereits besitzen. 
Nun kann zum Beispiel $U(x,y)$ als das Potential angeschaut werden
\begin{equation}
	\phi(x,y) = U(x,y).
\end{equation}
Orthogonal zum Potential ist das elektrische Feld
\begin{equation}
	E(x,y) = V(x,y).
\end{equation}
Um nun zu den parabolische Zylinderkoordinaten zu gelangen muss nur noch eine geeignete komplexe Funktion $F(s)$ gefunden werden, 
welche eine semi-infinite Platte beschreiben kann.
Die gesuchte Funktion in diesem Fall ist
\begin{equation}
	F(s) 
	= 
	\sqrt{s} 
	= 
	\sqrt{x + iy}.
\end{equation}
Dies kann umgeformt werden zu
\begin{equation}
	F(s) 
	= 
	\underbrace{\sqrt{\frac{\sqrt{x^2+y^2} + x}{2}}}_{U(x,y)} 
	+ 
	i\underbrace{\sqrt{\frac{\sqrt{x^2+y^2} - x}{2}}}_{V(x,y)}
	.
\end{equation}
Die Äquipotentialflächen können nun betrachtet werden, indem man die Funktion welche das Potential beschreibt gleich eine Konstante setzt,
\begin{equation}
	\sigma = U(x,y) = \sqrt{\frac{\sqrt{x^2+y^2} + x}{2}},
\end{equation}
und die Flächen mit der gleichen elektrischen Feldstärke können als
\begin{equation}
	\tau = V(x,y) = \sqrt{\frac{\sqrt{x^2+y^2} - x}{2}}
\end{equation}
beschrieben werden. Diese zwei Gleichungen zeigen nun wie man vom kartesischen Koordinatensystem ins parabolische Zylinderkoordinatensystem kommt. Werden diese Formeln nun nach x und y aufgelöst so beschreibe sie, wie man aus dem parabolischen Zylinderkoordinatensystem zurück ins kartesische rechnen kann
\begin{equation}
	x = \sigma \tau,
\end{equation}
\begin{equation}
	y = \frac{1}{2}\left ( \tau^2 - \sigma^2 \right )
\end{equation}





