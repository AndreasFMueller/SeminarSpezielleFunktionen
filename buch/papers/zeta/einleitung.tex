\section{Einleitung} \label{zeta:section:einleitung}
\rhead{Einleitung}

Die Riemannsche Zetafunktion $\zeta(s)$ ist für alle komplexe $s$ mit $\Re(s) > 1$ definiert als
\begin{equation}\label{zeta:equation1}
    \zeta(s)
    =
    \sum_{n=1}^{\infty}
    \frac{1}{n^s}.
\end{equation}
Die Zetafunktion ist bekannt als Bestandteil der Riemannschen Vermutung, welche besagt das alle nichttrivialen Nullstellen der Zetafunktion einen Realteil von $\frac{1}{2}$ haben.
Mithilfe dieser Vermutung kann eine gute Annäherung an die Primzahlfunktion gefunden werden.
Die Primzahlfunktion steigt immer an, sobald eine Primzahl vorkommt.
Eine Darstellung davon ist in Abbildung \ref{fig:zeta:primzahlfunktion} zu finden.
Die Riemannsche Vermutung ist eines der ungelösten Millennium-Probleme der Mathematik, auf deren Lösung eine Belohnung von einer Million Doller ausgesetzt ist \cite{zeta:online:millennium}.
Auf eine genauere Beschreibung der Riemannschen Vermutung wird im Rahmen dieses Papers nicht eingegangen.
\begin{figure}
    \centering
    %% Creator: Matplotlib, PGF backend
%%
%% To include the figure in your LaTeX document, write
%%   \input{<filename>.pgf}
%%
%% Make sure the required packages are loaded in your preamble
%%   \usepackage{pgf}
%%
%% and, on pdftex
%%   \usepackage[utf8]{inputenc}\DeclareUnicodeCharacter{2212}{-}
%%
%% or, on luatex and xetex
%%   \usepackage{unicode-math}
%%
%% Figures using additional raster images can only be included by \input if
%% they are in the same directory as the main LaTeX file. For loading figures
%% from other directories you can use the `import` package
%%   \usepackage{import}
%%
%% and then include the figures with
%%   \import{<path to file>}{<filename>.pgf}
%%
%% Matplotlib used the following preamble
%%
\begingroup%
\makeatletter%
\begin{pgfpicture}%
\pgfpathrectangle{\pgfpointorigin}{\pgfqpoint{5.440000in}{3.480000in}}%
\pgfusepath{use as bounding box, clip}%
\begin{pgfscope}%
\pgfsetbuttcap%
\pgfsetmiterjoin%
\definecolor{currentfill}{rgb}{1.000000,1.000000,1.000000}%
\pgfsetfillcolor{currentfill}%
\pgfsetlinewidth{0.000000pt}%
\definecolor{currentstroke}{rgb}{1.000000,1.000000,1.000000}%
\pgfsetstrokecolor{currentstroke}%
\pgfsetdash{}{0pt}%
\pgfpathmoveto{\pgfqpoint{0.000000in}{0.000000in}}%
\pgfpathlineto{\pgfqpoint{5.440000in}{0.000000in}}%
\pgfpathlineto{\pgfqpoint{5.440000in}{3.480000in}}%
\pgfpathlineto{\pgfqpoint{0.000000in}{3.480000in}}%
\pgfpathclose%
\pgfusepath{fill}%
\end{pgfscope}%
\begin{pgfscope}%
\pgfsetbuttcap%
\pgfsetmiterjoin%
\definecolor{currentfill}{rgb}{1.000000,1.000000,1.000000}%
\pgfsetfillcolor{currentfill}%
\pgfsetlinewidth{0.000000pt}%
\definecolor{currentstroke}{rgb}{0.000000,0.000000,0.000000}%
\pgfsetstrokecolor{currentstroke}%
\pgfsetstrokeopacity{0.000000}%
\pgfsetdash{}{0pt}%
\pgfpathmoveto{\pgfqpoint{0.680000in}{0.382800in}}%
\pgfpathlineto{\pgfqpoint{4.896000in}{0.382800in}}%
\pgfpathlineto{\pgfqpoint{4.896000in}{3.062400in}}%
\pgfpathlineto{\pgfqpoint{0.680000in}{3.062400in}}%
\pgfpathclose%
\pgfusepath{fill}%
\end{pgfscope}%
\begin{pgfscope}%
\pgfsetbuttcap%
\pgfsetroundjoin%
\definecolor{currentfill}{rgb}{0.000000,0.000000,0.000000}%
\pgfsetfillcolor{currentfill}%
\pgfsetlinewidth{0.803000pt}%
\definecolor{currentstroke}{rgb}{0.000000,0.000000,0.000000}%
\pgfsetstrokecolor{currentstroke}%
\pgfsetdash{}{0pt}%
\pgfsys@defobject{currentmarker}{\pgfqpoint{0.000000in}{-0.048611in}}{\pgfqpoint{0.000000in}{0.000000in}}{%
\pgfpathmoveto{\pgfqpoint{0.000000in}{0.000000in}}%
\pgfpathlineto{\pgfqpoint{0.000000in}{-0.048611in}}%
\pgfusepath{stroke,fill}%
}%
\begin{pgfscope}%
\pgfsys@transformshift{0.871636in}{0.382800in}%
\pgfsys@useobject{currentmarker}{}%
\end{pgfscope}%
\end{pgfscope}%
\begin{pgfscope}%
\definecolor{textcolor}{rgb}{0.000000,0.000000,0.000000}%
\pgfsetstrokecolor{textcolor}%
\pgfsetfillcolor{textcolor}%
\pgftext[x=0.871636in,y=0.285578in,,top]{\color{textcolor}\sffamily\fontsize{10.000000}{12.000000}\selectfont 0}%
\end{pgfscope}%
\begin{pgfscope}%
\pgfsetbuttcap%
\pgfsetroundjoin%
\definecolor{currentfill}{rgb}{0.000000,0.000000,0.000000}%
\pgfsetfillcolor{currentfill}%
\pgfsetlinewidth{0.803000pt}%
\definecolor{currentstroke}{rgb}{0.000000,0.000000,0.000000}%
\pgfsetstrokecolor{currentstroke}%
\pgfsetdash{}{0pt}%
\pgfsys@defobject{currentmarker}{\pgfqpoint{0.000000in}{-0.048611in}}{\pgfqpoint{0.000000in}{0.000000in}}{%
\pgfpathmoveto{\pgfqpoint{0.000000in}{0.000000in}}%
\pgfpathlineto{\pgfqpoint{0.000000in}{-0.048611in}}%
\pgfusepath{stroke,fill}%
}%
\begin{pgfscope}%
\pgfsys@transformshift{1.510424in}{0.382800in}%
\pgfsys@useobject{currentmarker}{}%
\end{pgfscope}%
\end{pgfscope}%
\begin{pgfscope}%
\definecolor{textcolor}{rgb}{0.000000,0.000000,0.000000}%
\pgfsetstrokecolor{textcolor}%
\pgfsetfillcolor{textcolor}%
\pgftext[x=1.510424in,y=0.285578in,,top]{\color{textcolor}\sffamily\fontsize{10.000000}{12.000000}\selectfont 5}%
\end{pgfscope}%
\begin{pgfscope}%
\pgfsetbuttcap%
\pgfsetroundjoin%
\definecolor{currentfill}{rgb}{0.000000,0.000000,0.000000}%
\pgfsetfillcolor{currentfill}%
\pgfsetlinewidth{0.803000pt}%
\definecolor{currentstroke}{rgb}{0.000000,0.000000,0.000000}%
\pgfsetstrokecolor{currentstroke}%
\pgfsetdash{}{0pt}%
\pgfsys@defobject{currentmarker}{\pgfqpoint{0.000000in}{-0.048611in}}{\pgfqpoint{0.000000in}{0.000000in}}{%
\pgfpathmoveto{\pgfqpoint{0.000000in}{0.000000in}}%
\pgfpathlineto{\pgfqpoint{0.000000in}{-0.048611in}}%
\pgfusepath{stroke,fill}%
}%
\begin{pgfscope}%
\pgfsys@transformshift{2.149212in}{0.382800in}%
\pgfsys@useobject{currentmarker}{}%
\end{pgfscope}%
\end{pgfscope}%
\begin{pgfscope}%
\definecolor{textcolor}{rgb}{0.000000,0.000000,0.000000}%
\pgfsetstrokecolor{textcolor}%
\pgfsetfillcolor{textcolor}%
\pgftext[x=2.149212in,y=0.285578in,,top]{\color{textcolor}\sffamily\fontsize{10.000000}{12.000000}\selectfont 10}%
\end{pgfscope}%
\begin{pgfscope}%
\pgfsetbuttcap%
\pgfsetroundjoin%
\definecolor{currentfill}{rgb}{0.000000,0.000000,0.000000}%
\pgfsetfillcolor{currentfill}%
\pgfsetlinewidth{0.803000pt}%
\definecolor{currentstroke}{rgb}{0.000000,0.000000,0.000000}%
\pgfsetstrokecolor{currentstroke}%
\pgfsetdash{}{0pt}%
\pgfsys@defobject{currentmarker}{\pgfqpoint{0.000000in}{-0.048611in}}{\pgfqpoint{0.000000in}{0.000000in}}{%
\pgfpathmoveto{\pgfqpoint{0.000000in}{0.000000in}}%
\pgfpathlineto{\pgfqpoint{0.000000in}{-0.048611in}}%
\pgfusepath{stroke,fill}%
}%
\begin{pgfscope}%
\pgfsys@transformshift{2.788000in}{0.382800in}%
\pgfsys@useobject{currentmarker}{}%
\end{pgfscope}%
\end{pgfscope}%
\begin{pgfscope}%
\definecolor{textcolor}{rgb}{0.000000,0.000000,0.000000}%
\pgfsetstrokecolor{textcolor}%
\pgfsetfillcolor{textcolor}%
\pgftext[x=2.788000in,y=0.285578in,,top]{\color{textcolor}\sffamily\fontsize{10.000000}{12.000000}\selectfont 15}%
\end{pgfscope}%
\begin{pgfscope}%
\pgfsetbuttcap%
\pgfsetroundjoin%
\definecolor{currentfill}{rgb}{0.000000,0.000000,0.000000}%
\pgfsetfillcolor{currentfill}%
\pgfsetlinewidth{0.803000pt}%
\definecolor{currentstroke}{rgb}{0.000000,0.000000,0.000000}%
\pgfsetstrokecolor{currentstroke}%
\pgfsetdash{}{0pt}%
\pgfsys@defobject{currentmarker}{\pgfqpoint{0.000000in}{-0.048611in}}{\pgfqpoint{0.000000in}{0.000000in}}{%
\pgfpathmoveto{\pgfqpoint{0.000000in}{0.000000in}}%
\pgfpathlineto{\pgfqpoint{0.000000in}{-0.048611in}}%
\pgfusepath{stroke,fill}%
}%
\begin{pgfscope}%
\pgfsys@transformshift{3.426788in}{0.382800in}%
\pgfsys@useobject{currentmarker}{}%
\end{pgfscope}%
\end{pgfscope}%
\begin{pgfscope}%
\definecolor{textcolor}{rgb}{0.000000,0.000000,0.000000}%
\pgfsetstrokecolor{textcolor}%
\pgfsetfillcolor{textcolor}%
\pgftext[x=3.426788in,y=0.285578in,,top]{\color{textcolor}\sffamily\fontsize{10.000000}{12.000000}\selectfont 20}%
\end{pgfscope}%
\begin{pgfscope}%
\pgfsetbuttcap%
\pgfsetroundjoin%
\definecolor{currentfill}{rgb}{0.000000,0.000000,0.000000}%
\pgfsetfillcolor{currentfill}%
\pgfsetlinewidth{0.803000pt}%
\definecolor{currentstroke}{rgb}{0.000000,0.000000,0.000000}%
\pgfsetstrokecolor{currentstroke}%
\pgfsetdash{}{0pt}%
\pgfsys@defobject{currentmarker}{\pgfqpoint{0.000000in}{-0.048611in}}{\pgfqpoint{0.000000in}{0.000000in}}{%
\pgfpathmoveto{\pgfqpoint{0.000000in}{0.000000in}}%
\pgfpathlineto{\pgfqpoint{0.000000in}{-0.048611in}}%
\pgfusepath{stroke,fill}%
}%
\begin{pgfscope}%
\pgfsys@transformshift{4.065576in}{0.382800in}%
\pgfsys@useobject{currentmarker}{}%
\end{pgfscope}%
\end{pgfscope}%
\begin{pgfscope}%
\definecolor{textcolor}{rgb}{0.000000,0.000000,0.000000}%
\pgfsetstrokecolor{textcolor}%
\pgfsetfillcolor{textcolor}%
\pgftext[x=4.065576in,y=0.285578in,,top]{\color{textcolor}\sffamily\fontsize{10.000000}{12.000000}\selectfont 25}%
\end{pgfscope}%
\begin{pgfscope}%
\pgfsetbuttcap%
\pgfsetroundjoin%
\definecolor{currentfill}{rgb}{0.000000,0.000000,0.000000}%
\pgfsetfillcolor{currentfill}%
\pgfsetlinewidth{0.803000pt}%
\definecolor{currentstroke}{rgb}{0.000000,0.000000,0.000000}%
\pgfsetstrokecolor{currentstroke}%
\pgfsetdash{}{0pt}%
\pgfsys@defobject{currentmarker}{\pgfqpoint{0.000000in}{-0.048611in}}{\pgfqpoint{0.000000in}{0.000000in}}{%
\pgfpathmoveto{\pgfqpoint{0.000000in}{0.000000in}}%
\pgfpathlineto{\pgfqpoint{0.000000in}{-0.048611in}}%
\pgfusepath{stroke,fill}%
}%
\begin{pgfscope}%
\pgfsys@transformshift{4.704364in}{0.382800in}%
\pgfsys@useobject{currentmarker}{}%
\end{pgfscope}%
\end{pgfscope}%
\begin{pgfscope}%
\definecolor{textcolor}{rgb}{0.000000,0.000000,0.000000}%
\pgfsetstrokecolor{textcolor}%
\pgfsetfillcolor{textcolor}%
\pgftext[x=4.704364in,y=0.285578in,,top]{\color{textcolor}\sffamily\fontsize{10.000000}{12.000000}\selectfont 30}%
\end{pgfscope}%
\begin{pgfscope}%
\pgfsetbuttcap%
\pgfsetroundjoin%
\definecolor{currentfill}{rgb}{0.000000,0.000000,0.000000}%
\pgfsetfillcolor{currentfill}%
\pgfsetlinewidth{0.803000pt}%
\definecolor{currentstroke}{rgb}{0.000000,0.000000,0.000000}%
\pgfsetstrokecolor{currentstroke}%
\pgfsetdash{}{0pt}%
\pgfsys@defobject{currentmarker}{\pgfqpoint{-0.048611in}{0.000000in}}{\pgfqpoint{-0.000000in}{0.000000in}}{%
\pgfpathmoveto{\pgfqpoint{-0.000000in}{0.000000in}}%
\pgfpathlineto{\pgfqpoint{-0.048611in}{0.000000in}}%
\pgfusepath{stroke,fill}%
}%
\begin{pgfscope}%
\pgfsys@transformshift{0.680000in}{0.504600in}%
\pgfsys@useobject{currentmarker}{}%
\end{pgfscope}%
\end{pgfscope}%
\begin{pgfscope}%
\definecolor{textcolor}{rgb}{0.000000,0.000000,0.000000}%
\pgfsetstrokecolor{textcolor}%
\pgfsetfillcolor{textcolor}%
\pgftext[x=0.513333in, y=0.456375in, left, base]{\color{textcolor}\sffamily\fontsize{10.000000}{12.000000}\selectfont 0}%
\end{pgfscope}%
\begin{pgfscope}%
\pgfsetbuttcap%
\pgfsetroundjoin%
\definecolor{currentfill}{rgb}{0.000000,0.000000,0.000000}%
\pgfsetfillcolor{currentfill}%
\pgfsetlinewidth{0.803000pt}%
\definecolor{currentstroke}{rgb}{0.000000,0.000000,0.000000}%
\pgfsetstrokecolor{currentstroke}%
\pgfsetdash{}{0pt}%
\pgfsys@defobject{currentmarker}{\pgfqpoint{-0.048611in}{0.000000in}}{\pgfqpoint{-0.000000in}{0.000000in}}{%
\pgfpathmoveto{\pgfqpoint{-0.000000in}{0.000000in}}%
\pgfpathlineto{\pgfqpoint{-0.048611in}{0.000000in}}%
\pgfusepath{stroke,fill}%
}%
\begin{pgfscope}%
\pgfsys@transformshift{0.680000in}{0.991800in}%
\pgfsys@useobject{currentmarker}{}%
\end{pgfscope}%
\end{pgfscope}%
\begin{pgfscope}%
\definecolor{textcolor}{rgb}{0.000000,0.000000,0.000000}%
\pgfsetstrokecolor{textcolor}%
\pgfsetfillcolor{textcolor}%
\pgftext[x=0.513333in, y=0.943575in, left, base]{\color{textcolor}\sffamily\fontsize{10.000000}{12.000000}\selectfont 2}%
\end{pgfscope}%
\begin{pgfscope}%
\pgfsetbuttcap%
\pgfsetroundjoin%
\definecolor{currentfill}{rgb}{0.000000,0.000000,0.000000}%
\pgfsetfillcolor{currentfill}%
\pgfsetlinewidth{0.803000pt}%
\definecolor{currentstroke}{rgb}{0.000000,0.000000,0.000000}%
\pgfsetstrokecolor{currentstroke}%
\pgfsetdash{}{0pt}%
\pgfsys@defobject{currentmarker}{\pgfqpoint{-0.048611in}{0.000000in}}{\pgfqpoint{-0.000000in}{0.000000in}}{%
\pgfpathmoveto{\pgfqpoint{-0.000000in}{0.000000in}}%
\pgfpathlineto{\pgfqpoint{-0.048611in}{0.000000in}}%
\pgfusepath{stroke,fill}%
}%
\begin{pgfscope}%
\pgfsys@transformshift{0.680000in}{1.479000in}%
\pgfsys@useobject{currentmarker}{}%
\end{pgfscope}%
\end{pgfscope}%
\begin{pgfscope}%
\definecolor{textcolor}{rgb}{0.000000,0.000000,0.000000}%
\pgfsetstrokecolor{textcolor}%
\pgfsetfillcolor{textcolor}%
\pgftext[x=0.513333in, y=1.430775in, left, base]{\color{textcolor}\sffamily\fontsize{10.000000}{12.000000}\selectfont 4}%
\end{pgfscope}%
\begin{pgfscope}%
\pgfsetbuttcap%
\pgfsetroundjoin%
\definecolor{currentfill}{rgb}{0.000000,0.000000,0.000000}%
\pgfsetfillcolor{currentfill}%
\pgfsetlinewidth{0.803000pt}%
\definecolor{currentstroke}{rgb}{0.000000,0.000000,0.000000}%
\pgfsetstrokecolor{currentstroke}%
\pgfsetdash{}{0pt}%
\pgfsys@defobject{currentmarker}{\pgfqpoint{-0.048611in}{0.000000in}}{\pgfqpoint{-0.000000in}{0.000000in}}{%
\pgfpathmoveto{\pgfqpoint{-0.000000in}{0.000000in}}%
\pgfpathlineto{\pgfqpoint{-0.048611in}{0.000000in}}%
\pgfusepath{stroke,fill}%
}%
\begin{pgfscope}%
\pgfsys@transformshift{0.680000in}{1.966200in}%
\pgfsys@useobject{currentmarker}{}%
\end{pgfscope}%
\end{pgfscope}%
\begin{pgfscope}%
\definecolor{textcolor}{rgb}{0.000000,0.000000,0.000000}%
\pgfsetstrokecolor{textcolor}%
\pgfsetfillcolor{textcolor}%
\pgftext[x=0.513333in, y=1.917975in, left, base]{\color{textcolor}\sffamily\fontsize{10.000000}{12.000000}\selectfont 6}%
\end{pgfscope}%
\begin{pgfscope}%
\pgfsetbuttcap%
\pgfsetroundjoin%
\definecolor{currentfill}{rgb}{0.000000,0.000000,0.000000}%
\pgfsetfillcolor{currentfill}%
\pgfsetlinewidth{0.803000pt}%
\definecolor{currentstroke}{rgb}{0.000000,0.000000,0.000000}%
\pgfsetstrokecolor{currentstroke}%
\pgfsetdash{}{0pt}%
\pgfsys@defobject{currentmarker}{\pgfqpoint{-0.048611in}{0.000000in}}{\pgfqpoint{-0.000000in}{0.000000in}}{%
\pgfpathmoveto{\pgfqpoint{-0.000000in}{0.000000in}}%
\pgfpathlineto{\pgfqpoint{-0.048611in}{0.000000in}}%
\pgfusepath{stroke,fill}%
}%
\begin{pgfscope}%
\pgfsys@transformshift{0.680000in}{2.453400in}%
\pgfsys@useobject{currentmarker}{}%
\end{pgfscope}%
\end{pgfscope}%
\begin{pgfscope}%
\definecolor{textcolor}{rgb}{0.000000,0.000000,0.000000}%
\pgfsetstrokecolor{textcolor}%
\pgfsetfillcolor{textcolor}%
\pgftext[x=0.513333in, y=2.405175in, left, base]{\color{textcolor}\sffamily\fontsize{10.000000}{12.000000}\selectfont 8}%
\end{pgfscope}%
\begin{pgfscope}%
\pgfsetbuttcap%
\pgfsetroundjoin%
\definecolor{currentfill}{rgb}{0.000000,0.000000,0.000000}%
\pgfsetfillcolor{currentfill}%
\pgfsetlinewidth{0.803000pt}%
\definecolor{currentstroke}{rgb}{0.000000,0.000000,0.000000}%
\pgfsetstrokecolor{currentstroke}%
\pgfsetdash{}{0pt}%
\pgfsys@defobject{currentmarker}{\pgfqpoint{-0.048611in}{0.000000in}}{\pgfqpoint{-0.000000in}{0.000000in}}{%
\pgfpathmoveto{\pgfqpoint{-0.000000in}{0.000000in}}%
\pgfpathlineto{\pgfqpoint{-0.048611in}{0.000000in}}%
\pgfusepath{stroke,fill}%
}%
\begin{pgfscope}%
\pgfsys@transformshift{0.680000in}{2.940600in}%
\pgfsys@useobject{currentmarker}{}%
\end{pgfscope}%
\end{pgfscope}%
\begin{pgfscope}%
\definecolor{textcolor}{rgb}{0.000000,0.000000,0.000000}%
\pgfsetstrokecolor{textcolor}%
\pgfsetfillcolor{textcolor}%
\pgftext[x=0.443888in, y=2.892375in, left, base]{\color{textcolor}\sffamily\fontsize{10.000000}{12.000000}\selectfont 10}%
\end{pgfscope}%
\begin{pgfscope}%
\pgfpathrectangle{\pgfqpoint{0.680000in}{0.382800in}}{\pgfqpoint{4.216000in}{2.679600in}}%
\pgfusepath{clip}%
\pgfsetrectcap%
\pgfsetroundjoin%
\pgfsetlinewidth{1.505625pt}%
\definecolor{currentstroke}{rgb}{0.121569,0.466667,0.705882}%
\pgfsetstrokecolor{currentstroke}%
\pgfsetdash{}{0pt}%
\pgfpathmoveto{\pgfqpoint{0.871636in}{0.504600in}}%
\pgfpathlineto{\pgfqpoint{0.999394in}{0.504600in}}%
\pgfpathlineto{\pgfqpoint{0.999394in}{0.504600in}}%
\pgfpathlineto{\pgfqpoint{1.127152in}{0.504600in}}%
\pgfpathlineto{\pgfqpoint{1.127152in}{0.748200in}}%
\pgfpathlineto{\pgfqpoint{1.254909in}{0.748200in}}%
\pgfpathlineto{\pgfqpoint{1.254909in}{0.991800in}}%
\pgfpathlineto{\pgfqpoint{1.382667in}{0.991800in}}%
\pgfpathlineto{\pgfqpoint{1.382667in}{0.991800in}}%
\pgfpathlineto{\pgfqpoint{1.510424in}{0.991800in}}%
\pgfpathlineto{\pgfqpoint{1.510424in}{1.235400in}}%
\pgfpathlineto{\pgfqpoint{1.638182in}{1.235400in}}%
\pgfpathlineto{\pgfqpoint{1.638182in}{1.235400in}}%
\pgfpathlineto{\pgfqpoint{1.765939in}{1.235400in}}%
\pgfpathlineto{\pgfqpoint{1.765939in}{1.479000in}}%
\pgfpathlineto{\pgfqpoint{1.893697in}{1.479000in}}%
\pgfpathlineto{\pgfqpoint{1.893697in}{1.479000in}}%
\pgfpathlineto{\pgfqpoint{2.021455in}{1.479000in}}%
\pgfpathlineto{\pgfqpoint{2.021455in}{1.479000in}}%
\pgfpathlineto{\pgfqpoint{2.149212in}{1.479000in}}%
\pgfpathlineto{\pgfqpoint{2.149212in}{1.479000in}}%
\pgfpathlineto{\pgfqpoint{2.276970in}{1.479000in}}%
\pgfpathlineto{\pgfqpoint{2.276970in}{1.722600in}}%
\pgfpathlineto{\pgfqpoint{2.404727in}{1.722600in}}%
\pgfpathlineto{\pgfqpoint{2.404727in}{1.722600in}}%
\pgfpathlineto{\pgfqpoint{2.532485in}{1.722600in}}%
\pgfpathlineto{\pgfqpoint{2.532485in}{1.966200in}}%
\pgfpathlineto{\pgfqpoint{2.660242in}{1.966200in}}%
\pgfpathlineto{\pgfqpoint{2.660242in}{1.966200in}}%
\pgfpathlineto{\pgfqpoint{2.788000in}{1.966200in}}%
\pgfpathlineto{\pgfqpoint{2.788000in}{1.966200in}}%
\pgfpathlineto{\pgfqpoint{2.915758in}{1.966200in}}%
\pgfpathlineto{\pgfqpoint{2.915758in}{1.966200in}}%
\pgfpathlineto{\pgfqpoint{3.043515in}{1.966200in}}%
\pgfpathlineto{\pgfqpoint{3.043515in}{2.209800in}}%
\pgfpathlineto{\pgfqpoint{3.171273in}{2.209800in}}%
\pgfpathlineto{\pgfqpoint{3.171273in}{2.209800in}}%
\pgfpathlineto{\pgfqpoint{3.299030in}{2.209800in}}%
\pgfpathlineto{\pgfqpoint{3.299030in}{2.453400in}}%
\pgfpathlineto{\pgfqpoint{3.426788in}{2.453400in}}%
\pgfpathlineto{\pgfqpoint{3.426788in}{2.453400in}}%
\pgfpathlineto{\pgfqpoint{3.554545in}{2.453400in}}%
\pgfpathlineto{\pgfqpoint{3.554545in}{2.453400in}}%
\pgfpathlineto{\pgfqpoint{3.682303in}{2.453400in}}%
\pgfpathlineto{\pgfqpoint{3.682303in}{2.453400in}}%
\pgfpathlineto{\pgfqpoint{3.810061in}{2.453400in}}%
\pgfpathlineto{\pgfqpoint{3.810061in}{2.697000in}}%
\pgfpathlineto{\pgfqpoint{3.937818in}{2.697000in}}%
\pgfpathlineto{\pgfqpoint{3.937818in}{2.697000in}}%
\pgfpathlineto{\pgfqpoint{4.065576in}{2.697000in}}%
\pgfpathlineto{\pgfqpoint{4.065576in}{2.697000in}}%
\pgfpathlineto{\pgfqpoint{4.193333in}{2.697000in}}%
\pgfpathlineto{\pgfqpoint{4.193333in}{2.697000in}}%
\pgfpathlineto{\pgfqpoint{4.321091in}{2.697000in}}%
\pgfpathlineto{\pgfqpoint{4.321091in}{2.697000in}}%
\pgfpathlineto{\pgfqpoint{4.448848in}{2.697000in}}%
\pgfpathlineto{\pgfqpoint{4.448848in}{2.697000in}}%
\pgfpathlineto{\pgfqpoint{4.576606in}{2.697000in}}%
\pgfpathlineto{\pgfqpoint{4.576606in}{2.940600in}}%
\pgfpathlineto{\pgfqpoint{4.704364in}{2.940600in}}%
\pgfusepath{stroke}%
\end{pgfscope}%
\begin{pgfscope}%
\pgfsetrectcap%
\pgfsetmiterjoin%
\pgfsetlinewidth{0.803000pt}%
\definecolor{currentstroke}{rgb}{0.000000,0.000000,0.000000}%
\pgfsetstrokecolor{currentstroke}%
\pgfsetdash{}{0pt}%
\pgfpathmoveto{\pgfqpoint{0.680000in}{0.382800in}}%
\pgfpathlineto{\pgfqpoint{0.680000in}{3.062400in}}%
\pgfusepath{stroke}%
\end{pgfscope}%
\begin{pgfscope}%
\pgfsetrectcap%
\pgfsetmiterjoin%
\pgfsetlinewidth{0.803000pt}%
\definecolor{currentstroke}{rgb}{0.000000,0.000000,0.000000}%
\pgfsetstrokecolor{currentstroke}%
\pgfsetdash{}{0pt}%
\pgfpathmoveto{\pgfqpoint{4.896000in}{0.382800in}}%
\pgfpathlineto{\pgfqpoint{4.896000in}{3.062400in}}%
\pgfusepath{stroke}%
\end{pgfscope}%
\begin{pgfscope}%
\pgfsetrectcap%
\pgfsetmiterjoin%
\pgfsetlinewidth{0.803000pt}%
\definecolor{currentstroke}{rgb}{0.000000,0.000000,0.000000}%
\pgfsetstrokecolor{currentstroke}%
\pgfsetdash{}{0pt}%
\pgfpathmoveto{\pgfqpoint{0.680000in}{0.382800in}}%
\pgfpathlineto{\pgfqpoint{4.896000in}{0.382800in}}%
\pgfusepath{stroke}%
\end{pgfscope}%
\begin{pgfscope}%
\pgfsetrectcap%
\pgfsetmiterjoin%
\pgfsetlinewidth{0.803000pt}%
\definecolor{currentstroke}{rgb}{0.000000,0.000000,0.000000}%
\pgfsetstrokecolor{currentstroke}%
\pgfsetdash{}{0pt}%
\pgfpathmoveto{\pgfqpoint{0.680000in}{3.062400in}}%
\pgfpathlineto{\pgfqpoint{4.896000in}{3.062400in}}%
\pgfusepath{stroke}%
\end{pgfscope}%
\end{pgfpicture}%
\makeatother%
\endgroup%

    \caption{Die Primzahlfunktion von $0$ bis $30$.}
    \label{fig:zeta:primzahlfunktion}
\end{figure}

Der grundlegende Zusammenhang der Primzahlen und der Zetafunktion wird im ersten Abschnitt \ref{zeta:section:eulerprodukt} über das Eulerprodukt gezeigt.
Danach folgt die Verbindung zur bereits bekannten Gammafunktion in Abschnitt \ref{zeta:section:zusammenhang_mit_gammafunktion}.
Schlussendlich folgt die Beschreibung der analytischen Fortsetzung die gesamte komplexe Ebene in Abschnitt \ref{zeta:section:analytische_fortsetzung}.

Diese analytische Fortsetzung wird für die Riemannsche Vermutung benötigt, ermöglicht aber auch andere interessante Aussagen.
So findet sich zum Beispiel immer wieder die aberwitzige Behauptung, das die Summe aller natürlichen Zahlen
\begin{equation*}
    \sum{n=1}^{\infty} n
    =
    \sum_{n=1}^{\infty}
    \frac{1}{n^{-1}}
    =
    -\frac{1}{12}
\end{equation*}
sei.
Obwohl diese Behauptung offensichtlich Falsch ist, hat sie doch ihre Berechtigung, wie durch die analytische Fortsetzung gezeigt werden wird.

Die folgenden mathematischen Herleitungen sind, sofern nicht anders gekennzeichnet, eigene Darstellungen basierend auf den überaus umfangreichen Wikipedia-Artikeln auf Deutsch \cite{zeta:online:wiki_de} und Englisch \cite{zeta:online:wiki_en} sowie einer Video-Playlist \cite{zeta:online:mryoumath}.
