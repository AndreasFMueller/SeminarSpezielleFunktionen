\section{Analytische Fortsetzung} \label{zeta:section:analytische_fortsetzung}
\rhead{Analytische Fortsetzung}

%TODO missing Text

\subsection{Fortsetzung auf $\Re(s) > 0$} \label{zeta:subsection:auf_bereich_ge_0}
Zuerst definieren die Dirichletsche Etafunktion als
\begin{equation}\label{zeta:equation:eta}
    \eta(s)
    =
    \sum_{n=1}^{\infty}
    \frac{(-1)^{n-1}}{n^s},
\end{equation}
wobei die Reihe bis auf die alternierenden Vorzeichen die selbe wie in der Zetafunktion ist.
Diese Etafunktion konvergiert gemäss dem Leibnitz-Kriterium im Bereich $\Re(s) > 0$, da dann die einzelnen Glieder monoton fallend sind.

Wenn wir es nun schaffen, die sehr ähnliche Zetafunktion mit der Etafunktion auszudrücken, dann haben die gesuchte Fortsetzung.
Die folgenden Schritte zeigen, wie man dazu kommt:
\begin{align}
    \zeta(s)
    &=
    \sum_{n=1}^{\infty}
    \frac{1}{n^s} \label{zeta:align1}
    \\
    \frac{1}{2^{s-1}}
    \zeta(s)
    &=
    \sum_{n=1}^{\infty}
    \frac{2}{(2n)^s} \label{zeta:align2}
    \\
    \left(1 - \frac{1}{2^{s-1}} \right)
    \zeta(s)
    &=
    \frac{1}{1^s}
    \underbrace{-\frac{2}{2^s} + \frac{1}{2^s}}_{-\frac{1}{2^s}}
    + \frac{1}{3^s}
    \underbrace{-\frac{2}{4^s} + \frac{1}{4^s}}_{-\frac{1}{4^s}}
    \ldots
    && \text{\eqref{zeta:align1}} - \text{\eqref{zeta:align2}}
    \\
    &= \eta(s)
    \\
    \zeta(s)
    &=
    \left(1 - \frac{1}{2^{s-1}} \right)^{-1} \eta(s).
\end{align}

\subsection{Fortsetzung auf ganz $\mathbb{C}$} \label{zeta:subsection:auf_ganz}
Für die Fortsetzung auf den Rest von $\mathbb{C}$, verwenden wir den Zusammenhang von Gamma- und Zetafunktion aus \ref{zeta:section:zusammenhang_mit_gammafunktion}.
Wir beginnen damit, die Gammafunktion für den halben Funktionswert zu berechnen als
\begin{equation}
    \Gamma \left( \frac{s}{2} \right)
    =
    \int_0^{\infty} t^{\frac{s}{2}-1} e^{-t} dt.
\end{equation}
Nun substituieren wir $t$ mit $t = \pi n^2 x$ und $dt=\pi n^2 dx$ und erhalten
\begin{align}
    \Gamma \left( \frac{s}{2} \right)
    &=
    \int_0^{\infty}
    (\pi n^2)^{\frac{s}{2}}
    x^{\frac{s}{2}-1}
    e^{-\pi n^2 x}
    dx
    && \text{Division durch } (\pi n^2)^{\frac{s}{2}}
    \\
    \frac{\Gamma \left( \frac{s}{2} \right)}{\pi^{\frac{s}{2}} n^s}
    &=
    \int_0^{\infty}
    x^{\frac{s}{2}-1}
    e^{-\pi n^2 x}
    dx
    && \text{Zeta durch Summenbildung } \sum_{n=1}^{\infty}
    \\
    \frac{\Gamma \left( \frac{s}{2} \right)}{\pi^{\frac{s}{2}}}
    \zeta(s)
    &=
    \int_0^{\infty}
    x^{\frac{s}{2}-1}
    \sum_{n=1}^{\infty}
    e^{-\pi n^2 x}
    dx. \label{zeta:equation:integral1}
\end{align}
Die Summe kürzen wir ab als $\psi(x) = \sum_{n=1}^{\infty} e^{-\pi n^2 x}$.
%TODO Wieso folgendes -> aus Fourier Signal
Es gilt
\begin{equation}\label{zeta:equation:psi}
    \psi(x)
    =
    - \frac{1}{2}
    + \frac{\psi\left(\frac{1}{x} \right)}{\sqrt{x}}
    + \frac{1}{2 \sqrt{x}}.
\end{equation}

Zunächst teilen wir nun das Integral aus \eqref{zeta:equation:integral1} auf als
\begin{equation}\label{zeta:equation:integral2}
    \int_0^{\infty}
    x^{\frac{s}{2}-1}
    \psi(x)
    dx
    =
    \int_0^{1}
    x^{\frac{s}{2}-1}
    \psi(x)
    dx
    +
    \int_1^{\infty}
    x^{\frac{s}{2}-1}
    \psi(x)
    dx,
\end{equation}
wobei wir uns nun auf den ersten Teil konzentrieren werden.
Dabei setzen wir das Wissen aus \eqref{zeta:equation:integral2} ein und erhalten
\begin{align}
    \int_0^{1}
    x^{\frac{s}{2}-1}
    \psi(x)
    dx
    &=
    \int_0^{1}
    x^{\frac{s}{2}-1}
    \left(
    - \frac{1}{2}
    + \frac{\psi\left(\frac{1}{x} \right)}{\sqrt{x}}
    + \frac{1}{2 \sqrt{x}}.
    \right)
    dx
    \\
    &=
    \int_0^{1}
    x^{\frac{s}{2}-\frac{3}{2}}
    \psi \left( \frac{1}{x} \right)
    + \frac{1}{2}
    \left(
    x^{\frac{s}{2}-\frac{3}{2}}
    -
    x^{\frac{s}{2}-1}
    \right)
    dx
    \\
    &=
    \int_0^{1}
    x^{\frac{s}{2}-\frac{3}{2}}
    \psi \left( \frac{1}{x} \right)
    dx
    + \frac{1}{2}
    \int_0^1
    x^{\frac{s}{2}-\frac{3}{2}}
    -
    x^{\frac{s}{2}-1}
    dx.
\end{align}
Dabei kann das zweite integral gelöst werden als
\begin{equation}
    \frac{1}{2}
    \int_0^1
    x^{\frac{s}{2}-\frac{3}{2}}
    -
    x^{\frac{s}{2}-1}
    dx
    =
    \frac{1}{s(s-1)}.
\end{equation}


