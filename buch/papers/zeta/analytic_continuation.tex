\section{Analytische Fortsetzung} \label{zeta:section:analytische_fortsetzung}
\rhead{Analytische Fortsetzung}

Die analytische Fortsetzung der Riemannschen Zetafunktion ist äusserst interessant.
Sie ermöglicht die Berechnung von $\zeta(-1)$ und weiterer spannender Werte.
So liegen zum Beispiel unendlich viele Nullstellen der Zetafunktion bei $\Re(s) = \frac{1}{2}$.
Wie bereits erwähnt sind diese Gegenstand der Riemannschen Vermutung.

Es werden zwei verschiedene Fortsetzungen benötigt.
Die erste erweitert die Zetafunktion auf $\Re(s) > 0$.
Die zweite verwendet eine Spiegelung an der $\Re(s) = \frac{1}{2}$ Geraden und erschliesst damit die ganze komplexe Ebene.
Eine grafische Darstellung dieses Plans ist in Abbildung \ref{zeta:fig:continuation_overview} zu sehen.
\begin{figure}
    \centering
    \begin{tikzpicture}[>=stealth', auto, node distance=0.9cm, scale=2,
    dot/.style={fill, circle, inner sep=0, minimum size=0.1cm}]

    \draw[->] (-2,0) -- (-1,0) node[dot]{} node[anchor=north]{$-1$} -- (0,0) node[anchor=north west]{$0$} -- (0.5,0) node[anchor=north west]{$0.5$}-- (1,0) node[anchor=north west]{$1$} -- (2,0) node[anchor=west]{$\Re(s)$};

    \draw[->] (0,-1.2) -- (0,1.2) node[anchor=south]{$\Im(s)$};
    \begin{scope}[yscale=0.1]
        \draw[] (1,-1) -- (1,1);
    \end{scope}
    \draw[dotted] (0.5,-1) -- (0.5,1);

    \begin{scope}[]
        \fill[opacity=0.2, red] (-1.8,1) rectangle (0, -1);
        \fill[opacity=0.2, blue] (0,1) rectangle (1, -1);
        \fill[opacity=0.2, green] (1,1) rectangle (1.8, -1);
    \end{scope}

\end{tikzpicture}

    \caption{
        Die verschiedenen Abschnitte der Riemannschen Zetafunktion.
        Die originale Definition von \eqref{zeta:equation1} ist im grünen Bereich gültig.
        Für den blauen Bereich gilt \eqref{zeta:equation:fortsetzung1}.
        Um den roten Bereich zu bekommen verwendet die Funktionalgleichung \eqref{zeta:equation:functional} eine Spiegelung an $\Re(s) = 0.5$.
    }
    \label{zeta:fig:continuation_overview}
\end{figure}

\subsection{Fortsetzung auf $\Re(s) > 0$} \label{zeta:subsection:auf_bereich_ge_0}
Zuerst definieren wir die Dirichletsche Etafunktion als
\begin{equation}\label{zeta:equation:eta}
    \eta(s)
    =
    \sum_{n=1}^{\infty}
    \frac{(-1)^{n-1}}{n^s},
\end{equation}
wobei die Reihe bis auf die alternierenden Vorzeichen die selbe wie in der Zetafunktion ist.
Diese Etafunktion konvergiert gemäss dem Leibnitz-Kriterium im Bereich $\Re(s) > 0$, da dann die einzelnen Glieder monoton fallend sind.

Wenn wir es nun schaffen, die sehr ähnliche Zetafunktion durch die Etafunktion auszudrücken, dann haben die gesuchte Fortsetzung.
Zuerst wiederholen wir zweimal die Definition der Zetafunktion \eqref{zeta:equation1}, wobei wir sie einmal durch $2^{s-1}$ teilen
\begin{align}
    \color{red}
        \zeta(s)
    &=
    \sum_{n=1}^{\infty}
    \color{red}
        \frac{1}{n^s} \label{zeta:align1}
    \\
    \color{blue}
        \frac{1}{2^{s-1}}
        \zeta(s)
    &=
    \sum_{n=1}^{\infty}
    \color{blue}
        \frac{2}{(2n)^s}. \label{zeta:align2}
\end{align}
Durch Subtraktion der beiden Gleichungen \eqref{zeta:align1} minus \eqref{zeta:align2}, ergibt sich
\begin{align}
    \left({\color{red}1} - {\color{blue}\frac{1}{2^{s-1}}} \right)
    \zeta(s)
    &=
    {\color{red}\frac{1}{1^s}}
    \underbrace{
    -
    {\color{blue}\frac{2}{2^s}}
    +
    {\color{red}\frac{1}{2^s}}
    }_{\displaystyle{-\frac{1}{2^s}}}
    +
    {\color{red}\frac{1}{3^s}}
    \underbrace{-
    {\color{blue}\frac{2}{4^s}}
    +
    {\color{red}\frac{1}{4^s}}
    }_{\displaystyle{-\frac{1}{4^s}}}
    \ldots
    \\
    &= \eta(s).
\end{align}
Dies ist die Fortsetzung auf den noch unbekannten Bereich $0 < \Re(s) < 1$
\begin{equation} \label{zeta:equation:fortsetzung1}
    \zeta(s)
    :=
    \left(1 - \frac{1}{2^{s-1}} \right)^{-1} \eta(s).
\end{equation}

\subsection{Fortsetzung auf ganz $\mathbb{C}$} \label{zeta:subsection:auf_ganz}
Für die Fortsetzung auf den Rest von $\mathbb{C}$, verwenden wir den Zusammenhang von Gamma- und Zetafunktion aus \ref{zeta:section:zusammenhang_mit_gammafunktion}.
Wir beginnen damit, die Gammafunktion für den halben Funktionswert zu berechnen als
\begin{equation}
    \Gamma \left( \frac{s}{2} \right)
    =
    \int_0^{\infty} t^{\frac{s}{2}-1} e^{-t} dt.
\end{equation}
Nun substituieren wir $t = \pi n^2 x$ und $dt=\pi n^2 dx$ und erhalten
\begin{equation}
    \Gamma \left( \frac{s}{2} \right)
    =
    \int_0^{\infty}
    (\pi n^2)^{\frac{s}{2}}
    x^{\frac{s}{2}-1}
    e^{-\pi n^2 x}
    \,dx.
\end{equation}
Analog zum Abschnitt \ref{zeta:section:zusammenhang_mit_gammafunktion} teilen wir durch $(\pi n^2)^{\frac{s}{2}}$
\begin{equation}
    \frac{\Gamma \left( \frac{s}{2} \right)}{\pi^{\frac{s}{2}}}
    \frac{1}{n^s}
     =
    \int_0^{\infty}
    x^{\frac{s}{2}-1}
    e^{-\pi n^2 x}
    \,dx,
\end{equation}
und finden $\zeta(s)$ durch die Summenbildung über alle $n$
\begin{align}
    \frac{\Gamma \left( \frac{s}{2} \right)}{\pi^{\frac{s}{2}}}
    \zeta(s)
    &=
    \int_0^{\infty}
    x^{\frac{s}{2}-1}
    \sum_{n=1}^{\infty}
    e^{-\pi n^2 x}
    \,dx\label{zeta:equation:integral1}
    \\
    &=
    \int_0^{\infty}
    x^{\frac{s}{2}-1}
    \psi(x)
    \,dx,
\end{align}
wobei die Summe $\sum_{n=1}^{\infty} e^{-\pi n^2 x}$ als $\psi(x)$ abgekürzt wird.
Zunächst teilen wir nun das Integral auf in zwei Teile
\begin{equation}\label{zeta:equation:integral2}
    \int_0^{\infty}
    x^{\frac{s}{2}-1}
    \psi(x)
    \,dx
    =
    \underbrace{
    \int_0^{1}
    x^{\frac{s}{2}-1}
    \psi(x)
    \,dx
    }_{\displaystyle{I_1}}
    +
    \underbrace{
    \int_1^{\infty}
    x^{\frac{s}{2}-1}
    \psi(x)
    \,dx
    }_{\displaystyle{I_2}}
    =
    I_1 + I_2.
\end{equation}
Abschnitt \ref{zeta:subsubsec:intcal} beschreibt wie das Integral $I_1$ umgestellt werden kann um ebenfalls die Integrationsgrenzen $1$ und $\infty$ zu bekommen.
Die Lösung, beschrieben in Gleichung \eqref{zeta:equation:intcal_res}, lautet
\begin{equation*}
    I_1
    =
    \int_0^{1}
    x^{\frac{s}{2}-1}
    \psi(x)
    \,dx
    =
    \int_{1}^{\infty}
    x^{(-1) \left(\frac{s}{2}+\frac{1}{2}\right)}
    \psi(x)
    \,dx
    +
    \frac{1}{s(s-1)}.
\end{equation*}
Dieses Resultat setzen wir nun ein in \eqref{zeta:equation:integral2}, um schlussendlich
\begin{align}
    \frac{\Gamma \left( \frac{s}{2} \right)}{\pi^{\frac{s}{2}}}
    \zeta(s)
    &=
    \int_0^{1}
    x^{\frac{s}{2}-1}
    \psi(x)
    \,dx
    +
    \int_1^{\infty}
    x^{\frac{s}{2}-1}
    \psi(x)
    \,dx
    \nonumber
    \\
    &=
    \frac{1}{s(s-1)}
    +
    \int_{1}^{\infty}
    x^{(-1) \left(\frac{s}{2}+\frac{1}{2}\right)}
    \psi(x)
    \,dx
    +
    \int_1^{\infty}
    x^{\frac{s}{2}-1}
    \psi(x)
    \,dx
    \\
    &=
    \frac{1}{s(s-1)}
    +
    \int_{1}^{\infty}
    \left(
    x^{-\frac{s}{2}-\frac{1}{2}}
    +
    x^{\frac{s}{2}-1}
    \right)
    \psi(x)
    \,dx
    \\
    &=
    \frac{-1}{s(1-s)}
    +
    \int_{1}^{\infty}
    \left(
    x^{\frac{1-s}{2}}
    +
    x^{\frac{s}{2}}
    \right)
    \frac{\psi(x)}{x}
    \,dx,
\end{align}
zu erhalten.
Wenn wir dieses Resultat genau anschauen, erkennen wir dass sich nichts verändert wenn $s$ mit $1-s$ ersetzt wird.
Somit haben wir die analytische Fortsetzung gefunden als
\begin{equation}\label{zeta:equation:functional}
    \frac{\Gamma \left( \frac{s}{2} \right)}{\pi^{\frac{s}{2}}}
    \zeta(s)
    =
    \frac{\Gamma \left( \frac{1-s}{2} \right)}{\pi^{\frac{1-s}{2}}}
    \zeta(1-s),
\end{equation}
was einer Spiegelung an der $\Re(s) = \frac{1}{2}$ Geraden entspricht.
Eine ganz ähnliche Spiegelungseigenschaft wurde bereits in Abschnitt \ref{buch:funktionentheorie:subsection:gammareflektion} für die Gammafunktion gefunden.

\subsection{Berechnung des Integrals $I_1 = \int_0^{1} x^{\frac{s}{2}-1} \psi(x) \,dx$} \label{zeta:subsubsec:intcal}

Ziel dieses Abschnittes ist, zu zeigen wie das Integral $I_1$ aus Gleichung \eqref{zeta:equation:integral2} durch ein neues Integral mit den Integrationsgrenzen $1$ und $\infty$ ersetzt werden kann.
Da dieser Schritt ziemlich aufwendig ist, wird er hier in einem eigenen Abschnitt behandelt.
Zunächst wird die poissonsche Summenformel hergeleitet \cite{zeta:online:poisson}, da diese verwendet werden kann um $\psi(x)$ zu berechnen.

Um die poissonsche Summenformel zu beweisen, berechnen wir zunächst die Fourierreihe der Dirac Delta Funktion.

\begin{lemma}
    Die Fourierreihe der periodischen Dirac $\delta$ Funktion $\sum \delta(x - 2\pi k)$ ist
    \begin{equation} \label{zeta:equation:fourier_dirac}
        \sum_{k=-\infty}^{\infty}
        \delta(x - 2\pi k)
        =
        \frac{1}{2\pi}
        \sum_{n=-\infty}^{\infty}
        e^{i n x}.
    \end{equation}
\end{lemma}

\begin{proof}[Beweis]
    Eine Fourierreihe einer beliebigen periodischen Funktion $f(x)$ berechnet sich als
    \begin{align}
        f(x)
        &=
        \sum_{n=-\infty}^{\infty}
        c_n
        e^{i n x} \\
        c_n
        &=
        \frac{1}{2\pi}
        \int_{-\pi}^{\pi}
        f(x)
        e^{-i n x}
        \, dx.
    \end{align}
    Wenn $f(x)=\delta(x)$ eingesetz wird ergeben sich konstante Koeffizienten
    \begin{equation}
        c_n
        =
        \frac{1}{2\pi}
        \int_{-\pi}^{\pi}
        \delta(x)
        e^{-i n x}
        \, dx
        =
        \frac{1}{2\pi},
    \end{equation}
    womit die sehr einfache Fourierreihe der Dirac Delta Funktion berechnet wäre.
\end{proof}

\begin{satz}[Poissonsche Summernformel]
    Die Summe einer Funktion $f(n)$ über alle ganzen Zahlen $n$ ist äquivalent zur Summe ihrer Fouriertransformation $F(k)$ über alle ganzen Zahlen $k$
    \begin{equation}
        \sum_{n=-\infty}^{\infty}
        f(n)
        =
        \sum_{k=-\infty}^{\infty}
        F(k).
    \end{equation}
\end{satz}

\begin{proof}[Beweis]
    Wir schreiben die Summe über die Fouriertransformation aus
    \begin{align}
        \sum_{k=-\infty}^{\infty}
        F(k)
        &=
        \sum_{k=-\infty}^{\infty}
        \int_{-\infty}^{\infty}
        f(x)
        e^{-i 2\pi x k}
        \, dx
        \\
        &=
        \int_{-\infty}^{\infty}
        f(x)
        \underbrace{
        \sum_{k=-\infty}^{\infty}
        e^{-i 2\pi x k}
        }_{\displaystyle{\text{\eqref{zeta:equation:fourier_dirac}}}}
        \, dx, \label{zeta:equation:1934}
    \end{align}
    und verwenden die Fouriertransformation der Dirac Funktion aus \eqref{zeta:equation:fourier_dirac}
    \begin{align}
        \sum_{k=-\infty}^{\infty}
        e^{-i 2\pi x k}
        &=
        2 \pi
        \sum_{k=-\infty}^{\infty}
        \delta(-2\pi x - 2\pi k)
        \\
        &=
        \frac{2 \pi}{2 \pi}
        \sum_{k=-\infty}^{\infty}
        \delta(x + k).
    \end{align}
    Wenn wir dies einsetzen in Gleichung \eqref{zeta:equation:1934} erhalten wir
    \begin{equation}
        \sum_{k=-\infty}^{\infty}
        F(k)
        =
        \int_{-\infty}^{\infty}
        f(x)
        \sum_{k=-\infty}^{\infty}
        \delta(x + k)
        \, dx
        =
        \sum_{k=-\infty}^{\infty}
        \int_{-\infty}^{\infty}
        f(x)
        \delta(x + k)
        \, dx
        =
        \sum_{k=-\infty}^{\infty}
        f(k),
    \end{equation}
    was der gesuchte Beweis für die poissonsche Summenformel ist.
\end{proof}

Erinnern wir uns nochmals an unser Integral aus Gleichung \eqref{zeta:equation:integral2}
\begin{align*}
    I_1
    &=
    \int_0^{1}
    x^{\frac{s}{2}-1}
    \sum_{n=1}^{\infty}
    e^{-\pi n^2 x}
    \,dx
    \\
    &=
    \int_0^{1}
    x^{\frac{s}{2}-1}
    \psi(x)
    \,dx
    .
\end{align*}

Wir wenden nun diese poissonsche Summenformel $\sum f(n) = \sum F(n)$ an auf $\psi(x)$.
In unserem Problem ist also $f(n) =  e^{-\pi n^2 x}$ und die zugehörige Fouriertransformation $F(n)$ ist
\begin{equation}
    F(n)
    =
    \mathcal{F}
    (
    e^{-\pi n^2 x}
    )
    =
    \frac{1}{\sqrt{x}}
    e^{\frac{-n^2 \pi}{x}}.
\end{equation}
Dadurch ergibt sich
\begin{equation}\label{zeta:equation:psi}
    \sum_{n=-\infty}^{\infty}
    e^{-\pi n^2 x}
    =
    \frac{1}{\sqrt{x}}
    \sum_{n=-\infty}^{\infty}
    e^{\frac{-n^2 \pi}{x}},
\end{equation}
wobei wir die Summen so verändern müssen, dass sie bei $n=1$ beginnen und wir $\psi(x)$ erhalten als
\begin{align}
    2
    \sum_{n=1}^{\infty}
    e^{-\pi n^2 x}
    +
    1
    &=
    \frac{1}{\sqrt{x}}
    \Biggl(
    2
    \sum_{n=1}^{\infty}
    e^{\frac{-n^2 \pi}{x}}
    +
    1
    \Biggr)
    \\
    2
    \psi(x)
    +
    1
    &=
    \frac{1}{\sqrt{x}}
    \left(
    2
    \psi\left(\frac{1}{x}\right)
    +
    1
    \right)
    \\
    \psi(x)
    &=
    - \frac{1}{2}
    + \frac{\psi\left(\frac{1}{x} \right)}{\sqrt{x}}
    + \frac{1}{2 \sqrt{x}}.\label{zeta:equation:psi}
\end{align}
Diese Form von $\psi(x)$ eingesetzt in $I_1$ ergibt
\begin{align}
    I_1
    =
    \int_0^{1}
    x^{\frac{s}{2}-1}
    \psi(x)
    \,dx
    &=
    \int_0^{1}
    x^{\frac{s}{2}-1}
    \Biggl(
    - \frac{1}{2}
    + \frac{\psi\left(\frac{1}{x} \right)}{\sqrt{x}}
    + \frac{1}{2 \sqrt{x}}
    \Biggr)
    \,dx
    \\
    &=
    \int_0^{1}
    x^{\frac{s}{2}-\frac{3}{2}}
    \psi \left( \frac{1}{x} \right)
    + \frac{1}{2}
    \biggl(
    x^{\frac{s}{2}-\frac{3}{2}}
    -
    x^{\frac{s}{2}-1}
    \biggl)
    \,dx
    \\
    &=
    \underbrace{
    \int_0^{1}
    x^{\frac{s}{2}-\frac{3}{2}}
    \psi \left( \frac{1}{x} \right)
    \,dx
    }_{\displaystyle{I_3}}
    +
    \underbrace{
    \frac{1}{2}
    \int_0^1
    x^{\frac{s}{2}-\frac{3}{2}}
    -
    x^{\frac{s}{2}-1}
    \,dx
    }_{\displaystyle{I_4}}. \label{zeta:equation:integral3}
\end{align}
Darin kann für das zweite Integral $I_4$ eine Lösung gefunden werden als
\begin{equation}
    I_4
    =
    \frac{1}{2}
    \int_0^1
    x^{\frac{s}{2}-\frac{3}{2}}
    -
    x^{\frac{s}{2}-1}
    \,dx
    =
    \frac{1}{s(s-1)}.
\end{equation}
Das erste Integral $I_3$ aus \eqref{zeta:equation:integral3} mit $\psi \left(\frac{1}{x} \right)$ ist hingegen nicht lösbar in dieser Form.
Deshalb substituieren wir $x = \frac{1}{u}$ und $dx = -\frac{1}{u^2}du$.
Die untere Integralgrenze wechselt ebenfalls zu $x_0 = 0 \rightarrow u_0 = \infty$.
Dies ergibt
\begin{align}
    I_3
    =
    \int_{\infty}^{1}
    \left(
    \frac{1}{u}
    \right)^{\frac{s}{2}-\frac{3}{2}}
    \psi(u)
    \frac{-du}{u^2}
    &=
    \int_{1}^{\infty}
    \left(
    \frac{1}{u}
    \right)^{\frac{s}{2}-\frac{3}{2}}
    \psi(u)
    \frac{du}{u^2}
    \\
    &=
    \int_{1}^{\infty}
    x^{(-1) \left(\frac{s}{2}+\frac{1}{2}\right)}
    \psi(x)
    \,dx,
\end{align}
wobei wir durch Multiplikation mit $(-1)$ die Integralgrenzen tauschen dürfen.
Es ist zu beachten das diese Grenzen nun identisch mit den Grenzen des zweiten Integrals $I_2$ von \eqref{zeta:equation:integral2} sind.
Wir setzen beide Lösungen in Gleichung \eqref{zeta:equation:integral3} ein und erhalten
\begin{equation}
    I_1
    =
    \int_0^{1}
    x^{\frac{s}{2}-1}
    \psi(x)
    \,dx
    =
    \int_{1}^{\infty}
    x^{(-1) \left(\frac{s}{2}+\frac{1}{2}\right)}
    \psi(x)
    \,dx
    +
    \frac{1}{s(s-1)}. \label{zeta:equation:intcal_res}
\end{equation}
Diese Form des Integrals $I_1$ hat die gewünschten Integrationsgrenzen und ein essentieller Bestandteil des Beweises der Funktionalgleichung in Abschnitt \ref{zeta:subsection:auf_ganz}.
