\section{Analytische Fortsetzung} \label{zeta:section:analytische_fortsetzung}
\rhead{Analytische Fortsetzung}

Die analytische Fortsetzung der Riemannschen Zetafunktion ist äusserst interessant.
Sie ermöglicht die Berechnung von $\zeta(-1)$ und weiterer spannender Werte.
So liegen zum Beispiel unendlich viele Nullstellen der Zetafunktion bei $\Re(s) = 0.5$.
Diese sind relevant für die Primzahlverteilung und sind Gegenstand der Riemannschen Vermutung.

Es werden zwei verschiedene Fortsetzungen benötigt.
Die erste erweitert die Zetafunktion auf $\Re(s) > 0$.
Die zweite verwendet eine Spiegelung an der $\Re(s) = 0.5$ Linie und erschliesst damit die ganze komplexe Ebene.
Eine grafische Darstellung dieses Plans ist in Abbildung \ref{zeta:fig:continuation_overview} zu sehen.
\begin{figure}
    \centering
    \begin{tikzpicture}[>=stealth', auto, node distance=0.9cm, scale=2,
    dot/.style={fill, circle, inner sep=0, minimum size=0.1cm}]

    \draw[->] (-2,0) -- (-1,0) node[dot]{} node[anchor=north]{$-1$} -- (0,0) node[anchor=north west]{$0$} -- (0.5,0) node[anchor=north west]{$0.5$}-- (1,0) node[anchor=north west]{$1$} -- (2,0) node[anchor=west]{$\Re(s)$};

    \draw[->] (0,-1.2) -- (0,1.2) node[anchor=south]{$\Im(s)$};
    \begin{scope}[yscale=0.1]
        \draw[] (1,-1) -- (1,1);
    \end{scope}
    \draw[dotted] (0.5,-1) -- (0.5,1);

    \begin{scope}[]
        \fill[opacity=0.2, red] (-1.8,1) rectangle (0, -1);
        \fill[opacity=0.2, blue] (0,1) rectangle (1, -1);
        \fill[opacity=0.2, green] (1,1) rectangle (1.8, -1);
    \end{scope}

\end{tikzpicture}

    \caption{
        Die verschiedenen Abschnitte der Riemannschen Zetafunktion. 
        Die originale Definition von \eqref{zeta:equation1} ist im grünen Bereich gültig.
        Für den blauen Bereich gilt \eqref{zeta:equation:fortsetzung1}.
        Um den roten Bereich zu bekommen verwendet die Funktionalgleichung \eqref{zeta:equation:functional} eine Spiegelung an $\Re(s) = 0.5$.
    }
    \label{zeta:fig:continuation_overview}
\end{figure}

\subsection{Fortsetzung auf $\Re(s) > 0$} \label{zeta:subsection:auf_bereich_ge_0}
Zuerst definieren die Dirichletsche Etafunktion als
\begin{equation}\label{zeta:equation:eta}
    \eta(s)
    =
    \sum_{n=1}^{\infty}
    \frac{(-1)^{n-1}}{n^s},
\end{equation}
wobei die Reihe bis auf die alternierenden Vorzeichen die selbe wie in der Zetafunktion ist.
Diese Etafunktion konvergiert gemäss dem Leibnitz-Kriterium im Bereich $\Re(s) > 0$, da dann die einzelnen Glieder monoton fallend sind.

Wenn wir es nun schaffen, die sehr ähnliche Zetafunktion durch die Etafunktion auszudrücken, dann haben die gesuchte Fortsetzung.
Zuerst wiederholen wir zweimal die Definition der Zetafunktion \eqref{zeta:equation1}, wobei wir sie einmal durch $2^{s-1}$ teilen
\begin{align}
    \zeta(s)
    &=
    \sum_{n=1}^{\infty}
    \frac{1}{n^s} \label{zeta:align1}
    \\
    \frac{1}{2^{s-1}}
    \zeta(s)
    &=
    \sum_{n=1}^{\infty}
    \frac{2}{(2n)^s}. \label{zeta:align2}
\end{align}
Durch Subtraktion der beiden Gleichungen \eqref{zeta:align1} minus \eqref{zeta:align2}, ergibt sich
\begin{align}
    \left(1 - \frac{1}{2^{s-1}} \right)
    \zeta(s)
    &=
    \frac{1}{1^s}
    \underbrace{-\frac{2}{2^s} + \frac{1}{2^s}}_{-\frac{1}{2^s}}
    + \frac{1}{3^s}
    \underbrace{-\frac{2}{4^s} + \frac{1}{4^s}}_{-\frac{1}{4^s}}
    \ldots
    \\
    &= \eta(s).
\end{align}
Dies ist die Fortsetzung auf den noch unbekannten Bereich $0 < \Re(s) < 1$
\begin{equation} \label{zeta:equation:fortsetzung1}
    \zeta(s)
    :=
    \left(1 - \frac{1}{2^{s-1}} \right)^{-1} \eta(s).
\end{equation}

\subsection{Fortsetzung auf ganz $\mathbb{C}$} \label{zeta:subsection:auf_ganz}
Für die Fortsetzung auf den Rest von $\mathbb{C}$, verwenden wir den Zusammenhang von Gamma- und Zetafunktion aus \ref{zeta:section:zusammenhang_mit_gammafunktion}.
Wir beginnen damit, die Gammafunktion für den halben Funktionswert zu berechnen als
\begin{equation}
    \Gamma \left( \frac{s}{2} \right)
    =
    \int_0^{\infty} t^{\frac{s}{2}-1} e^{-t} dt.
\end{equation}
Nun substituieren wir $t$ mit $t = \pi n^2 x$ und $dt=\pi n^2 dx$ und erhalten
\begin{align}
    \Gamma \left( \frac{s}{2} \right)
    &=
    \int_0^{\infty}
    (\pi n^2)^{\frac{s}{2}}
    x^{\frac{s}{2}-1}
    e^{-\pi n^2 x}
    \,dx
    && \text{Division durch } (\pi n^2)^{\frac{s}{2}}
    \\
    \frac{\Gamma \left( \frac{s}{2} \right)}{\pi^{\frac{s}{2}} n^s}
    &=
    \int_0^{\infty}
    x^{\frac{s}{2}-1}
    e^{-\pi n^2 x}
    \,dx
    && \text{Zeta durch Summenbildung } \sum_{n=1}^{\infty}
    \\
    \frac{\Gamma \left( \frac{s}{2} \right)}{\pi^{\frac{s}{2}}}
    \zeta(s)
    &=
    \int_0^{\infty}
    x^{\frac{s}{2}-1}
    \sum_{n=1}^{\infty}
    e^{-\pi n^2 x}
    \,dx. \label{zeta:equation:integral1}
\end{align}
Die Summe kürzen wir ab als $\psi(x) = \sum_{n=1}^{\infty} e^{-\pi n^2 x}$.
%TODO Wieso folgendes -> aus Fourier Signal
Es gilt
\begin{equation}\label{zeta:equation:psi}
    \psi(x)
    =
    - \frac{1}{2}
    + \frac{\psi\left(\frac{1}{x} \right)}{\sqrt{x}}
    + \frac{1}{2 \sqrt{x}}.
\end{equation}

Zunächst teilen wir nun das Integral aus \eqref{zeta:equation:integral1} auf als
\begin{equation}\label{zeta:equation:integral2}
    \int_0^{\infty}
    x^{\frac{s}{2}-1}
    \psi(x)
    \,dx
    =
    \underbrace{
    \int_0^{1}
    x^{\frac{s}{2}-1}
    \psi(x)
    \,dx
    }_{I_1}
    +
    \underbrace{
    \int_1^{\infty}
    x^{\frac{s}{2}-1}
    \psi(x)
    \,dx
    }_{I_2}
    =
    I_1 + I_2,
\end{equation}
wobei wir uns nun auf den ersten Teil $I_1$ konzentrieren werden.
Dabei setzen wir die Definition von $\psi(x)$ aus \eqref{zeta:equation:psi} ein und erhalten
\begin{align}
    I_1
    =
    \int_0^{1}
    x^{\frac{s}{2}-1}
    \psi(x)
    \,dx
    &=
    \int_0^{1}
    x^{\frac{s}{2}-1}
    \left(
    - \frac{1}{2}
    + \frac{\psi\left(\frac{1}{x} \right)}{\sqrt{x}}
    + \frac{1}{2 \sqrt{x}}
    \right)
    \,dx
    \\
    &=
    \int_0^{1}
    x^{\frac{s}{2}-\frac{3}{2}}
    \psi \left( \frac{1}{x} \right)
    + \frac{1}{2}
    \biggl(
    x^{\frac{s}{2}-\frac{3}{2}}
    -
    x^{\frac{s}{2}-1}
    \biggl)
    \,dx
    \\
    &=
    \underbrace{
    \int_0^{1}
    x^{\frac{s}{2}-\frac{3}{2}}
    \psi \left( \frac{1}{x} \right)
    \,dx
    }_{I_3}
    +
    \underbrace{
    \frac{1}{2}
    \int_0^1
    x^{\frac{s}{2}-\frac{3}{2}}
    -
    x^{\frac{s}{2}-1}
    \,dx
    }_{I_4}. \label{zeta:equation:integral3}
\end{align}
Dabei kann das zweite Integral $I_4$ gelöst werden als
\begin{equation}
    I_4
    =
    \frac{1}{2}
    \int_0^1
    x^{\frac{s}{2}-\frac{3}{2}}
    -
    x^{\frac{s}{2}-1}
    \,dx
    =
    \frac{1}{s(s-1)}.
\end{equation}
Das erste Integral $I_3$ aus \eqref{zeta:equation:integral3} mit $\psi \left(\frac{1}{x} \right)$ ist nicht lösbar in dieser Form.
Deshalb substituieren wir $x = \frac{1}{u}$ und $dx = -\frac{1}{u^2}du$.
Die untere Integralgrenze wechselt ebenfalls zu $x_0 = 0 \rightarrow u_0 = \infty$.
Dies ergibt
\begin{align}
    I_3
    =
    \int_{\infty}^{1}
    \left(
    \frac{1}{u}
    \right)^{\frac{s}{2}-\frac{3}{2}}
    \psi(u)
    \frac{-du}{u^2}
    &=
    \int_{1}^{\infty}
    \left(
    \frac{1}{u}
    \right)^{\frac{s}{2}-\frac{3}{2}}
    \psi(u)
    \frac{du}{u^2}
    \\
    &=
    \int_{1}^{\infty}
    x^{(-1) \left(\frac{s}{2}+\frac{1}{2}\right)}
    \psi(x)
    \,dx,
\end{align}
wobei wir durch Multiplikation mit $(-1)$ die Integralgrenzen tauschen dürfen.
Es ist zu beachten das diese Grenzen nun identisch mit den Grenzen des zweiten Integrals von \eqref{zeta:equation:integral2} sind.
Wir setzen beide Lösungen ein in Gleichung \eqref{zeta:equation:integral3} und erhalten
\begin{equation}
    I_1
    =
    \int_0^{1}
    x^{\frac{s}{2}-1}
    \psi(x)
    \,dx
    =
    \int_{1}^{\infty}
    x^{(-1) \left(\frac{s}{2}+\frac{1}{2}\right)}
    \psi(x)
    \,dx
    +
    \frac{1}{s(s-1)}.
\end{equation}
Dieses Resultat setzen wir wiederum ein in \eqref{zeta:equation:integral2}, um schlussendlich
\begin{align}
    \frac{\Gamma \left( \frac{s}{2} \right)}{\pi^{\frac{s}{2}}}
    \zeta(s)
    &=
    \int_0^{1}
    x^{\frac{s}{2}-1}
    \psi(x)
    \,dx
    +
    \int_1^{\infty}
    x^{\frac{s}{2}-1}
    \psi(x)
    \,dx
    \nonumber
    \\
    &=
    \frac{1}{s(s-1)}
    +
    \int_{1}^{\infty}
    x^{(-1) \left(\frac{s}{2}+\frac{1}{2}\right)}
    \psi(x)
    \,dx
    +
    \int_1^{\infty}
    x^{\frac{s}{2}-1}
    \psi(x)
    \,dx
    \\
    &=
    \frac{1}{s(s-1)}
    +
    \int_{1}^{\infty}
    \left(
    x^{-\frac{s}{2}-\frac{1}{2}}
    +
    x^{\frac{s}{2}-1}
    \right)
    \psi(x)
    \,dx
    \\
    &=
    \frac{-1}{s(1-s)}
    +
    \int_{1}^{\infty}
    \left(
    x^{\frac{1-s}{2}}
    +
    x^{\frac{s}{2}}
    \right)
    \frac{\psi(x)}{x}
    \,dx,
\end{align}
zu erhalten.
Wenn wir dieses Resultat genau anschauen, erkennen wir dass sich nichts verändert wenn $s$ mit $1-s$ ersetzt wird.
Somit haben wir die analytische Fortsetzung gefunden als
\begin{equation}\label{zeta:equation:functional}
    \frac{\Gamma \left( \frac{s}{2} \right)}{\pi^{\frac{s}{2}}}
    \zeta(s)
    =
    \frac{\Gamma \left( \frac{1-s}{2} \right)}{\pi^{\frac{1-s}{2}}}
    \zeta(1-s).
\end{equation}
%TODO Definitionen und Gleichungen klarer unterscheiden
