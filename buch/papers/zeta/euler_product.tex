\section{Eulerprodukt} \label{zeta:section:eulerprodukt}
\rhead{Eulerprodukt}

Das Euler-Produkt stellt die gesuchte Verbindung der Zeta-Funktion und der Primzahlen her.
\index{Euler-Produkt}%
Wie der Name bereits sagt, wurde das Eulerprodukt bereits 1727 von Euler entdeckt.
Um daraus die Riemannsche Vermutung herzuleiten, wäre aber noch einiges mehr nötig.

\begin{satz}
    Für alle Zahlen $s$ mit $\Re(s) > 1$ ist die Zeta-Funktion identisch mit dem unendlichen Eulerprodukt
    \begin{equation}\label{zeta:eq:eulerprodukt}
        \zeta(s)
        =
        \sum_{n=1}^\infty
        \frac{1}{n^s}
        =
        \prod_{p \in P}
        \frac{1}{1-p^{-s}}
    \end{equation}
    wobei $P$ die Menge aller Primzahlen darstellt.
\index{Menge der Primzahlen}%
\end{satz}

\begin{proof}[Beweis]
    Der Beweis startet mit dem Eulerprodukt und stellt dieses so um, dass die Zeta-Funktion erscheint.
    Als erstes ersetzen wir die Faktoren durch geometrische Reihen
    \begin{equation}
        \prod_{p\in P}
        \frac{1}{1-p^{-s}}
        =
        \prod_{p \in P}
        \sum_{k_p=0}^{\infty}
        \biggl(
        \frac{1}{p^s}
        \biggr)^{k_p}
        =
        \prod_{p \in P}
        \sum_{k_p=0}^{\infty}
        \frac{1}{p^{s k_p}},
    \end{equation}
    dabei iteriert der Index $i$ über alle Primzahlen $p_i$.
    Durch Ausschreiben der Multiplikation und Ausklammern der Summen erhalten wir
    \begin{align}
        \prod_{p \in P}
        \sum_{k_p=0}^{\infty}
        \frac{1}{p^{s k_p}}
        &=
        \sum_{k_2=0}^{\infty}
        \frac{1}{2^{s k_2}}
        \sum_{k_3=0}^{\infty}
        \frac{1}{3^{s k_3}}
        \sum_{k_5=0}^{\infty}
        \frac{1}{5^{s k_5}}
        \sum_{k_7=0}^{\infty}
        \frac{1}{7^{s k_7}}
        \ldots
        \nonumber \\
        &=
        \sum_{k_2=0}^{\infty}
        \sum_{k_3=0}^{\infty}
        \sum_{k_5=0}^{\infty}
        \ldots
        \biggl(
        \frac{1}{2^{k_2}}
        \frac{1}{3^{k_3}}
        \frac{1}{5^{k_5}}
        \ldots
        \biggr)^s.
        \label{zeta:equation:eulerprodukt2}
    \end{align}
    Der Fundamentalsatz der Arithmetik (Primfaktorzerlegung) besagt, dass jede beliebige Zahl $n \in \mathbb{N}$ durch eine eindeutige Primfaktorzerlegung
    \begin{equation}
        n = \prod_p p^{k_p} \qquad \forall \; n \in \mathbb{N}
    \end{equation}
beschrieben werden kann.
    Jeder Summand der Summen in \eqref{zeta:equation:eulerprodukt2} ist somit der Kehrwert genau einer natürlichen Zahl $n \in \mathbb{N}$.
    Da die Summen alle möglichen Kombinationen von Exponenten und Primzahlen in \eqref{zeta:equation:eulerprodukt2} enthält, haben wir
    \begin{equation}
        \sum_{k_2=0}^{\infty}
        \sum_{k_3=0}^{\infty}
        \sum_{k_5=0}^{\infty}
        \ldots
        \biggl(
        \frac{1}{2^{k_2}}
        \frac{1}{3^{k_3}}
        \frac{1}{5^{k_5}}
        \ldots
        \biggr)^s
        =
        \sum_{n=1}^\infty
        \frac{1}{n^s}
        =
        \zeta(s),
    \end{equation}
    wodurch das Eulerprodukt bewiesen ist.
\end{proof}

