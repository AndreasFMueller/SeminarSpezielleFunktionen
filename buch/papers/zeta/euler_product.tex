\section{Eulerprodukt} \label{zeta:section:eulerprodukt}
\rhead{Eulerprodukt}

Das Eulerprodukt stellt die Verbindung der Zetafunktion und der Primzahlen her.
Diese Verbindung ist sehr wichtig, da durch sie eine Aussage zur Primzahlverteilung gemacht werden kann.
Die Verteilung der Primzahlen ist Gegenstand der Riemannschen Vermutung, welche eines der grössten ungelösten Probleme der Mathematik ist.

\begin{satz}
    Für alle Zahlen $s$ mit $\Re(s) > 1$ ist die Zetafunktion identisch mit dem unendlichen Eulerprodukt
    \begin{equation}\label{zeta:eq:eulerprodukt}
        \zeta(s)
        =
        \sum_{n=1}^\infty
        \frac{1}{n^s}
        =
        \prod_{p \in P}
        \frac{1}{1-p^{-s}}
    \end{equation}
    wobei $P$ die Menge aller Primzahlen darstellt.
\end{satz}

\begin{proof}[Beweis]
    Der Beweis startet mit dem Eulerprodukt und stellt dieses so um, dass die Zetafunktion erscheint.
    Als erstes ersetzen wir die Faktoren durch geometrische Reihen
    \begin{equation}
        \prod_{i=1}^{\infty}
        \frac{1}{1-p^{-s}}
        =
        \prod_{p \in P}
        \sum_{k_i=0}^{\infty}
        \left(
        \frac{1}{p_i^s}
        \right)^{k_i}
        =
        \prod_{p \in P}
        \sum_{k_i=0}^{\infty}
        \frac{1}{p_i^{s k_i}},
    \end{equation}
    dabei iteriert der Index $i$ über alle Primzahlen $p_i$.
    Durch Ausschreiben der Multiplikation und Ausklammern der Summen erhalten wir
    \begin{align}
        \prod_{p \in P}
        \sum_{k_i=0}^{\infty}
        \frac{1}{p_i^{s k_i}}
        &=
        \sum_{k_1=0}^{\infty}
        \frac{1}{p_1^{s k_1}}
        \sum_{k_2=0}^{\infty}
        \frac{1}{p_2^{s k_2}}
        \ldots
        \nonumber \\
        &=
        \sum_{k_1=0}^{\infty}
        \sum_{k_2=0}^{\infty}
        \ldots
        \left(
        \frac{1}{p_1^{k_1}}
        \frac{1}{p_2^{k_2}}
        \ldots
        \right)^s.
        \label{zeta:equation:eulerprodukt2}
    \end{align}
    Der Fundamentalsatz der Arithmetik (Primfaktorzerlegung) besagt, dass jede beliebige Zahl $n \in \mathbb{N}$ durch eine eindeutige Primfaktorzerlegung beschrieben werden kann
    \begin{equation}
        n = \prod_i p_i^{k_i} \quad \forall \quad n \in \mathbb{N}.
    \end{equation}
    Jeder Summand der Summen in \eqref{zeta:equation:eulerprodukt2} ist somit eine Zahl $n$.
    Da die Summen alle möglichen Kombinationen von Exponenten und Primzahlen in \eqref{zeta:equation:eulerprodukt2} enthält haben wir
    \begin{equation}
        \sum_{k_1=0}^{\infty}
        \sum_{k_2=0}^{\infty}
        \ldots
        \left(
        \frac{1}{p_1^{k_1}}
        \frac{1}{p_2^{k_2}}
        \ldots
        \right)^s
        =
        \sum_{n=1}^\infty
        \frac{1}{n^s}
        =
        \zeta(s)
    \end{equation}
\end{proof}

