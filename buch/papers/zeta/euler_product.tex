\section{Eulerprodukt} \label{zeta:section:eulerprodukt}
\rhead{Eulerprodukt}

Das Eulerprodukt stellt die gesuchte Verbindung der Zetafunktion und der Primzahlen her.
Wie der Name bereits sagt, wurde das Eulerprodukt bereits 1727 von Euler entdeckt.
Um daraus die Riemannsche Vermutung herzuleiten, wäre aber noch einiges mehr nötig.

\begin{satz}
    Für alle Zahlen $s$ mit $\Re(s) > 1$ ist die Zetafunktion identisch mit dem unendlichen Eulerprodukt
    \begin{equation}\label{zeta:eq:eulerprodukt}
        \zeta(s)
        =
        \sum_{n=1}^\infty
        \frac{1}{n^s}
        =
        \prod_{p \in P}
        \frac{1}{1-p^{-s}}
    \end{equation}
    wobei $P$ die Menge aller Primzahlen darstellt.
\end{satz}

\begin{proof}[Beweis]
    Der Beweis startet mit dem Eulerprodukt und stellt dieses so um, dass die Zetafunktion erscheint.
    Als erstes ersetzen wir die Faktoren durch geometrische Reihen
    \begin{equation}
        \prod_{i=1}^{\infty}
        \frac{1}{1-p^{-s}}
        =
        \prod_{p \in P}
        \sum_{k_i=0}^{\infty}
        \biggl(
        \frac{1}{p_i^s}
        \biggr)^{k_i}
        =
        \prod_{p \in P}
        \sum_{k_i=0}^{\infty}
        \frac{1}{p_i^{s k_i}},
    \end{equation}
    dabei iteriert der Index $i$ über alle Primzahlen $p_i$.
    Durch Ausschreiben der Multiplikation und Ausklammern der Summen erhalten wir
    \begin{align}
        \prod_{p \in P}
        \sum_{k_i=0}^{\infty}
        \frac{1}{p_i^{s k_i}}
        &=
        \sum_{k_1=0}^{\infty}
        \frac{1}{p_1^{s k_1}}
        \sum_{k_2=0}^{\infty}
        \frac{1}{p_2^{s k_2}}
        \ldots
        \nonumber \\
        &=
        \sum_{k_1=0}^{\infty}
        \sum_{k_2=0}^{\infty}
        \ldots
        \biggl(
        \frac{1}{p_1^{k_1}}
        \frac{1}{p_2^{k_2}}
        \ldots
        \biggr)^s.
        \label{zeta:equation:eulerprodukt2}
    \end{align}
    Der Fundamentalsatz der Arithmetik (Primfaktorzerlegung) besagt, dass jede beliebige Zahl $n \in \mathbb{N}$ durch eine eindeutige Primfaktorzerlegung beschrieben werden kann
    \begin{equation}
        n = \prod_i p_i^{k_i} \quad \forall \quad n \in \mathbb{N}.
    \end{equation}
    Jeder Summand der Summen in \eqref{zeta:equation:eulerprodukt2} ist somit der Kehrwert genau einer natürlichen Zahl $n \in \mathbb{N}$.
    Da die Summen alle möglichen Kombinationen von Exponenten und Primzahlen in \eqref{zeta:equation:eulerprodukt2} enthält, haben wir
    \begin{equation}
        \sum_{k_1=0}^{\infty}
        \sum_{k_2=0}^{\infty}
        \ldots
        \biggl(
        \frac{1}{p_1^{k_1}}
        \frac{1}{p_2^{k_2}}
        \ldots
        \biggr)^s
        =
        \sum_{n=1}^\infty
        \frac{1}{n^s}
        =
        \zeta(s),
    \end{equation}
    wodurch das Eulerprodukt bewiesen ist.
\end{proof}

