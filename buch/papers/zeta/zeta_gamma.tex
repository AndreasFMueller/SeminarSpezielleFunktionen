\section{Zusammenhang mit der Gammafunktion} \label{zeta:section:zusammenhang_mit_gammafunktion}
\rhead{Zusammenhang mit der Gammafunktion}

In diesem Abschnitt wird gezeigt, wie sich die Zetafunktion durch die Gammafunktion $\Gamma(s)$ ausdrücken lässt.
Dieser Zusammenhang der Art $\zeta(s) = f(\Gamma(s))$ ist nicht nur interessant, er wird später auch für die Herleitung der analytischen Fortsetzung gebraucht.

Wir erinnern uns an die Definition der Gammafunktion in \eqref{buch:rekursion:gamma:integralbeweis}
\begin{equation*}
    \Gamma(s)
    =
    \int_0^{\infty} t^{s-1} e^{-t} \,dt,
\end{equation*}
wobei die Notation an die Zetafunktion angepasst ist.
Durch die Substitution $t = nu$ und $dt = n\,du$ wird daraus
\begin{align*}
    \Gamma(s)
    &=
    \int_0^{\infty} n^{s-1}u^{s-1} e^{-nu} n \,du \\
    &=
    \int_0^{\infty} n^s u^{s-1} e^{-nu} \,du.
\end{align*}
Durch Division durch $n^s$ ergeben sich die Quotienten
\begin{equation*}
    \frac{\Gamma(s)}{n^s}
    =
    \int_0^{\infty} u^{s-1} e^{-nu} \,du,
\end{equation*}
welche sich zur Zetafunktion summieren
\begin{equation}
    \sum_{n=1}^{\infty} \frac{\Gamma(s)}{n^s}
    =
    \Gamma(s) \zeta(s)
    =
    \int_0^{\infty} u^{s-1}
    \sum_{n=1}^{\infty}e^{-nu}
    \,du.
    \label{zeta:equation:zeta_gamma1}
\end{equation}
Die Summe über $e^{-nu}$ können wir als geometrische Reihe schreiben und erhalten
\begin{align}
    \sum_{n=1}^{\infty}\left(e^{-u}\right)^n
    &=
    \sum_{n=0}^{\infty}\left(e^{-u}\right)^n
    -
    1
    \\
    &=
    \frac{1}{1 - e^{-u}} - 1
    \\
    &=
    \frac{1}{e^u - 1}.
\end{align}
Wenn wir dieses Resultat einsetzen in \eqref{zeta:equation:zeta_gamma1} und durch $\Gamma(s)$ teilen, erhalten wir den gewünschten Zusammenhang
\begin{equation}\label{zeta:equation:zeta_gamma_final}
    \zeta(s)
    =
    \frac{1}{\Gamma(s)}
    \int_0^{\infty}
    \frac{u^{s-1}}{e^u -1}
    du.
\end{equation}
