\section{Zusammenhang mit Gammafunktion} \label{zeta:section:zusammenhang_mit_gammafunktion}
\rhead{Zusammenhang mit Gammafunktion}

Dieser Abschnitt stellt die Verbindung zwischen der Gamma- und der Zetafunktion her.

%TODO ref Gamma
Wenn in der Gammafunkion die Integrationsvariable $t$ substituieren mit $t = nu$ und $dt = n du$, dann können wir die Gleichung umstellen und erhalten den Zusammenhang mit der Zetafunktion
\begin{align}
    \Gamma(s)
    &=
    \int_0^{\infty} t^{s-1} e^{-t} dt
    \\
    &=
    \int_0^{\infty} n^{s\cancel{-1}}u^{s-1} e^{-nu} \cancel{n}du
    &&
    \text{Division durch }n^s
    \\
    \frac{\Gamma(s)}{n^s}
    &=
    \int_0^{\infty} u^{s-1} e^{-nu}du
    &&
    \text{Zeta durch Summenbildung } \sum_{n=1}^{\infty}
    \\
    \Gamma(s) \zeta(s)
    &=
    \int_0^{\infty} u^{s-1}
    \sum_{n=1}^{\infty}e^{-nu}
    du.
    \label{zeta:equation:zeta_gamma1}
\end{align}
Die Summe über $e^{-nu}$ können wir als geometrische Reihe schreiben und erhalten
\begin{align}
    \sum_{n=1}^{\infty}e^{-u^n}
    &=
    \sum_{n=0}^{\infty}e^{-u^n}
    -
    1
    \\
    &=
    \frac{1}{1 - e^{-u}} - 1
    \\
    &=
    \frac{1}{e^u - 1}.
\end{align}
Wenn wir dieses Resultat einsetzen in \eqref{zeta:equation:zeta_gamma1} und durch $\Gamma(s)$ teilen, erhalten wir
\begin{equation}\label{zeta:equation:zeta_gamma_final}
    \zeta(s)
    =
    \frac{1}{\Gamma(s)}
    \int_0^{\infty}
    \frac{u^{s-1}}{e^u -1}
    du.
\end{equation}
