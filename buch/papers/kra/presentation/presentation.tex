\documentclass[ngerman, aspectratio=169, xcolor={rgb}]{beamer}

% style
\mode<presentation>{
	\usetheme{Frankfurt}
}
%packages
\usepackage[utf8]{inputenc}\DeclareUnicodeCharacter{2212}{-}
\usepackage[english]{babel}
\usepackage{graphicx}
\usepackage{array}

\newcolumntype{L}[1]{>{\raggedright\let\newline\\\arraybackslash\hspace{0pt}}m{#1}}
\usepackage{ragged2e}

\usepackage{bm} % bold math
\usepackage{amsfonts}
\usepackage{amssymb}
\usepackage{mathtools}
\usepackage{amsmath}
\usepackage{multirow} % multi row in tables
\usepackage{booktabs} %toprule midrule bottomrue in tables
\usepackage{scrextend}
\usepackage{textgreek}
\usepackage[rgb]{xcolor}

\usepackage[normalem]{ulem} % \sout

\usepackage{ marvosym } % \Lightning

\usepackage{multimedia} % embedded videos

\usepackage{tikz}
\usepackage{pgf}
\usepackage{pgfplots}

\usepackage{algorithmic}

%citations
\usepackage[style=verbose,backend=biber]{biblatex}
\addbibresource{references.bib}


%math font
\usefonttheme[onlymath]{serif}

%Beamer Template modifications
%\definecolor{mainColor}{HTML}{0065A3} % HSR blue
\definecolor{mainColor}{HTML}{D72864} % OST pink
\definecolor{invColor}{HTML}{28d79b} % OST pink
\definecolor{dgreen}{HTML}{38ad36} % Dark green

%\definecolor{mainColor}{HTML}{000000} % HSR blue
\setbeamercolor{palette primary}{bg=white,fg=mainColor}
\setbeamercolor{palette secondary}{bg=orange,fg=mainColor}
\setbeamercolor{palette tertiary}{bg=yellow,fg=red}
\setbeamercolor{palette quaternary}{bg=mainColor,fg=white} %bg = Top bar, fg = active top bar topic
\setbeamercolor{structure}{fg=black} % itemize, enumerate, etc (bullet points)
\setbeamercolor{section in toc}{fg=black} % TOC sections
\setbeamertemplate{section in toc}[sections numbered]
\setbeamertemplate{subsection in toc}{%
	\hspace{1.2em}{$\bullet$}~\inserttocsubsection\par}

\setbeamertemplate{itemize items}[circle]
\setbeamertemplate{description item}[circle]
\setbeamertemplate{title page}[default][colsep=-4bp,rounded=true]
\beamertemplatenavigationsymbolsempty

\setbeamercolor{footline}{fg=gray}
\setbeamertemplate{footline}{%
	\hfill\usebeamertemplate***{navigation symbols}
	\hspace{0.5cm}
	\insertframenumber{}\hspace{0.2cm}\vspace{0.2cm}
}

\usepackage{caption}
\captionsetup{labelformat=empty}

%Title Page
\title{KRA}
\subtitle{Kalman Riccati Abel}
\author{Samuel Niederer}
% \institute{OST Ostschweizer Fachhochschule}
% \institute{\includegraphics[scale=0.3]{../img/ost_logo.png}}
\date{\today}

%
% packages.tex -- packages required by the paper sturmliouville
%
% (c) 2019 Prof Dr Andreas Müller, Hochschule Rapperswil
%

% if your paper needs special packages, add package commands as in the
% following example
%\usepackage{packagename}



\newcommand*{\QED}{\hfill\ensuremath{\blacksquare}}%

\newcommand*{\HL}{\textcolor{mainColor}}
\newcommand*{\RD}{\textcolor{red}}
\newcommand*{\BL}{\textcolor{blue}}
\newcommand*{\GN}{\textcolor{dgreen}}
\newcommand{\dt}[0]{\frac{d}{dt}}

\definecolor{darkgreen}{rgb}{0,0.6,0}


\makeatletter
\newcount\my@repeat@count
\newcommand{\myrepeat}[2]{%
	\begingroup
	\my@repeat@count=\z@
	\@whilenum\my@repeat@count<#1\do{#2\advance\my@repeat@count\@ne}%
	\endgroup
}
\makeatother

\usetikzlibrary{automata,arrows,positioning,calc,shapes.geometric, fadings}

\begin{document}

\begin{frame}
	\titlepage
\end{frame}

\begin{frame}
	\frametitle{Content}
	\tableofcontents
\end{frame}

\section{Einführung}

\begin{frame}
	\begin{itemize}
		\item<1|only@1> \textbf{K}alman
		\item<1|only@1> \textbf{R}iccati
		\item<1|only@1> \textbf{A}bel

		\item<2|only@2> \textcolor{red}{\sout{\textbf{K}alman}}
		\item<2|only@2> \textbf{R}iccati
		\item<2|only@2> \textbf{A}bel

		\item<3|only@3> \textcolor{red}{\sout{\textbf{K}alman}} \textcolor{green}{Federmassesytem}
		\item<3|only@3> \textbf{R}iccati
		\item<3|only@3> \textbf{A}bel

		\item<4|only@4> \textcolor{red}{\sout{\textbf{K}alman}} \textcolor{green}{Federmassesytem}
		\item<4|only@4> \textbf{R}iccati
		\item<4|only@4> \uwave{\textbf{A}bel}
	\end{itemize}
\end{frame}

\section{Riccati}

\begin{frame}
	\frametitle{Riccatische Differentialgleichung}
	\begin{equation*}
		% y'(x) = f(x)y^2(x) + g(x)y(x) + h(x)
		x'(t) = f(t)x^2(t) + g(t)x(t) + h(t)
	\end{equation*}

	\pause

	\begin{equation*}
		\dot{X}(t) = - X(t)BX(t) - X(t)A + DX(t) + C
	\end{equation*}

	% \pause
	% Anwendungen
	% \begin{itemize}
	% 	\item Zeitkontinuierlicher Kalmanfilter
	% 	\item Regelungstechnik LQ-Regler
	% 	\item Federmassesyteme
	% \end{itemize}
\end{frame}

\begin{frame}
	\frametitle{Auftreten der Gleichung}
	\begin{columns}
		\column{0.4 \textwidth}
		\begin{equation*}
			\dt
			\begin{pmatrix}
				X \\
				Y
			\end{pmatrix}
			=
			\underbrace{
				\begin{pmatrix}
					A & B \\
					C & D
				\end{pmatrix}
			}_{H}
			\begin{pmatrix}
				X \\
				Y
			\end{pmatrix}
		\end{equation*}

		\pause

		\column{0.4 \textwidth}
		\begin{equation*}
			U = YX^{-1} \qquad \dt U = ?
		\end{equation*}
	\end{columns}

	\pause

	\begin{align*}
		\dt U         & = \dot{Y} X^{-1} + Y \dt X^{-1}                                                                                                    \\
		\uncover<4->{ & = (CX + DY) X^{-1} - Y (X^{-1} \dot{X} X^{-1})\\}
		\uncover<5->{ & = C\underbrace{XX^{-1}}_\text{I} + D\underbrace{YX^{-1}}_\text{U} - Y(X^{-1} (AX + BY) X^{-1})\\}
		\uncover<6->{ & = C + DU - \underbrace{YX^{-1}}_\text{U}(A\underbrace{XX^{-1}}_\text{I} + B\underbrace{YX^{-1}}_\text{U})\\}
		\uncover<7->{ & = C  + DU - UA - UBU}
	\end{align*}
\end{frame}

\begin{frame}
	\frametitle{Lösen der Gleichung}
	\begin{equation*}
		\begin{pmatrix}
			X(t) \\
			Y(t)
		\end{pmatrix}
		=
		\Phi(t_0, t)
		\begin{pmatrix}
			I(t) \\
			U_0(t)
		\end{pmatrix}
		=
		\begin{pmatrix}
			\Phi_{11}(t_0, t) & \Phi_{12}(t_0, t) \\
			\Phi_{21}(t_0, t) & \Phi_{22}(t_0, t)
		\end{pmatrix}
		\begin{pmatrix}
			I(t) \\
			U_0(t)
		\end{pmatrix}
	\end{equation*}

	\pause

	\begin{equation*}
		U(t) =
		\begin{pmatrix}
			\Phi_{21}(t_0, t) + \Phi_{22}(t_0, t) U_0(t)
		\end{pmatrix}
		\begin{pmatrix}
			\Phi_{11}(t_0, t) + \Phi_{12}(t_0, t) U_0(t)
		\end{pmatrix}
		^{-1}
	\end{equation*}

	\pause

	% wobei $\Phi(t, t_0)$ die sogennante Zustandsübergangsmatrix ist.

	\begin{equation*}
		\Phi(t_0, t) = e^{H(t - t_0)}
	\end{equation*}
\end{frame}

\section{Federmassystem}
\begin{frame}
	\frametitle{Federmassesystem}
	\begin{columns}
		\column{0.5 \textwidth}
		% create tikz drawing of a simple mass spring system

\tikzstyle{hmline}=[{Latex[length=3.3,width=2.2]}-{Latex[length=3.3,width=2.2]},line width=0.3]
\tikzstyle{vmline}=[red, dashed,line width=0.4,dash pattern=on 1pt off 1pt]
\tikzstyle{ground}=[pattern=north east lines]
\tikzstyle{mass}=[line width=0.6,red!30!black,fill=red!40!black!10,rounded corners=1,top color=red!40!black!20,bottom color=red!40!black!10,shading angle=20]
\tikzstyle{spring}=[line width=0.8,blue!7!black!80,snake=coil,segment amplitude=5,line cap=round]

\begin{tikzpicture}[scale=2]
    \newcommand{\ticks}[2]
    {
        % arguments: x, y coordinates
        \draw[thick] (#1, #2 - 0.1 / 2) --++ (0, 0.1);
    }

    \tikzmath{
        \hWall = 1.2;
        \wWall = 0.3;
        \lWall = 3.5;
        \hMass = 0.6;
        \wMass = 1.1;
        \xMass1 = 1.2;
        \xMass2 = 2.2;
        \xAxisYpos = 0;
        \originX1 = 0;
        \originY1 = 0.5;
        \originX2 = 0;
        \originY2 = -2;
        \springscale=7;
    }

    % create x axis
    \draw[->,thick] (0,\xAxisYpos) --+ (\lWall, 0) node[right]{$x$};

    % create ticks on x axis
    \ticks{\wWall}{\xAxisYpos}
    \ticks{\xMass1}{\xAxisYpos}
    \ticks{\xMass2}{\xAxisYpos}

    % create underground
    \draw[ground] (\originX1, \originY1) ++ (0, 0) --+(\lWall,0) --+(\lWall, \wWall) --+(\wWall, \wWall) --+(\wWall, \hWall) --+(0, \hWall) -- cycle;
    \draw[ground] (\originX2, \originY2) ++ (0, 0) --+(\lWall,0) --+(\lWall, \wWall) --+(\wWall, \wWall) --+(\wWall, \hWall) --+(0, \hWall) -- cycle;

    % create masses
    \draw[mass] (\originX1, \originY1) ++ (\xMass1, \wWall) rectangle ++ (\wMass,\hMass) node[midway] {$m$};
    \draw[mass] (\originX2, \originY2) ++ (\xMass2, \wWall) rectangle ++ (\wMass,\hMass) node[midway] {$m$};

    % create springs
    \draw[spring, segment length=(\xMass1 - \wWall) * \springscale] (\originX1, \originY1) ++
    (\wWall, \wWall + \hMass / 2) --++ (\xMass1 - \wWall, 0) node[midway,above=0.2] {$k$};
    \draw[spring, segment length=(\xMass2 - \wWall) * \springscale] (\originX2, \originY2) ++
    (\wWall, \wWall + \hMass / 2) --++ (\xMass2 - \wWall, 0) node[midway,above=0.2] {$k$};

    % create vertical measurement line 
    \draw[vmline] (\xMass1, \xAxisYpos) --+(0, \originY1 + \wWall);
    \draw[vmline] (\xMass2, \xAxisYpos) --+(0, \originY2 + \hMass+\wWall);
    \draw[vmline] (\wWall, \originY1+\wWall) --(\wWall, \originY2 + \hWall);

    % create horizontal measurement line
    \draw[hmline] (\wWall, \xAxisYpos + 0.2) -- (\xMass1, \xAxisYpos + 0.2) node[midway,fill=white,inner sep=0] {$\ell_0$};
    \draw[hmline] (\xMass1, \xAxisYpos + 0.2) -- (\xMass2, \xAxisYpos + 0.2) node[midway,fill=white,inner sep=0] {$\Delta_{x}$};
    \draw[hmline] (\wWall, \xAxisYpos - 0.3) -- (\xMass2, \xAxisYpos - 0.3) node[midway,fill=white,inner sep=0] {$\ell_{1}$};

    %  create force arrow
    \draw[->,blue, very thick,line cap=round] (\xMass2 + \wMass / 2, \originY2 + \wWall + \hMass + 0.15) node[above] {$\vb{F_{R}}$} --+ (-0.5, 0);
\end{tikzpicture}

		\column{0.5 \textwidth}
		\begin{align*}
			\uncover<2->{F_R      & = k \Delta_x  \\}
			\uncover<3->{F_a      & = am = \ddot{x} m \\}
			\uncover<4->{F_R      & = F_a \Leftrightarrow k \Delta_x = \ddot{x} m\\}
			\uncover<5->{\ddot{x} & = \frac{k \Delta_x}{m} \\}
			\uncover<6->{x(t)     & = A \cos(\omega_0 + \Phi), \quad \omega_0 = \sqrt{\frac{k}{m}}}
		\end{align*}
	\end{columns}
\end{frame}

\begin{frame}
	\frametitle{Phasenraum}
	\begin{columns}
		\column{0.3 \textwidth}
		\begin{tikzpicture}[scale=3]
			\draw[->, thick] (0, 0) -- (1,0) node[right] {$q$};
			\draw[->, thick] (0.5, -0.5) -- (0.5,0.5) node[above]{$p$};
		\end{tikzpicture}
		\column{0.7 \textwidth}
		Impulskoordinaten $p = (p_{1}, p_{2}, ..., p_{n}), \quad p=mv$ \\
		Ortskoordinaten $q = (q_{1}, q_{2}, ..., q_{n})$ \\



		\begin{align*}
			\uncover<2->{\mathcal{H}(q, p) & = \underbrace{T(p)}_{E_{kin}} + \underbrace{V(q)}_{E_{pot}} = E_{tot} \\}
			\uncover<3->{                  & = \frac{p^2}{2m}+ \frac{k q^2}{2}}
		\end{align*}



		\begin{equation*}
			\uncover<4->{
				\dot{q_{k}} = \frac{\partial \mathcal{H}}{\partial p_k}
				\qquad
				\dot{p_{k}} = -\frac{\partial \mathcal{H}}{\partial q_k}
			}
		\end{equation*}

		\pause

		\begin{equation*}
			\uncover<5->{
				\begin{pmatrix}
					\dot{q} \\
					\dot{p}
				\end{pmatrix}
				=
				\begin{pmatrix}
					0  & \frac{1}{m} \\
					-k & 0
				\end{pmatrix}
				\begin{pmatrix}
					q \\
					p
				\end{pmatrix}
			}
		\end{equation*}

	\end{columns}
\end{frame}

\begin{frame}
	\frametitle{Phasenraum}
	\colorlet{mypurple}{red!50!blue!90!black!80}

% style to create arrows
\tikzset{
    traj/.style 2 args={thick, postaction={decorate},decoration={markings,
                    mark=at position #1 with {\arrow{<}},
                    mark=at position #2 with {\arrow{<}}}
        }
}

\begin{tikzpicture}[scale=0.6]
    % p(t=0) = 0, q(t=0) = A, max(p) = mwA
    \tikzmath{
        \axh = 5.2;
        \axw1 = 4.2;
        \axw2 = 4.8;
        \d1 = 0.9;
        \a0 = 1;
        \b0 = 2;
        \a1 = \a0 + \d1;
        \b1 = \b0 + \d1;
        \a2 = \a1 + \d1;
        \b2 = \b1 + \d1;
        \a3 = \a2 + \d1;
        \b3 = \b2 + \d1;
        \d2 = 0.75;
        \aa0 = 2;
        \bb0 = 1;
        \aa1 = \aa0 + \d2;
        \bb1 = \bb0 + \d2;
        \aa2 = \aa1 + \d2;
        \bb2 = \bb1 + \d2;
        \aa3 = \aa2 + \d2;
        \bb3 = \bb2 + \d2;
    }

    \draw[->,thick] (-\axw1,0) -- (\axw1,0) node[right] {$q$};
    \draw[->,thick] (0,-\axh) -- (0,\axh) node[above] {$p$};

    \draw[traj={0.375}{0.875},darkgreen] ellipse (\a0 and \b0);
    \draw[traj={0.375}{0.875},blue] ellipse (\a1 and \b1);
    \draw[traj={0.375}{0.875},cyan] ellipse (\a2 and \b2);
    \draw[traj={0.375}{0.875},mypurple] ellipse (\a3 and \b3);

    \node[right,darkgreen] at (45:{\a0} and {\b0}) {$E_A$};
    \node[right, blue] at (45:{\a1} and {\b1}) {$E_B$};
    \node[right, cyan] at (45:{\a2} and {\b2}) {$E_C$};
    \node[right, mypurple] at (45:{\a3} and {\b3}) {$E_D$};
    \node[above left] at (110:\b3 + 0.1) {grosses $\omega$};

    \begin{scope}[xshift=12cm]
        \draw[->,thick] (-\axw2,0) -- (\axw2,0) node[right] {$q$};
        \draw[->,thick] (0,-\axh) -- (0,\axh) node[above] {$p$};

        \draw[traj={0.375}{0.875},darkgreen] ellipse (\aa0 and \bb0);
        \draw[traj={0.375}{0.875},blue] ellipse (\aa1 and \bb1);
        \draw[traj={0.375}{0.875},cyan] ellipse (\aa2 and \bb2);
        \draw[traj={0.375}{0.875},mypurple] ellipse (\aa3 and \bb3);

        \node[above, darkgreen] at (45:{\aa0} and {\bb0}) {$E_A$};
        \node[above, blue] at (45:{\aa1} and {\bb1}) {$E_B$};
        \node[above, cyan] at (45:{\aa2} and {\bb2}) {$E_C$};
        \node[above, mypurple] at (45:{\aa3} and {\bb3}) {$E_D$};

        \node[above left] at (110:\b3 + 0.1) {kleines $\omega$};
    \end{scope}
\end{tikzpicture}
\end{frame}

\begin{frame}
	\frametitle{Federmassesystem}
	\begin{columns}
		\column{0.6 \textwidth}
		\scalebox{0.8}{% create tikz drawing of a multi mass multi spring system

\tikzstyle{vmline}=[red, dashed,line width=0.4,dash pattern=on 1pt off 1pt]
\tikzstyle{ground}=[pattern=north east lines]
\tikzstyle{mass}=[line width=0.6,red!30!black,fill=red!40!black!10,rounded corners=1,top color=red!40!black!20,bottom color=red!40!black!10,shading angle=20]
\tikzstyle{spring}=[line width=0.8,blue!7!black!80,snake=coil,segment amplitude=5,line cap=round]

\begin{tikzpicture}[scale=2]
    \newcommand{\ticks}[3]
    {
        % x, y coordinates
        \draw[thick] (#1, #2 - 0.1 / 2) --++ (0, 0.1) node[scale=0.8,below=0.2] {#3};
    }
    \tikzmath{
        \hWall = 1.2;
        \wWall = 0.3;
        \lWall = 5;
        \hMass = 0.6;
        \wMass = 1.1;
        \xMass1 = 1.0;
        \xMass2 = 3.0;
        \xAxisYpos = 0;
        \originX1 = 0;
        \originY1 = 0.5;
        \springscale=7;
    }

    % create axis
    \draw[->,thick] (0,\xAxisYpos) --+ (\xMass2 + \wMass, 0) node[right]{$q$};
    % create ticks on x / q axis
    \ticks{\xMass1}{\xAxisYpos}{$q_{1}$}
    \ticks{\xMass2}{\xAxisYpos}{$q_{2}$}

    % create non-moving backgrounds
    \draw[ground] (\originX1, \originY1) ++ (0, 0) --+(\lWall,0) --+(\lWall, \hWall)
    --+ (\lWall - \wWall, \hWall) --+(\lWall - \wWall, \wWall) --+ (\wWall, \wWall)  --+(\wWall, \hWall) --+(0, \hWall) -- cycle;

    % create masses
    \draw[mass] (\originX1, \originY1) ++ (\xMass1, \wWall) rectangle ++ (\wMass,\hMass) node[midway] {$m_{1}$};
    \draw[mass] (\originX1, \originY1) ++ (\xMass2, \wWall) rectangle ++ (\wMass,\hMass) node[midway] {$m_{2}$};

    % create springs
    \draw[spring, segment length=(\xMass1 - \wWall) * \springscale] (\originX1, \originY1) ++
    (\wWall, \wWall + \hMass / 2) --++ (\xMass1 - \wWall, 0) node[midway,above=0.2] {$k_1$};
    \draw[spring, segment length=(\xMass1 - \wWall) * \springscale] (\originX1, \originY1) ++
    (\xMass1 + \wMass, \wWall + \hMass / 2) --++ (\xMass2 - \xMass1 - \wMass, 0) node[midway,above=0.2] {$k_c$};
    \draw[spring, segment length=(\xMass1 - \wWall) * \springscale] (\originX1, \originY1) ++
    (\xMass2 + \wMass, \wWall + \hMass / 2) --++ (\lWall - \xMass2 - \wMass - \wWall, 0) node[midway,above=0.2] {$k_2$};

    % create vertical measurement line 
    \draw[vmline] (\xMass1, \xAxisYpos) --+(0, \originY1 + \wWall);
    \draw[vmline] (\xMass2, \xAxisYpos) --+(0, \originY1 + \wWall);

\end{tikzpicture}
}
		\begin{align*}
			\uncover<2->{\mathcal{H} & = T + V \\}
			\uncover<7->{            & = \frac{p_1^2}{2m_1} + \frac{p_2^2}{2m_2} + \frac{k_1 q_1^2}{2} + \frac{k_c (q_2 - q_1)^2}{2} + \frac{k_2 q_2^2}{2}}
		\end{align*}

		\column{0.4 \textwidth}
		\begin{align*}
			\uncover<3->{T & = T_1 + T_2}                                                               \\
			\uncover<5->{  & = \frac{p_1^2}{2m_1} + \frac{p_2^2}{2m_2}   }                              \\
			\uncover<4->{V & = V_1 + V_c + V_2 }                                                        \\
			\uncover<6->{  & = \frac{k_1 q_1^2}{2} + \frac{k_c (q_2 - q_1)^2}{2} + \frac{k_2 q_2^2}{2}}
		\end{align*}
	\end{columns}
\end{frame}

\begin{frame}
	\frametitle{Federmassesystem}
	\begin{equation*}
		\begin{pmatrix}
			\dot{q_1} \\
			\dot{q_2} \\
			\dot{p_1} \\
			\dot{p_2} \\
		\end{pmatrix}
		=
		\begin{pmatrix}
			0            & 0            & \frac{1}{2m_1} & 0              \\
			0            & 0            & 0              & \frac{1}{2m_2} \\
			-(k_1 + k_c) & k_c          & 0              & 0              \\
			k_c          & -(k_c + k_2) & 0              & 0              \\
		\end{pmatrix}
		\begin{pmatrix}
			q_1 \\
			q_2 \\
			p_1 \\
			p_2 \\
		\end{pmatrix}
		\Leftrightarrow
		\dt
		\begin{pmatrix}
			Q \\
			P \\
		\end{pmatrix}
		\underbrace{
			\begin{pmatrix}
				0 & M \\
				K & 0
			\end{pmatrix}
		}_{H}
		\begin{pmatrix}
			Q \\
			P \\
		\end{pmatrix}
	\end{equation*}

	\pause

	$U = PQ^{-1} \qquad \dt U = ?$

	\pause

	\begin{align*}
		\dt U & = C  + DU - UA - UBU \\
		      & = K - UMU
	\end{align*}

\end{frame}

\begin{frame}
	\frametitle{Einfluss der Anfangsbedingung:}
	\begin{columns}
		\column{0.4 \textwidth}
		\begin{equation*}
			\uncover<2->{q_0 =
				\begin{pmatrix}
					q_{10} \\
					q_{20}
				\end{pmatrix}
				=
				\begin{pmatrix}
					3 \\
					1
				\end{pmatrix}
			}
		\end{equation*}
		\begin{equation*}
			\uncover<3->{q_0 =
				\begin{pmatrix}
					q_{10} \\
					q_{20}
				\end{pmatrix}
				=
				\begin{pmatrix}
					3 \\
					3
				\end{pmatrix}
			}
		\end{equation*}
		\begin{equation*}
			\uncover<4->{q_0 =
				\begin{pmatrix}
					q_{10} \\
					q_{20}
				\end{pmatrix}
				=
				\begin{pmatrix}
					2 \\
					-2
				\end{pmatrix}
			}
		\end{equation*}
		\column{0.6 \textwidth}
		\scalebox{0.8}{% create tikz drawing of a multi mass multi spring system

\tikzstyle{vmline}=[red, dashed,line width=0.4,dash pattern=on 1pt off 1pt]
\tikzstyle{ground}=[pattern=north east lines]
\tikzstyle{mass}=[line width=0.6,red!30!black,fill=red!40!black!10,rounded corners=1,top color=red!40!black!20,bottom color=red!40!black!10,shading angle=20]
\tikzstyle{spring}=[line width=0.8,blue!7!black!80,snake=coil,segment amplitude=5,line cap=round]

\begin{tikzpicture}[scale=2]
    \newcommand{\ticks}[3]
    {
        % x, y coordinates
        \draw[thick] (#1, #2 - 0.1 / 2) --++ (0, 0.1) node[scale=0.8,below=0.2] {#3};
    }
    \tikzmath{
        \hWall = 1.2;
        \wWall = 0.3;
        \lWall = 5;
        \hMass = 0.6;
        \wMass = 1.1;
        \xMass1 = 1.0;
        \xMass2 = 3.0;
        \xAxisYpos = 0;
        \originX1 = 0;
        \originY1 = 0.5;
        \springscale=7;
    }

    % create axis
    \draw[->,thick] (0,\xAxisYpos) --+ (\xMass2 + \wMass, 0) node[right]{$q$};
    % create ticks on x / q axis
    \ticks{\xMass1}{\xAxisYpos}{$q_{1}$}
    \ticks{\xMass2}{\xAxisYpos}{$q_{2}$}

    % create non-moving backgrounds
    \draw[ground] (\originX1, \originY1) ++ (0, 0) --+(\lWall,0) --+(\lWall, \hWall)
    --+ (\lWall - \wWall, \hWall) --+(\lWall - \wWall, \wWall) --+ (\wWall, \wWall)  --+(\wWall, \hWall) --+(0, \hWall) -- cycle;

    % create masses
    \draw[mass] (\originX1, \originY1) ++ (\xMass1, \wWall) rectangle ++ (\wMass,\hMass) node[midway] {$m_{1}$};
    \draw[mass] (\originX1, \originY1) ++ (\xMass2, \wWall) rectangle ++ (\wMass,\hMass) node[midway] {$m_{2}$};

    % create springs
    \draw[spring, segment length=(\xMass1 - \wWall) * \springscale] (\originX1, \originY1) ++
    (\wWall, \wWall + \hMass / 2) --++ (\xMass1 - \wWall, 0) node[midway,above=0.2] {$k_1$};
    \draw[spring, segment length=(\xMass1 - \wWall) * \springscale] (\originX1, \originY1) ++
    (\xMass1 + \wMass, \wWall + \hMass / 2) --++ (\xMass2 - \xMass1 - \wMass, 0) node[midway,above=0.2] {$k_c$};
    \draw[spring, segment length=(\xMass1 - \wWall) * \springscale] (\originX1, \originY1) ++
    (\xMass2 + \wMass, \wWall + \hMass / 2) --++ (\lWall - \xMass2 - \wMass - \wWall, 0) node[midway,above=0.2] {$k_2$};

    % create vertical measurement line 
    \draw[vmline] (\xMass1, \xAxisYpos) --+(0, \originY1 + \wWall);
    \draw[vmline] (\xMass2, \xAxisYpos) --+(0, \originY1 + \wWall);

\end{tikzpicture}
}
	\end{columns}
\end{frame}

\section{Schlussteil}
\begin{frame}
	\frametitle{Zusammenfassung}
	\begin{itemize}
		\pause
		\item{Riccatische Differentialgleichung}
		      \pause
		      \begin{itemize}
			      \item{Ausgansgleichung}
			            \pause
			      \item{Lösung}
		      \end{itemize}
		      \pause
		\item{Harmonischer Ozillator}
		      \pause
		      \begin{itemize}
			      \item{Hamiltonfunktion}
			            \pause
			      \item{Phasenraum}
		      \end{itemize}
		      \pause
		\item{Gekoppelter harmonischer Ozillator}
		      \pause
		      \begin{itemize}
			      \item{Riccatische Differentialgleichung}
			            \pause
			      \item{Einfluss der Anfangsbedingungen}
		      \end{itemize}
		      \pause
		\item{\uwave{Abel}}
		      \begin{itemize}
			      \pause
			      \item{Nichtlineare Federkonstante}
		      \end{itemize}

	\end{itemize}
\end{frame}

\end{document}
