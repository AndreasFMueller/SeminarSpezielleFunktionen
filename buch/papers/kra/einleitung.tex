\section{Einleitung} \label{kra:section:einleitung}
\rhead{Einleitung}
Die riccatische Differentialgleichung ist eine nicht lineare gewöhnliche Differentialgleichung erster Ordnung der Form
\begin{equation}
    \label{kra:equation:riccati}
    y' = f(x)y + g(x)y^2 + h(x).
\end{equation}
Sie ist benannt nach dem italienischen Grafen Jacopo Francesco Riccati (1676–1754) der sich mit der Klassifizierung von Differentialgleichungen befasste.
Als Riccati Gleichung werden auch Matrixgleichungen der Form
\begin{equation}
    \label{kra:equation:matrixriccati}
    \dot{X}(t) = C + DX(t) - X(t)A -X(t)BX(t)
\end{equation}
bezeichnet, welche aufgrund ihres quadratischen Terms eine gewisse Ähnlichkeit aufweisen \cite{kra:riccati} \cite{kra:ethz}.
