\section{Einleitung} \label{kra:section:einleitung}
\rhead{Einleitung}
Die riccatische Differentialgleichung ist eine nichtlineare gewöhnliche Differentialgleichunge erster Ordnung der form
\begin{equation}
    \label{kra:riccati}
    y'(x) = f(x)y^2(x) + g(x)y(x) + h(x)
\end{equation}
Sie ist bennant nach dem italienischen Grafen Jacopo Francesco Riccati (1676–1754) der sich mit der Klassifizierung von Differentialgleichungen befasste und Methoden zur Verringerung der Ordnung von Gleichungen entwickelte.
Als Riccati Gleichung werden auch Matrixgleichugen der Form
\begin{equation}
    \label{kra:matrixriccati}
    \dot{U}(t) = DU(t) - UA(t) - U(t)BU(t) % +Q ?
\end{equation}
bezeichnet, welche aufgrund ihres quadratischen Terms eine gewisse ähnlichkeit aufweisen.