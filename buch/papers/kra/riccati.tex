\section{Riccati
  \label{kra:section:riccati}}
\rhead{Riccati}

\begin{equation}
    y'(x) = f(x)y^2(x) + g(x)y(x) + h(x)
\end{equation}
% einfache (normale riccati gleichung und ihre loesung)
% (kann man diese bei einfachem federmasse system benutzten?)
% matrix riccati gleichung


Die zeitkontinuierliche Riccati-Matrix-Gleichung hat die Form
\begin{equation}
    \label{kra:riccati:riccatiequation}
    \dot{U(t)} = DU(t) - UA(t) - U(t)BU(t)
\end{equation}

Betrachten wir das Differentialgleichungssystem \ref{kra:riccati:derivation}

\begin{equation}
    \label{kra:riccati:derivation}
    \dt
    \begin{pmatrix}
        X \\
        Y
    \end{pmatrix}
    =
    \underbrace{
        \begin{pmatrix}
            A & B \\
            C & D
        \end{pmatrix}
    }_{H}
    \begin{pmatrix}
        X \\
        Y
    \end{pmatrix}
\end{equation}

interessieren wir uns für die zeitliche Änderung der Grösse $U = YX^{-1}$, so erhalten wir durch einsetzten

\begin{align*}
    \dt U & = \dot{Y} X^{-1} + Y \dt X^{-1}                                                                           \\
          & = (CX + DY) X^{-1} - Y (X^{-1} \dot{X} X^{-1})                                                            \\
          & = C\underbrace{XX^{-1}}_\text{I} + D\underbrace{YX^{-1}}_\text{U} - Y(X^{-1} (AX + BY) X^{-1})            \\
          & = C + DU - \underbrace{YX^{-1}}_\text{U}(A\underbrace{XX^{-1}}_\text{I} + B\underbrace{YX^{-1}}_\text{U}) \\
          & = C  + DU - UA - UBU
\end{align*}

was uns auf die Riccati-Matrix-Gleichung \ref{kra:riccati:riccatiequation} führt.
Die Lösung dieser Gleichung erhalten wir nach \cite{kra:kalmanisae} folgendermassen
\begin{equation}
    \begin{pmatrix}
        X(t) \\
        Y(t)
    \end{pmatrix}
    =
    \Phi(t_0, t)
    \begin{pmatrix}
        I(t) \\
        U_0(t)
    \end{pmatrix}
    =
    \begin{pmatrix}
        \Phi_{11}(t_0, t) & \Phi_{12}(t_0, t) \\
        \Phi_{21}(t_0, t) & \Phi_{22}(t_0, t)
    \end{pmatrix}
    \begin{pmatrix}
        I(t) \\
        U_0(t)
    \end{pmatrix}
\end{equation}

\begin{equation}
    U(t) =
    \begin{pmatrix}
        \Phi_{21}(t_0, t) + \Phi_{22}(t_0, t)
    \end{pmatrix}
    \begin{pmatrix}
        \Phi_{11}(t_0, t) + \Phi_{12}(t_0, t)
    \end{pmatrix}
    ^{-1}
\end{equation}

wobei $\Phi(t, t_0)$ die sogennante Zustandsübergangsmatrix ist.

\begin{equation}
    \Phi(t_0, t) = e^{H(t - t_0)}
\end{equation}



