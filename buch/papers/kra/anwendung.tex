\section{Anwendung \label{kra:section:anwendung}}
\rhead{Anwendung}
\newcommand{\dt}[0]{\frac{d}{dt}}

Die Matrix-Riccati Differentialgleichung findet unter anderem Anwendung in der Regelungstechnik beim RQ- und RQG-Regler oder aber auch beim Kalmanfilter.
Im folgenden Abschnitt möchten wir uns an einem Beispiel anschauen wie wir mit Hilfe der Matrix-Riccati Differentialgleichung (\ref{kra:equation:matrixriccati}) ein Feder-Masse-System untersuchen können \cite{kra:riccati}.

\subsection{Feder-Masse-System}
Die einfachste Form eines Feder-Masse-Systems ist dargestellt in Abbildung \ref{kra:fig:simple_mass_spring}.
Es besteht aus einer reibungsfrei gelagerten Masse $m$ ,welche an eine Feder mit der Federkonstante $k$ gekoppelt ist.
Die im System wirkenden Kräfte teilen sich auf in die auf dem hookeschen Gesetz basierenden Rückstellkraft $F_R = k \Delta_x$ und der auf dem Aktionsprinzip basierenden Kraft $F_a = am = \ddot{x} m$.
Das Kräftegleichgewicht fordert $F_R = F_a$ woraus folgt, dass

\begin{equation*}
    k \Delta_x = \ddot{x} m \Leftrightarrow \ddot{x} = \frac{k \Delta_x}{m}
\end{equation*}
Die Funktion die diese Differentialgleichung löst, ist die harmonische Schwingung
\begin{equation}
    x(t) = A \cos(\omega_0 t + \Phi), \quad \omega_0 = \sqrt{\frac{k}{m}}
\end{equation}
\begin{figure}
    % move image to standalone because the physics package is
    % incompatible with underbrace
    \includegraphics{papers/kra/images/simple.pdf}
    %% create tikz drawing of a simple mass spring system

\tikzstyle{hmline}=[{Latex[length=3.3,width=2.2]}-{Latex[length=3.3,width=2.2]},line width=0.3]
\tikzstyle{vmline}=[red, dashed,line width=0.4,dash pattern=on 1pt off 1pt]
\tikzstyle{ground}=[pattern=north east lines]
\tikzstyle{mass}=[line width=0.6,red!30!black,fill=red!40!black!10,rounded corners=1,top color=red!40!black!20,bottom color=red!40!black!10,shading angle=20]
\tikzstyle{spring}=[line width=0.8,blue!7!black!80,snake=coil,segment amplitude=5,line cap=round]

\begin{tikzpicture}[scale=2]
    \newcommand{\ticks}[2]
    {
        % arguments: x, y coordinates
        \draw[thick] (#1, #2 - 0.1 / 2) --++ (0, 0.1);
    }

    \tikzmath{
        \hWall = 1.5;
        \wWall = 0.3;
        \lWall = 3.5;
        \hMass = 0.6;
        \wMass = 1.1;
        \xMass1 = 1.2;
        \xMass2 = 2.2;
        \xAxisYpos = 0;
        \originX1 = 0;
        \originY1 = 0.5;
        \originX2 = 0;
        \originY2 = -2;
        \springscale=7;
    }

    % create x axis
    \draw[->,thick] (0,\xAxisYpos) --+ (\lWall, 0) node[right]{$x$};

    % create ticks on x axis
    \ticks{\wWall}{\xAxisYpos}
    \ticks{\xMass1}{\xAxisYpos}
    \ticks{\xMass2}{\xAxisYpos}

    % create underground
    \draw[ground] (\originX1, \originY1) ++ (0, 0) --+(\lWall,0) --+(\lWall, \wWall) --+(\wWall, \wWall) --+(\wWall, \hWall) --+(0, \hWall) -- cycle;
    \draw[ground] (\originX2, \originY2) ++ (0, 0) --+(\lWall,0) --+(\lWall, \wWall) --+(\wWall, \wWall) --+(\wWall, \hWall) --+(0, \hWall) -- cycle;

    % create masses
    \draw[mass] (\originX1, \originY1) ++ (\xMass1, \wWall) rectangle ++ (\wMass,\hMass) node[midway] {$m$};
    \draw[mass] (\originX2, \originY2) ++ (\xMass2, \wWall) rectangle ++ (\wMass,\hMass) node[midway] {$m$};

    % create springs
    \draw[spring, segment length=(\xMass1 - \wWall) * \springscale] (\originX1, \originY1) ++ (\wWall, \wWall + \hMass / 2) --+ (\xMass1 - \wWall, 0);
    \draw[spring, segment length=(\xMass2 - \wWall) * \springscale] (\originX2, \originY2) ++ (\wWall, \wWall + \hMass / 2) --+ (\xMass2 - \wWall, 0);

    % create vertical measurement line 
    \draw[vmline] (\xMass1, \xAxisYpos) --+(0, \originY1 + \wWall);
    \draw[vmline] (\xMass2, \xAxisYpos) --+(0, \originY2 + \hMass+\wWall);
    \draw[vmline] (\wWall, \originY1+\wWall) --(\wWall, \originY2 + \hWall);

    % create horizontal measurement line
    \draw[hmline] (\wWall, \xAxisYpos + 0.2) -- (\xMass1, \xAxisYpos + 0.2) node[midway,fill=white,inner sep=0] {$\ell_0$};
    \draw[hmline] (\xMass1, \xAxisYpos + 0.2) -- (\xMass2, \xAxisYpos + 0.2) node[midway,fill=white,inner sep=0] {$\Delta_{x}$};
    \draw[hmline] (\wWall, \xAxisYpos - 0.3) -- (\xMass2, \xAxisYpos - 0.3) node[midway,fill=white,inner sep=0] {$\ell_{1}$};

    %  create force arrow
    \draw[->,blue, very thick,line cap=round] (\xMass2 + \wMass / 2, \originY2 + \wWall + \hMass + 0.15) node[above] {$\vb{F_{R}}$} --+ (-0.5, 0);
\end{tikzpicture}
    \caption{Einfaches Feder-Masse-System.}
    \label{kra:fig:simple_mass_spring}
\end{figure}
\begin{figure}
    % create tikz drawing of a multi mass multi spring system

\tikzstyle{vmline}=[red, dashed,line width=0.4,dash pattern=on 1pt off 1pt]
\tikzstyle{ground}=[pattern=north east lines]
\tikzstyle{mass}=[line width=0.6,red!30!black,fill=red!40!black!10,rounded corners=1,top color=red!40!black!20,bottom color=red!40!black!10,shading angle=20]
\tikzstyle{spring}=[line width=0.8,blue!7!black!80,snake=coil,segment amplitude=5,line cap=round]

\begin{tikzpicture}[scale=2, >=latex]
    \newcommand{\ticks}[3]
    {
        % x, y coordinates
        \draw[thick] (#1, #2 - 0.1 / 2) --++ (0, 0.1) node[scale=0.8,below=0.2] {#3};
    }
    \tikzmath{
        \hWall = 1.2;
        \wWall = 0.3;
        \lWall = 5;
        \hMass = 0.6;
        \wMass = 1.1;
        \xMass1 = 1.0;
        \xMass2 = 3.0;
        \xAxisYpos = 0;
        \originX1 = 0;
        \originY1 = 0.5;
        \springscale=7;
    }

    % create axis
    \draw[->,thick] (0,\xAxisYpos) --+ (\xMass2 + \wMass, 0) node[right]{$q$};
    % create ticks on x / q axis
    \ticks{\xMass1}{\xAxisYpos}{$q_{1}$}
    \ticks{\xMass2}{\xAxisYpos}{$q_{2}$}

    % create non-moving backgrounds
    \draw[ground] (\originX1, \originY1) ++ (0, 0) --+(\lWall,0) --+(\lWall, \hWall)
    --+ (\lWall - \wWall, \hWall) --+(\lWall - \wWall, \wWall) --+ (\wWall, \wWall)  --+(\wWall, \hWall) --+(0, \hWall) -- cycle;

    % create masses
    \draw[mass] (\originX1, \originY1) ++ (\xMass1, \wWall) rectangle ++ (\wMass,\hMass) node[midway] {$m_{1}$};
    \draw[mass] (\originX1, \originY1) ++ (\xMass2, \wWall) rectangle ++ (\wMass,\hMass) node[midway] {$m_{2}$};

    % create springs
    \draw[spring, segment length=(\xMass1 - \wWall) * \springscale] (\originX1, \originY1) ++
    (\wWall, \wWall + \hMass / 2) --++ (\xMass1 - \wWall, 0) node[midway,above=0.2] {$k_1$};
    \draw[spring, segment length=(\xMass1 - \wWall) * \springscale] (\originX1, \originY1) ++
    (\xMass1 + \wMass, \wWall + \hMass / 2) --++ (\xMass2 - \xMass1 - \wMass, 0) node[midway,above=0.2] {$k_c$};
    \draw[spring, segment length=(\xMass1 - \wWall) * \springscale] (\originX1, \originY1) ++
    (\xMass2 + \wMass, \wWall + \hMass / 2) --++ (\lWall - \xMass2 - \wMass - \wWall, 0) node[midway,above=0.2] {$k_2$};

    % create vertical measurement line 
    \draw[vmline] (\xMass1, \xAxisYpos) --+(0, \originY1 + \wWall);
    \draw[vmline] (\xMass2, \xAxisYpos) --+(0, \originY1 + \wWall);

\end{tikzpicture}

    \caption{Feder-Masse-System mit zwei Massen und drei Federn.}
    \label{kra:fig:multi_mass_spring}
\end{figure}

\subsection{Hamilton-Funktion}
Die Bewegung der Masse $m$ kann mit Hilfe der hamiltonschen Mechanik im Phasenraum untersucht werden.
Die hamiltonschen Gleichungen verwenden dafür die verallgemeinerten Ortskoordinaten
$q = (q_{1}, q_{2}, ..., q_{n})$ und die verallgemeinerten Impulskoordinaten $p = (p_{1}, p_{2}, ..., p_{n})$, wobei der Impuls definiert ist als $p_k = m_k \cdot v_k$.
Liegen keine zeitabhängigen Zwangsbedingungen vor, so entspricht die Hamitlon-Funktion der Gesamtenergie des Systems \cite{kra:hamilton}.
Im Falle des einfachen Feder-Masse-Systems, Abbildung \ref{kra:fig:simple_mass_spring}, setzt sich die Hamilton-Funktion aus kinetischer und potentieller Energie zusammen.
\begin{equation}
    \label{kra:harmonischer_oszillator}
    \begin{split}
        \mathcal{H}(q, p) &= T(p) + V(q) = E \\
        &= \underbrace{\frac{p^2}{2m}}_{E_{kin}} + \underbrace{\frac{k q^2}{2}}_{E_{pot}}
    \end{split}
\end{equation}
Die Hamiltonschen Bewegungsgleichungen liefern \cite{kra:kanonischegleichungen}
\begin{equation}
    \label{kra:hamilton:bewegungsgleichung}
    \dot{q_{k}} = \frac{\partial \mathcal{H}}{\partial p_k}
    \qquad
    \dot{p_{k}} = -\frac{\partial \mathcal{H}}{\partial q_k}
\end{equation}
daraus folgt
\[
    \dot{q} = \frac{p}{m}
    \qquad
    \dot{p} = -kq
\]
in Matrixschreibweise erhalten wir also
\[
    \begin{pmatrix}
        \dot{q} \\
        \dot{p}
    \end{pmatrix}
    =
    \begin{pmatrix}
        0  & \frac{1}{m} \\
        -k & 0
    \end{pmatrix}
    \begin{pmatrix}
        q \\
        p
    \end{pmatrix}
\]
Für das erweiterte Federmassesystem, Abbildung \ref{kra:fig:multi_mass_spring}, können wir analog vorgehen.
Die kinetische Energie setzt sich nun aus den kinetischen Energien der einzelnen Massen $m_1$ und $m_2$ zusammen.
Die Potentielle Energie erhalten wir aus der Summe der kinetischen Energien der einzelnen Federn mit den Federkonstanten $k_1$, $k_c$ und $k_2$.
\begin{align*}
    \begin{split}
        T   &= T_1 + T_2 \\
        &= \frac{p_1^2}{2m_1} + \frac{p_2^2}{2m_2}
    \end{split}
    \\
    \begin{split}
        V   &= V_1 + V_c + V_2 \\
        &= \frac{k_1 q_1^2}{2} + \frac{k_c (q_2 - q_1)^2}{2} + \frac{k_2 q_2^2}{2}
    \end{split}
\end{align*}
Die Hamilton-Funktion ist also
\begin{align*}
    \begin{split}
        \mathcal{H}     &= T + V \\
        &= \frac{p_1^2}{2m_1} + \frac{p_2^2}{2m_2} + \frac{k_1 q_1^2}{2} + \frac{k_c (q_2 - q_1)^2}{2} + \frac{k_2 q_2^2}{2}
    \end{split}
\end{align*}
Die Bewegungsgleichungen \ref{kra:hamilton:bewegungsgleichung} liefern
\begin{align*}
    \frac{\partial \mathcal{H}}{\partial p_k}  & = \dot{q_k}
    \Rightarrow
    \left\{
    \begin{alignedat}{2}
        \dot{q_1}   &= \frac{2p_1}{2m_1}    &&= \frac{p_1}{m_1}\\
        \dot{q_2}   &= \frac{2p_2}{2m_2}    &&= \frac{p_2}{m_2}
    \end{alignedat}
    \right.
    \\
    -\frac{\partial \mathcal{H}}{\partial q_k} & = \dot{p_k}
    \Rightarrow
    \left\{
    \begin{alignedat}{2}
        \dot{p_1}   &= -(\frac{2k_1q_1}{2} - \frac{2k_c(q_2-q_1)}{2})  &&= -q_1(k_1+k_c) + q_2k_c \\
        \dot{p_1}   &= -(\frac{2k_c(q_2-q_1)}{2} - \frac{2k_2q_2}{2})  &&= q_1k_c - (k_c + k_2)
    \end{alignedat}
    \right.
\end{align*}
In Matrixschreibweise erhalten wir
\begin{equation}
    \label{kra:hamilton:multispringmass}
    \begin{pmatrix}
        \dot{q_1} \\
        \dot{q_2} \\
        \dot{p_1} \\
        \dot{p_2} \\
    \end{pmatrix}
    =
    \begin{pmatrix}
        0            & 0            & \frac{1}{2m_1} & 0              \\
        0            & 0            & 0              & \frac{1}{2m_2} \\
        -(k_1 + k_c) & k_c          & 0              & 0              \\
        k_c          & -(k_c + k_2) & 0              & 0              \\
    \end{pmatrix}
    \begin{pmatrix}
        q_1 \\
        q_2 \\
        p_1 \\
        p_2 \\
    \end{pmatrix}
    \Leftrightarrow
    \dt
    \begin{pmatrix}
        Q \\
        P \\
    \end{pmatrix}
    =
    \underbrace{
        \begin{pmatrix}
            0 & M \\
            K & 0
        \end{pmatrix}
    }_{G}
    \begin{pmatrix}
        Q \\
        P \\
    \end{pmatrix}
\end{equation}

\subsection{Phasenraum}
Der Phasenraum erlaubt die eindeutige Beschreibung aller möglichen Bewegungszustände eines mechanischen Systems durch einen Punkt.
Die Phasenraumdarstellung eignet sich somit sehr gut für die systematische Untersuchung der Feder-Masse-Systeme.

\subsubsection{Harmonischer Oszillator}
Die Hamiltonfunktion des harmonischen Oszillators \ref{kra:harmonischer_oszillator} führt auf eine Lösung der Form
\begin{equation*}
    q(t) = A \cos(\omega_0 T + \Phi), \quad p(t) = -m \omega_0 A \sin(\omega_0 t + \Phi)
\end{equation*}
die Phasenraumtrajektorien bilden also Ellipsen mit Zentrum $q=0, p=0$ und Halbachsen $A$ und $m \omega A$.
Abbildung \ref{kra:fig:phasenraum} zeigt Phasenraumtrajektorien mit den Energien $E_{x \in \{A, B, C, D\}}$ und verschiedenen Werten von $\omega$.
\begin{figure}
    \colorlet{mypurple}{red!50!blue!90!black!80}

% style to create arrows
\tikzset{
    traj/.style 2 args={thick, postaction={decorate},decoration={markings,
                    mark=at position #1 with {\arrow{<}},
                    mark=at position #2 with {\arrow{<}}}
        }
}

\begin{tikzpicture}[scale=0.6, >=latex]
    % p(t=0) = 0, q(t=0) = A, max(p) = mwA
    \tikzmath{
        \axh = 5.2;
        \axw1 = 4.2;
        \axw2 = 4.8;
        \d1 = 0.9;
        \a0 = 1;
        \b0 = 2;
        \a1 = \a0 + \d1;
        \b1 = \b0 + \d1;
        \a2 = \a1 + \d1;
        \b2 = \b1 + \d1;
        \a3 = \a2 + \d1;
        \b3 = \b2 + \d1;
        \d2 = 0.75;
        \aa0 = 2;
        \bb0 = 1;
        \aa1 = \aa0 + \d2;
        \bb1 = \bb0 + \d2;
        \aa2 = \aa1 + \d2;
        \bb2 = \bb1 + \d2;
        \aa3 = \aa2 + \d2;
        \bb3 = \bb2 + \d2;
    }

    \draw[->,thick] (-\axw1,0) -- (\axw1,0) node[right] {$q$};
    \draw[->,thick] (0,-\axh) -- (0,\axh) node[above] {$p$};

    \draw[traj={0.375}{0.875},darkgreen] ellipse (\a0 and \b0);
    \draw[traj={0.375}{0.875},blue] ellipse (\a1 and \b1);
    \draw[traj={0.375}{0.875},cyan] ellipse (\a2 and \b2);
    \draw[traj={0.375}{0.875},mypurple] ellipse (\a3 and \b3);

    \node[right,darkgreen] at (45:{\a0} and {\b0}) {$E_A$};
    \node[right, blue] at (45:{\a1} and {\b1}) {$E_B$};
    \node[right, cyan] at (45:{\a2} and {\b2}) {$E_C$};
    \node[right, mypurple] at (45:{\a3} and {\b3}) {$E_D$};
    \node[above left] at (110:\b3 + 0.1) {grosses $\omega$};

    \begin{scope}[xshift=12cm]
        \draw[->,thick] (-\axw2,0) -- (\axw2,0) node[right] {$q$};
        \draw[->,thick] (0,-\axh) -- (0,\axh) node[above] {$p$};

        \draw[traj={0.375}{0.875},darkgreen] ellipse (\aa0 and \bb0);
        \draw[traj={0.375}{0.875},blue] ellipse (\aa1 and \bb1);
        \draw[traj={0.375}{0.875},cyan] ellipse (\aa2 and \bb2);
        \draw[traj={0.375}{0.875},mypurple] ellipse (\aa3 and \bb3);

        \node[above, darkgreen] at (45:{\aa0} and {\bb0}) {$E_A$};
        \node[above, blue] at (45:{\aa1} and {\bb1}) {$E_B$};
        \node[above, cyan] at (45:{\aa2} and {\bb2}) {$E_C$};
        \node[above, mypurple] at (45:{\aa3} and {\bb3}) {$E_D$};

        \node[above left] at (110:\b3 + 0.1) {kleines $\omega$};
    \end{scope}
\end{tikzpicture}
    \caption{Phasenraumdarstellung des einfachen Feder-Masse-Systems.}
    \label{kra:fig:phasenraum}
\end{figure}

\subsubsection{Erweitertes Feder-Masse-System}
Wir intressieren uns nun dafür wie der Phasenwinkel $U = PQ^{-1}$ von der Zeit abhängt,
wir suchen also die Grösse $\Theta = \dt U$.
Ersetzten wir in der Gleichung \ref{kra:hamilton:multispringmass} die Matrix $G$ mit $\tilde{G}$ so erhalten wir
\begin{equation}
    \dt
    \begin{pmatrix}
        Q \\
        P
    \end{pmatrix}
    =
    \underbrace{
        \begin{pmatrix}
            A & B \\
            C & D
        \end{pmatrix}
    }_{\tilde{G}}
    \begin{pmatrix}
        Q \\
        P
    \end{pmatrix}
\end{equation}
Mit einsetzten folgt
\begin{align*}
    \dot{Q} = AQ + BP \\
    \dot{P} = CQ + DP
\end{align*}
\begin{equation}
    \begin{split}
        \dt U   &= \dot{P} Q^{-1} + P \dt Q^{-1} \\
        &= (CQ + DP) Q^{-1} - P (Q^{-1} \dot{Q} Q^{-1}) \\
        &= C\underbrace{QQ^{-1}}_\text{I} + D\underbrace{PQ^{-1}}_\text{U} - P(Q^{-1} (AQ + BP) Q^{-1}) \\
        &= C + DU - \underbrace{PQ^{-1}}_\text{U}(A\underbrace{QQ^{-1}}_\text{I} + B\underbrace{PQ^{-1}}_\text{U}) \\
        &= C  + DU - UA - UBU
    \end{split}
\end{equation}
was uns auf die Matrix-Riccati Gleichung \ref{kra:equation:matrixriccati} führt.

% @TODO Einfluss auf anfangsbedingungen, plots?
% @TODO Fazit ?
