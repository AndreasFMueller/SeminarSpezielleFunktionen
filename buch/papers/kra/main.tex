%
% main.tex -- Paper zum Thema <kra>
%
% (c) 2020 Hochschule Rapperswil
%
\chapter{Kalman, Riccati und Abel\label{chapter:kra}}
\lhead{Kalman, Riccati und Abel}
\begin{refsection}
    \chapterauthor{Samuel Niederer}

    Ein paar Hinweise für die korrekte Formatierung des Textes
    \begin{itemize}
        \item
              Absätze werden gebildet, indem man eine Leerzeile einfügt.
              Die Verwendung von \verb+\\+ ist nur in Tabellen und Arrays gestattet.
        \item
              Die explizite Platzierung von Bildern ist nicht erlaubt, entsprechende
              Optionen werden gelöscht.
              Verwenden Sie Labels und Verweise, um auf Bilder hinzuweisen.
        \item
              Beginnen Sie jeden Satz auf einer neuen Zeile.
              Damit ermöglichen Sie dem Versionsverwaltungssysteme, Änderungen
              in verschiedenen Sätzen von verschiedenen Autoren ohne Konflikt
              anzuwenden.
        \item
              Bilden Sie auch für Formeln kurze Zeilen, einerseits der besseren
              Übersicht wegen, aber auch um GIT die Arbeit zu erleichtern.
    \end{itemize}

    %
% einleitung.tex -- Beispiel-File für die Einleitung
%
% (c) 2020 Prof Dr Andreas Müller, Hochschule Rapperswil
%
\section{Teil 0\label{fresnel:section:teil0}}
\rhead{Teil 0}
Lorem ipsum dolor sit amet, consetetur sadipscing elitr, sed diam
nonumy eirmod tempor invidunt ut labore et dolore magna aliquyam
erat, sed diam voluptua \cite{fresnel:bibtex}.
At vero eos et accusam et justo duo dolores et ea rebum.
Stet clita kasd gubergren, no sea takimata sanctus est Lorem ipsum
dolor sit amet.

Lorem ipsum dolor sit amet, consetetur sadipscing elitr, sed diam
nonumy eirmod tempor invidunt ut labore et dolore magna aliquyam
erat, sed diam voluptua.
At vero eos et accusam et justo duo dolores et ea rebum.  Stet clita
kasd gubergren, no sea takimata sanctus est Lorem ipsum dolor sit
amet.



    %
% teil1.tex -- Beispiel-File für das Paper
%
% (c) 2020 Prof Dr Andreas Müller, Hochschule Rapperswil
%
\section{Lösung
\label{parzyl:section:teil1}}
\rhead{Problemstellung}
Die Differentialgleichungen \eqref{parzyl:sep_dgl_1} und \eqref{parzyl:sep_dgl_2} können mit
Hilfe der Whittaker Gleichung gelöst werden.
\begin{definition}
    Die Funktion 
    \begin{equation*}
        W_{k,m}(z) = 
    e^{-z/2} z^{m+1/2} \,
    {}_{1} F_{1}
    (
        {\textstyle \frac{1}{2}} 
        + m - k, 1 + 2m; z)
    \end{equation*}
    heisst Whittaker Funktion und ist eine Lösung
    von der Whittaker Differentialgleichung
    \begin{equation}
        \frac{d^2W}{d z^2} +
        \left(-\frac{1}{4}  + \frac{k}{z} + \frac{\frac{1}{4} - m^2}{z^2} \right) W = 0.
        \label{parzyl:eq:whitDiffEq}
    \end{equation}
\end{definition}
Es wird nun die Differentialgleichung bestimmt, welche
\begin{equation}
    w = z^{-1/2} W_{k,-1/4} \left({\textstyle \frac{1}{2}} z^2\right)
\end{equation}
als Lösung hat.
Dafür wird $w$ in \eqref{parzyl:eq:whitDiffEq} eingesetzt woraus
\begin{equation}
    \frac{d^2 w}{dz^2} - \left(\frac{1}{4} z^2 - 2k\right) w = 0
\label{parzyl:eq:weberDiffEq}
\end{equation}
resultiert. DIese Differentialgleichung ist dieselbe wie 
\eqref{parzyl:sep_dgl_2} und \eqref{parzyl:sep_dgl_2}, welche somit
$w$ als Lösung haben.
Da es sich um eine Differentialgleichung zweiter Ordnung handelt, hat sie nicht nur
eine sondern zwei Lösungen.
Die zweite Lösung der Whittaker-Gleichung ist $W_{k,-m} (z)$.
Somit hat \eqref{parzyl:eq:weberDiffEq}
\begin{align}
    w_1 & = z^{-1/2} W_{k,-1/4} \left({\textstyle \frac{1}{2}} z^2\right)\\
    w_2 & = z^{-1/2} W_{k,1/4} \left({\textstyle \frac{1}{2}} z^2\right)
\end{align}
als Lösungen.

Ausgeschrieben ergeben sich als Lösungen
\begin{align}
    w_1 &= e^{-z^2/4} \,
    {}_{1} F_{1}
    (
        {\textstyle \frac{1}{4}} 
         - k, {\textstyle \frac{1}{2}} ; {\textstyle \frac{1}{2}}z^2) \\
    w_2 & = z e^{-z^2/4} \,
         {}_{1} F_{1}
         ({\textstyle \frac{3}{4}} 
              - k, {\textstyle \frac{3}{2}} ; {\textstyle \frac{1}{2}}z^2)
\end{align}
    %
% teil2.tex -- Umsetzung in C Programmen
%
% (c) 2022 Fabian Dünki, Hochschule Rapperswil
%
\section{Umsetzung
\label{0f1:section:teil2}}
\rhead{Umsetzung}
Zur Umsetzung wurden drei verschiedene Ansätze gewählt, die in
vollständiger Form auf Github \cite{0f1:code} zu finden sind.
Dabei wurde der Schwerpunkt auf die Funktionalität und eine gute
Lesbarkeit des Codes gelegt.
Die Unterprogramme wurde jeweils, wie die GNU Scientific Library,
\index{GNU Scientific Library}%
in C geschrieben.
Die Zwischenresultate wurden vom Hauptprogramm
in einem CSV-File gespeichert.
\index{CSV}%
Anschliessen wurde mit der Matplot-Library
\index{Matplot-Library}%
\index{Python}%
in Python die Resultate geplottet.

\subsection{Potenzreihe
\label{0f1:subsection:potenzreihe}}
Die naheliegendste Lösung ist die Programmierung der Potenzreihe
\begin{align}
    \label{0f1:umsetzung:0f1:eq}
    \mathstrut_0F_1(;c;z)
    &=
    \sum_{k=0}^\infty
    \frac{z^k}{(c)_k \cdot k!}
    &= 
    \frac{1}{c}
    +\frac{z^1}{(c+1) \cdot 1}
    + \cdots
    + \frac{z^{20}}{c(c+1)(c+2)\cdots(c+19) \cdot 2.4 \cdot 10^{18}}.
\end{align}

\lstinputlisting[style=C,float,caption={Potenzreihe.},label={0f1:listing:potenzreihe}, firstline=59]{papers/0f1/listings/potenzreihe.c}

\subsection{Kettenbruch
\label{0f1:subsection:kettenbruch}}
Eine weitere Variante zur Berechnung von $\mathstrut_0F_1(;c;z)$ ist die Umsetzung als Kettenbruch.
\index{Kettenbruch}
Der Vorteil einer Umsetzung als Kettenbruch gegenüber der Potenzreihe ist die schnellere Konvergenz.

\subsubsection{Grundlage}
Ein endlicher Kettenbruch \cite{0f1:wiki-kettenbruch} ist ein Bruch der Form
\begin{equation*}
a_0 + \cfrac{b_1}{a_1+\cfrac{b_2}{a_2+\cfrac{b_3}{a_3+\cdots}}},
\end{equation*}
in welchem $a_0, a_1,\dots,a_n$ und $b_1,b_2,\dots,b_n$ ganze Zahlen sind.

\subsubsection{Rekursionsbeziehungen und Kettenbrüche}

Nimmt man die Gleichung
Wenn es für die analytischen Funktionen $f_i(z)$ eine Relation der Form
\begin{equation}
	f_{i-1}(z) - f_i(z) = k_i z f_{i+1}(z)
\label{0f1:relation}
\end{equation}
für ganzzahlige positive $i$ und Konstanten $k_i$
gibt,
dann gibt es einen Kettenbruch für das Verhältnis
$\frac{f_i(z)}{f_{i-1}(z)}$ \cite{0f1:wiki-fraction}. 
Aus der Relation~\eqref{0f1:relation}
ergibt sich der Zusammenhang
\begin{equation}
	\cfrac{f_i(z)}{f_{i-1}(z)}
	=
	\cfrac{1}{1+k_iz\cfrac{f_{i+1}(z)}{f_i(z)}}.
\label{0f1:bruchrelation}
\end{equation}
Geht man einen Schritt weiter und nimmt für
$g_i(z) = \frac{f_i(z)}{f_{i-1}(z)}$ an, kommt man zur Formel
\begin{equation*}
	g_i(z) = \cfrac{1}{1+k_izg_{i+1}(z)}.
\end{equation*}
Setzt man dies nun für $g_1$ in den Bruch ein, ergibt sich
\begin{equation*}
	g_1(z) = \cfrac{f_1(z)}{f_0(z)} = \cfrac{1}{1+k_izg_2(z)} = \cfrac{1}{1+\cfrac{k_1z}{1+k_2zg_3(z)}} = \cdots
\end{equation*}
Wiederholt man dies unendlich, erhält man einen Kettenbruch in der Form:
\begin{equation}
	\label{0f1:math:rekursion:eq}
	\cfrac{f_1(z)}{f_0(z)}
	=
	\cfrac{1}{1+\cfrac{k_1z}{1+\cfrac{k_2z}{1+\cfrac{k_3z}{\cdots}}}}.
\end{equation}

\subsubsection{Rekursion für $\mathstrut_0F_1$}
Angewendet auf die Potenzreihe
\begin{equation}
	\label{0f1:math:potenzreihe:0f1:eq}
	\mathstrut_0F_1(;c;z)
	=
	1 + \frac{z}{c\cdot1!} + \frac{z^2}{c(c+1)\cdot2!} + \frac{z^3}{c(c+1)(c+2)\cdot3!} + \cdots
\end{equation}
kann durch Substitution bewiesen werden, dass
\begin{equation*}
	\mathstrut_0F_1(;c-1;z) - \mathstrut_0F_1(;c;z)
	=
	\frac{z}{c(c-1)} \cdot \mathstrut_0F_1(;c+1;z)
\end{equation*}
eine Relation der Art \eqref{0f1:relation} dazu ist.
Wenn man für $f_i$ und $k_i$ die Annahme
\begin{align*}
	f_i &= \mathstrut_0F_1(;c+i;z)\\
	k_i	&= \frac{1}{(c+i)(c+i-1)}
\end{align*}
trifft und in die Formel \eqref{0f1:math:rekursion:eq} einsetzt, erhält man:
\begin{equation*}
	\cfrac{\mathstrut_0F_1(;c+1;z)}{\mathstrut_0F_1(;c;z)}
	=
	\cfrac{1}{1+\cfrac{\cfrac{z}{c(c+1)}}{1+\cfrac{\cfrac{z}{(c+1)(c+2)}}{1+\cfrac{\cfrac{z}{(c+2)(c+3)}}{\cdots}}}}.
\end{equation*}

\subsubsection{Algorithmus}
Da mit obigen Formeln nur ein Verhältnis zwischen
$\frac{\mathstrut_0F_1(;c+1;z)}{\mathstrut_0F_1(;c;z)}$
berechnet wurde, braucht es weitere Relationen um $\mathstrut_0F_1(;c;z)$
zu erhalten.
So ergeben ähnliche Relationen nach Wolfram Alpha \cite{0f1:wolfram-0f1} den Kettenbruch
\begin{equation}
	\label{0f1:math:kettenbruch:0f1:eq}
	\mathstrut_0F_1(;c;z) = 1 + \cfrac{\cfrac{z}{c}}{1+\cfrac{-\cfrac{z}{2(c+1)}}{1+\cfrac{z}{2(c+1)}+\cfrac{-\cfrac{z}{3(c+2)}}{1+\cfrac{z}{5(c+4)} + \cdots}}},
\end{equation}
der als Code (Listing \ref{0f1:listing:kettenbruchIterativ})  umgesetzt wurde. 

\lstinputlisting[style=C,float,caption={Iterativ umgesetzter Kettenbruch.},label={0f1:listing:kettenbruchIterativ},  firstline=8]{papers/0f1/listings/kettenbruchIterativ.c}

\subsection{Rekursionsformel
\label{0f1:subsection:rekursionsformel}}
Wesentlich stabiler zur Berechnung eines Kettenbruches ist die
Rekursionsformel.
\index{Rekursionsformel}%
Nachfolgend wird die verkürzte Herleitung vom
Kettenbruch zur Rekursionsformel aufgezeigt.
Eine vollständige Schritt für Schritt Herleitung ist im Seminarbuch Numerik
\cite{0f1:kettenbrueche}
im Kapitel Kettenbrüche zu finden.

\subsubsection{Herleitung}
Ein Näherungsbruch in der Form
\index{Näherungsbruch}%
\begin{align*}
	\cfrac{A_k}{B_k} = a_k + \cfrac{b_{k + 1}}{a_{k + 1} + \cfrac{p}{q}}
\end{align*}
lässt sich zu
\begin{align*}
	\cfrac{A_k}{B_k} = \cfrac{b_{k+1}}{a_{k+1} + \cfrac{p}{q}} = \frac{b_{k+1} \cdot q}{a_{k+1} \cdot q + p}
\end{align*}
umformen.
Dies lässt sich auch durch die Matrizenschreibweise
\index{Matrixschreibeweise eines Kettenbruchs}%
\begin{equation*}
	\begin{pmatrix}
		A_k\\
		B_k
	\end{pmatrix}
	= 		\begin{pmatrix}
		b_{k+1} \cdot q\\
		a_{k+1} \cdot q + p
	\end{pmatrix}
	=\begin{pmatrix}
		0&	b_{k+1}\\
		1&	a_{k+1}
	\end{pmatrix}
	\begin{pmatrix}
		p \\
		q
	\end{pmatrix}
	%\label{0f1:math:rekursionsformel:herleitung}
\end{equation*}
ausdrücken.
Wendet man dies nun auf den Kettenbruch in der Form
\begin{equation*}
	\frac{A_k}{B_k} = a_0 + \cfrac{b_1}{a_1+\cfrac{b_2}{a_2+\cfrac{\cdots}{\cdots+\cfrac{b_{k-1}}{a_{k-1} + \cfrac{b_k}{a_k}}}}}
\end{equation*}
an, ergibt sich die Matrixdarstellungen:
\begin{align*}
	\begin{pmatrix}
		A_k\\
		B_k
	\end{pmatrix}
	&=
	\begin{pmatrix}
		1& a_0\\
		0& 1
	\end{pmatrix}
	\begin{pmatrix}
		0& b_1\\
		1& a_1
	\end{pmatrix}
	\cdots
	\begin{pmatrix}
		0& b_{k-1}\\
		1& a_{k-1}
	\end{pmatrix}
	\begin{pmatrix}
		b_k\\
		a_k
	\end{pmatrix}.
\end{align*}
Nach vollständiger Induktion ergibt sich für den Schritt $k$, die Matrix
\begin{equation}
	\label{0f1:math:matrix:ende:eq}
	 \begin{pmatrix}
		A_{k}\\
		B_{k}			
	\end{pmatrix} 
	=
		\begin{pmatrix}
		A_{k-2}& A_{k-1}\\
		B_{k-2}& B_{k-1}			
	\end{pmatrix}
		\begin{pmatrix}
		b_k\\
		a_k
	\end{pmatrix}.
\end{equation}
Und schlussendlich kann der Näherungsbruch
\[
\frac{A_k}{B_k}
\] 
berechnet werden.

\subsubsection{Algorithmus}
Die Berechnung von $A_k, B_k$ gemäss \eqref{0f1:math:matrix:ende:eq} kann man auch ohne die Matrizenschreibweise \cite{0f1:kettenbrueche} aufschreiben:
\begin{itemize}
\item Startbedingungen:
\begin{align*}
A_{-1} &= 0		&		A_0 &= a_0 \\
B_{-1} &= 1		&		B_0 &= 1 
\end{align*}
\item Schritt $k\to k+1$:
\[
\begin{aligned}
\label{0f1:math:loesung:eq}
A_{k+1} &= A_{k-1} \cdot b_k + A_k \cdot a_k \\
B_{k+1} &= B_{k-1} \cdot b_k + B_k \cdot a_k
\end{aligned}
\]
\item
Näherungsbruch: \qquad$\displaystyle\frac{A_k}{B_k}$.
\end{itemize}
Ein grosser Vorteil dieser Umsetzung
als Rekursionsformel
(Listing~\ref{0f1:listing:kettenbruchRekursion}) ist,
dass im Vergleich zum Code (Listing~\ref{0f1:listing:kettenbruchIterativ})
eine Division gespart werden kann und somit weniger Rundungsfehler
entstehen können.

%Code
\lstinputlisting[style=C,float,caption={Rekursionsformel für Kettenbruch.},label={0f1:listing:kettenbruchRekursion},  firstline=8]{papers/0f1/listings/kettenbruchRekursion.c}

    %
% teil3.tex -- Beispiel-File für Teil 3
%
% (c) 2020 Prof Dr Andreas Müller, Hochschule Rapperswil
%
\section{Eigenschaften
\label{parzyl:section:Eigenschaften}}
\rhead{Eigenschaften}

\subsection{Potenzreihenentwicklung
	\label{parzyl:potenz}}
Die parabolischen Zylinderfunktionen, welche in Gleichung \ref{parzyl:eq:solution_dgl} gegeben sind, können auch als Potenzreihen geschrieben werden
\begin{align}
	w_1(k,z)
	&=  
	e^{-z^2/4} \,
	{}_{1} F_{1}
	(
	{\textstyle \frac{1}{4}} 
	- k, {\textstyle \frac{1}{2}} ; {\textstyle \frac{1}{2}}z^2) 
	= 
	e^{-\frac{z^2}{4}}
	\sum^{\infty}_{n=0}
	\frac{\left ( \frac{1}{4} - k \right )_{n}}{\left ( \frac{1}{2}\right )_{n}}
	\frac{\left ( \frac{1}{2} z^2\right )^n}{n!} \\
	&=
	e^{-\frac{z^2}{4}}
	\left ( 
	1 
	+
	\left ( \frac{1}{2} - 2k \right )\frac{z^2}{2!}
	+
	\left ( \frac{1}{2} - 2k \right )\left ( \frac{5}{2} - 2k \right )\frac{z^4}{4!}  
	+
	\dots
	\right )
\end{align}
und
\begin{align}
	w_2(k,z)
	&=  
	ze^{-z^2/4} \,
	{}_{1} F_{1}
	(
	{\textstyle \frac{3}{4}} 
	- k, {\textstyle \frac{3}{2}} ; {\textstyle \frac{1}{2}}z^2) 
	= 
	ze^{-\frac{z^2}{4}}
	\sum^{\infty}_{n=0}
	\frac{\left ( \frac{3}{4} - k \right )_{n}}{\left ( \frac{3}{2}\right )_{n}}
	\frac{\left ( \frac{1}{2} z^2\right )^n}{n!} \\
	&=
	e^{-\frac{z^2}{4}}
	\left ( 
	z 
	+
	\left ( \frac{3}{2} - 2k \right )\frac{z^3}{3!}
	+
	\left ( \frac{3}{2} - 2k \right )\left ( \frac{7}{2} - 2k \right )\frac{z^5}{5!}  
	+
	\dots
	\right ).
\end{align}
Bei den Potenzreihen sieht man gut, dass die Ordnung des Polynoms im generellen ins unendliche geht. Es gibt allerdings die Möglichkeit für bestimmte k das die Terme in der Klammer gleich null werden und das Polynom somit eine endliche Ordnung $n$ hat. Dies geschieht bei $w_1(k,z)$ falls
\begin{equation}
	k = \frac{1}{4} + n \qquad n \in \mathbb{N}_0
\end{equation}
und bei $w_2(k,z)$ falls
\begin{equation}
	k = \frac{3}{4} + n \qquad n \in \mathbb{N}_0.
\end{equation}

\subsection{Ableitung}
Es kann gezeigt werden, dass die Ableitungen $\frac{\partial w_1(z,k)}{\partial z}$ und $\frac{\partial w_2(z,k)}{\partial z}$ einen Zusammenhang zwischen $w_1(z,k)$ und $w_2(z,k)$ zeigen. Die Ableitung von $w_1(z,k)$ nach $z$ kann über die Produktregel berechnet werden und ist gegeben als
\begin{equation}
	\frac{\partial w_1(z,k)}{\partial z} = \left (\frac{1}{2} - 2k \right ) w_2(z, k -\frac{1}{2}) - \frac{1}{2} z w_1(z,k),
\end{equation} 
und die Ableitung von $w_2(z,k)$ als
\begin{equation}
	\frac{\partial w_2(z,k)}{\partial z} = w_1(z, k -\frac{1}{2}) - \frac{1}{2} z w_2(z,k).
\end{equation}
Über diese Eigenschaft können einfach weitere Ableitungen berechnet werden. 



    \printbibliography[heading=subbibliography]
\end{refsection}
