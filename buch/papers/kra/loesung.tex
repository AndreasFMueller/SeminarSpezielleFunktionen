\section{Lösungsmethoden} \label{kra:section:loesung}
\rhead{Lösungsmethoden}

\subsection{Riccatische Differentialgleichung} \label{kra:loesung:riccati}
Eine allgemeine analytische Lösung der Riccati Differentialgleichung ist nicht möglich.
Es gibt aber Spezialfälle, in denen sich die Gleichung vereinfachen lässt und so eine analytische Lösung gefunden werden kann.
Diese wollen wir im folgenden Abschnitt genauer anschauen.

\subsubsection{Fall 1: Konstante Koeffizienten}
Im Fall von konstanten Koeffizienten $f(x), g(x), h(x)$, wird die Gleichung \eqref{kra:equation:riccati} zu
\begin{equation}
    y' = fy^2 + gy + h.
\end{equation}
Durch Ausschreiben des Differentialquotienten
\begin{equation}
    \frac{dy}{dx} = fy^2 + gy + h
\end{equation}
erkennt man, dass die Differentialgleichung separierbar ist. Die Lösung findet man nun durch die Berechnung des Integrals
\begin{equation} \label{kra:equation:case1_int}
    \int \frac{dy}{fy^2 + gy + h} = \int dx.
\end{equation}

\subsubsection{Fall 2: Bekannte spezielle Lösung}
Kennt man eine spezielle Lösung $y_p$, so kann die riccatische Differentialgleichung mit Hilfe einer Substitution auf eine lineare Gleichung reduziert werden.
Wir wählen als Substitution
\begin{equation} \label{kra:equation:substitution}
    z = \frac{1}{y - y_p},
\end{equation}
durch Umstellen von \eqref{kra:equation:substitution} folgt
\begin{equation}
    y = y_p + \frac{1}{z^2} \label{kra:equation:backsubstitution}
\end{equation}
\begin{equation}
    y' = y_p' - \frac{1}{z^2}z',
\end{equation}
mit Einsetzten in die Differentialgleichung \eqref{kra:equation:riccati} resultiert
\begin{equation}
    y_p' - \frac{1}{z^2}z' = f(x)(y_p + \frac{1}{z}) + g(x)(y_p + \frac{1}{z})^2 + h(x)
\end{equation}
\begin{equation}
    -z^{2}y_p' + z' = -z^2\underbrace{(y_{p}f(x) + g(x)y_p^2 + h(x))}_{\displaystyle{y_p'}} - z(f(x) + 2y_{p}g(x)) - g(x)
\end{equation}
was uns direkt auf die lineare Differentialgleichung 1. Ordnung
\begin{equation}
    z' = -z(f(x) + 2y_{p}g(x)) - g(x)
\end{equation}
führt.
Diese kann nun mit den Methoden zur Lösung von linearen Differentialgleichungen 1. Ordnung gelöst werden.
Durch die Rücksubstitution \eqref{kra:equation:backsubstitution} erhält man dann die Lösung von \eqref{kra:equation:riccati}.

\subsection{Matrix-Riccati-Differentialgleichung} \label{kra:loesung:riccati}
Im Folgenden wollen wir uns anschauen wie die Matrix-Riccati-Differentialgleichung entsteht und wie sie gelöst werden kann.

\subsubsection{Entstehung}
Der Ausgangspunkt bildet die Matrix-Differentialgleichung
\begin{equation}
    \label{kra:equation:matrix-dgl}
    \begin{pmatrix}
        \dot{X}(t) \\
        \dot{Y}(t)
    \end{pmatrix}
    =
    \underbrace{
        \begin{pmatrix}
            A & B \\
            C & D
        \end{pmatrix}
    }_{\displaystyle{H}}
    \begin{pmatrix}
        X(t) \\
        Y(t)
    \end{pmatrix}
\end{equation}
mit den allgemeinen quadratischen Matrizen $A, B, C$ und $D$, welche in der sogenannten Hamiltonschen-Matrix $H$ zusammengefasst werden können.
Wir führen eine neue Grösse
\[
    U(t) = Y(t)X(t)^{-1}
\]
ein, für dessen Ableitung $\dt U(t)$ wir mit
\[
    \dot{X}(t) = AX(t) + BY(t) \quad \text{und} \quad \dot{Y}(t) = CX(t) + DY(t)
\]
folgendes Ergebnis erhalten
\begin{equation}
    \label{kra:equation:feder-masse-riccati-matrix}
    \begin{split}
        \dt U(t)   &= \dot{Y}(t) X(t)^{-1} + Y(t) \dt X(t)^{-1} \\
        &= (CX(t) + DY(t)) X(t)^{-1} - Y(t) (X(t)^{-1} \dot{X}(t) X(t)^{-1}) \\
        &= C\underbrace{X(t)X(t)^{-1}}_\text{$I$} + D\underbrace{Y(t)X(t)^{-1}}_\text{$U(t)$} - Y(t)(X(t)^{-1} (AX(t) + BY(t)) X(t)^{-1}) \\
        &= C + DU(t) - \underbrace{Y(t)X(t)^{-1}}_\text{$U(t)$}(A\underbrace{X(t)X(t)^{-1}}_\text{$I$} + B\underbrace{Y(t)X(t)^{-1}}_\text{$U(t)$}) \\
        &= C  + DU(t) - U(t)A - U(t)BU(t).
    \end{split}
\end{equation}
\begin{satz}
    \label{kra:satz:riccati-matrix-dgl}
    Die Ableitung $\dt U(t) = \dt (Y(t)X(t)^{-1})$ ist eine Matrix-Riccati-Differentialgleichung.
\end{satz}

\subsubsection{Lösung}
Sei
\[
    V(t)
    =
    \begin{pmatrix}
        X(t) \\
        Y(t)
    \end{pmatrix},
    \quad
    \dot{V}(t) = HV(t)
\]
eine Matrix-Differentialgleichung 1. Ordnung, dann ist
\[
    V(t) = e^{H(t)} V(0)
\]
eine Lösung.
Die Berechnung des Matrixexpontentials $e^{H(t)}$ kann mittels Diagonalisierung
\[
    H = Q \Lambda Q^{-1}
\]
effizient berechnet werden.
Es folgt dann, dass
\[
    e^{Ht}
    =
    Q
    e^{\Lambda t}
    Q^{-1}
    =
    Q
    \begin{pmatrix}
        e^{\lambda_1 t} & 0               & \dots  & 0               \\
        0               & e^{\lambda_2 t} & \ddots & \vdots          \\
        \vdots          & \ddots          & \ddots & 0               \\
        0               & \dots           & 0      & e^{\lambda_n t}
    \end{pmatrix}
    Q^{-1}
\]
ist. Die Lösung der Matrix-Riccati-Differentialgleichung erhalten wir analog mit
\begin{equation}
    \label{kra:matrixriccati-solution}
    \begin{pmatrix}
        X(t) \\
        Y(t)
    \end{pmatrix}
    =
    \Phi(t_0, t)
    \begin{pmatrix}
        I(t) \\
        P_0(t)
    \end{pmatrix}
    =
    \begin{pmatrix}
        \Phi_{11}(t_0, t) & \Phi_{12}(t_0, t) \\
        \Phi_{21}(t_0, t) & \Phi_{22}(t_0, t)
    \end{pmatrix}
    \begin{pmatrix}
        I(t) \\
        P_0(t)
    \end{pmatrix}
\end{equation}
\begin{equation}
    P(t) =
    \begin{pmatrix}
        \Phi_{21}(t_0, t) + \Phi_{22}(t_0, t)
    \end{pmatrix}
    \begin{pmatrix}
        \Phi_{11}(t_0, t) + \Phi_{12}(t_0, t)
    \end{pmatrix}
    ^{-1}
\end{equation}
wobei $\Phi(t_0, t) = e^{H(t - t_0)}$ die sogenannte Zustandsübergangsmatrix von \eqref{kra:equation:matrix-dgl} ist,
welche die Zeitentwicklung der einzelnen Lösungen beschreibt \cite{kra:kalmanisae}.
