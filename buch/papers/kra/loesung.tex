\section{Lösungsmethoden} \label{kra:section:loesung}
\rhead{Lösungsmethoden}

\subsection{Riccatische Differentialgleichung} \label{kra:loesung:riccati}
Eine allgemeine analytische Lösung der Riccati Differentialgleichung ist nicht möglich.
Es gibt aber Spezialfälle, in denen sich die Gleichung vereinfachen lässt und so eine analytische Lösung gefunden werden kann.
Diese wollen wir im folgenden Abschnitt genauer anschauen.

\subsubsection{Fall 1: Konstante Koeffizienten}
Im Fall von konstanten Koeffizienten $f(x), g(x), h(x)$, wird die Gleichung \eqref{kra:equation:riccati} zu
\begin{equation}
    y' = fy^2 + gy + h.
\end{equation}
Durch Ausschreiben des Differentialquotienten
\begin{equation}
    \frac{dy}{dx} = fy^2 + gy + h
\end{equation}
erkennt man, dass die DGL separierbar ist. Die Lösung findet man nun durch die Berechnung des Integrals
\begin{equation} \label{kra:equation:case1_int}
    \int \frac{dy}{fy^2 + gy + h} = \int dx.
\end{equation}

\subsubsection{Fall 2: Bekannte spezielle Lösung}
Kennt man eine spezielle Lösung $y_p$, so kann die riccatische DGL mit Hilfe einer Substitution auf eine lineare Gleichung reduziert werden.
Wir wählen als Substitution
\begin{equation} \label{kra:equation:substitution}
    z = \frac{1}{y - y_p},
\end{equation}
durch Umstellen von \eqref{kra:equation:substitution} folgt
\begin{equation}
    y = y_p + \frac{1}{z^2} \label{kra:equation:backsubstitution}
\end{equation}
\begin{equation}
    y' = y_p' - \frac{1}{z^2}z',
\end{equation}
mit Einsetzten in die DGL \eqref{kra:equation:riccati} resultiert
\begin{equation}
    y_p' - \frac{1}{z^2}z' = f(x)(y_p + \frac{1}{z}) + g(x)(y_p + \frac{1}{z})^2 + h(x)
\end{equation}
\begin{equation}
    -z^{2}y_p' + z' = -z^2\underbrace{(y_{p}f(x) + g(x)y_p^2 + h(x))}_{\displaystyle{y_p'}} - z(f(x) + 2y_{p}g(x)) - g(x)
\end{equation}
was uns direkt auf die lineare Differentialgleichung 1. Ordnung
\begin{equation}
    z' = -z(f(x) + 2y_{p}g(x)) - g(x)
\end{equation}
führt.
Diese kann nun mit den Methoden zur Lösung von linearen Differentialgleichungen 1. Ordnung gelöst werden.
Durch die Rücksubstitution \eqref{kra:equation:backsubstitution} erhält man dann die Lösung von \eqref{kra:equation:riccati}.

\subsection{Matrix-Riccati-Differentialgleichung} \label{kra:loesung:riccati}
% Lösung matrix riccati
Die Lösung der Matrix-Riccati-Gleichung \ref{kra:equation:matrixriccati} erhalten wir nach \cite{kra:kalmanisae} folgendermassen
\begin{equation}
    \label{kra:matrixriccati-solution}
    \begin{pmatrix}
        X(t) \\
        Y(t)
    \end{pmatrix}
    =
    \Phi(t_0, t)
    \begin{pmatrix}
        I(t) \\
        U_0(t)
    \end{pmatrix}
    =
    \begin{pmatrix}
        \Phi_{11}(t_0, t) & \Phi_{12}(t_0, t) \\
        \Phi_{21}(t_0, t) & \Phi_{22}(t_0, t)
    \end{pmatrix}
    \begin{pmatrix}
        I(t) \\
        U_0(t)
    \end{pmatrix}
\end{equation}
\begin{equation}
    U(t) =
    \begin{pmatrix}
        \Phi_{21}(t_0, t) + \Phi_{22}(t_0, t)
    \end{pmatrix}
    \begin{pmatrix}
        \Phi_{11}(t_0, t) + \Phi_{12}(t_0, t)
    \end{pmatrix}
    ^{-1}
\end{equation}
wobei $\Phi(t, t_0)$ die sogenannte Zustandsübergangsmatrix ist.
\begin{equation}
    \Phi(t_0, t) = e^{H(t - t_0)}
\end{equation}
