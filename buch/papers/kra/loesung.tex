\section{Lösungsmethoden} \label{kra:section:loesung}
\rhead{Lösungsmethoden}
% @TODO Lösung normal riccati
Lösung der Riccatischen Differentialgleichung \ref{kra:riccati}.


% Lösung matrix riccati
Die Lösung der Matrix-Riccati Gleichung \ref{kra:matrixriccati} erhalten wir nach \cite{kra:kalmanisae} folgendermassen
\begin{equation}
    \label{kra:matrixriccati-solution}
    \begin{pmatrix}
        X(t) \\
        Y(t)
    \end{pmatrix}
    =
    \Phi(t_0, t)
    \begin{pmatrix}
        I(t) \\
        U_0(t)
    \end{pmatrix}
    =
    \begin{pmatrix}
        \Phi_{11}(t_0, t) & \Phi_{12}(t_0, t) \\
        \Phi_{21}(t_0, t) & \Phi_{22}(t_0, t)
    \end{pmatrix}
    \begin{pmatrix}
        I(t) \\
        U_0(t)
    \end{pmatrix}
\end{equation}

\begin{equation}
    U(t) =
    \begin{pmatrix}
        \Phi_{21}(t_0, t) + \Phi_{22}(t_0, t)
    \end{pmatrix}
    \begin{pmatrix}
        \Phi_{11}(t_0, t) + \Phi_{12}(t_0, t)
    \end{pmatrix}
    ^{-1}
\end{equation}

wobei $\Phi(t, t_0)$ die sogennante Zustandsübergangsmatrix ist.

\begin{equation}
    \Phi(t_0, t) = e^{H(t - t_0)}
\end{equation}
