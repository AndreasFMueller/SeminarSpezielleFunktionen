\newcommand{\dt}[0]{\frac{d}{dt}}

\section{Teil abc\label{kra:section:teilabc}}
\rhead{Teil abc}

\subsection{Hamilton-Funktion}
Die Bewegung der Masse $m$ kann mit Hilfe der hamiltonschen Mechanik im Phasenraum untersucht werden.
Die hamiltonschen Gleichungen verwenden dafür die veralgemeinerten Ortskoordinaten
$q = (q_{1}, q_{2}, ..., q_{n})$ und die verallgemeinerten Impulskoordinaten $p = (p_{1}, p_{2}, ..., p_{n})$,
wobei der Impuls definiert ist als $p_k = m_k \cdot v_k$.
Liegen keine zeitabhängigen Zwangsbedingungen vor, so entspricht die Hamitlon-Funktion der Gesamtenergie des Systems \cite{kra:hamilton}.
Im Falle des einfachen Federmassesystems, Abbildung \ref{kra:fig:simple_spring_mass},
setzt sich die Hamilton-Funktion aus kinetischer und potentieller Energie zusammen.

\begin{equation}
    \label{hamilton}
    \begin{split}
        \mathcal{H}(q, p) &= T(p) + V(q) = E \\
        &= \underbrace{\frac{p^2}{2m}}_{E_{kin}} + \underbrace{\frac{k q^2}{2}}_{E_{pot}}
    \end{split}
\end{equation}

Die Hamiltonschen Bewegungsgleichungen liefern \cite{kra:kanonischegleichungen}
\begin{equation}
    \label{kra:hamilton:bewegungsgleichung}
    \dot{q_{k}} = \frac{\partial \mathcal{H}}{\partial p_k}
    \qquad
    \dot{p_{k}} = -\frac{\partial \mathcal{H}}{\partial q_k}
\end{equation}

daraus folgt

\[
    \dot{q} = \frac{p}{m}
    \qquad
    \dot{p} = -kq
\]

in Matrixschreibweise erhalten wir also

\[
    \begin{pmatrix}
        \dot{q} \\
        \dot{p}
    \end{pmatrix}
    =
    \begin{pmatrix}
        0  & \frac{1}{m} \\
        -k & 0
    \end{pmatrix}
    \begin{pmatrix}
        q \\
        p
    \end{pmatrix}
\]

Für das erweiterte Federmassesystem, Abbildung \ref{kra:fig:multi_spring_mass}, können wir analog vorgehen.
Die kinetische Energie setzt sich nun aus den kinetischen Energien der einzelnen Massen $m_1$ und $m_2$ zusammen.
Die Potentielle Energie erhalten wir aus der Summe der kinetischen Energien der einzelnen Federn mit den Federkonstanten $k_1$, $k_c$ und $k_2$.

\begin{align*}
    \begin{split}
        T   &= T_1 + T_2 \\
        &= \frac{p_1^2}{2m_1} + \frac{p_2^2}{2m_2}
    \end{split}
    \\
    \begin{split}
        V   &= V_1 + V_c + V_2 \\
        &= \frac{k_1 q_1^2}{2} + \frac{k_c (q_2 - q_1)^2}{2} + \frac{k_2 q_2^2}{2}
    \end{split}
\end{align*}

Die Hamilton-Funktion ist also

\begin{align*}
    \begin{split}
        \mathcal{H}     &= T + V \\
        &= \frac{p_1^2}{2m_1} + \frac{p_2^2}{2m_2} + \frac{k_1 q_1^2}{2} + \frac{k_c (q_2 - q_1)^2}{2} + \frac{k_2 q_2^2}{2}
    \end{split}
\end{align*}

Die Bewegungsgleichungen \ref{kra:hamilton:bewegungsgleichung} liefern
\begin{align*}
    \frac{\partial \mathcal{H}}{\partial p_k}  & = \dot{q_k}
    \Rightarrow
    \left\{
    \begin{alignedat}{2}
        \dot{q_1}   &= \frac{2p_1}{2m_1}    &&= \frac{p_1}{m_1}\\
        \dot{q_2}   &= \frac{2p_2}{2m_2}    &&= \frac{p_2}{m_2}
    \end{alignedat}
    \right.
    \\
    -\frac{\partial \mathcal{H}}{\partial q_k} & = \dot{p_k}
    \Rightarrow
    \left\{
    \begin{alignedat}{2}
        \dot{p_1}   &= -(\frac{2k_1q_1}{2} - \frac{2k_c(q_2-q_1)}{2})  &&= -q_1(k_1+k_c) + q_2k_c \\
        \dot{p_1}   &= -(\frac{2k_c(q_2-q_1)}{2} - \frac{2k_2q_2}{2})  &&= q_1k_c - (k_c + k_2)
    \end{alignedat}
    \right.
\end{align*}

In Matrixschreibweise erhalten wir

\begin{equation}
    \label{kra:hamilton:multispringmass}
    \begin{pmatrix}
        \dot{q_1} \\
        \dot{q_2} \\
        \dot{p_1} \\
        \dot{p_2} \\
    \end{pmatrix}
    =
    \begin{pmatrix}
        0            & 0            & \frac{1}{2m_1} & 0              \\
        0            & 0            & 0              & \frac{1}{2m_2} \\
        -(k_1 + k_c) & k_c          & 0              & 0              \\
        k_c          & -(k_c + k_2) & 0              & 0              \\
    \end{pmatrix}
    \begin{pmatrix}
        q_1 \\
        q_2 \\
        p_1 \\
        p_2 \\
    \end{pmatrix}
    \Leftrightarrow
    \dt
    \begin{pmatrix}
        Q \\
        P \\
    \end{pmatrix}
    \underbrace{
        \begin{pmatrix}
            0 & M \\
            K & 0
        \end{pmatrix}
    }_{G}
    \begin{pmatrix}
        Q \\
        P \\
    \end{pmatrix}
\end{equation}


Wir intressieren uns nun dafür wie der Phasenwinkel $U = PQ^{-1}$ von der Zeit abhängt,
wir suchen also die Grösse $\Theta = \dt U$.

Ersetzten wir in der Gleichung \ref{kra:hamilton:multispringmass} die Matrix $G$ mit $\tilde{G}$ so erhalten wir
\begin{equation}
    \dt
    \begin{pmatrix}
        Q \\
        P
    \end{pmatrix}
    =
    \underbrace{
        \begin{pmatrix}
            A & B \\
            C & D
        \end{pmatrix}
    }_{\tilde{G}}
    \begin{pmatrix}
        Q \\
        P
    \end{pmatrix}
\end{equation}

Mit einsetzten folgt

\begin{align*}
    \dot{Q} = AQ + BP \\
    \dot{P} = CQ + DP
\end{align*}
\begin{equation}
    \begin{split}
        \dt U   &= \dot{P} Q^{-1} + P \dt Q^{-1} \\
        &= (CQ + DP) Q^{-1} - P (Q^{-1} \dot{Q} Q^{-1}) \\
        &= C\underbrace{QQ^{-1}}_\text{I} + D\underbrace{PQ^{-1}}_\text{U} - P(Q^{-1} (AQ + BP) Q^{-1}) \\
        &= C + DU - \underbrace{PQ^{-1}}_\text{U}(A\underbrace{QQ^{-1}}_\text{I} + B\underbrace{PQ^{-1}}_\text{U}) \\
        &= C  + DU - UA - UBU
    \end{split}
\end{equation}

was uns auf die zeitkontinuierliche Matrix-Riccati-Gleichung führt.

