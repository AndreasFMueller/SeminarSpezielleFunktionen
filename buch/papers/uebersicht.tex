%
% uebersicht.tex -- Uebersicht ueber die Seminar-Arbeiten
%
% (c) 2022 Prof Dr Andreas Mueller, OST Ostschweizer Fachhochschule
%
\chapter*{Übersicht}
\lhead{Übersicht}
\rhead{}
\label{buch:uebersicht}
Im zweiten Teil kommen die Teilnehmer des Seminars selbst zu Wort.
Die im ersten Teil dargelegten mathematischen Methoden und
grundlegenden Modelle werden dabei verfeinert, verallgemeinert
und auch numerisch überprüft.

Die Lambert-$W$-Funktion ermöglicht, Exponentialgleichungen der
Form $xe^x=b$ zu lösen.
{\em David Hugentobler}
\index{David Hugentobler}%
\index{Hugentobler, David}%
und
{\em Yanik Kuster}
\index{Yanik Kuster}%
\index{Kuster, Yanik}%
zeigen in Kapitel~\ref{chapter:lambertw}, wie genau so eine Kombination
auch in der Lösung der Differentialgleichung für eine Verfolgungskurve
entsteht.
Die Lambert-$W$-Funktion ermöglicht daher, die Lösung ``in geschlossener
From'' hinzuschreiben.

Die Bessel-Funktionen sind über über den Koordinatenwechsel zu
Polarkoordinaten mit der Fourier-Transformation verbunden.
Es ist daher nicht so überraschend, dass sie auch in einem 
fundamentalen Problem der Elektrotechnik auftauchen.
Frequenzmodulation, im analogen Radio und anderen Anwendungen
von grosser praktischer Bedeutung.
Die Modulationsart ist jedoch nicht linear, selbst bei der Modulation
mit einem harmonischen Signal treten viele weitere Frequenz im 
Frequenzspektrum des modulierten Signals auf.
{\em Joshua Bär}
\index{Joshua Bär}%
\index{Bär, Joshua}%
berechnet in seiner Arbeit (Kapitel~\ref{chapter:fm}) die Amplituden
dieser Frequenzen als Werte der Bessel-Funktionen.

Spezielle Funktionen treten oft bei der Lösung von
partiellen Differentialgleichungen mit der Separationsmethode auf.
{\em Alain Keller} 
\index{Alain Keller}%
\index{Keller, Alain}%
und
{\em Thierry Schwaller}
\index{Thierry Schwaller}%
\index{Schwaller, Thierry}%
führen die Separationsmethode für die Helmholzgleichung in parabolischen
Koordinaten in Kapitel~\ref{chapter:parzyl}
durch, gehen dann aber noch einen Schritt weiter und 
stellen die parabolischen Zylinderfunktionen,
die Webersche und die Whittakersche Differentialgleichung und die
Whittaker-Funktionen vor.

Die Fresnel-Integrale treten einerseits in der Optik auf, sie können aber
auch eine Klothoide parametrisieren.
Kapitel~\ref{chapter:fresnel} zeigt diesen Zusammenhang und verbindet
ihn auch mit einer alltäglichen Beobachtung beim Schälen eines Apfels.

Als weiteres partielles Differentialgleichungsproblem
untersuchen 
{\em Andrea Mozzini Vellen}
\index{Andrea Mozzini Vellen}%
\index{Mozzini Vellen, Andrea}%
und
{\em Tim Tönz}
\index{Tim Tönz}%
\index{Tönz, Tim}%
in Kapitel~\ref{chapter:kreismembran}
die Wellengleichung in der Ebene in Polarkoordinaten.
Sie stellen die Separationsmethode der Transformationsmethode
gegenüber und kommen damit auf die Hankel-Transformation zu sprechen.
Ausserdem illustrieren Sie ihre Arbeit mit interessanten numerischen
Simulationen.

Das Sturm-Liouville-Problem ist eine Familie von Randwertproblemen
gewöhnlicher Differentialgleichungen, die zum Beispiel bei der
Separation von partiellen Differentialgleichungen oft auftritt.
{\em Réda Haddouche}
\index{Réda Haddouche}%
\index{Haddouche, Réda}%
und
{\em Erik Löffler}
\index{Erik Löffler}%
\index{Löffler, Erik}%
spielen dieses Szenario in Kapitel~\ref{chapter:sturmliouville} an
naheliegenden Beispielen durch und rechnen so im Detail nach, wie bei
solchen Problemen immer orthogonale Funktionenfamilien bezüglich eines
aus der Problemstellung ablesbaren Skalarproduktes entstehen.

Orthogonale Polynome bilden die Basis für das ausserordentlich
genaue numerische Integrationsverfahren der Gauss-Quadratur.
{\em Patrik Müller}
\index{Patrik Müller}%
\index{Müller, Patrick}%
wendet den Fall der Laguerre-Gauss-Quadratur auf die Integraldefinition
der Gamma-Funktion an und erhält ein Verfahren, welches Werte hoher
Genauigkeit mit sehr wenigen Auswertungen des Integranden liefern kann.

Die Riemannsche Zeta-Funktion $\zeta(s)$ hat auf Youtube Kultstatus
erworden wegen der daraus ableitbaren Aussage, die Summe aller natürlichen
Zahlen sei $-\frac1{12}$.
{\em Raphael Unterer}
\index{Raphael Unterer}%
\index{Unterer, Raphael}%
geht in Kapitel~\ref{chapter:zeta} genau dieser Aussage nach und zeigt,
wie man durch Konstruktion der analytischen Fortsetzung auf diesen
``verrückten'' Wert kommen kann.

Spezielle Funktionen werden in Computer-Lösungen genau dann nützlich,
wenn man effiziente Algorithmen zu deren Berechnung hat.
Für die Funktion $\mathstrut_0F_1$ enthält die GNU Scientific Library
\index{GSL}%
\index{GNU Scientific Library}%
zwar eine Implementation, die aber für gewisse Argumentwerte 
auf Linux oder MacOS eine Exception wirft.
{\em Fabian Dünki} 
\index{Fabian Dünki}%
\index{Dünki, Fabian}%
schlägt in Kapitel~\ref{chapter:0f1} verschiedene alternative
Implementationen vor, stösst aber auch auf interessante numerische Probleme.

Wie in Kapitel~\ref{buch:chapter:geometrie} angedeutet, gehören die
trigonometrischen Funktionen zu den ältesten speziellen Funktionen.
Sie wurden schon im Altertum zur Berechnung von Dreiecken auf der
Kugeloberfläche verwendet.
{\em Enez Erdem}
\index{Enez Erdem}%
\index{Erdem, Enez}%
und 
{\em Marc Kühne}
\index{Marc Kühne}%
\index{Kühne, Marc}%
lösen in Kapitel~\ref{chapter:nav} die Aufgabe, aus Beobachtungen
mit einem Sextanten die eigene Position auf der Erde zu bestimmen.

Die Tangens-Hyperbolicus-Funktion ist von grosser Bedeutung bei
der Auswertung von neuronalen Netzwerken.
Dabei kommt es vor allem auf Geschwindkeit an, weniger auf die
Genauigkeit.
{\em Marc Benz}
\index{Marc Benz}%
\index{Benz, Marc}%
stellt in Kapitel~\ref{chapter:transfer} eine geeignete
Berechnungsmethode vor.

Elliptische Funktionen ermöglichen, gewisse nichtlineare
Differentialgleichungen zu lösen.
Umgekehrt kann man fragen, welche Arten von nichtlinearen
Differentialgleichungen einer Lösung mit speziellen Funktionen
zugänglich sind.
{\em Samuel Niederer} geht der Frage in Kapitel~\ref{chapter:kra} 
\index{Samuel Niederer}%
\index{Niederer, Samuel}%
nach, wo er die Riccati-Differentialgleichung als nichtlineare
Differentialgleichung mit einiger Bedeutung in den Anwendungen
unter die Lupe nimmt.

Die Fourier-Theorie besagt, dass sich jede periodische Funktion in eine
Fourier-Reihe entwickeln lässt.
Eine solche Aussage gilt auch für Funktionen auf der Kugeloberfläche: 
jede Funktion auf der Kugeloberfläche lässt sich in eine Reihe
entwickeln, deren Summanden sogenannte Kugelfunktionen sind.
Die Theorie der Kugelfunktionen ist einer der Höhepunkte der Theorie
der speziellen Funktionen: die Kugelfunktionen sind eine orthogonale
Funktionenfamilie, die sich aus orthogonalen Polynomen, den sogenannten
Legendre-Polynomen und assoziierten Legendre-Polynome und den
klassischen Exponentialfunktionen aufbauen lassen.
{\em Manuel Cattaneo}
\index{Manuel Cattaneo}%
\index{Cattaneo, Manuel}%
und
{\em Naoki Pross}
\index{Naoki Pross}%
\index{Pross, Naoki}%
zeigen in Kapitel~\ref{chapter:kugel} mit viel Details und
Anwendungsbeispielen, wie diese besondere Theorie fast alle im ersten
Teil des Buches entwickelten Ideen illustrieren kann.

{\em Nicolas Toble}
\index{Nicolas Tobler}%
\index{Tobler, Nicolas}%
beantwortet in Kapitel~\ref{chapter:ellfilter} die Frage,
warum alle die bekannten Filter-Designs, die Elektroingenieure in
ihrer Ausbildung kennenlernen, nach irgendwelchen speziellen Funktionen
benannt sind.
Er findet heraus, dass dies damit zusammenhängt, dass die Filter
durch eine Optimalitätseigenschaft im Durchlass- oder im Sperrbereich
definiert sind, die die jeweiligen speziellen Funktionen erfüllen.
Als Verallgemeinerung des Tschebyscheff-Filters zeigt er dann, wie
mit elliptischen Funktionen ein Filter konstruiert werden kann,
welches sowohl im Durchlassbereich wie auch im Sperrbereich eine 
Optimalitätsbedingung erfüllen.
Solche elliptischen Filter stehen dem Ingenieur als fertige integrierte
Schaltungen für seine Problemlösungen zur Verfügung.
Die Analyse zeigt, wie das Verständnis elliptischer Integrale 
im komplexen zu den Optimalitätsbedingungen für elliptische
Filter führt.

Im Kapitel~\ref{chapter:dreieck} schliesslich wird gezeigt,
wie sich aus den eigenschaften der hermiteschen Polynome
ein Kriterium dafür entwickeln lässt, ob eine Funktion der Form
$P(t)e^{-t^2}$ mit einem Polynom $P(t)$ in geschlossener Form
integriert werden kann.
