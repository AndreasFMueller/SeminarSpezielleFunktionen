\documentclass[12pt]{scrartcl}
\usepackage{ucs}
\usepackage[utf8]{inputenc}
\usepackage[T1]{fontenc}
\usepackage{graphicx}

\begin{document}
\section{Geschichte der sphärischen Navigation}
Die Orientierung mit Hilfe der Sterne und der sphärischen Trigonometrie bewegt die Menschheit schon seit mehreren tausend Jahren. 
Nach Hinweisen und Schätzungen von Forscher haben schon vor 4000 Jahren die Ägypter und Gelehrten aus Babylon mit Hilfe der Astronomie den Lauf der Gestirne (Himmelskörper) zu berechnen versucht, jedoch ohne Erfolg. 
Etwa 350 vor Christus waren es die Griechen, welche den damaligen Astronomen Hilfestellungen mittels Kugel-Geometrien leisten konnten. 
Aus diesen Geometrien wurden erste mathematische Sätze aufgestellt und ein paar Jahrhunderte später kamen zu diesem Thema auch Berechnungen dazu. 
Ebenso wurden Kartenmaterial mit Sternenbilder angefertigt. 
Die Sinusfunktion war noch nicht bekannt, jedoch kamen zu dieser Zeit die ersten Ansätze der Cosinusfunktion aus Indien. 
Von diesen Hilfen darauf aufbauend konnte um 900 die Araber der Sinussatz entwickeln. 
Doch ein paar weitere Jahrhunderte vergingen bis zu diesem Thema wieder verstärkt Forschung betrieben wurde. 
Dies aus dem Grund, da im 15. Jahrhundert grosse Entdeckungsreisen, hauptsächlich per Schiff, erfolgten und die Orientierung vermehrt an Wichtigkeit gewann. 
Auch die Verwendung der Tangens- und Sinusfunktion sowie der neu entwickelte Seitencosinussatz trugen zu einer Verbesserung der Orientierung herbei. 
Im 16. Jahrhundert wurde dann ein weiterer trigonometrischer Satz, der Winkelcosinussatz hergeleitet. Stück für Stück wurden infolge der Entdeckung des Logarithmus im 17. Jahrhundert viele neue Methoden entwickelt. 
Auch eine Verbesserung der kartographischen Verwendung der Kugelgeometrie wurde vorgenommen.  
Es folgten weitere Entwicklungen in nicht euklidische Geometrien und im 19. Jahrhundert sowie auch im 20. Jahrhundert wurde zudem für die Relativitätstheorie auch die sphärische Trigonometrie beigezogen.
\end{document}