

\section{Einleitung}
Heutzutage ist die Navigation ein Teil des Lebens. 
Man sendet dem Kollegen seinen eigenen Standort, um sich das ewige Erklären zu sparen oder gibt die Adresse des Ziels ein, damit man seinen Aufenthaltsort zum Beispiel auf einer riesigen Wiese am See findet. 
Dies wird durch Technologien wie Funknavigation, welches ein auf Langzeitmessung beruhendes Hyperbelverfahren mit Langwellen ist oder die verbreitete Satellitennavigation, welche vier Satelliten für eine Messung zur Standortbestimmung nutzt.
Vor all diesen technologischen Fortschritten gab es lediglich die Astronavigation, welche heute noch auf Schiffen verwendet wird im Falle eines Stromausfalls. 
Aber wie funktioniert die Navigation mit den Sternen? Welche Hilfsmittel benötigt man, welche Rolle spielt die Mathematik und weshalb kann die Erde nicht flach sein? 
In diesem Kapitel werden genau diese Fragen mithilfe des nautischen Dreiecks, der sphärischen Trigonometrie und einigen Hilfsmitteln und Messgeräten beantwortet.