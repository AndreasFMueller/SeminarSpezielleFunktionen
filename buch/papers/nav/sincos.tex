


\section{Sphärische Navigation und Winkelfunktionen}
\kopfrechts{Sphärische Navigation und Winkelfunktionen}
Es gibt Hinweise, dass sich schon die Babylonier und Ägypter vor
4000~Jahren mit Problemen der sphärischen Trigonometrie beschäftigt
haben, um den Lauf von Gestirnen zu berechnen.
\index{Agypten@Ägypten}%
\index{sphärische Trigonometrie}%
Jedoch konnten sie dieses Problem nicht lösen. 
Die Geschichte der sphärischen Trigonometrie ist daher eng mit der Astronomie verknüpft.
\index{Astronomie}%
Ca.~350 BCE dachten die Griechen über Kugelgeometrie nach, sie wurde damit zu einer Hilfswissenschaft der Astronomen.

Zwischen 190 v.~Chr.~und 120 v.~Chr.~lebte ein griechischer Astronom namens Hipparchos.
\index{Hipparchos}%
Dieser entwickelte unter anderem die Chordentafeln, welche die Chordfunktion,
auch Chord genannt, beinhalten.
\index{Chord}%
Chord ist der Vorgänger der Sinusfunktion und galt damals als wichtigste
Grundlage der Trigonometrie.
Damals kannte man die Sinusfunktionen noch nicht.
In dieser Zeit wurden auch die ersten Sternenkarten angefertigt.

Die Definition der trigonometrischen Funktionen aus Griechenland
ermöglicht nur, rechtwinklige Dreiecke zu berechnen.
Die Beziehung zwischen Seiten und Winkeln allgemeiner Dreiecke
sind komplizierter und als Sinus- und Kosinussätze bekannt.
Aus Indien stammten die ersten Ansätze zu den Kosinussätzen.
\index{Kosinussatz}%
Aufbauend auf den indischen und griechischen Forschungen entwickeln
die Araber um das 9.~Jahrhundert den Sinussatz. 
\index{Sinussatz}%
Doch ein paar weitere Jahrhunderte vergingen bis zu diesem Thema
wieder verstärkt Forschung betrieben wurde, da im 15.~Jahrhundert
grosse Entdeckungsreisen, hauptsächlich per Schiff, erfolgten und
die Orientierung mit Sternen vermehrt an Wichtigkeit gewann.
Man nutzte für die Kartographie nun die Kugelgeometrie, um die
Genauigkeit zu erhöhen.
Der Sinussatz, die Tangensfunktion und der neu entwickelte
Seitenkosinussatz wurden in dieser Zeit bereits verwendet und im
\index{Seitenkosinussatz}%
darauffolgenden Jahrhundert folgte der Winkelkosinussatz.
\index{Winkelkosinussatz}%

Durch weitere mathematische Entwicklungen wie dem Logarithmus wurden
im Laufe des nächsten Jahrhunderts viele neue Methoden und
kartographische Anwendungen der Kugelgeometrie entdeckt.
\index{Logarithmus}%
Im 19.~und 20.~Jahrhundert wurden weitere nicht-euklidische Geometrien
entwickelt und die sphärische Trigonometrie fand auch ihre Anwendung
in der Relativitätstheorie.

