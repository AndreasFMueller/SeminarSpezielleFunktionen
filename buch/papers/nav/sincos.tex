


\section{Sphärische Navigation und Winkelfunktionen}
Es gibt Hinweise, dass sich schon die Babylonier und Ägypter vor 4000 Jahren sich mit Problemen der sphärischen Trigonometrie beschäftigt haben um den Lauf von Gestirnen zu berechnen. 
Jedoch konnten sie dieses Problem nicht lösen. 

Die Geschichte der sphärischen Trigonometrie ist daher eng mit der Astronomie verknüpft. Ca. 350 vor Christus dachten die Griechen über Kugelgeometrie nach und sie wurde zu einer Hilfswissenschaft der Astronomen. 
In Folge werden auch die ersten Sätze aufgestellt und wenige Jahrhunderte später wurden Berechnungen mithilfe des Sternkataloges von Hipparchos angestellt und darauffolgend Kartenmaterial erstellt.
In dieser Zeit wurden auch die ersten Sternenkarten angefertigt, jedoch kannte man damals die Sinusfunktion noch nicht. 
Aus Indien stammten die ersten Ansätze zu den Kosinussätzen.
Aufbauend auf den indischen und griechischen Forschungen entwickeln die Araber um das 9. Jahrhundert den Sinussatz. 
Doch ein paar weitere Jahrhunderte vergingen bis zu diesem Thema wieder verstärkt Forschung betrieben wurde, da im 15. Jahrhundert grosse Entdeckungsreisen, hauptsächlich per Schiff, erfolgten und die Orientierung mit Sternen vermehrt an Wichtigkeit gewann.
Man nutzte für die Kartographie nun die Kugelgeometrie, um die Genauigkeit zu erhöhen.
Der Sinussatz, die Tangensfunktion und der neu entwickelte Seitenkosinussatz wurden in dieser Zeit bereits verwendet und im darauffolgenden Jahrhundert folgte der Winkelkosinussatz. 


Durch weitere mathematische Entwicklungen wie den Logarithmus wurden im Laufe des nächsten Jahrhunderts viele neue Methoden und kartographische Anwendungen der Kugelgeometrie entdeckt. 
Im 19. und 20. Jahrhundert wurden weitere nicht-euklidische Geometrien entwickelt und die sphärische Trigonometrie fand auch ihre Anwendung in der Relativitätstheorie.