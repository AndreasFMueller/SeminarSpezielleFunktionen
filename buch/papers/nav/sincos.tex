


\section{Warum sind die Sinus- und Kosinusfunktionen spezielle Funktionen?}
Es gibt Hinweise, dass sich schon die Babylonier und Ägypter vor 4000 Jahren sich mit Problemen der sphärischen Trigonometrie beschäftigt haben um den Lauf von Gestirnen (Himmelskörper) zu berechnen. 
Jedoch konnten sie sie nicht lösen. 
Die Geschichte der sphärischen Trigonometrie ist daher eng mit der Astronomie verknüpft. Ca. 350 vor Christus dachten die Griechen über Kugelgeometrie nach und wurde zu einer Hilfswissenschaft der Astronomen. 
In Folge werden auch die ersten Sätze aufgestellt und wenige Jahrhunderte später wurden Berechnungen zu diesem Thema angestellt. 
In dieser Zeit wurden auch die ersten Sternenkarten angefertigt, jedoch kannte man damals die Sinusfunktion noch nicht. 
Aus Indien stammten die ersten Ansätze zu den Kosinussätzen.
Aufbauend auf den indischen und griechischen Forschungen entwickeln die Araber um 900 den Sinussatz. 
Zur Zeit der großen Entdeckungsreisen im 15. Jahrhundert wurden die Forschungen in sphärischer Trigonometrie wieder forciert. 
Der Sinussatz, die Tangensfunktion und der neu entwickelte Seitenkosinussatz wurden in dieser Zeit bereits verwendet. 
Im nächsten Jahrhundert folgte der Winkelkosinussatz. 
Durch weitere mathematische Entwicklungen wie den Logarithmus wurden im Laufe des nächsten Jahrhunderts viele neue Methoden und kartographische Anwendungen der Kugelgeometrie entdeckt. 
Im 19. und 20. Jahrhundert wurden weitere nicht-euklidische Geometrien entwickelt und die sphärische Trigonometrie fand auch ihre Anwendung in der Relativitätstheorie.