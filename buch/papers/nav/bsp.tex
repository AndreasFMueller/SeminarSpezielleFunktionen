\section{Beispielrechnung}

\subsection{Einführung}
In diesem Abschnitt wird die Theorie vom Abschnitt 21.6 in einem Praxisbeispiel angewendet. 
Wir haben die Deklination, Rektaszension, Höhe der beiden Planeten Deneb und Arktur und die Sternzeit von Greenwich als Ausgangslage.
Die Deklinationen und Rektaszensionen sind von einem vergangenen Datum und die Höhe der Gestirne und die Sternzeit wurden von einem uns unbekannten Ort aus gemessen.
Diesen Punkt gilt es mit dem erlangten Wissen herauszufinden.

\subsection{Vorgehen}

\begin{center}
	\begin{tabular}{l l l}
		1. & Koordinaten der Bildpunkte der Gestirne bestimmen \\
		2. & Dreiecke aufzeichnen und richtig beschriften\\
		3. & Dreieck ABC bestimmmen\\
		4. & Dreieck BPC bestimmen \\
		5. & Dreieck ABP bestimmen \\ 
		6. & Geographische Breite bestimmen \\
		7. & Geographische Länge bestimmen \\
	\end{tabular}
\end{center}

\subsection{Ausgangslage}
Die Rektaszension und die Sternzeit sind in der Regeln in Stunden angegeben.
Für die Umrechnung in Grad kann folgender Zusammenhang verwendet werden:
\[ Stunden \cdot 15 = Grad\].
Dies wurde hier bereits gemacht.
\begin{center}
	\begin{tabular}{l l l}
		Sternzeit $s$ & $118.610804^\circ$ \\
		Deneb&\\
		& Rektaszension $RA_{Deneb}$& $310.55058^\circ$ \\
		& Deklination $DEC_{Deneb}$& $45.361194^\circ$ \\
		& Höhe $H_{Deneb}$ & $50.256027^\circ$ \\ 
		Arktur &\\
		& Rektaszension $RA_{Arktur}$& $214.17558^\circ$ \\
		& Deklination $DEC_{Arktur}$& $19.063222^\circ$ \\
		& Höhe $H_{Arktur}$ & $47.427444^\circ$ \\  
	\end{tabular}
\end{center}
\subsection{Koordinaten der Bildpunkte}
Als erstes benötigen wir die Koordinaten der Bildpunkte von Arktur und Deneb. 
$\delta$ ist die Breite, $\lambda$ die Länge.
\begin{align}
\delta_{Deneb}&=DEC_{Deneb} = \underline{\underline{45.361194^\circ}} \nonumber \\ 
\lambda_{Deneb}&=RA_{Deneb} - s = 310.55058^\circ -118.610804^\circ =\underline{\underline{191.939776^\circ}}   \nonumber \\ 
\delta_{Arktur}&=DEC_{Arktur} =  \underline{\underline{19.063222^\circ}} \nonumber \\ 
\lambda_{Arktur}&=RA_{Arktur} - s = 214.17558^\circ -118.610804^\circ = \underline{\underline{5.5647759^\circ}}  \nonumber  
\end{align}


\subsection{Dreiecke definieren}
Das Festlegen der Dreiecke ist essenziell für die korrekten Berechnungen.
BILD
\subsection{Dreieck ABC}
Nun berechnen wir alle Seitenlängen $a$, $b$, $c$ und die Innnenwinkel $\alpha$, $\beta$ und $\gamma$
Wir können $b$ und $c$ mit den geltenten Zusammenhängen des nautischen Dreiecks wie folgt bestimmen:
\begin{align}
	b=90^\circ-DEC_{Deneb} = 90^\circ - 45.361194^\circ = \underline{\underline{44.638806^\circ}}\nonumber \\
	c=90^\circ-DEC_{Arktur} = 90^\circ - 19.063222^\circ = \underline{\underline{70.936778^\circ}} \nonumber 
\end{align}
Um $a$ zu bestimmen, benötigen wir zuerst den Winkel \[\alpha= RA_{Deneb} - RA_{Arktur} = 310.55058^\circ -214.17558^\circ = \underline{\underline{96.375^\circ}}.\]
Danach nutzen wir den sphärischen Winkelkosinussatz, um  $a$ zu berechnen:
\begin{align}
	a &= \cos^{-1}(\cos(b) \cdot \cos(c) + \sin(b) \cdot \sin(c) \cdot \cos(\alpha)) \nonumber \\
	 &= \cos^{-1}(\cos(44.638806) \cdot \cos(70.936778) + \sin(44.638806) \cdot \sin(70.936778) \cdot \cos(96.375)) \nonumber \\
	 &= \underline{\underline{80.8707801^\circ}} \nonumber
\end{align}
Für $\beta$ und $\gamma$ nutzen wir den sphärischen Seitenkosinussatz:
\begin{align}
	\beta &= \cos^{-1}  \bigg[\frac{\cos(b)-\cos(a) \cdot \cos(c)}{\sin(a) \cdot \sin(c)}\bigg] \nonumber \\
	&= \cos^{-1}  \bigg[\frac{\cos(44.638806)-\cos(80.8707801) \cdot \cos(70.936778)}{\sin(80.8707801) \cdot \sin(70.936778)}\bigg] \nonumber \\
	&= \underline{\underline{45.0115314^\circ}} \nonumber
\end{align}

	\begin{align}
	\gamma &=  \cos^{-1}  \bigg[\frac{\cos(c)-\cos(b) \cdot \cos(a)}{\sin(a) \cdot \sin(b)}\bigg] \nonumber \\
	&=  \cos^{-1}  \bigg[\frac{\cos(70.936778)-\cos(44.638806) \cdot \cos(80.8707801)}{\sin(80.8707801) \cdot \sin(44.638806)}\bigg] \nonumber \\
	&=\underline{\underline{72.0573328^\circ}} \nonumber
\end{align}

