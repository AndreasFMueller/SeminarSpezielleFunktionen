%
% dreieck.tex -- sphärische Dreiecke für Positionsbestimmung
%
% (c) 2021 Prof Dr Andreas Müller, OST Ostschweizer Fachhochschule
%
\documentclass[tikz]{standalone}
\usepackage{amsmath}
\usepackage{times}
\usepackage{txfonts}
\usepackage{pgfplots}
\usepackage{csvsimple}
\usetikzlibrary{arrows,intersections,math,calc}
\begin{document}

\definecolor{darkgreen}{rgb}{0,0.6,0}

\def\skala{1}

\def\punkt#1#2{
	\fill[color=#2] #1 circle[radius=0.08];
}

\begin{tikzpicture}[>=latex,thick,scale=\skala]

\coordinate (P) at (0.0000,0.0000);
\coordinate (A) at (1.0000,8.0000);
\coordinate (B) at (-3.0000,3.0000);
\coordinate (C) at (4.0000,4.0000);
\def\kanteAB{(1.0000,8.0000) arc (114.77514:167.90524:7.1589)}
\def\kanteBA{(-3.0000,3.0000) arc (167.90524:114.77514:7.1589)}
\def\kanteAC{(1.0000,8.0000) arc (63.43495:10.30485:5.5902)}
\def\kanteCA{(4.0000,4.0000) arc (10.30485:63.43495:5.5902)}
\def\kanteAP{(1.0000,8.0000) arc (146.30993:199.44003:9.0139)}
\def\kantePA{(0.0000,0.0000) arc (199.44003:146.30993:9.0139)}
\def\kanteBC{(-3.0000,3.0000) arc (-95.90614:-67.83365:14.5774)}
\def\kanteCB{(4.0000,4.0000) arc (-67.83365:-95.90614:14.5774)}
\def\kanteBP{(-3.0000,3.0000) arc (-161.56505:-108.43495:4.7434)}
\def\kantePB{(0.0000,0.0000) arc (-108.43495:-161.56505:4.7434)}
\def\kanteCP{(4.0000,4.0000) arc (-30.96376:-59.03624:11.6619)}
\def\kantePC{(0.0000,0.0000) arc (-59.03624:-30.96376:11.6619)}

\def\winkelAlpha#1{
	\begin{scope}
		\clip (A) circle[radius=1.1];
		\fill[color=#1!20] \kanteAB -- \kanteCA -- cycle;
	\end{scope}
	\node[color=#1] at ($(A)+(222:0.82)$) {$\alpha$};
}

\def\winkelOmega#1{
	\begin{scope}
		\clip (A) circle[radius=0.7];
		\fill[color=#1!20] \kanteAP -- \kanteCA -- cycle;
	\end{scope}
	\node[color=#1] at ($(A)+(285:0.50)$) {$\omega$};
}

\def\winkelGamma#1{
	\begin{scope}
		\clip (C) circle[radius=1.0];
		\fill[color=#1!20] \kanteCA -- \kanteBC -- cycle;
	\end{scope}
	\node[color=#1] at ($(C)+(155:0.60)$) {$\gamma$};
}

\def\winkelKappa#1{
	\begin{scope}
		\clip (B) circle[radius=1.2];
		\fill[color=#1!20] \kanteBP -- \kanteAB -- cycle;
	\end{scope}
	\node[color=#1] at ($(B)+(15:1.00)$) {$\kappa$};
}

\def\winkelBeta#1{
	\begin{scope}
		\clip (B) circle[radius=0.8];
		\fill[color=#1!20] \kanteBC -- \kanteAB -- cycle;
	\end{scope}
	\node[color=#1] at ($(B)+(35:0.40)$) {$\beta$};
}

\def\winkelBetaEins#1{
	\begin{scope}
		\clip (B) circle[radius=0.8];
		\fill[color=#1!20] \kanteBP -- \kanteCB -- cycle;
	\end{scope}
	\node[color=#1] at ($(B)+(330:0.60)$) {$\beta_1$};
}

\def\seiteC#1{ \node[color=#1] at (-1.9,5.9) {$c$}; }
\def\seiteB#1{ \node[color=#1] at (3.2,6.5) {$b$}; }
\def\seiteL#1{ \node[color=#1] at (-0.2,4.5) {$l$}; }
\def\seiteA#1{ \node[color=#1] at (2,3) {$a$}; }
\def\seitePB#1{ \node[color=#1] at (-2.1,1) {$p_b$}; }
\def\seitePC#1{ \node[color=#1] at (2.5,1.5) {$p_c$}; }


\winkelAlpha{red}
\winkelGamma{blue}
\winkelBeta{darkgreen}

\seiteC{black}
\seiteB{black}
\seiteA{black}

%\seiteL{gray}
\seitePB{gray}
\seitePC{gray}

\draw[line width=1.4pt] \kanteAB;
\draw[line width=1.4pt] \kanteAC;
%\draw[color=gray] \kanteAP;
\draw[line width=1.4pt] \kanteBC;
\draw[color=gray] \kanteBP;
\draw[color=gray] \kanteCP;

\punkt{(A)}{black};
\punkt{(B)}{black};
\punkt{(C)}{black};
\punkt{(P)}{gray};

\node at (A) [above] {$A$};
\node at (B) [left] {$B$};
\node at (C) [right] {$C$};
\node[color=gray] at (P) [below] {$P$};

\end{tikzpicture}
\end{document}

