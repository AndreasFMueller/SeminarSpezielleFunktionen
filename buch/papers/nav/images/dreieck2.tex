%
% dreieck2.tex -- sphärische Dreiecke für Positionsbestimmung
%
% (c) 2021 Prof Dr Andreas Müller, OST Ostschweizer Fachhochschule
%
\documentclass[tikz]{standalone}
\usepackage{amsmath}
\usepackage{times}
\usepackage{txfonts}
\usepackage{pgfplots}
\usepackage{csvsimple}
\usetikzlibrary{arrows,intersections,math,calc}
\begin{document}

\definecolor{darkgreen}{rgb}{0,0.6,0}

\def\skala{1}

\def\punkt#1#2{
	\fill[color=#2] #1 circle[radius=0.08];
}

\begin{tikzpicture}[>=latex,thick,scale=\skala]

\coordinate (P) at (0.0000,0.0000);
\coordinate (A) at (1.0000,8.0000);
\coordinate (B) at (-3.0000,3.0000);
\coordinate (C) at (4.0000,4.0000);
\def\kanteAB{(1.0000,8.0000) arc (114.77514:167.90524:7.1589)}
\def\kanteBA{(-3.0000,3.0000) arc (167.90524:114.77514:7.1589)}
\def\kanteAC{(1.0000,8.0000) arc (63.43495:10.30485:5.5902)}
\def\kanteCA{(4.0000,4.0000) arc (10.30485:63.43495:5.5902)}
\def\kanteAP{(1.0000,8.0000) arc (146.30993:199.44003:9.0139)}
\def\kantePA{(0.0000,0.0000) arc (199.44003:146.30993:9.0139)}
\def\kanteBC{(-3.0000,3.0000) arc (-95.90614:-67.83365:14.5774)}
\def\kanteCB{(4.0000,4.0000) arc (-67.83365:-95.90614:14.5774)}
\def\kanteBP{(-3.0000,3.0000) arc (-161.56505:-108.43495:4.7434)}
\def\kantePB{(0.0000,0.0000) arc (-108.43495:-161.56505:4.7434)}
\def\kanteCP{(4.0000,4.0000) arc (-30.96376:-59.03624:11.6619)}
\def\kantePC{(0.0000,0.0000) arc (-59.03624:-30.96376:11.6619)}

%
% macros.tex -- some common macro definitions
%
% (c) 2021 Prof Dr Andreas Müller, OST Ostschweizer Fachhochschule
%
\hypersetup{
    linktoc=all,
    linkcolor=blue
}
\newcounter{beispiel}
\newenvironment{beispiele}{
\bgroup\smallskip\parindent0pt\bf Beispiele\egroup

\begin{list}{\arabic{beispiel}.}
  {\usecounter{beispiel}
  \setlength{\labelsep}{5mm}
  \setlength{\rightmargin}{0pt}
}}{\end{list}}
\newcounter{uebungsaufgabezaehler}
% environment fuer uebungsaufgaben
\newenvironment{uebungsaufgaben}{%
}{%
\vfill\pagebreak}
\newenvironment{teilaufgaben}{
\begin{enumerate}
\renewcommand{\theenumi}{\alph{enumi})}
%\renewcommand{\labelenumi}{\alph{enumi})}
\renewcommand{\labelenumi}{\theenumi}
}{\end{enumerate}}
% Aufgabe
\newcounter{problemcounter}[chapter]
\def\aufgabenpath{chapters/uebungsaufgaben/}
\def\ainput#1{\input\aufgabenpath/#1}
\def\verbatimainput#1{\expandafter\verbatiminput{\aufgabenpath/#1}}
\def\aufgabetoplevel#1{%
\expandafter\def\expandafter\inputpath{#1}%
\let\aufgabepath=\inputpath
}
\def\includeagraphics[#1]#2{\expandafter\includegraphics[#1]{\aufgabepath#2}}
% \aufgabe
\renewcommand\theproblemcounter{\thechapter.\arabic{problemcounter}}
\newcommand{\uebungsaufgabe}[1]{%
\refstepcounter{problemcounter}%
\label{#1}%
\bigskip{\parindent0pt\strut}\hbox{\bf\theproblemcounter. }%
\expandafter\def\csname aufgabenpath\endcsname{\inputpath/}%
\expandafter\input{\aufgabenpath/#1.tex}
}
% linsys
\newcolumntype{\linsysR}{>{$}r<{$}}
\newcolumntype{\linsysL}{>{$}l<{$}}
\newcolumntype{\linsysC}{>{$}c<{$}}
\newenvironment{linsys}[1]{%
\begin{tabular}{*{#1}{\linsysR@{\;}\linsysC}@{\;}\linsysR}}%
{\end{tabular}}

% Loesung
\def\swallow#1{
%nothing
}
\NewEnviron{loesung}[1][Lösung]{%
\begin{proof}[#1]%
\renewcommand{\qedsymbol}{$\bigcirc$}
\BODY
\end{proof}
}
\NewEnviron{bewertung}{%
\begin{proof}[Bewertung]%
\renewcommand{\qedsymbol}{}
\BODY
\end{proof}
}
\NewEnviron{diskussion}{%
\begin{proof}[Diskussion]%
\renewcommand{\qedsymbol}{}
\BODY
\end{proof}
}
\NewEnviron{hinweis}{%
\begin{proof}[Hinweis]%
\renewcommand{\qedsymbol}{}
\BODY
\end{proof}
}
\def\keineloesungen{%
\RenewEnviron{loesung}{\relax}
\RenewEnviron{bewertung}{\relax}
\RenewEnviron{diskussion}{\relax}
}
\newenvironment{beispiel}{%
\begin{proof}[Beispiel]%
\renewcommand{\qedsymbol}{$\bigcirc$}
}{\end{proof}}

\allowdisplaybreaks

\lhead{Inhaltsverzeichnis}
\rhead{}
\tableofcontents
\newtheorem{satz}{Satz}[chapter]
\newtheorem{hilfssatz}[satz]{Hilfssatz}
\newtheorem{korollar}[satz]{Korollar}
\newtheorem{lemma}[satz]{Lemma}
\newtheorem{definition}[satz]{Definition}
\newtheorem{annahme}[satz]{Annahme}
\newtheorem{frage}[satz]{Frage}
\newtheorem{problem}[satz]{Problem}
\newtheorem{aufgabe}[satz]{Aufgabe}
\newtheorem{prinzip}[satz]{Prinzip}
\newtheorem*{problem*}{Problem}
\newtheorem{forderung}{Forderung}[chapter]
\newtheorem{konsequenz}[satz]{Konsequenz}
\newtheorem{algorithmus}[satz]{Algorithmus}

% English variants
\newtheorem{theorem}[satz]{Theorem}

\renewcommand{\floatpagefraction}{0.7}

\definecolor{darkgreen}{rgb}{0,0.6,0}
\definecolor{darkred}{rgb}{0.8,0,0}
\definecolor{orange}{rgb}{1,0.6,0}
%
% Kopfzeilen
%
\def\kopfrechts#1{\rhead{\theshortsection. #1}}
\def\kopflinks#1{\lhead{\thechapter. #1}
\rhead{}}
%
% Anwendungsabschnitte
%
\newcommand{\theshortsection}{\arabic{section}}
\def\anwendungen{%
\setcounter{section}{0}
\renewcommand{\thesection}{\thechapter.\Alph{section}}
\renewcommand{\theshortsection}{\Alph{section}}
}
\def\endanwendungen{%
\renewcommand{\thesection}{\thechapter.\arabic{section}}
\renewcommand{\theshortsection}{\arabic{section}}
}



%\winkelAlpha{red}
%\winkelGamma{blue}
%\winkelBeta{darkgreen}

\seiteC{black}
\seiteB{black}
%\seiteA{black}

%\seiteL{gray}
\seitePB{gray}
\seitePC{gray}

\draw[line width=1.4pt] \kanteAB;
\draw[line width=1.4pt] \kanteAC;
%\draw[color=gray] \kanteAP;
\draw[line width=1.4pt] \kanteBC;
\draw[color=gray] \kanteBP;
\draw[color=gray] \kanteCP;

\punkt{(A)}{black};
\punkt{(B)}{black};
\punkt{(C)}{black};
\punkt{(P)}{gray};

\node at (A) [above] {$A$};
\node at (B) [left] {$B$};
\node at (C) [right] {$C$};
\node[color=gray] at (P) [below] {$P$};

\end{tikzpicture}
\end{document}

