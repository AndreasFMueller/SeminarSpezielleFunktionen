\section{Rationale elliptische Funktionen}

Kommen wir nun zum eigentlichen Teil dieses Papers, den rationalen elliptischen Funktionen \cite{ellfilter:bib:orfanidis}
\begin{align}
    R_N(\xi, w) &= \cd \left(N~f_1(\xi)~\cd^{-1}(w, 1/\xi), f_2(\xi)\right) \label{ellfilter:eq:elliptic}\\
                &= \cd \left(N~\frac{K_1}{K}~\cd^{-1}(w, k), k_1\right) , \quad k= 1/\xi, k_1 = 1/f(\xi) \\
                &= \cd \left(N~K_1~z , k_1 \right), \quad w= \cd(z K, k)
\end{align}
Beim Betrachten dieser Definition, fällt die Ähnlichkeit zur trigonometrische Darstellung der Tsche\-byschef-Polynome \eqref{ellfilter:eq:chebychef_polynomials} auf.
Wie bei den Tschebyscheff-Polynomen ist die Formel mit speziellen Funktionen geschrieben.
Es kann jedoch gezeigt werden, dass es sich tatsächlich um rationale Funktionen handelt, wie es für ein lineares Filter vorausgesetzt wird.
Die elliptischen Funktionen werden also genau so eingesetzt, dass die resultierenden Nullstellen und Pole eine rationale Funktion ergeben. 
Anstelle des Kosinus bei den Tschebyscheff-Polynomen kommt hier die $\cd$-Funktion zum Einsatz.
Die Ordnungszahl $N$ kommt auch als Faktor for.
Zusätzlich werden noch zwei verschiedene elliptische Moduli $k$ und $k_1$ gebraucht.
Bei $k = k_1 = 0$ wird der $\cd$ zum Kosinus und wir erhalten in diesem Spezialfall die Tschebyschef-Polynome.

Durch das Konzept vom fundamentalen Rechteck, siehe Abbildung \ref{buch:elliptisch:fig:ellall} können für alle inversen Jacobi elliptischen Funktionen die Positionen der Null- und Polstellen anhand eines Diagramms ermittelt werden.
Die $\cd^{-1}(w, k)$-Funktion ist um $K$ verschoben zur $\sn^{-1}(w, k)$-Funktion, wie ersichtlich in Abbildung \ref{ellfilter:fig:cd}.
\begin{figure}
    \centering
    \begin{tikzpicture}[>=stealth', auto, node distance=2cm, scale=1.2, thick]

    \tikzstyle{zero} = [draw, circle, inner sep =0, minimum height=0.15cm]

    \tikzset{pole/.style={cross out, draw=black, minimum size=(0.15cm-\pgflinewidth), inner sep=0pt, outer sep=0pt}}

    \begin{scope}[xscale=0.9, yscale=1.8]

        \draw[gray, ->] (0,-1.5) -- (0,1.5) node[anchor=south]{$\mathrm{Im}~z$};
        \draw[gray, ->] (-5,0) -- (5,0) node[anchor=west]{$\mathrm{Re}~z$};


        \begin{scope}[xshift=0cm]

            \clip(-4.5,-1.25) rectangle (4.5,1.25);

            \fill[yellow!30] (0,0) rectangle (1, 0.5);

            \foreach \i in {-2,...,1} {
                \foreach \j in {-2,...,1} {
                    \begin{scope}[xshift=\i*4cm, yshift=\j*1cm]
                        \draw[->, thick, orange!50] (0, 0) -- (0,0.5);
                        \draw[->, thick, darkgreen!50] (1, 0) -- (0,0);
                        \draw[->, thick, cyan!50] (2, 0) -- (1,0);
                        \draw[->, thick, blue!50] (2,0.5) -- (2, 0);
                        \draw[->, thick, purple!50] (1, 0.5) -- (2,0.5);
                        \draw[->, thick, red!50] (0, 0.5) -- (1,0.5);
                        \draw[->, thick, orange!50] (0,1) -- (0,0.5);
                        \draw[->, thick, blue!50] (2,0.5) -- (2, 1);
                        \draw[->, thick, purple!50] (3, 0.5) -- (2,0.5);
                        \draw[->, thick, red!50] (4, 0.5) -- (3,0.5);
                        \draw[->, thick, cyan!50] (2, 0) -- (3,0);
                        \draw[->, thick, darkgreen!50] (3, 0) -- (4,0);
                    \end{scope}
                }
            }

            \draw[ultra thick, ->, orange] (0, 0) -- (0,0.5);
            \draw[ultra thick, ->, darkgreen] (1, 0) -- (0,0);
            \draw[ultra thick, ->, cyan] (2, 0) -- (1,0);
            \draw[ultra thick, ->, blue] (2,0.5) -- (2, 0);
            \draw[ultra thick, ->, purple] (1, 0.5) -- (2,0.5);
            \draw[ultra thick, ->, red] (0, 0.5) -- (1,0.5);

            \foreach \i in {-2,...,1} {
                \foreach \j in {-2,...,1} {
                    \begin{scope}[xshift=\i*4cm, yshift=\j*1cm]
                        \node[zero] at ( 1, 0) {};
                        \node[zero] at ( 3, 0) {};
                        \node[pole] at ( 1,0.5) {};
                        \node[pole] at ( 3,0.5) {};

                    \end{scope}
                }
            }

        \end{scope}

        \draw[gray] ( 1,0) +(0,0.05) -- +(0, -0.05) node[inner sep=0, anchor=north west] {\small $K$};
        \draw[gray]  (0, 0.5) +(0.1, 0) -- +(-0.1, 0) node[inner sep=0, anchor=south east]{\small $jK^\prime$};

    \end{scope}


    \node[zero] at (4,3) (n) {};
    \node[anchor=west] at (n.east) {Nullstelle};
    \node[pole, below=0.25cm of n] (n) {};
    \node[anchor=west] at (n.east) {Polstelle};

    \begin{scope}[yshift=-4cm, xscale=0.75]

        \draw[gray, ->] (-6,0) -- (6,0) node[anchor=west]{$w$};

        \draw[ultra thick, ->, purple] (-5, 0) -- (-3, 0);
        \draw[ultra thick, ->, blue]      (-3, 0) -- (-2, 0);
        \draw[ultra thick, ->, cyan]       (-2, 0) -- (0, 0);
        \draw[ultra thick, ->, darkgreen]    (0, 0) -- (2, 0);
        \draw[ultra thick, ->, orange] (2, 0) -- (3, 0);
        \draw[ultra thick, ->, red] (3, 0) -- (5, 0);

        \node[anchor=south] at (-5,0) {$-\infty$};
        \node[anchor=south] at (-3,0) {$-1/k$};
        \node[anchor=south] at (-2,0) {$-1$};
        \node[anchor=south] at (0,0) {$0$};
        \node[anchor=south] at (2,0) {$1$};
        \node[anchor=south] at (3,0) {$1/k$};
        \node[anchor=south] at (5,0) {$\infty$};

    \end{scope}

\end{tikzpicture}
    \caption{
        $z$-Ebene der Funktion $z = \cd^{-1}(w, k)$.
        Die Funktion ist in der realen Achse $4K$-periodisch und in der imaginären Achse $2jK^\prime$-periodisch.
    }
    \label{ellfilter:fig:cd}
\end{figure}
Auffallend an der $w = \cd(z, k)$-Funktion ist, dass sich $w$ auf der reellen Achse wie der Kosinus immer zwischen $-1$ und $1$ bewegt, während bei $\mathrm{Im(z) = K^\prime}$ die Werte zwischen $\pm 1/k$ und $\pm \infty$ verlaufen.
Die Idee des elliptischen Filter ist es, diese zwei Equiripple-Zonen abzufahren, wie ersichtlich in Abbildung \ref{ellfilter:fig:cd2}, welche analog zu Abbildung \ref{ellfilter:fig:arccos2} gesehen werden kann.
\begin{figure}
    \centering
    \begin{tikzpicture}[>=stealth', auto, node distance=2cm, scale=1.2]

    \tikzstyle{zero} = [draw, circle, inner sep =0, minimum height=0.15cm]
    \tikzstyle{dot} = [fill, circle, inner sep =0, minimum height=0.1cm]

    \tikzset{pole/.style={cross out, draw=black, minimum size=(0.15cm-\pgflinewidth), inner sep=0pt, outer sep=0pt}}

    \begin{scope}[xscale=1.25, yscale=3.5]

        \draw[gray, ->] (0,-0.55) -- (0,1.05) node[anchor=south]{$\mathrm{Im}~z_1$};
        \draw[gray, ->] (-1.5,0) -- (6,0) node[anchor=west]{$\mathrm{Re}~z_1$};

        \draw[gray] ( 1,0) +(0,0.05) -- +(0, -0.05) node[inner sep=0, anchor=north] {\small $K_1$};
        \draw[gray] ( 5,0) +(0,0.05) -- +(0, -0.05) node[inner sep=0, anchor=north] {\small $5K_1$};
        \draw[gray]  (0, 0.5) +(0.1, 0) -- +(-0.1, 0) node[inner sep=0, anchor=east]{\small $jK^\prime_1$};

        \begin{scope}

            \clip(-1.5,-0.75) rectangle (6.8,1.25);

            % \draw[>->, line width=0.05, thick, blue]   (1, 0.45) -- (2, 0.45) -- (2, 0.05) -- ( 0.1, 0.05) -- ( 0.1,0.45) -- (1, 0.45);
            % \draw[>->, line width=0.05, thick, orange] (2, 0.5 ) -- (4, 0.5 ) -- (4, 0   ) -- ( 0  , 0   ) -- ( 0  ,0.5 ) -- (2, 0.5 );
            % \draw[>->, line width=0.05, thick, red]    (3, 0.55) -- (6, 0.55) -- (6,-0.05) -- (-0.1,-0.05) -- (-0.1,0.55) -- (3, 0.55);
            % \node[blue] at (1, 0.25) {$N=1$};
            % \node[orange] at (3, 0.25) {$N=2$};
            % \node[red] at (5, 0.25) {$N=3$};



            % \draw[line width=0.1cm, fill, red!50] (0,0) rectangle (3, 0.5);
            % \draw[line width=0.05cm, fill, orange!50] (0,0) rectangle (2, 0.5);
            % \fill[yellow!50] (0,0) rectangle (1, 0.5);
            % \node[] at (0.5, 0.25) {\small $N=1$};
            % \node[] at (1.5, 0.25) {\small $N=2$};
            % \node[] at (2.5, 0.25) {\small $N=3$};

            \fill[orange!30] (0,0) rectangle (5, 0.5);
            % \fill[yellow!30] (0,0) rectangle (1, 0.1);
            \node[] at (2.5, 0.25) {\small $N=5$};


            \draw[decorate,decoration={brace,amplitude=3pt,mirror}, yshift=0.05cm]
                (5,0.5) node(t_k_unten){} -- node[above, yshift=0.1cm]{$NK_1$}
                (0,0.5) node(t_k_opt_unten){};

            \draw[decorate,decoration={brace,amplitude=3pt,mirror}, xshift=0.1cm]
                (5,0) node(t_k_unten){} -- node[right, xshift=0.1cm]{$K^\prime \frac{K_1N}{K} = K^\prime_1$}
                (5,0.5) node(t_k_opt_unten){};


            \draw[ultra thick, ->, darkgreen] (5, 0) -- node[yshift=-0.5cm]{Durchlassbereich} (0,0);
            \draw[ultra thick, ->, orange] (-0, 0) --  node[align=center]{Übergangs-\\berech} (0,0.5);
            \draw[ultra thick, ->, red] (0,0.5) -- node[align=center, yshift=0.7cm]{Sperrbereich} (5, 0.5);

            \draw (4,0  )  node[dot]{} node[anchor=south]      {\small $1$};
            \draw (2,0  )  node[dot]{} node[anchor=south]      {\small $-1$};
            \draw (0,0  )  node[dot]{} node[anchor=south west] {\small $1$};
            \draw (0,0.5)  node[dot]{} node[anchor=north west] {\small $1/k$};
            \draw (2,0.5)  node[dot]{} node[anchor=north]      {\small $-1/k$};
            \draw (4,0.5)  node[dot]{} node[anchor=north]      {\small $1/k$};

            \foreach \i in {-2,...,1} {
                \foreach \j in {-2,...,1} {
                    \begin{scope}[xshift=\i*4cm, yshift=\j*1cm]

                        \node[zero] at ( 1, 0) {};
                        \node[zero] at ( 3, 0) {};
                        \node[pole] at ( 1,0.5) {};
                        \node[pole] at ( 3,0.5) {};

                    \end{scope}
                }
            }

        \end{scope}

    \end{scope}

\end{tikzpicture}
    \caption{
        $z_1=N\frac{K_1}{K}\cd^{-1}(w, k)$-Ebene der rationalen elliptischen Funktionen.
        Je grösser die Ordnung $N$ gewählt wird, desto mehr Nullstellen werden passiert.
        Als Vereinfachung ist die Funktion nur für $w>0$ dargestellt.
    }
    \label{ellfilter:fig:cd2}
\end{figure}
Das elliptische Filter hat im Gegensatz zum Tschebyscheff-Filter drei Zonen.
Im Durchlassbereich werden wie beim Tschebyscheff-Filter die Nullstellen durchlaufen.
Statt dass $z_1$ für alle $w>1$ in die imaginäre Richtung geht, bewegen wir uns im Sperrbereich wieder in reeller Richtung, wo Pole und Punkte mit $\pm 1/k$ durchlaufen werden.
Aus dieser Sicht kann der Sperrbereich vom Tschebyscheff-Filter als unendlich langer Übergangsbereich angesehen werden.
% Falls es möglich ist diese Werte abzufahren im Stil der Tschebyscheff-Polynome, kann ein Filter gebaut werden, dass Equiripple-Verhalten im Durchlass- und Sperrbereich aufweist.
Abbildung \ref{ellfilter:fig:elliptic_freq} zeigt eine rationale elliptische Funktion und die Frequenzantwort des daraus resultierenden Filters.
\begin{figure}
    \centering
    %% Creator: Matplotlib, PGF backend
%%
%% To include the figure in your LaTeX document, write
%%   \input{<filename>.pgf}
%%
%% Make sure the required packages are loaded in your preamble
%%   \usepackage{pgf}
%%
%% Also ensure that all the required font packages are loaded; for instance,
%% the lmodern package is sometimes necessary when using math font.
%%   \usepackage{lmodern}
%%
%% Figures using additional raster images can only be included by \input if
%% they are in the same directory as the main LaTeX file. For loading figures
%% from other directories you can use the `import` package
%%   \usepackage{import}
%%
%% and then include the figures with
%%   \import{<path to file>}{<filename>.pgf}
%%
%% Matplotlib used the following preamble
%%
\begingroup%
\makeatletter%
\begin{pgfpicture}%
\pgfpathrectangle{\pgfpointorigin}{\pgfqpoint{5.000000in}{3.000000in}}%
\pgfusepath{use as bounding box, clip}%
\begin{pgfscope}%
\pgfsetbuttcap%
\pgfsetmiterjoin%
\pgfsetlinewidth{0.000000pt}%
\definecolor{currentstroke}{rgb}{1.000000,1.000000,1.000000}%
\pgfsetstrokecolor{currentstroke}%
\pgfsetstrokeopacity{0.000000}%
\pgfsetdash{}{0pt}%
\pgfpathmoveto{\pgfqpoint{0.000000in}{0.000000in}}%
\pgfpathlineto{\pgfqpoint{5.000000in}{0.000000in}}%
\pgfpathlineto{\pgfqpoint{5.000000in}{3.000000in}}%
\pgfpathlineto{\pgfqpoint{0.000000in}{3.000000in}}%
\pgfpathlineto{\pgfqpoint{0.000000in}{0.000000in}}%
\pgfpathclose%
\pgfusepath{}%
\end{pgfscope}%
\begin{pgfscope}%
\pgfsetbuttcap%
\pgfsetmiterjoin%
\definecolor{currentfill}{rgb}{1.000000,1.000000,1.000000}%
\pgfsetfillcolor{currentfill}%
\pgfsetlinewidth{0.000000pt}%
\definecolor{currentstroke}{rgb}{0.000000,0.000000,0.000000}%
\pgfsetstrokecolor{currentstroke}%
\pgfsetstrokeopacity{0.000000}%
\pgfsetdash{}{0pt}%
\pgfpathmoveto{\pgfqpoint{0.730012in}{1.798407in}}%
\pgfpathlineto{\pgfqpoint{4.711458in}{1.798407in}}%
\pgfpathlineto{\pgfqpoint{4.711458in}{2.731117in}}%
\pgfpathlineto{\pgfqpoint{0.730012in}{2.731117in}}%
\pgfpathlineto{\pgfqpoint{0.730012in}{1.798407in}}%
\pgfpathclose%
\pgfusepath{fill}%
\end{pgfscope}%
\begin{pgfscope}%
\pgfpathrectangle{\pgfqpoint{0.730012in}{1.798407in}}{\pgfqpoint{3.981446in}{0.932710in}}%
\pgfusepath{clip}%
\pgfsetbuttcap%
\pgfsetmiterjoin%
\definecolor{currentfill}{rgb}{0.000000,0.501961,0.000000}%
\pgfsetfillcolor{currentfill}%
\pgfsetfillopacity{0.200000}%
\pgfsetlinewidth{0.000000pt}%
\definecolor{currentstroke}{rgb}{0.000000,0.000000,0.000000}%
\pgfsetstrokecolor{currentstroke}%
\pgfsetstrokeopacity{0.200000}%
\pgfsetdash{}{0pt}%
\pgfpathmoveto{\pgfqpoint{0.730012in}{-91.099555in}}%
\pgfpathlineto{\pgfqpoint{2.720735in}{-91.099555in}}%
\pgfpathlineto{\pgfqpoint{2.720735in}{2.171491in}}%
\pgfpathlineto{\pgfqpoint{0.730012in}{2.171491in}}%
\pgfpathlineto{\pgfqpoint{0.730012in}{-91.099555in}}%
\pgfpathclose%
\pgfusepath{fill}%
\end{pgfscope}%
\begin{pgfscope}%
\pgfpathrectangle{\pgfqpoint{0.730012in}{1.798407in}}{\pgfqpoint{3.981446in}{0.932710in}}%
\pgfusepath{clip}%
\pgfsetbuttcap%
\pgfsetmiterjoin%
\definecolor{currentfill}{rgb}{1.000000,0.647059,0.000000}%
\pgfsetfillcolor{currentfill}%
\pgfsetfillopacity{0.200000}%
\pgfsetlinewidth{0.000000pt}%
\definecolor{currentstroke}{rgb}{0.000000,0.000000,0.000000}%
\pgfsetstrokecolor{currentstroke}%
\pgfsetstrokeopacity{0.200000}%
\pgfsetdash{}{0pt}%
\pgfpathmoveto{\pgfqpoint{2.720735in}{2.171491in}}%
\pgfpathlineto{\pgfqpoint{2.740483in}{2.171491in}}%
\pgfpathlineto{\pgfqpoint{2.740483in}{2.358033in}}%
\pgfpathlineto{\pgfqpoint{2.720735in}{2.358033in}}%
\pgfpathlineto{\pgfqpoint{2.720735in}{2.171491in}}%
\pgfpathclose%
\pgfusepath{fill}%
\end{pgfscope}%
\begin{pgfscope}%
\pgfpathrectangle{\pgfqpoint{0.730012in}{1.798407in}}{\pgfqpoint{3.981446in}{0.932710in}}%
\pgfusepath{clip}%
\pgfsetbuttcap%
\pgfsetmiterjoin%
\definecolor{currentfill}{rgb}{1.000000,0.000000,0.000000}%
\pgfsetfillcolor{currentfill}%
\pgfsetfillopacity{0.200000}%
\pgfsetlinewidth{0.000000pt}%
\definecolor{currentstroke}{rgb}{0.000000,0.000000,0.000000}%
\pgfsetstrokecolor{currentstroke}%
\pgfsetstrokeopacity{0.200000}%
\pgfsetdash{}{0pt}%
\pgfpathmoveto{\pgfqpoint{2.740483in}{2.358033in}}%
\pgfpathlineto{\pgfqpoint{4.731206in}{2.358033in}}%
\pgfpathlineto{\pgfqpoint{4.731206in}{2.731121in}}%
\pgfpathlineto{\pgfqpoint{2.740483in}{2.731121in}}%
\pgfpathlineto{\pgfqpoint{2.740483in}{2.358033in}}%
\pgfpathclose%
\pgfusepath{fill}%
\end{pgfscope}%
\begin{pgfscope}%
\pgfpathrectangle{\pgfqpoint{0.730012in}{1.798407in}}{\pgfqpoint{3.981446in}{0.932710in}}%
\pgfusepath{clip}%
\pgfsetrectcap%
\pgfsetroundjoin%
\pgfsetlinewidth{0.803000pt}%
\definecolor{currentstroke}{rgb}{0.690196,0.690196,0.690196}%
\pgfsetstrokecolor{currentstroke}%
\pgfsetdash{}{0pt}%
\pgfpathmoveto{\pgfqpoint{0.730012in}{1.798407in}}%
\pgfpathlineto{\pgfqpoint{0.730012in}{2.731117in}}%
\pgfusepath{stroke}%
\end{pgfscope}%
\begin{pgfscope}%
\pgfpathrectangle{\pgfqpoint{0.730012in}{1.798407in}}{\pgfqpoint{3.981446in}{0.932710in}}%
\pgfusepath{clip}%
\pgfsetrectcap%
\pgfsetroundjoin%
\pgfsetlinewidth{0.803000pt}%
\definecolor{currentstroke}{rgb}{0.690196,0.690196,0.690196}%
\pgfsetstrokecolor{currentstroke}%
\pgfsetdash{}{0pt}%
\pgfpathmoveto{\pgfqpoint{1.227693in}{1.798407in}}%
\pgfpathlineto{\pgfqpoint{1.227693in}{2.731117in}}%
\pgfusepath{stroke}%
\end{pgfscope}%
\begin{pgfscope}%
\pgfpathrectangle{\pgfqpoint{0.730012in}{1.798407in}}{\pgfqpoint{3.981446in}{0.932710in}}%
\pgfusepath{clip}%
\pgfsetrectcap%
\pgfsetroundjoin%
\pgfsetlinewidth{0.803000pt}%
\definecolor{currentstroke}{rgb}{0.690196,0.690196,0.690196}%
\pgfsetstrokecolor{currentstroke}%
\pgfsetdash{}{0pt}%
\pgfpathmoveto{\pgfqpoint{1.725373in}{1.798407in}}%
\pgfpathlineto{\pgfqpoint{1.725373in}{2.731117in}}%
\pgfusepath{stroke}%
\end{pgfscope}%
\begin{pgfscope}%
\pgfpathrectangle{\pgfqpoint{0.730012in}{1.798407in}}{\pgfqpoint{3.981446in}{0.932710in}}%
\pgfusepath{clip}%
\pgfsetrectcap%
\pgfsetroundjoin%
\pgfsetlinewidth{0.803000pt}%
\definecolor{currentstroke}{rgb}{0.690196,0.690196,0.690196}%
\pgfsetstrokecolor{currentstroke}%
\pgfsetdash{}{0pt}%
\pgfpathmoveto{\pgfqpoint{2.223054in}{1.798407in}}%
\pgfpathlineto{\pgfqpoint{2.223054in}{2.731117in}}%
\pgfusepath{stroke}%
\end{pgfscope}%
\begin{pgfscope}%
\pgfpathrectangle{\pgfqpoint{0.730012in}{1.798407in}}{\pgfqpoint{3.981446in}{0.932710in}}%
\pgfusepath{clip}%
\pgfsetrectcap%
\pgfsetroundjoin%
\pgfsetlinewidth{0.803000pt}%
\definecolor{currentstroke}{rgb}{0.690196,0.690196,0.690196}%
\pgfsetstrokecolor{currentstroke}%
\pgfsetdash{}{0pt}%
\pgfpathmoveto{\pgfqpoint{2.720735in}{1.798407in}}%
\pgfpathlineto{\pgfqpoint{2.720735in}{2.731117in}}%
\pgfusepath{stroke}%
\end{pgfscope}%
\begin{pgfscope}%
\pgfpathrectangle{\pgfqpoint{0.730012in}{1.798407in}}{\pgfqpoint{3.981446in}{0.932710in}}%
\pgfusepath{clip}%
\pgfsetrectcap%
\pgfsetroundjoin%
\pgfsetlinewidth{0.803000pt}%
\definecolor{currentstroke}{rgb}{0.690196,0.690196,0.690196}%
\pgfsetstrokecolor{currentstroke}%
\pgfsetdash{}{0pt}%
\pgfpathmoveto{\pgfqpoint{3.218416in}{1.798407in}}%
\pgfpathlineto{\pgfqpoint{3.218416in}{2.731117in}}%
\pgfusepath{stroke}%
\end{pgfscope}%
\begin{pgfscope}%
\pgfpathrectangle{\pgfqpoint{0.730012in}{1.798407in}}{\pgfqpoint{3.981446in}{0.932710in}}%
\pgfusepath{clip}%
\pgfsetrectcap%
\pgfsetroundjoin%
\pgfsetlinewidth{0.803000pt}%
\definecolor{currentstroke}{rgb}{0.690196,0.690196,0.690196}%
\pgfsetstrokecolor{currentstroke}%
\pgfsetdash{}{0pt}%
\pgfpathmoveto{\pgfqpoint{3.716097in}{1.798407in}}%
\pgfpathlineto{\pgfqpoint{3.716097in}{2.731117in}}%
\pgfusepath{stroke}%
\end{pgfscope}%
\begin{pgfscope}%
\pgfpathrectangle{\pgfqpoint{0.730012in}{1.798407in}}{\pgfqpoint{3.981446in}{0.932710in}}%
\pgfusepath{clip}%
\pgfsetrectcap%
\pgfsetroundjoin%
\pgfsetlinewidth{0.803000pt}%
\definecolor{currentstroke}{rgb}{0.690196,0.690196,0.690196}%
\pgfsetstrokecolor{currentstroke}%
\pgfsetdash{}{0pt}%
\pgfpathmoveto{\pgfqpoint{4.213778in}{1.798407in}}%
\pgfpathlineto{\pgfqpoint{4.213778in}{2.731117in}}%
\pgfusepath{stroke}%
\end{pgfscope}%
\begin{pgfscope}%
\pgfpathrectangle{\pgfqpoint{0.730012in}{1.798407in}}{\pgfqpoint{3.981446in}{0.932710in}}%
\pgfusepath{clip}%
\pgfsetrectcap%
\pgfsetroundjoin%
\pgfsetlinewidth{0.803000pt}%
\definecolor{currentstroke}{rgb}{0.690196,0.690196,0.690196}%
\pgfsetstrokecolor{currentstroke}%
\pgfsetdash{}{0pt}%
\pgfpathmoveto{\pgfqpoint{4.711458in}{1.798407in}}%
\pgfpathlineto{\pgfqpoint{4.711458in}{2.731117in}}%
\pgfusepath{stroke}%
\end{pgfscope}%
\begin{pgfscope}%
\pgfpathrectangle{\pgfqpoint{0.730012in}{1.798407in}}{\pgfqpoint{3.981446in}{0.932710in}}%
\pgfusepath{clip}%
\pgfsetrectcap%
\pgfsetroundjoin%
\pgfsetlinewidth{0.803000pt}%
\definecolor{currentstroke}{rgb}{0.690196,0.690196,0.690196}%
\pgfsetstrokecolor{currentstroke}%
\pgfsetdash{}{0pt}%
\pgfpathmoveto{\pgfqpoint{0.730012in}{1.798407in}}%
\pgfpathlineto{\pgfqpoint{4.711458in}{1.798407in}}%
\pgfusepath{stroke}%
\end{pgfscope}%
\begin{pgfscope}%
\pgfsetbuttcap%
\pgfsetroundjoin%
\definecolor{currentfill}{rgb}{0.000000,0.000000,0.000000}%
\pgfsetfillcolor{currentfill}%
\pgfsetlinewidth{0.803000pt}%
\definecolor{currentstroke}{rgb}{0.000000,0.000000,0.000000}%
\pgfsetstrokecolor{currentstroke}%
\pgfsetdash{}{0pt}%
\pgfsys@defobject{currentmarker}{\pgfqpoint{-0.048611in}{0.000000in}}{\pgfqpoint{-0.000000in}{0.000000in}}{%
\pgfpathmoveto{\pgfqpoint{-0.000000in}{0.000000in}}%
\pgfpathlineto{\pgfqpoint{-0.048611in}{0.000000in}}%
\pgfusepath{stroke,fill}%
}%
\begin{pgfscope}%
\pgfsys@transformshift{0.730012in}{1.798407in}%
\pgfsys@useobject{currentmarker}{}%
\end{pgfscope}%
\end{pgfscope}%
\begin{pgfscope}%
\definecolor{textcolor}{rgb}{0.000000,0.000000,0.000000}%
\pgfsetstrokecolor{textcolor}%
\pgfsetfillcolor{textcolor}%
\pgftext[x=0.344787in, y=1.750182in, left, base]{\color{textcolor}\rmfamily\fontsize{10.000000}{12.000000}\selectfont \(\displaystyle {10^{-4}}\)}%
\end{pgfscope}%
\begin{pgfscope}%
\pgfpathrectangle{\pgfqpoint{0.730012in}{1.798407in}}{\pgfqpoint{3.981446in}{0.932710in}}%
\pgfusepath{clip}%
\pgfsetrectcap%
\pgfsetroundjoin%
\pgfsetlinewidth{0.803000pt}%
\definecolor{currentstroke}{rgb}{0.690196,0.690196,0.690196}%
\pgfsetstrokecolor{currentstroke}%
\pgfsetdash{}{0pt}%
\pgfpathmoveto{\pgfqpoint{0.730012in}{2.171491in}}%
\pgfpathlineto{\pgfqpoint{4.711458in}{2.171491in}}%
\pgfusepath{stroke}%
\end{pgfscope}%
\begin{pgfscope}%
\pgfsetbuttcap%
\pgfsetroundjoin%
\definecolor{currentfill}{rgb}{0.000000,0.000000,0.000000}%
\pgfsetfillcolor{currentfill}%
\pgfsetlinewidth{0.803000pt}%
\definecolor{currentstroke}{rgb}{0.000000,0.000000,0.000000}%
\pgfsetstrokecolor{currentstroke}%
\pgfsetdash{}{0pt}%
\pgfsys@defobject{currentmarker}{\pgfqpoint{-0.048611in}{0.000000in}}{\pgfqpoint{-0.000000in}{0.000000in}}{%
\pgfpathmoveto{\pgfqpoint{-0.000000in}{0.000000in}}%
\pgfpathlineto{\pgfqpoint{-0.048611in}{0.000000in}}%
\pgfusepath{stroke,fill}%
}%
\begin{pgfscope}%
\pgfsys@transformshift{0.730012in}{2.171491in}%
\pgfsys@useobject{currentmarker}{}%
\end{pgfscope}%
\end{pgfscope}%
\begin{pgfscope}%
\definecolor{textcolor}{rgb}{0.000000,0.000000,0.000000}%
\pgfsetstrokecolor{textcolor}%
\pgfsetfillcolor{textcolor}%
\pgftext[x=0.431593in, y=2.123266in, left, base]{\color{textcolor}\rmfamily\fontsize{10.000000}{12.000000}\selectfont \(\displaystyle {10^{0}}\)}%
\end{pgfscope}%
\begin{pgfscope}%
\pgfpathrectangle{\pgfqpoint{0.730012in}{1.798407in}}{\pgfqpoint{3.981446in}{0.932710in}}%
\pgfusepath{clip}%
\pgfsetrectcap%
\pgfsetroundjoin%
\pgfsetlinewidth{0.803000pt}%
\definecolor{currentstroke}{rgb}{0.690196,0.690196,0.690196}%
\pgfsetstrokecolor{currentstroke}%
\pgfsetdash{}{0pt}%
\pgfpathmoveto{\pgfqpoint{0.730012in}{2.544575in}}%
\pgfpathlineto{\pgfqpoint{4.711458in}{2.544575in}}%
\pgfusepath{stroke}%
\end{pgfscope}%
\begin{pgfscope}%
\pgfsetbuttcap%
\pgfsetroundjoin%
\definecolor{currentfill}{rgb}{0.000000,0.000000,0.000000}%
\pgfsetfillcolor{currentfill}%
\pgfsetlinewidth{0.803000pt}%
\definecolor{currentstroke}{rgb}{0.000000,0.000000,0.000000}%
\pgfsetstrokecolor{currentstroke}%
\pgfsetdash{}{0pt}%
\pgfsys@defobject{currentmarker}{\pgfqpoint{-0.048611in}{0.000000in}}{\pgfqpoint{-0.000000in}{0.000000in}}{%
\pgfpathmoveto{\pgfqpoint{-0.000000in}{0.000000in}}%
\pgfpathlineto{\pgfqpoint{-0.048611in}{0.000000in}}%
\pgfusepath{stroke,fill}%
}%
\begin{pgfscope}%
\pgfsys@transformshift{0.730012in}{2.544575in}%
\pgfsys@useobject{currentmarker}{}%
\end{pgfscope}%
\end{pgfscope}%
\begin{pgfscope}%
\definecolor{textcolor}{rgb}{0.000000,0.000000,0.000000}%
\pgfsetstrokecolor{textcolor}%
\pgfsetfillcolor{textcolor}%
\pgftext[x=0.431593in, y=2.496350in, left, base]{\color{textcolor}\rmfamily\fontsize{10.000000}{12.000000}\selectfont \(\displaystyle {10^{4}}\)}%
\end{pgfscope}%
\begin{pgfscope}%
\definecolor{textcolor}{rgb}{0.000000,0.000000,0.000000}%
\pgfsetstrokecolor{textcolor}%
\pgfsetfillcolor{textcolor}%
\pgftext[x=0.289232in,y=2.264762in,,bottom,rotate=90.000000]{\color{textcolor}\rmfamily\fontsize{10.000000}{12.000000}\selectfont \(\displaystyle |F_N(w)|^2\)}%
\end{pgfscope}%
\begin{pgfscope}%
\pgfpathrectangle{\pgfqpoint{0.730012in}{1.798407in}}{\pgfqpoint{3.981446in}{0.932710in}}%
\pgfusepath{clip}%
\pgfsetrectcap%
\pgfsetroundjoin%
\pgfsetlinewidth{1.003750pt}%
\definecolor{currentstroke}{rgb}{0.000000,0.501961,0.000000}%
\pgfsetstrokecolor{currentstroke}%
\pgfsetdash{}{0pt}%
\pgfpathmoveto{\pgfqpoint{0.737191in}{1.784518in}}%
\pgfpathlineto{\pgfqpoint{0.742958in}{1.832403in}}%
\pgfpathlineto{\pgfqpoint{0.750925in}{1.871252in}}%
\pgfpathlineto{\pgfqpoint{0.759888in}{1.900142in}}%
\pgfpathlineto{\pgfqpoint{0.770842in}{1.925439in}}%
\pgfpathlineto{\pgfqpoint{0.783788in}{1.947737in}}%
\pgfpathlineto{\pgfqpoint{0.799722in}{1.968737in}}%
\pgfpathlineto{\pgfqpoint{0.817647in}{1.987242in}}%
\pgfpathlineto{\pgfqpoint{0.839556in}{2.005268in}}%
\pgfpathlineto{\pgfqpoint{0.865449in}{2.022379in}}%
\pgfpathlineto{\pgfqpoint{0.896320in}{2.038901in}}%
\pgfpathlineto{\pgfqpoint{0.933167in}{2.054946in}}%
\pgfpathlineto{\pgfqpoint{0.976985in}{2.070524in}}%
\pgfpathlineto{\pgfqpoint{1.028770in}{2.085590in}}%
\pgfpathlineto{\pgfqpoint{1.090513in}{2.100289in}}%
\pgfpathlineto{\pgfqpoint{1.164207in}{2.114587in}}%
\pgfpathlineto{\pgfqpoint{1.251842in}{2.128336in}}%
\pgfpathlineto{\pgfqpoint{1.353420in}{2.141062in}}%
\pgfpathlineto{\pgfqpoint{1.468940in}{2.152368in}}%
\pgfpathlineto{\pgfqpoint{1.595414in}{2.161615in}}%
\pgfpathlineto{\pgfqpoint{1.725871in}{2.168064in}}%
\pgfpathlineto{\pgfqpoint{1.850354in}{2.171188in}}%
\pgfpathlineto{\pgfqpoint{1.961890in}{2.170972in}}%
\pgfpathlineto{\pgfqpoint{2.056497in}{2.167777in}}%
\pgfpathlineto{\pgfqpoint{2.134174in}{2.162158in}}%
\pgfpathlineto{\pgfqpoint{2.196913in}{2.154666in}}%
\pgfpathlineto{\pgfqpoint{2.248698in}{2.145463in}}%
\pgfpathlineto{\pgfqpoint{2.290524in}{2.134993in}}%
\pgfpathlineto{\pgfqpoint{2.324383in}{2.123493in}}%
\pgfpathlineto{\pgfqpoint{2.352267in}{2.110935in}}%
\pgfpathlineto{\pgfqpoint{2.375172in}{2.097452in}}%
\pgfpathlineto{\pgfqpoint{2.395089in}{2.082154in}}%
\pgfpathlineto{\pgfqpoint{2.411023in}{2.066192in}}%
\pgfpathlineto{\pgfqpoint{2.424965in}{2.047880in}}%
\pgfpathlineto{\pgfqpoint{2.436915in}{2.026837in}}%
\pgfpathlineto{\pgfqpoint{2.446874in}{2.002671in}}%
\pgfpathlineto{\pgfqpoint{2.454841in}{1.975182in}}%
\pgfpathlineto{\pgfqpoint{2.461812in}{1.938571in}}%
\pgfpathlineto{\pgfqpoint{2.466791in}{1.894821in}}%
\pgfpathlineto{\pgfqpoint{2.469778in}{1.848425in}}%
\pgfpathlineto{\pgfqpoint{2.471839in}{1.784518in}}%
\pgfpathmoveto{\pgfqpoint{2.475385in}{1.784518in}}%
\pgfpathlineto{\pgfqpoint{2.477745in}{1.856374in}}%
\pgfpathlineto{\pgfqpoint{2.482725in}{1.920971in}}%
\pgfpathlineto{\pgfqpoint{2.489696in}{1.967944in}}%
\pgfpathlineto{\pgfqpoint{2.497663in}{2.001653in}}%
\pgfpathlineto{\pgfqpoint{2.507621in}{2.031108in}}%
\pgfpathlineto{\pgfqpoint{2.519571in}{2.057133in}}%
\pgfpathlineto{\pgfqpoint{2.533513in}{2.080426in}}%
\pgfpathlineto{\pgfqpoint{2.550443in}{2.102664in}}%
\pgfpathlineto{\pgfqpoint{2.569364in}{2.122536in}}%
\pgfpathlineto{\pgfqpoint{2.591273in}{2.141008in}}%
\pgfpathlineto{\pgfqpoint{2.613182in}{2.155578in}}%
\pgfpathlineto{\pgfqpoint{2.633099in}{2.165435in}}%
\pgfpathlineto{\pgfqpoint{2.650029in}{2.170549in}}%
\pgfpathlineto{\pgfqpoint{2.662975in}{2.171353in}}%
\pgfpathlineto{\pgfqpoint{2.672934in}{2.168930in}}%
\pgfpathlineto{\pgfqpoint{2.680901in}{2.163776in}}%
\pgfpathlineto{\pgfqpoint{2.687872in}{2.155218in}}%
\pgfpathlineto{\pgfqpoint{2.693847in}{2.142465in}}%
\pgfpathlineto{\pgfqpoint{2.698826in}{2.124457in}}%
\pgfpathlineto{\pgfqpoint{2.702810in}{2.099873in}}%
\pgfpathlineto{\pgfqpoint{2.706793in}{2.050915in}}%
\pgfpathlineto{\pgfqpoint{2.708785in}{1.994774in}}%
\pgfpathlineto{\pgfqpoint{2.710776in}{1.833392in}}%
\pgfpathlineto{\pgfqpoint{2.713764in}{2.056695in}}%
\pgfpathlineto{\pgfqpoint{2.718743in}{2.147396in}}%
\pgfpathlineto{\pgfqpoint{2.720735in}{2.171491in}}%
\pgfpathlineto{\pgfqpoint{2.720735in}{2.171491in}}%
\pgfusepath{stroke}%
\end{pgfscope}%
\begin{pgfscope}%
\pgfpathrectangle{\pgfqpoint{0.730012in}{1.798407in}}{\pgfqpoint{3.981446in}{0.932710in}}%
\pgfusepath{clip}%
\pgfsetrectcap%
\pgfsetroundjoin%
\pgfsetlinewidth{1.003750pt}%
\definecolor{currentstroke}{rgb}{1.000000,0.647059,0.000000}%
\pgfsetstrokecolor{currentstroke}%
\pgfsetdash{}{0pt}%
\pgfpathmoveto{\pgfqpoint{2.720735in}{2.171491in}}%
\pgfpathlineto{\pgfqpoint{2.730416in}{2.263242in}}%
\pgfpathlineto{\pgfqpoint{2.740483in}{2.357944in}}%
\pgfpathlineto{\pgfqpoint{2.740483in}{2.357944in}}%
\pgfusepath{stroke}%
\end{pgfscope}%
\begin{pgfscope}%
\pgfpathrectangle{\pgfqpoint{0.730012in}{1.798407in}}{\pgfqpoint{3.981446in}{0.932710in}}%
\pgfusepath{clip}%
\pgfsetrectcap%
\pgfsetroundjoin%
\pgfsetlinewidth{1.003750pt}%
\definecolor{currentstroke}{rgb}{1.000000,0.000000,0.000000}%
\pgfsetstrokecolor{currentstroke}%
\pgfsetdash{}{0pt}%
\pgfpathmoveto{\pgfqpoint{2.740483in}{2.357944in}}%
\pgfpathlineto{\pgfqpoint{2.745413in}{2.425975in}}%
\pgfpathlineto{\pgfqpoint{2.748371in}{2.498777in}}%
\pgfpathlineto{\pgfqpoint{2.750343in}{2.638009in}}%
\pgfpathlineto{\pgfqpoint{2.751329in}{2.632169in}}%
\pgfpathlineto{\pgfqpoint{2.754287in}{2.485572in}}%
\pgfpathlineto{\pgfqpoint{2.758231in}{2.433629in}}%
\pgfpathlineto{\pgfqpoint{2.763161in}{2.403439in}}%
\pgfpathlineto{\pgfqpoint{2.769076in}{2.384105in}}%
\pgfpathlineto{\pgfqpoint{2.775978in}{2.371545in}}%
\pgfpathlineto{\pgfqpoint{2.783866in}{2.363704in}}%
\pgfpathlineto{\pgfqpoint{2.792740in}{2.359387in}}%
\pgfpathlineto{\pgfqpoint{2.803586in}{2.357822in}}%
\pgfpathlineto{\pgfqpoint{2.817390in}{2.359308in}}%
\pgfpathlineto{\pgfqpoint{2.835137in}{2.364651in}}%
\pgfpathlineto{\pgfqpoint{2.856829in}{2.374526in}}%
\pgfpathlineto{\pgfqpoint{2.882464in}{2.389713in}}%
\pgfpathlineto{\pgfqpoint{2.908100in}{2.408525in}}%
\pgfpathlineto{\pgfqpoint{2.930777in}{2.428838in}}%
\pgfpathlineto{\pgfqpoint{2.950497in}{2.450483in}}%
\pgfpathlineto{\pgfqpoint{2.967259in}{2.473386in}}%
\pgfpathlineto{\pgfqpoint{2.981062in}{2.497332in}}%
\pgfpathlineto{\pgfqpoint{2.992894in}{2.524185in}}%
\pgfpathlineto{\pgfqpoint{3.002754in}{2.554872in}}%
\pgfpathlineto{\pgfqpoint{3.010642in}{2.590647in}}%
\pgfpathlineto{\pgfqpoint{3.016558in}{2.632873in}}%
\pgfpathlineto{\pgfqpoint{3.020502in}{2.681015in}}%
\pgfpathlineto{\pgfqpoint{3.023145in}{2.745006in}}%
\pgfpathmoveto{\pgfqpoint{3.027755in}{2.745006in}}%
\pgfpathlineto{\pgfqpoint{3.031347in}{2.668074in}}%
\pgfpathlineto{\pgfqpoint{3.037263in}{2.612733in}}%
\pgfpathlineto{\pgfqpoint{3.044165in}{2.576495in}}%
\pgfpathlineto{\pgfqpoint{3.053039in}{2.546365in}}%
\pgfpathlineto{\pgfqpoint{3.063885in}{2.521091in}}%
\pgfpathlineto{\pgfqpoint{3.076702in}{2.499633in}}%
\pgfpathlineto{\pgfqpoint{3.091492in}{2.481219in}}%
\pgfpathlineto{\pgfqpoint{3.108254in}{2.465278in}}%
\pgfpathlineto{\pgfqpoint{3.127973in}{2.450741in}}%
\pgfpathlineto{\pgfqpoint{3.151637in}{2.437187in}}%
\pgfpathlineto{\pgfqpoint{3.180230in}{2.424515in}}%
\pgfpathlineto{\pgfqpoint{3.214740in}{2.412770in}}%
\pgfpathlineto{\pgfqpoint{3.256151in}{2.402052in}}%
\pgfpathlineto{\pgfqpoint{3.307422in}{2.392129in}}%
\pgfpathlineto{\pgfqpoint{3.370525in}{2.383222in}}%
\pgfpathlineto{\pgfqpoint{3.449403in}{2.375370in}}%
\pgfpathlineto{\pgfqpoint{3.548987in}{2.368720in}}%
\pgfpathlineto{\pgfqpoint{3.677165in}{2.363420in}}%
\pgfpathlineto{\pgfqpoint{3.845767in}{2.359706in}}%
\pgfpathlineto{\pgfqpoint{4.076487in}{2.357907in}}%
\pgfpathlineto{\pgfqpoint{4.409748in}{2.358631in}}%
\pgfpathlineto{\pgfqpoint{4.711458in}{2.360903in}}%
\pgfpathlineto{\pgfqpoint{4.711458in}{2.360903in}}%
\pgfusepath{stroke}%
\end{pgfscope}%
\begin{pgfscope}%
\pgfsetbuttcap%
\pgfsetroundjoin%
\definecolor{currentfill}{rgb}{0.000000,0.000000,0.000000}%
\pgfsetfillcolor{currentfill}%
\pgfsetfillopacity{0.000000}%
\pgfsetlinewidth{1.003750pt}%
\definecolor{currentstroke}{rgb}{0.000000,0.000000,0.000000}%
\pgfsetstrokecolor{currentstroke}%
\pgfsetdash{}{0pt}%
\pgfsys@defobject{currentmarker}{\pgfqpoint{-0.041667in}{-0.041667in}}{\pgfqpoint{0.041667in}{0.041667in}}{%
\pgfpathmoveto{\pgfqpoint{0.000000in}{-0.041667in}}%
\pgfpathcurveto{\pgfqpoint{0.011050in}{-0.041667in}}{\pgfqpoint{0.021649in}{-0.037276in}}{\pgfqpoint{0.029463in}{-0.029463in}}%
\pgfpathcurveto{\pgfqpoint{0.037276in}{-0.021649in}}{\pgfqpoint{0.041667in}{-0.011050in}}{\pgfqpoint{0.041667in}{0.000000in}}%
\pgfpathcurveto{\pgfqpoint{0.041667in}{0.011050in}}{\pgfqpoint{0.037276in}{0.021649in}}{\pgfqpoint{0.029463in}{0.029463in}}%
\pgfpathcurveto{\pgfqpoint{0.021649in}{0.037276in}}{\pgfqpoint{0.011050in}{0.041667in}}{\pgfqpoint{0.000000in}{0.041667in}}%
\pgfpathcurveto{\pgfqpoint{-0.011050in}{0.041667in}}{\pgfqpoint{-0.021649in}{0.037276in}}{\pgfqpoint{-0.029463in}{0.029463in}}%
\pgfpathcurveto{\pgfqpoint{-0.037276in}{0.021649in}}{\pgfqpoint{-0.041667in}{0.011050in}}{\pgfqpoint{-0.041667in}{0.000000in}}%
\pgfpathcurveto{\pgfqpoint{-0.041667in}{-0.011050in}}{\pgfqpoint{-0.037276in}{-0.021649in}}{\pgfqpoint{-0.029463in}{-0.029463in}}%
\pgfpathcurveto{\pgfqpoint{-0.021649in}{-0.037276in}}{\pgfqpoint{-0.011050in}{-0.041667in}}{\pgfqpoint{0.000000in}{-0.041667in}}%
\pgfpathlineto{\pgfqpoint{0.000000in}{-0.041667in}}%
\pgfpathclose%
\pgfusepath{stroke,fill}%
}%
\begin{pgfscope}%
\pgfsys@transformshift{0.730012in}{1.728454in}%
\pgfsys@useobject{currentmarker}{}%
\end{pgfscope}%
\begin{pgfscope}%
\pgfsys@transformshift{2.461941in}{1.728454in}%
\pgfsys@useobject{currentmarker}{}%
\end{pgfscope}%
\begin{pgfscope}%
\pgfsys@transformshift{2.710781in}{1.728454in}%
\pgfsys@useobject{currentmarker}{}%
\end{pgfscope}%
\end{pgfscope}%
\begin{pgfscope}%
\pgfsetbuttcap%
\pgfsetroundjoin%
\definecolor{currentfill}{rgb}{0.000000,0.000000,0.000000}%
\pgfsetfillcolor{currentfill}%
\pgfsetfillopacity{0.000000}%
\pgfsetlinewidth{1.003750pt}%
\definecolor{currentstroke}{rgb}{0.000000,0.000000,0.000000}%
\pgfsetstrokecolor{currentstroke}%
\pgfsetdash{}{0pt}%
\pgfsys@defobject{currentmarker}{\pgfqpoint{-0.041667in}{-0.041667in}}{\pgfqpoint{0.041667in}{0.041667in}}{%
\pgfpathmoveto{\pgfqpoint{-0.041667in}{-0.041667in}}%
\pgfpathlineto{\pgfqpoint{0.041667in}{0.041667in}}%
\pgfpathmoveto{\pgfqpoint{-0.041667in}{0.041667in}}%
\pgfpathlineto{\pgfqpoint{0.041667in}{-0.041667in}}%
\pgfusepath{stroke,fill}%
}%
\begin{pgfscope}%
\pgfsys@transformshift{2.740642in}{2.801071in}%
\pgfsys@useobject{currentmarker}{}%
\end{pgfscope}%
\begin{pgfscope}%
\pgfsys@transformshift{3.029297in}{2.801071in}%
\pgfsys@useobject{currentmarker}{}%
\end{pgfscope}%
\begin{pgfscope}%
\pgfsys@transformshift{4.810994in}{2.801071in}%
\pgfsys@useobject{currentmarker}{}%
\end{pgfscope}%
\end{pgfscope}%
\begin{pgfscope}%
\pgfsetrectcap%
\pgfsetmiterjoin%
\pgfsetlinewidth{0.803000pt}%
\definecolor{currentstroke}{rgb}{0.000000,0.000000,0.000000}%
\pgfsetstrokecolor{currentstroke}%
\pgfsetdash{}{0pt}%
\pgfpathmoveto{\pgfqpoint{0.730012in}{1.798407in}}%
\pgfpathlineto{\pgfqpoint{0.730012in}{2.731117in}}%
\pgfusepath{stroke}%
\end{pgfscope}%
\begin{pgfscope}%
\pgfsetrectcap%
\pgfsetmiterjoin%
\pgfsetlinewidth{0.803000pt}%
\definecolor{currentstroke}{rgb}{0.000000,0.000000,0.000000}%
\pgfsetstrokecolor{currentstroke}%
\pgfsetdash{}{0pt}%
\pgfpathmoveto{\pgfqpoint{4.711458in}{1.798407in}}%
\pgfpathlineto{\pgfqpoint{4.711458in}{2.731117in}}%
\pgfusepath{stroke}%
\end{pgfscope}%
\begin{pgfscope}%
\pgfsetrectcap%
\pgfsetmiterjoin%
\pgfsetlinewidth{0.803000pt}%
\definecolor{currentstroke}{rgb}{0.000000,0.000000,0.000000}%
\pgfsetstrokecolor{currentstroke}%
\pgfsetdash{}{0pt}%
\pgfpathmoveto{\pgfqpoint{0.730012in}{1.798407in}}%
\pgfpathlineto{\pgfqpoint{4.711458in}{1.798407in}}%
\pgfusepath{stroke}%
\end{pgfscope}%
\begin{pgfscope}%
\pgfsetrectcap%
\pgfsetmiterjoin%
\pgfsetlinewidth{0.803000pt}%
\definecolor{currentstroke}{rgb}{0.000000,0.000000,0.000000}%
\pgfsetstrokecolor{currentstroke}%
\pgfsetdash{}{0pt}%
\pgfpathmoveto{\pgfqpoint{0.730012in}{2.731117in}}%
\pgfpathlineto{\pgfqpoint{4.711458in}{2.731117in}}%
\pgfusepath{stroke}%
\end{pgfscope}%
\begin{pgfscope}%
\pgfsetbuttcap%
\pgfsetmiterjoin%
\definecolor{currentfill}{rgb}{1.000000,1.000000,1.000000}%
\pgfsetfillcolor{currentfill}%
\pgfsetlinewidth{0.000000pt}%
\definecolor{currentstroke}{rgb}{0.000000,0.000000,0.000000}%
\pgfsetstrokecolor{currentstroke}%
\pgfsetstrokeopacity{0.000000}%
\pgfsetdash{}{0pt}%
\pgfpathmoveto{\pgfqpoint{0.730012in}{0.548769in}}%
\pgfpathlineto{\pgfqpoint{4.711458in}{0.548769in}}%
\pgfpathlineto{\pgfqpoint{4.711458in}{1.481479in}}%
\pgfpathlineto{\pgfqpoint{0.730012in}{1.481479in}}%
\pgfpathlineto{\pgfqpoint{0.730012in}{0.548769in}}%
\pgfpathclose%
\pgfusepath{fill}%
\end{pgfscope}%
\begin{pgfscope}%
\pgfpathrectangle{\pgfqpoint{0.730012in}{0.548769in}}{\pgfqpoint{3.981446in}{0.932710in}}%
\pgfusepath{clip}%
\pgfsetbuttcap%
\pgfsetmiterjoin%
\definecolor{currentfill}{rgb}{0.000000,0.501961,0.000000}%
\pgfsetfillcolor{currentfill}%
\pgfsetfillopacity{0.200000}%
\pgfsetlinewidth{0.000000pt}%
\definecolor{currentstroke}{rgb}{0.000000,0.000000,0.000000}%
\pgfsetstrokecolor{currentstroke}%
\pgfsetstrokeopacity{0.200000}%
\pgfsetdash{}{0pt}%
\pgfpathmoveto{\pgfqpoint{0.730012in}{1.208295in}}%
\pgfpathlineto{\pgfqpoint{2.720735in}{1.208295in}}%
\pgfpathlineto{\pgfqpoint{2.720735in}{2.141005in}}%
\pgfpathlineto{\pgfqpoint{0.730012in}{2.141005in}}%
\pgfpathlineto{\pgfqpoint{0.730012in}{1.208295in}}%
\pgfpathclose%
\pgfusepath{fill}%
\end{pgfscope}%
\begin{pgfscope}%
\pgfpathrectangle{\pgfqpoint{0.730012in}{0.548769in}}{\pgfqpoint{3.981446in}{0.932710in}}%
\pgfusepath{clip}%
\pgfsetbuttcap%
\pgfsetmiterjoin%
\definecolor{currentfill}{rgb}{1.000000,0.647059,0.000000}%
\pgfsetfillcolor{currentfill}%
\pgfsetfillopacity{0.200000}%
\pgfsetlinewidth{0.000000pt}%
\definecolor{currentstroke}{rgb}{0.000000,0.000000,0.000000}%
\pgfsetstrokecolor{currentstroke}%
\pgfsetstrokeopacity{0.200000}%
\pgfsetdash{}{0pt}%
\pgfpathmoveto{\pgfqpoint{2.720735in}{0.642040in}}%
\pgfpathlineto{\pgfqpoint{2.740483in}{0.642040in}}%
\pgfpathlineto{\pgfqpoint{2.740483in}{1.208295in}}%
\pgfpathlineto{\pgfqpoint{2.720735in}{1.208295in}}%
\pgfpathlineto{\pgfqpoint{2.720735in}{0.642040in}}%
\pgfpathclose%
\pgfusepath{fill}%
\end{pgfscope}%
\begin{pgfscope}%
\pgfpathrectangle{\pgfqpoint{0.730012in}{0.548769in}}{\pgfqpoint{3.981446in}{0.932710in}}%
\pgfusepath{clip}%
\pgfsetbuttcap%
\pgfsetmiterjoin%
\definecolor{currentfill}{rgb}{1.000000,0.000000,0.000000}%
\pgfsetfillcolor{currentfill}%
\pgfsetfillopacity{0.200000}%
\pgfsetlinewidth{0.000000pt}%
\definecolor{currentstroke}{rgb}{0.000000,0.000000,0.000000}%
\pgfsetstrokecolor{currentstroke}%
\pgfsetstrokeopacity{0.200000}%
\pgfsetdash{}{0pt}%
\pgfpathmoveto{\pgfqpoint{2.740483in}{0.548769in}}%
\pgfpathlineto{\pgfqpoint{4.731206in}{0.548769in}}%
\pgfpathlineto{\pgfqpoint{4.731206in}{0.642040in}}%
\pgfpathlineto{\pgfqpoint{2.740483in}{0.642040in}}%
\pgfpathlineto{\pgfqpoint{2.740483in}{0.548769in}}%
\pgfpathclose%
\pgfusepath{fill}%
\end{pgfscope}%
\begin{pgfscope}%
\pgfpathrectangle{\pgfqpoint{0.730012in}{0.548769in}}{\pgfqpoint{3.981446in}{0.932710in}}%
\pgfusepath{clip}%
\pgfsetrectcap%
\pgfsetroundjoin%
\pgfsetlinewidth{0.803000pt}%
\definecolor{currentstroke}{rgb}{0.690196,0.690196,0.690196}%
\pgfsetstrokecolor{currentstroke}%
\pgfsetdash{}{0pt}%
\pgfpathmoveto{\pgfqpoint{0.730012in}{0.548769in}}%
\pgfpathlineto{\pgfqpoint{0.730012in}{1.481479in}}%
\pgfusepath{stroke}%
\end{pgfscope}%
\begin{pgfscope}%
\pgfsetbuttcap%
\pgfsetroundjoin%
\definecolor{currentfill}{rgb}{0.000000,0.000000,0.000000}%
\pgfsetfillcolor{currentfill}%
\pgfsetlinewidth{0.803000pt}%
\definecolor{currentstroke}{rgb}{0.000000,0.000000,0.000000}%
\pgfsetstrokecolor{currentstroke}%
\pgfsetdash{}{0pt}%
\pgfsys@defobject{currentmarker}{\pgfqpoint{0.000000in}{-0.048611in}}{\pgfqpoint{0.000000in}{0.000000in}}{%
\pgfpathmoveto{\pgfqpoint{0.000000in}{0.000000in}}%
\pgfpathlineto{\pgfqpoint{0.000000in}{-0.048611in}}%
\pgfusepath{stroke,fill}%
}%
\begin{pgfscope}%
\pgfsys@transformshift{0.730012in}{0.548769in}%
\pgfsys@useobject{currentmarker}{}%
\end{pgfscope}%
\end{pgfscope}%
\begin{pgfscope}%
\definecolor{textcolor}{rgb}{0.000000,0.000000,0.000000}%
\pgfsetstrokecolor{textcolor}%
\pgfsetfillcolor{textcolor}%
\pgftext[x=0.730012in,y=0.451547in,,top]{\color{textcolor}\rmfamily\fontsize{10.000000}{12.000000}\selectfont \(\displaystyle {0.00}\)}%
\end{pgfscope}%
\begin{pgfscope}%
\pgfpathrectangle{\pgfqpoint{0.730012in}{0.548769in}}{\pgfqpoint{3.981446in}{0.932710in}}%
\pgfusepath{clip}%
\pgfsetrectcap%
\pgfsetroundjoin%
\pgfsetlinewidth{0.803000pt}%
\definecolor{currentstroke}{rgb}{0.690196,0.690196,0.690196}%
\pgfsetstrokecolor{currentstroke}%
\pgfsetdash{}{0pt}%
\pgfpathmoveto{\pgfqpoint{1.227693in}{0.548769in}}%
\pgfpathlineto{\pgfqpoint{1.227693in}{1.481479in}}%
\pgfusepath{stroke}%
\end{pgfscope}%
\begin{pgfscope}%
\pgfsetbuttcap%
\pgfsetroundjoin%
\definecolor{currentfill}{rgb}{0.000000,0.000000,0.000000}%
\pgfsetfillcolor{currentfill}%
\pgfsetlinewidth{0.803000pt}%
\definecolor{currentstroke}{rgb}{0.000000,0.000000,0.000000}%
\pgfsetstrokecolor{currentstroke}%
\pgfsetdash{}{0pt}%
\pgfsys@defobject{currentmarker}{\pgfqpoint{0.000000in}{-0.048611in}}{\pgfqpoint{0.000000in}{0.000000in}}{%
\pgfpathmoveto{\pgfqpoint{0.000000in}{0.000000in}}%
\pgfpathlineto{\pgfqpoint{0.000000in}{-0.048611in}}%
\pgfusepath{stroke,fill}%
}%
\begin{pgfscope}%
\pgfsys@transformshift{1.227693in}{0.548769in}%
\pgfsys@useobject{currentmarker}{}%
\end{pgfscope}%
\end{pgfscope}%
\begin{pgfscope}%
\definecolor{textcolor}{rgb}{0.000000,0.000000,0.000000}%
\pgfsetstrokecolor{textcolor}%
\pgfsetfillcolor{textcolor}%
\pgftext[x=1.227693in,y=0.451547in,,top]{\color{textcolor}\rmfamily\fontsize{10.000000}{12.000000}\selectfont \(\displaystyle {0.25}\)}%
\end{pgfscope}%
\begin{pgfscope}%
\pgfpathrectangle{\pgfqpoint{0.730012in}{0.548769in}}{\pgfqpoint{3.981446in}{0.932710in}}%
\pgfusepath{clip}%
\pgfsetrectcap%
\pgfsetroundjoin%
\pgfsetlinewidth{0.803000pt}%
\definecolor{currentstroke}{rgb}{0.690196,0.690196,0.690196}%
\pgfsetstrokecolor{currentstroke}%
\pgfsetdash{}{0pt}%
\pgfpathmoveto{\pgfqpoint{1.725373in}{0.548769in}}%
\pgfpathlineto{\pgfqpoint{1.725373in}{1.481479in}}%
\pgfusepath{stroke}%
\end{pgfscope}%
\begin{pgfscope}%
\pgfsetbuttcap%
\pgfsetroundjoin%
\definecolor{currentfill}{rgb}{0.000000,0.000000,0.000000}%
\pgfsetfillcolor{currentfill}%
\pgfsetlinewidth{0.803000pt}%
\definecolor{currentstroke}{rgb}{0.000000,0.000000,0.000000}%
\pgfsetstrokecolor{currentstroke}%
\pgfsetdash{}{0pt}%
\pgfsys@defobject{currentmarker}{\pgfqpoint{0.000000in}{-0.048611in}}{\pgfqpoint{0.000000in}{0.000000in}}{%
\pgfpathmoveto{\pgfqpoint{0.000000in}{0.000000in}}%
\pgfpathlineto{\pgfqpoint{0.000000in}{-0.048611in}}%
\pgfusepath{stroke,fill}%
}%
\begin{pgfscope}%
\pgfsys@transformshift{1.725373in}{0.548769in}%
\pgfsys@useobject{currentmarker}{}%
\end{pgfscope}%
\end{pgfscope}%
\begin{pgfscope}%
\definecolor{textcolor}{rgb}{0.000000,0.000000,0.000000}%
\pgfsetstrokecolor{textcolor}%
\pgfsetfillcolor{textcolor}%
\pgftext[x=1.725373in,y=0.451547in,,top]{\color{textcolor}\rmfamily\fontsize{10.000000}{12.000000}\selectfont \(\displaystyle {0.50}\)}%
\end{pgfscope}%
\begin{pgfscope}%
\pgfpathrectangle{\pgfqpoint{0.730012in}{0.548769in}}{\pgfqpoint{3.981446in}{0.932710in}}%
\pgfusepath{clip}%
\pgfsetrectcap%
\pgfsetroundjoin%
\pgfsetlinewidth{0.803000pt}%
\definecolor{currentstroke}{rgb}{0.690196,0.690196,0.690196}%
\pgfsetstrokecolor{currentstroke}%
\pgfsetdash{}{0pt}%
\pgfpathmoveto{\pgfqpoint{2.223054in}{0.548769in}}%
\pgfpathlineto{\pgfqpoint{2.223054in}{1.481479in}}%
\pgfusepath{stroke}%
\end{pgfscope}%
\begin{pgfscope}%
\pgfsetbuttcap%
\pgfsetroundjoin%
\definecolor{currentfill}{rgb}{0.000000,0.000000,0.000000}%
\pgfsetfillcolor{currentfill}%
\pgfsetlinewidth{0.803000pt}%
\definecolor{currentstroke}{rgb}{0.000000,0.000000,0.000000}%
\pgfsetstrokecolor{currentstroke}%
\pgfsetdash{}{0pt}%
\pgfsys@defobject{currentmarker}{\pgfqpoint{0.000000in}{-0.048611in}}{\pgfqpoint{0.000000in}{0.000000in}}{%
\pgfpathmoveto{\pgfqpoint{0.000000in}{0.000000in}}%
\pgfpathlineto{\pgfqpoint{0.000000in}{-0.048611in}}%
\pgfusepath{stroke,fill}%
}%
\begin{pgfscope}%
\pgfsys@transformshift{2.223054in}{0.548769in}%
\pgfsys@useobject{currentmarker}{}%
\end{pgfscope}%
\end{pgfscope}%
\begin{pgfscope}%
\definecolor{textcolor}{rgb}{0.000000,0.000000,0.000000}%
\pgfsetstrokecolor{textcolor}%
\pgfsetfillcolor{textcolor}%
\pgftext[x=2.223054in,y=0.451547in,,top]{\color{textcolor}\rmfamily\fontsize{10.000000}{12.000000}\selectfont \(\displaystyle {0.75}\)}%
\end{pgfscope}%
\begin{pgfscope}%
\pgfpathrectangle{\pgfqpoint{0.730012in}{0.548769in}}{\pgfqpoint{3.981446in}{0.932710in}}%
\pgfusepath{clip}%
\pgfsetrectcap%
\pgfsetroundjoin%
\pgfsetlinewidth{0.803000pt}%
\definecolor{currentstroke}{rgb}{0.690196,0.690196,0.690196}%
\pgfsetstrokecolor{currentstroke}%
\pgfsetdash{}{0pt}%
\pgfpathmoveto{\pgfqpoint{2.720735in}{0.548769in}}%
\pgfpathlineto{\pgfqpoint{2.720735in}{1.481479in}}%
\pgfusepath{stroke}%
\end{pgfscope}%
\begin{pgfscope}%
\pgfsetbuttcap%
\pgfsetroundjoin%
\definecolor{currentfill}{rgb}{0.000000,0.000000,0.000000}%
\pgfsetfillcolor{currentfill}%
\pgfsetlinewidth{0.803000pt}%
\definecolor{currentstroke}{rgb}{0.000000,0.000000,0.000000}%
\pgfsetstrokecolor{currentstroke}%
\pgfsetdash{}{0pt}%
\pgfsys@defobject{currentmarker}{\pgfqpoint{0.000000in}{-0.048611in}}{\pgfqpoint{0.000000in}{0.000000in}}{%
\pgfpathmoveto{\pgfqpoint{0.000000in}{0.000000in}}%
\pgfpathlineto{\pgfqpoint{0.000000in}{-0.048611in}}%
\pgfusepath{stroke,fill}%
}%
\begin{pgfscope}%
\pgfsys@transformshift{2.720735in}{0.548769in}%
\pgfsys@useobject{currentmarker}{}%
\end{pgfscope}%
\end{pgfscope}%
\begin{pgfscope}%
\definecolor{textcolor}{rgb}{0.000000,0.000000,0.000000}%
\pgfsetstrokecolor{textcolor}%
\pgfsetfillcolor{textcolor}%
\pgftext[x=2.720735in,y=0.451547in,,top]{\color{textcolor}\rmfamily\fontsize{10.000000}{12.000000}\selectfont \(\displaystyle {1.00}\)}%
\end{pgfscope}%
\begin{pgfscope}%
\pgfpathrectangle{\pgfqpoint{0.730012in}{0.548769in}}{\pgfqpoint{3.981446in}{0.932710in}}%
\pgfusepath{clip}%
\pgfsetrectcap%
\pgfsetroundjoin%
\pgfsetlinewidth{0.803000pt}%
\definecolor{currentstroke}{rgb}{0.690196,0.690196,0.690196}%
\pgfsetstrokecolor{currentstroke}%
\pgfsetdash{}{0pt}%
\pgfpathmoveto{\pgfqpoint{3.218416in}{0.548769in}}%
\pgfpathlineto{\pgfqpoint{3.218416in}{1.481479in}}%
\pgfusepath{stroke}%
\end{pgfscope}%
\begin{pgfscope}%
\pgfsetbuttcap%
\pgfsetroundjoin%
\definecolor{currentfill}{rgb}{0.000000,0.000000,0.000000}%
\pgfsetfillcolor{currentfill}%
\pgfsetlinewidth{0.803000pt}%
\definecolor{currentstroke}{rgb}{0.000000,0.000000,0.000000}%
\pgfsetstrokecolor{currentstroke}%
\pgfsetdash{}{0pt}%
\pgfsys@defobject{currentmarker}{\pgfqpoint{0.000000in}{-0.048611in}}{\pgfqpoint{0.000000in}{0.000000in}}{%
\pgfpathmoveto{\pgfqpoint{0.000000in}{0.000000in}}%
\pgfpathlineto{\pgfqpoint{0.000000in}{-0.048611in}}%
\pgfusepath{stroke,fill}%
}%
\begin{pgfscope}%
\pgfsys@transformshift{3.218416in}{0.548769in}%
\pgfsys@useobject{currentmarker}{}%
\end{pgfscope}%
\end{pgfscope}%
\begin{pgfscope}%
\definecolor{textcolor}{rgb}{0.000000,0.000000,0.000000}%
\pgfsetstrokecolor{textcolor}%
\pgfsetfillcolor{textcolor}%
\pgftext[x=3.218416in,y=0.451547in,,top]{\color{textcolor}\rmfamily\fontsize{10.000000}{12.000000}\selectfont \(\displaystyle {1.25}\)}%
\end{pgfscope}%
\begin{pgfscope}%
\pgfpathrectangle{\pgfqpoint{0.730012in}{0.548769in}}{\pgfqpoint{3.981446in}{0.932710in}}%
\pgfusepath{clip}%
\pgfsetrectcap%
\pgfsetroundjoin%
\pgfsetlinewidth{0.803000pt}%
\definecolor{currentstroke}{rgb}{0.690196,0.690196,0.690196}%
\pgfsetstrokecolor{currentstroke}%
\pgfsetdash{}{0pt}%
\pgfpathmoveto{\pgfqpoint{3.716097in}{0.548769in}}%
\pgfpathlineto{\pgfqpoint{3.716097in}{1.481479in}}%
\pgfusepath{stroke}%
\end{pgfscope}%
\begin{pgfscope}%
\pgfsetbuttcap%
\pgfsetroundjoin%
\definecolor{currentfill}{rgb}{0.000000,0.000000,0.000000}%
\pgfsetfillcolor{currentfill}%
\pgfsetlinewidth{0.803000pt}%
\definecolor{currentstroke}{rgb}{0.000000,0.000000,0.000000}%
\pgfsetstrokecolor{currentstroke}%
\pgfsetdash{}{0pt}%
\pgfsys@defobject{currentmarker}{\pgfqpoint{0.000000in}{-0.048611in}}{\pgfqpoint{0.000000in}{0.000000in}}{%
\pgfpathmoveto{\pgfqpoint{0.000000in}{0.000000in}}%
\pgfpathlineto{\pgfqpoint{0.000000in}{-0.048611in}}%
\pgfusepath{stroke,fill}%
}%
\begin{pgfscope}%
\pgfsys@transformshift{3.716097in}{0.548769in}%
\pgfsys@useobject{currentmarker}{}%
\end{pgfscope}%
\end{pgfscope}%
\begin{pgfscope}%
\definecolor{textcolor}{rgb}{0.000000,0.000000,0.000000}%
\pgfsetstrokecolor{textcolor}%
\pgfsetfillcolor{textcolor}%
\pgftext[x=3.716097in,y=0.451547in,,top]{\color{textcolor}\rmfamily\fontsize{10.000000}{12.000000}\selectfont \(\displaystyle {1.50}\)}%
\end{pgfscope}%
\begin{pgfscope}%
\pgfpathrectangle{\pgfqpoint{0.730012in}{0.548769in}}{\pgfqpoint{3.981446in}{0.932710in}}%
\pgfusepath{clip}%
\pgfsetrectcap%
\pgfsetroundjoin%
\pgfsetlinewidth{0.803000pt}%
\definecolor{currentstroke}{rgb}{0.690196,0.690196,0.690196}%
\pgfsetstrokecolor{currentstroke}%
\pgfsetdash{}{0pt}%
\pgfpathmoveto{\pgfqpoint{4.213778in}{0.548769in}}%
\pgfpathlineto{\pgfqpoint{4.213778in}{1.481479in}}%
\pgfusepath{stroke}%
\end{pgfscope}%
\begin{pgfscope}%
\pgfsetbuttcap%
\pgfsetroundjoin%
\definecolor{currentfill}{rgb}{0.000000,0.000000,0.000000}%
\pgfsetfillcolor{currentfill}%
\pgfsetlinewidth{0.803000pt}%
\definecolor{currentstroke}{rgb}{0.000000,0.000000,0.000000}%
\pgfsetstrokecolor{currentstroke}%
\pgfsetdash{}{0pt}%
\pgfsys@defobject{currentmarker}{\pgfqpoint{0.000000in}{-0.048611in}}{\pgfqpoint{0.000000in}{0.000000in}}{%
\pgfpathmoveto{\pgfqpoint{0.000000in}{0.000000in}}%
\pgfpathlineto{\pgfqpoint{0.000000in}{-0.048611in}}%
\pgfusepath{stroke,fill}%
}%
\begin{pgfscope}%
\pgfsys@transformshift{4.213778in}{0.548769in}%
\pgfsys@useobject{currentmarker}{}%
\end{pgfscope}%
\end{pgfscope}%
\begin{pgfscope}%
\definecolor{textcolor}{rgb}{0.000000,0.000000,0.000000}%
\pgfsetstrokecolor{textcolor}%
\pgfsetfillcolor{textcolor}%
\pgftext[x=4.213778in,y=0.451547in,,top]{\color{textcolor}\rmfamily\fontsize{10.000000}{12.000000}\selectfont \(\displaystyle {1.75}\)}%
\end{pgfscope}%
\begin{pgfscope}%
\pgfpathrectangle{\pgfqpoint{0.730012in}{0.548769in}}{\pgfqpoint{3.981446in}{0.932710in}}%
\pgfusepath{clip}%
\pgfsetrectcap%
\pgfsetroundjoin%
\pgfsetlinewidth{0.803000pt}%
\definecolor{currentstroke}{rgb}{0.690196,0.690196,0.690196}%
\pgfsetstrokecolor{currentstroke}%
\pgfsetdash{}{0pt}%
\pgfpathmoveto{\pgfqpoint{4.711458in}{0.548769in}}%
\pgfpathlineto{\pgfqpoint{4.711458in}{1.481479in}}%
\pgfusepath{stroke}%
\end{pgfscope}%
\begin{pgfscope}%
\pgfsetbuttcap%
\pgfsetroundjoin%
\definecolor{currentfill}{rgb}{0.000000,0.000000,0.000000}%
\pgfsetfillcolor{currentfill}%
\pgfsetlinewidth{0.803000pt}%
\definecolor{currentstroke}{rgb}{0.000000,0.000000,0.000000}%
\pgfsetstrokecolor{currentstroke}%
\pgfsetdash{}{0pt}%
\pgfsys@defobject{currentmarker}{\pgfqpoint{0.000000in}{-0.048611in}}{\pgfqpoint{0.000000in}{0.000000in}}{%
\pgfpathmoveto{\pgfqpoint{0.000000in}{0.000000in}}%
\pgfpathlineto{\pgfqpoint{0.000000in}{-0.048611in}}%
\pgfusepath{stroke,fill}%
}%
\begin{pgfscope}%
\pgfsys@transformshift{4.711458in}{0.548769in}%
\pgfsys@useobject{currentmarker}{}%
\end{pgfscope}%
\end{pgfscope}%
\begin{pgfscope}%
\definecolor{textcolor}{rgb}{0.000000,0.000000,0.000000}%
\pgfsetstrokecolor{textcolor}%
\pgfsetfillcolor{textcolor}%
\pgftext[x=4.711458in,y=0.451547in,,top]{\color{textcolor}\rmfamily\fontsize{10.000000}{12.000000}\selectfont \(\displaystyle {2.00}\)}%
\end{pgfscope}%
\begin{pgfscope}%
\definecolor{textcolor}{rgb}{0.000000,0.000000,0.000000}%
\pgfsetstrokecolor{textcolor}%
\pgfsetfillcolor{textcolor}%
\pgftext[x=2.720735in,y=0.272534in,,top]{\color{textcolor}\rmfamily\fontsize{10.000000}{12.000000}\selectfont \(\displaystyle w\)}%
\end{pgfscope}%
\begin{pgfscope}%
\pgfpathrectangle{\pgfqpoint{0.730012in}{0.548769in}}{\pgfqpoint{3.981446in}{0.932710in}}%
\pgfusepath{clip}%
\pgfsetrectcap%
\pgfsetroundjoin%
\pgfsetlinewidth{0.803000pt}%
\definecolor{currentstroke}{rgb}{0.690196,0.690196,0.690196}%
\pgfsetstrokecolor{currentstroke}%
\pgfsetdash{}{0pt}%
\pgfpathmoveto{\pgfqpoint{0.730012in}{0.548769in}}%
\pgfpathlineto{\pgfqpoint{4.711458in}{0.548769in}}%
\pgfusepath{stroke}%
\end{pgfscope}%
\begin{pgfscope}%
\pgfsetbuttcap%
\pgfsetroundjoin%
\definecolor{currentfill}{rgb}{0.000000,0.000000,0.000000}%
\pgfsetfillcolor{currentfill}%
\pgfsetlinewidth{0.803000pt}%
\definecolor{currentstroke}{rgb}{0.000000,0.000000,0.000000}%
\pgfsetstrokecolor{currentstroke}%
\pgfsetdash{}{0pt}%
\pgfsys@defobject{currentmarker}{\pgfqpoint{-0.048611in}{0.000000in}}{\pgfqpoint{-0.000000in}{0.000000in}}{%
\pgfpathmoveto{\pgfqpoint{-0.000000in}{0.000000in}}%
\pgfpathlineto{\pgfqpoint{-0.048611in}{0.000000in}}%
\pgfusepath{stroke,fill}%
}%
\begin{pgfscope}%
\pgfsys@transformshift{0.730012in}{0.548769in}%
\pgfsys@useobject{currentmarker}{}%
\end{pgfscope}%
\end{pgfscope}%
\begin{pgfscope}%
\definecolor{textcolor}{rgb}{0.000000,0.000000,0.000000}%
\pgfsetstrokecolor{textcolor}%
\pgfsetfillcolor{textcolor}%
\pgftext[x=0.455320in, y=0.500544in, left, base]{\color{textcolor}\rmfamily\fontsize{10.000000}{12.000000}\selectfont \(\displaystyle {0.0}\)}%
\end{pgfscope}%
\begin{pgfscope}%
\pgfpathrectangle{\pgfqpoint{0.730012in}{0.548769in}}{\pgfqpoint{3.981446in}{0.932710in}}%
\pgfusepath{clip}%
\pgfsetrectcap%
\pgfsetroundjoin%
\pgfsetlinewidth{0.803000pt}%
\definecolor{currentstroke}{rgb}{0.690196,0.690196,0.690196}%
\pgfsetstrokecolor{currentstroke}%
\pgfsetdash{}{0pt}%
\pgfpathmoveto{\pgfqpoint{0.730012in}{1.015124in}}%
\pgfpathlineto{\pgfqpoint{4.711458in}{1.015124in}}%
\pgfusepath{stroke}%
\end{pgfscope}%
\begin{pgfscope}%
\pgfsetbuttcap%
\pgfsetroundjoin%
\definecolor{currentfill}{rgb}{0.000000,0.000000,0.000000}%
\pgfsetfillcolor{currentfill}%
\pgfsetlinewidth{0.803000pt}%
\definecolor{currentstroke}{rgb}{0.000000,0.000000,0.000000}%
\pgfsetstrokecolor{currentstroke}%
\pgfsetdash{}{0pt}%
\pgfsys@defobject{currentmarker}{\pgfqpoint{-0.048611in}{0.000000in}}{\pgfqpoint{-0.000000in}{0.000000in}}{%
\pgfpathmoveto{\pgfqpoint{-0.000000in}{0.000000in}}%
\pgfpathlineto{\pgfqpoint{-0.048611in}{0.000000in}}%
\pgfusepath{stroke,fill}%
}%
\begin{pgfscope}%
\pgfsys@transformshift{0.730012in}{1.015124in}%
\pgfsys@useobject{currentmarker}{}%
\end{pgfscope}%
\end{pgfscope}%
\begin{pgfscope}%
\definecolor{textcolor}{rgb}{0.000000,0.000000,0.000000}%
\pgfsetstrokecolor{textcolor}%
\pgfsetfillcolor{textcolor}%
\pgftext[x=0.455320in, y=0.966899in, left, base]{\color{textcolor}\rmfamily\fontsize{10.000000}{12.000000}\selectfont \(\displaystyle {0.5}\)}%
\end{pgfscope}%
\begin{pgfscope}%
\pgfpathrectangle{\pgfqpoint{0.730012in}{0.548769in}}{\pgfqpoint{3.981446in}{0.932710in}}%
\pgfusepath{clip}%
\pgfsetrectcap%
\pgfsetroundjoin%
\pgfsetlinewidth{0.803000pt}%
\definecolor{currentstroke}{rgb}{0.690196,0.690196,0.690196}%
\pgfsetstrokecolor{currentstroke}%
\pgfsetdash{}{0pt}%
\pgfpathmoveto{\pgfqpoint{0.730012in}{1.481479in}}%
\pgfpathlineto{\pgfqpoint{4.711458in}{1.481479in}}%
\pgfusepath{stroke}%
\end{pgfscope}%
\begin{pgfscope}%
\pgfsetbuttcap%
\pgfsetroundjoin%
\definecolor{currentfill}{rgb}{0.000000,0.000000,0.000000}%
\pgfsetfillcolor{currentfill}%
\pgfsetlinewidth{0.803000pt}%
\definecolor{currentstroke}{rgb}{0.000000,0.000000,0.000000}%
\pgfsetstrokecolor{currentstroke}%
\pgfsetdash{}{0pt}%
\pgfsys@defobject{currentmarker}{\pgfqpoint{-0.048611in}{0.000000in}}{\pgfqpoint{-0.000000in}{0.000000in}}{%
\pgfpathmoveto{\pgfqpoint{-0.000000in}{0.000000in}}%
\pgfpathlineto{\pgfqpoint{-0.048611in}{0.000000in}}%
\pgfusepath{stroke,fill}%
}%
\begin{pgfscope}%
\pgfsys@transformshift{0.730012in}{1.481479in}%
\pgfsys@useobject{currentmarker}{}%
\end{pgfscope}%
\end{pgfscope}%
\begin{pgfscope}%
\definecolor{textcolor}{rgb}{0.000000,0.000000,0.000000}%
\pgfsetstrokecolor{textcolor}%
\pgfsetfillcolor{textcolor}%
\pgftext[x=0.455320in, y=1.433254in, left, base]{\color{textcolor}\rmfamily\fontsize{10.000000}{12.000000}\selectfont \(\displaystyle {1.0}\)}%
\end{pgfscope}%
\begin{pgfscope}%
\definecolor{textcolor}{rgb}{0.000000,0.000000,0.000000}%
\pgfsetstrokecolor{textcolor}%
\pgfsetfillcolor{textcolor}%
\pgftext[x=0.399764in,y=1.015124in,,bottom,rotate=90.000000]{\color{textcolor}\rmfamily\fontsize{10.000000}{12.000000}\selectfont \(\displaystyle |H(w)|\)}%
\end{pgfscope}%
\begin{pgfscope}%
\pgfpathrectangle{\pgfqpoint{0.730012in}{0.548769in}}{\pgfqpoint{3.981446in}{0.932710in}}%
\pgfusepath{clip}%
\pgfsetrectcap%
\pgfsetroundjoin%
\pgfsetlinewidth{1.003750pt}%
\definecolor{currentstroke}{rgb}{0.000000,0.501961,0.000000}%
\pgfsetstrokecolor{currentstroke}%
\pgfsetdash{}{0pt}%
\pgfpathmoveto{\pgfqpoint{0.730012in}{1.481479in}}%
\pgfpathlineto{\pgfqpoint{0.778809in}{1.479958in}}%
\pgfpathlineto{\pgfqpoint{0.828602in}{1.475328in}}%
\pgfpathlineto{\pgfqpoint{0.880387in}{1.467407in}}%
\pgfpathlineto{\pgfqpoint{0.936155in}{1.455696in}}%
\pgfpathlineto{\pgfqpoint{0.997898in}{1.439486in}}%
\pgfpathlineto{\pgfqpoint{1.070596in}{1.417072in}}%
\pgfpathlineto{\pgfqpoint{1.169186in}{1.383167in}}%
\pgfpathlineto{\pgfqpoint{1.400225in}{1.302696in}}%
\pgfpathlineto{\pgfqpoint{1.490849in}{1.275037in}}%
\pgfpathlineto{\pgfqpoint{1.571513in}{1.253580in}}%
\pgfpathlineto{\pgfqpoint{1.646203in}{1.236862in}}%
\pgfpathlineto{\pgfqpoint{1.715913in}{1.224372in}}%
\pgfpathlineto{\pgfqpoint{1.781640in}{1.215702in}}%
\pgfpathlineto{\pgfqpoint{1.843383in}{1.210669in}}%
\pgfpathlineto{\pgfqpoint{1.900147in}{1.209077in}}%
\pgfpathlineto{\pgfqpoint{1.952927in}{1.210605in}}%
\pgfpathlineto{\pgfqpoint{2.001724in}{1.215011in}}%
\pgfpathlineto{\pgfqpoint{2.047534in}{1.222202in}}%
\pgfpathlineto{\pgfqpoint{2.090356in}{1.232057in}}%
\pgfpathlineto{\pgfqpoint{2.130190in}{1.244399in}}%
\pgfpathlineto{\pgfqpoint{2.167037in}{1.258981in}}%
\pgfpathlineto{\pgfqpoint{2.202888in}{1.276524in}}%
\pgfpathlineto{\pgfqpoint{2.236747in}{1.296522in}}%
\pgfpathlineto{\pgfqpoint{2.270607in}{1.320174in}}%
\pgfpathlineto{\pgfqpoint{2.304466in}{1.347623in}}%
\pgfpathlineto{\pgfqpoint{2.342308in}{1.382404in}}%
\pgfpathlineto{\pgfqpoint{2.424965in}{1.460278in}}%
\pgfpathlineto{\pgfqpoint{2.443886in}{1.472813in}}%
\pgfpathlineto{\pgfqpoint{2.458824in}{1.479192in}}%
\pgfpathlineto{\pgfqpoint{2.471770in}{1.481443in}}%
\pgfpathlineto{\pgfqpoint{2.483721in}{1.480301in}}%
\pgfpathlineto{\pgfqpoint{2.494675in}{1.476203in}}%
\pgfpathlineto{\pgfqpoint{2.505629in}{1.468966in}}%
\pgfpathlineto{\pgfqpoint{2.517580in}{1.457374in}}%
\pgfpathlineto{\pgfqpoint{2.530526in}{1.440517in}}%
\pgfpathlineto{\pgfqpoint{2.545464in}{1.415946in}}%
\pgfpathlineto{\pgfqpoint{2.564385in}{1.378434in}}%
\pgfpathlineto{\pgfqpoint{2.598244in}{1.302559in}}%
\pgfpathlineto{\pgfqpoint{2.623141in}{1.250407in}}%
\pgfpathlineto{\pgfqpoint{2.638079in}{1.225864in}}%
\pgfpathlineto{\pgfqpoint{2.648037in}{1.214516in}}%
\pgfpathlineto{\pgfqpoint{2.656004in}{1.209662in}}%
\pgfpathlineto{\pgfqpoint{2.661979in}{1.209352in}}%
\pgfpathlineto{\pgfqpoint{2.666959in}{1.211923in}}%
\pgfpathlineto{\pgfqpoint{2.671938in}{1.217767in}}%
\pgfpathlineto{\pgfqpoint{2.677913in}{1.230396in}}%
\pgfpathlineto{\pgfqpoint{2.683888in}{1.251298in}}%
\pgfpathlineto{\pgfqpoint{2.689863in}{1.283737in}}%
\pgfpathlineto{\pgfqpoint{2.696834in}{1.341847in}}%
\pgfpathlineto{\pgfqpoint{2.710776in}{1.481369in}}%
\pgfpathlineto{\pgfqpoint{2.712768in}{1.469582in}}%
\pgfpathlineto{\pgfqpoint{2.715756in}{1.408523in}}%
\pgfpathlineto{\pgfqpoint{2.720735in}{1.209077in}}%
\pgfpathlineto{\pgfqpoint{2.720735in}{1.209077in}}%
\pgfusepath{stroke}%
\end{pgfscope}%
\begin{pgfscope}%
\pgfpathrectangle{\pgfqpoint{0.730012in}{0.548769in}}{\pgfqpoint{3.981446in}{0.932710in}}%
\pgfusepath{clip}%
\pgfsetrectcap%
\pgfsetroundjoin%
\pgfsetlinewidth{1.003750pt}%
\definecolor{currentstroke}{rgb}{1.000000,0.647059,0.000000}%
\pgfsetstrokecolor{currentstroke}%
\pgfsetdash{}{0pt}%
\pgfpathmoveto{\pgfqpoint{2.720735in}{1.209077in}}%
\pgfpathlineto{\pgfqpoint{2.730426in}{0.835162in}}%
\pgfpathlineto{\pgfqpoint{2.737243in}{0.688588in}}%
\pgfpathlineto{\pgfqpoint{2.740483in}{0.641897in}}%
\pgfpathlineto{\pgfqpoint{2.740483in}{0.641897in}}%
\pgfusepath{stroke}%
\end{pgfscope}%
\begin{pgfscope}%
\pgfpathrectangle{\pgfqpoint{0.730012in}{0.548769in}}{\pgfqpoint{3.981446in}{0.932710in}}%
\pgfusepath{clip}%
\pgfsetrectcap%
\pgfsetroundjoin%
\pgfsetlinewidth{1.003750pt}%
\definecolor{currentstroke}{rgb}{1.000000,0.000000,0.000000}%
\pgfsetstrokecolor{currentstroke}%
\pgfsetdash{}{0pt}%
\pgfpathmoveto{\pgfqpoint{2.740483in}{0.641897in}}%
\pgfpathlineto{\pgfqpoint{2.747385in}{0.572668in}}%
\pgfpathlineto{\pgfqpoint{2.750343in}{0.551719in}}%
\pgfpathlineto{\pgfqpoint{2.751329in}{0.551940in}}%
\pgfpathlineto{\pgfqpoint{2.759217in}{0.589234in}}%
\pgfpathlineto{\pgfqpoint{2.767105in}{0.612151in}}%
\pgfpathlineto{\pgfqpoint{2.774992in}{0.626293in}}%
\pgfpathlineto{\pgfqpoint{2.782880in}{0.634794in}}%
\pgfpathlineto{\pgfqpoint{2.791754in}{0.639917in}}%
\pgfpathlineto{\pgfqpoint{2.801614in}{0.641963in}}%
\pgfpathlineto{\pgfqpoint{2.813446in}{0.641168in}}%
\pgfpathlineto{\pgfqpoint{2.829221in}{0.636806in}}%
\pgfpathlineto{\pgfqpoint{2.852885in}{0.626689in}}%
\pgfpathlineto{\pgfqpoint{2.998810in}{0.558502in}}%
\pgfpathlineto{\pgfqpoint{3.026417in}{0.549109in}}%
\pgfpathlineto{\pgfqpoint{3.078674in}{0.565615in}}%
\pgfpathlineto{\pgfqpoint{3.135861in}{0.580381in}}%
\pgfpathlineto{\pgfqpoint{3.198964in}{0.593453in}}%
\pgfpathlineto{\pgfqpoint{3.270941in}{0.605140in}}%
\pgfpathlineto{\pgfqpoint{3.353763in}{0.615363in}}%
\pgfpathlineto{\pgfqpoint{3.450389in}{0.624079in}}%
\pgfpathlineto{\pgfqpoint{3.565749in}{0.631263in}}%
\pgfpathlineto{\pgfqpoint{3.705758in}{0.636764in}}%
\pgfpathlineto{\pgfqpoint{3.881263in}{0.640439in}}%
\pgfpathlineto{\pgfqpoint{4.111982in}{0.642013in}}%
\pgfpathlineto{\pgfqpoint{4.439328in}{0.640922in}}%
\pgfpathlineto{\pgfqpoint{4.711458in}{0.638589in}}%
\pgfpathlineto{\pgfqpoint{4.711458in}{0.638589in}}%
\pgfusepath{stroke}%
\end{pgfscope}%
\begin{pgfscope}%
\pgfsetrectcap%
\pgfsetmiterjoin%
\pgfsetlinewidth{0.803000pt}%
\definecolor{currentstroke}{rgb}{0.000000,0.000000,0.000000}%
\pgfsetstrokecolor{currentstroke}%
\pgfsetdash{}{0pt}%
\pgfpathmoveto{\pgfqpoint{0.730012in}{0.548769in}}%
\pgfpathlineto{\pgfqpoint{0.730012in}{1.481479in}}%
\pgfusepath{stroke}%
\end{pgfscope}%
\begin{pgfscope}%
\pgfsetrectcap%
\pgfsetmiterjoin%
\pgfsetlinewidth{0.803000pt}%
\definecolor{currentstroke}{rgb}{0.000000,0.000000,0.000000}%
\pgfsetstrokecolor{currentstroke}%
\pgfsetdash{}{0pt}%
\pgfpathmoveto{\pgfqpoint{4.711458in}{0.548769in}}%
\pgfpathlineto{\pgfqpoint{4.711458in}{1.481479in}}%
\pgfusepath{stroke}%
\end{pgfscope}%
\begin{pgfscope}%
\pgfsetrectcap%
\pgfsetmiterjoin%
\pgfsetlinewidth{0.803000pt}%
\definecolor{currentstroke}{rgb}{0.000000,0.000000,0.000000}%
\pgfsetstrokecolor{currentstroke}%
\pgfsetdash{}{0pt}%
\pgfpathmoveto{\pgfqpoint{0.730012in}{0.548769in}}%
\pgfpathlineto{\pgfqpoint{4.711458in}{0.548769in}}%
\pgfusepath{stroke}%
\end{pgfscope}%
\begin{pgfscope}%
\pgfsetrectcap%
\pgfsetmiterjoin%
\pgfsetlinewidth{0.803000pt}%
\definecolor{currentstroke}{rgb}{0.000000,0.000000,0.000000}%
\pgfsetstrokecolor{currentstroke}%
\pgfsetdash{}{0pt}%
\pgfpathmoveto{\pgfqpoint{0.730012in}{1.481479in}}%
\pgfpathlineto{\pgfqpoint{4.711458in}{1.481479in}}%
\pgfusepath{stroke}%
\end{pgfscope}%
\end{pgfpicture}%
\makeatother%
\endgroup%

    \caption{$F_N$ und die resultierende Frequenzantwort eines elliptischen Filters.}
    \label{ellfilter:fig:elliptic_freq}
\end{figure}

Da sich die Funktion im Übergangsbereich nur zur nächsten Reihe von Polstellen bewegt, ist der Übergangsbereich monoton steigend.
Theoretisch könnte eine gleiches Durchlass- und Sperrbereichsverhalten erreicht werden, wenn die Funktion auf eine andere Reihe ansteigen würde.
Dies würde jedoch zu Oszillationen zwischen $1$ und $1/k$ im Übergangsbereich führen.

\subsection{Gradgleichung}

Damit die Pol- und Nullstellen genau in dieser Konstellation durchfahren werden, müssen die elliptischen Moduli des inneren und äusseren $\cd$ aufeinander abgestimmt werden.
In der reellen Richtung müssen sich die Periodizitäten $K$ und $K_1$ um den Faktor $N$ unterscheiden, während die imagiäre Periodizitäten $K^\prime$ und $K^\prime_1$ gleich bleiben müssen.
Zur Erinnerung, $K$ und $K^\prime$ sind durch elliptische Integrale definiert und vom Modul $k$ abhängig wie ersichtlich in Abbildung \ref{ellfilter:fig:kprime}.
\begin{figure}
    \centering
    %% Creator: Matplotlib, PGF backend
%%
%% To include the figure in your LaTeX document, write
%%   \input{<filename>.pgf}
%%
%% Make sure the required packages are loaded in your preamble
%%   \usepackage{pgf}
%%
%% Also ensure that all the required font packages are loaded; for instance,
%% the lmodern package is sometimes necessary when using math font.
%%   \usepackage{lmodern}
%%
%% Figures using additional raster images can only be included by \input if
%% they are in the same directory as the main LaTeX file. For loading figures
%% from other directories you can use the `import` package
%%   \usepackage{import}
%%
%% and then include the figures with
%%   \import{<path to file>}{<filename>.pgf}
%%
%% Matplotlib used the following preamble
%%
\begingroup%
\makeatletter%
\begin{pgfpicture}%
\pgfpathrectangle{\pgfpointorigin}{\pgfqpoint{5.000000in}{2.500000in}}%
\pgfusepath{use as bounding box, clip}%
\begin{pgfscope}%
\pgfsetbuttcap%
\pgfsetmiterjoin%
\pgfsetlinewidth{0.000000pt}%
\definecolor{currentstroke}{rgb}{1.000000,1.000000,1.000000}%
\pgfsetstrokecolor{currentstroke}%
\pgfsetstrokeopacity{0.000000}%
\pgfsetdash{}{0pt}%
\pgfpathmoveto{\pgfqpoint{0.000000in}{0.000000in}}%
\pgfpathlineto{\pgfqpoint{5.000000in}{0.000000in}}%
\pgfpathlineto{\pgfqpoint{5.000000in}{2.500000in}}%
\pgfpathlineto{\pgfqpoint{0.000000in}{2.500000in}}%
\pgfpathlineto{\pgfqpoint{0.000000in}{0.000000in}}%
\pgfpathclose%
\pgfusepath{}%
\end{pgfscope}%
\begin{pgfscope}%
\pgfsetbuttcap%
\pgfsetmiterjoin%
\definecolor{currentfill}{rgb}{1.000000,1.000000,1.000000}%
\pgfsetfillcolor{currentfill}%
\pgfsetlinewidth{0.000000pt}%
\definecolor{currentstroke}{rgb}{0.000000,0.000000,0.000000}%
\pgfsetstrokecolor{currentstroke}%
\pgfsetstrokeopacity{0.000000}%
\pgfsetdash{}{0pt}%
\pgfpathmoveto{\pgfqpoint{0.316407in}{0.548769in}}%
\pgfpathlineto{\pgfqpoint{2.256930in}{0.548769in}}%
\pgfpathlineto{\pgfqpoint{2.256930in}{2.301955in}}%
\pgfpathlineto{\pgfqpoint{0.316407in}{2.301955in}}%
\pgfpathlineto{\pgfqpoint{0.316407in}{0.548769in}}%
\pgfpathclose%
\pgfusepath{fill}%
\end{pgfscope}%
\begin{pgfscope}%
\pgfsetbuttcap%
\pgfsetroundjoin%
\definecolor{currentfill}{rgb}{0.000000,0.000000,0.000000}%
\pgfsetfillcolor{currentfill}%
\pgfsetlinewidth{0.803000pt}%
\definecolor{currentstroke}{rgb}{0.000000,0.000000,0.000000}%
\pgfsetstrokecolor{currentstroke}%
\pgfsetdash{}{0pt}%
\pgfsys@defobject{currentmarker}{\pgfqpoint{0.000000in}{-0.048611in}}{\pgfqpoint{0.000000in}{0.000000in}}{%
\pgfpathmoveto{\pgfqpoint{0.000000in}{0.000000in}}%
\pgfpathlineto{\pgfqpoint{0.000000in}{-0.048611in}}%
\pgfusepath{stroke,fill}%
}%
\begin{pgfscope}%
\pgfsys@transformshift{0.316407in}{0.548769in}%
\pgfsys@useobject{currentmarker}{}%
\end{pgfscope}%
\end{pgfscope}%
\begin{pgfscope}%
\definecolor{textcolor}{rgb}{0.000000,0.000000,0.000000}%
\pgfsetstrokecolor{textcolor}%
\pgfsetfillcolor{textcolor}%
\pgftext[x=0.316407in,y=0.451547in,,top]{\color{textcolor}\rmfamily\fontsize{10.000000}{12.000000}\selectfont \(\displaystyle {0.00}\)}%
\end{pgfscope}%
\begin{pgfscope}%
\pgfsetbuttcap%
\pgfsetroundjoin%
\definecolor{currentfill}{rgb}{0.000000,0.000000,0.000000}%
\pgfsetfillcolor{currentfill}%
\pgfsetlinewidth{0.803000pt}%
\definecolor{currentstroke}{rgb}{0.000000,0.000000,0.000000}%
\pgfsetstrokecolor{currentstroke}%
\pgfsetdash{}{0pt}%
\pgfsys@defobject{currentmarker}{\pgfqpoint{0.000000in}{-0.048611in}}{\pgfqpoint{0.000000in}{0.000000in}}{%
\pgfpathmoveto{\pgfqpoint{0.000000in}{0.000000in}}%
\pgfpathlineto{\pgfqpoint{0.000000in}{-0.048611in}}%
\pgfusepath{stroke,fill}%
}%
\begin{pgfscope}%
\pgfsys@transformshift{0.801538in}{0.548769in}%
\pgfsys@useobject{currentmarker}{}%
\end{pgfscope}%
\end{pgfscope}%
\begin{pgfscope}%
\definecolor{textcolor}{rgb}{0.000000,0.000000,0.000000}%
\pgfsetstrokecolor{textcolor}%
\pgfsetfillcolor{textcolor}%
\pgftext[x=0.801538in,y=0.451547in,,top]{\color{textcolor}\rmfamily\fontsize{10.000000}{12.000000}\selectfont \(\displaystyle {0.25}\)}%
\end{pgfscope}%
\begin{pgfscope}%
\pgfsetbuttcap%
\pgfsetroundjoin%
\definecolor{currentfill}{rgb}{0.000000,0.000000,0.000000}%
\pgfsetfillcolor{currentfill}%
\pgfsetlinewidth{0.803000pt}%
\definecolor{currentstroke}{rgb}{0.000000,0.000000,0.000000}%
\pgfsetstrokecolor{currentstroke}%
\pgfsetdash{}{0pt}%
\pgfsys@defobject{currentmarker}{\pgfqpoint{0.000000in}{-0.048611in}}{\pgfqpoint{0.000000in}{0.000000in}}{%
\pgfpathmoveto{\pgfqpoint{0.000000in}{0.000000in}}%
\pgfpathlineto{\pgfqpoint{0.000000in}{-0.048611in}}%
\pgfusepath{stroke,fill}%
}%
\begin{pgfscope}%
\pgfsys@transformshift{1.286669in}{0.548769in}%
\pgfsys@useobject{currentmarker}{}%
\end{pgfscope}%
\end{pgfscope}%
\begin{pgfscope}%
\definecolor{textcolor}{rgb}{0.000000,0.000000,0.000000}%
\pgfsetstrokecolor{textcolor}%
\pgfsetfillcolor{textcolor}%
\pgftext[x=1.286669in,y=0.451547in,,top]{\color{textcolor}\rmfamily\fontsize{10.000000}{12.000000}\selectfont \(\displaystyle {0.50}\)}%
\end{pgfscope}%
\begin{pgfscope}%
\pgfsetbuttcap%
\pgfsetroundjoin%
\definecolor{currentfill}{rgb}{0.000000,0.000000,0.000000}%
\pgfsetfillcolor{currentfill}%
\pgfsetlinewidth{0.803000pt}%
\definecolor{currentstroke}{rgb}{0.000000,0.000000,0.000000}%
\pgfsetstrokecolor{currentstroke}%
\pgfsetdash{}{0pt}%
\pgfsys@defobject{currentmarker}{\pgfqpoint{0.000000in}{-0.048611in}}{\pgfqpoint{0.000000in}{0.000000in}}{%
\pgfpathmoveto{\pgfqpoint{0.000000in}{0.000000in}}%
\pgfpathlineto{\pgfqpoint{0.000000in}{-0.048611in}}%
\pgfusepath{stroke,fill}%
}%
\begin{pgfscope}%
\pgfsys@transformshift{1.771800in}{0.548769in}%
\pgfsys@useobject{currentmarker}{}%
\end{pgfscope}%
\end{pgfscope}%
\begin{pgfscope}%
\definecolor{textcolor}{rgb}{0.000000,0.000000,0.000000}%
\pgfsetstrokecolor{textcolor}%
\pgfsetfillcolor{textcolor}%
\pgftext[x=1.771800in,y=0.451547in,,top]{\color{textcolor}\rmfamily\fontsize{10.000000}{12.000000}\selectfont \(\displaystyle {0.75}\)}%
\end{pgfscope}%
\begin{pgfscope}%
\pgfsetbuttcap%
\pgfsetroundjoin%
\definecolor{currentfill}{rgb}{0.000000,0.000000,0.000000}%
\pgfsetfillcolor{currentfill}%
\pgfsetlinewidth{0.803000pt}%
\definecolor{currentstroke}{rgb}{0.000000,0.000000,0.000000}%
\pgfsetstrokecolor{currentstroke}%
\pgfsetdash{}{0pt}%
\pgfsys@defobject{currentmarker}{\pgfqpoint{0.000000in}{-0.048611in}}{\pgfqpoint{0.000000in}{0.000000in}}{%
\pgfpathmoveto{\pgfqpoint{0.000000in}{0.000000in}}%
\pgfpathlineto{\pgfqpoint{0.000000in}{-0.048611in}}%
\pgfusepath{stroke,fill}%
}%
\begin{pgfscope}%
\pgfsys@transformshift{2.256930in}{0.548769in}%
\pgfsys@useobject{currentmarker}{}%
\end{pgfscope}%
\end{pgfscope}%
\begin{pgfscope}%
\definecolor{textcolor}{rgb}{0.000000,0.000000,0.000000}%
\pgfsetstrokecolor{textcolor}%
\pgfsetfillcolor{textcolor}%
\pgftext[x=2.256930in,y=0.451547in,,top]{\color{textcolor}\rmfamily\fontsize{10.000000}{12.000000}\selectfont \(\displaystyle {1.00}\)}%
\end{pgfscope}%
\begin{pgfscope}%
\definecolor{textcolor}{rgb}{0.000000,0.000000,0.000000}%
\pgfsetstrokecolor{textcolor}%
\pgfsetfillcolor{textcolor}%
\pgftext[x=1.286669in,y=0.272534in,,top]{\color{textcolor}\rmfamily\fontsize{10.000000}{12.000000}\selectfont \(\displaystyle k\)}%
\end{pgfscope}%
\begin{pgfscope}%
\pgfsetbuttcap%
\pgfsetroundjoin%
\definecolor{currentfill}{rgb}{0.000000,0.000000,0.000000}%
\pgfsetfillcolor{currentfill}%
\pgfsetlinewidth{0.803000pt}%
\definecolor{currentstroke}{rgb}{0.000000,0.000000,0.000000}%
\pgfsetstrokecolor{currentstroke}%
\pgfsetdash{}{0pt}%
\pgfsys@defobject{currentmarker}{\pgfqpoint{-0.048611in}{0.000000in}}{\pgfqpoint{-0.000000in}{0.000000in}}{%
\pgfpathmoveto{\pgfqpoint{-0.000000in}{0.000000in}}%
\pgfpathlineto{\pgfqpoint{-0.048611in}{0.000000in}}%
\pgfusepath{stroke,fill}%
}%
\begin{pgfscope}%
\pgfsys@transformshift{0.316407in}{0.548769in}%
\pgfsys@useobject{currentmarker}{}%
\end{pgfscope}%
\end{pgfscope}%
\begin{pgfscope}%
\definecolor{textcolor}{rgb}{0.000000,0.000000,0.000000}%
\pgfsetstrokecolor{textcolor}%
\pgfsetfillcolor{textcolor}%
\pgftext[x=0.149740in, y=0.500544in, left, base]{\color{textcolor}\rmfamily\fontsize{10.000000}{12.000000}\selectfont \(\displaystyle {0}\)}%
\end{pgfscope}%
\begin{pgfscope}%
\pgfsetbuttcap%
\pgfsetroundjoin%
\definecolor{currentfill}{rgb}{0.000000,0.000000,0.000000}%
\pgfsetfillcolor{currentfill}%
\pgfsetlinewidth{0.803000pt}%
\definecolor{currentstroke}{rgb}{0.000000,0.000000,0.000000}%
\pgfsetstrokecolor{currentstroke}%
\pgfsetdash{}{0pt}%
\pgfsys@defobject{currentmarker}{\pgfqpoint{-0.048611in}{0.000000in}}{\pgfqpoint{-0.000000in}{0.000000in}}{%
\pgfpathmoveto{\pgfqpoint{-0.000000in}{0.000000in}}%
\pgfpathlineto{\pgfqpoint{-0.048611in}{0.000000in}}%
\pgfusepath{stroke,fill}%
}%
\begin{pgfscope}%
\pgfsys@transformshift{0.316407in}{0.987065in}%
\pgfsys@useobject{currentmarker}{}%
\end{pgfscope}%
\end{pgfscope}%
\begin{pgfscope}%
\definecolor{textcolor}{rgb}{0.000000,0.000000,0.000000}%
\pgfsetstrokecolor{textcolor}%
\pgfsetfillcolor{textcolor}%
\pgftext[x=0.149740in, y=0.938840in, left, base]{\color{textcolor}\rmfamily\fontsize{10.000000}{12.000000}\selectfont \(\displaystyle {1}\)}%
\end{pgfscope}%
\begin{pgfscope}%
\pgfsetbuttcap%
\pgfsetroundjoin%
\definecolor{currentfill}{rgb}{0.000000,0.000000,0.000000}%
\pgfsetfillcolor{currentfill}%
\pgfsetlinewidth{0.803000pt}%
\definecolor{currentstroke}{rgb}{0.000000,0.000000,0.000000}%
\pgfsetstrokecolor{currentstroke}%
\pgfsetdash{}{0pt}%
\pgfsys@defobject{currentmarker}{\pgfqpoint{-0.048611in}{0.000000in}}{\pgfqpoint{-0.000000in}{0.000000in}}{%
\pgfpathmoveto{\pgfqpoint{-0.000000in}{0.000000in}}%
\pgfpathlineto{\pgfqpoint{-0.048611in}{0.000000in}}%
\pgfusepath{stroke,fill}%
}%
\begin{pgfscope}%
\pgfsys@transformshift{0.316407in}{1.425362in}%
\pgfsys@useobject{currentmarker}{}%
\end{pgfscope}%
\end{pgfscope}%
\begin{pgfscope}%
\definecolor{textcolor}{rgb}{0.000000,0.000000,0.000000}%
\pgfsetstrokecolor{textcolor}%
\pgfsetfillcolor{textcolor}%
\pgftext[x=0.149740in, y=1.377137in, left, base]{\color{textcolor}\rmfamily\fontsize{10.000000}{12.000000}\selectfont \(\displaystyle {2}\)}%
\end{pgfscope}%
\begin{pgfscope}%
\pgfsetbuttcap%
\pgfsetroundjoin%
\definecolor{currentfill}{rgb}{0.000000,0.000000,0.000000}%
\pgfsetfillcolor{currentfill}%
\pgfsetlinewidth{0.803000pt}%
\definecolor{currentstroke}{rgb}{0.000000,0.000000,0.000000}%
\pgfsetstrokecolor{currentstroke}%
\pgfsetdash{}{0pt}%
\pgfsys@defobject{currentmarker}{\pgfqpoint{-0.048611in}{0.000000in}}{\pgfqpoint{-0.000000in}{0.000000in}}{%
\pgfpathmoveto{\pgfqpoint{-0.000000in}{0.000000in}}%
\pgfpathlineto{\pgfqpoint{-0.048611in}{0.000000in}}%
\pgfusepath{stroke,fill}%
}%
\begin{pgfscope}%
\pgfsys@transformshift{0.316407in}{1.863658in}%
\pgfsys@useobject{currentmarker}{}%
\end{pgfscope}%
\end{pgfscope}%
\begin{pgfscope}%
\definecolor{textcolor}{rgb}{0.000000,0.000000,0.000000}%
\pgfsetstrokecolor{textcolor}%
\pgfsetfillcolor{textcolor}%
\pgftext[x=0.149740in, y=1.815433in, left, base]{\color{textcolor}\rmfamily\fontsize{10.000000}{12.000000}\selectfont \(\displaystyle {3}\)}%
\end{pgfscope}%
\begin{pgfscope}%
\pgfsetbuttcap%
\pgfsetroundjoin%
\definecolor{currentfill}{rgb}{0.000000,0.000000,0.000000}%
\pgfsetfillcolor{currentfill}%
\pgfsetlinewidth{0.803000pt}%
\definecolor{currentstroke}{rgb}{0.000000,0.000000,0.000000}%
\pgfsetstrokecolor{currentstroke}%
\pgfsetdash{}{0pt}%
\pgfsys@defobject{currentmarker}{\pgfqpoint{-0.048611in}{0.000000in}}{\pgfqpoint{-0.000000in}{0.000000in}}{%
\pgfpathmoveto{\pgfqpoint{-0.000000in}{0.000000in}}%
\pgfpathlineto{\pgfqpoint{-0.048611in}{0.000000in}}%
\pgfusepath{stroke,fill}%
}%
\begin{pgfscope}%
\pgfsys@transformshift{0.316407in}{2.301955in}%
\pgfsys@useobject{currentmarker}{}%
\end{pgfscope}%
\end{pgfscope}%
\begin{pgfscope}%
\definecolor{textcolor}{rgb}{0.000000,0.000000,0.000000}%
\pgfsetstrokecolor{textcolor}%
\pgfsetfillcolor{textcolor}%
\pgftext[x=0.149740in, y=2.253730in, left, base]{\color{textcolor}\rmfamily\fontsize{10.000000}{12.000000}\selectfont \(\displaystyle {4}\)}%
\end{pgfscope}%
\begin{pgfscope}%
\pgfpathrectangle{\pgfqpoint{0.316407in}{0.548769in}}{\pgfqpoint{1.940523in}{1.753186in}}%
\pgfusepath{clip}%
\pgfsetrectcap%
\pgfsetroundjoin%
\pgfsetlinewidth{1.003750pt}%
\definecolor{currentstroke}{rgb}{0.121569,0.466667,0.705882}%
\pgfsetstrokecolor{currentstroke}%
\pgfsetdash{}{0pt}%
\pgfpathmoveto{\pgfqpoint{0.316427in}{1.237243in}}%
\pgfpathlineto{\pgfqpoint{0.316601in}{1.237243in}}%
\pgfpathlineto{\pgfqpoint{0.318348in}{1.237244in}}%
\pgfpathlineto{\pgfqpoint{0.335813in}{1.237261in}}%
\pgfpathlineto{\pgfqpoint{0.355218in}{1.237312in}}%
\pgfpathlineto{\pgfqpoint{0.374623in}{1.237398in}}%
\pgfpathlineto{\pgfqpoint{0.394028in}{1.237519in}}%
\pgfpathlineto{\pgfqpoint{0.413434in}{1.237674in}}%
\pgfpathlineto{\pgfqpoint{0.432839in}{1.237864in}}%
\pgfpathlineto{\pgfqpoint{0.452244in}{1.238089in}}%
\pgfpathlineto{\pgfqpoint{0.471649in}{1.238349in}}%
\pgfpathlineto{\pgfqpoint{0.491054in}{1.238644in}}%
\pgfpathlineto{\pgfqpoint{0.510460in}{1.238974in}}%
\pgfpathlineto{\pgfqpoint{0.529865in}{1.239340in}}%
\pgfpathlineto{\pgfqpoint{0.549270in}{1.239742in}}%
\pgfpathlineto{\pgfqpoint{0.568675in}{1.240180in}}%
\pgfpathlineto{\pgfqpoint{0.588081in}{1.240655in}}%
\pgfpathlineto{\pgfqpoint{0.607486in}{1.241166in}}%
\pgfpathlineto{\pgfqpoint{0.626891in}{1.241714in}}%
\pgfpathlineto{\pgfqpoint{0.646296in}{1.242300in}}%
\pgfpathlineto{\pgfqpoint{0.665702in}{1.242924in}}%
\pgfpathlineto{\pgfqpoint{0.685107in}{1.243586in}}%
\pgfpathlineto{\pgfqpoint{0.704512in}{1.244287in}}%
\pgfpathlineto{\pgfqpoint{0.723917in}{1.245028in}}%
\pgfpathlineto{\pgfqpoint{0.743322in}{1.245809in}}%
\pgfpathlineto{\pgfqpoint{0.762728in}{1.246630in}}%
\pgfpathlineto{\pgfqpoint{0.782133in}{1.247492in}}%
\pgfpathlineto{\pgfqpoint{0.801538in}{1.248396in}}%
\pgfpathlineto{\pgfqpoint{0.820943in}{1.249343in}}%
\pgfpathlineto{\pgfqpoint{0.840349in}{1.250333in}}%
\pgfpathlineto{\pgfqpoint{0.859754in}{1.251367in}}%
\pgfpathlineto{\pgfqpoint{0.879159in}{1.252446in}}%
\pgfpathlineto{\pgfqpoint{0.898564in}{1.253571in}}%
\pgfpathlineto{\pgfqpoint{0.917969in}{1.254743in}}%
\pgfpathlineto{\pgfqpoint{0.937375in}{1.255962in}}%
\pgfpathlineto{\pgfqpoint{0.956780in}{1.257230in}}%
\pgfpathlineto{\pgfqpoint{0.976185in}{1.258548in}}%
\pgfpathlineto{\pgfqpoint{0.995590in}{1.259917in}}%
\pgfpathlineto{\pgfqpoint{1.014996in}{1.261339in}}%
\pgfpathlineto{\pgfqpoint{1.034401in}{1.262814in}}%
\pgfpathlineto{\pgfqpoint{1.053806in}{1.264344in}}%
\pgfpathlineto{\pgfqpoint{1.073211in}{1.265930in}}%
\pgfpathlineto{\pgfqpoint{1.092617in}{1.267575in}}%
\pgfpathlineto{\pgfqpoint{1.112022in}{1.269279in}}%
\pgfpathlineto{\pgfqpoint{1.131427in}{1.271045in}}%
\pgfpathlineto{\pgfqpoint{1.150832in}{1.272874in}}%
\pgfpathlineto{\pgfqpoint{1.170237in}{1.274768in}}%
\pgfpathlineto{\pgfqpoint{1.189643in}{1.276729in}}%
\pgfpathlineto{\pgfqpoint{1.209048in}{1.278760in}}%
\pgfpathlineto{\pgfqpoint{1.228453in}{1.280863in}}%
\pgfpathlineto{\pgfqpoint{1.247858in}{1.283040in}}%
\pgfpathlineto{\pgfqpoint{1.267264in}{1.285294in}}%
\pgfpathlineto{\pgfqpoint{1.286669in}{1.287627in}}%
\pgfpathlineto{\pgfqpoint{1.306074in}{1.290044in}}%
\pgfpathlineto{\pgfqpoint{1.325479in}{1.292546in}}%
\pgfpathlineto{\pgfqpoint{1.344884in}{1.295137in}}%
\pgfpathlineto{\pgfqpoint{1.364290in}{1.297822in}}%
\pgfpathlineto{\pgfqpoint{1.383695in}{1.300603in}}%
\pgfpathlineto{\pgfqpoint{1.403100in}{1.303485in}}%
\pgfpathlineto{\pgfqpoint{1.422505in}{1.306473in}}%
\pgfpathlineto{\pgfqpoint{1.441911in}{1.309570in}}%
\pgfpathlineto{\pgfqpoint{1.461316in}{1.312784in}}%
\pgfpathlineto{\pgfqpoint{1.480721in}{1.316118in}}%
\pgfpathlineto{\pgfqpoint{1.500126in}{1.319579in}}%
\pgfpathlineto{\pgfqpoint{1.519532in}{1.323174in}}%
\pgfpathlineto{\pgfqpoint{1.538937in}{1.326910in}}%
\pgfpathlineto{\pgfqpoint{1.558342in}{1.330793in}}%
\pgfpathlineto{\pgfqpoint{1.577747in}{1.334833in}}%
\pgfpathlineto{\pgfqpoint{1.597152in}{1.339039in}}%
\pgfpathlineto{\pgfqpoint{1.616558in}{1.343420in}}%
\pgfpathlineto{\pgfqpoint{1.635963in}{1.347988in}}%
\pgfpathlineto{\pgfqpoint{1.655368in}{1.352753in}}%
\pgfpathlineto{\pgfqpoint{1.674773in}{1.357730in}}%
\pgfpathlineto{\pgfqpoint{1.694179in}{1.362933in}}%
\pgfpathlineto{\pgfqpoint{1.713584in}{1.368377in}}%
\pgfpathlineto{\pgfqpoint{1.732989in}{1.374081in}}%
\pgfpathlineto{\pgfqpoint{1.752394in}{1.380064in}}%
\pgfpathlineto{\pgfqpoint{1.771800in}{1.386349in}}%
\pgfpathlineto{\pgfqpoint{1.791205in}{1.392961in}}%
\pgfpathlineto{\pgfqpoint{1.810610in}{1.399927in}}%
\pgfpathlineto{\pgfqpoint{1.830015in}{1.407281in}}%
\pgfpathlineto{\pgfqpoint{1.849420in}{1.415059in}}%
\pgfpathlineto{\pgfqpoint{1.868826in}{1.423303in}}%
\pgfpathlineto{\pgfqpoint{1.888231in}{1.432062in}}%
\pgfpathlineto{\pgfqpoint{1.907636in}{1.441392in}}%
\pgfpathlineto{\pgfqpoint{1.927041in}{1.451361in}}%
\pgfpathlineto{\pgfqpoint{1.946447in}{1.462048in}}%
\pgfpathlineto{\pgfqpoint{1.965852in}{1.473546in}}%
\pgfpathlineto{\pgfqpoint{1.985257in}{1.485971in}}%
\pgfpathlineto{\pgfqpoint{2.004662in}{1.499462in}}%
\pgfpathlineto{\pgfqpoint{2.024067in}{1.514194in}}%
\pgfpathlineto{\pgfqpoint{2.043473in}{1.530388in}}%
\pgfpathlineto{\pgfqpoint{2.062878in}{1.548326in}}%
\pgfpathlineto{\pgfqpoint{2.082283in}{1.568383in}}%
\pgfpathlineto{\pgfqpoint{2.101688in}{1.591069in}}%
\pgfpathlineto{\pgfqpoint{2.121094in}{1.617098in}}%
\pgfpathlineto{\pgfqpoint{2.140499in}{1.647519in}}%
\pgfpathlineto{\pgfqpoint{2.159904in}{1.683962in}}%
\pgfpathlineto{\pgfqpoint{2.179309in}{1.729164in}}%
\pgfpathlineto{\pgfqpoint{2.198715in}{1.788269in}}%
\pgfpathlineto{\pgfqpoint{2.218120in}{1.872854in}}%
\pgfpathlineto{\pgfqpoint{2.237525in}{2.019955in}}%
\pgfpathlineto{\pgfqpoint{2.247876in}{2.315844in}}%
\pgfusepath{stroke}%
\end{pgfscope}%
\begin{pgfscope}%
\pgfpathrectangle{\pgfqpoint{0.316407in}{0.548769in}}{\pgfqpoint{1.940523in}{1.753186in}}%
\pgfusepath{clip}%
\pgfsetrectcap%
\pgfsetroundjoin%
\pgfsetlinewidth{1.003750pt}%
\definecolor{currentstroke}{rgb}{1.000000,0.498039,0.054902}%
\pgfsetstrokecolor{currentstroke}%
\pgfsetdash{}{0pt}%
\pgfpathmoveto{\pgfqpoint{0.454821in}{2.315844in}}%
\pgfpathlineto{\pgfqpoint{0.471649in}{2.265444in}}%
\pgfpathlineto{\pgfqpoint{0.491054in}{2.214262in}}%
\pgfpathlineto{\pgfqpoint{0.510460in}{2.168554in}}%
\pgfpathlineto{\pgfqpoint{0.529865in}{2.127278in}}%
\pgfpathlineto{\pgfqpoint{0.549270in}{2.089666in}}%
\pgfpathlineto{\pgfqpoint{0.568675in}{2.055132in}}%
\pgfpathlineto{\pgfqpoint{0.588081in}{2.023222in}}%
\pgfpathlineto{\pgfqpoint{0.607486in}{1.993575in}}%
\pgfpathlineto{\pgfqpoint{0.626891in}{1.965899in}}%
\pgfpathlineto{\pgfqpoint{0.646296in}{1.939958in}}%
\pgfpathlineto{\pgfqpoint{0.665702in}{1.915554in}}%
\pgfpathlineto{\pgfqpoint{0.685107in}{1.892522in}}%
\pgfpathlineto{\pgfqpoint{0.704512in}{1.870720in}}%
\pgfpathlineto{\pgfqpoint{0.723917in}{1.850031in}}%
\pgfpathlineto{\pgfqpoint{0.743322in}{1.830351in}}%
\pgfpathlineto{\pgfqpoint{0.762728in}{1.811591in}}%
\pgfpathlineto{\pgfqpoint{0.782133in}{1.793672in}}%
\pgfpathlineto{\pgfqpoint{0.801538in}{1.776528in}}%
\pgfpathlineto{\pgfqpoint{0.820943in}{1.760096in}}%
\pgfpathlineto{\pgfqpoint{0.840349in}{1.744324in}}%
\pgfpathlineto{\pgfqpoint{0.859754in}{1.729164in}}%
\pgfpathlineto{\pgfqpoint{0.879159in}{1.714573in}}%
\pgfpathlineto{\pgfqpoint{0.898564in}{1.700513in}}%
\pgfpathlineto{\pgfqpoint{0.917969in}{1.686948in}}%
\pgfpathlineto{\pgfqpoint{0.937375in}{1.673849in}}%
\pgfpathlineto{\pgfqpoint{0.956780in}{1.661185in}}%
\pgfpathlineto{\pgfqpoint{0.976185in}{1.648932in}}%
\pgfpathlineto{\pgfqpoint{0.995590in}{1.637065in}}%
\pgfpathlineto{\pgfqpoint{1.014996in}{1.625563in}}%
\pgfpathlineto{\pgfqpoint{1.034401in}{1.614406in}}%
\pgfpathlineto{\pgfqpoint{1.053806in}{1.603575in}}%
\pgfpathlineto{\pgfqpoint{1.073211in}{1.593053in}}%
\pgfpathlineto{\pgfqpoint{1.092617in}{1.582826in}}%
\pgfpathlineto{\pgfqpoint{1.112022in}{1.572877in}}%
\pgfpathlineto{\pgfqpoint{1.131427in}{1.563195in}}%
\pgfpathlineto{\pgfqpoint{1.150832in}{1.553766in}}%
\pgfpathlineto{\pgfqpoint{1.170237in}{1.544578in}}%
\pgfpathlineto{\pgfqpoint{1.189643in}{1.535621in}}%
\pgfpathlineto{\pgfqpoint{1.209048in}{1.526884in}}%
\pgfpathlineto{\pgfqpoint{1.228453in}{1.518359in}}%
\pgfpathlineto{\pgfqpoint{1.247858in}{1.510036in}}%
\pgfpathlineto{\pgfqpoint{1.267264in}{1.501906in}}%
\pgfpathlineto{\pgfqpoint{1.286669in}{1.493962in}}%
\pgfpathlineto{\pgfqpoint{1.306074in}{1.486197in}}%
\pgfpathlineto{\pgfqpoint{1.325479in}{1.478603in}}%
\pgfpathlineto{\pgfqpoint{1.344884in}{1.471174in}}%
\pgfpathlineto{\pgfqpoint{1.364290in}{1.463903in}}%
\pgfpathlineto{\pgfqpoint{1.383695in}{1.456785in}}%
\pgfpathlineto{\pgfqpoint{1.403100in}{1.449815in}}%
\pgfpathlineto{\pgfqpoint{1.422505in}{1.442986in}}%
\pgfpathlineto{\pgfqpoint{1.441911in}{1.436294in}}%
\pgfpathlineto{\pgfqpoint{1.461316in}{1.429735in}}%
\pgfpathlineto{\pgfqpoint{1.480721in}{1.423303in}}%
\pgfpathlineto{\pgfqpoint{1.500126in}{1.416995in}}%
\pgfpathlineto{\pgfqpoint{1.519532in}{1.410805in}}%
\pgfpathlineto{\pgfqpoint{1.538937in}{1.404732in}}%
\pgfpathlineto{\pgfqpoint{1.558342in}{1.398770in}}%
\pgfpathlineto{\pgfqpoint{1.577747in}{1.392916in}}%
\pgfpathlineto{\pgfqpoint{1.597152in}{1.387167in}}%
\pgfpathlineto{\pgfqpoint{1.616558in}{1.381520in}}%
\pgfpathlineto{\pgfqpoint{1.635963in}{1.375971in}}%
\pgfpathlineto{\pgfqpoint{1.655368in}{1.370518in}}%
\pgfpathlineto{\pgfqpoint{1.674773in}{1.365158in}}%
\pgfpathlineto{\pgfqpoint{1.694179in}{1.359888in}}%
\pgfpathlineto{\pgfqpoint{1.713584in}{1.354705in}}%
\pgfpathlineto{\pgfqpoint{1.732989in}{1.349607in}}%
\pgfpathlineto{\pgfqpoint{1.752394in}{1.344593in}}%
\pgfpathlineto{\pgfqpoint{1.771800in}{1.339658in}}%
\pgfpathlineto{\pgfqpoint{1.791205in}{1.334802in}}%
\pgfpathlineto{\pgfqpoint{1.810610in}{1.330022in}}%
\pgfpathlineto{\pgfqpoint{1.830015in}{1.325316in}}%
\pgfpathlineto{\pgfqpoint{1.849420in}{1.320682in}}%
\pgfpathlineto{\pgfqpoint{1.868826in}{1.316118in}}%
\pgfpathlineto{\pgfqpoint{1.888231in}{1.311623in}}%
\pgfpathlineto{\pgfqpoint{1.907636in}{1.307195in}}%
\pgfpathlineto{\pgfqpoint{1.927041in}{1.302831in}}%
\pgfpathlineto{\pgfqpoint{1.946447in}{1.298531in}}%
\pgfpathlineto{\pgfqpoint{1.965852in}{1.294293in}}%
\pgfpathlineto{\pgfqpoint{1.985257in}{1.290116in}}%
\pgfpathlineto{\pgfqpoint{2.004662in}{1.285997in}}%
\pgfpathlineto{\pgfqpoint{2.024067in}{1.281936in}}%
\pgfpathlineto{\pgfqpoint{2.043473in}{1.277931in}}%
\pgfpathlineto{\pgfqpoint{2.062878in}{1.273982in}}%
\pgfpathlineto{\pgfqpoint{2.082283in}{1.270085in}}%
\pgfpathlineto{\pgfqpoint{2.101688in}{1.266241in}}%
\pgfpathlineto{\pgfqpoint{2.121094in}{1.262449in}}%
\pgfpathlineto{\pgfqpoint{2.140499in}{1.258706in}}%
\pgfpathlineto{\pgfqpoint{2.159904in}{1.255013in}}%
\pgfpathlineto{\pgfqpoint{2.179309in}{1.251367in}}%
\pgfpathlineto{\pgfqpoint{2.198715in}{1.247768in}}%
\pgfpathlineto{\pgfqpoint{2.218120in}{1.244215in}}%
\pgfpathlineto{\pgfqpoint{2.237525in}{1.240707in}}%
\pgfpathlineto{\pgfqpoint{2.254990in}{1.237588in}}%
\pgfpathlineto{\pgfqpoint{2.256736in}{1.237278in}}%
\pgfpathlineto{\pgfqpoint{2.256911in}{1.237247in}}%
\pgfusepath{stroke}%
\end{pgfscope}%
\begin{pgfscope}%
\pgfsetrectcap%
\pgfsetmiterjoin%
\pgfsetlinewidth{0.803000pt}%
\definecolor{currentstroke}{rgb}{0.000000,0.000000,0.000000}%
\pgfsetstrokecolor{currentstroke}%
\pgfsetdash{}{0pt}%
\pgfpathmoveto{\pgfqpoint{0.316407in}{0.548769in}}%
\pgfpathlineto{\pgfqpoint{0.316407in}{2.301955in}}%
\pgfusepath{stroke}%
\end{pgfscope}%
\begin{pgfscope}%
\pgfsetrectcap%
\pgfsetmiterjoin%
\pgfsetlinewidth{0.803000pt}%
\definecolor{currentstroke}{rgb}{0.000000,0.000000,0.000000}%
\pgfsetstrokecolor{currentstroke}%
\pgfsetdash{}{0pt}%
\pgfpathmoveto{\pgfqpoint{2.256930in}{0.548769in}}%
\pgfpathlineto{\pgfqpoint{2.256930in}{2.301955in}}%
\pgfusepath{stroke}%
\end{pgfscope}%
\begin{pgfscope}%
\pgfsetrectcap%
\pgfsetmiterjoin%
\pgfsetlinewidth{0.803000pt}%
\definecolor{currentstroke}{rgb}{0.000000,0.000000,0.000000}%
\pgfsetstrokecolor{currentstroke}%
\pgfsetdash{}{0pt}%
\pgfpathmoveto{\pgfqpoint{0.316407in}{0.548769in}}%
\pgfpathlineto{\pgfqpoint{2.256930in}{0.548769in}}%
\pgfusepath{stroke}%
\end{pgfscope}%
\begin{pgfscope}%
\pgfsetrectcap%
\pgfsetmiterjoin%
\pgfsetlinewidth{0.803000pt}%
\definecolor{currentstroke}{rgb}{0.000000,0.000000,0.000000}%
\pgfsetstrokecolor{currentstroke}%
\pgfsetdash{}{0pt}%
\pgfpathmoveto{\pgfqpoint{0.316407in}{2.301955in}}%
\pgfpathlineto{\pgfqpoint{2.256930in}{2.301955in}}%
\pgfusepath{stroke}%
\end{pgfscope}%
\begin{pgfscope}%
\definecolor{textcolor}{rgb}{0.000000,0.000000,0.000000}%
\pgfsetstrokecolor{textcolor}%
\pgfsetfillcolor{textcolor}%
\pgftext[x=0.859754in,y=1.295197in,left,base]{\color{textcolor}\rmfamily\fontsize{10.000000}{12.000000}\selectfont \(\displaystyle K\)}%
\end{pgfscope}%
\begin{pgfscope}%
\definecolor{textcolor}{rgb}{0.000000,0.000000,0.000000}%
\pgfsetstrokecolor{textcolor}%
\pgfsetfillcolor{textcolor}%
\pgftext[x=0.859754in,y=1.772994in,left,base]{\color{textcolor}\rmfamily\fontsize{10.000000}{12.000000}\selectfont \(\displaystyle K^\prime\)}%
\end{pgfscope}%
\begin{pgfscope}%
\pgfsetbuttcap%
\pgfsetmiterjoin%
\definecolor{currentfill}{rgb}{1.000000,1.000000,1.000000}%
\pgfsetfillcolor{currentfill}%
\pgfsetlinewidth{0.000000pt}%
\definecolor{currentstroke}{rgb}{0.000000,0.000000,0.000000}%
\pgfsetstrokecolor{currentstroke}%
\pgfsetstrokeopacity{0.000000}%
\pgfsetdash{}{0pt}%
\pgfpathmoveto{\pgfqpoint{2.874885in}{0.548769in}}%
\pgfpathlineto{\pgfqpoint{4.815407in}{0.548769in}}%
\pgfpathlineto{\pgfqpoint{4.815407in}{2.301955in}}%
\pgfpathlineto{\pgfqpoint{2.874885in}{2.301955in}}%
\pgfpathlineto{\pgfqpoint{2.874885in}{0.548769in}}%
\pgfpathclose%
\pgfusepath{fill}%
\end{pgfscope}%
\begin{pgfscope}%
\pgfsetbuttcap%
\pgfsetroundjoin%
\definecolor{currentfill}{rgb}{0.000000,0.000000,0.000000}%
\pgfsetfillcolor{currentfill}%
\pgfsetlinewidth{0.803000pt}%
\definecolor{currentstroke}{rgb}{0.000000,0.000000,0.000000}%
\pgfsetstrokecolor{currentstroke}%
\pgfsetdash{}{0pt}%
\pgfsys@defobject{currentmarker}{\pgfqpoint{0.000000in}{-0.048611in}}{\pgfqpoint{0.000000in}{0.000000in}}{%
\pgfpathmoveto{\pgfqpoint{0.000000in}{0.000000in}}%
\pgfpathlineto{\pgfqpoint{0.000000in}{-0.048611in}}%
\pgfusepath{stroke,fill}%
}%
\begin{pgfscope}%
\pgfsys@transformshift{2.874885in}{0.548769in}%
\pgfsys@useobject{currentmarker}{}%
\end{pgfscope}%
\end{pgfscope}%
\begin{pgfscope}%
\definecolor{textcolor}{rgb}{0.000000,0.000000,0.000000}%
\pgfsetstrokecolor{textcolor}%
\pgfsetfillcolor{textcolor}%
\pgftext[x=2.874885in,y=0.451547in,,top]{\color{textcolor}\rmfamily\fontsize{10.000000}{12.000000}\selectfont \(\displaystyle {0}\)}%
\end{pgfscope}%
\begin{pgfscope}%
\pgfsetbuttcap%
\pgfsetroundjoin%
\definecolor{currentfill}{rgb}{0.000000,0.000000,0.000000}%
\pgfsetfillcolor{currentfill}%
\pgfsetlinewidth{0.803000pt}%
\definecolor{currentstroke}{rgb}{0.000000,0.000000,0.000000}%
\pgfsetstrokecolor{currentstroke}%
\pgfsetdash{}{0pt}%
\pgfsys@defobject{currentmarker}{\pgfqpoint{0.000000in}{-0.048611in}}{\pgfqpoint{0.000000in}{0.000000in}}{%
\pgfpathmoveto{\pgfqpoint{0.000000in}{0.000000in}}%
\pgfpathlineto{\pgfqpoint{0.000000in}{-0.048611in}}%
\pgfusepath{stroke,fill}%
}%
\begin{pgfscope}%
\pgfsys@transformshift{3.521726in}{0.548769in}%
\pgfsys@useobject{currentmarker}{}%
\end{pgfscope}%
\end{pgfscope}%
\begin{pgfscope}%
\definecolor{textcolor}{rgb}{0.000000,0.000000,0.000000}%
\pgfsetstrokecolor{textcolor}%
\pgfsetfillcolor{textcolor}%
\pgftext[x=3.521726in,y=0.451547in,,top]{\color{textcolor}\rmfamily\fontsize{10.000000}{12.000000}\selectfont \(\displaystyle {2}\)}%
\end{pgfscope}%
\begin{pgfscope}%
\pgfsetbuttcap%
\pgfsetroundjoin%
\definecolor{currentfill}{rgb}{0.000000,0.000000,0.000000}%
\pgfsetfillcolor{currentfill}%
\pgfsetlinewidth{0.803000pt}%
\definecolor{currentstroke}{rgb}{0.000000,0.000000,0.000000}%
\pgfsetstrokecolor{currentstroke}%
\pgfsetdash{}{0pt}%
\pgfsys@defobject{currentmarker}{\pgfqpoint{0.000000in}{-0.048611in}}{\pgfqpoint{0.000000in}{0.000000in}}{%
\pgfpathmoveto{\pgfqpoint{0.000000in}{0.000000in}}%
\pgfpathlineto{\pgfqpoint{0.000000in}{-0.048611in}}%
\pgfusepath{stroke,fill}%
}%
\begin{pgfscope}%
\pgfsys@transformshift{4.168566in}{0.548769in}%
\pgfsys@useobject{currentmarker}{}%
\end{pgfscope}%
\end{pgfscope}%
\begin{pgfscope}%
\definecolor{textcolor}{rgb}{0.000000,0.000000,0.000000}%
\pgfsetstrokecolor{textcolor}%
\pgfsetfillcolor{textcolor}%
\pgftext[x=4.168566in,y=0.451547in,,top]{\color{textcolor}\rmfamily\fontsize{10.000000}{12.000000}\selectfont \(\displaystyle {4}\)}%
\end{pgfscope}%
\begin{pgfscope}%
\pgfsetbuttcap%
\pgfsetroundjoin%
\definecolor{currentfill}{rgb}{0.000000,0.000000,0.000000}%
\pgfsetfillcolor{currentfill}%
\pgfsetlinewidth{0.803000pt}%
\definecolor{currentstroke}{rgb}{0.000000,0.000000,0.000000}%
\pgfsetstrokecolor{currentstroke}%
\pgfsetdash{}{0pt}%
\pgfsys@defobject{currentmarker}{\pgfqpoint{0.000000in}{-0.048611in}}{\pgfqpoint{0.000000in}{0.000000in}}{%
\pgfpathmoveto{\pgfqpoint{0.000000in}{0.000000in}}%
\pgfpathlineto{\pgfqpoint{0.000000in}{-0.048611in}}%
\pgfusepath{stroke,fill}%
}%
\begin{pgfscope}%
\pgfsys@transformshift{4.815407in}{0.548769in}%
\pgfsys@useobject{currentmarker}{}%
\end{pgfscope}%
\end{pgfscope}%
\begin{pgfscope}%
\definecolor{textcolor}{rgb}{0.000000,0.000000,0.000000}%
\pgfsetstrokecolor{textcolor}%
\pgfsetfillcolor{textcolor}%
\pgftext[x=4.815407in,y=0.451547in,,top]{\color{textcolor}\rmfamily\fontsize{10.000000}{12.000000}\selectfont \(\displaystyle {6}\)}%
\end{pgfscope}%
\begin{pgfscope}%
\definecolor{textcolor}{rgb}{0.000000,0.000000,0.000000}%
\pgfsetstrokecolor{textcolor}%
\pgfsetfillcolor{textcolor}%
\pgftext[x=3.845146in,y=0.272534in,,top]{\color{textcolor}\rmfamily\fontsize{10.000000}{12.000000}\selectfont \(\displaystyle K\)}%
\end{pgfscope}%
\begin{pgfscope}%
\pgfsetbuttcap%
\pgfsetroundjoin%
\definecolor{currentfill}{rgb}{0.000000,0.000000,0.000000}%
\pgfsetfillcolor{currentfill}%
\pgfsetlinewidth{0.803000pt}%
\definecolor{currentstroke}{rgb}{0.000000,0.000000,0.000000}%
\pgfsetstrokecolor{currentstroke}%
\pgfsetdash{}{0pt}%
\pgfsys@defobject{currentmarker}{\pgfqpoint{-0.048611in}{0.000000in}}{\pgfqpoint{-0.000000in}{0.000000in}}{%
\pgfpathmoveto{\pgfqpoint{-0.000000in}{0.000000in}}%
\pgfpathlineto{\pgfqpoint{-0.048611in}{0.000000in}}%
\pgfusepath{stroke,fill}%
}%
\begin{pgfscope}%
\pgfsys@transformshift{2.874885in}{0.548769in}%
\pgfsys@useobject{currentmarker}{}%
\end{pgfscope}%
\end{pgfscope}%
\begin{pgfscope}%
\definecolor{textcolor}{rgb}{0.000000,0.000000,0.000000}%
\pgfsetstrokecolor{textcolor}%
\pgfsetfillcolor{textcolor}%
\pgftext[x=2.708218in, y=0.500544in, left, base]{\color{textcolor}\rmfamily\fontsize{10.000000}{12.000000}\selectfont \(\displaystyle {0}\)}%
\end{pgfscope}%
\begin{pgfscope}%
\pgfsetbuttcap%
\pgfsetroundjoin%
\definecolor{currentfill}{rgb}{0.000000,0.000000,0.000000}%
\pgfsetfillcolor{currentfill}%
\pgfsetlinewidth{0.803000pt}%
\definecolor{currentstroke}{rgb}{0.000000,0.000000,0.000000}%
\pgfsetstrokecolor{currentstroke}%
\pgfsetdash{}{0pt}%
\pgfsys@defobject{currentmarker}{\pgfqpoint{-0.048611in}{0.000000in}}{\pgfqpoint{-0.000000in}{0.000000in}}{%
\pgfpathmoveto{\pgfqpoint{-0.000000in}{0.000000in}}%
\pgfpathlineto{\pgfqpoint{-0.048611in}{0.000000in}}%
\pgfusepath{stroke,fill}%
}%
\begin{pgfscope}%
\pgfsys@transformshift{2.874885in}{0.899406in}%
\pgfsys@useobject{currentmarker}{}%
\end{pgfscope}%
\end{pgfscope}%
\begin{pgfscope}%
\definecolor{textcolor}{rgb}{0.000000,0.000000,0.000000}%
\pgfsetstrokecolor{textcolor}%
\pgfsetfillcolor{textcolor}%
\pgftext[x=2.708218in, y=0.851181in, left, base]{\color{textcolor}\rmfamily\fontsize{10.000000}{12.000000}\selectfont \(\displaystyle {1}\)}%
\end{pgfscope}%
\begin{pgfscope}%
\pgfsetbuttcap%
\pgfsetroundjoin%
\definecolor{currentfill}{rgb}{0.000000,0.000000,0.000000}%
\pgfsetfillcolor{currentfill}%
\pgfsetlinewidth{0.803000pt}%
\definecolor{currentstroke}{rgb}{0.000000,0.000000,0.000000}%
\pgfsetstrokecolor{currentstroke}%
\pgfsetdash{}{0pt}%
\pgfsys@defobject{currentmarker}{\pgfqpoint{-0.048611in}{0.000000in}}{\pgfqpoint{-0.000000in}{0.000000in}}{%
\pgfpathmoveto{\pgfqpoint{-0.000000in}{0.000000in}}%
\pgfpathlineto{\pgfqpoint{-0.048611in}{0.000000in}}%
\pgfusepath{stroke,fill}%
}%
\begin{pgfscope}%
\pgfsys@transformshift{2.874885in}{1.250043in}%
\pgfsys@useobject{currentmarker}{}%
\end{pgfscope}%
\end{pgfscope}%
\begin{pgfscope}%
\definecolor{textcolor}{rgb}{0.000000,0.000000,0.000000}%
\pgfsetstrokecolor{textcolor}%
\pgfsetfillcolor{textcolor}%
\pgftext[x=2.708218in, y=1.201818in, left, base]{\color{textcolor}\rmfamily\fontsize{10.000000}{12.000000}\selectfont \(\displaystyle {2}\)}%
\end{pgfscope}%
\begin{pgfscope}%
\pgfsetbuttcap%
\pgfsetroundjoin%
\definecolor{currentfill}{rgb}{0.000000,0.000000,0.000000}%
\pgfsetfillcolor{currentfill}%
\pgfsetlinewidth{0.803000pt}%
\definecolor{currentstroke}{rgb}{0.000000,0.000000,0.000000}%
\pgfsetstrokecolor{currentstroke}%
\pgfsetdash{}{0pt}%
\pgfsys@defobject{currentmarker}{\pgfqpoint{-0.048611in}{0.000000in}}{\pgfqpoint{-0.000000in}{0.000000in}}{%
\pgfpathmoveto{\pgfqpoint{-0.000000in}{0.000000in}}%
\pgfpathlineto{\pgfqpoint{-0.048611in}{0.000000in}}%
\pgfusepath{stroke,fill}%
}%
\begin{pgfscope}%
\pgfsys@transformshift{2.874885in}{1.600680in}%
\pgfsys@useobject{currentmarker}{}%
\end{pgfscope}%
\end{pgfscope}%
\begin{pgfscope}%
\definecolor{textcolor}{rgb}{0.000000,0.000000,0.000000}%
\pgfsetstrokecolor{textcolor}%
\pgfsetfillcolor{textcolor}%
\pgftext[x=2.708218in, y=1.552455in, left, base]{\color{textcolor}\rmfamily\fontsize{10.000000}{12.000000}\selectfont \(\displaystyle {3}\)}%
\end{pgfscope}%
\begin{pgfscope}%
\pgfsetbuttcap%
\pgfsetroundjoin%
\definecolor{currentfill}{rgb}{0.000000,0.000000,0.000000}%
\pgfsetfillcolor{currentfill}%
\pgfsetlinewidth{0.803000pt}%
\definecolor{currentstroke}{rgb}{0.000000,0.000000,0.000000}%
\pgfsetstrokecolor{currentstroke}%
\pgfsetdash{}{0pt}%
\pgfsys@defobject{currentmarker}{\pgfqpoint{-0.048611in}{0.000000in}}{\pgfqpoint{-0.000000in}{0.000000in}}{%
\pgfpathmoveto{\pgfqpoint{-0.000000in}{0.000000in}}%
\pgfpathlineto{\pgfqpoint{-0.048611in}{0.000000in}}%
\pgfusepath{stroke,fill}%
}%
\begin{pgfscope}%
\pgfsys@transformshift{2.874885in}{1.951318in}%
\pgfsys@useobject{currentmarker}{}%
\end{pgfscope}%
\end{pgfscope}%
\begin{pgfscope}%
\definecolor{textcolor}{rgb}{0.000000,0.000000,0.000000}%
\pgfsetstrokecolor{textcolor}%
\pgfsetfillcolor{textcolor}%
\pgftext[x=2.708218in, y=1.903092in, left, base]{\color{textcolor}\rmfamily\fontsize{10.000000}{12.000000}\selectfont \(\displaystyle {4}\)}%
\end{pgfscope}%
\begin{pgfscope}%
\pgfsetbuttcap%
\pgfsetroundjoin%
\definecolor{currentfill}{rgb}{0.000000,0.000000,0.000000}%
\pgfsetfillcolor{currentfill}%
\pgfsetlinewidth{0.803000pt}%
\definecolor{currentstroke}{rgb}{0.000000,0.000000,0.000000}%
\pgfsetstrokecolor{currentstroke}%
\pgfsetdash{}{0pt}%
\pgfsys@defobject{currentmarker}{\pgfqpoint{-0.048611in}{0.000000in}}{\pgfqpoint{-0.000000in}{0.000000in}}{%
\pgfpathmoveto{\pgfqpoint{-0.000000in}{0.000000in}}%
\pgfpathlineto{\pgfqpoint{-0.048611in}{0.000000in}}%
\pgfusepath{stroke,fill}%
}%
\begin{pgfscope}%
\pgfsys@transformshift{2.874885in}{2.301955in}%
\pgfsys@useobject{currentmarker}{}%
\end{pgfscope}%
\end{pgfscope}%
\begin{pgfscope}%
\definecolor{textcolor}{rgb}{0.000000,0.000000,0.000000}%
\pgfsetstrokecolor{textcolor}%
\pgfsetfillcolor{textcolor}%
\pgftext[x=2.708218in, y=2.253730in, left, base]{\color{textcolor}\rmfamily\fontsize{10.000000}{12.000000}\selectfont \(\displaystyle {5}\)}%
\end{pgfscope}%
\begin{pgfscope}%
\definecolor{textcolor}{rgb}{0.000000,0.000000,0.000000}%
\pgfsetstrokecolor{textcolor}%
\pgfsetfillcolor{textcolor}%
\pgftext[x=2.652662in,y=1.425362in,,bottom,rotate=90.000000]{\color{textcolor}\rmfamily\fontsize{10.000000}{12.000000}\selectfont \(\displaystyle K^\prime\)}%
\end{pgfscope}%
\begin{pgfscope}%
\pgfpathrectangle{\pgfqpoint{2.874885in}{0.548769in}}{\pgfqpoint{1.940523in}{1.753186in}}%
\pgfusepath{clip}%
\pgfsetrectcap%
\pgfsetroundjoin%
\pgfsetlinewidth{0.501875pt}%
\definecolor{currentstroke}{rgb}{0.501961,0.501961,0.501961}%
\pgfsetstrokecolor{currentstroke}%
\pgfsetdash{}{0pt}%
\pgfpathmoveto{\pgfqpoint{3.382912in}{0.548769in}}%
\pgfpathlineto{\pgfqpoint{3.382912in}{2.301955in}}%
\pgfusepath{stroke}%
\end{pgfscope}%
\begin{pgfscope}%
\pgfpathrectangle{\pgfqpoint{2.874885in}{0.548769in}}{\pgfqpoint{1.940523in}{1.753186in}}%
\pgfusepath{clip}%
\pgfsetrectcap%
\pgfsetroundjoin%
\pgfsetlinewidth{0.501875pt}%
\definecolor{currentstroke}{rgb}{0.501961,0.501961,0.501961}%
\pgfsetstrokecolor{currentstroke}%
\pgfsetdash{}{0pt}%
\pgfpathmoveto{\pgfqpoint{2.874885in}{1.099548in}}%
\pgfpathlineto{\pgfqpoint{4.815407in}{1.099548in}}%
\pgfusepath{stroke}%
\end{pgfscope}%
\begin{pgfscope}%
\pgfpathrectangle{\pgfqpoint{2.874885in}{0.548769in}}{\pgfqpoint{1.940523in}{1.753186in}}%
\pgfusepath{clip}%
\pgfsetrectcap%
\pgfsetroundjoin%
\pgfsetlinewidth{1.003750pt}%
\definecolor{currentstroke}{rgb}{0.121569,0.466667,0.705882}%
\pgfsetstrokecolor{currentstroke}%
\pgfsetdash{}{0pt}%
\pgfpathmoveto{\pgfqpoint{3.383004in}{2.315844in}}%
\pgfpathlineto{\pgfqpoint{3.383027in}{2.264692in}}%
\pgfpathlineto{\pgfqpoint{3.383116in}{2.164019in}}%
\pgfpathlineto{\pgfqpoint{3.383230in}{2.086013in}}%
\pgfpathlineto{\pgfqpoint{3.383370in}{2.022353in}}%
\pgfpathlineto{\pgfqpoint{3.383536in}{1.968602in}}%
\pgfpathlineto{\pgfqpoint{3.383728in}{1.922109in}}%
\pgfpathlineto{\pgfqpoint{3.383946in}{1.881163in}}%
\pgfpathlineto{\pgfqpoint{3.384190in}{1.844597in}}%
\pgfpathlineto{\pgfqpoint{3.384460in}{1.811577in}}%
\pgfpathlineto{\pgfqpoint{3.384756in}{1.781487in}}%
\pgfpathlineto{\pgfqpoint{3.385079in}{1.753859in}}%
\pgfpathlineto{\pgfqpoint{3.385429in}{1.728331in}}%
\pgfpathlineto{\pgfqpoint{3.385807in}{1.704613in}}%
\pgfpathlineto{\pgfqpoint{3.386211in}{1.682473in}}%
\pgfpathlineto{\pgfqpoint{3.386644in}{1.661720in}}%
\pgfpathlineto{\pgfqpoint{3.387104in}{1.642197in}}%
\pgfpathlineto{\pgfqpoint{3.387593in}{1.623771in}}%
\pgfpathlineto{\pgfqpoint{3.388110in}{1.606330in}}%
\pgfpathlineto{\pgfqpoint{3.388657in}{1.589779in}}%
\pgfpathlineto{\pgfqpoint{3.389233in}{1.574035in}}%
\pgfpathlineto{\pgfqpoint{3.389839in}{1.559026in}}%
\pgfpathlineto{\pgfqpoint{3.390475in}{1.544692in}}%
\pgfpathlineto{\pgfqpoint{3.391142in}{1.530976in}}%
\pgfpathlineto{\pgfqpoint{3.391841in}{1.517831in}}%
\pgfpathlineto{\pgfqpoint{3.392571in}{1.505213in}}%
\pgfpathlineto{\pgfqpoint{3.393334in}{1.493085in}}%
\pgfpathlineto{\pgfqpoint{3.394130in}{1.481412in}}%
\pgfpathlineto{\pgfqpoint{3.394960in}{1.470164in}}%
\pgfpathlineto{\pgfqpoint{3.395825in}{1.459313in}}%
\pgfpathlineto{\pgfqpoint{3.396725in}{1.448833in}}%
\pgfpathlineto{\pgfqpoint{3.397661in}{1.438702in}}%
\pgfpathlineto{\pgfqpoint{3.398633in}{1.428899in}}%
\pgfpathlineto{\pgfqpoint{3.399643in}{1.419406in}}%
\pgfpathlineto{\pgfqpoint{3.400692in}{1.410204in}}%
\pgfpathlineto{\pgfqpoint{3.401781in}{1.401278in}}%
\pgfpathlineto{\pgfqpoint{3.402910in}{1.392614in}}%
\pgfpathlineto{\pgfqpoint{3.404081in}{1.384197in}}%
\pgfpathlineto{\pgfqpoint{3.405294in}{1.376014in}}%
\pgfpathlineto{\pgfqpoint{3.406552in}{1.368056in}}%
\pgfpathlineto{\pgfqpoint{3.407855in}{1.360310in}}%
\pgfpathlineto{\pgfqpoint{3.409204in}{1.352766in}}%
\pgfpathlineto{\pgfqpoint{3.410602in}{1.345416in}}%
\pgfpathlineto{\pgfqpoint{3.412049in}{1.338250in}}%
\pgfpathlineto{\pgfqpoint{3.413548in}{1.331261in}}%
\pgfpathlineto{\pgfqpoint{3.415099in}{1.324441in}}%
\pgfpathlineto{\pgfqpoint{3.416706in}{1.317782in}}%
\pgfpathlineto{\pgfqpoint{3.418369in}{1.311279in}}%
\pgfpathlineto{\pgfqpoint{3.420091in}{1.304923in}}%
\pgfpathlineto{\pgfqpoint{3.421874in}{1.298711in}}%
\pgfpathlineto{\pgfqpoint{3.423720in}{1.292636in}}%
\pgfpathlineto{\pgfqpoint{3.425632in}{1.286693in}}%
\pgfpathlineto{\pgfqpoint{3.427613in}{1.280876in}}%
\pgfpathlineto{\pgfqpoint{3.429665in}{1.275182in}}%
\pgfpathlineto{\pgfqpoint{3.431792in}{1.269606in}}%
\pgfpathlineto{\pgfqpoint{3.433997in}{1.264143in}}%
\pgfpathlineto{\pgfqpoint{3.436283in}{1.258789in}}%
\pgfpathlineto{\pgfqpoint{3.438654in}{1.253542in}}%
\pgfpathlineto{\pgfqpoint{3.441114in}{1.248396in}}%
\pgfpathlineto{\pgfqpoint{3.443668in}{1.243349in}}%
\pgfpathlineto{\pgfqpoint{3.446321in}{1.238398in}}%
\pgfpathlineto{\pgfqpoint{3.449077in}{1.233539in}}%
\pgfpathlineto{\pgfqpoint{3.451943in}{1.228770in}}%
\pgfpathlineto{\pgfqpoint{3.454924in}{1.224087in}}%
\pgfpathlineto{\pgfqpoint{3.458028in}{1.219487in}}%
\pgfpathlineto{\pgfqpoint{3.461261in}{1.214970in}}%
\pgfpathlineto{\pgfqpoint{3.464631in}{1.210531in}}%
\pgfpathlineto{\pgfqpoint{3.468147in}{1.206168in}}%
\pgfpathlineto{\pgfqpoint{3.471820in}{1.201880in}}%
\pgfpathlineto{\pgfqpoint{3.475659in}{1.197664in}}%
\pgfpathlineto{\pgfqpoint{3.479676in}{1.193518in}}%
\pgfpathlineto{\pgfqpoint{3.483885in}{1.189440in}}%
\pgfpathlineto{\pgfqpoint{3.488300in}{1.185428in}}%
\pgfpathlineto{\pgfqpoint{3.492938in}{1.181480in}}%
\pgfpathlineto{\pgfqpoint{3.497817in}{1.177595in}}%
\pgfpathlineto{\pgfqpoint{3.502957in}{1.173771in}}%
\pgfpathlineto{\pgfqpoint{3.508384in}{1.170006in}}%
\pgfpathlineto{\pgfqpoint{3.514123in}{1.166299in}}%
\pgfpathlineto{\pgfqpoint{3.520206in}{1.162648in}}%
\pgfpathlineto{\pgfqpoint{3.526670in}{1.159052in}}%
\pgfpathlineto{\pgfqpoint{3.533555in}{1.155509in}}%
\pgfpathlineto{\pgfqpoint{3.540911in}{1.152019in}}%
\pgfpathlineto{\pgfqpoint{3.548796in}{1.148579in}}%
\pgfpathlineto{\pgfqpoint{3.557281in}{1.145189in}}%
\pgfpathlineto{\pgfqpoint{3.566449in}{1.141846in}}%
\pgfpathlineto{\pgfqpoint{3.576405in}{1.138552in}}%
\pgfpathlineto{\pgfqpoint{3.587275in}{1.135303in}}%
\pgfpathlineto{\pgfqpoint{3.599224in}{1.132099in}}%
\pgfpathlineto{\pgfqpoint{3.612461in}{1.128939in}}%
\pgfpathlineto{\pgfqpoint{3.627261in}{1.125822in}}%
\pgfpathlineto{\pgfqpoint{3.644002in}{1.122747in}}%
\pgfpathlineto{\pgfqpoint{3.663208in}{1.119713in}}%
\pgfpathlineto{\pgfqpoint{3.685656in}{1.116719in}}%
\pgfpathlineto{\pgfqpoint{3.712547in}{1.113764in}}%
\pgfpathlineto{\pgfqpoint{3.745902in}{1.110847in}}%
\pgfpathlineto{\pgfqpoint{3.789516in}{1.107968in}}%
\pgfpathlineto{\pgfqpoint{3.851932in}{1.105126in}}%
\pgfpathlineto{\pgfqpoint{3.960478in}{1.102320in}}%
\pgfpathlineto{\pgfqpoint{4.328852in}{1.099824in}}%
\pgfpathlineto{\pgfqpoint{4.700641in}{1.099576in}}%
\pgfpathlineto{\pgfqpoint{4.829296in}{1.099567in}}%
\pgfusepath{stroke}%
\end{pgfscope}%
\begin{pgfscope}%
\pgfpathrectangle{\pgfqpoint{2.874885in}{0.548769in}}{\pgfqpoint{1.940523in}{1.753186in}}%
\pgfusepath{clip}%
\pgfsetbuttcap%
\pgfsetroundjoin%
\definecolor{currentfill}{rgb}{0.121569,0.466667,0.705882}%
\pgfsetfillcolor{currentfill}%
\pgfsetlinewidth{1.003750pt}%
\definecolor{currentstroke}{rgb}{0.121569,0.466667,0.705882}%
\pgfsetstrokecolor{currentstroke}%
\pgfsetdash{}{0pt}%
\pgfsys@defobject{currentmarker}{\pgfqpoint{-0.006944in}{-0.006944in}}{\pgfqpoint{0.006944in}{0.006944in}}{%
\pgfpathmoveto{\pgfqpoint{0.000000in}{-0.006944in}}%
\pgfpathcurveto{\pgfqpoint{0.001842in}{-0.006944in}}{\pgfqpoint{0.003608in}{-0.006213in}}{\pgfqpoint{0.004910in}{-0.004910in}}%
\pgfpathcurveto{\pgfqpoint{0.006213in}{-0.003608in}}{\pgfqpoint{0.006944in}{-0.001842in}}{\pgfqpoint{0.006944in}{0.000000in}}%
\pgfpathcurveto{\pgfqpoint{0.006944in}{0.001842in}}{\pgfqpoint{0.006213in}{0.003608in}}{\pgfqpoint{0.004910in}{0.004910in}}%
\pgfpathcurveto{\pgfqpoint{0.003608in}{0.006213in}}{\pgfqpoint{0.001842in}{0.006944in}}{\pgfqpoint{0.000000in}{0.006944in}}%
\pgfpathcurveto{\pgfqpoint{-0.001842in}{0.006944in}}{\pgfqpoint{-0.003608in}{0.006213in}}{\pgfqpoint{-0.004910in}{0.004910in}}%
\pgfpathcurveto{\pgfqpoint{-0.006213in}{0.003608in}}{\pgfqpoint{-0.006944in}{0.001842in}}{\pgfqpoint{-0.006944in}{0.000000in}}%
\pgfpathcurveto{\pgfqpoint{-0.006944in}{-0.001842in}}{\pgfqpoint{-0.006213in}{-0.003608in}}{\pgfqpoint{-0.004910in}{-0.004910in}}%
\pgfpathcurveto{\pgfqpoint{-0.003608in}{-0.006213in}}{\pgfqpoint{-0.001842in}{-0.006944in}}{\pgfqpoint{0.000000in}{-0.006944in}}%
\pgfpathlineto{\pgfqpoint{0.000000in}{-0.006944in}}%
\pgfpathclose%
\pgfusepath{stroke,fill}%
}%
\begin{pgfscope}%
\pgfsys@transformshift{3.384190in}{1.844597in}%
\pgfsys@useobject{currentmarker}{}%
\end{pgfscope}%
\begin{pgfscope}%
\pgfsys@transformshift{3.388110in}{1.606330in}%
\pgfsys@useobject{currentmarker}{}%
\end{pgfscope}%
\begin{pgfscope}%
\pgfsys@transformshift{3.405294in}{1.376014in}%
\pgfsys@useobject{currentmarker}{}%
\end{pgfscope}%
\begin{pgfscope}%
\pgfsys@transformshift{3.441114in}{1.248396in}%
\pgfsys@useobject{currentmarker}{}%
\end{pgfscope}%
\begin{pgfscope}%
\pgfsys@transformshift{3.612461in}{1.128939in}%
\pgfsys@useobject{currentmarker}{}%
\end{pgfscope}%
\begin{pgfscope}%
\pgfsys@transformshift{3.960478in}{1.102320in}%
\pgfsys@useobject{currentmarker}{}%
\end{pgfscope}%
\end{pgfscope}%
\begin{pgfscope}%
\pgfsetrectcap%
\pgfsetmiterjoin%
\pgfsetlinewidth{0.803000pt}%
\definecolor{currentstroke}{rgb}{0.000000,0.000000,0.000000}%
\pgfsetstrokecolor{currentstroke}%
\pgfsetdash{}{0pt}%
\pgfpathmoveto{\pgfqpoint{2.874885in}{0.548769in}}%
\pgfpathlineto{\pgfqpoint{2.874885in}{2.301955in}}%
\pgfusepath{stroke}%
\end{pgfscope}%
\begin{pgfscope}%
\pgfsetrectcap%
\pgfsetmiterjoin%
\pgfsetlinewidth{0.803000pt}%
\definecolor{currentstroke}{rgb}{0.000000,0.000000,0.000000}%
\pgfsetstrokecolor{currentstroke}%
\pgfsetdash{}{0pt}%
\pgfpathmoveto{\pgfqpoint{4.815407in}{0.548769in}}%
\pgfpathlineto{\pgfqpoint{4.815407in}{2.301955in}}%
\pgfusepath{stroke}%
\end{pgfscope}%
\begin{pgfscope}%
\pgfsetrectcap%
\pgfsetmiterjoin%
\pgfsetlinewidth{0.803000pt}%
\definecolor{currentstroke}{rgb}{0.000000,0.000000,0.000000}%
\pgfsetstrokecolor{currentstroke}%
\pgfsetdash{}{0pt}%
\pgfpathmoveto{\pgfqpoint{2.874885in}{0.548769in}}%
\pgfpathlineto{\pgfqpoint{4.815407in}{0.548769in}}%
\pgfusepath{stroke}%
\end{pgfscope}%
\begin{pgfscope}%
\pgfsetrectcap%
\pgfsetmiterjoin%
\pgfsetlinewidth{0.803000pt}%
\definecolor{currentstroke}{rgb}{0.000000,0.000000,0.000000}%
\pgfsetstrokecolor{currentstroke}%
\pgfsetdash{}{0pt}%
\pgfpathmoveto{\pgfqpoint{2.874885in}{2.301955in}}%
\pgfpathlineto{\pgfqpoint{4.815407in}{2.301955in}}%
\pgfusepath{stroke}%
\end{pgfscope}%
\begin{pgfscope}%
\definecolor{textcolor}{rgb}{0.000000,0.000000,0.000000}%
\pgfsetstrokecolor{textcolor}%
\pgfsetfillcolor{textcolor}%
\pgftext[x=2.907227in,y=1.134612in,left,base]{\color{textcolor}\rmfamily\fontsize{10.000000}{12.000000}\selectfont \(\displaystyle \pi/2\)}%
\end{pgfscope}%
\begin{pgfscope}%
\definecolor{textcolor}{rgb}{0.000000,0.000000,0.000000}%
\pgfsetstrokecolor{textcolor}%
\pgfsetfillcolor{textcolor}%
\pgftext[x=3.415254in,y=0.583833in,left,base]{\color{textcolor}\rmfamily\fontsize{10.000000}{12.000000}\selectfont \(\displaystyle \pi/2\)}%
\end{pgfscope}%
\begin{pgfscope}%
\definecolor{textcolor}{rgb}{0.000000,0.000000,0.000000}%
\pgfsetstrokecolor{textcolor}%
\pgfsetfillcolor{textcolor}%
\pgftext[x=3.416532in,y=1.879661in,left,base]{\color{textcolor}\rmfamily\fontsize{10.000000}{12.000000}\selectfont \(\displaystyle k=0.10\)}%
\end{pgfscope}%
\begin{pgfscope}%
\definecolor{textcolor}{rgb}{0.000000,0.000000,0.000000}%
\pgfsetstrokecolor{textcolor}%
\pgfsetfillcolor{textcolor}%
\pgftext[x=3.420452in,y=1.641394in,left,base]{\color{textcolor}\rmfamily\fontsize{10.000000}{12.000000}\selectfont \(\displaystyle k=0.20\)}%
\end{pgfscope}%
\begin{pgfscope}%
\definecolor{textcolor}{rgb}{0.000000,0.000000,0.000000}%
\pgfsetstrokecolor{textcolor}%
\pgfsetfillcolor{textcolor}%
\pgftext[x=3.437636in,y=1.411078in,left,base]{\color{textcolor}\rmfamily\fontsize{10.000000}{12.000000}\selectfont \(\displaystyle k=0.40\)}%
\end{pgfscope}%
\begin{pgfscope}%
\definecolor{textcolor}{rgb}{0.000000,0.000000,0.000000}%
\pgfsetstrokecolor{textcolor}%
\pgfsetfillcolor{textcolor}%
\pgftext[x=3.473456in,y=1.283460in,left,base]{\color{textcolor}\rmfamily\fontsize{10.000000}{12.000000}\selectfont \(\displaystyle k=0.60\)}%
\end{pgfscope}%
\begin{pgfscope}%
\definecolor{textcolor}{rgb}{0.000000,0.000000,0.000000}%
\pgfsetstrokecolor{textcolor}%
\pgfsetfillcolor{textcolor}%
\pgftext[x=3.644803in,y=1.164003in,left,base]{\color{textcolor}\rmfamily\fontsize{10.000000}{12.000000}\selectfont \(\displaystyle k=0.90\)}%
\end{pgfscope}%
\begin{pgfscope}%
\definecolor{textcolor}{rgb}{0.000000,0.000000,0.000000}%
\pgfsetstrokecolor{textcolor}%
\pgfsetfillcolor{textcolor}%
\pgftext[x=3.992820in,y=1.137383in,left,base]{\color{textcolor}\rmfamily\fontsize{10.000000}{12.000000}\selectfont \(\displaystyle k=0.99\)}%
\end{pgfscope}%
\end{pgfpicture}%
\makeatother%
\endgroup%

    \caption{Die Periodizitäten in realer und imaginärer Richtung in Abhängigkeit vom elliptischen Modul $k$.}
    \label{ellfilter:fig:kprime}
\end{figure}
$K$ und $K^\prime$ sind durch die Ortskurve $K + jK^\prime$ aneinander gebunden und benötigen den Zusatzfaktor $K_1/K$ in \eqref{ellfilter:eq:elliptic}, um die genanten Forderungen einzuhalten.
Abbildung \ref{ellfilter:fig:degree_eq} zeigt das Problem geometrisch auf, wobei zwei Punkte $K+jK^\prime$ und $K_1+jK_1^\prime$ auf der Ortskurve gesucht sind.
\begin{figure}
    \centering
    
\def\d{0.2}
\def\n{3}
\def\nn{2}
\def\a{2.5}

\begin{tikzpicture}[>=stealth', auto, node distance=2cm, scale=1.2]

    \tikzstyle{zero} = [draw, circle, inner sep =0, minimum height=0.15cm]
    \tikzstyle{dot} = [fill, circle, inner sep =0, minimum height=0.1cm]

    \tikzset{pole/.style={cross out, draw, minimum size=(0.15cm-\pgflinewidth), inner sep=0pt, outer sep=0pt}}

    \begin{scope}[xscale=3, yscale=3]

    \begin{scope}[]
        % \onslide<4->{
        \fill[orange!30, scale=1.735] (0,0) rectangle (\d*\a+0.5, \d/\a+0.5);
        % }
        % \onslide<2->{
        \fill[yellow!30] (0,0) rectangle (\d*\a+0.5, \d/\a+0.5);
        % }

        \begin{scope}[]
            \clip(0,0) rectangle (2,1.25);
            \draw[thick, scale=1, domain=0.1:10,  variable=\x, smooth, samples=200] plot ({\d*\x1+0.5}, {\d/\x+0.5});
            \node at(1.25,0.7) {$K + jK^\prime$ Ortskurve};
        \end{scope}

        % \onslide<2->{
        \begin{scope}[blue]
            \draw[] (0,0) rectangle (\d*\a+0.5, \d/\a+0.5);


            \node[pole] at ( \d*\a+0.5, \d/\a+0.5) {};
            \node[zero] at ( \d*\a+0.5, 0) {};

            \draw[] ( \d*\a+0.5,0)  node[anchor=north] {\small $K$};
            \draw[]  (0, \d/\a+0.5)  node[anchor=east]{\small $jK^\prime$};

            % \onslide<3->{

            \foreach \i in {1,...,\nn} {
                \draw[gray, dotted] (\i*\d*\a/\n+\i*0.5/\n, 0) -- (\i*\d*\a/\n+\i*0.5/\n, \d/\a+0.5);
            }

            \node[dot, gray] at (\d*\a/\n+0.5/\n, \d/\a+0.5) {};
            \node[above] at (0.5*\d*\a/\n+0.5*0.5/\n, \d/\a+0.5) {\small $K/N$};
            % }
        \end{scope}
        % }

        % \onslide<4->{
        \begin{scope}[scale=1.735, red]
            \draw (0,0) rectangle (\d*\a/\n+0.5/\n, \d/\a+0.5);
            \draw[gray] (0,0) -- (\d*\a/\n+0.5/\n, \d/\a+0.5);

            \node[pole] at ( \d*\a/\n+0.5/\n, \d/\a+0.5) {};
            \node[zero] at ( \d*\a/\n+0.5/\n, 0) {};


            \draw[] ( \d*\a/\n+0.5/\n,0)  node[anchor=north] {\small $K_1$};
            \draw[]  (0, \d/\a+0.5)  node[anchor=east]{\small $jK_1^\prime$};

        \end{scope}
        % }

        \draw[gray, ->] (0,-0.25) -- (0,1.25) node[anchor=south]{$\mathrm{Im}$};
        \draw[gray, ->] (-0.25,0) -- (2,0) node[anchor=west]{$\mathrm{Re}$};

    \end{scope}

\end{scope}

\end{tikzpicture}

    \caption{Die Gradgleichung als geometrisches Problem ($N=3$).}
    \label{ellfilter:fig:degree_eq}
\end{figure}
Algebraisch kann so die Gradgleichung
\begin{equation}
    N \frac{K^\prime}{K} = \frac{K^\prime_1}{K_1}
\end{equation}
aufgestellt werden, dessen Lösung ist gegeben durch
\begin{equation}\label{ellfilter:eq:degeqsol}
k_1 = k^N \prod_{i=1}^L \sn^4 \Bigg( \frac{2i - 1}{N} K, k \Bigg),
\quad \text{wobei} \quad
N = 2L+r.
\end{equation}
Die Herleitung ist sehr umfassend und wird in \cite{ellfilter:bib:orfanidis} im Detail angeschaut.

\subsection{Berechnung der rationalen Funktion}

$k_1$ muss jedoch gar nicht berechnet werden, um $R_N$ in der Form einer rationale Funktion erhalten.
Die Ordnung $N$ und der Parameter $k$ können frei gewählt werden.
% $k_1$ muss dann mit \eqref{ellfilter:eq:degeqsol} oder mit numerischen Methoden berechnet werden.
Je kleiner $k$ gewählt wird, desto grösser wird die Dämpfung des Filters im Sperrbereich im Verhältnis zum Durchlassbereich.
Allerdings verliert das Filter dabei auch an Steilheit.
Wenn $k$ und $N$ bekannt sind, können die Position der Pol- und Nullstellen $p_i$ und $n_i$ in einem Raster konstruiert werden, wie dargestellt in Abbildung \ref{ellfilter:fig:pn}.
\begin{figure}
    \centering
    \begin{tikzpicture}[>=stealth', auto, node distance=2cm, scale=1.2, thick]

    \tikzstyle{zero} = [draw, circle, inner sep =0, minimum height=0.15cm]
    \tikzstyle{dot} = [fill, circle, inner sep =0, minimum height=0.1cm]

    \tikzset{pole/.style={cross out, draw=black, minimum size=(0.15cm-\pgflinewidth), inner sep=0pt, outer sep=0pt}}

    \begin{scope}[xscale=0.75, yscale=2.5]

        \fill[orange!30] (0,0) rectangle (5, 0.5);
        % \fill[yellow!30] (0,0) rectangle (1, 0.1);
        \node[] at (5, 0.25) {\small $N=5$};

        \draw[gray, ->] (0,-0.25) -- (0,0.75) node[anchor=south]{$\mathrm{Im}~z$};
        \draw[gray, ->] (-1,0) -- (11,0) node[anchor=west]{$\mathrm{Re}~z$};

        \draw[gray] ( 5,0) +(0,0.035) -- +(0, -0.035) node[inner sep=0, anchor=north] {\small $K$};
        \draw[gray]  (0, 0.5) +(0.1, 0) -- +(-0.1, 0) node[inner sep=0, anchor=east]{\small $jK^\prime$};

        \begin{scope}

            \draw[ultra thick, ->, purple] (5, 0.5) -- (10,0.5);
            \draw[ultra thick, ->, blue] (10, 0.5) --  (10,0);
            \draw[ultra thick, ->, cyan] (10, 0) --  (5,0);
            \draw[ultra thick, ->, darkgreen] (5, 0) --  (0,0);
            \draw[ultra thick, ->, orange] (-0, 0) -- (0,0.5);
            \draw[ultra thick, ->, red] (0,0.5) -- (5, 0.5);

            \foreach \i in {1,...,5} {
                \begin{scope}[xshift=(\i-1)*2cm]
                    \node[zero] at ( 1, 0) {};
                    \node[anchor=south west] at ( 1, 0) {$n_\i$};
                    \node[pole] at ( 1,0.5) {};
                    \node[anchor=south west] at ( 1, 0.5) {$p_\i$};
                \end{scope}
            }

        \end{scope}

    \end{scope}

    % \begin{scope}[yshift=-1.5cm, xshift=3.75cm, xscale=0.75]

    %     \draw[gray, ->] (-6,0) -- (6,0) node[anchor=west]{$w$};

    %     \draw[ultra thick, ->, purple] (-5, 0) -- (-3, 0);
    %     \draw[ultra thick, ->, blue]      (-3, 0) -- (-2, 0);
    %     \draw[ultra thick, ->, cyan]       (-2, 0) -- (0, 0);
    %     \draw[ultra thick, ->, darkgreen]    (0, 0) -- (2, 0);
    %     \draw[ultra thick, ->, orange] (2, 0) -- (3, 0);
    %     \draw[ultra thick, ->, red] (3, 0) -- (5, 0);

    %     \node[anchor=south] at (-5,0) {$-\infty$};
    %     \node[anchor=south] at (-3,0) {$-1/k$};
    %     \node[anchor=south] at (-2,0) {$-1$};
    %     \node[anchor=south] at (0,0) {$0$};
    %     \node[anchor=south] at (2,0) {$1$};
    %     \node[anchor=south] at (3,0) {$1/k$};
    %     \node[anchor=south] at (5,0) {$\infty$};

    % \end{scope}

\end{tikzpicture}

    \caption{
        Pole und Nullstellen in der $z = \cd^{-1}(w, k)$-Ebene für die Rücktransformation zur einer rationalen Funktion.
    }
    \label{ellfilter:fig:pn}
\end{figure}
Dabei muss aufgepasst werden, dass insgesamt nur $N$ Nullstellen und $N$ Pole gesetzt werden, da bei der transformation mit dem $\cd$ mehrere Werte auf einen abgebildet werden und mehrfache Pole und Nullstellen nicht erwünscht sind.
Wegen der Periodizität sind diese in der komplexen $z$-Ebene linear angeordnet:
\begin{align}
    n_i(k) &= K\frac{2i+1}{N} \\
    p_i(k) &= n_i + jK^\prime.
\end{align}
Durch das Rücktransformieren mit der $\cd$-Funktion gelangt man schlussendlich zu der rationalen Funktion
\begin{equation}
    R_N(w, k) = r_0 \prod_{i=1}^N \frac{w - \cd \big(n_i(k), k \big)}{w - \cd \big(p_i(k), k \big)},
\end{equation}
wobei $r_0$ so gewählt werden muss, dass $R_N(w, k) = 1$.

\section{Elliptisches Filter}

Um ein elliptisches Filter auszulegen werden aber nicht die Pol- und Nullstellen der rationalen Funktion gebraucht, sondern diejenigen der Übertragungsfunktion $H(s)$ der komplexen Frequenz $s = j\Omega + \sigma$.
Der Bezug zum quadratischen Amplitudengang \eqref{ellfilter:eq:quadratic_transfer} ist dabei
\begin{equation}
    |H(\Omega)|^2 = H(s) H(s^*),
\end{equation}
wobei $*$ die komplexe Konjugation kennzeichnet.
Die genaue Berechnung geht einiges tiefer in die Filtertheorie, und verlässt das Gebiet der speziellen Funktionen.
Der interessierte Leser wird auf \cite[Kapitel~5]{ellfilter:bib:orfanidis} verwiesen.

% \subsection{Schlussfolgerung}

% Die elliptischen Filter können als direkte Erweiterung der Tschebyscheff-Filter verstanden werden.
% Bei den Tschebyscheff-Polynomen haben wir gesehen, dass die Trigonometrische Formel zu einfachen Polynomen umgewandelt werden kann.
% Im elliptischen Fall entstehen so rationale Funktionen mit Nullstellen und auch Pole.
