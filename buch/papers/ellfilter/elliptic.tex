\section{Elliptische rationale Funktionen}

Kommen wir nun zum eigentlichen Teil dieses Papers, den elliptischen rationalen Funktionen
\begin{align}
    R_N(\xi, w) &= \cd \left(N~f_1(\xi)~\cd^{-1}(w, 1/\xi), f_2(\xi)\right) \\
                &= \cd \left(N~\frac{K_1}{K}~\cd^{-1}(w, k), k_1)\right) , \quad k= 1/\xi, k_1 = 1/f(\xi) \\
                &= \cd \left(N~K_1~z , k_1 \right), \quad w= \cd(z K, k)
\end{align}


sieht ähnlich aus wie die trigonometrische Darstellung der Tschebyschef-Polynome \eqref{ellfilter:eq:chebychef_polynomials}
Anstelle vom Kosinus kommt hier die $\cd$-Funktion zum Einsatz.
Die Ordnungszahl $N$ kommt auch als Faktor for.
Zusätzlich werden noch zwei verschiedene elliptische Module $k$ und $k_1$ gebraucht.



Sinus entspricht $\sn$

Damit die Nullstellen an ähnlichen Positionen zu liegen kommen wie bei den Tschebyscheff-Polynomen, muss die $\cd$-Funktion gewählt werden.

Die $\cd^{-1}(w, k)$-Funktion ist um $K$ verschoben zur $\sn^{-1}(w, k)$-Funktion, wie ersichtlich in Abbildung \ref{ellfilter:fig:cd}.
\begin{figure}
    \centering
    \begin{tikzpicture}[>=stealth', auto, node distance=2cm, scale=1.2]

    \tikzstyle{zero} = [draw, circle, inner sep =0, minimum height=0.15cm]

    \tikzset{pole/.style={cross out, draw=black, minimum size=(0.15cm-\pgflinewidth), inner sep=0pt, outer sep=0pt}}

    \begin{scope}[xscale=0.9, yscale=1.8]

        \draw[gray, ->] (0,-1.5) -- (0,1.5) node[anchor=south]{$\mathrm{Im}~z$};
        \draw[gray, ->] (-5,0) -- (5,0) node[anchor=west]{$\mathrm{Re}~z$};

        \draw[gray] ( 1,0) +(0,0.1) -- +(0, -0.1) node[inner sep=0, anchor=north] {\small $K$};

        \draw[gray]  (0, 0.5) +(0.1, 0) -- +(-0.1, 0) node[inner sep=0, anchor=east]{\small $jK^\prime$};


        \begin{scope}

            \begin{scope}[xshift=0cm]

                \clip(-4.5,-1.25) rectangle (4.5,1.25);

                \fill[yellow!30] (0,0) rectangle (1, 0.5);


                \draw[ultra thick, ->, darkgreen] (0, 0) -- (0,0.5);
                \draw[ultra thick, ->, orange] (1, 0) -- (0,0);
                \draw[ultra thick, ->, red] (2, 0) -- (1,0);
                \draw[ultra thick, ->, blue] (2,0.5) -- (2, 0);
                \draw[ultra thick, ->, purple] (1, 0.5) -- (2,0.5);
                \draw[ultra thick, ->, cyan] (0, 0.5) -- (1,0.5);



                \foreach \i in {-2,...,1} {
                    \foreach \j in {-2,...,1} {
                        \begin{scope}[xshift=\i*4cm, yshift=\j*1cm]
                            \draw[opacity=0.5, ->, darkgreen] (0, 0) -- (0,0.5);
                            \draw[opacity=0.5, ->, orange] (1, 0) -- (0,0);
                            \draw[opacity=0.5, ->, red] (2, 0) -- (1,0);
                            \draw[opacity=0.5, ->, blue] (2,0.5) -- (2, 0);
                            \draw[opacity=0.5, ->, purple] (1, 0.5) -- (2,0.5);
                            \draw[opacity=0.5, ->, cyan] (0, 0.5) -- (1,0.5);
                            \draw[opacity=0.5, ->, darkgreen] (0,1) -- (0,0.5);
                            \draw[opacity=0.5, ->, blue] (2,0.5) -- (2, 1);
                            \draw[opacity=0.5, ->, purple] (3, 0.5) -- (2,0.5);
                            \draw[opacity=0.5, ->, cyan] (4, 0.5) -- (3,0.5);
                            \draw[opacity=0.5, ->, red] (2, 0) -- (3,0);
                            \draw[opacity=0.5, ->, orange] (3, 0) -- (4,0);

                            \node[zero] at ( 1, 0) {};
                            \node[zero] at ( 3, 0) {};
                            \node[pole] at ( 1,0.5) {};
                            \node[pole] at ( 3,0.5) {};

                        \end{scope}
                    }
                }

            \end{scope}

        \end{scope}

    \end{scope}

    \node[zero] at (4,3) (n) {};
    \node[anchor=west] at (n.east) {Zero};
    \node[pole, below=0.25cm of n] (n) {};
    \node[anchor=west] at (n.east) {Pole};

    \begin{scope}[yshift=-4cm, xscale=0.75]

        \draw[gray, ->] (-6,0) -- (6,0) node[anchor=west]{$w$};

        \draw[thick, ->, purple] (-5, 0) -- (-3, 0);
        \draw[thick, ->, blue]      (-3, 0) -- (-2, 0);
        \draw[thick, ->, red]       (-2, 0) -- (0, 0);
        \draw[thick, ->, orange]    (0, 0) -- (2, 0);
        \draw[thick, ->, darkgreen] (2, 0) -- (3, 0);
        \draw[thick, ->, cyan] (3, 0) -- (5, 0);

        \node[anchor=south] at (-5,0) {$-\infty$};
        \node[anchor=south] at (-3,0) {$-1/k$};
        \node[anchor=south] at (-2,0) {$-1$};
        \node[anchor=south] at (0,0) {$0$};
        \node[anchor=south] at (2,0) {$1$};
        \node[anchor=south] at (3,0) {$1/k$};
        \node[anchor=south] at (5,0) {$\infty$};

    \end{scope}

\end{tikzpicture}
    \caption{
        $z$-Ebene der Funktion $z = \sn^{-1}(w, k)$.
        Die Funktion ist in der realen Achse $4K$-periodisch und in der imaginären Achse $2jK^\prime$-periodisch.
    }
    \label{ellfilter:fig:cd}
\end{figure}
Auffallend ist, dass sich alle Nullstellen und Polstellen um $K$ verschoben haben.

Durch das Konzept vom fundamentalen Rechteck, siehe Abbildung \ref{ellfilter:fig:fundamental_rectangle} können für alle inversen Jaccobi elliptischen Funktionen die Positionen der Null- und Polstellen anhand eines Diagramms ermittelt werden.
Der erste Buchstabe bestimmt die Position der Nullstelle und der zweite Buchstabe die Polstelle.
\begin{figure}
    \centering
    \begin{tikzpicture}[>=stealth', auto, node distance=2cm, scale=1.2]

    \tikzstyle{zero} = [draw, circle, inner sep =0, minimum height=0.15cm]

    \tikzset{pole/.style={cross out, draw=black, minimum size=(0.15cm-\pgflinewidth), inner sep=0pt, outer sep=0pt}}

    \begin{scope}[xscale=2, yscale=2]

        \draw[gray, ->] (0,-0.25) -- (0,1.25) node[anchor=south]{$\mathrm{Im}~z$};
        \draw[gray, ->] (-0.25,0) -- (1.5,0) node[anchor=west]{$\mathrm{Re}~z$};

        \draw[gray] ( 1,0) +(0,0.05) -- +(0, -0.05) node[inner sep=0, anchor=north] {\small $K$};

        \draw[gray]  (0, 1) +(0.05, 0) -- +(-0.05, 0) node[inner sep=0, anchor=east]{\small $jK^\prime$};

        \fill[yellow!50] (0,0) rectangle (1, 1);

        \node[anchor=south east] at ( 1,0) {$c$};
        \node[anchor=north east] at ( 1,1) {$d$};
        \node[anchor=north west] at ( 0,1) {$n$};
        \node[anchor=south west] at ( 0,0) {$s$};

    \end{scope}


\end{tikzpicture}
    \caption{
        Fundamentales Rechteck der inversen Jaccobi elliptischen Funktionen.
    }
    \label{ellfilter:fig:fundamental_rectangle}
\end{figure}

Auffallend an der $w = \sn(z, k)$-Funktion ist, dass sich $w$ auf der reellen Achse wie der Kosinus immer zwischen $-1$ und $1$ bewegt, während bei $\mathrm{Im(z) = K^\prime}$ die Werte zwischen $\pm 1/k$ und $\pm \infty$ verlaufen.
Die Funktion hat also Equirippel-Verhalten um $w=0$ und um $w=\pm \infty$.
Falls es möglich ist diese Werte abzufahren im Sti der Tschebyscheff-Polynome, kann ein Filter gebaut werden, dass Equirippel-Verhalten im Durchlass- und Sperrbereich aufweist.



Analog zu Abbildung \ref{ellfilter:fig:arccos2} können wir auch bei den elliptisch rationalen Funktionen die komplexe $z$-Ebene betrachten, wie ersichtlich in Abbildung \ref{ellfilter:fig:cd2}, um die besser zu verstehen.
\begin{figure}
    \centering
    \input{papers/ellfilter/tikz/cd2.tikz.tex}
    \caption{
        $z_1$-Ebene der elliptischen rationalen Funktionen.
        Je grösser die Ordnung $N$ gewählt wird, desto mehr Nullstellen passiert.
    }
    \label{ellfilter:fig:cd2}
\end{figure}
% Da die $\cd^{-1}$-Funktion 



\begin{figure}
    \centering
    %% Creator: Matplotlib, PGF backend
%%
%% To include the figure in your LaTeX document, write
%%   \input{<filename>.pgf}
%%
%% Make sure the required packages are loaded in your preamble
%%   \usepackage{pgf}
%%
%% Also ensure that all the required font packages are loaded; for instance,
%% the lmodern package is sometimes necessary when using math font.
%%   \usepackage{lmodern}
%%
%% Figures using additional raster images can only be included by \input if
%% they are in the same directory as the main LaTeX file. For loading figures
%% from other directories you can use the `import` package
%%   \usepackage{import}
%%
%% and then include the figures with
%%   \import{<path to file>}{<filename>.pgf}
%%
%% Matplotlib used the following preamble
%%
\begingroup%
\makeatletter%
\begin{pgfpicture}%
\pgfpathrectangle{\pgfpointorigin}{\pgfqpoint{4.000000in}{2.500000in}}%
\pgfusepath{use as bounding box, clip}%
\begin{pgfscope}%
\pgfsetbuttcap%
\pgfsetmiterjoin%
\pgfsetlinewidth{0.000000pt}%
\definecolor{currentstroke}{rgb}{1.000000,1.000000,1.000000}%
\pgfsetstrokecolor{currentstroke}%
\pgfsetstrokeopacity{0.000000}%
\pgfsetdash{}{0pt}%
\pgfpathmoveto{\pgfqpoint{0.000000in}{0.000000in}}%
\pgfpathlineto{\pgfqpoint{4.000000in}{0.000000in}}%
\pgfpathlineto{\pgfqpoint{4.000000in}{2.500000in}}%
\pgfpathlineto{\pgfqpoint{0.000000in}{2.500000in}}%
\pgfpathlineto{\pgfqpoint{0.000000in}{0.000000in}}%
\pgfpathclose%
\pgfusepath{}%
\end{pgfscope}%
\begin{pgfscope}%
\pgfsetbuttcap%
\pgfsetmiterjoin%
\definecolor{currentfill}{rgb}{1.000000,1.000000,1.000000}%
\pgfsetfillcolor{currentfill}%
\pgfsetlinewidth{0.000000pt}%
\definecolor{currentstroke}{rgb}{0.000000,0.000000,0.000000}%
\pgfsetstrokecolor{currentstroke}%
\pgfsetstrokeopacity{0.000000}%
\pgfsetdash{}{0pt}%
\pgfpathmoveto{\pgfqpoint{0.733531in}{0.548769in}}%
\pgfpathlineto{\pgfqpoint{3.761597in}{0.548769in}}%
\pgfpathlineto{\pgfqpoint{3.761597in}{2.301955in}}%
\pgfpathlineto{\pgfqpoint{0.733531in}{2.301955in}}%
\pgfpathlineto{\pgfqpoint{0.733531in}{0.548769in}}%
\pgfpathclose%
\pgfusepath{fill}%
\end{pgfscope}%
\begin{pgfscope}%
\pgfpathrectangle{\pgfqpoint{0.733531in}{0.548769in}}{\pgfqpoint{3.028066in}{1.753186in}}%
\pgfusepath{clip}%
\pgfsetbuttcap%
\pgfsetmiterjoin%
\definecolor{currentfill}{rgb}{0.000000,0.501961,0.000000}%
\pgfsetfillcolor{currentfill}%
\pgfsetfillopacity{0.200000}%
\pgfsetlinewidth{0.000000pt}%
\definecolor{currentstroke}{rgb}{0.000000,0.000000,0.000000}%
\pgfsetstrokecolor{currentstroke}%
\pgfsetstrokeopacity{0.200000}%
\pgfsetdash{}{0pt}%
\pgfpathmoveto{\pgfqpoint{0.733531in}{-174.068564in}}%
\pgfpathlineto{\pgfqpoint{2.247564in}{-174.068564in}}%
\pgfpathlineto{\pgfqpoint{2.247564in}{1.250043in}}%
\pgfpathlineto{\pgfqpoint{0.733531in}{1.250043in}}%
\pgfpathlineto{\pgfqpoint{0.733531in}{-174.068564in}}%
\pgfpathclose%
\pgfusepath{fill}%
\end{pgfscope}%
\begin{pgfscope}%
\pgfpathrectangle{\pgfqpoint{0.733531in}{0.548769in}}{\pgfqpoint{3.028066in}{1.753186in}}%
\pgfusepath{clip}%
\pgfsetbuttcap%
\pgfsetmiterjoin%
\definecolor{currentfill}{rgb}{1.000000,0.647059,0.000000}%
\pgfsetfillcolor{currentfill}%
\pgfsetfillopacity{0.200000}%
\pgfsetlinewidth{0.000000pt}%
\definecolor{currentstroke}{rgb}{0.000000,0.000000,0.000000}%
\pgfsetstrokecolor{currentstroke}%
\pgfsetstrokeopacity{0.200000}%
\pgfsetdash{}{0pt}%
\pgfpathmoveto{\pgfqpoint{2.247564in}{1.250043in}}%
\pgfpathlineto{\pgfqpoint{2.262704in}{1.250043in}}%
\pgfpathlineto{\pgfqpoint{2.262704in}{1.600680in}}%
\pgfpathlineto{\pgfqpoint{2.247564in}{1.600680in}}%
\pgfpathlineto{\pgfqpoint{2.247564in}{1.250043in}}%
\pgfpathclose%
\pgfusepath{fill}%
\end{pgfscope}%
\begin{pgfscope}%
\pgfpathrectangle{\pgfqpoint{0.733531in}{0.548769in}}{\pgfqpoint{3.028066in}{1.753186in}}%
\pgfusepath{clip}%
\pgfsetbuttcap%
\pgfsetmiterjoin%
\definecolor{currentfill}{rgb}{1.000000,0.000000,0.000000}%
\pgfsetfillcolor{currentfill}%
\pgfsetfillopacity{0.200000}%
\pgfsetlinewidth{0.000000pt}%
\definecolor{currentstroke}{rgb}{0.000000,0.000000,0.000000}%
\pgfsetstrokecolor{currentstroke}%
\pgfsetstrokeopacity{0.200000}%
\pgfsetdash{}{0pt}%
\pgfpathmoveto{\pgfqpoint{2.262704in}{1.600680in}}%
\pgfpathlineto{\pgfqpoint{3.776737in}{1.600680in}}%
\pgfpathlineto{\pgfqpoint{3.776737in}{2.301962in}}%
\pgfpathlineto{\pgfqpoint{2.262704in}{2.301962in}}%
\pgfpathlineto{\pgfqpoint{2.262704in}{1.600680in}}%
\pgfpathclose%
\pgfusepath{fill}%
\end{pgfscope}%
\begin{pgfscope}%
\pgfpathrectangle{\pgfqpoint{0.733531in}{0.548769in}}{\pgfqpoint{3.028066in}{1.753186in}}%
\pgfusepath{clip}%
\pgfsetrectcap%
\pgfsetroundjoin%
\pgfsetlinewidth{0.803000pt}%
\definecolor{currentstroke}{rgb}{0.690196,0.690196,0.690196}%
\pgfsetstrokecolor{currentstroke}%
\pgfsetdash{}{0pt}%
\pgfpathmoveto{\pgfqpoint{0.733531in}{0.548769in}}%
\pgfpathlineto{\pgfqpoint{0.733531in}{2.301955in}}%
\pgfusepath{stroke}%
\end{pgfscope}%
\begin{pgfscope}%
\pgfsetbuttcap%
\pgfsetroundjoin%
\definecolor{currentfill}{rgb}{0.000000,0.000000,0.000000}%
\pgfsetfillcolor{currentfill}%
\pgfsetlinewidth{0.803000pt}%
\definecolor{currentstroke}{rgb}{0.000000,0.000000,0.000000}%
\pgfsetstrokecolor{currentstroke}%
\pgfsetdash{}{0pt}%
\pgfsys@defobject{currentmarker}{\pgfqpoint{0.000000in}{-0.048611in}}{\pgfqpoint{0.000000in}{0.000000in}}{%
\pgfpathmoveto{\pgfqpoint{0.000000in}{0.000000in}}%
\pgfpathlineto{\pgfqpoint{0.000000in}{-0.048611in}}%
\pgfusepath{stroke,fill}%
}%
\begin{pgfscope}%
\pgfsys@transformshift{0.733531in}{0.548769in}%
\pgfsys@useobject{currentmarker}{}%
\end{pgfscope}%
\end{pgfscope}%
\begin{pgfscope}%
\definecolor{textcolor}{rgb}{0.000000,0.000000,0.000000}%
\pgfsetstrokecolor{textcolor}%
\pgfsetfillcolor{textcolor}%
\pgftext[x=0.733531in,y=0.451547in,,top]{\color{textcolor}\rmfamily\fontsize{10.000000}{12.000000}\selectfont \(\displaystyle {0.0}\)}%
\end{pgfscope}%
\begin{pgfscope}%
\pgfpathrectangle{\pgfqpoint{0.733531in}{0.548769in}}{\pgfqpoint{3.028066in}{1.753186in}}%
\pgfusepath{clip}%
\pgfsetrectcap%
\pgfsetroundjoin%
\pgfsetlinewidth{0.803000pt}%
\definecolor{currentstroke}{rgb}{0.690196,0.690196,0.690196}%
\pgfsetstrokecolor{currentstroke}%
\pgfsetdash{}{0pt}%
\pgfpathmoveto{\pgfqpoint{1.490547in}{0.548769in}}%
\pgfpathlineto{\pgfqpoint{1.490547in}{2.301955in}}%
\pgfusepath{stroke}%
\end{pgfscope}%
\begin{pgfscope}%
\pgfsetbuttcap%
\pgfsetroundjoin%
\definecolor{currentfill}{rgb}{0.000000,0.000000,0.000000}%
\pgfsetfillcolor{currentfill}%
\pgfsetlinewidth{0.803000pt}%
\definecolor{currentstroke}{rgb}{0.000000,0.000000,0.000000}%
\pgfsetstrokecolor{currentstroke}%
\pgfsetdash{}{0pt}%
\pgfsys@defobject{currentmarker}{\pgfqpoint{0.000000in}{-0.048611in}}{\pgfqpoint{0.000000in}{0.000000in}}{%
\pgfpathmoveto{\pgfqpoint{0.000000in}{0.000000in}}%
\pgfpathlineto{\pgfqpoint{0.000000in}{-0.048611in}}%
\pgfusepath{stroke,fill}%
}%
\begin{pgfscope}%
\pgfsys@transformshift{1.490547in}{0.548769in}%
\pgfsys@useobject{currentmarker}{}%
\end{pgfscope}%
\end{pgfscope}%
\begin{pgfscope}%
\definecolor{textcolor}{rgb}{0.000000,0.000000,0.000000}%
\pgfsetstrokecolor{textcolor}%
\pgfsetfillcolor{textcolor}%
\pgftext[x=1.490547in,y=0.451547in,,top]{\color{textcolor}\rmfamily\fontsize{10.000000}{12.000000}\selectfont \(\displaystyle {0.5}\)}%
\end{pgfscope}%
\begin{pgfscope}%
\pgfpathrectangle{\pgfqpoint{0.733531in}{0.548769in}}{\pgfqpoint{3.028066in}{1.753186in}}%
\pgfusepath{clip}%
\pgfsetrectcap%
\pgfsetroundjoin%
\pgfsetlinewidth{0.803000pt}%
\definecolor{currentstroke}{rgb}{0.690196,0.690196,0.690196}%
\pgfsetstrokecolor{currentstroke}%
\pgfsetdash{}{0pt}%
\pgfpathmoveto{\pgfqpoint{2.247564in}{0.548769in}}%
\pgfpathlineto{\pgfqpoint{2.247564in}{2.301955in}}%
\pgfusepath{stroke}%
\end{pgfscope}%
\begin{pgfscope}%
\pgfsetbuttcap%
\pgfsetroundjoin%
\definecolor{currentfill}{rgb}{0.000000,0.000000,0.000000}%
\pgfsetfillcolor{currentfill}%
\pgfsetlinewidth{0.803000pt}%
\definecolor{currentstroke}{rgb}{0.000000,0.000000,0.000000}%
\pgfsetstrokecolor{currentstroke}%
\pgfsetdash{}{0pt}%
\pgfsys@defobject{currentmarker}{\pgfqpoint{0.000000in}{-0.048611in}}{\pgfqpoint{0.000000in}{0.000000in}}{%
\pgfpathmoveto{\pgfqpoint{0.000000in}{0.000000in}}%
\pgfpathlineto{\pgfqpoint{0.000000in}{-0.048611in}}%
\pgfusepath{stroke,fill}%
}%
\begin{pgfscope}%
\pgfsys@transformshift{2.247564in}{0.548769in}%
\pgfsys@useobject{currentmarker}{}%
\end{pgfscope}%
\end{pgfscope}%
\begin{pgfscope}%
\definecolor{textcolor}{rgb}{0.000000,0.000000,0.000000}%
\pgfsetstrokecolor{textcolor}%
\pgfsetfillcolor{textcolor}%
\pgftext[x=2.247564in,y=0.451547in,,top]{\color{textcolor}\rmfamily\fontsize{10.000000}{12.000000}\selectfont \(\displaystyle {1.0}\)}%
\end{pgfscope}%
\begin{pgfscope}%
\pgfpathrectangle{\pgfqpoint{0.733531in}{0.548769in}}{\pgfqpoint{3.028066in}{1.753186in}}%
\pgfusepath{clip}%
\pgfsetrectcap%
\pgfsetroundjoin%
\pgfsetlinewidth{0.803000pt}%
\definecolor{currentstroke}{rgb}{0.690196,0.690196,0.690196}%
\pgfsetstrokecolor{currentstroke}%
\pgfsetdash{}{0pt}%
\pgfpathmoveto{\pgfqpoint{3.004580in}{0.548769in}}%
\pgfpathlineto{\pgfqpoint{3.004580in}{2.301955in}}%
\pgfusepath{stroke}%
\end{pgfscope}%
\begin{pgfscope}%
\pgfsetbuttcap%
\pgfsetroundjoin%
\definecolor{currentfill}{rgb}{0.000000,0.000000,0.000000}%
\pgfsetfillcolor{currentfill}%
\pgfsetlinewidth{0.803000pt}%
\definecolor{currentstroke}{rgb}{0.000000,0.000000,0.000000}%
\pgfsetstrokecolor{currentstroke}%
\pgfsetdash{}{0pt}%
\pgfsys@defobject{currentmarker}{\pgfqpoint{0.000000in}{-0.048611in}}{\pgfqpoint{0.000000in}{0.000000in}}{%
\pgfpathmoveto{\pgfqpoint{0.000000in}{0.000000in}}%
\pgfpathlineto{\pgfqpoint{0.000000in}{-0.048611in}}%
\pgfusepath{stroke,fill}%
}%
\begin{pgfscope}%
\pgfsys@transformshift{3.004580in}{0.548769in}%
\pgfsys@useobject{currentmarker}{}%
\end{pgfscope}%
\end{pgfscope}%
\begin{pgfscope}%
\definecolor{textcolor}{rgb}{0.000000,0.000000,0.000000}%
\pgfsetstrokecolor{textcolor}%
\pgfsetfillcolor{textcolor}%
\pgftext[x=3.004580in,y=0.451547in,,top]{\color{textcolor}\rmfamily\fontsize{10.000000}{12.000000}\selectfont \(\displaystyle {1.5}\)}%
\end{pgfscope}%
\begin{pgfscope}%
\pgfpathrectangle{\pgfqpoint{0.733531in}{0.548769in}}{\pgfqpoint{3.028066in}{1.753186in}}%
\pgfusepath{clip}%
\pgfsetrectcap%
\pgfsetroundjoin%
\pgfsetlinewidth{0.803000pt}%
\definecolor{currentstroke}{rgb}{0.690196,0.690196,0.690196}%
\pgfsetstrokecolor{currentstroke}%
\pgfsetdash{}{0pt}%
\pgfpathmoveto{\pgfqpoint{3.761597in}{0.548769in}}%
\pgfpathlineto{\pgfqpoint{3.761597in}{2.301955in}}%
\pgfusepath{stroke}%
\end{pgfscope}%
\begin{pgfscope}%
\pgfsetbuttcap%
\pgfsetroundjoin%
\definecolor{currentfill}{rgb}{0.000000,0.000000,0.000000}%
\pgfsetfillcolor{currentfill}%
\pgfsetlinewidth{0.803000pt}%
\definecolor{currentstroke}{rgb}{0.000000,0.000000,0.000000}%
\pgfsetstrokecolor{currentstroke}%
\pgfsetdash{}{0pt}%
\pgfsys@defobject{currentmarker}{\pgfqpoint{0.000000in}{-0.048611in}}{\pgfqpoint{0.000000in}{0.000000in}}{%
\pgfpathmoveto{\pgfqpoint{0.000000in}{0.000000in}}%
\pgfpathlineto{\pgfqpoint{0.000000in}{-0.048611in}}%
\pgfusepath{stroke,fill}%
}%
\begin{pgfscope}%
\pgfsys@transformshift{3.761597in}{0.548769in}%
\pgfsys@useobject{currentmarker}{}%
\end{pgfscope}%
\end{pgfscope}%
\begin{pgfscope}%
\definecolor{textcolor}{rgb}{0.000000,0.000000,0.000000}%
\pgfsetstrokecolor{textcolor}%
\pgfsetfillcolor{textcolor}%
\pgftext[x=3.761597in,y=0.451547in,,top]{\color{textcolor}\rmfamily\fontsize{10.000000}{12.000000}\selectfont \(\displaystyle {2.0}\)}%
\end{pgfscope}%
\begin{pgfscope}%
\definecolor{textcolor}{rgb}{0.000000,0.000000,0.000000}%
\pgfsetstrokecolor{textcolor}%
\pgfsetfillcolor{textcolor}%
\pgftext[x=2.247564in,y=0.272534in,,top]{\color{textcolor}\rmfamily\fontsize{10.000000}{12.000000}\selectfont \(\displaystyle w\)}%
\end{pgfscope}%
\begin{pgfscope}%
\pgfpathrectangle{\pgfqpoint{0.733531in}{0.548769in}}{\pgfqpoint{3.028066in}{1.753186in}}%
\pgfusepath{clip}%
\pgfsetrectcap%
\pgfsetroundjoin%
\pgfsetlinewidth{0.803000pt}%
\definecolor{currentstroke}{rgb}{0.690196,0.690196,0.690196}%
\pgfsetstrokecolor{currentstroke}%
\pgfsetdash{}{0pt}%
\pgfpathmoveto{\pgfqpoint{0.733531in}{0.548769in}}%
\pgfpathlineto{\pgfqpoint{3.761597in}{0.548769in}}%
\pgfusepath{stroke}%
\end{pgfscope}%
\begin{pgfscope}%
\pgfsetbuttcap%
\pgfsetroundjoin%
\definecolor{currentfill}{rgb}{0.000000,0.000000,0.000000}%
\pgfsetfillcolor{currentfill}%
\pgfsetlinewidth{0.803000pt}%
\definecolor{currentstroke}{rgb}{0.000000,0.000000,0.000000}%
\pgfsetstrokecolor{currentstroke}%
\pgfsetdash{}{0pt}%
\pgfsys@defobject{currentmarker}{\pgfqpoint{-0.048611in}{0.000000in}}{\pgfqpoint{-0.000000in}{0.000000in}}{%
\pgfpathmoveto{\pgfqpoint{-0.000000in}{0.000000in}}%
\pgfpathlineto{\pgfqpoint{-0.048611in}{0.000000in}}%
\pgfusepath{stroke,fill}%
}%
\begin{pgfscope}%
\pgfsys@transformshift{0.733531in}{0.548769in}%
\pgfsys@useobject{currentmarker}{}%
\end{pgfscope}%
\end{pgfscope}%
\begin{pgfscope}%
\definecolor{textcolor}{rgb}{0.000000,0.000000,0.000000}%
\pgfsetstrokecolor{textcolor}%
\pgfsetfillcolor{textcolor}%
\pgftext[x=0.348306in, y=0.500544in, left, base]{\color{textcolor}\rmfamily\fontsize{10.000000}{12.000000}\selectfont \(\displaystyle {10^{-4}}\)}%
\end{pgfscope}%
\begin{pgfscope}%
\pgfpathrectangle{\pgfqpoint{0.733531in}{0.548769in}}{\pgfqpoint{3.028066in}{1.753186in}}%
\pgfusepath{clip}%
\pgfsetrectcap%
\pgfsetroundjoin%
\pgfsetlinewidth{0.803000pt}%
\definecolor{currentstroke}{rgb}{0.690196,0.690196,0.690196}%
\pgfsetstrokecolor{currentstroke}%
\pgfsetdash{}{0pt}%
\pgfpathmoveto{\pgfqpoint{0.733531in}{0.899406in}}%
\pgfpathlineto{\pgfqpoint{3.761597in}{0.899406in}}%
\pgfusepath{stroke}%
\end{pgfscope}%
\begin{pgfscope}%
\pgfsetbuttcap%
\pgfsetroundjoin%
\definecolor{currentfill}{rgb}{0.000000,0.000000,0.000000}%
\pgfsetfillcolor{currentfill}%
\pgfsetlinewidth{0.803000pt}%
\definecolor{currentstroke}{rgb}{0.000000,0.000000,0.000000}%
\pgfsetstrokecolor{currentstroke}%
\pgfsetdash{}{0pt}%
\pgfsys@defobject{currentmarker}{\pgfqpoint{-0.048611in}{0.000000in}}{\pgfqpoint{-0.000000in}{0.000000in}}{%
\pgfpathmoveto{\pgfqpoint{-0.000000in}{0.000000in}}%
\pgfpathlineto{\pgfqpoint{-0.048611in}{0.000000in}}%
\pgfusepath{stroke,fill}%
}%
\begin{pgfscope}%
\pgfsys@transformshift{0.733531in}{0.899406in}%
\pgfsys@useobject{currentmarker}{}%
\end{pgfscope}%
\end{pgfscope}%
\begin{pgfscope}%
\definecolor{textcolor}{rgb}{0.000000,0.000000,0.000000}%
\pgfsetstrokecolor{textcolor}%
\pgfsetfillcolor{textcolor}%
\pgftext[x=0.348306in, y=0.851181in, left, base]{\color{textcolor}\rmfamily\fontsize{10.000000}{12.000000}\selectfont \(\displaystyle {10^{-2}}\)}%
\end{pgfscope}%
\begin{pgfscope}%
\pgfpathrectangle{\pgfqpoint{0.733531in}{0.548769in}}{\pgfqpoint{3.028066in}{1.753186in}}%
\pgfusepath{clip}%
\pgfsetrectcap%
\pgfsetroundjoin%
\pgfsetlinewidth{0.803000pt}%
\definecolor{currentstroke}{rgb}{0.690196,0.690196,0.690196}%
\pgfsetstrokecolor{currentstroke}%
\pgfsetdash{}{0pt}%
\pgfpathmoveto{\pgfqpoint{0.733531in}{1.250043in}}%
\pgfpathlineto{\pgfqpoint{3.761597in}{1.250043in}}%
\pgfusepath{stroke}%
\end{pgfscope}%
\begin{pgfscope}%
\pgfsetbuttcap%
\pgfsetroundjoin%
\definecolor{currentfill}{rgb}{0.000000,0.000000,0.000000}%
\pgfsetfillcolor{currentfill}%
\pgfsetlinewidth{0.803000pt}%
\definecolor{currentstroke}{rgb}{0.000000,0.000000,0.000000}%
\pgfsetstrokecolor{currentstroke}%
\pgfsetdash{}{0pt}%
\pgfsys@defobject{currentmarker}{\pgfqpoint{-0.048611in}{0.000000in}}{\pgfqpoint{-0.000000in}{0.000000in}}{%
\pgfpathmoveto{\pgfqpoint{-0.000000in}{0.000000in}}%
\pgfpathlineto{\pgfqpoint{-0.048611in}{0.000000in}}%
\pgfusepath{stroke,fill}%
}%
\begin{pgfscope}%
\pgfsys@transformshift{0.733531in}{1.250043in}%
\pgfsys@useobject{currentmarker}{}%
\end{pgfscope}%
\end{pgfscope}%
\begin{pgfscope}%
\definecolor{textcolor}{rgb}{0.000000,0.000000,0.000000}%
\pgfsetstrokecolor{textcolor}%
\pgfsetfillcolor{textcolor}%
\pgftext[x=0.435112in, y=1.201818in, left, base]{\color{textcolor}\rmfamily\fontsize{10.000000}{12.000000}\selectfont \(\displaystyle {10^{0}}\)}%
\end{pgfscope}%
\begin{pgfscope}%
\pgfpathrectangle{\pgfqpoint{0.733531in}{0.548769in}}{\pgfqpoint{3.028066in}{1.753186in}}%
\pgfusepath{clip}%
\pgfsetrectcap%
\pgfsetroundjoin%
\pgfsetlinewidth{0.803000pt}%
\definecolor{currentstroke}{rgb}{0.690196,0.690196,0.690196}%
\pgfsetstrokecolor{currentstroke}%
\pgfsetdash{}{0pt}%
\pgfpathmoveto{\pgfqpoint{0.733531in}{1.600680in}}%
\pgfpathlineto{\pgfqpoint{3.761597in}{1.600680in}}%
\pgfusepath{stroke}%
\end{pgfscope}%
\begin{pgfscope}%
\pgfsetbuttcap%
\pgfsetroundjoin%
\definecolor{currentfill}{rgb}{0.000000,0.000000,0.000000}%
\pgfsetfillcolor{currentfill}%
\pgfsetlinewidth{0.803000pt}%
\definecolor{currentstroke}{rgb}{0.000000,0.000000,0.000000}%
\pgfsetstrokecolor{currentstroke}%
\pgfsetdash{}{0pt}%
\pgfsys@defobject{currentmarker}{\pgfqpoint{-0.048611in}{0.000000in}}{\pgfqpoint{-0.000000in}{0.000000in}}{%
\pgfpathmoveto{\pgfqpoint{-0.000000in}{0.000000in}}%
\pgfpathlineto{\pgfqpoint{-0.048611in}{0.000000in}}%
\pgfusepath{stroke,fill}%
}%
\begin{pgfscope}%
\pgfsys@transformshift{0.733531in}{1.600680in}%
\pgfsys@useobject{currentmarker}{}%
\end{pgfscope}%
\end{pgfscope}%
\begin{pgfscope}%
\definecolor{textcolor}{rgb}{0.000000,0.000000,0.000000}%
\pgfsetstrokecolor{textcolor}%
\pgfsetfillcolor{textcolor}%
\pgftext[x=0.435112in, y=1.552455in, left, base]{\color{textcolor}\rmfamily\fontsize{10.000000}{12.000000}\selectfont \(\displaystyle {10^{2}}\)}%
\end{pgfscope}%
\begin{pgfscope}%
\pgfpathrectangle{\pgfqpoint{0.733531in}{0.548769in}}{\pgfqpoint{3.028066in}{1.753186in}}%
\pgfusepath{clip}%
\pgfsetrectcap%
\pgfsetroundjoin%
\pgfsetlinewidth{0.803000pt}%
\definecolor{currentstroke}{rgb}{0.690196,0.690196,0.690196}%
\pgfsetstrokecolor{currentstroke}%
\pgfsetdash{}{0pt}%
\pgfpathmoveto{\pgfqpoint{0.733531in}{1.951318in}}%
\pgfpathlineto{\pgfqpoint{3.761597in}{1.951318in}}%
\pgfusepath{stroke}%
\end{pgfscope}%
\begin{pgfscope}%
\pgfsetbuttcap%
\pgfsetroundjoin%
\definecolor{currentfill}{rgb}{0.000000,0.000000,0.000000}%
\pgfsetfillcolor{currentfill}%
\pgfsetlinewidth{0.803000pt}%
\definecolor{currentstroke}{rgb}{0.000000,0.000000,0.000000}%
\pgfsetstrokecolor{currentstroke}%
\pgfsetdash{}{0pt}%
\pgfsys@defobject{currentmarker}{\pgfqpoint{-0.048611in}{0.000000in}}{\pgfqpoint{-0.000000in}{0.000000in}}{%
\pgfpathmoveto{\pgfqpoint{-0.000000in}{0.000000in}}%
\pgfpathlineto{\pgfqpoint{-0.048611in}{0.000000in}}%
\pgfusepath{stroke,fill}%
}%
\begin{pgfscope}%
\pgfsys@transformshift{0.733531in}{1.951318in}%
\pgfsys@useobject{currentmarker}{}%
\end{pgfscope}%
\end{pgfscope}%
\begin{pgfscope}%
\definecolor{textcolor}{rgb}{0.000000,0.000000,0.000000}%
\pgfsetstrokecolor{textcolor}%
\pgfsetfillcolor{textcolor}%
\pgftext[x=0.435112in, y=1.903092in, left, base]{\color{textcolor}\rmfamily\fontsize{10.000000}{12.000000}\selectfont \(\displaystyle {10^{4}}\)}%
\end{pgfscope}%
\begin{pgfscope}%
\pgfpathrectangle{\pgfqpoint{0.733531in}{0.548769in}}{\pgfqpoint{3.028066in}{1.753186in}}%
\pgfusepath{clip}%
\pgfsetrectcap%
\pgfsetroundjoin%
\pgfsetlinewidth{0.803000pt}%
\definecolor{currentstroke}{rgb}{0.690196,0.690196,0.690196}%
\pgfsetstrokecolor{currentstroke}%
\pgfsetdash{}{0pt}%
\pgfpathmoveto{\pgfqpoint{0.733531in}{2.301955in}}%
\pgfpathlineto{\pgfqpoint{3.761597in}{2.301955in}}%
\pgfusepath{stroke}%
\end{pgfscope}%
\begin{pgfscope}%
\pgfsetbuttcap%
\pgfsetroundjoin%
\definecolor{currentfill}{rgb}{0.000000,0.000000,0.000000}%
\pgfsetfillcolor{currentfill}%
\pgfsetlinewidth{0.803000pt}%
\definecolor{currentstroke}{rgb}{0.000000,0.000000,0.000000}%
\pgfsetstrokecolor{currentstroke}%
\pgfsetdash{}{0pt}%
\pgfsys@defobject{currentmarker}{\pgfqpoint{-0.048611in}{0.000000in}}{\pgfqpoint{-0.000000in}{0.000000in}}{%
\pgfpathmoveto{\pgfqpoint{-0.000000in}{0.000000in}}%
\pgfpathlineto{\pgfqpoint{-0.048611in}{0.000000in}}%
\pgfusepath{stroke,fill}%
}%
\begin{pgfscope}%
\pgfsys@transformshift{0.733531in}{2.301955in}%
\pgfsys@useobject{currentmarker}{}%
\end{pgfscope}%
\end{pgfscope}%
\begin{pgfscope}%
\definecolor{textcolor}{rgb}{0.000000,0.000000,0.000000}%
\pgfsetstrokecolor{textcolor}%
\pgfsetfillcolor{textcolor}%
\pgftext[x=0.435112in, y=2.253730in, left, base]{\color{textcolor}\rmfamily\fontsize{10.000000}{12.000000}\selectfont \(\displaystyle {10^{6}}\)}%
\end{pgfscope}%
\begin{pgfscope}%
\definecolor{textcolor}{rgb}{0.000000,0.000000,0.000000}%
\pgfsetstrokecolor{textcolor}%
\pgfsetfillcolor{textcolor}%
\pgftext[x=0.292751in,y=1.425362in,,bottom,rotate=90.000000]{\color{textcolor}\rmfamily\fontsize{10.000000}{12.000000}\selectfont \(\displaystyle F^2_N(w)\)}%
\end{pgfscope}%
\begin{pgfscope}%
\pgfpathrectangle{\pgfqpoint{0.733531in}{0.548769in}}{\pgfqpoint{3.028066in}{1.753186in}}%
\pgfusepath{clip}%
\pgfsetrectcap%
\pgfsetroundjoin%
\pgfsetlinewidth{1.003750pt}%
\definecolor{currentstroke}{rgb}{0.121569,0.466667,0.705882}%
\pgfsetstrokecolor{currentstroke}%
\pgfsetdash{}{0pt}%
\pgfpathmoveto{\pgfqpoint{0.739446in}{0.534880in}}%
\pgfpathlineto{\pgfqpoint{0.744132in}{0.623916in}}%
\pgfpathlineto{\pgfqpoint{0.750947in}{0.699506in}}%
\pgfpathlineto{\pgfqpoint{0.759276in}{0.759013in}}%
\pgfpathlineto{\pgfqpoint{0.769120in}{0.808295in}}%
\pgfpathlineto{\pgfqpoint{0.781235in}{0.852871in}}%
\pgfpathlineto{\pgfqpoint{0.794865in}{0.891083in}}%
\pgfpathlineto{\pgfqpoint{0.810009in}{0.924604in}}%
\pgfpathlineto{\pgfqpoint{0.827425in}{0.955729in}}%
\pgfpathlineto{\pgfqpoint{0.847112in}{0.984554in}}%
\pgfpathlineto{\pgfqpoint{0.869071in}{1.011252in}}%
\pgfpathlineto{\pgfqpoint{0.894059in}{1.036721in}}%
\pgfpathlineto{\pgfqpoint{0.922075in}{1.060823in}}%
\pgfpathlineto{\pgfqpoint{0.953878in}{1.084028in}}%
\pgfpathlineto{\pgfqpoint{0.989467in}{1.106127in}}%
\pgfpathlineto{\pgfqpoint{1.029598in}{1.127375in}}%
\pgfpathlineto{\pgfqpoint{1.075031in}{1.147865in}}%
\pgfpathlineto{\pgfqpoint{1.125764in}{1.167300in}}%
\pgfpathlineto{\pgfqpoint{1.182554in}{1.185675in}}%
\pgfpathlineto{\pgfqpoint{1.244645in}{1.202480in}}%
\pgfpathlineto{\pgfqpoint{1.312036in}{1.217494in}}%
\pgfpathlineto{\pgfqpoint{1.383214in}{1.230171in}}%
\pgfpathlineto{\pgfqpoint{1.455905in}{1.239991in}}%
\pgfpathlineto{\pgfqpoint{1.527083in}{1.246540in}}%
\pgfpathlineto{\pgfqpoint{1.594474in}{1.249707in}}%
\pgfpathlineto{\pgfqpoint{1.655808in}{1.249589in}}%
\pgfpathlineto{\pgfqpoint{1.711084in}{1.246442in}}%
\pgfpathlineto{\pgfqpoint{1.758788in}{1.240733in}}%
\pgfpathlineto{\pgfqpoint{1.800434in}{1.232740in}}%
\pgfpathlineto{\pgfqpoint{1.836780in}{1.222684in}}%
\pgfpathlineto{\pgfqpoint{1.867825in}{1.211013in}}%
\pgfpathlineto{\pgfqpoint{1.895085in}{1.197575in}}%
\pgfpathlineto{\pgfqpoint{1.919315in}{1.182199in}}%
\pgfpathlineto{\pgfqpoint{1.940517in}{1.165082in}}%
\pgfpathlineto{\pgfqpoint{1.959447in}{1.145758in}}%
\pgfpathlineto{\pgfqpoint{1.976106in}{1.124277in}}%
\pgfpathlineto{\pgfqpoint{1.991250in}{1.099472in}}%
\pgfpathlineto{\pgfqpoint{2.004122in}{1.072523in}}%
\pgfpathlineto{\pgfqpoint{2.015480in}{1.041896in}}%
\pgfpathlineto{\pgfqpoint{2.026081in}{1.004016in}}%
\pgfpathlineto{\pgfqpoint{2.035168in}{0.959254in}}%
\pgfpathlineto{\pgfqpoint{2.042740in}{0.905583in}}%
\pgfpathlineto{\pgfqpoint{2.048797in}{0.840043in}}%
\pgfpathlineto{\pgfqpoint{2.053341in}{0.758643in}}%
\pgfpathlineto{\pgfqpoint{2.056369in}{0.659102in}}%
\pgfpathlineto{\pgfqpoint{2.058129in}{0.534880in}}%
\pgfpathmoveto{\pgfqpoint{2.061041in}{0.534880in}}%
\pgfpathlineto{\pgfqpoint{2.064699in}{0.731366in}}%
\pgfpathlineto{\pgfqpoint{2.069999in}{0.841854in}}%
\pgfpathlineto{\pgfqpoint{2.076814in}{0.921040in}}%
\pgfpathlineto{\pgfqpoint{2.085143in}{0.984050in}}%
\pgfpathlineto{\pgfqpoint{2.095744in}{1.040507in}}%
\pgfpathlineto{\pgfqpoint{2.107859in}{1.088435in}}%
\pgfpathlineto{\pgfqpoint{2.121489in}{1.130355in}}%
\pgfpathlineto{\pgfqpoint{2.136633in}{1.167522in}}%
\pgfpathlineto{\pgfqpoint{2.153292in}{1.200289in}}%
\pgfpathlineto{\pgfqpoint{2.169193in}{1.224889in}}%
\pgfpathlineto{\pgfqpoint{2.182823in}{1.240496in}}%
\pgfpathlineto{\pgfqpoint{2.192666in}{1.247725in}}%
\pgfpathlineto{\pgfqpoint{2.200239in}{1.250017in}}%
\pgfpathlineto{\pgfqpoint{2.206296in}{1.248902in}}%
\pgfpathlineto{\pgfqpoint{2.211597in}{1.244804in}}%
\pgfpathlineto{\pgfqpoint{2.216897in}{1.236352in}}%
\pgfpathlineto{\pgfqpoint{2.222197in}{1.220917in}}%
\pgfpathlineto{\pgfqpoint{2.226741in}{1.197982in}}%
\pgfpathlineto{\pgfqpoint{2.231284in}{1.157051in}}%
\pgfpathlineto{\pgfqpoint{2.235070in}{1.089329in}}%
\pgfpathlineto{\pgfqpoint{2.237342in}{1.003949in}}%
\pgfpathlineto{\pgfqpoint{2.238856in}{0.869518in}}%
\pgfpathlineto{\pgfqpoint{2.239613in}{0.638914in}}%
\pgfpathlineto{\pgfqpoint{2.240370in}{0.794881in}}%
\pgfpathlineto{\pgfqpoint{2.243399in}{1.100517in}}%
\pgfpathlineto{\pgfqpoint{2.248700in}{1.280424in}}%
\pgfpathlineto{\pgfqpoint{2.266873in}{1.753784in}}%
\pgfpathlineto{\pgfqpoint{2.269144in}{1.924021in}}%
\pgfpathlineto{\pgfqpoint{2.270659in}{2.202839in}}%
\pgfpathlineto{\pgfqpoint{2.272930in}{1.848446in}}%
\pgfpathlineto{\pgfqpoint{2.276716in}{1.730165in}}%
\pgfpathlineto{\pgfqpoint{2.281260in}{1.672036in}}%
\pgfpathlineto{\pgfqpoint{2.286560in}{1.637950in}}%
\pgfpathlineto{\pgfqpoint{2.292618in}{1.617444in}}%
\pgfpathlineto{\pgfqpoint{2.298675in}{1.606779in}}%
\pgfpathlineto{\pgfqpoint{2.304733in}{1.601737in}}%
\pgfpathlineto{\pgfqpoint{2.311548in}{1.600286in}}%
\pgfpathlineto{\pgfqpoint{2.319120in}{1.602150in}}%
\pgfpathlineto{\pgfqpoint{2.328206in}{1.607676in}}%
\pgfpathlineto{\pgfqpoint{2.340322in}{1.618928in}}%
\pgfpathlineto{\pgfqpoint{2.355466in}{1.637536in}}%
\pgfpathlineto{\pgfqpoint{2.372881in}{1.664058in}}%
\pgfpathlineto{\pgfqpoint{2.391054in}{1.697587in}}%
\pgfpathlineto{\pgfqpoint{2.407713in}{1.734758in}}%
\pgfpathlineto{\pgfqpoint{2.422857in}{1.776122in}}%
\pgfpathlineto{\pgfqpoint{2.435729in}{1.820082in}}%
\pgfpathlineto{\pgfqpoint{2.447088in}{1.870149in}}%
\pgfpathlineto{\pgfqpoint{2.456174in}{1.923894in}}%
\pgfpathlineto{\pgfqpoint{2.463746in}{1.987030in}}%
\pgfpathlineto{\pgfqpoint{2.469804in}{2.064340in}}%
\pgfpathlineto{\pgfqpoint{2.474347in}{2.165039in}}%
\pgfpathlineto{\pgfqpoint{2.477435in}{2.315844in}}%
\pgfpathmoveto{\pgfqpoint{2.481180in}{2.315844in}}%
\pgfpathlineto{\pgfqpoint{2.484948in}{2.149178in}}%
\pgfpathlineto{\pgfqpoint{2.490248in}{2.050240in}}%
\pgfpathlineto{\pgfqpoint{2.497063in}{1.978983in}}%
\pgfpathlineto{\pgfqpoint{2.505392in}{1.923413in}}%
\pgfpathlineto{\pgfqpoint{2.515236in}{1.878185in}}%
\pgfpathlineto{\pgfqpoint{2.526594in}{1.840393in}}%
\pgfpathlineto{\pgfqpoint{2.539467in}{1.808260in}}%
\pgfpathlineto{\pgfqpoint{2.553854in}{1.780613in}}%
\pgfpathlineto{\pgfqpoint{2.569755in}{1.756622in}}%
\pgfpathlineto{\pgfqpoint{2.587928in}{1.734871in}}%
\pgfpathlineto{\pgfqpoint{2.608372in}{1.715370in}}%
\pgfpathlineto{\pgfqpoint{2.631089in}{1.698028in}}%
\pgfpathlineto{\pgfqpoint{2.656834in}{1.682284in}}%
\pgfpathlineto{\pgfqpoint{2.686365in}{1.667895in}}%
\pgfpathlineto{\pgfqpoint{2.720439in}{1.654789in}}%
\pgfpathlineto{\pgfqpoint{2.759814in}{1.642992in}}%
\pgfpathlineto{\pgfqpoint{2.806760in}{1.632261in}}%
\pgfpathlineto{\pgfqpoint{2.862036in}{1.622901in}}%
\pgfpathlineto{\pgfqpoint{2.928670in}{1.614877in}}%
\pgfpathlineto{\pgfqpoint{3.008934in}{1.608422in}}%
\pgfpathlineto{\pgfqpoint{3.108128in}{1.603650in}}%
\pgfpathlineto{\pgfqpoint{3.233824in}{1.600841in}}%
\pgfpathlineto{\pgfqpoint{3.396624in}{1.600449in}}%
\pgfpathlineto{\pgfqpoint{3.619242in}{1.603198in}}%
\pgfpathlineto{\pgfqpoint{3.761597in}{1.606074in}}%
\pgfpathlineto{\pgfqpoint{3.761597in}{1.606074in}}%
\pgfusepath{stroke}%
\end{pgfscope}%
\begin{pgfscope}%
\pgfpathrectangle{\pgfqpoint{0.733531in}{0.548769in}}{\pgfqpoint{3.028066in}{1.753186in}}%
\pgfusepath{clip}%
\pgfsetbuttcap%
\pgfsetroundjoin%
\definecolor{currentfill}{rgb}{0.000000,0.000000,0.000000}%
\pgfsetfillcolor{currentfill}%
\pgfsetfillopacity{0.000000}%
\pgfsetlinewidth{1.003750pt}%
\definecolor{currentstroke}{rgb}{0.000000,0.000000,0.000000}%
\pgfsetstrokecolor{currentstroke}%
\pgfsetdash{}{0pt}%
\pgfsys@defobject{currentmarker}{\pgfqpoint{-0.041667in}{-0.041667in}}{\pgfqpoint{0.041667in}{0.041667in}}{%
\pgfpathmoveto{\pgfqpoint{0.000000in}{-0.041667in}}%
\pgfpathcurveto{\pgfqpoint{0.011050in}{-0.041667in}}{\pgfqpoint{0.021649in}{-0.037276in}}{\pgfqpoint{0.029463in}{-0.029463in}}%
\pgfpathcurveto{\pgfqpoint{0.037276in}{-0.021649in}}{\pgfqpoint{0.041667in}{-0.011050in}}{\pgfqpoint{0.041667in}{0.000000in}}%
\pgfpathcurveto{\pgfqpoint{0.041667in}{0.011050in}}{\pgfqpoint{0.037276in}{0.021649in}}{\pgfqpoint{0.029463in}{0.029463in}}%
\pgfpathcurveto{\pgfqpoint{0.021649in}{0.037276in}}{\pgfqpoint{0.011050in}{0.041667in}}{\pgfqpoint{0.000000in}{0.041667in}}%
\pgfpathcurveto{\pgfqpoint{-0.011050in}{0.041667in}}{\pgfqpoint{-0.021649in}{0.037276in}}{\pgfqpoint{-0.029463in}{0.029463in}}%
\pgfpathcurveto{\pgfqpoint{-0.037276in}{0.021649in}}{\pgfqpoint{-0.041667in}{0.011050in}}{\pgfqpoint{-0.041667in}{0.000000in}}%
\pgfpathcurveto{\pgfqpoint{-0.041667in}{-0.011050in}}{\pgfqpoint{-0.037276in}{-0.021649in}}{\pgfqpoint{-0.029463in}{-0.029463in}}%
\pgfpathcurveto{\pgfqpoint{-0.021649in}{-0.037276in}}{\pgfqpoint{-0.011050in}{-0.041667in}}{\pgfqpoint{0.000000in}{-0.041667in}}%
\pgfpathlineto{\pgfqpoint{0.000000in}{-0.041667in}}%
\pgfpathclose%
\pgfusepath{stroke,fill}%
}%
\begin{pgfscope}%
\pgfsys@transformshift{0.733531in}{0.548769in}%
\pgfsys@useobject{currentmarker}{}%
\end{pgfscope}%
\begin{pgfscope}%
\pgfsys@transformshift{2.050740in}{0.548769in}%
\pgfsys@useobject{currentmarker}{}%
\end{pgfscope}%
\begin{pgfscope}%
\pgfsys@transformshift{2.247564in}{0.548769in}%
\pgfsys@useobject{currentmarker}{}%
\end{pgfscope}%
\end{pgfscope}%
\begin{pgfscope}%
\pgfpathrectangle{\pgfqpoint{0.733531in}{0.548769in}}{\pgfqpoint{3.028066in}{1.753186in}}%
\pgfusepath{clip}%
\pgfsetbuttcap%
\pgfsetroundjoin%
\definecolor{currentfill}{rgb}{0.000000,0.000000,0.000000}%
\pgfsetfillcolor{currentfill}%
\pgfsetfillopacity{0.000000}%
\pgfsetlinewidth{1.003750pt}%
\definecolor{currentstroke}{rgb}{0.000000,0.000000,0.000000}%
\pgfsetstrokecolor{currentstroke}%
\pgfsetdash{}{0pt}%
\pgfsys@defobject{currentmarker}{\pgfqpoint{-0.041667in}{-0.041667in}}{\pgfqpoint{0.041667in}{0.041667in}}{%
\pgfpathmoveto{\pgfqpoint{-0.041667in}{-0.041667in}}%
\pgfpathlineto{\pgfqpoint{0.041667in}{0.041667in}}%
\pgfpathmoveto{\pgfqpoint{-0.041667in}{0.041667in}}%
\pgfpathlineto{\pgfqpoint{0.041667in}{-0.041667in}}%
\pgfusepath{stroke,fill}%
}%
\begin{pgfscope}%
\pgfsys@transformshift{2.262704in}{2.301955in}%
\pgfsys@useobject{currentmarker}{}%
\end{pgfscope}%
\begin{pgfscope}%
\pgfsys@transformshift{2.482239in}{2.301955in}%
\pgfsys@useobject{currentmarker}{}%
\end{pgfscope}%
\end{pgfscope}%
\begin{pgfscope}%
\pgfsetrectcap%
\pgfsetmiterjoin%
\pgfsetlinewidth{0.803000pt}%
\definecolor{currentstroke}{rgb}{0.000000,0.000000,0.000000}%
\pgfsetstrokecolor{currentstroke}%
\pgfsetdash{}{0pt}%
\pgfpathmoveto{\pgfqpoint{0.733531in}{0.548769in}}%
\pgfpathlineto{\pgfqpoint{0.733531in}{2.301955in}}%
\pgfusepath{stroke}%
\end{pgfscope}%
\begin{pgfscope}%
\pgfsetrectcap%
\pgfsetmiterjoin%
\pgfsetlinewidth{0.803000pt}%
\definecolor{currentstroke}{rgb}{0.000000,0.000000,0.000000}%
\pgfsetstrokecolor{currentstroke}%
\pgfsetdash{}{0pt}%
\pgfpathmoveto{\pgfqpoint{3.761597in}{0.548769in}}%
\pgfpathlineto{\pgfqpoint{3.761597in}{2.301955in}}%
\pgfusepath{stroke}%
\end{pgfscope}%
\begin{pgfscope}%
\pgfsetrectcap%
\pgfsetmiterjoin%
\pgfsetlinewidth{0.803000pt}%
\definecolor{currentstroke}{rgb}{0.000000,0.000000,0.000000}%
\pgfsetstrokecolor{currentstroke}%
\pgfsetdash{}{0pt}%
\pgfpathmoveto{\pgfqpoint{0.733531in}{0.548769in}}%
\pgfpathlineto{\pgfqpoint{3.761597in}{0.548769in}}%
\pgfusepath{stroke}%
\end{pgfscope}%
\begin{pgfscope}%
\pgfsetrectcap%
\pgfsetmiterjoin%
\pgfsetlinewidth{0.803000pt}%
\definecolor{currentstroke}{rgb}{0.000000,0.000000,0.000000}%
\pgfsetstrokecolor{currentstroke}%
\pgfsetdash{}{0pt}%
\pgfpathmoveto{\pgfqpoint{0.733531in}{2.301955in}}%
\pgfpathlineto{\pgfqpoint{3.761597in}{2.301955in}}%
\pgfusepath{stroke}%
\end{pgfscope}%
\begin{pgfscope}%
\pgfsetbuttcap%
\pgfsetmiterjoin%
\definecolor{currentfill}{rgb}{1.000000,1.000000,1.000000}%
\pgfsetfillcolor{currentfill}%
\pgfsetfillopacity{0.800000}%
\pgfsetlinewidth{1.003750pt}%
\definecolor{currentstroke}{rgb}{0.800000,0.800000,0.800000}%
\pgfsetstrokecolor{currentstroke}%
\pgfsetstrokeopacity{0.800000}%
\pgfsetdash{}{0pt}%
\pgfpathmoveto{\pgfqpoint{0.830753in}{1.997171in}}%
\pgfpathlineto{\pgfqpoint{2.157621in}{1.997171in}}%
\pgfpathquadraticcurveto{\pgfqpoint{2.185399in}{1.997171in}}{\pgfqpoint{2.185399in}{2.024949in}}%
\pgfpathlineto{\pgfqpoint{2.185399in}{2.204733in}}%
\pgfpathquadraticcurveto{\pgfqpoint{2.185399in}{2.232510in}}{\pgfqpoint{2.157621in}{2.232510in}}%
\pgfpathlineto{\pgfqpoint{0.830753in}{2.232510in}}%
\pgfpathquadraticcurveto{\pgfqpoint{0.802975in}{2.232510in}}{\pgfqpoint{0.802975in}{2.204733in}}%
\pgfpathlineto{\pgfqpoint{0.802975in}{2.024949in}}%
\pgfpathquadraticcurveto{\pgfqpoint{0.802975in}{1.997171in}}{\pgfqpoint{0.830753in}{1.997171in}}%
\pgfpathlineto{\pgfqpoint{0.830753in}{1.997171in}}%
\pgfpathclose%
\pgfusepath{stroke,fill}%
\end{pgfscope}%
\begin{pgfscope}%
\pgfsetrectcap%
\pgfsetroundjoin%
\pgfsetlinewidth{1.003750pt}%
\definecolor{currentstroke}{rgb}{0.121569,0.466667,0.705882}%
\pgfsetstrokecolor{currentstroke}%
\pgfsetdash{}{0pt}%
\pgfpathmoveto{\pgfqpoint{0.858531in}{2.128344in}}%
\pgfpathlineto{\pgfqpoint{0.997420in}{2.128344in}}%
\pgfpathlineto{\pgfqpoint{1.136309in}{2.128344in}}%
\pgfusepath{stroke}%
\end{pgfscope}%
\begin{pgfscope}%
\definecolor{textcolor}{rgb}{0.000000,0.000000,0.000000}%
\pgfsetstrokecolor{textcolor}%
\pgfsetfillcolor{textcolor}%
\pgftext[x=1.247420in,y=2.079733in,left,base]{\color{textcolor}\rmfamily\fontsize{10.000000}{12.000000}\selectfont \(\displaystyle N=5, k=0.1\)}%
\end{pgfscope}%
\end{pgfpicture}%
\makeatother%
\endgroup%

    \caption{$F_N$ für ein elliptischs filter.}
    \label{ellfilter:fig:elliptic}
\end{figure}


\begin{figure}
    \centering
    %% Creator: Matplotlib, PGF backend
%%
%% To include the figure in your LaTeX document, write
%%   \input{<filename>.pgf}
%%
%% Make sure the required packages are loaded in your preamble
%%   \usepackage{pgf}
%%
%% Also ensure that all the required font packages are loaded; for instance,
%% the lmodern package is sometimes necessary when using math font.
%%   \usepackage{lmodern}
%%
%% Figures using additional raster images can only be included by \input if
%% they are in the same directory as the main LaTeX file. For loading figures
%% from other directories you can use the `import` package
%%   \usepackage{import}
%%
%% and then include the figures with
%%   \import{<path to file>}{<filename>.pgf}
%%
%% Matplotlib used the following preamble
%%
\begingroup%
\makeatletter%
\begin{pgfpicture}%
\pgfpathrectangle{\pgfpointorigin}{\pgfqpoint{4.000000in}{2.500000in}}%
\pgfusepath{use as bounding box, clip}%
\begin{pgfscope}%
\pgfsetbuttcap%
\pgfsetmiterjoin%
\pgfsetlinewidth{0.000000pt}%
\definecolor{currentstroke}{rgb}{1.000000,1.000000,1.000000}%
\pgfsetstrokecolor{currentstroke}%
\pgfsetstrokeopacity{0.000000}%
\pgfsetdash{}{0pt}%
\pgfpathmoveto{\pgfqpoint{0.000000in}{0.000000in}}%
\pgfpathlineto{\pgfqpoint{4.000000in}{0.000000in}}%
\pgfpathlineto{\pgfqpoint{4.000000in}{2.500000in}}%
\pgfpathlineto{\pgfqpoint{0.000000in}{2.500000in}}%
\pgfpathlineto{\pgfqpoint{0.000000in}{0.000000in}}%
\pgfpathclose%
\pgfusepath{}%
\end{pgfscope}%
\begin{pgfscope}%
\pgfsetbuttcap%
\pgfsetmiterjoin%
\definecolor{currentfill}{rgb}{1.000000,1.000000,1.000000}%
\pgfsetfillcolor{currentfill}%
\pgfsetlinewidth{0.000000pt}%
\definecolor{currentstroke}{rgb}{0.000000,0.000000,0.000000}%
\pgfsetstrokecolor{currentstroke}%
\pgfsetstrokeopacity{0.000000}%
\pgfsetdash{}{0pt}%
\pgfpathmoveto{\pgfqpoint{0.617954in}{0.548769in}}%
\pgfpathlineto{\pgfqpoint{3.761597in}{0.548769in}}%
\pgfpathlineto{\pgfqpoint{3.761597in}{2.301955in}}%
\pgfpathlineto{\pgfqpoint{0.617954in}{2.301955in}}%
\pgfpathlineto{\pgfqpoint{0.617954in}{0.548769in}}%
\pgfpathclose%
\pgfusepath{fill}%
\end{pgfscope}%
\begin{pgfscope}%
\pgfpathrectangle{\pgfqpoint{0.617954in}{0.548769in}}{\pgfqpoint{3.143642in}{1.753186in}}%
\pgfusepath{clip}%
\pgfsetbuttcap%
\pgfsetmiterjoin%
\definecolor{currentfill}{rgb}{0.000000,0.501961,0.000000}%
\pgfsetfillcolor{currentfill}%
\pgfsetfillopacity{0.200000}%
\pgfsetlinewidth{0.000000pt}%
\definecolor{currentstroke}{rgb}{0.000000,0.000000,0.000000}%
\pgfsetstrokecolor{currentstroke}%
\pgfsetstrokeopacity{0.200000}%
\pgfsetdash{}{0pt}%
\pgfpathmoveto{\pgfqpoint{0.617954in}{1.788459in}}%
\pgfpathlineto{\pgfqpoint{2.189776in}{1.788459in}}%
\pgfpathlineto{\pgfqpoint{2.189776in}{3.541645in}}%
\pgfpathlineto{\pgfqpoint{0.617954in}{3.541645in}}%
\pgfpathlineto{\pgfqpoint{0.617954in}{1.788459in}}%
\pgfpathclose%
\pgfusepath{fill}%
\end{pgfscope}%
\begin{pgfscope}%
\pgfpathrectangle{\pgfqpoint{0.617954in}{0.548769in}}{\pgfqpoint{3.143642in}{1.753186in}}%
\pgfusepath{clip}%
\pgfsetbuttcap%
\pgfsetmiterjoin%
\definecolor{currentfill}{rgb}{1.000000,0.647059,0.000000}%
\pgfsetfillcolor{currentfill}%
\pgfsetfillopacity{0.200000}%
\pgfsetlinewidth{0.000000pt}%
\definecolor{currentstroke}{rgb}{0.000000,0.000000,0.000000}%
\pgfsetstrokecolor{currentstroke}%
\pgfsetstrokeopacity{0.200000}%
\pgfsetdash{}{0pt}%
\pgfpathmoveto{\pgfqpoint{2.189776in}{0.724087in}}%
\pgfpathlineto{\pgfqpoint{2.205494in}{0.724087in}}%
\pgfpathlineto{\pgfqpoint{2.205494in}{1.788459in}}%
\pgfpathlineto{\pgfqpoint{2.189776in}{1.788459in}}%
\pgfpathlineto{\pgfqpoint{2.189776in}{0.724087in}}%
\pgfpathclose%
\pgfusepath{fill}%
\end{pgfscope}%
\begin{pgfscope}%
\pgfpathrectangle{\pgfqpoint{0.617954in}{0.548769in}}{\pgfqpoint{3.143642in}{1.753186in}}%
\pgfusepath{clip}%
\pgfsetbuttcap%
\pgfsetmiterjoin%
\definecolor{currentfill}{rgb}{1.000000,0.000000,0.000000}%
\pgfsetfillcolor{currentfill}%
\pgfsetfillopacity{0.200000}%
\pgfsetlinewidth{0.000000pt}%
\definecolor{currentstroke}{rgb}{0.000000,0.000000,0.000000}%
\pgfsetstrokecolor{currentstroke}%
\pgfsetstrokeopacity{0.200000}%
\pgfsetdash{}{0pt}%
\pgfpathmoveto{\pgfqpoint{2.205494in}{0.548769in}}%
\pgfpathlineto{\pgfqpoint{3.777315in}{0.548769in}}%
\pgfpathlineto{\pgfqpoint{3.777315in}{0.724087in}}%
\pgfpathlineto{\pgfqpoint{2.205494in}{0.724087in}}%
\pgfpathlineto{\pgfqpoint{2.205494in}{0.548769in}}%
\pgfpathclose%
\pgfusepath{fill}%
\end{pgfscope}%
\begin{pgfscope}%
\pgfpathrectangle{\pgfqpoint{0.617954in}{0.548769in}}{\pgfqpoint{3.143642in}{1.753186in}}%
\pgfusepath{clip}%
\pgfsetrectcap%
\pgfsetroundjoin%
\pgfsetlinewidth{0.803000pt}%
\definecolor{currentstroke}{rgb}{0.690196,0.690196,0.690196}%
\pgfsetstrokecolor{currentstroke}%
\pgfsetdash{}{0pt}%
\pgfpathmoveto{\pgfqpoint{0.617954in}{0.548769in}}%
\pgfpathlineto{\pgfqpoint{0.617954in}{2.301955in}}%
\pgfusepath{stroke}%
\end{pgfscope}%
\begin{pgfscope}%
\pgfsetbuttcap%
\pgfsetroundjoin%
\definecolor{currentfill}{rgb}{0.000000,0.000000,0.000000}%
\pgfsetfillcolor{currentfill}%
\pgfsetlinewidth{0.803000pt}%
\definecolor{currentstroke}{rgb}{0.000000,0.000000,0.000000}%
\pgfsetstrokecolor{currentstroke}%
\pgfsetdash{}{0pt}%
\pgfsys@defobject{currentmarker}{\pgfqpoint{0.000000in}{-0.048611in}}{\pgfqpoint{0.000000in}{0.000000in}}{%
\pgfpathmoveto{\pgfqpoint{0.000000in}{0.000000in}}%
\pgfpathlineto{\pgfqpoint{0.000000in}{-0.048611in}}%
\pgfusepath{stroke,fill}%
}%
\begin{pgfscope}%
\pgfsys@transformshift{0.617954in}{0.548769in}%
\pgfsys@useobject{currentmarker}{}%
\end{pgfscope}%
\end{pgfscope}%
\begin{pgfscope}%
\definecolor{textcolor}{rgb}{0.000000,0.000000,0.000000}%
\pgfsetstrokecolor{textcolor}%
\pgfsetfillcolor{textcolor}%
\pgftext[x=0.617954in,y=0.451547in,,top]{\color{textcolor}\rmfamily\fontsize{10.000000}{12.000000}\selectfont \(\displaystyle {0.0}\)}%
\end{pgfscope}%
\begin{pgfscope}%
\pgfpathrectangle{\pgfqpoint{0.617954in}{0.548769in}}{\pgfqpoint{3.143642in}{1.753186in}}%
\pgfusepath{clip}%
\pgfsetrectcap%
\pgfsetroundjoin%
\pgfsetlinewidth{0.803000pt}%
\definecolor{currentstroke}{rgb}{0.690196,0.690196,0.690196}%
\pgfsetstrokecolor{currentstroke}%
\pgfsetdash{}{0pt}%
\pgfpathmoveto{\pgfqpoint{1.403865in}{0.548769in}}%
\pgfpathlineto{\pgfqpoint{1.403865in}{2.301955in}}%
\pgfusepath{stroke}%
\end{pgfscope}%
\begin{pgfscope}%
\pgfsetbuttcap%
\pgfsetroundjoin%
\definecolor{currentfill}{rgb}{0.000000,0.000000,0.000000}%
\pgfsetfillcolor{currentfill}%
\pgfsetlinewidth{0.803000pt}%
\definecolor{currentstroke}{rgb}{0.000000,0.000000,0.000000}%
\pgfsetstrokecolor{currentstroke}%
\pgfsetdash{}{0pt}%
\pgfsys@defobject{currentmarker}{\pgfqpoint{0.000000in}{-0.048611in}}{\pgfqpoint{0.000000in}{0.000000in}}{%
\pgfpathmoveto{\pgfqpoint{0.000000in}{0.000000in}}%
\pgfpathlineto{\pgfqpoint{0.000000in}{-0.048611in}}%
\pgfusepath{stroke,fill}%
}%
\begin{pgfscope}%
\pgfsys@transformshift{1.403865in}{0.548769in}%
\pgfsys@useobject{currentmarker}{}%
\end{pgfscope}%
\end{pgfscope}%
\begin{pgfscope}%
\definecolor{textcolor}{rgb}{0.000000,0.000000,0.000000}%
\pgfsetstrokecolor{textcolor}%
\pgfsetfillcolor{textcolor}%
\pgftext[x=1.403865in,y=0.451547in,,top]{\color{textcolor}\rmfamily\fontsize{10.000000}{12.000000}\selectfont \(\displaystyle {0.5}\)}%
\end{pgfscope}%
\begin{pgfscope}%
\pgfpathrectangle{\pgfqpoint{0.617954in}{0.548769in}}{\pgfqpoint{3.143642in}{1.753186in}}%
\pgfusepath{clip}%
\pgfsetrectcap%
\pgfsetroundjoin%
\pgfsetlinewidth{0.803000pt}%
\definecolor{currentstroke}{rgb}{0.690196,0.690196,0.690196}%
\pgfsetstrokecolor{currentstroke}%
\pgfsetdash{}{0pt}%
\pgfpathmoveto{\pgfqpoint{2.189776in}{0.548769in}}%
\pgfpathlineto{\pgfqpoint{2.189776in}{2.301955in}}%
\pgfusepath{stroke}%
\end{pgfscope}%
\begin{pgfscope}%
\pgfsetbuttcap%
\pgfsetroundjoin%
\definecolor{currentfill}{rgb}{0.000000,0.000000,0.000000}%
\pgfsetfillcolor{currentfill}%
\pgfsetlinewidth{0.803000pt}%
\definecolor{currentstroke}{rgb}{0.000000,0.000000,0.000000}%
\pgfsetstrokecolor{currentstroke}%
\pgfsetdash{}{0pt}%
\pgfsys@defobject{currentmarker}{\pgfqpoint{0.000000in}{-0.048611in}}{\pgfqpoint{0.000000in}{0.000000in}}{%
\pgfpathmoveto{\pgfqpoint{0.000000in}{0.000000in}}%
\pgfpathlineto{\pgfqpoint{0.000000in}{-0.048611in}}%
\pgfusepath{stroke,fill}%
}%
\begin{pgfscope}%
\pgfsys@transformshift{2.189776in}{0.548769in}%
\pgfsys@useobject{currentmarker}{}%
\end{pgfscope}%
\end{pgfscope}%
\begin{pgfscope}%
\definecolor{textcolor}{rgb}{0.000000,0.000000,0.000000}%
\pgfsetstrokecolor{textcolor}%
\pgfsetfillcolor{textcolor}%
\pgftext[x=2.189776in,y=0.451547in,,top]{\color{textcolor}\rmfamily\fontsize{10.000000}{12.000000}\selectfont \(\displaystyle {1.0}\)}%
\end{pgfscope}%
\begin{pgfscope}%
\pgfpathrectangle{\pgfqpoint{0.617954in}{0.548769in}}{\pgfqpoint{3.143642in}{1.753186in}}%
\pgfusepath{clip}%
\pgfsetrectcap%
\pgfsetroundjoin%
\pgfsetlinewidth{0.803000pt}%
\definecolor{currentstroke}{rgb}{0.690196,0.690196,0.690196}%
\pgfsetstrokecolor{currentstroke}%
\pgfsetdash{}{0pt}%
\pgfpathmoveto{\pgfqpoint{2.975686in}{0.548769in}}%
\pgfpathlineto{\pgfqpoint{2.975686in}{2.301955in}}%
\pgfusepath{stroke}%
\end{pgfscope}%
\begin{pgfscope}%
\pgfsetbuttcap%
\pgfsetroundjoin%
\definecolor{currentfill}{rgb}{0.000000,0.000000,0.000000}%
\pgfsetfillcolor{currentfill}%
\pgfsetlinewidth{0.803000pt}%
\definecolor{currentstroke}{rgb}{0.000000,0.000000,0.000000}%
\pgfsetstrokecolor{currentstroke}%
\pgfsetdash{}{0pt}%
\pgfsys@defobject{currentmarker}{\pgfqpoint{0.000000in}{-0.048611in}}{\pgfqpoint{0.000000in}{0.000000in}}{%
\pgfpathmoveto{\pgfqpoint{0.000000in}{0.000000in}}%
\pgfpathlineto{\pgfqpoint{0.000000in}{-0.048611in}}%
\pgfusepath{stroke,fill}%
}%
\begin{pgfscope}%
\pgfsys@transformshift{2.975686in}{0.548769in}%
\pgfsys@useobject{currentmarker}{}%
\end{pgfscope}%
\end{pgfscope}%
\begin{pgfscope}%
\definecolor{textcolor}{rgb}{0.000000,0.000000,0.000000}%
\pgfsetstrokecolor{textcolor}%
\pgfsetfillcolor{textcolor}%
\pgftext[x=2.975686in,y=0.451547in,,top]{\color{textcolor}\rmfamily\fontsize{10.000000}{12.000000}\selectfont \(\displaystyle {1.5}\)}%
\end{pgfscope}%
\begin{pgfscope}%
\pgfpathrectangle{\pgfqpoint{0.617954in}{0.548769in}}{\pgfqpoint{3.143642in}{1.753186in}}%
\pgfusepath{clip}%
\pgfsetrectcap%
\pgfsetroundjoin%
\pgfsetlinewidth{0.803000pt}%
\definecolor{currentstroke}{rgb}{0.690196,0.690196,0.690196}%
\pgfsetstrokecolor{currentstroke}%
\pgfsetdash{}{0pt}%
\pgfpathmoveto{\pgfqpoint{3.761597in}{0.548769in}}%
\pgfpathlineto{\pgfqpoint{3.761597in}{2.301955in}}%
\pgfusepath{stroke}%
\end{pgfscope}%
\begin{pgfscope}%
\pgfsetbuttcap%
\pgfsetroundjoin%
\definecolor{currentfill}{rgb}{0.000000,0.000000,0.000000}%
\pgfsetfillcolor{currentfill}%
\pgfsetlinewidth{0.803000pt}%
\definecolor{currentstroke}{rgb}{0.000000,0.000000,0.000000}%
\pgfsetstrokecolor{currentstroke}%
\pgfsetdash{}{0pt}%
\pgfsys@defobject{currentmarker}{\pgfqpoint{0.000000in}{-0.048611in}}{\pgfqpoint{0.000000in}{0.000000in}}{%
\pgfpathmoveto{\pgfqpoint{0.000000in}{0.000000in}}%
\pgfpathlineto{\pgfqpoint{0.000000in}{-0.048611in}}%
\pgfusepath{stroke,fill}%
}%
\begin{pgfscope}%
\pgfsys@transformshift{3.761597in}{0.548769in}%
\pgfsys@useobject{currentmarker}{}%
\end{pgfscope}%
\end{pgfscope}%
\begin{pgfscope}%
\definecolor{textcolor}{rgb}{0.000000,0.000000,0.000000}%
\pgfsetstrokecolor{textcolor}%
\pgfsetfillcolor{textcolor}%
\pgftext[x=3.761597in,y=0.451547in,,top]{\color{textcolor}\rmfamily\fontsize{10.000000}{12.000000}\selectfont \(\displaystyle {2.0}\)}%
\end{pgfscope}%
\begin{pgfscope}%
\definecolor{textcolor}{rgb}{0.000000,0.000000,0.000000}%
\pgfsetstrokecolor{textcolor}%
\pgfsetfillcolor{textcolor}%
\pgftext[x=2.189776in,y=0.272534in,,top]{\color{textcolor}\rmfamily\fontsize{10.000000}{12.000000}\selectfont \(\displaystyle w\)}%
\end{pgfscope}%
\begin{pgfscope}%
\pgfpathrectangle{\pgfqpoint{0.617954in}{0.548769in}}{\pgfqpoint{3.143642in}{1.753186in}}%
\pgfusepath{clip}%
\pgfsetrectcap%
\pgfsetroundjoin%
\pgfsetlinewidth{0.803000pt}%
\definecolor{currentstroke}{rgb}{0.690196,0.690196,0.690196}%
\pgfsetstrokecolor{currentstroke}%
\pgfsetdash{}{0pt}%
\pgfpathmoveto{\pgfqpoint{0.617954in}{0.548769in}}%
\pgfpathlineto{\pgfqpoint{3.761597in}{0.548769in}}%
\pgfusepath{stroke}%
\end{pgfscope}%
\begin{pgfscope}%
\pgfsetbuttcap%
\pgfsetroundjoin%
\definecolor{currentfill}{rgb}{0.000000,0.000000,0.000000}%
\pgfsetfillcolor{currentfill}%
\pgfsetlinewidth{0.803000pt}%
\definecolor{currentstroke}{rgb}{0.000000,0.000000,0.000000}%
\pgfsetstrokecolor{currentstroke}%
\pgfsetdash{}{0pt}%
\pgfsys@defobject{currentmarker}{\pgfqpoint{-0.048611in}{0.000000in}}{\pgfqpoint{-0.000000in}{0.000000in}}{%
\pgfpathmoveto{\pgfqpoint{-0.000000in}{0.000000in}}%
\pgfpathlineto{\pgfqpoint{-0.048611in}{0.000000in}}%
\pgfusepath{stroke,fill}%
}%
\begin{pgfscope}%
\pgfsys@transformshift{0.617954in}{0.548769in}%
\pgfsys@useobject{currentmarker}{}%
\end{pgfscope}%
\end{pgfscope}%
\begin{pgfscope}%
\definecolor{textcolor}{rgb}{0.000000,0.000000,0.000000}%
\pgfsetstrokecolor{textcolor}%
\pgfsetfillcolor{textcolor}%
\pgftext[x=0.343262in, y=0.500544in, left, base]{\color{textcolor}\rmfamily\fontsize{10.000000}{12.000000}\selectfont \(\displaystyle {0.0}\)}%
\end{pgfscope}%
\begin{pgfscope}%
\pgfpathrectangle{\pgfqpoint{0.617954in}{0.548769in}}{\pgfqpoint{3.143642in}{1.753186in}}%
\pgfusepath{clip}%
\pgfsetrectcap%
\pgfsetroundjoin%
\pgfsetlinewidth{0.803000pt}%
\definecolor{currentstroke}{rgb}{0.690196,0.690196,0.690196}%
\pgfsetstrokecolor{currentstroke}%
\pgfsetdash{}{0pt}%
\pgfpathmoveto{\pgfqpoint{0.617954in}{0.899406in}}%
\pgfpathlineto{\pgfqpoint{3.761597in}{0.899406in}}%
\pgfusepath{stroke}%
\end{pgfscope}%
\begin{pgfscope}%
\pgfsetbuttcap%
\pgfsetroundjoin%
\definecolor{currentfill}{rgb}{0.000000,0.000000,0.000000}%
\pgfsetfillcolor{currentfill}%
\pgfsetlinewidth{0.803000pt}%
\definecolor{currentstroke}{rgb}{0.000000,0.000000,0.000000}%
\pgfsetstrokecolor{currentstroke}%
\pgfsetdash{}{0pt}%
\pgfsys@defobject{currentmarker}{\pgfqpoint{-0.048611in}{0.000000in}}{\pgfqpoint{-0.000000in}{0.000000in}}{%
\pgfpathmoveto{\pgfqpoint{-0.000000in}{0.000000in}}%
\pgfpathlineto{\pgfqpoint{-0.048611in}{0.000000in}}%
\pgfusepath{stroke,fill}%
}%
\begin{pgfscope}%
\pgfsys@transformshift{0.617954in}{0.899406in}%
\pgfsys@useobject{currentmarker}{}%
\end{pgfscope}%
\end{pgfscope}%
\begin{pgfscope}%
\definecolor{textcolor}{rgb}{0.000000,0.000000,0.000000}%
\pgfsetstrokecolor{textcolor}%
\pgfsetfillcolor{textcolor}%
\pgftext[x=0.343262in, y=0.851181in, left, base]{\color{textcolor}\rmfamily\fontsize{10.000000}{12.000000}\selectfont \(\displaystyle {0.2}\)}%
\end{pgfscope}%
\begin{pgfscope}%
\pgfpathrectangle{\pgfqpoint{0.617954in}{0.548769in}}{\pgfqpoint{3.143642in}{1.753186in}}%
\pgfusepath{clip}%
\pgfsetrectcap%
\pgfsetroundjoin%
\pgfsetlinewidth{0.803000pt}%
\definecolor{currentstroke}{rgb}{0.690196,0.690196,0.690196}%
\pgfsetstrokecolor{currentstroke}%
\pgfsetdash{}{0pt}%
\pgfpathmoveto{\pgfqpoint{0.617954in}{1.250043in}}%
\pgfpathlineto{\pgfqpoint{3.761597in}{1.250043in}}%
\pgfusepath{stroke}%
\end{pgfscope}%
\begin{pgfscope}%
\pgfsetbuttcap%
\pgfsetroundjoin%
\definecolor{currentfill}{rgb}{0.000000,0.000000,0.000000}%
\pgfsetfillcolor{currentfill}%
\pgfsetlinewidth{0.803000pt}%
\definecolor{currentstroke}{rgb}{0.000000,0.000000,0.000000}%
\pgfsetstrokecolor{currentstroke}%
\pgfsetdash{}{0pt}%
\pgfsys@defobject{currentmarker}{\pgfqpoint{-0.048611in}{0.000000in}}{\pgfqpoint{-0.000000in}{0.000000in}}{%
\pgfpathmoveto{\pgfqpoint{-0.000000in}{0.000000in}}%
\pgfpathlineto{\pgfqpoint{-0.048611in}{0.000000in}}%
\pgfusepath{stroke,fill}%
}%
\begin{pgfscope}%
\pgfsys@transformshift{0.617954in}{1.250043in}%
\pgfsys@useobject{currentmarker}{}%
\end{pgfscope}%
\end{pgfscope}%
\begin{pgfscope}%
\definecolor{textcolor}{rgb}{0.000000,0.000000,0.000000}%
\pgfsetstrokecolor{textcolor}%
\pgfsetfillcolor{textcolor}%
\pgftext[x=0.343262in, y=1.201818in, left, base]{\color{textcolor}\rmfamily\fontsize{10.000000}{12.000000}\selectfont \(\displaystyle {0.4}\)}%
\end{pgfscope}%
\begin{pgfscope}%
\pgfpathrectangle{\pgfqpoint{0.617954in}{0.548769in}}{\pgfqpoint{3.143642in}{1.753186in}}%
\pgfusepath{clip}%
\pgfsetrectcap%
\pgfsetroundjoin%
\pgfsetlinewidth{0.803000pt}%
\definecolor{currentstroke}{rgb}{0.690196,0.690196,0.690196}%
\pgfsetstrokecolor{currentstroke}%
\pgfsetdash{}{0pt}%
\pgfpathmoveto{\pgfqpoint{0.617954in}{1.600680in}}%
\pgfpathlineto{\pgfqpoint{3.761597in}{1.600680in}}%
\pgfusepath{stroke}%
\end{pgfscope}%
\begin{pgfscope}%
\pgfsetbuttcap%
\pgfsetroundjoin%
\definecolor{currentfill}{rgb}{0.000000,0.000000,0.000000}%
\pgfsetfillcolor{currentfill}%
\pgfsetlinewidth{0.803000pt}%
\definecolor{currentstroke}{rgb}{0.000000,0.000000,0.000000}%
\pgfsetstrokecolor{currentstroke}%
\pgfsetdash{}{0pt}%
\pgfsys@defobject{currentmarker}{\pgfqpoint{-0.048611in}{0.000000in}}{\pgfqpoint{-0.000000in}{0.000000in}}{%
\pgfpathmoveto{\pgfqpoint{-0.000000in}{0.000000in}}%
\pgfpathlineto{\pgfqpoint{-0.048611in}{0.000000in}}%
\pgfusepath{stroke,fill}%
}%
\begin{pgfscope}%
\pgfsys@transformshift{0.617954in}{1.600680in}%
\pgfsys@useobject{currentmarker}{}%
\end{pgfscope}%
\end{pgfscope}%
\begin{pgfscope}%
\definecolor{textcolor}{rgb}{0.000000,0.000000,0.000000}%
\pgfsetstrokecolor{textcolor}%
\pgfsetfillcolor{textcolor}%
\pgftext[x=0.343262in, y=1.552455in, left, base]{\color{textcolor}\rmfamily\fontsize{10.000000}{12.000000}\selectfont \(\displaystyle {0.6}\)}%
\end{pgfscope}%
\begin{pgfscope}%
\pgfpathrectangle{\pgfqpoint{0.617954in}{0.548769in}}{\pgfqpoint{3.143642in}{1.753186in}}%
\pgfusepath{clip}%
\pgfsetrectcap%
\pgfsetroundjoin%
\pgfsetlinewidth{0.803000pt}%
\definecolor{currentstroke}{rgb}{0.690196,0.690196,0.690196}%
\pgfsetstrokecolor{currentstroke}%
\pgfsetdash{}{0pt}%
\pgfpathmoveto{\pgfqpoint{0.617954in}{1.951318in}}%
\pgfpathlineto{\pgfqpoint{3.761597in}{1.951318in}}%
\pgfusepath{stroke}%
\end{pgfscope}%
\begin{pgfscope}%
\pgfsetbuttcap%
\pgfsetroundjoin%
\definecolor{currentfill}{rgb}{0.000000,0.000000,0.000000}%
\pgfsetfillcolor{currentfill}%
\pgfsetlinewidth{0.803000pt}%
\definecolor{currentstroke}{rgb}{0.000000,0.000000,0.000000}%
\pgfsetstrokecolor{currentstroke}%
\pgfsetdash{}{0pt}%
\pgfsys@defobject{currentmarker}{\pgfqpoint{-0.048611in}{0.000000in}}{\pgfqpoint{-0.000000in}{0.000000in}}{%
\pgfpathmoveto{\pgfqpoint{-0.000000in}{0.000000in}}%
\pgfpathlineto{\pgfqpoint{-0.048611in}{0.000000in}}%
\pgfusepath{stroke,fill}%
}%
\begin{pgfscope}%
\pgfsys@transformshift{0.617954in}{1.951318in}%
\pgfsys@useobject{currentmarker}{}%
\end{pgfscope}%
\end{pgfscope}%
\begin{pgfscope}%
\definecolor{textcolor}{rgb}{0.000000,0.000000,0.000000}%
\pgfsetstrokecolor{textcolor}%
\pgfsetfillcolor{textcolor}%
\pgftext[x=0.343262in, y=1.903092in, left, base]{\color{textcolor}\rmfamily\fontsize{10.000000}{12.000000}\selectfont \(\displaystyle {0.8}\)}%
\end{pgfscope}%
\begin{pgfscope}%
\pgfpathrectangle{\pgfqpoint{0.617954in}{0.548769in}}{\pgfqpoint{3.143642in}{1.753186in}}%
\pgfusepath{clip}%
\pgfsetrectcap%
\pgfsetroundjoin%
\pgfsetlinewidth{0.803000pt}%
\definecolor{currentstroke}{rgb}{0.690196,0.690196,0.690196}%
\pgfsetstrokecolor{currentstroke}%
\pgfsetdash{}{0pt}%
\pgfpathmoveto{\pgfqpoint{0.617954in}{2.301955in}}%
\pgfpathlineto{\pgfqpoint{3.761597in}{2.301955in}}%
\pgfusepath{stroke}%
\end{pgfscope}%
\begin{pgfscope}%
\pgfsetbuttcap%
\pgfsetroundjoin%
\definecolor{currentfill}{rgb}{0.000000,0.000000,0.000000}%
\pgfsetfillcolor{currentfill}%
\pgfsetlinewidth{0.803000pt}%
\definecolor{currentstroke}{rgb}{0.000000,0.000000,0.000000}%
\pgfsetstrokecolor{currentstroke}%
\pgfsetdash{}{0pt}%
\pgfsys@defobject{currentmarker}{\pgfqpoint{-0.048611in}{0.000000in}}{\pgfqpoint{-0.000000in}{0.000000in}}{%
\pgfpathmoveto{\pgfqpoint{-0.000000in}{0.000000in}}%
\pgfpathlineto{\pgfqpoint{-0.048611in}{0.000000in}}%
\pgfusepath{stroke,fill}%
}%
\begin{pgfscope}%
\pgfsys@transformshift{0.617954in}{2.301955in}%
\pgfsys@useobject{currentmarker}{}%
\end{pgfscope}%
\end{pgfscope}%
\begin{pgfscope}%
\definecolor{textcolor}{rgb}{0.000000,0.000000,0.000000}%
\pgfsetstrokecolor{textcolor}%
\pgfsetfillcolor{textcolor}%
\pgftext[x=0.343262in, y=2.253730in, left, base]{\color{textcolor}\rmfamily\fontsize{10.000000}{12.000000}\selectfont \(\displaystyle {1.0}\)}%
\end{pgfscope}%
\begin{pgfscope}%
\definecolor{textcolor}{rgb}{0.000000,0.000000,0.000000}%
\pgfsetstrokecolor{textcolor}%
\pgfsetfillcolor{textcolor}%
\pgftext[x=0.287707in,y=1.425362in,,bottom,rotate=90.000000]{\color{textcolor}\rmfamily\fontsize{10.000000}{12.000000}\selectfont \(\displaystyle |H(w)|\)}%
\end{pgfscope}%
\begin{pgfscope}%
\pgfpathrectangle{\pgfqpoint{0.617954in}{0.548769in}}{\pgfqpoint{3.143642in}{1.753186in}}%
\pgfusepath{clip}%
\pgfsetrectcap%
\pgfsetroundjoin%
\pgfsetlinewidth{1.003750pt}%
\definecolor{currentstroke}{rgb}{0.121569,0.466667,0.705882}%
\pgfsetstrokecolor{currentstroke}%
\pgfsetdash{}{0pt}%
\pgfpathmoveto{\pgfqpoint{0.617954in}{2.301955in}}%
\pgfpathlineto{\pgfqpoint{0.646254in}{2.300410in}}%
\pgfpathlineto{\pgfqpoint{0.674554in}{2.295805in}}%
\pgfpathlineto{\pgfqpoint{0.703640in}{2.287983in}}%
\pgfpathlineto{\pgfqpoint{0.734298in}{2.276528in}}%
\pgfpathlineto{\pgfqpoint{0.767315in}{2.260797in}}%
\pgfpathlineto{\pgfqpoint{0.802690in}{2.240472in}}%
\pgfpathlineto{\pgfqpoint{0.842781in}{2.213774in}}%
\pgfpathlineto{\pgfqpoint{0.889947in}{2.178486in}}%
\pgfpathlineto{\pgfqpoint{0.952050in}{2.127836in}}%
\pgfpathlineto{\pgfqpoint{1.147791in}{1.965399in}}%
\pgfpathlineto{\pgfqpoint{1.205963in}{1.922684in}}%
\pgfpathlineto{\pgfqpoint{1.257846in}{1.888382in}}%
\pgfpathlineto{\pgfqpoint{1.305012in}{1.860803in}}%
\pgfpathlineto{\pgfqpoint{1.349034in}{1.838524in}}%
\pgfpathlineto{\pgfqpoint{1.390698in}{1.820826in}}%
\pgfpathlineto{\pgfqpoint{1.430003in}{1.807434in}}%
\pgfpathlineto{\pgfqpoint{1.466950in}{1.798051in}}%
\pgfpathlineto{\pgfqpoint{1.501539in}{1.792361in}}%
\pgfpathlineto{\pgfqpoint{1.534555in}{1.790011in}}%
\pgfpathlineto{\pgfqpoint{1.566000in}{1.790875in}}%
\pgfpathlineto{\pgfqpoint{1.595872in}{1.794823in}}%
\pgfpathlineto{\pgfqpoint{1.624172in}{1.801709in}}%
\pgfpathlineto{\pgfqpoint{1.650899in}{1.811365in}}%
\pgfpathlineto{\pgfqpoint{1.676841in}{1.824030in}}%
\pgfpathlineto{\pgfqpoint{1.701996in}{1.839797in}}%
\pgfpathlineto{\pgfqpoint{1.726365in}{1.858757in}}%
\pgfpathlineto{\pgfqpoint{1.749949in}{1.880985in}}%
\pgfpathlineto{\pgfqpoint{1.773532in}{1.907483in}}%
\pgfpathlineto{\pgfqpoint{1.797115in}{1.938719in}}%
\pgfpathlineto{\pgfqpoint{1.820698in}{1.975139in}}%
\pgfpathlineto{\pgfqpoint{1.845068in}{2.018557in}}%
\pgfpathlineto{\pgfqpoint{1.871009in}{2.071217in}}%
\pgfpathlineto{\pgfqpoint{1.903240in}{2.144258in}}%
\pgfpathlineto{\pgfqpoint{1.949620in}{2.249449in}}%
\pgfpathlineto{\pgfqpoint{1.965342in}{2.277408in}}%
\pgfpathlineto{\pgfqpoint{1.977134in}{2.292569in}}%
\pgfpathlineto{\pgfqpoint{1.986567in}{2.299856in}}%
\pgfpathlineto{\pgfqpoint{1.993642in}{2.301922in}}%
\pgfpathlineto{\pgfqpoint{2.000717in}{2.300685in}}%
\pgfpathlineto{\pgfqpoint{2.007792in}{2.295839in}}%
\pgfpathlineto{\pgfqpoint{2.014867in}{2.287135in}}%
\pgfpathlineto{\pgfqpoint{2.023514in}{2.271025in}}%
\pgfpathlineto{\pgfqpoint{2.032947in}{2.246492in}}%
\pgfpathlineto{\pgfqpoint{2.043953in}{2.209014in}}%
\pgfpathlineto{\pgfqpoint{2.057317in}{2.152165in}}%
\pgfpathlineto{\pgfqpoint{2.076183in}{2.056775in}}%
\pgfpathlineto{\pgfqpoint{2.114702in}{1.858945in}}%
\pgfpathlineto{\pgfqpoint{2.126494in}{1.815249in}}%
\pgfpathlineto{\pgfqpoint{2.134355in}{1.796399in}}%
\pgfpathlineto{\pgfqpoint{2.139858in}{1.790308in}}%
\pgfpathlineto{\pgfqpoint{2.143002in}{1.790254in}}%
\pgfpathlineto{\pgfqpoint{2.146147in}{1.793264in}}%
\pgfpathlineto{\pgfqpoint{2.150077in}{1.802263in}}%
\pgfpathlineto{\pgfqpoint{2.154794in}{1.822852in}}%
\pgfpathlineto{\pgfqpoint{2.159510in}{1.857784in}}%
\pgfpathlineto{\pgfqpoint{2.165013in}{1.924261in}}%
\pgfpathlineto{\pgfqpoint{2.170516in}{2.030210in}}%
\pgfpathlineto{\pgfqpoint{2.181521in}{2.301670in}}%
\pgfpathlineto{\pgfqpoint{2.182308in}{2.299748in}}%
\pgfpathlineto{\pgfqpoint{2.183880in}{2.267565in}}%
\pgfpathlineto{\pgfqpoint{2.186238in}{2.135783in}}%
\pgfpathlineto{\pgfqpoint{2.192527in}{1.496420in}}%
\pgfpathlineto{\pgfqpoint{2.198816in}{1.001269in}}%
\pgfpathlineto{\pgfqpoint{2.205105in}{0.731898in}}%
\pgfpathlineto{\pgfqpoint{2.211393in}{0.583247in}}%
\pgfpathlineto{\pgfqpoint{2.213752in}{0.552138in}}%
\pgfpathlineto{\pgfqpoint{2.220827in}{0.630495in}}%
\pgfpathlineto{\pgfqpoint{2.227902in}{0.675607in}}%
\pgfpathlineto{\pgfqpoint{2.234977in}{0.701566in}}%
\pgfpathlineto{\pgfqpoint{2.241266in}{0.714626in}}%
\pgfpathlineto{\pgfqpoint{2.247554in}{0.721456in}}%
\pgfpathlineto{\pgfqpoint{2.253843in}{0.723972in}}%
\pgfpathlineto{\pgfqpoint{2.260918in}{0.723210in}}%
\pgfpathlineto{\pgfqpoint{2.269565in}{0.718770in}}%
\pgfpathlineto{\pgfqpoint{2.281357in}{0.708826in}}%
\pgfpathlineto{\pgfqpoint{2.300224in}{0.688160in}}%
\pgfpathlineto{\pgfqpoint{2.385123in}{0.590361in}}%
\pgfpathlineto{\pgfqpoint{2.417354in}{0.559882in}}%
\pgfpathlineto{\pgfqpoint{2.430717in}{0.549063in}}%
\pgfpathlineto{\pgfqpoint{2.463734in}{0.574407in}}%
\pgfpathlineto{\pgfqpoint{2.498323in}{0.597083in}}%
\pgfpathlineto{\pgfqpoint{2.535270in}{0.617584in}}%
\pgfpathlineto{\pgfqpoint{2.574575in}{0.635868in}}%
\pgfpathlineto{\pgfqpoint{2.617811in}{0.652529in}}%
\pgfpathlineto{\pgfqpoint{2.664977in}{0.667359in}}%
\pgfpathlineto{\pgfqpoint{2.717646in}{0.680619in}}%
\pgfpathlineto{\pgfqpoint{2.776604in}{0.692216in}}%
\pgfpathlineto{\pgfqpoint{2.843424in}{0.702155in}}%
\pgfpathlineto{\pgfqpoint{2.920462in}{0.710422in}}%
\pgfpathlineto{\pgfqpoint{3.010864in}{0.716921in}}%
\pgfpathlineto{\pgfqpoint{3.118561in}{0.721464in}}%
\pgfpathlineto{\pgfqpoint{3.250627in}{0.723829in}}%
\pgfpathlineto{\pgfqpoint{3.419640in}{0.723619in}}%
\pgfpathlineto{\pgfqpoint{3.651542in}{0.720038in}}%
\pgfpathlineto{\pgfqpoint{3.761597in}{0.717600in}}%
\pgfpathlineto{\pgfqpoint{3.761597in}{0.717600in}}%
\pgfusepath{stroke}%
\end{pgfscope}%
\begin{pgfscope}%
\pgfsetrectcap%
\pgfsetmiterjoin%
\pgfsetlinewidth{0.803000pt}%
\definecolor{currentstroke}{rgb}{0.000000,0.000000,0.000000}%
\pgfsetstrokecolor{currentstroke}%
\pgfsetdash{}{0pt}%
\pgfpathmoveto{\pgfqpoint{0.617954in}{0.548769in}}%
\pgfpathlineto{\pgfqpoint{0.617954in}{2.301955in}}%
\pgfusepath{stroke}%
\end{pgfscope}%
\begin{pgfscope}%
\pgfsetrectcap%
\pgfsetmiterjoin%
\pgfsetlinewidth{0.803000pt}%
\definecolor{currentstroke}{rgb}{0.000000,0.000000,0.000000}%
\pgfsetstrokecolor{currentstroke}%
\pgfsetdash{}{0pt}%
\pgfpathmoveto{\pgfqpoint{3.761597in}{0.548769in}}%
\pgfpathlineto{\pgfqpoint{3.761597in}{2.301955in}}%
\pgfusepath{stroke}%
\end{pgfscope}%
\begin{pgfscope}%
\pgfsetrectcap%
\pgfsetmiterjoin%
\pgfsetlinewidth{0.803000pt}%
\definecolor{currentstroke}{rgb}{0.000000,0.000000,0.000000}%
\pgfsetstrokecolor{currentstroke}%
\pgfsetdash{}{0pt}%
\pgfpathmoveto{\pgfqpoint{0.617954in}{0.548769in}}%
\pgfpathlineto{\pgfqpoint{3.761597in}{0.548769in}}%
\pgfusepath{stroke}%
\end{pgfscope}%
\begin{pgfscope}%
\pgfsetrectcap%
\pgfsetmiterjoin%
\pgfsetlinewidth{0.803000pt}%
\definecolor{currentstroke}{rgb}{0.000000,0.000000,0.000000}%
\pgfsetstrokecolor{currentstroke}%
\pgfsetdash{}{0pt}%
\pgfpathmoveto{\pgfqpoint{0.617954in}{2.301955in}}%
\pgfpathlineto{\pgfqpoint{3.761597in}{2.301955in}}%
\pgfusepath{stroke}%
\end{pgfscope}%
\end{pgfpicture}%
\makeatother%
\endgroup%

    \caption{Die resultierende frequenzantwort eines elliptischs filter.}
    \label{ellfilter:fig:elliptic_freq}
\end{figure}

\subsection{Gradgleichung}

Der $\cd^{-1}$ Term muss so verzogen werden, dass die umgebene $\cd$-Funktion die Nullstellen und Pole trifft.
Dies trifft ein wenn die Degree Equation erfüllt ist.

\begin{equation}
    N \frac{K^\prime}{K} = \frac{K^\prime_1}{K_1}
\end{equation}


Leider ist das lösen dieser Gleichung nicht trivial.
Die Rechnung wird in \ref{ellfilter:bib:orfanidis} im Detail angeschaut.

\begin{figure}
    \centering
    %% Creator: Matplotlib, PGF backend
%%
%% To include the figure in your LaTeX document, write
%%   \input{<filename>.pgf}
%%
%% Make sure the required packages are loaded in your preamble
%%   \usepackage{pgf}
%%
%% Also ensure that all the required font packages are loaded; for instance,
%% the lmodern package is sometimes necessary when using math font.
%%   \usepackage{lmodern}
%%
%% Figures using additional raster images can only be included by \input if
%% they are in the same directory as the main LaTeX file. For loading figures
%% from other directories you can use the `import` package
%%   \usepackage{import}
%%
%% and then include the figures with
%%   \import{<path to file>}{<filename>.pgf}
%%
%% Matplotlib used the following preamble
%%
\begingroup%
\makeatletter%
\begin{pgfpicture}%
\pgfpathrectangle{\pgfpointorigin}{\pgfqpoint{5.000000in}{2.500000in}}%
\pgfusepath{use as bounding box, clip}%
\begin{pgfscope}%
\pgfsetbuttcap%
\pgfsetmiterjoin%
\pgfsetlinewidth{0.000000pt}%
\definecolor{currentstroke}{rgb}{1.000000,1.000000,1.000000}%
\pgfsetstrokecolor{currentstroke}%
\pgfsetstrokeopacity{0.000000}%
\pgfsetdash{}{0pt}%
\pgfpathmoveto{\pgfqpoint{0.000000in}{0.000000in}}%
\pgfpathlineto{\pgfqpoint{5.000000in}{0.000000in}}%
\pgfpathlineto{\pgfqpoint{5.000000in}{2.500000in}}%
\pgfpathlineto{\pgfqpoint{0.000000in}{2.500000in}}%
\pgfpathlineto{\pgfqpoint{0.000000in}{0.000000in}}%
\pgfpathclose%
\pgfusepath{}%
\end{pgfscope}%
\begin{pgfscope}%
\pgfsetbuttcap%
\pgfsetmiterjoin%
\definecolor{currentfill}{rgb}{1.000000,1.000000,1.000000}%
\pgfsetfillcolor{currentfill}%
\pgfsetlinewidth{0.000000pt}%
\definecolor{currentstroke}{rgb}{0.000000,0.000000,0.000000}%
\pgfsetstrokecolor{currentstroke}%
\pgfsetstrokeopacity{0.000000}%
\pgfsetdash{}{0pt}%
\pgfpathmoveto{\pgfqpoint{0.316407in}{0.548769in}}%
\pgfpathlineto{\pgfqpoint{2.256930in}{0.548769in}}%
\pgfpathlineto{\pgfqpoint{2.256930in}{2.301955in}}%
\pgfpathlineto{\pgfqpoint{0.316407in}{2.301955in}}%
\pgfpathlineto{\pgfqpoint{0.316407in}{0.548769in}}%
\pgfpathclose%
\pgfusepath{fill}%
\end{pgfscope}%
\begin{pgfscope}%
\pgfsetbuttcap%
\pgfsetroundjoin%
\definecolor{currentfill}{rgb}{0.000000,0.000000,0.000000}%
\pgfsetfillcolor{currentfill}%
\pgfsetlinewidth{0.803000pt}%
\definecolor{currentstroke}{rgb}{0.000000,0.000000,0.000000}%
\pgfsetstrokecolor{currentstroke}%
\pgfsetdash{}{0pt}%
\pgfsys@defobject{currentmarker}{\pgfqpoint{0.000000in}{-0.048611in}}{\pgfqpoint{0.000000in}{0.000000in}}{%
\pgfpathmoveto{\pgfqpoint{0.000000in}{0.000000in}}%
\pgfpathlineto{\pgfqpoint{0.000000in}{-0.048611in}}%
\pgfusepath{stroke,fill}%
}%
\begin{pgfscope}%
\pgfsys@transformshift{0.316407in}{0.548769in}%
\pgfsys@useobject{currentmarker}{}%
\end{pgfscope}%
\end{pgfscope}%
\begin{pgfscope}%
\definecolor{textcolor}{rgb}{0.000000,0.000000,0.000000}%
\pgfsetstrokecolor{textcolor}%
\pgfsetfillcolor{textcolor}%
\pgftext[x=0.316407in,y=0.451547in,,top]{\color{textcolor}\rmfamily\fontsize{10.000000}{12.000000}\selectfont \(\displaystyle {0.00}\)}%
\end{pgfscope}%
\begin{pgfscope}%
\pgfsetbuttcap%
\pgfsetroundjoin%
\definecolor{currentfill}{rgb}{0.000000,0.000000,0.000000}%
\pgfsetfillcolor{currentfill}%
\pgfsetlinewidth{0.803000pt}%
\definecolor{currentstroke}{rgb}{0.000000,0.000000,0.000000}%
\pgfsetstrokecolor{currentstroke}%
\pgfsetdash{}{0pt}%
\pgfsys@defobject{currentmarker}{\pgfqpoint{0.000000in}{-0.048611in}}{\pgfqpoint{0.000000in}{0.000000in}}{%
\pgfpathmoveto{\pgfqpoint{0.000000in}{0.000000in}}%
\pgfpathlineto{\pgfqpoint{0.000000in}{-0.048611in}}%
\pgfusepath{stroke,fill}%
}%
\begin{pgfscope}%
\pgfsys@transformshift{0.801538in}{0.548769in}%
\pgfsys@useobject{currentmarker}{}%
\end{pgfscope}%
\end{pgfscope}%
\begin{pgfscope}%
\definecolor{textcolor}{rgb}{0.000000,0.000000,0.000000}%
\pgfsetstrokecolor{textcolor}%
\pgfsetfillcolor{textcolor}%
\pgftext[x=0.801538in,y=0.451547in,,top]{\color{textcolor}\rmfamily\fontsize{10.000000}{12.000000}\selectfont \(\displaystyle {0.25}\)}%
\end{pgfscope}%
\begin{pgfscope}%
\pgfsetbuttcap%
\pgfsetroundjoin%
\definecolor{currentfill}{rgb}{0.000000,0.000000,0.000000}%
\pgfsetfillcolor{currentfill}%
\pgfsetlinewidth{0.803000pt}%
\definecolor{currentstroke}{rgb}{0.000000,0.000000,0.000000}%
\pgfsetstrokecolor{currentstroke}%
\pgfsetdash{}{0pt}%
\pgfsys@defobject{currentmarker}{\pgfqpoint{0.000000in}{-0.048611in}}{\pgfqpoint{0.000000in}{0.000000in}}{%
\pgfpathmoveto{\pgfqpoint{0.000000in}{0.000000in}}%
\pgfpathlineto{\pgfqpoint{0.000000in}{-0.048611in}}%
\pgfusepath{stroke,fill}%
}%
\begin{pgfscope}%
\pgfsys@transformshift{1.286669in}{0.548769in}%
\pgfsys@useobject{currentmarker}{}%
\end{pgfscope}%
\end{pgfscope}%
\begin{pgfscope}%
\definecolor{textcolor}{rgb}{0.000000,0.000000,0.000000}%
\pgfsetstrokecolor{textcolor}%
\pgfsetfillcolor{textcolor}%
\pgftext[x=1.286669in,y=0.451547in,,top]{\color{textcolor}\rmfamily\fontsize{10.000000}{12.000000}\selectfont \(\displaystyle {0.50}\)}%
\end{pgfscope}%
\begin{pgfscope}%
\pgfsetbuttcap%
\pgfsetroundjoin%
\definecolor{currentfill}{rgb}{0.000000,0.000000,0.000000}%
\pgfsetfillcolor{currentfill}%
\pgfsetlinewidth{0.803000pt}%
\definecolor{currentstroke}{rgb}{0.000000,0.000000,0.000000}%
\pgfsetstrokecolor{currentstroke}%
\pgfsetdash{}{0pt}%
\pgfsys@defobject{currentmarker}{\pgfqpoint{0.000000in}{-0.048611in}}{\pgfqpoint{0.000000in}{0.000000in}}{%
\pgfpathmoveto{\pgfqpoint{0.000000in}{0.000000in}}%
\pgfpathlineto{\pgfqpoint{0.000000in}{-0.048611in}}%
\pgfusepath{stroke,fill}%
}%
\begin{pgfscope}%
\pgfsys@transformshift{1.771800in}{0.548769in}%
\pgfsys@useobject{currentmarker}{}%
\end{pgfscope}%
\end{pgfscope}%
\begin{pgfscope}%
\definecolor{textcolor}{rgb}{0.000000,0.000000,0.000000}%
\pgfsetstrokecolor{textcolor}%
\pgfsetfillcolor{textcolor}%
\pgftext[x=1.771800in,y=0.451547in,,top]{\color{textcolor}\rmfamily\fontsize{10.000000}{12.000000}\selectfont \(\displaystyle {0.75}\)}%
\end{pgfscope}%
\begin{pgfscope}%
\pgfsetbuttcap%
\pgfsetroundjoin%
\definecolor{currentfill}{rgb}{0.000000,0.000000,0.000000}%
\pgfsetfillcolor{currentfill}%
\pgfsetlinewidth{0.803000pt}%
\definecolor{currentstroke}{rgb}{0.000000,0.000000,0.000000}%
\pgfsetstrokecolor{currentstroke}%
\pgfsetdash{}{0pt}%
\pgfsys@defobject{currentmarker}{\pgfqpoint{0.000000in}{-0.048611in}}{\pgfqpoint{0.000000in}{0.000000in}}{%
\pgfpathmoveto{\pgfqpoint{0.000000in}{0.000000in}}%
\pgfpathlineto{\pgfqpoint{0.000000in}{-0.048611in}}%
\pgfusepath{stroke,fill}%
}%
\begin{pgfscope}%
\pgfsys@transformshift{2.256930in}{0.548769in}%
\pgfsys@useobject{currentmarker}{}%
\end{pgfscope}%
\end{pgfscope}%
\begin{pgfscope}%
\definecolor{textcolor}{rgb}{0.000000,0.000000,0.000000}%
\pgfsetstrokecolor{textcolor}%
\pgfsetfillcolor{textcolor}%
\pgftext[x=2.256930in,y=0.451547in,,top]{\color{textcolor}\rmfamily\fontsize{10.000000}{12.000000}\selectfont \(\displaystyle {1.00}\)}%
\end{pgfscope}%
\begin{pgfscope}%
\definecolor{textcolor}{rgb}{0.000000,0.000000,0.000000}%
\pgfsetstrokecolor{textcolor}%
\pgfsetfillcolor{textcolor}%
\pgftext[x=1.286669in,y=0.272534in,,top]{\color{textcolor}\rmfamily\fontsize{10.000000}{12.000000}\selectfont \(\displaystyle k\)}%
\end{pgfscope}%
\begin{pgfscope}%
\pgfsetbuttcap%
\pgfsetroundjoin%
\definecolor{currentfill}{rgb}{0.000000,0.000000,0.000000}%
\pgfsetfillcolor{currentfill}%
\pgfsetlinewidth{0.803000pt}%
\definecolor{currentstroke}{rgb}{0.000000,0.000000,0.000000}%
\pgfsetstrokecolor{currentstroke}%
\pgfsetdash{}{0pt}%
\pgfsys@defobject{currentmarker}{\pgfqpoint{-0.048611in}{0.000000in}}{\pgfqpoint{-0.000000in}{0.000000in}}{%
\pgfpathmoveto{\pgfqpoint{-0.000000in}{0.000000in}}%
\pgfpathlineto{\pgfqpoint{-0.048611in}{0.000000in}}%
\pgfusepath{stroke,fill}%
}%
\begin{pgfscope}%
\pgfsys@transformshift{0.316407in}{0.548769in}%
\pgfsys@useobject{currentmarker}{}%
\end{pgfscope}%
\end{pgfscope}%
\begin{pgfscope}%
\definecolor{textcolor}{rgb}{0.000000,0.000000,0.000000}%
\pgfsetstrokecolor{textcolor}%
\pgfsetfillcolor{textcolor}%
\pgftext[x=0.149740in, y=0.500544in, left, base]{\color{textcolor}\rmfamily\fontsize{10.000000}{12.000000}\selectfont \(\displaystyle {0}\)}%
\end{pgfscope}%
\begin{pgfscope}%
\pgfsetbuttcap%
\pgfsetroundjoin%
\definecolor{currentfill}{rgb}{0.000000,0.000000,0.000000}%
\pgfsetfillcolor{currentfill}%
\pgfsetlinewidth{0.803000pt}%
\definecolor{currentstroke}{rgb}{0.000000,0.000000,0.000000}%
\pgfsetstrokecolor{currentstroke}%
\pgfsetdash{}{0pt}%
\pgfsys@defobject{currentmarker}{\pgfqpoint{-0.048611in}{0.000000in}}{\pgfqpoint{-0.000000in}{0.000000in}}{%
\pgfpathmoveto{\pgfqpoint{-0.000000in}{0.000000in}}%
\pgfpathlineto{\pgfqpoint{-0.048611in}{0.000000in}}%
\pgfusepath{stroke,fill}%
}%
\begin{pgfscope}%
\pgfsys@transformshift{0.316407in}{0.987065in}%
\pgfsys@useobject{currentmarker}{}%
\end{pgfscope}%
\end{pgfscope}%
\begin{pgfscope}%
\definecolor{textcolor}{rgb}{0.000000,0.000000,0.000000}%
\pgfsetstrokecolor{textcolor}%
\pgfsetfillcolor{textcolor}%
\pgftext[x=0.149740in, y=0.938840in, left, base]{\color{textcolor}\rmfamily\fontsize{10.000000}{12.000000}\selectfont \(\displaystyle {1}\)}%
\end{pgfscope}%
\begin{pgfscope}%
\pgfsetbuttcap%
\pgfsetroundjoin%
\definecolor{currentfill}{rgb}{0.000000,0.000000,0.000000}%
\pgfsetfillcolor{currentfill}%
\pgfsetlinewidth{0.803000pt}%
\definecolor{currentstroke}{rgb}{0.000000,0.000000,0.000000}%
\pgfsetstrokecolor{currentstroke}%
\pgfsetdash{}{0pt}%
\pgfsys@defobject{currentmarker}{\pgfqpoint{-0.048611in}{0.000000in}}{\pgfqpoint{-0.000000in}{0.000000in}}{%
\pgfpathmoveto{\pgfqpoint{-0.000000in}{0.000000in}}%
\pgfpathlineto{\pgfqpoint{-0.048611in}{0.000000in}}%
\pgfusepath{stroke,fill}%
}%
\begin{pgfscope}%
\pgfsys@transformshift{0.316407in}{1.425362in}%
\pgfsys@useobject{currentmarker}{}%
\end{pgfscope}%
\end{pgfscope}%
\begin{pgfscope}%
\definecolor{textcolor}{rgb}{0.000000,0.000000,0.000000}%
\pgfsetstrokecolor{textcolor}%
\pgfsetfillcolor{textcolor}%
\pgftext[x=0.149740in, y=1.377137in, left, base]{\color{textcolor}\rmfamily\fontsize{10.000000}{12.000000}\selectfont \(\displaystyle {2}\)}%
\end{pgfscope}%
\begin{pgfscope}%
\pgfsetbuttcap%
\pgfsetroundjoin%
\definecolor{currentfill}{rgb}{0.000000,0.000000,0.000000}%
\pgfsetfillcolor{currentfill}%
\pgfsetlinewidth{0.803000pt}%
\definecolor{currentstroke}{rgb}{0.000000,0.000000,0.000000}%
\pgfsetstrokecolor{currentstroke}%
\pgfsetdash{}{0pt}%
\pgfsys@defobject{currentmarker}{\pgfqpoint{-0.048611in}{0.000000in}}{\pgfqpoint{-0.000000in}{0.000000in}}{%
\pgfpathmoveto{\pgfqpoint{-0.000000in}{0.000000in}}%
\pgfpathlineto{\pgfqpoint{-0.048611in}{0.000000in}}%
\pgfusepath{stroke,fill}%
}%
\begin{pgfscope}%
\pgfsys@transformshift{0.316407in}{1.863658in}%
\pgfsys@useobject{currentmarker}{}%
\end{pgfscope}%
\end{pgfscope}%
\begin{pgfscope}%
\definecolor{textcolor}{rgb}{0.000000,0.000000,0.000000}%
\pgfsetstrokecolor{textcolor}%
\pgfsetfillcolor{textcolor}%
\pgftext[x=0.149740in, y=1.815433in, left, base]{\color{textcolor}\rmfamily\fontsize{10.000000}{12.000000}\selectfont \(\displaystyle {3}\)}%
\end{pgfscope}%
\begin{pgfscope}%
\pgfsetbuttcap%
\pgfsetroundjoin%
\definecolor{currentfill}{rgb}{0.000000,0.000000,0.000000}%
\pgfsetfillcolor{currentfill}%
\pgfsetlinewidth{0.803000pt}%
\definecolor{currentstroke}{rgb}{0.000000,0.000000,0.000000}%
\pgfsetstrokecolor{currentstroke}%
\pgfsetdash{}{0pt}%
\pgfsys@defobject{currentmarker}{\pgfqpoint{-0.048611in}{0.000000in}}{\pgfqpoint{-0.000000in}{0.000000in}}{%
\pgfpathmoveto{\pgfqpoint{-0.000000in}{0.000000in}}%
\pgfpathlineto{\pgfqpoint{-0.048611in}{0.000000in}}%
\pgfusepath{stroke,fill}%
}%
\begin{pgfscope}%
\pgfsys@transformshift{0.316407in}{2.301955in}%
\pgfsys@useobject{currentmarker}{}%
\end{pgfscope}%
\end{pgfscope}%
\begin{pgfscope}%
\definecolor{textcolor}{rgb}{0.000000,0.000000,0.000000}%
\pgfsetstrokecolor{textcolor}%
\pgfsetfillcolor{textcolor}%
\pgftext[x=0.149740in, y=2.253730in, left, base]{\color{textcolor}\rmfamily\fontsize{10.000000}{12.000000}\selectfont \(\displaystyle {4}\)}%
\end{pgfscope}%
\begin{pgfscope}%
\pgfpathrectangle{\pgfqpoint{0.316407in}{0.548769in}}{\pgfqpoint{1.940523in}{1.753186in}}%
\pgfusepath{clip}%
\pgfsetrectcap%
\pgfsetroundjoin%
\pgfsetlinewidth{1.003750pt}%
\definecolor{currentstroke}{rgb}{0.121569,0.466667,0.705882}%
\pgfsetstrokecolor{currentstroke}%
\pgfsetdash{}{0pt}%
\pgfpathmoveto{\pgfqpoint{0.316427in}{1.237243in}}%
\pgfpathlineto{\pgfqpoint{0.316601in}{1.237243in}}%
\pgfpathlineto{\pgfqpoint{0.318348in}{1.237244in}}%
\pgfpathlineto{\pgfqpoint{0.335813in}{1.237261in}}%
\pgfpathlineto{\pgfqpoint{0.355218in}{1.237312in}}%
\pgfpathlineto{\pgfqpoint{0.374623in}{1.237398in}}%
\pgfpathlineto{\pgfqpoint{0.394028in}{1.237519in}}%
\pgfpathlineto{\pgfqpoint{0.413434in}{1.237674in}}%
\pgfpathlineto{\pgfqpoint{0.432839in}{1.237864in}}%
\pgfpathlineto{\pgfqpoint{0.452244in}{1.238089in}}%
\pgfpathlineto{\pgfqpoint{0.471649in}{1.238349in}}%
\pgfpathlineto{\pgfqpoint{0.491054in}{1.238644in}}%
\pgfpathlineto{\pgfqpoint{0.510460in}{1.238974in}}%
\pgfpathlineto{\pgfqpoint{0.529865in}{1.239340in}}%
\pgfpathlineto{\pgfqpoint{0.549270in}{1.239742in}}%
\pgfpathlineto{\pgfqpoint{0.568675in}{1.240180in}}%
\pgfpathlineto{\pgfqpoint{0.588081in}{1.240655in}}%
\pgfpathlineto{\pgfqpoint{0.607486in}{1.241166in}}%
\pgfpathlineto{\pgfqpoint{0.626891in}{1.241714in}}%
\pgfpathlineto{\pgfqpoint{0.646296in}{1.242300in}}%
\pgfpathlineto{\pgfqpoint{0.665702in}{1.242924in}}%
\pgfpathlineto{\pgfqpoint{0.685107in}{1.243586in}}%
\pgfpathlineto{\pgfqpoint{0.704512in}{1.244287in}}%
\pgfpathlineto{\pgfqpoint{0.723917in}{1.245028in}}%
\pgfpathlineto{\pgfqpoint{0.743322in}{1.245809in}}%
\pgfpathlineto{\pgfqpoint{0.762728in}{1.246630in}}%
\pgfpathlineto{\pgfqpoint{0.782133in}{1.247492in}}%
\pgfpathlineto{\pgfqpoint{0.801538in}{1.248396in}}%
\pgfpathlineto{\pgfqpoint{0.820943in}{1.249343in}}%
\pgfpathlineto{\pgfqpoint{0.840349in}{1.250333in}}%
\pgfpathlineto{\pgfqpoint{0.859754in}{1.251367in}}%
\pgfpathlineto{\pgfqpoint{0.879159in}{1.252446in}}%
\pgfpathlineto{\pgfqpoint{0.898564in}{1.253571in}}%
\pgfpathlineto{\pgfqpoint{0.917969in}{1.254743in}}%
\pgfpathlineto{\pgfqpoint{0.937375in}{1.255962in}}%
\pgfpathlineto{\pgfqpoint{0.956780in}{1.257230in}}%
\pgfpathlineto{\pgfqpoint{0.976185in}{1.258548in}}%
\pgfpathlineto{\pgfqpoint{0.995590in}{1.259917in}}%
\pgfpathlineto{\pgfqpoint{1.014996in}{1.261339in}}%
\pgfpathlineto{\pgfqpoint{1.034401in}{1.262814in}}%
\pgfpathlineto{\pgfqpoint{1.053806in}{1.264344in}}%
\pgfpathlineto{\pgfqpoint{1.073211in}{1.265930in}}%
\pgfpathlineto{\pgfqpoint{1.092617in}{1.267575in}}%
\pgfpathlineto{\pgfqpoint{1.112022in}{1.269279in}}%
\pgfpathlineto{\pgfqpoint{1.131427in}{1.271045in}}%
\pgfpathlineto{\pgfqpoint{1.150832in}{1.272874in}}%
\pgfpathlineto{\pgfqpoint{1.170237in}{1.274768in}}%
\pgfpathlineto{\pgfqpoint{1.189643in}{1.276729in}}%
\pgfpathlineto{\pgfqpoint{1.209048in}{1.278760in}}%
\pgfpathlineto{\pgfqpoint{1.228453in}{1.280863in}}%
\pgfpathlineto{\pgfqpoint{1.247858in}{1.283040in}}%
\pgfpathlineto{\pgfqpoint{1.267264in}{1.285294in}}%
\pgfpathlineto{\pgfqpoint{1.286669in}{1.287627in}}%
\pgfpathlineto{\pgfqpoint{1.306074in}{1.290044in}}%
\pgfpathlineto{\pgfqpoint{1.325479in}{1.292546in}}%
\pgfpathlineto{\pgfqpoint{1.344884in}{1.295137in}}%
\pgfpathlineto{\pgfqpoint{1.364290in}{1.297822in}}%
\pgfpathlineto{\pgfqpoint{1.383695in}{1.300603in}}%
\pgfpathlineto{\pgfqpoint{1.403100in}{1.303485in}}%
\pgfpathlineto{\pgfqpoint{1.422505in}{1.306473in}}%
\pgfpathlineto{\pgfqpoint{1.441911in}{1.309570in}}%
\pgfpathlineto{\pgfqpoint{1.461316in}{1.312784in}}%
\pgfpathlineto{\pgfqpoint{1.480721in}{1.316118in}}%
\pgfpathlineto{\pgfqpoint{1.500126in}{1.319579in}}%
\pgfpathlineto{\pgfqpoint{1.519532in}{1.323174in}}%
\pgfpathlineto{\pgfqpoint{1.538937in}{1.326910in}}%
\pgfpathlineto{\pgfqpoint{1.558342in}{1.330793in}}%
\pgfpathlineto{\pgfqpoint{1.577747in}{1.334833in}}%
\pgfpathlineto{\pgfqpoint{1.597152in}{1.339039in}}%
\pgfpathlineto{\pgfqpoint{1.616558in}{1.343420in}}%
\pgfpathlineto{\pgfqpoint{1.635963in}{1.347988in}}%
\pgfpathlineto{\pgfqpoint{1.655368in}{1.352753in}}%
\pgfpathlineto{\pgfqpoint{1.674773in}{1.357730in}}%
\pgfpathlineto{\pgfqpoint{1.694179in}{1.362933in}}%
\pgfpathlineto{\pgfqpoint{1.713584in}{1.368377in}}%
\pgfpathlineto{\pgfqpoint{1.732989in}{1.374081in}}%
\pgfpathlineto{\pgfqpoint{1.752394in}{1.380064in}}%
\pgfpathlineto{\pgfqpoint{1.771800in}{1.386349in}}%
\pgfpathlineto{\pgfqpoint{1.791205in}{1.392961in}}%
\pgfpathlineto{\pgfqpoint{1.810610in}{1.399927in}}%
\pgfpathlineto{\pgfqpoint{1.830015in}{1.407281in}}%
\pgfpathlineto{\pgfqpoint{1.849420in}{1.415059in}}%
\pgfpathlineto{\pgfqpoint{1.868826in}{1.423303in}}%
\pgfpathlineto{\pgfqpoint{1.888231in}{1.432062in}}%
\pgfpathlineto{\pgfqpoint{1.907636in}{1.441392in}}%
\pgfpathlineto{\pgfqpoint{1.927041in}{1.451361in}}%
\pgfpathlineto{\pgfqpoint{1.946447in}{1.462048in}}%
\pgfpathlineto{\pgfqpoint{1.965852in}{1.473546in}}%
\pgfpathlineto{\pgfqpoint{1.985257in}{1.485971in}}%
\pgfpathlineto{\pgfqpoint{2.004662in}{1.499462in}}%
\pgfpathlineto{\pgfqpoint{2.024067in}{1.514194in}}%
\pgfpathlineto{\pgfqpoint{2.043473in}{1.530388in}}%
\pgfpathlineto{\pgfqpoint{2.062878in}{1.548326in}}%
\pgfpathlineto{\pgfqpoint{2.082283in}{1.568383in}}%
\pgfpathlineto{\pgfqpoint{2.101688in}{1.591069in}}%
\pgfpathlineto{\pgfqpoint{2.121094in}{1.617098in}}%
\pgfpathlineto{\pgfqpoint{2.140499in}{1.647519in}}%
\pgfpathlineto{\pgfqpoint{2.159904in}{1.683962in}}%
\pgfpathlineto{\pgfqpoint{2.179309in}{1.729164in}}%
\pgfpathlineto{\pgfqpoint{2.198715in}{1.788269in}}%
\pgfpathlineto{\pgfqpoint{2.218120in}{1.872854in}}%
\pgfpathlineto{\pgfqpoint{2.237525in}{2.019955in}}%
\pgfpathlineto{\pgfqpoint{2.247876in}{2.315844in}}%
\pgfusepath{stroke}%
\end{pgfscope}%
\begin{pgfscope}%
\pgfpathrectangle{\pgfqpoint{0.316407in}{0.548769in}}{\pgfqpoint{1.940523in}{1.753186in}}%
\pgfusepath{clip}%
\pgfsetrectcap%
\pgfsetroundjoin%
\pgfsetlinewidth{1.003750pt}%
\definecolor{currentstroke}{rgb}{1.000000,0.498039,0.054902}%
\pgfsetstrokecolor{currentstroke}%
\pgfsetdash{}{0pt}%
\pgfpathmoveto{\pgfqpoint{0.454821in}{2.315844in}}%
\pgfpathlineto{\pgfqpoint{0.471649in}{2.265444in}}%
\pgfpathlineto{\pgfqpoint{0.491054in}{2.214262in}}%
\pgfpathlineto{\pgfqpoint{0.510460in}{2.168554in}}%
\pgfpathlineto{\pgfqpoint{0.529865in}{2.127278in}}%
\pgfpathlineto{\pgfqpoint{0.549270in}{2.089666in}}%
\pgfpathlineto{\pgfqpoint{0.568675in}{2.055132in}}%
\pgfpathlineto{\pgfqpoint{0.588081in}{2.023222in}}%
\pgfpathlineto{\pgfqpoint{0.607486in}{1.993575in}}%
\pgfpathlineto{\pgfqpoint{0.626891in}{1.965899in}}%
\pgfpathlineto{\pgfqpoint{0.646296in}{1.939958in}}%
\pgfpathlineto{\pgfqpoint{0.665702in}{1.915554in}}%
\pgfpathlineto{\pgfqpoint{0.685107in}{1.892522in}}%
\pgfpathlineto{\pgfqpoint{0.704512in}{1.870720in}}%
\pgfpathlineto{\pgfqpoint{0.723917in}{1.850031in}}%
\pgfpathlineto{\pgfqpoint{0.743322in}{1.830351in}}%
\pgfpathlineto{\pgfqpoint{0.762728in}{1.811591in}}%
\pgfpathlineto{\pgfqpoint{0.782133in}{1.793672in}}%
\pgfpathlineto{\pgfqpoint{0.801538in}{1.776528in}}%
\pgfpathlineto{\pgfqpoint{0.820943in}{1.760096in}}%
\pgfpathlineto{\pgfqpoint{0.840349in}{1.744324in}}%
\pgfpathlineto{\pgfqpoint{0.859754in}{1.729164in}}%
\pgfpathlineto{\pgfqpoint{0.879159in}{1.714573in}}%
\pgfpathlineto{\pgfqpoint{0.898564in}{1.700513in}}%
\pgfpathlineto{\pgfqpoint{0.917969in}{1.686948in}}%
\pgfpathlineto{\pgfqpoint{0.937375in}{1.673849in}}%
\pgfpathlineto{\pgfqpoint{0.956780in}{1.661185in}}%
\pgfpathlineto{\pgfqpoint{0.976185in}{1.648932in}}%
\pgfpathlineto{\pgfqpoint{0.995590in}{1.637065in}}%
\pgfpathlineto{\pgfqpoint{1.014996in}{1.625563in}}%
\pgfpathlineto{\pgfqpoint{1.034401in}{1.614406in}}%
\pgfpathlineto{\pgfqpoint{1.053806in}{1.603575in}}%
\pgfpathlineto{\pgfqpoint{1.073211in}{1.593053in}}%
\pgfpathlineto{\pgfqpoint{1.092617in}{1.582826in}}%
\pgfpathlineto{\pgfqpoint{1.112022in}{1.572877in}}%
\pgfpathlineto{\pgfqpoint{1.131427in}{1.563195in}}%
\pgfpathlineto{\pgfqpoint{1.150832in}{1.553766in}}%
\pgfpathlineto{\pgfqpoint{1.170237in}{1.544578in}}%
\pgfpathlineto{\pgfqpoint{1.189643in}{1.535621in}}%
\pgfpathlineto{\pgfqpoint{1.209048in}{1.526884in}}%
\pgfpathlineto{\pgfqpoint{1.228453in}{1.518359in}}%
\pgfpathlineto{\pgfqpoint{1.247858in}{1.510036in}}%
\pgfpathlineto{\pgfqpoint{1.267264in}{1.501906in}}%
\pgfpathlineto{\pgfqpoint{1.286669in}{1.493962in}}%
\pgfpathlineto{\pgfqpoint{1.306074in}{1.486197in}}%
\pgfpathlineto{\pgfqpoint{1.325479in}{1.478603in}}%
\pgfpathlineto{\pgfqpoint{1.344884in}{1.471174in}}%
\pgfpathlineto{\pgfqpoint{1.364290in}{1.463903in}}%
\pgfpathlineto{\pgfqpoint{1.383695in}{1.456785in}}%
\pgfpathlineto{\pgfqpoint{1.403100in}{1.449815in}}%
\pgfpathlineto{\pgfqpoint{1.422505in}{1.442986in}}%
\pgfpathlineto{\pgfqpoint{1.441911in}{1.436294in}}%
\pgfpathlineto{\pgfqpoint{1.461316in}{1.429735in}}%
\pgfpathlineto{\pgfqpoint{1.480721in}{1.423303in}}%
\pgfpathlineto{\pgfqpoint{1.500126in}{1.416995in}}%
\pgfpathlineto{\pgfqpoint{1.519532in}{1.410805in}}%
\pgfpathlineto{\pgfqpoint{1.538937in}{1.404732in}}%
\pgfpathlineto{\pgfqpoint{1.558342in}{1.398770in}}%
\pgfpathlineto{\pgfqpoint{1.577747in}{1.392916in}}%
\pgfpathlineto{\pgfqpoint{1.597152in}{1.387167in}}%
\pgfpathlineto{\pgfqpoint{1.616558in}{1.381520in}}%
\pgfpathlineto{\pgfqpoint{1.635963in}{1.375971in}}%
\pgfpathlineto{\pgfqpoint{1.655368in}{1.370518in}}%
\pgfpathlineto{\pgfqpoint{1.674773in}{1.365158in}}%
\pgfpathlineto{\pgfqpoint{1.694179in}{1.359888in}}%
\pgfpathlineto{\pgfqpoint{1.713584in}{1.354705in}}%
\pgfpathlineto{\pgfqpoint{1.732989in}{1.349607in}}%
\pgfpathlineto{\pgfqpoint{1.752394in}{1.344593in}}%
\pgfpathlineto{\pgfqpoint{1.771800in}{1.339658in}}%
\pgfpathlineto{\pgfqpoint{1.791205in}{1.334802in}}%
\pgfpathlineto{\pgfqpoint{1.810610in}{1.330022in}}%
\pgfpathlineto{\pgfqpoint{1.830015in}{1.325316in}}%
\pgfpathlineto{\pgfqpoint{1.849420in}{1.320682in}}%
\pgfpathlineto{\pgfqpoint{1.868826in}{1.316118in}}%
\pgfpathlineto{\pgfqpoint{1.888231in}{1.311623in}}%
\pgfpathlineto{\pgfqpoint{1.907636in}{1.307195in}}%
\pgfpathlineto{\pgfqpoint{1.927041in}{1.302831in}}%
\pgfpathlineto{\pgfqpoint{1.946447in}{1.298531in}}%
\pgfpathlineto{\pgfqpoint{1.965852in}{1.294293in}}%
\pgfpathlineto{\pgfqpoint{1.985257in}{1.290116in}}%
\pgfpathlineto{\pgfqpoint{2.004662in}{1.285997in}}%
\pgfpathlineto{\pgfqpoint{2.024067in}{1.281936in}}%
\pgfpathlineto{\pgfqpoint{2.043473in}{1.277931in}}%
\pgfpathlineto{\pgfqpoint{2.062878in}{1.273982in}}%
\pgfpathlineto{\pgfqpoint{2.082283in}{1.270085in}}%
\pgfpathlineto{\pgfqpoint{2.101688in}{1.266241in}}%
\pgfpathlineto{\pgfqpoint{2.121094in}{1.262449in}}%
\pgfpathlineto{\pgfqpoint{2.140499in}{1.258706in}}%
\pgfpathlineto{\pgfqpoint{2.159904in}{1.255013in}}%
\pgfpathlineto{\pgfqpoint{2.179309in}{1.251367in}}%
\pgfpathlineto{\pgfqpoint{2.198715in}{1.247768in}}%
\pgfpathlineto{\pgfqpoint{2.218120in}{1.244215in}}%
\pgfpathlineto{\pgfqpoint{2.237525in}{1.240707in}}%
\pgfpathlineto{\pgfqpoint{2.254990in}{1.237588in}}%
\pgfpathlineto{\pgfqpoint{2.256736in}{1.237278in}}%
\pgfpathlineto{\pgfqpoint{2.256911in}{1.237247in}}%
\pgfusepath{stroke}%
\end{pgfscope}%
\begin{pgfscope}%
\pgfsetrectcap%
\pgfsetmiterjoin%
\pgfsetlinewidth{0.803000pt}%
\definecolor{currentstroke}{rgb}{0.000000,0.000000,0.000000}%
\pgfsetstrokecolor{currentstroke}%
\pgfsetdash{}{0pt}%
\pgfpathmoveto{\pgfqpoint{0.316407in}{0.548769in}}%
\pgfpathlineto{\pgfqpoint{0.316407in}{2.301955in}}%
\pgfusepath{stroke}%
\end{pgfscope}%
\begin{pgfscope}%
\pgfsetrectcap%
\pgfsetmiterjoin%
\pgfsetlinewidth{0.803000pt}%
\definecolor{currentstroke}{rgb}{0.000000,0.000000,0.000000}%
\pgfsetstrokecolor{currentstroke}%
\pgfsetdash{}{0pt}%
\pgfpathmoveto{\pgfqpoint{2.256930in}{0.548769in}}%
\pgfpathlineto{\pgfqpoint{2.256930in}{2.301955in}}%
\pgfusepath{stroke}%
\end{pgfscope}%
\begin{pgfscope}%
\pgfsetrectcap%
\pgfsetmiterjoin%
\pgfsetlinewidth{0.803000pt}%
\definecolor{currentstroke}{rgb}{0.000000,0.000000,0.000000}%
\pgfsetstrokecolor{currentstroke}%
\pgfsetdash{}{0pt}%
\pgfpathmoveto{\pgfqpoint{0.316407in}{0.548769in}}%
\pgfpathlineto{\pgfqpoint{2.256930in}{0.548769in}}%
\pgfusepath{stroke}%
\end{pgfscope}%
\begin{pgfscope}%
\pgfsetrectcap%
\pgfsetmiterjoin%
\pgfsetlinewidth{0.803000pt}%
\definecolor{currentstroke}{rgb}{0.000000,0.000000,0.000000}%
\pgfsetstrokecolor{currentstroke}%
\pgfsetdash{}{0pt}%
\pgfpathmoveto{\pgfqpoint{0.316407in}{2.301955in}}%
\pgfpathlineto{\pgfqpoint{2.256930in}{2.301955in}}%
\pgfusepath{stroke}%
\end{pgfscope}%
\begin{pgfscope}%
\definecolor{textcolor}{rgb}{0.000000,0.000000,0.000000}%
\pgfsetstrokecolor{textcolor}%
\pgfsetfillcolor{textcolor}%
\pgftext[x=0.859754in,y=1.295197in,left,base]{\color{textcolor}\rmfamily\fontsize{10.000000}{12.000000}\selectfont \(\displaystyle K\)}%
\end{pgfscope}%
\begin{pgfscope}%
\definecolor{textcolor}{rgb}{0.000000,0.000000,0.000000}%
\pgfsetstrokecolor{textcolor}%
\pgfsetfillcolor{textcolor}%
\pgftext[x=0.859754in,y=1.772994in,left,base]{\color{textcolor}\rmfamily\fontsize{10.000000}{12.000000}\selectfont \(\displaystyle K^\prime\)}%
\end{pgfscope}%
\begin{pgfscope}%
\pgfsetbuttcap%
\pgfsetmiterjoin%
\definecolor{currentfill}{rgb}{1.000000,1.000000,1.000000}%
\pgfsetfillcolor{currentfill}%
\pgfsetlinewidth{0.000000pt}%
\definecolor{currentstroke}{rgb}{0.000000,0.000000,0.000000}%
\pgfsetstrokecolor{currentstroke}%
\pgfsetstrokeopacity{0.000000}%
\pgfsetdash{}{0pt}%
\pgfpathmoveto{\pgfqpoint{2.874885in}{0.548769in}}%
\pgfpathlineto{\pgfqpoint{4.815407in}{0.548769in}}%
\pgfpathlineto{\pgfqpoint{4.815407in}{2.301955in}}%
\pgfpathlineto{\pgfqpoint{2.874885in}{2.301955in}}%
\pgfpathlineto{\pgfqpoint{2.874885in}{0.548769in}}%
\pgfpathclose%
\pgfusepath{fill}%
\end{pgfscope}%
\begin{pgfscope}%
\pgfsetbuttcap%
\pgfsetroundjoin%
\definecolor{currentfill}{rgb}{0.000000,0.000000,0.000000}%
\pgfsetfillcolor{currentfill}%
\pgfsetlinewidth{0.803000pt}%
\definecolor{currentstroke}{rgb}{0.000000,0.000000,0.000000}%
\pgfsetstrokecolor{currentstroke}%
\pgfsetdash{}{0pt}%
\pgfsys@defobject{currentmarker}{\pgfqpoint{0.000000in}{-0.048611in}}{\pgfqpoint{0.000000in}{0.000000in}}{%
\pgfpathmoveto{\pgfqpoint{0.000000in}{0.000000in}}%
\pgfpathlineto{\pgfqpoint{0.000000in}{-0.048611in}}%
\pgfusepath{stroke,fill}%
}%
\begin{pgfscope}%
\pgfsys@transformshift{2.874885in}{0.548769in}%
\pgfsys@useobject{currentmarker}{}%
\end{pgfscope}%
\end{pgfscope}%
\begin{pgfscope}%
\definecolor{textcolor}{rgb}{0.000000,0.000000,0.000000}%
\pgfsetstrokecolor{textcolor}%
\pgfsetfillcolor{textcolor}%
\pgftext[x=2.874885in,y=0.451547in,,top]{\color{textcolor}\rmfamily\fontsize{10.000000}{12.000000}\selectfont \(\displaystyle {0}\)}%
\end{pgfscope}%
\begin{pgfscope}%
\pgfsetbuttcap%
\pgfsetroundjoin%
\definecolor{currentfill}{rgb}{0.000000,0.000000,0.000000}%
\pgfsetfillcolor{currentfill}%
\pgfsetlinewidth{0.803000pt}%
\definecolor{currentstroke}{rgb}{0.000000,0.000000,0.000000}%
\pgfsetstrokecolor{currentstroke}%
\pgfsetdash{}{0pt}%
\pgfsys@defobject{currentmarker}{\pgfqpoint{0.000000in}{-0.048611in}}{\pgfqpoint{0.000000in}{0.000000in}}{%
\pgfpathmoveto{\pgfqpoint{0.000000in}{0.000000in}}%
\pgfpathlineto{\pgfqpoint{0.000000in}{-0.048611in}}%
\pgfusepath{stroke,fill}%
}%
\begin{pgfscope}%
\pgfsys@transformshift{3.521726in}{0.548769in}%
\pgfsys@useobject{currentmarker}{}%
\end{pgfscope}%
\end{pgfscope}%
\begin{pgfscope}%
\definecolor{textcolor}{rgb}{0.000000,0.000000,0.000000}%
\pgfsetstrokecolor{textcolor}%
\pgfsetfillcolor{textcolor}%
\pgftext[x=3.521726in,y=0.451547in,,top]{\color{textcolor}\rmfamily\fontsize{10.000000}{12.000000}\selectfont \(\displaystyle {2}\)}%
\end{pgfscope}%
\begin{pgfscope}%
\pgfsetbuttcap%
\pgfsetroundjoin%
\definecolor{currentfill}{rgb}{0.000000,0.000000,0.000000}%
\pgfsetfillcolor{currentfill}%
\pgfsetlinewidth{0.803000pt}%
\definecolor{currentstroke}{rgb}{0.000000,0.000000,0.000000}%
\pgfsetstrokecolor{currentstroke}%
\pgfsetdash{}{0pt}%
\pgfsys@defobject{currentmarker}{\pgfqpoint{0.000000in}{-0.048611in}}{\pgfqpoint{0.000000in}{0.000000in}}{%
\pgfpathmoveto{\pgfqpoint{0.000000in}{0.000000in}}%
\pgfpathlineto{\pgfqpoint{0.000000in}{-0.048611in}}%
\pgfusepath{stroke,fill}%
}%
\begin{pgfscope}%
\pgfsys@transformshift{4.168566in}{0.548769in}%
\pgfsys@useobject{currentmarker}{}%
\end{pgfscope}%
\end{pgfscope}%
\begin{pgfscope}%
\definecolor{textcolor}{rgb}{0.000000,0.000000,0.000000}%
\pgfsetstrokecolor{textcolor}%
\pgfsetfillcolor{textcolor}%
\pgftext[x=4.168566in,y=0.451547in,,top]{\color{textcolor}\rmfamily\fontsize{10.000000}{12.000000}\selectfont \(\displaystyle {4}\)}%
\end{pgfscope}%
\begin{pgfscope}%
\pgfsetbuttcap%
\pgfsetroundjoin%
\definecolor{currentfill}{rgb}{0.000000,0.000000,0.000000}%
\pgfsetfillcolor{currentfill}%
\pgfsetlinewidth{0.803000pt}%
\definecolor{currentstroke}{rgb}{0.000000,0.000000,0.000000}%
\pgfsetstrokecolor{currentstroke}%
\pgfsetdash{}{0pt}%
\pgfsys@defobject{currentmarker}{\pgfqpoint{0.000000in}{-0.048611in}}{\pgfqpoint{0.000000in}{0.000000in}}{%
\pgfpathmoveto{\pgfqpoint{0.000000in}{0.000000in}}%
\pgfpathlineto{\pgfqpoint{0.000000in}{-0.048611in}}%
\pgfusepath{stroke,fill}%
}%
\begin{pgfscope}%
\pgfsys@transformshift{4.815407in}{0.548769in}%
\pgfsys@useobject{currentmarker}{}%
\end{pgfscope}%
\end{pgfscope}%
\begin{pgfscope}%
\definecolor{textcolor}{rgb}{0.000000,0.000000,0.000000}%
\pgfsetstrokecolor{textcolor}%
\pgfsetfillcolor{textcolor}%
\pgftext[x=4.815407in,y=0.451547in,,top]{\color{textcolor}\rmfamily\fontsize{10.000000}{12.000000}\selectfont \(\displaystyle {6}\)}%
\end{pgfscope}%
\begin{pgfscope}%
\definecolor{textcolor}{rgb}{0.000000,0.000000,0.000000}%
\pgfsetstrokecolor{textcolor}%
\pgfsetfillcolor{textcolor}%
\pgftext[x=3.845146in,y=0.272534in,,top]{\color{textcolor}\rmfamily\fontsize{10.000000}{12.000000}\selectfont \(\displaystyle K\)}%
\end{pgfscope}%
\begin{pgfscope}%
\pgfsetbuttcap%
\pgfsetroundjoin%
\definecolor{currentfill}{rgb}{0.000000,0.000000,0.000000}%
\pgfsetfillcolor{currentfill}%
\pgfsetlinewidth{0.803000pt}%
\definecolor{currentstroke}{rgb}{0.000000,0.000000,0.000000}%
\pgfsetstrokecolor{currentstroke}%
\pgfsetdash{}{0pt}%
\pgfsys@defobject{currentmarker}{\pgfqpoint{-0.048611in}{0.000000in}}{\pgfqpoint{-0.000000in}{0.000000in}}{%
\pgfpathmoveto{\pgfqpoint{-0.000000in}{0.000000in}}%
\pgfpathlineto{\pgfqpoint{-0.048611in}{0.000000in}}%
\pgfusepath{stroke,fill}%
}%
\begin{pgfscope}%
\pgfsys@transformshift{2.874885in}{0.548769in}%
\pgfsys@useobject{currentmarker}{}%
\end{pgfscope}%
\end{pgfscope}%
\begin{pgfscope}%
\definecolor{textcolor}{rgb}{0.000000,0.000000,0.000000}%
\pgfsetstrokecolor{textcolor}%
\pgfsetfillcolor{textcolor}%
\pgftext[x=2.708218in, y=0.500544in, left, base]{\color{textcolor}\rmfamily\fontsize{10.000000}{12.000000}\selectfont \(\displaystyle {0}\)}%
\end{pgfscope}%
\begin{pgfscope}%
\pgfsetbuttcap%
\pgfsetroundjoin%
\definecolor{currentfill}{rgb}{0.000000,0.000000,0.000000}%
\pgfsetfillcolor{currentfill}%
\pgfsetlinewidth{0.803000pt}%
\definecolor{currentstroke}{rgb}{0.000000,0.000000,0.000000}%
\pgfsetstrokecolor{currentstroke}%
\pgfsetdash{}{0pt}%
\pgfsys@defobject{currentmarker}{\pgfqpoint{-0.048611in}{0.000000in}}{\pgfqpoint{-0.000000in}{0.000000in}}{%
\pgfpathmoveto{\pgfqpoint{-0.000000in}{0.000000in}}%
\pgfpathlineto{\pgfqpoint{-0.048611in}{0.000000in}}%
\pgfusepath{stroke,fill}%
}%
\begin{pgfscope}%
\pgfsys@transformshift{2.874885in}{0.899406in}%
\pgfsys@useobject{currentmarker}{}%
\end{pgfscope}%
\end{pgfscope}%
\begin{pgfscope}%
\definecolor{textcolor}{rgb}{0.000000,0.000000,0.000000}%
\pgfsetstrokecolor{textcolor}%
\pgfsetfillcolor{textcolor}%
\pgftext[x=2.708218in, y=0.851181in, left, base]{\color{textcolor}\rmfamily\fontsize{10.000000}{12.000000}\selectfont \(\displaystyle {1}\)}%
\end{pgfscope}%
\begin{pgfscope}%
\pgfsetbuttcap%
\pgfsetroundjoin%
\definecolor{currentfill}{rgb}{0.000000,0.000000,0.000000}%
\pgfsetfillcolor{currentfill}%
\pgfsetlinewidth{0.803000pt}%
\definecolor{currentstroke}{rgb}{0.000000,0.000000,0.000000}%
\pgfsetstrokecolor{currentstroke}%
\pgfsetdash{}{0pt}%
\pgfsys@defobject{currentmarker}{\pgfqpoint{-0.048611in}{0.000000in}}{\pgfqpoint{-0.000000in}{0.000000in}}{%
\pgfpathmoveto{\pgfqpoint{-0.000000in}{0.000000in}}%
\pgfpathlineto{\pgfqpoint{-0.048611in}{0.000000in}}%
\pgfusepath{stroke,fill}%
}%
\begin{pgfscope}%
\pgfsys@transformshift{2.874885in}{1.250043in}%
\pgfsys@useobject{currentmarker}{}%
\end{pgfscope}%
\end{pgfscope}%
\begin{pgfscope}%
\definecolor{textcolor}{rgb}{0.000000,0.000000,0.000000}%
\pgfsetstrokecolor{textcolor}%
\pgfsetfillcolor{textcolor}%
\pgftext[x=2.708218in, y=1.201818in, left, base]{\color{textcolor}\rmfamily\fontsize{10.000000}{12.000000}\selectfont \(\displaystyle {2}\)}%
\end{pgfscope}%
\begin{pgfscope}%
\pgfsetbuttcap%
\pgfsetroundjoin%
\definecolor{currentfill}{rgb}{0.000000,0.000000,0.000000}%
\pgfsetfillcolor{currentfill}%
\pgfsetlinewidth{0.803000pt}%
\definecolor{currentstroke}{rgb}{0.000000,0.000000,0.000000}%
\pgfsetstrokecolor{currentstroke}%
\pgfsetdash{}{0pt}%
\pgfsys@defobject{currentmarker}{\pgfqpoint{-0.048611in}{0.000000in}}{\pgfqpoint{-0.000000in}{0.000000in}}{%
\pgfpathmoveto{\pgfqpoint{-0.000000in}{0.000000in}}%
\pgfpathlineto{\pgfqpoint{-0.048611in}{0.000000in}}%
\pgfusepath{stroke,fill}%
}%
\begin{pgfscope}%
\pgfsys@transformshift{2.874885in}{1.600680in}%
\pgfsys@useobject{currentmarker}{}%
\end{pgfscope}%
\end{pgfscope}%
\begin{pgfscope}%
\definecolor{textcolor}{rgb}{0.000000,0.000000,0.000000}%
\pgfsetstrokecolor{textcolor}%
\pgfsetfillcolor{textcolor}%
\pgftext[x=2.708218in, y=1.552455in, left, base]{\color{textcolor}\rmfamily\fontsize{10.000000}{12.000000}\selectfont \(\displaystyle {3}\)}%
\end{pgfscope}%
\begin{pgfscope}%
\pgfsetbuttcap%
\pgfsetroundjoin%
\definecolor{currentfill}{rgb}{0.000000,0.000000,0.000000}%
\pgfsetfillcolor{currentfill}%
\pgfsetlinewidth{0.803000pt}%
\definecolor{currentstroke}{rgb}{0.000000,0.000000,0.000000}%
\pgfsetstrokecolor{currentstroke}%
\pgfsetdash{}{0pt}%
\pgfsys@defobject{currentmarker}{\pgfqpoint{-0.048611in}{0.000000in}}{\pgfqpoint{-0.000000in}{0.000000in}}{%
\pgfpathmoveto{\pgfqpoint{-0.000000in}{0.000000in}}%
\pgfpathlineto{\pgfqpoint{-0.048611in}{0.000000in}}%
\pgfusepath{stroke,fill}%
}%
\begin{pgfscope}%
\pgfsys@transformshift{2.874885in}{1.951318in}%
\pgfsys@useobject{currentmarker}{}%
\end{pgfscope}%
\end{pgfscope}%
\begin{pgfscope}%
\definecolor{textcolor}{rgb}{0.000000,0.000000,0.000000}%
\pgfsetstrokecolor{textcolor}%
\pgfsetfillcolor{textcolor}%
\pgftext[x=2.708218in, y=1.903092in, left, base]{\color{textcolor}\rmfamily\fontsize{10.000000}{12.000000}\selectfont \(\displaystyle {4}\)}%
\end{pgfscope}%
\begin{pgfscope}%
\pgfsetbuttcap%
\pgfsetroundjoin%
\definecolor{currentfill}{rgb}{0.000000,0.000000,0.000000}%
\pgfsetfillcolor{currentfill}%
\pgfsetlinewidth{0.803000pt}%
\definecolor{currentstroke}{rgb}{0.000000,0.000000,0.000000}%
\pgfsetstrokecolor{currentstroke}%
\pgfsetdash{}{0pt}%
\pgfsys@defobject{currentmarker}{\pgfqpoint{-0.048611in}{0.000000in}}{\pgfqpoint{-0.000000in}{0.000000in}}{%
\pgfpathmoveto{\pgfqpoint{-0.000000in}{0.000000in}}%
\pgfpathlineto{\pgfqpoint{-0.048611in}{0.000000in}}%
\pgfusepath{stroke,fill}%
}%
\begin{pgfscope}%
\pgfsys@transformshift{2.874885in}{2.301955in}%
\pgfsys@useobject{currentmarker}{}%
\end{pgfscope}%
\end{pgfscope}%
\begin{pgfscope}%
\definecolor{textcolor}{rgb}{0.000000,0.000000,0.000000}%
\pgfsetstrokecolor{textcolor}%
\pgfsetfillcolor{textcolor}%
\pgftext[x=2.708218in, y=2.253730in, left, base]{\color{textcolor}\rmfamily\fontsize{10.000000}{12.000000}\selectfont \(\displaystyle {5}\)}%
\end{pgfscope}%
\begin{pgfscope}%
\definecolor{textcolor}{rgb}{0.000000,0.000000,0.000000}%
\pgfsetstrokecolor{textcolor}%
\pgfsetfillcolor{textcolor}%
\pgftext[x=2.652662in,y=1.425362in,,bottom,rotate=90.000000]{\color{textcolor}\rmfamily\fontsize{10.000000}{12.000000}\selectfont \(\displaystyle K^\prime\)}%
\end{pgfscope}%
\begin{pgfscope}%
\pgfpathrectangle{\pgfqpoint{2.874885in}{0.548769in}}{\pgfqpoint{1.940523in}{1.753186in}}%
\pgfusepath{clip}%
\pgfsetrectcap%
\pgfsetroundjoin%
\pgfsetlinewidth{0.501875pt}%
\definecolor{currentstroke}{rgb}{0.501961,0.501961,0.501961}%
\pgfsetstrokecolor{currentstroke}%
\pgfsetdash{}{0pt}%
\pgfpathmoveto{\pgfqpoint{3.382912in}{0.548769in}}%
\pgfpathlineto{\pgfqpoint{3.382912in}{2.301955in}}%
\pgfusepath{stroke}%
\end{pgfscope}%
\begin{pgfscope}%
\pgfpathrectangle{\pgfqpoint{2.874885in}{0.548769in}}{\pgfqpoint{1.940523in}{1.753186in}}%
\pgfusepath{clip}%
\pgfsetrectcap%
\pgfsetroundjoin%
\pgfsetlinewidth{0.501875pt}%
\definecolor{currentstroke}{rgb}{0.501961,0.501961,0.501961}%
\pgfsetstrokecolor{currentstroke}%
\pgfsetdash{}{0pt}%
\pgfpathmoveto{\pgfqpoint{2.874885in}{1.099548in}}%
\pgfpathlineto{\pgfqpoint{4.815407in}{1.099548in}}%
\pgfusepath{stroke}%
\end{pgfscope}%
\begin{pgfscope}%
\pgfpathrectangle{\pgfqpoint{2.874885in}{0.548769in}}{\pgfqpoint{1.940523in}{1.753186in}}%
\pgfusepath{clip}%
\pgfsetrectcap%
\pgfsetroundjoin%
\pgfsetlinewidth{1.003750pt}%
\definecolor{currentstroke}{rgb}{0.121569,0.466667,0.705882}%
\pgfsetstrokecolor{currentstroke}%
\pgfsetdash{}{0pt}%
\pgfpathmoveto{\pgfqpoint{3.383004in}{2.315844in}}%
\pgfpathlineto{\pgfqpoint{3.383027in}{2.264692in}}%
\pgfpathlineto{\pgfqpoint{3.383116in}{2.164019in}}%
\pgfpathlineto{\pgfqpoint{3.383230in}{2.086013in}}%
\pgfpathlineto{\pgfqpoint{3.383370in}{2.022353in}}%
\pgfpathlineto{\pgfqpoint{3.383536in}{1.968602in}}%
\pgfpathlineto{\pgfqpoint{3.383728in}{1.922109in}}%
\pgfpathlineto{\pgfqpoint{3.383946in}{1.881163in}}%
\pgfpathlineto{\pgfqpoint{3.384190in}{1.844597in}}%
\pgfpathlineto{\pgfqpoint{3.384460in}{1.811577in}}%
\pgfpathlineto{\pgfqpoint{3.384756in}{1.781487in}}%
\pgfpathlineto{\pgfqpoint{3.385079in}{1.753859in}}%
\pgfpathlineto{\pgfqpoint{3.385429in}{1.728331in}}%
\pgfpathlineto{\pgfqpoint{3.385807in}{1.704613in}}%
\pgfpathlineto{\pgfqpoint{3.386211in}{1.682473in}}%
\pgfpathlineto{\pgfqpoint{3.386644in}{1.661720in}}%
\pgfpathlineto{\pgfqpoint{3.387104in}{1.642197in}}%
\pgfpathlineto{\pgfqpoint{3.387593in}{1.623771in}}%
\pgfpathlineto{\pgfqpoint{3.388110in}{1.606330in}}%
\pgfpathlineto{\pgfqpoint{3.388657in}{1.589779in}}%
\pgfpathlineto{\pgfqpoint{3.389233in}{1.574035in}}%
\pgfpathlineto{\pgfqpoint{3.389839in}{1.559026in}}%
\pgfpathlineto{\pgfqpoint{3.390475in}{1.544692in}}%
\pgfpathlineto{\pgfqpoint{3.391142in}{1.530976in}}%
\pgfpathlineto{\pgfqpoint{3.391841in}{1.517831in}}%
\pgfpathlineto{\pgfqpoint{3.392571in}{1.505213in}}%
\pgfpathlineto{\pgfqpoint{3.393334in}{1.493085in}}%
\pgfpathlineto{\pgfqpoint{3.394130in}{1.481412in}}%
\pgfpathlineto{\pgfqpoint{3.394960in}{1.470164in}}%
\pgfpathlineto{\pgfqpoint{3.395825in}{1.459313in}}%
\pgfpathlineto{\pgfqpoint{3.396725in}{1.448833in}}%
\pgfpathlineto{\pgfqpoint{3.397661in}{1.438702in}}%
\pgfpathlineto{\pgfqpoint{3.398633in}{1.428899in}}%
\pgfpathlineto{\pgfqpoint{3.399643in}{1.419406in}}%
\pgfpathlineto{\pgfqpoint{3.400692in}{1.410204in}}%
\pgfpathlineto{\pgfqpoint{3.401781in}{1.401278in}}%
\pgfpathlineto{\pgfqpoint{3.402910in}{1.392614in}}%
\pgfpathlineto{\pgfqpoint{3.404081in}{1.384197in}}%
\pgfpathlineto{\pgfqpoint{3.405294in}{1.376014in}}%
\pgfpathlineto{\pgfqpoint{3.406552in}{1.368056in}}%
\pgfpathlineto{\pgfqpoint{3.407855in}{1.360310in}}%
\pgfpathlineto{\pgfqpoint{3.409204in}{1.352766in}}%
\pgfpathlineto{\pgfqpoint{3.410602in}{1.345416in}}%
\pgfpathlineto{\pgfqpoint{3.412049in}{1.338250in}}%
\pgfpathlineto{\pgfqpoint{3.413548in}{1.331261in}}%
\pgfpathlineto{\pgfqpoint{3.415099in}{1.324441in}}%
\pgfpathlineto{\pgfqpoint{3.416706in}{1.317782in}}%
\pgfpathlineto{\pgfqpoint{3.418369in}{1.311279in}}%
\pgfpathlineto{\pgfqpoint{3.420091in}{1.304923in}}%
\pgfpathlineto{\pgfqpoint{3.421874in}{1.298711in}}%
\pgfpathlineto{\pgfqpoint{3.423720in}{1.292636in}}%
\pgfpathlineto{\pgfqpoint{3.425632in}{1.286693in}}%
\pgfpathlineto{\pgfqpoint{3.427613in}{1.280876in}}%
\pgfpathlineto{\pgfqpoint{3.429665in}{1.275182in}}%
\pgfpathlineto{\pgfqpoint{3.431792in}{1.269606in}}%
\pgfpathlineto{\pgfqpoint{3.433997in}{1.264143in}}%
\pgfpathlineto{\pgfqpoint{3.436283in}{1.258789in}}%
\pgfpathlineto{\pgfqpoint{3.438654in}{1.253542in}}%
\pgfpathlineto{\pgfqpoint{3.441114in}{1.248396in}}%
\pgfpathlineto{\pgfqpoint{3.443668in}{1.243349in}}%
\pgfpathlineto{\pgfqpoint{3.446321in}{1.238398in}}%
\pgfpathlineto{\pgfqpoint{3.449077in}{1.233539in}}%
\pgfpathlineto{\pgfqpoint{3.451943in}{1.228770in}}%
\pgfpathlineto{\pgfqpoint{3.454924in}{1.224087in}}%
\pgfpathlineto{\pgfqpoint{3.458028in}{1.219487in}}%
\pgfpathlineto{\pgfqpoint{3.461261in}{1.214970in}}%
\pgfpathlineto{\pgfqpoint{3.464631in}{1.210531in}}%
\pgfpathlineto{\pgfqpoint{3.468147in}{1.206168in}}%
\pgfpathlineto{\pgfqpoint{3.471820in}{1.201880in}}%
\pgfpathlineto{\pgfqpoint{3.475659in}{1.197664in}}%
\pgfpathlineto{\pgfqpoint{3.479676in}{1.193518in}}%
\pgfpathlineto{\pgfqpoint{3.483885in}{1.189440in}}%
\pgfpathlineto{\pgfqpoint{3.488300in}{1.185428in}}%
\pgfpathlineto{\pgfqpoint{3.492938in}{1.181480in}}%
\pgfpathlineto{\pgfqpoint{3.497817in}{1.177595in}}%
\pgfpathlineto{\pgfqpoint{3.502957in}{1.173771in}}%
\pgfpathlineto{\pgfqpoint{3.508384in}{1.170006in}}%
\pgfpathlineto{\pgfqpoint{3.514123in}{1.166299in}}%
\pgfpathlineto{\pgfqpoint{3.520206in}{1.162648in}}%
\pgfpathlineto{\pgfqpoint{3.526670in}{1.159052in}}%
\pgfpathlineto{\pgfqpoint{3.533555in}{1.155509in}}%
\pgfpathlineto{\pgfqpoint{3.540911in}{1.152019in}}%
\pgfpathlineto{\pgfqpoint{3.548796in}{1.148579in}}%
\pgfpathlineto{\pgfqpoint{3.557281in}{1.145189in}}%
\pgfpathlineto{\pgfqpoint{3.566449in}{1.141846in}}%
\pgfpathlineto{\pgfqpoint{3.576405in}{1.138552in}}%
\pgfpathlineto{\pgfqpoint{3.587275in}{1.135303in}}%
\pgfpathlineto{\pgfqpoint{3.599224in}{1.132099in}}%
\pgfpathlineto{\pgfqpoint{3.612461in}{1.128939in}}%
\pgfpathlineto{\pgfqpoint{3.627261in}{1.125822in}}%
\pgfpathlineto{\pgfqpoint{3.644002in}{1.122747in}}%
\pgfpathlineto{\pgfqpoint{3.663208in}{1.119713in}}%
\pgfpathlineto{\pgfqpoint{3.685656in}{1.116719in}}%
\pgfpathlineto{\pgfqpoint{3.712547in}{1.113764in}}%
\pgfpathlineto{\pgfqpoint{3.745902in}{1.110847in}}%
\pgfpathlineto{\pgfqpoint{3.789516in}{1.107968in}}%
\pgfpathlineto{\pgfqpoint{3.851932in}{1.105126in}}%
\pgfpathlineto{\pgfqpoint{3.960478in}{1.102320in}}%
\pgfpathlineto{\pgfqpoint{4.328852in}{1.099824in}}%
\pgfpathlineto{\pgfqpoint{4.700641in}{1.099576in}}%
\pgfpathlineto{\pgfqpoint{4.829296in}{1.099567in}}%
\pgfusepath{stroke}%
\end{pgfscope}%
\begin{pgfscope}%
\pgfsetrectcap%
\pgfsetmiterjoin%
\pgfsetlinewidth{0.803000pt}%
\definecolor{currentstroke}{rgb}{0.000000,0.000000,0.000000}%
\pgfsetstrokecolor{currentstroke}%
\pgfsetdash{}{0pt}%
\pgfpathmoveto{\pgfqpoint{2.874885in}{0.548769in}}%
\pgfpathlineto{\pgfqpoint{2.874885in}{2.301955in}}%
\pgfusepath{stroke}%
\end{pgfscope}%
\begin{pgfscope}%
\pgfsetrectcap%
\pgfsetmiterjoin%
\pgfsetlinewidth{0.803000pt}%
\definecolor{currentstroke}{rgb}{0.000000,0.000000,0.000000}%
\pgfsetstrokecolor{currentstroke}%
\pgfsetdash{}{0pt}%
\pgfpathmoveto{\pgfqpoint{4.815407in}{0.548769in}}%
\pgfpathlineto{\pgfqpoint{4.815407in}{2.301955in}}%
\pgfusepath{stroke}%
\end{pgfscope}%
\begin{pgfscope}%
\pgfsetrectcap%
\pgfsetmiterjoin%
\pgfsetlinewidth{0.803000pt}%
\definecolor{currentstroke}{rgb}{0.000000,0.000000,0.000000}%
\pgfsetstrokecolor{currentstroke}%
\pgfsetdash{}{0pt}%
\pgfpathmoveto{\pgfqpoint{2.874885in}{0.548769in}}%
\pgfpathlineto{\pgfqpoint{4.815407in}{0.548769in}}%
\pgfusepath{stroke}%
\end{pgfscope}%
\begin{pgfscope}%
\pgfsetrectcap%
\pgfsetmiterjoin%
\pgfsetlinewidth{0.803000pt}%
\definecolor{currentstroke}{rgb}{0.000000,0.000000,0.000000}%
\pgfsetstrokecolor{currentstroke}%
\pgfsetdash{}{0pt}%
\pgfpathmoveto{\pgfqpoint{2.874885in}{2.301955in}}%
\pgfpathlineto{\pgfqpoint{4.815407in}{2.301955in}}%
\pgfusepath{stroke}%
\end{pgfscope}%
\begin{pgfscope}%
\definecolor{textcolor}{rgb}{0.000000,0.000000,0.000000}%
\pgfsetstrokecolor{textcolor}%
\pgfsetfillcolor{textcolor}%
\pgftext[x=2.907227in,y=1.134612in,left,base]{\color{textcolor}\rmfamily\fontsize{10.000000}{12.000000}\selectfont \(\displaystyle \pi/2\)}%
\end{pgfscope}%
\begin{pgfscope}%
\definecolor{textcolor}{rgb}{0.000000,0.000000,0.000000}%
\pgfsetstrokecolor{textcolor}%
\pgfsetfillcolor{textcolor}%
\pgftext[x=3.415254in,y=0.583833in,left,base]{\color{textcolor}\rmfamily\fontsize{10.000000}{12.000000}\selectfont \(\displaystyle \pi/2\)}%
\end{pgfscope}%
\end{pgfpicture}%
\makeatother%
\endgroup%

    \caption{Die Periodizitäten in realer und imaginärer Richtung in Abhängigkeit vom elliptischen Modul $k$.}
\end{figure}

\begin{figure}
    \centering
    
\def\d{0.2}
\def\n{3}
\def\nn{2}
\def\a{2.5}

\begin{tikzpicture}[>=stealth', auto, node distance=2cm, scale=1.2]

    \tikzstyle{zero} = [draw, circle, inner sep =0, minimum height=0.15cm]
    \tikzstyle{dot} = [fill, circle, inner sep =0, minimum height=0.1cm]

    \tikzset{pole/.style={cross out, draw=black, minimum size=(0.15cm-\pgflinewidth), inner sep=0pt, outer sep=0pt}}

    \begin{scope}[xscale=3, yscale=3]

    \begin{scope}[]

        \fill[orange!30, scale=1.735] (0,0) rectangle (\d*\a+0.5, \d/\a+0.5);
        \fill[yellow!30] (0,0) rectangle (\d*\a+0.5, \d/\a+0.5);

        \begin{scope}[scale=1.735, red]
            \draw (0,0) rectangle (\d*\a/\n+0.5/\n, \d/\a+0.5);
            \draw[gray] (0,0) -- (\d*\a/\n+0.5/\n, \d/\a+0.5);

            \node[zero] at ( \d*\a/\n+0.5/\n, \d/\a+0.5) {};
            \node[pole, color=red] at ( \d*\a/\n+0.5/\n, 0) {};


            \draw[] ( \d*\a/\n+0.5/\n,0)  node[anchor=north] {\small $K_1$};
            \draw[]  (0, \d/\a+0.5)  node[anchor=east]{\small $jK_1^\prime$};

        \end{scope}

        \begin{scope}[blue]
            \draw[] (0,0) rectangle (\d*\a+0.5, \d/\a+0.5);
            \foreach \i in {1,...,\nn} {
                \draw[gray, dotted] (\i*\d*\a/\n+\i*0.5/\n, 0) -- (\i*\d*\a/\n+\i*0.5/\n, \d/\a+0.5);
            }

            \node[zero] at ( \d*\a+0.5, \d/\a+0.5) {};
            \node[pole, color=blue] at ( \d*\a+0.5, 0) {};

            \draw[] ( \d*\a+0.5,0)  node[anchor=north] {\small $K$};
            \draw[]  (0, \d/\a+0.5)  node[anchor=east]{\small $jK^\prime$};

            \node[dot, gray] at (\d*\a/\n+0.5/\n, \d/\a+0.5) {};
            \node[above] at (0.5*\d*\a/\n+0.5*0.5/\n, \d/\a+0.5) {\small $K/N$};

        \end{scope}

        \draw[thick, gray, ->] (0,-0.25) -- (0,1.25) node[anchor=south]{$\mathrm{Im}$};
        \draw[thick, gray, ->] (-0.25,0) -- (2,0) node[anchor=west]{$\mathrm{Re}$};

        \begin{scope}[]
            \clip(0,0) rectangle (2,1.25);
            \draw[scale=1, domain=0.1:10,  variable=\x, smooth, samples=200] plot ({\d*\x1+0.5}, {\d/\x+0.5});

        \end{scope}
    \end{scope}


\end{scope}

\end{tikzpicture}

    \caption{Die Gradgleichung als geometrisches Problem.}
\end{figure}



\subsection{Polynome?}

Bei den Tschebyscheff-Polynomen haben wir gesehen, dass die Trigonometrische Formel zu einfachen Polynomen umgewandelt werden kann.
Im gegensatz zum $\cos^{-1}$ hat der $\cd^{-1}$ nicht nur Nullstellen sondern auch Pole.
Somit entstehen bei den elliptischen rationalen Funktionen, wie es der name auch deutet, rationale Funktionen, also ein Bruch von zwei Polynomen.

Da Transformationen einer rationalen Funktionen mit Grundrechenarten, wie es in \eqref{ellfilter:eq:h_omega} der Fall ist, immer noch rationale Funktionen ergeben, stellt dies kein Problem für die Implementierung dar.
