\section{Elliptische rationale Funktionen}

Kommen wir nun zum eigentlichen Teil dieses Papers, den elliptischen rationalen Funktionen
\begin{align}
    R_N(\xi, w) &= \cd \left(N~f_1(\xi)~\cd^{-1}(w, 1/\xi), f_2(\xi)\right) \\
                &= \cd \left(N~\frac{K_1}{K}~\cd^{-1}(w, k), k_1)\right) , \quad k= 1/\xi, k_1 = 1/f(\xi) \\
                &= \cd \left(N~K_1~z , k_1 \right), \quad w= \cd(z K, k)
\end{align}


sieht ähnlich aus wie die trigonometrische Darstellung der Tschebyschef-Polynome \eqref{ellfilter:eq:chebychef_polynomials}
Anstelle vom Kosinus kommt hier die $\cd$-Funktion zum Einsatz.
Die Ordnungszahl $N$ kommt auch als Faktor for.
Zusätzlich werden noch zwei verschiedene elliptische Module $k$ und $k_1$ gebraucht.



Sinus entspricht $\sn$

Damit die Nullstellen an ähnlichen Positionen zu liegen kommen wie bei den Tschebyscheff-Polynomen, muss die $\cd$-Funktion gewählt werden.

Die $\cd^{-1}(w, k)$-Funktion ist um $K$ verschoben zur $\sn^{-1}(w, k)$-Funktion, wie ersichtlich in Abbildung \ref{ellfilter:fig:cd}.
\begin{figure}
    \centering
    \begin{tikzpicture}[>=stealth', auto, node distance=2cm, scale=1.2, thick]

    \tikzstyle{zero} = [draw, circle, inner sep =0, minimum height=0.15cm]

    \tikzset{pole/.style={cross out, draw=black, minimum size=(0.15cm-\pgflinewidth), inner sep=0pt, outer sep=0pt}}

    \begin{scope}[xscale=0.9, yscale=1.8]

        \draw[gray, ->] (0,-1.5) -- (0,1.5) node[anchor=south]{$\mathrm{Im}~z$};
        \draw[gray, ->] (-5,0) -- (5,0) node[anchor=west]{$\mathrm{Re}~z$};


        \begin{scope}[xshift=0cm]

            \clip(-4.5,-1.25) rectangle (4.5,1.25);

            \fill[yellow!30] (0,0) rectangle (1, 0.5);

            \foreach \i in {-2,...,1} {
                \foreach \j in {-2,...,1} {
                    \begin{scope}[xshift=\i*4cm, yshift=\j*1cm]
                        \draw[->, thick, orange!50] (0, 0) -- (0,0.5);
                        \draw[->, thick, darkgreen!50] (1, 0) -- (0,0);
                        \draw[->, thick, cyan!50] (2, 0) -- (1,0);
                        \draw[->, thick, blue!50] (2,0.5) -- (2, 0);
                        \draw[->, thick, purple!50] (1, 0.5) -- (2,0.5);
                        \draw[->, thick, red!50] (0, 0.5) -- (1,0.5);
                        \draw[->, thick, orange!50] (0,1) -- (0,0.5);
                        \draw[->, thick, blue!50] (2,0.5) -- (2, 1);
                        \draw[->, thick, purple!50] (3, 0.5) -- (2,0.5);
                        \draw[->, thick, red!50] (4, 0.5) -- (3,0.5);
                        \draw[->, thick, cyan!50] (2, 0) -- (3,0);
                        \draw[->, thick, darkgreen!50] (3, 0) -- (4,0);
                    \end{scope}
                }
            }

            \draw[ultra thick, ->, orange] (0, 0) -- (0,0.5);
            \draw[ultra thick, ->, darkgreen] (1, 0) -- (0,0);
            \draw[ultra thick, ->, cyan] (2, 0) -- (1,0);
            \draw[ultra thick, ->, blue] (2,0.5) -- (2, 0);
            \draw[ultra thick, ->, purple] (1, 0.5) -- (2,0.5);
            \draw[ultra thick, ->, red] (0, 0.5) -- (1,0.5);

            \foreach \i in {-2,...,1} {
                \foreach \j in {-2,...,1} {
                    \begin{scope}[xshift=\i*4cm, yshift=\j*1cm]
                        \node[zero] at ( 1, 0) {};
                        \node[zero] at ( 3, 0) {};
                        \node[pole] at ( 1,0.5) {};
                        \node[pole] at ( 3,0.5) {};

                    \end{scope}
                }
            }

        \end{scope}

        \draw[gray] ( 1,0) +(0,0.05) -- +(0, -0.05) node[inner sep=0, anchor=north west] {\small $K$};
        \draw[gray]  (0, 0.5) +(0.1, 0) -- +(-0.1, 0) node[inner sep=0, anchor=south east]{\small $jK^\prime$};

    \end{scope}


    \node[zero] at (4,3) (n) {};
    \node[anchor=west] at (n.east) {Nullstelle};
    \node[pole, below=0.25cm of n] (n) {};
    \node[anchor=west] at (n.east) {Polstelle};

    \begin{scope}[yshift=-4cm, xscale=0.75]

        \draw[gray, ->] (-6,0) -- (6,0) node[anchor=west]{$w$};

        \draw[ultra thick, ->, purple] (-5, 0) -- (-3, 0);
        \draw[ultra thick, ->, blue]      (-3, 0) -- (-2, 0);
        \draw[ultra thick, ->, cyan]       (-2, 0) -- (0, 0);
        \draw[ultra thick, ->, darkgreen]    (0, 0) -- (2, 0);
        \draw[ultra thick, ->, orange] (2, 0) -- (3, 0);
        \draw[ultra thick, ->, red] (3, 0) -- (5, 0);

        \node[anchor=south] at (-5,0) {$-\infty$};
        \node[anchor=south] at (-3,0) {$-1/k$};
        \node[anchor=south] at (-2,0) {$-1$};
        \node[anchor=south] at (0,0) {$0$};
        \node[anchor=south] at (2,0) {$1$};
        \node[anchor=south] at (3,0) {$1/k$};
        \node[anchor=south] at (5,0) {$\infty$};

    \end{scope}

\end{tikzpicture}
    \caption{
        $z$-Ebene der Funktion $z = \sn^{-1}(w, k)$.
        Die Funktion ist in der realen Achse $4K$-periodisch und in der imaginären Achse $2jK^\prime$-periodisch.
    }
    \label{ellfilter:fig:cd}
\end{figure}
Auffallend ist, dass sich alle Nullstellen und Polstellen um $K$ verschoben haben.

Durch das Konzept vom fundamentalen Rechteck, siehe Abbildung \ref{ellfilter:fig:fundamental_rectangle} können für alle inversen Jaccobi elliptischen Funktionen die Positionen der Null- und Polstellen anhand eines Diagramms ermittelt werden.
Der erste Buchstabe bestimmt die Position der Nullstelle und der zweite Buchstabe die Polstelle.
\begin{figure}
    \centering
    \begin{tikzpicture}[>=stealth', auto, node distance=2cm, scale=1.2]

    \tikzstyle{zero} = [draw, circle, inner sep =0, minimum height=0.15cm]

    \tikzset{pole/.style={cross out, draw=black, minimum size=(0.15cm-\pgflinewidth), inner sep=0pt, outer sep=0pt}}

    \begin{scope}[xscale=2, yscale=2]

        \draw[gray, ->] (0,-0.25) -- (0,1.25) node[anchor=south]{$\mathrm{Im}~z$};
        \draw[gray, ->] (-0.25,0) -- (1.5,0) node[anchor=west]{$\mathrm{Re}~z$};

        \draw[gray] ( 1,0) +(0,0.05) -- +(0, -0.05) node[inner sep=0, anchor=north] {\small $K$};

        \draw[gray]  (0, 1) +(0.05, 0) -- +(-0.05, 0) node[inner sep=0, anchor=east]{\small $jK^\prime$};

        \fill[yellow!50] (0,0) rectangle (1, 1);

        \node[anchor=south east] at ( 1,0) {$c$};
        \node[anchor=north east] at ( 1,1) {$d$};
        \node[anchor=north west] at ( 0,1) {$n$};
        \node[anchor=south west] at ( 0,0) {$s$};

    \end{scope}


\end{tikzpicture}
    \caption{
        Fundamentales Rechteck der inversen Jaccobi elliptischen Funktionen.
    }
    \label{ellfilter:fig:fundamental_rectangle}
\end{figure}

Auffallend an der $w = \sn(z, k)$-Funktion ist, dass sich $w$ auf der reellen Achse wie der Kosinus immer zwischen $-1$ und $1$ bewegt, während bei $\mathrm{Im(z) = K^\prime}$ die Werte zwischen $\pm 1/k$ und $\pm \infty$ verlaufen.
Die Funktion hat also Equirippel-Verhalten um $w=0$ und um $w=\pm \infty$.
Falls es möglich ist diese Werte abzufahren im Sti der Tschebyscheff-Polynome, kann ein Filter gebaut werden, dass Equirippel-Verhalten im Durchlass- und Sperrbereich aufweist.



Analog zu Abbildung \ref{ellfilter:fig:arccos2} können wir auch bei den elliptisch rationalen Funktionen die komplexe $z$-Ebene betrachten, wie ersichtlich in Abbildung \ref{ellfilter:fig:cd2}, um die besser zu verstehen.
\begin{figure}
    \centering
    \begin{tikzpicture}[>=stealth', auto, node distance=2cm, scale=1.2]

    \tikzstyle{zero} = [draw, circle, inner sep =0, minimum height=0.15cm]
    \tikzstyle{dot} = [fill, circle, inner sep =0, minimum height=0.1cm]

    \tikzset{pole/.style={cross out, draw=black, minimum size=(0.15cm-\pgflinewidth), inner sep=0pt, outer sep=0pt}}

    \begin{scope}[xscale=1.25, yscale=3.5]

        \draw[gray, ->] (0,-0.55) -- (0,1.05) node[anchor=south]{$\mathrm{Im}~z_1$};
        \draw[gray, ->] (-1.5,0) -- (6,0) node[anchor=west]{$\mathrm{Re}~z_1$};

        \draw[gray] ( 1,0) +(0,0.05) -- +(0, -0.05) node[inner sep=0, anchor=north] {\small $K_1$};
        \draw[gray] ( 5,0) +(0,0.05) -- +(0, -0.05) node[inner sep=0, anchor=north] {\small $5K_1$};
        \draw[gray]  (0, 0.5) +(0.1, 0) -- +(-0.1, 0) node[inner sep=0, anchor=east]{\small $jK^\prime_1$};

        \begin{scope}

            \clip(-1.5,-0.75) rectangle (6.8,1.25);

            % \draw[>->, line width=0.05, thick, blue]   (1, 0.45) -- (2, 0.45) -- (2, 0.05) -- ( 0.1, 0.05) -- ( 0.1,0.45) -- (1, 0.45);
            % \draw[>->, line width=0.05, thick, orange] (2, 0.5 ) -- (4, 0.5 ) -- (4, 0   ) -- ( 0  , 0   ) -- ( 0  ,0.5 ) -- (2, 0.5 );
            % \draw[>->, line width=0.05, thick, red]    (3, 0.55) -- (6, 0.55) -- (6,-0.05) -- (-0.1,-0.05) -- (-0.1,0.55) -- (3, 0.55);
            % \node[blue] at (1, 0.25) {$N=1$};
            % \node[orange] at (3, 0.25) {$N=2$};
            % \node[red] at (5, 0.25) {$N=3$};



            % \draw[line width=0.1cm, fill, red!50] (0,0) rectangle (3, 0.5);
            % \draw[line width=0.05cm, fill, orange!50] (0,0) rectangle (2, 0.5);
            % \fill[yellow!50] (0,0) rectangle (1, 0.5);
            % \node[] at (0.5, 0.25) {\small $N=1$};
            % \node[] at (1.5, 0.25) {\small $N=2$};
            % \node[] at (2.5, 0.25) {\small $N=3$};

            \fill[orange!30] (0,0) rectangle (5, 0.5);
            % \fill[yellow!30] (0,0) rectangle (1, 0.1);
            \node[] at (2.5, 0.25) {\small $N=5$};


            \draw[decorate,decoration={brace,amplitude=3pt,mirror}, yshift=0.05cm]
                (5,0.5) node(t_k_unten){} -- node[above, yshift=0.1cm]{$NK_1$}
                (0,0.5) node(t_k_opt_unten){};

            \draw[decorate,decoration={brace,amplitude=3pt,mirror}, xshift=0.1cm]
                (5,0) node(t_k_unten){} -- node[right, xshift=0.1cm]{$K^\prime \frac{K_1N}{K} = K^\prime_1$}
                (5,0.5) node(t_k_opt_unten){};


            \draw[ultra thick, ->, darkgreen] (5, 0) -- node[yshift=-0.5cm]{Durchlassbereich} (0,0);
            \draw[ultra thick, ->, orange] (-0, 0) --  node[align=center]{Übergangs-\\berech} (0,0.5);
            \draw[ultra thick, ->, red] (0,0.5) -- node[align=center, yshift=0.7cm]{Sperrbereich} (5, 0.5);

            \draw (4,0  )  node[dot]{} node[anchor=south]      {\small $1$};
            \draw (2,0  )  node[dot]{} node[anchor=south]      {\small $-1$};
            \draw (0,0  )  node[dot]{} node[anchor=south west] {\small $1$};
            \draw (0,0.5)  node[dot]{} node[anchor=north west] {\small $1/k$};
            \draw (2,0.5)  node[dot]{} node[anchor=north]      {\small $-1/k$};
            \draw (4,0.5)  node[dot]{} node[anchor=north]      {\small $1/k$};

            \foreach \i in {-2,...,1} {
                \foreach \j in {-2,...,1} {
                    \begin{scope}[xshift=\i*4cm, yshift=\j*1cm]

                        \node[zero] at ( 1, 0) {};
                        \node[zero] at ( 3, 0) {};
                        \node[pole] at ( 1,0.5) {};
                        \node[pole] at ( 3,0.5) {};

                    \end{scope}
                }
            }

        \end{scope}

    \end{scope}

\end{tikzpicture}
    \caption{
        $z_1$-Ebene der elliptischen rationalen Funktionen.
        Je grösser die Ordnung $N$ gewählt wird, desto mehr Nullstellen passiert.
    }
    \label{ellfilter:fig:cd2}
\end{figure}
% Da die $\cd^{-1}$-Funktion 



\begin{figure}
    \centering
    %% Creator: Matplotlib, PGF backend
%%
%% To include the figure in your LaTeX document, write
%%   \input{<filename>.pgf}
%%
%% Make sure the required packages are loaded in your preamble
%%   \usepackage{pgf}
%%
%% Also ensure that all the required font packages are loaded; for instance,
%% the lmodern package is sometimes necessary when using math font.
%%   \usepackage{lmodern}
%%
%% Figures using additional raster images can only be included by \input if
%% they are in the same directory as the main LaTeX file. For loading figures
%% from other directories you can use the `import` package
%%   \usepackage{import}
%%
%% and then include the figures with
%%   \import{<path to file>}{<filename>.pgf}
%%
%% Matplotlib used the following preamble
%%
\begingroup%
\makeatletter%
\begin{pgfpicture}%
\pgfpathrectangle{\pgfpointorigin}{\pgfqpoint{4.000000in}{2.500000in}}%
\pgfusepath{use as bounding box, clip}%
\begin{pgfscope}%
\pgfsetbuttcap%
\pgfsetmiterjoin%
\pgfsetlinewidth{0.000000pt}%
\definecolor{currentstroke}{rgb}{1.000000,1.000000,1.000000}%
\pgfsetstrokecolor{currentstroke}%
\pgfsetstrokeopacity{0.000000}%
\pgfsetdash{}{0pt}%
\pgfpathmoveto{\pgfqpoint{0.000000in}{0.000000in}}%
\pgfpathlineto{\pgfqpoint{4.000000in}{0.000000in}}%
\pgfpathlineto{\pgfqpoint{4.000000in}{2.500000in}}%
\pgfpathlineto{\pgfqpoint{0.000000in}{2.500000in}}%
\pgfpathlineto{\pgfqpoint{0.000000in}{0.000000in}}%
\pgfpathclose%
\pgfusepath{}%
\end{pgfscope}%
\begin{pgfscope}%
\pgfsetbuttcap%
\pgfsetmiterjoin%
\definecolor{currentfill}{rgb}{1.000000,1.000000,1.000000}%
\pgfsetfillcolor{currentfill}%
\pgfsetlinewidth{0.000000pt}%
\definecolor{currentstroke}{rgb}{0.000000,0.000000,0.000000}%
\pgfsetstrokecolor{currentstroke}%
\pgfsetstrokeopacity{0.000000}%
\pgfsetdash{}{0pt}%
\pgfpathmoveto{\pgfqpoint{0.733531in}{0.548769in}}%
\pgfpathlineto{\pgfqpoint{3.761597in}{0.548769in}}%
\pgfpathlineto{\pgfqpoint{3.761597in}{2.301955in}}%
\pgfpathlineto{\pgfqpoint{0.733531in}{2.301955in}}%
\pgfpathlineto{\pgfqpoint{0.733531in}{0.548769in}}%
\pgfpathclose%
\pgfusepath{fill}%
\end{pgfscope}%
\begin{pgfscope}%
\pgfpathrectangle{\pgfqpoint{0.733531in}{0.548769in}}{\pgfqpoint{3.028066in}{1.753186in}}%
\pgfusepath{clip}%
\pgfsetbuttcap%
\pgfsetmiterjoin%
\definecolor{currentfill}{rgb}{0.000000,0.501961,0.000000}%
\pgfsetfillcolor{currentfill}%
\pgfsetfillopacity{0.200000}%
\pgfsetlinewidth{0.000000pt}%
\definecolor{currentstroke}{rgb}{0.000000,0.000000,0.000000}%
\pgfsetstrokecolor{currentstroke}%
\pgfsetstrokeopacity{0.200000}%
\pgfsetdash{}{0pt}%
\pgfpathmoveto{\pgfqpoint{0.733531in}{-174.068564in}}%
\pgfpathlineto{\pgfqpoint{2.247564in}{-174.068564in}}%
\pgfpathlineto{\pgfqpoint{2.247564in}{1.250043in}}%
\pgfpathlineto{\pgfqpoint{0.733531in}{1.250043in}}%
\pgfpathlineto{\pgfqpoint{0.733531in}{-174.068564in}}%
\pgfpathclose%
\pgfusepath{fill}%
\end{pgfscope}%
\begin{pgfscope}%
\pgfpathrectangle{\pgfqpoint{0.733531in}{0.548769in}}{\pgfqpoint{3.028066in}{1.753186in}}%
\pgfusepath{clip}%
\pgfsetbuttcap%
\pgfsetmiterjoin%
\definecolor{currentfill}{rgb}{1.000000,0.647059,0.000000}%
\pgfsetfillcolor{currentfill}%
\pgfsetfillopacity{0.200000}%
\pgfsetlinewidth{0.000000pt}%
\definecolor{currentstroke}{rgb}{0.000000,0.000000,0.000000}%
\pgfsetstrokecolor{currentstroke}%
\pgfsetstrokeopacity{0.200000}%
\pgfsetdash{}{0pt}%
\pgfpathmoveto{\pgfqpoint{2.247564in}{1.250043in}}%
\pgfpathlineto{\pgfqpoint{2.262583in}{1.250043in}}%
\pgfpathlineto{\pgfqpoint{2.262583in}{1.600680in}}%
\pgfpathlineto{\pgfqpoint{2.247564in}{1.600680in}}%
\pgfpathlineto{\pgfqpoint{2.247564in}{1.250043in}}%
\pgfpathclose%
\pgfusepath{fill}%
\end{pgfscope}%
\begin{pgfscope}%
\pgfpathrectangle{\pgfqpoint{0.733531in}{0.548769in}}{\pgfqpoint{3.028066in}{1.753186in}}%
\pgfusepath{clip}%
\pgfsetbuttcap%
\pgfsetmiterjoin%
\definecolor{currentfill}{rgb}{1.000000,0.000000,0.000000}%
\pgfsetfillcolor{currentfill}%
\pgfsetfillopacity{0.200000}%
\pgfsetlinewidth{0.000000pt}%
\definecolor{currentstroke}{rgb}{0.000000,0.000000,0.000000}%
\pgfsetstrokecolor{currentstroke}%
\pgfsetstrokeopacity{0.200000}%
\pgfsetdash{}{0pt}%
\pgfpathmoveto{\pgfqpoint{2.262583in}{1.600680in}}%
\pgfpathlineto{\pgfqpoint{3.776616in}{1.600680in}}%
\pgfpathlineto{\pgfqpoint{3.776616in}{2.301962in}}%
\pgfpathlineto{\pgfqpoint{2.262583in}{2.301962in}}%
\pgfpathlineto{\pgfqpoint{2.262583in}{1.600680in}}%
\pgfpathclose%
\pgfusepath{fill}%
\end{pgfscope}%
\begin{pgfscope}%
\pgfpathrectangle{\pgfqpoint{0.733531in}{0.548769in}}{\pgfqpoint{3.028066in}{1.753186in}}%
\pgfusepath{clip}%
\pgfsetrectcap%
\pgfsetroundjoin%
\pgfsetlinewidth{0.803000pt}%
\definecolor{currentstroke}{rgb}{0.690196,0.690196,0.690196}%
\pgfsetstrokecolor{currentstroke}%
\pgfsetdash{}{0pt}%
\pgfpathmoveto{\pgfqpoint{0.733531in}{0.548769in}}%
\pgfpathlineto{\pgfqpoint{0.733531in}{2.301955in}}%
\pgfusepath{stroke}%
\end{pgfscope}%
\begin{pgfscope}%
\pgfsetbuttcap%
\pgfsetroundjoin%
\definecolor{currentfill}{rgb}{0.000000,0.000000,0.000000}%
\pgfsetfillcolor{currentfill}%
\pgfsetlinewidth{0.803000pt}%
\definecolor{currentstroke}{rgb}{0.000000,0.000000,0.000000}%
\pgfsetstrokecolor{currentstroke}%
\pgfsetdash{}{0pt}%
\pgfsys@defobject{currentmarker}{\pgfqpoint{0.000000in}{-0.048611in}}{\pgfqpoint{0.000000in}{0.000000in}}{%
\pgfpathmoveto{\pgfqpoint{0.000000in}{0.000000in}}%
\pgfpathlineto{\pgfqpoint{0.000000in}{-0.048611in}}%
\pgfusepath{stroke,fill}%
}%
\begin{pgfscope}%
\pgfsys@transformshift{0.733531in}{0.548769in}%
\pgfsys@useobject{currentmarker}{}%
\end{pgfscope}%
\end{pgfscope}%
\begin{pgfscope}%
\definecolor{textcolor}{rgb}{0.000000,0.000000,0.000000}%
\pgfsetstrokecolor{textcolor}%
\pgfsetfillcolor{textcolor}%
\pgftext[x=0.733531in,y=0.451547in,,top]{\color{textcolor}\rmfamily\fontsize{10.000000}{12.000000}\selectfont \(\displaystyle {0.0}\)}%
\end{pgfscope}%
\begin{pgfscope}%
\pgfpathrectangle{\pgfqpoint{0.733531in}{0.548769in}}{\pgfqpoint{3.028066in}{1.753186in}}%
\pgfusepath{clip}%
\pgfsetrectcap%
\pgfsetroundjoin%
\pgfsetlinewidth{0.803000pt}%
\definecolor{currentstroke}{rgb}{0.690196,0.690196,0.690196}%
\pgfsetstrokecolor{currentstroke}%
\pgfsetdash{}{0pt}%
\pgfpathmoveto{\pgfqpoint{1.490547in}{0.548769in}}%
\pgfpathlineto{\pgfqpoint{1.490547in}{2.301955in}}%
\pgfusepath{stroke}%
\end{pgfscope}%
\begin{pgfscope}%
\pgfsetbuttcap%
\pgfsetroundjoin%
\definecolor{currentfill}{rgb}{0.000000,0.000000,0.000000}%
\pgfsetfillcolor{currentfill}%
\pgfsetlinewidth{0.803000pt}%
\definecolor{currentstroke}{rgb}{0.000000,0.000000,0.000000}%
\pgfsetstrokecolor{currentstroke}%
\pgfsetdash{}{0pt}%
\pgfsys@defobject{currentmarker}{\pgfqpoint{0.000000in}{-0.048611in}}{\pgfqpoint{0.000000in}{0.000000in}}{%
\pgfpathmoveto{\pgfqpoint{0.000000in}{0.000000in}}%
\pgfpathlineto{\pgfqpoint{0.000000in}{-0.048611in}}%
\pgfusepath{stroke,fill}%
}%
\begin{pgfscope}%
\pgfsys@transformshift{1.490547in}{0.548769in}%
\pgfsys@useobject{currentmarker}{}%
\end{pgfscope}%
\end{pgfscope}%
\begin{pgfscope}%
\definecolor{textcolor}{rgb}{0.000000,0.000000,0.000000}%
\pgfsetstrokecolor{textcolor}%
\pgfsetfillcolor{textcolor}%
\pgftext[x=1.490547in,y=0.451547in,,top]{\color{textcolor}\rmfamily\fontsize{10.000000}{12.000000}\selectfont \(\displaystyle {0.5}\)}%
\end{pgfscope}%
\begin{pgfscope}%
\pgfpathrectangle{\pgfqpoint{0.733531in}{0.548769in}}{\pgfqpoint{3.028066in}{1.753186in}}%
\pgfusepath{clip}%
\pgfsetrectcap%
\pgfsetroundjoin%
\pgfsetlinewidth{0.803000pt}%
\definecolor{currentstroke}{rgb}{0.690196,0.690196,0.690196}%
\pgfsetstrokecolor{currentstroke}%
\pgfsetdash{}{0pt}%
\pgfpathmoveto{\pgfqpoint{2.247564in}{0.548769in}}%
\pgfpathlineto{\pgfqpoint{2.247564in}{2.301955in}}%
\pgfusepath{stroke}%
\end{pgfscope}%
\begin{pgfscope}%
\pgfsetbuttcap%
\pgfsetroundjoin%
\definecolor{currentfill}{rgb}{0.000000,0.000000,0.000000}%
\pgfsetfillcolor{currentfill}%
\pgfsetlinewidth{0.803000pt}%
\definecolor{currentstroke}{rgb}{0.000000,0.000000,0.000000}%
\pgfsetstrokecolor{currentstroke}%
\pgfsetdash{}{0pt}%
\pgfsys@defobject{currentmarker}{\pgfqpoint{0.000000in}{-0.048611in}}{\pgfqpoint{0.000000in}{0.000000in}}{%
\pgfpathmoveto{\pgfqpoint{0.000000in}{0.000000in}}%
\pgfpathlineto{\pgfqpoint{0.000000in}{-0.048611in}}%
\pgfusepath{stroke,fill}%
}%
\begin{pgfscope}%
\pgfsys@transformshift{2.247564in}{0.548769in}%
\pgfsys@useobject{currentmarker}{}%
\end{pgfscope}%
\end{pgfscope}%
\begin{pgfscope}%
\definecolor{textcolor}{rgb}{0.000000,0.000000,0.000000}%
\pgfsetstrokecolor{textcolor}%
\pgfsetfillcolor{textcolor}%
\pgftext[x=2.247564in,y=0.451547in,,top]{\color{textcolor}\rmfamily\fontsize{10.000000}{12.000000}\selectfont \(\displaystyle {1.0}\)}%
\end{pgfscope}%
\begin{pgfscope}%
\pgfpathrectangle{\pgfqpoint{0.733531in}{0.548769in}}{\pgfqpoint{3.028066in}{1.753186in}}%
\pgfusepath{clip}%
\pgfsetrectcap%
\pgfsetroundjoin%
\pgfsetlinewidth{0.803000pt}%
\definecolor{currentstroke}{rgb}{0.690196,0.690196,0.690196}%
\pgfsetstrokecolor{currentstroke}%
\pgfsetdash{}{0pt}%
\pgfpathmoveto{\pgfqpoint{3.004580in}{0.548769in}}%
\pgfpathlineto{\pgfqpoint{3.004580in}{2.301955in}}%
\pgfusepath{stroke}%
\end{pgfscope}%
\begin{pgfscope}%
\pgfsetbuttcap%
\pgfsetroundjoin%
\definecolor{currentfill}{rgb}{0.000000,0.000000,0.000000}%
\pgfsetfillcolor{currentfill}%
\pgfsetlinewidth{0.803000pt}%
\definecolor{currentstroke}{rgb}{0.000000,0.000000,0.000000}%
\pgfsetstrokecolor{currentstroke}%
\pgfsetdash{}{0pt}%
\pgfsys@defobject{currentmarker}{\pgfqpoint{0.000000in}{-0.048611in}}{\pgfqpoint{0.000000in}{0.000000in}}{%
\pgfpathmoveto{\pgfqpoint{0.000000in}{0.000000in}}%
\pgfpathlineto{\pgfqpoint{0.000000in}{-0.048611in}}%
\pgfusepath{stroke,fill}%
}%
\begin{pgfscope}%
\pgfsys@transformshift{3.004580in}{0.548769in}%
\pgfsys@useobject{currentmarker}{}%
\end{pgfscope}%
\end{pgfscope}%
\begin{pgfscope}%
\definecolor{textcolor}{rgb}{0.000000,0.000000,0.000000}%
\pgfsetstrokecolor{textcolor}%
\pgfsetfillcolor{textcolor}%
\pgftext[x=3.004580in,y=0.451547in,,top]{\color{textcolor}\rmfamily\fontsize{10.000000}{12.000000}\selectfont \(\displaystyle {1.5}\)}%
\end{pgfscope}%
\begin{pgfscope}%
\pgfpathrectangle{\pgfqpoint{0.733531in}{0.548769in}}{\pgfqpoint{3.028066in}{1.753186in}}%
\pgfusepath{clip}%
\pgfsetrectcap%
\pgfsetroundjoin%
\pgfsetlinewidth{0.803000pt}%
\definecolor{currentstroke}{rgb}{0.690196,0.690196,0.690196}%
\pgfsetstrokecolor{currentstroke}%
\pgfsetdash{}{0pt}%
\pgfpathmoveto{\pgfqpoint{3.761597in}{0.548769in}}%
\pgfpathlineto{\pgfqpoint{3.761597in}{2.301955in}}%
\pgfusepath{stroke}%
\end{pgfscope}%
\begin{pgfscope}%
\pgfsetbuttcap%
\pgfsetroundjoin%
\definecolor{currentfill}{rgb}{0.000000,0.000000,0.000000}%
\pgfsetfillcolor{currentfill}%
\pgfsetlinewidth{0.803000pt}%
\definecolor{currentstroke}{rgb}{0.000000,0.000000,0.000000}%
\pgfsetstrokecolor{currentstroke}%
\pgfsetdash{}{0pt}%
\pgfsys@defobject{currentmarker}{\pgfqpoint{0.000000in}{-0.048611in}}{\pgfqpoint{0.000000in}{0.000000in}}{%
\pgfpathmoveto{\pgfqpoint{0.000000in}{0.000000in}}%
\pgfpathlineto{\pgfqpoint{0.000000in}{-0.048611in}}%
\pgfusepath{stroke,fill}%
}%
\begin{pgfscope}%
\pgfsys@transformshift{3.761597in}{0.548769in}%
\pgfsys@useobject{currentmarker}{}%
\end{pgfscope}%
\end{pgfscope}%
\begin{pgfscope}%
\definecolor{textcolor}{rgb}{0.000000,0.000000,0.000000}%
\pgfsetstrokecolor{textcolor}%
\pgfsetfillcolor{textcolor}%
\pgftext[x=3.761597in,y=0.451547in,,top]{\color{textcolor}\rmfamily\fontsize{10.000000}{12.000000}\selectfont \(\displaystyle {2.0}\)}%
\end{pgfscope}%
\begin{pgfscope}%
\definecolor{textcolor}{rgb}{0.000000,0.000000,0.000000}%
\pgfsetstrokecolor{textcolor}%
\pgfsetfillcolor{textcolor}%
\pgftext[x=2.247564in,y=0.272534in,,top]{\color{textcolor}\rmfamily\fontsize{10.000000}{12.000000}\selectfont \(\displaystyle w\)}%
\end{pgfscope}%
\begin{pgfscope}%
\pgfpathrectangle{\pgfqpoint{0.733531in}{0.548769in}}{\pgfqpoint{3.028066in}{1.753186in}}%
\pgfusepath{clip}%
\pgfsetrectcap%
\pgfsetroundjoin%
\pgfsetlinewidth{0.803000pt}%
\definecolor{currentstroke}{rgb}{0.690196,0.690196,0.690196}%
\pgfsetstrokecolor{currentstroke}%
\pgfsetdash{}{0pt}%
\pgfpathmoveto{\pgfqpoint{0.733531in}{0.548769in}}%
\pgfpathlineto{\pgfqpoint{3.761597in}{0.548769in}}%
\pgfusepath{stroke}%
\end{pgfscope}%
\begin{pgfscope}%
\pgfsetbuttcap%
\pgfsetroundjoin%
\definecolor{currentfill}{rgb}{0.000000,0.000000,0.000000}%
\pgfsetfillcolor{currentfill}%
\pgfsetlinewidth{0.803000pt}%
\definecolor{currentstroke}{rgb}{0.000000,0.000000,0.000000}%
\pgfsetstrokecolor{currentstroke}%
\pgfsetdash{}{0pt}%
\pgfsys@defobject{currentmarker}{\pgfqpoint{-0.048611in}{0.000000in}}{\pgfqpoint{-0.000000in}{0.000000in}}{%
\pgfpathmoveto{\pgfqpoint{-0.000000in}{0.000000in}}%
\pgfpathlineto{\pgfqpoint{-0.048611in}{0.000000in}}%
\pgfusepath{stroke,fill}%
}%
\begin{pgfscope}%
\pgfsys@transformshift{0.733531in}{0.548769in}%
\pgfsys@useobject{currentmarker}{}%
\end{pgfscope}%
\end{pgfscope}%
\begin{pgfscope}%
\definecolor{textcolor}{rgb}{0.000000,0.000000,0.000000}%
\pgfsetstrokecolor{textcolor}%
\pgfsetfillcolor{textcolor}%
\pgftext[x=0.348306in, y=0.500544in, left, base]{\color{textcolor}\rmfamily\fontsize{10.000000}{12.000000}\selectfont \(\displaystyle {10^{-4}}\)}%
\end{pgfscope}%
\begin{pgfscope}%
\pgfpathrectangle{\pgfqpoint{0.733531in}{0.548769in}}{\pgfqpoint{3.028066in}{1.753186in}}%
\pgfusepath{clip}%
\pgfsetrectcap%
\pgfsetroundjoin%
\pgfsetlinewidth{0.803000pt}%
\definecolor{currentstroke}{rgb}{0.690196,0.690196,0.690196}%
\pgfsetstrokecolor{currentstroke}%
\pgfsetdash{}{0pt}%
\pgfpathmoveto{\pgfqpoint{0.733531in}{0.899406in}}%
\pgfpathlineto{\pgfqpoint{3.761597in}{0.899406in}}%
\pgfusepath{stroke}%
\end{pgfscope}%
\begin{pgfscope}%
\pgfsetbuttcap%
\pgfsetroundjoin%
\definecolor{currentfill}{rgb}{0.000000,0.000000,0.000000}%
\pgfsetfillcolor{currentfill}%
\pgfsetlinewidth{0.803000pt}%
\definecolor{currentstroke}{rgb}{0.000000,0.000000,0.000000}%
\pgfsetstrokecolor{currentstroke}%
\pgfsetdash{}{0pt}%
\pgfsys@defobject{currentmarker}{\pgfqpoint{-0.048611in}{0.000000in}}{\pgfqpoint{-0.000000in}{0.000000in}}{%
\pgfpathmoveto{\pgfqpoint{-0.000000in}{0.000000in}}%
\pgfpathlineto{\pgfqpoint{-0.048611in}{0.000000in}}%
\pgfusepath{stroke,fill}%
}%
\begin{pgfscope}%
\pgfsys@transformshift{0.733531in}{0.899406in}%
\pgfsys@useobject{currentmarker}{}%
\end{pgfscope}%
\end{pgfscope}%
\begin{pgfscope}%
\definecolor{textcolor}{rgb}{0.000000,0.000000,0.000000}%
\pgfsetstrokecolor{textcolor}%
\pgfsetfillcolor{textcolor}%
\pgftext[x=0.348306in, y=0.851181in, left, base]{\color{textcolor}\rmfamily\fontsize{10.000000}{12.000000}\selectfont \(\displaystyle {10^{-2}}\)}%
\end{pgfscope}%
\begin{pgfscope}%
\pgfpathrectangle{\pgfqpoint{0.733531in}{0.548769in}}{\pgfqpoint{3.028066in}{1.753186in}}%
\pgfusepath{clip}%
\pgfsetrectcap%
\pgfsetroundjoin%
\pgfsetlinewidth{0.803000pt}%
\definecolor{currentstroke}{rgb}{0.690196,0.690196,0.690196}%
\pgfsetstrokecolor{currentstroke}%
\pgfsetdash{}{0pt}%
\pgfpathmoveto{\pgfqpoint{0.733531in}{1.250043in}}%
\pgfpathlineto{\pgfqpoint{3.761597in}{1.250043in}}%
\pgfusepath{stroke}%
\end{pgfscope}%
\begin{pgfscope}%
\pgfsetbuttcap%
\pgfsetroundjoin%
\definecolor{currentfill}{rgb}{0.000000,0.000000,0.000000}%
\pgfsetfillcolor{currentfill}%
\pgfsetlinewidth{0.803000pt}%
\definecolor{currentstroke}{rgb}{0.000000,0.000000,0.000000}%
\pgfsetstrokecolor{currentstroke}%
\pgfsetdash{}{0pt}%
\pgfsys@defobject{currentmarker}{\pgfqpoint{-0.048611in}{0.000000in}}{\pgfqpoint{-0.000000in}{0.000000in}}{%
\pgfpathmoveto{\pgfqpoint{-0.000000in}{0.000000in}}%
\pgfpathlineto{\pgfqpoint{-0.048611in}{0.000000in}}%
\pgfusepath{stroke,fill}%
}%
\begin{pgfscope}%
\pgfsys@transformshift{0.733531in}{1.250043in}%
\pgfsys@useobject{currentmarker}{}%
\end{pgfscope}%
\end{pgfscope}%
\begin{pgfscope}%
\definecolor{textcolor}{rgb}{0.000000,0.000000,0.000000}%
\pgfsetstrokecolor{textcolor}%
\pgfsetfillcolor{textcolor}%
\pgftext[x=0.435112in, y=1.201818in, left, base]{\color{textcolor}\rmfamily\fontsize{10.000000}{12.000000}\selectfont \(\displaystyle {10^{0}}\)}%
\end{pgfscope}%
\begin{pgfscope}%
\pgfpathrectangle{\pgfqpoint{0.733531in}{0.548769in}}{\pgfqpoint{3.028066in}{1.753186in}}%
\pgfusepath{clip}%
\pgfsetrectcap%
\pgfsetroundjoin%
\pgfsetlinewidth{0.803000pt}%
\definecolor{currentstroke}{rgb}{0.690196,0.690196,0.690196}%
\pgfsetstrokecolor{currentstroke}%
\pgfsetdash{}{0pt}%
\pgfpathmoveto{\pgfqpoint{0.733531in}{1.600680in}}%
\pgfpathlineto{\pgfqpoint{3.761597in}{1.600680in}}%
\pgfusepath{stroke}%
\end{pgfscope}%
\begin{pgfscope}%
\pgfsetbuttcap%
\pgfsetroundjoin%
\definecolor{currentfill}{rgb}{0.000000,0.000000,0.000000}%
\pgfsetfillcolor{currentfill}%
\pgfsetlinewidth{0.803000pt}%
\definecolor{currentstroke}{rgb}{0.000000,0.000000,0.000000}%
\pgfsetstrokecolor{currentstroke}%
\pgfsetdash{}{0pt}%
\pgfsys@defobject{currentmarker}{\pgfqpoint{-0.048611in}{0.000000in}}{\pgfqpoint{-0.000000in}{0.000000in}}{%
\pgfpathmoveto{\pgfqpoint{-0.000000in}{0.000000in}}%
\pgfpathlineto{\pgfqpoint{-0.048611in}{0.000000in}}%
\pgfusepath{stroke,fill}%
}%
\begin{pgfscope}%
\pgfsys@transformshift{0.733531in}{1.600680in}%
\pgfsys@useobject{currentmarker}{}%
\end{pgfscope}%
\end{pgfscope}%
\begin{pgfscope}%
\definecolor{textcolor}{rgb}{0.000000,0.000000,0.000000}%
\pgfsetstrokecolor{textcolor}%
\pgfsetfillcolor{textcolor}%
\pgftext[x=0.435112in, y=1.552455in, left, base]{\color{textcolor}\rmfamily\fontsize{10.000000}{12.000000}\selectfont \(\displaystyle {10^{2}}\)}%
\end{pgfscope}%
\begin{pgfscope}%
\pgfpathrectangle{\pgfqpoint{0.733531in}{0.548769in}}{\pgfqpoint{3.028066in}{1.753186in}}%
\pgfusepath{clip}%
\pgfsetrectcap%
\pgfsetroundjoin%
\pgfsetlinewidth{0.803000pt}%
\definecolor{currentstroke}{rgb}{0.690196,0.690196,0.690196}%
\pgfsetstrokecolor{currentstroke}%
\pgfsetdash{}{0pt}%
\pgfpathmoveto{\pgfqpoint{0.733531in}{1.951318in}}%
\pgfpathlineto{\pgfqpoint{3.761597in}{1.951318in}}%
\pgfusepath{stroke}%
\end{pgfscope}%
\begin{pgfscope}%
\pgfsetbuttcap%
\pgfsetroundjoin%
\definecolor{currentfill}{rgb}{0.000000,0.000000,0.000000}%
\pgfsetfillcolor{currentfill}%
\pgfsetlinewidth{0.803000pt}%
\definecolor{currentstroke}{rgb}{0.000000,0.000000,0.000000}%
\pgfsetstrokecolor{currentstroke}%
\pgfsetdash{}{0pt}%
\pgfsys@defobject{currentmarker}{\pgfqpoint{-0.048611in}{0.000000in}}{\pgfqpoint{-0.000000in}{0.000000in}}{%
\pgfpathmoveto{\pgfqpoint{-0.000000in}{0.000000in}}%
\pgfpathlineto{\pgfqpoint{-0.048611in}{0.000000in}}%
\pgfusepath{stroke,fill}%
}%
\begin{pgfscope}%
\pgfsys@transformshift{0.733531in}{1.951318in}%
\pgfsys@useobject{currentmarker}{}%
\end{pgfscope}%
\end{pgfscope}%
\begin{pgfscope}%
\definecolor{textcolor}{rgb}{0.000000,0.000000,0.000000}%
\pgfsetstrokecolor{textcolor}%
\pgfsetfillcolor{textcolor}%
\pgftext[x=0.435112in, y=1.903092in, left, base]{\color{textcolor}\rmfamily\fontsize{10.000000}{12.000000}\selectfont \(\displaystyle {10^{4}}\)}%
\end{pgfscope}%
\begin{pgfscope}%
\pgfpathrectangle{\pgfqpoint{0.733531in}{0.548769in}}{\pgfqpoint{3.028066in}{1.753186in}}%
\pgfusepath{clip}%
\pgfsetrectcap%
\pgfsetroundjoin%
\pgfsetlinewidth{0.803000pt}%
\definecolor{currentstroke}{rgb}{0.690196,0.690196,0.690196}%
\pgfsetstrokecolor{currentstroke}%
\pgfsetdash{}{0pt}%
\pgfpathmoveto{\pgfqpoint{0.733531in}{2.301955in}}%
\pgfpathlineto{\pgfqpoint{3.761597in}{2.301955in}}%
\pgfusepath{stroke}%
\end{pgfscope}%
\begin{pgfscope}%
\pgfsetbuttcap%
\pgfsetroundjoin%
\definecolor{currentfill}{rgb}{0.000000,0.000000,0.000000}%
\pgfsetfillcolor{currentfill}%
\pgfsetlinewidth{0.803000pt}%
\definecolor{currentstroke}{rgb}{0.000000,0.000000,0.000000}%
\pgfsetstrokecolor{currentstroke}%
\pgfsetdash{}{0pt}%
\pgfsys@defobject{currentmarker}{\pgfqpoint{-0.048611in}{0.000000in}}{\pgfqpoint{-0.000000in}{0.000000in}}{%
\pgfpathmoveto{\pgfqpoint{-0.000000in}{0.000000in}}%
\pgfpathlineto{\pgfqpoint{-0.048611in}{0.000000in}}%
\pgfusepath{stroke,fill}%
}%
\begin{pgfscope}%
\pgfsys@transformshift{0.733531in}{2.301955in}%
\pgfsys@useobject{currentmarker}{}%
\end{pgfscope}%
\end{pgfscope}%
\begin{pgfscope}%
\definecolor{textcolor}{rgb}{0.000000,0.000000,0.000000}%
\pgfsetstrokecolor{textcolor}%
\pgfsetfillcolor{textcolor}%
\pgftext[x=0.435112in, y=2.253730in, left, base]{\color{textcolor}\rmfamily\fontsize{10.000000}{12.000000}\selectfont \(\displaystyle {10^{6}}\)}%
\end{pgfscope}%
\begin{pgfscope}%
\definecolor{textcolor}{rgb}{0.000000,0.000000,0.000000}%
\pgfsetstrokecolor{textcolor}%
\pgfsetfillcolor{textcolor}%
\pgftext[x=0.292751in,y=1.425362in,,bottom,rotate=90.000000]{\color{textcolor}\rmfamily\fontsize{10.000000}{12.000000}\selectfont \(\displaystyle F^2_N(w)\)}%
\end{pgfscope}%
\begin{pgfscope}%
\pgfpathrectangle{\pgfqpoint{0.733531in}{0.548769in}}{\pgfqpoint{3.028066in}{1.753186in}}%
\pgfusepath{clip}%
\pgfsetrectcap%
\pgfsetroundjoin%
\pgfsetlinewidth{1.003750pt}%
\definecolor{currentstroke}{rgb}{0.000000,0.501961,0.000000}%
\pgfsetstrokecolor{currentstroke}%
\pgfsetdash{}{0pt}%
\pgfpathmoveto{\pgfqpoint{0.739446in}{0.534880in}}%
\pgfpathlineto{\pgfqpoint{0.744135in}{0.623954in}}%
\pgfpathlineto{\pgfqpoint{0.750951in}{0.699544in}}%
\pgfpathlineto{\pgfqpoint{0.759282in}{0.759051in}}%
\pgfpathlineto{\pgfqpoint{0.769129in}{0.808333in}}%
\pgfpathlineto{\pgfqpoint{0.781247in}{0.852909in}}%
\pgfpathlineto{\pgfqpoint{0.794880in}{0.891121in}}%
\pgfpathlineto{\pgfqpoint{0.810028in}{0.924642in}}%
\pgfpathlineto{\pgfqpoint{0.827448in}{0.955767in}}%
\pgfpathlineto{\pgfqpoint{0.847140in}{0.984592in}}%
\pgfpathlineto{\pgfqpoint{0.869105in}{1.011289in}}%
\pgfpathlineto{\pgfqpoint{0.894099in}{1.036759in}}%
\pgfpathlineto{\pgfqpoint{0.922122in}{1.060860in}}%
\pgfpathlineto{\pgfqpoint{0.953933in}{1.084065in}}%
\pgfpathlineto{\pgfqpoint{0.989531in}{1.106163in}}%
\pgfpathlineto{\pgfqpoint{1.029673in}{1.127411in}}%
\pgfpathlineto{\pgfqpoint{1.075116in}{1.147900in}}%
\pgfpathlineto{\pgfqpoint{1.125862in}{1.167334in}}%
\pgfpathlineto{\pgfqpoint{1.182666in}{1.185708in}}%
\pgfpathlineto{\pgfqpoint{1.244773in}{1.202512in}}%
\pgfpathlineto{\pgfqpoint{1.312181in}{1.217523in}}%
\pgfpathlineto{\pgfqpoint{1.383376in}{1.230197in}}%
\pgfpathlineto{\pgfqpoint{1.456086in}{1.240011in}}%
\pgfpathlineto{\pgfqpoint{1.527281in}{1.246554in}}%
\pgfpathlineto{\pgfqpoint{1.594689in}{1.249711in}}%
\pgfpathlineto{\pgfqpoint{1.656038in}{1.249582in}}%
\pgfpathlineto{\pgfqpoint{1.710571in}{1.246487in}}%
\pgfpathlineto{\pgfqpoint{1.759044in}{1.240693in}}%
\pgfpathlineto{\pgfqpoint{1.800701in}{1.232678in}}%
\pgfpathlineto{\pgfqpoint{1.837056in}{1.222595in}}%
\pgfpathlineto{\pgfqpoint{1.868109in}{1.210890in}}%
\pgfpathlineto{\pgfqpoint{1.895375in}{1.197412in}}%
\pgfpathlineto{\pgfqpoint{1.919612in}{1.181986in}}%
\pgfpathlineto{\pgfqpoint{1.940819in}{1.164808in}}%
\pgfpathlineto{\pgfqpoint{1.959754in}{1.145405in}}%
\pgfpathlineto{\pgfqpoint{1.976416in}{1.123825in}}%
\pgfpathlineto{\pgfqpoint{1.990807in}{1.100289in}}%
\pgfpathlineto{\pgfqpoint{2.003683in}{1.073560in}}%
\pgfpathlineto{\pgfqpoint{2.015044in}{1.043233in}}%
\pgfpathlineto{\pgfqpoint{2.025647in}{1.005813in}}%
\pgfpathlineto{\pgfqpoint{2.034736in}{0.961762in}}%
\pgfpathlineto{\pgfqpoint{2.042310in}{0.909257in}}%
\pgfpathlineto{\pgfqpoint{2.048369in}{0.845806in}}%
\pgfpathlineto{\pgfqpoint{2.052913in}{0.768537in}}%
\pgfpathlineto{\pgfqpoint{2.056700in}{0.642785in}}%
\pgfpathlineto{\pgfqpoint{2.058170in}{0.534880in}}%
\pgfpathmoveto{\pgfqpoint{2.061062in}{0.534880in}}%
\pgfpathlineto{\pgfqpoint{2.063517in}{0.690755in}}%
\pgfpathlineto{\pgfqpoint{2.068061in}{0.809745in}}%
\pgfpathlineto{\pgfqpoint{2.074120in}{0.894170in}}%
\pgfpathlineto{\pgfqpoint{2.082452in}{0.966161in}}%
\pgfpathlineto{\pgfqpoint{2.092298in}{1.024095in}}%
\pgfpathlineto{\pgfqpoint{2.103659in}{1.073241in}}%
\pgfpathlineto{\pgfqpoint{2.117292in}{1.118465in}}%
\pgfpathlineto{\pgfqpoint{2.132440in}{1.158024in}}%
\pgfpathlineto{\pgfqpoint{2.148345in}{1.191332in}}%
\pgfpathlineto{\pgfqpoint{2.164250in}{1.217932in}}%
\pgfpathlineto{\pgfqpoint{2.178641in}{1.236315in}}%
\pgfpathlineto{\pgfqpoint{2.190002in}{1.246177in}}%
\pgfpathlineto{\pgfqpoint{2.198333in}{1.249773in}}%
\pgfpathlineto{\pgfqpoint{2.205150in}{1.249364in}}%
\pgfpathlineto{\pgfqpoint{2.210452in}{1.245998in}}%
\pgfpathlineto{\pgfqpoint{2.215753in}{1.238648in}}%
\pgfpathlineto{\pgfqpoint{2.221055in}{1.225044in}}%
\pgfpathlineto{\pgfqpoint{2.225599in}{1.204942in}}%
\pgfpathlineto{\pgfqpoint{2.230144in}{1.169937in}}%
\pgfpathlineto{\pgfqpoint{2.233931in}{1.115425in}}%
\pgfpathlineto{\pgfqpoint{2.236960in}{1.023401in}}%
\pgfpathlineto{\pgfqpoint{2.238475in}{0.917874in}}%
\pgfpathlineto{\pgfqpoint{2.239990in}{0.614529in}}%
\pgfpathlineto{\pgfqpoint{2.242262in}{1.034265in}}%
\pgfpathlineto{\pgfqpoint{2.246806in}{1.228220in}}%
\pgfpathlineto{\pgfqpoint{2.247564in}{1.250043in}}%
\pgfpathlineto{\pgfqpoint{2.247564in}{1.250043in}}%
\pgfusepath{stroke}%
\end{pgfscope}%
\begin{pgfscope}%
\pgfpathrectangle{\pgfqpoint{0.733531in}{0.548769in}}{\pgfqpoint{3.028066in}{1.753186in}}%
\pgfusepath{clip}%
\pgfsetrectcap%
\pgfsetroundjoin%
\pgfsetlinewidth{1.003750pt}%
\definecolor{currentstroke}{rgb}{1.000000,0.647059,0.000000}%
\pgfsetstrokecolor{currentstroke}%
\pgfsetdash{}{0pt}%
\pgfpathmoveto{\pgfqpoint{2.247564in}{1.250043in}}%
\pgfpathlineto{\pgfqpoint{2.256527in}{1.456923in}}%
\pgfpathlineto{\pgfqpoint{2.262583in}{1.600512in}}%
\pgfpathlineto{\pgfqpoint{2.262583in}{1.600512in}}%
\pgfusepath{stroke}%
\end{pgfscope}%
\begin{pgfscope}%
\pgfpathrectangle{\pgfqpoint{0.733531in}{0.548769in}}{\pgfqpoint{3.028066in}{1.753186in}}%
\pgfusepath{clip}%
\pgfsetrectcap%
\pgfsetroundjoin%
\pgfsetlinewidth{1.003750pt}%
\definecolor{currentstroke}{rgb}{1.000000,0.000000,0.000000}%
\pgfsetstrokecolor{currentstroke}%
\pgfsetdash{}{0pt}%
\pgfpathmoveto{\pgfqpoint{2.262583in}{1.600512in}}%
\pgfpathlineto{\pgfqpoint{2.267082in}{1.764522in}}%
\pgfpathlineto{\pgfqpoint{2.269332in}{1.949179in}}%
\pgfpathlineto{\pgfqpoint{2.270082in}{2.126941in}}%
\pgfpathlineto{\pgfqpoint{2.270832in}{2.115965in}}%
\pgfpathlineto{\pgfqpoint{2.273831in}{1.806954in}}%
\pgfpathlineto{\pgfqpoint{2.278331in}{1.704361in}}%
\pgfpathlineto{\pgfqpoint{2.283580in}{1.654377in}}%
\pgfpathlineto{\pgfqpoint{2.289579in}{1.626078in}}%
\pgfpathlineto{\pgfqpoint{2.295578in}{1.611340in}}%
\pgfpathlineto{\pgfqpoint{2.301577in}{1.603818in}}%
\pgfpathlineto{\pgfqpoint{2.307576in}{1.600689in}}%
\pgfpathlineto{\pgfqpoint{2.314325in}{1.600615in}}%
\pgfpathlineto{\pgfqpoint{2.322574in}{1.603892in}}%
\pgfpathlineto{\pgfqpoint{2.333072in}{1.611738in}}%
\pgfpathlineto{\pgfqpoint{2.346570in}{1.626063in}}%
\pgfpathlineto{\pgfqpoint{2.363067in}{1.648464in}}%
\pgfpathlineto{\pgfqpoint{2.381064in}{1.678367in}}%
\pgfpathlineto{\pgfqpoint{2.398312in}{1.712908in}}%
\pgfpathlineto{\pgfqpoint{2.414809in}{1.753046in}}%
\pgfpathlineto{\pgfqpoint{2.429057in}{1.796016in}}%
\pgfpathlineto{\pgfqpoint{2.441055in}{1.841839in}}%
\pgfpathlineto{\pgfqpoint{2.451553in}{1.894485in}}%
\pgfpathlineto{\pgfqpoint{2.459802in}{1.951191in}}%
\pgfpathlineto{\pgfqpoint{2.466551in}{2.018321in}}%
\pgfpathlineto{\pgfqpoint{2.471800in}{2.100984in}}%
\pgfpathlineto{\pgfqpoint{2.475549in}{2.207779in}}%
\pgfpathlineto{\pgfqpoint{2.477412in}{2.315844in}}%
\pgfpathmoveto{\pgfqpoint{2.481215in}{2.315844in}}%
\pgfpathlineto{\pgfqpoint{2.484548in}{2.160223in}}%
\pgfpathlineto{\pgfqpoint{2.489797in}{2.056487in}}%
\pgfpathlineto{\pgfqpoint{2.496546in}{1.983295in}}%
\pgfpathlineto{\pgfqpoint{2.504795in}{1.926727in}}%
\pgfpathlineto{\pgfqpoint{2.514543in}{1.880902in}}%
\pgfpathlineto{\pgfqpoint{2.525792in}{1.842715in}}%
\pgfpathlineto{\pgfqpoint{2.538540in}{1.810301in}}%
\pgfpathlineto{\pgfqpoint{2.552787in}{1.782438in}}%
\pgfpathlineto{\pgfqpoint{2.568535in}{1.758275in}}%
\pgfpathlineto{\pgfqpoint{2.586532in}{1.736373in}}%
\pgfpathlineto{\pgfqpoint{2.606779in}{1.716740in}}%
\pgfpathlineto{\pgfqpoint{2.629275in}{1.699278in}}%
\pgfpathlineto{\pgfqpoint{2.654771in}{1.683422in}}%
\pgfpathlineto{\pgfqpoint{2.684017in}{1.668923in}}%
\pgfpathlineto{\pgfqpoint{2.717761in}{1.655709in}}%
\pgfpathlineto{\pgfqpoint{2.756755in}{1.643804in}}%
\pgfpathlineto{\pgfqpoint{2.802498in}{1.633115in}}%
\pgfpathlineto{\pgfqpoint{2.857239in}{1.623602in}}%
\pgfpathlineto{\pgfqpoint{2.922479in}{1.615504in}}%
\pgfpathlineto{\pgfqpoint{3.001966in}{1.608875in}}%
\pgfpathlineto{\pgfqpoint{3.099451in}{1.603960in}}%
\pgfpathlineto{\pgfqpoint{3.222432in}{1.600986in}}%
\pgfpathlineto{\pgfqpoint{3.382157in}{1.600379in}}%
\pgfpathlineto{\pgfqpoint{3.598872in}{1.602843in}}%
\pgfpathlineto{\pgfqpoint{3.761597in}{1.606074in}}%
\pgfpathlineto{\pgfqpoint{3.761597in}{1.606074in}}%
\pgfusepath{stroke}%
\end{pgfscope}%
\begin{pgfscope}%
\pgfpathrectangle{\pgfqpoint{0.733531in}{0.548769in}}{\pgfqpoint{3.028066in}{1.753186in}}%
\pgfusepath{clip}%
\pgfsetbuttcap%
\pgfsetroundjoin%
\definecolor{currentfill}{rgb}{0.000000,0.000000,0.000000}%
\pgfsetfillcolor{currentfill}%
\pgfsetfillopacity{0.000000}%
\pgfsetlinewidth{1.003750pt}%
\definecolor{currentstroke}{rgb}{0.000000,0.000000,0.000000}%
\pgfsetstrokecolor{currentstroke}%
\pgfsetdash{}{0pt}%
\pgfsys@defobject{currentmarker}{\pgfqpoint{-0.041667in}{-0.041667in}}{\pgfqpoint{0.041667in}{0.041667in}}{%
\pgfpathmoveto{\pgfqpoint{0.000000in}{-0.041667in}}%
\pgfpathcurveto{\pgfqpoint{0.011050in}{-0.041667in}}{\pgfqpoint{0.021649in}{-0.037276in}}{\pgfqpoint{0.029463in}{-0.029463in}}%
\pgfpathcurveto{\pgfqpoint{0.037276in}{-0.021649in}}{\pgfqpoint{0.041667in}{-0.011050in}}{\pgfqpoint{0.041667in}{0.000000in}}%
\pgfpathcurveto{\pgfqpoint{0.041667in}{0.011050in}}{\pgfqpoint{0.037276in}{0.021649in}}{\pgfqpoint{0.029463in}{0.029463in}}%
\pgfpathcurveto{\pgfqpoint{0.021649in}{0.037276in}}{\pgfqpoint{0.011050in}{0.041667in}}{\pgfqpoint{0.000000in}{0.041667in}}%
\pgfpathcurveto{\pgfqpoint{-0.011050in}{0.041667in}}{\pgfqpoint{-0.021649in}{0.037276in}}{\pgfqpoint{-0.029463in}{0.029463in}}%
\pgfpathcurveto{\pgfqpoint{-0.037276in}{0.021649in}}{\pgfqpoint{-0.041667in}{0.011050in}}{\pgfqpoint{-0.041667in}{0.000000in}}%
\pgfpathcurveto{\pgfqpoint{-0.041667in}{-0.011050in}}{\pgfqpoint{-0.037276in}{-0.021649in}}{\pgfqpoint{-0.029463in}{-0.029463in}}%
\pgfpathcurveto{\pgfqpoint{-0.021649in}{-0.037276in}}{\pgfqpoint{-0.011050in}{-0.041667in}}{\pgfqpoint{0.000000in}{-0.041667in}}%
\pgfpathlineto{\pgfqpoint{0.000000in}{-0.041667in}}%
\pgfpathclose%
\pgfusepath{stroke,fill}%
}%
\begin{pgfscope}%
\pgfsys@transformshift{0.733531in}{0.548769in}%
\pgfsys@useobject{currentmarker}{}%
\end{pgfscope}%
\begin{pgfscope}%
\pgfsys@transformshift{2.050740in}{0.548769in}%
\pgfsys@useobject{currentmarker}{}%
\end{pgfscope}%
\begin{pgfscope}%
\pgfsys@transformshift{2.239994in}{0.548769in}%
\pgfsys@useobject{currentmarker}{}%
\end{pgfscope}%
\end{pgfscope}%
\begin{pgfscope}%
\pgfpathrectangle{\pgfqpoint{0.733531in}{0.548769in}}{\pgfqpoint{3.028066in}{1.753186in}}%
\pgfusepath{clip}%
\pgfsetbuttcap%
\pgfsetroundjoin%
\definecolor{currentfill}{rgb}{0.000000,0.000000,0.000000}%
\pgfsetfillcolor{currentfill}%
\pgfsetfillopacity{0.000000}%
\pgfsetlinewidth{1.003750pt}%
\definecolor{currentstroke}{rgb}{0.000000,0.000000,0.000000}%
\pgfsetstrokecolor{currentstroke}%
\pgfsetdash{}{0pt}%
\pgfsys@defobject{currentmarker}{\pgfqpoint{-0.041667in}{-0.041667in}}{\pgfqpoint{0.041667in}{0.041667in}}{%
\pgfpathmoveto{\pgfqpoint{-0.041667in}{-0.041667in}}%
\pgfpathlineto{\pgfqpoint{0.041667in}{0.041667in}}%
\pgfpathmoveto{\pgfqpoint{-0.041667in}{0.041667in}}%
\pgfpathlineto{\pgfqpoint{0.041667in}{-0.041667in}}%
\pgfusepath{stroke,fill}%
}%
\begin{pgfscope}%
\pgfsys@transformshift{2.262704in}{2.301955in}%
\pgfsys@useobject{currentmarker}{}%
\end{pgfscope}%
\begin{pgfscope}%
\pgfsys@transformshift{2.482239in}{2.301955in}%
\pgfsys@useobject{currentmarker}{}%
\end{pgfscope}%
\end{pgfscope}%
\begin{pgfscope}%
\pgfsetrectcap%
\pgfsetmiterjoin%
\pgfsetlinewidth{0.803000pt}%
\definecolor{currentstroke}{rgb}{0.000000,0.000000,0.000000}%
\pgfsetstrokecolor{currentstroke}%
\pgfsetdash{}{0pt}%
\pgfpathmoveto{\pgfqpoint{0.733531in}{0.548769in}}%
\pgfpathlineto{\pgfqpoint{0.733531in}{2.301955in}}%
\pgfusepath{stroke}%
\end{pgfscope}%
\begin{pgfscope}%
\pgfsetrectcap%
\pgfsetmiterjoin%
\pgfsetlinewidth{0.803000pt}%
\definecolor{currentstroke}{rgb}{0.000000,0.000000,0.000000}%
\pgfsetstrokecolor{currentstroke}%
\pgfsetdash{}{0pt}%
\pgfpathmoveto{\pgfqpoint{3.761597in}{0.548769in}}%
\pgfpathlineto{\pgfqpoint{3.761597in}{2.301955in}}%
\pgfusepath{stroke}%
\end{pgfscope}%
\begin{pgfscope}%
\pgfsetrectcap%
\pgfsetmiterjoin%
\pgfsetlinewidth{0.803000pt}%
\definecolor{currentstroke}{rgb}{0.000000,0.000000,0.000000}%
\pgfsetstrokecolor{currentstroke}%
\pgfsetdash{}{0pt}%
\pgfpathmoveto{\pgfqpoint{0.733531in}{0.548769in}}%
\pgfpathlineto{\pgfqpoint{3.761597in}{0.548769in}}%
\pgfusepath{stroke}%
\end{pgfscope}%
\begin{pgfscope}%
\pgfsetrectcap%
\pgfsetmiterjoin%
\pgfsetlinewidth{0.803000pt}%
\definecolor{currentstroke}{rgb}{0.000000,0.000000,0.000000}%
\pgfsetstrokecolor{currentstroke}%
\pgfsetdash{}{0pt}%
\pgfpathmoveto{\pgfqpoint{0.733531in}{2.301955in}}%
\pgfpathlineto{\pgfqpoint{3.761597in}{2.301955in}}%
\pgfusepath{stroke}%
\end{pgfscope}%
\end{pgfpicture}%
\makeatother%
\endgroup%

    \caption{$F_N$ für ein elliptischs filter.}
    \label{ellfilter:fig:elliptic}
\end{figure}

\subsection{Degree Equation}

Der $\cd^{-1}$ Term muss so verzogen werden, dass die umgebene $\cd$-Funktion die Nullstellen und Pole trifft.
Dies trifft ein wenn die Degree Equation erfüllt ist.

\begin{equation}
    N \frac{K^\prime}{K} = \frac{K^\prime_1}{K_1}
\end{equation}


Leider ist das lösen dieser Gleichung nicht trivial.
Die Rechnung wird in \ref{ellfilter:bib:orfanidis} im Detail angeschaut.


\subsection{Polynome?}

Bei den Tschebyscheff-Polynomen haben wir gesehen, dass die Trigonometrische Formel zu einfachen Polynomen umgewandelt werden kann.
Im gegensatz zum $\cos^{-1}$ hat der $\cd^{-1}$ nicht nur Nullstellen sondern auch Pole.
Somit entstehen bei den elliptischen rationalen Funktionen, wie es der name auch deutet, rationale Funktionen, also ein Bruch von zwei Polynomen.

Da Transformationen einer rationalen Funktionen mit Grundrechenarten, wie es in \eqref{ellfilter:eq:h_omega} der Fall ist, immer noch rationale Funktionen ergeben, stellt dies kein Problem für die Implementierung dar.
