%
% main.tex -- Paper zum Thema Elliptische Filter
%
% (c) 2020 Hochschule Rapperswil
%
\chapter{Elliptische Filter\label{chapter:ellfilter}}
\lhead{Elliptische Filter}
\begin{refsection}
\chapterauthor{Nicolas Tobler}


\section{Einleitung}

Lineare filter

Filter, Signalverarbeitung


Der womöglich wichtigste Filtertyp ist das Tiefpassfilter.
Dieses soll im Durchlassbereich unter der Grenzfrequenz $\Omega_p$ unverstärkt durchlassen und alle anderen Frequenzen vollständig auslöschen.

Bei der Implementierung von Filtern


In der Elektrotechnik führen Schaltungen mit linearen Bauelementen wie Kondensatoren, Spulen und Widerständen immer zu linearen zeitinvarianten Systemen (LTI-System von englich \textit{time-invariant system}).
Die Übertragungsfunktion im Frequenzbereich $|H(\Omega)|$ eines solchen Systems ist dabei immer eine rationale Funktion, also eine Division von zwei Polynomen.
Die Polynome habe dabei immer reelle oder komplex-konjugierte Nullstellen.


\begin{equation} \label{ellfilter:eq:h_omega}
    | H(\Omega)|^2 = \frac{1}{1 + \varepsilon_p^2 F_N^2(w)}, \quad w=\frac{\Omega}{\Omega_p}
\end{equation}

$\Omega = 2 \pi f$ ist die analoge Frequenz


% Linear filter
Damit das Filter implementierbar und stabil ist, muss $H(\Omega)^2$ eine rationale Funktion sein, deren Nullstellen und Pole auf der linken Halbebene liegen.

$N \in \mathbb{N} $ gibt dabei die Ordnung des Filters vor, also die maximale Anzahl Pole oder Nullstellen.

% In \eqref{ellfilter:eq:h_omega} wird $F_N(w)$ so verzogen, dass $F_N(w) \forall |w| < 1$


Damit ein Filter die Passband Kondition erfüllt muss $|F_N(w)| \leq 1 \forall |w| \leq 1$ und für $|w| \geq 1$ sollte die Funktion möglichst schnell divergieren.
Eine einfaches Polynom, dass das erfüllt, erhalten wir wenn $F_N(w) = w^N$.
Tatsächlich erhalten wir damit das Butterworth Filter, wie in Abbildung \ref{ellfilter:fig:butterworth} ersichtlich.
\begin{figure}
    \centering
    \includegraphics[scale=1]{papers/ellfilter/python/F_N_butterworth.pdf}
    \caption{$F_N$ für Butterworth filter. Der grüne Bereich definiert die erlaubten Werte für alle $F_N$-Funktionen.}
    \label{ellfilter:fig:butterworth}
\end{figure}

wenn $F_N(w)$ eine rationale Funktion ist, ist auch $H(\Omega)$ eine rationale Funktion und daher ein lineares Filter. %proof?

\begin{align}
    F_N(w) & =
    \begin{cases}
        w^N                            & \text{Butterworth} \\
        T_N(w)                         & \text{Tschebyscheff, Typ 1}  \\
        [k_1 T_N (k^{-1} w^{-1})]^{-1} & \text{Tschebyscheff, Typ 2}  \\
        R_N(w)                         & \text{Elliptisch (Cauer)}    \\
    \end{cases}
\end{align}

Mit der Ausnahme vom Butterworth filter sind alle Filter nach speziellen Funktionen benannt.
Alle diese Filter sind optimal für unterschiedliche Anwendungsgebiete.
Das Butterworth-Filter, zum Beispiel, ist maximal flach im Durchlassbereich.
Das Tschebyscheff-1 Filter sind maximal steil für eine definierte Welligkeit im Durchlassbereich, währendem es im Sperrbereich monoton abfallend ist.
Es scheint so als sind gewisse Eigenschaften dieser speziellen Funktionen verantwortlich für die Optimalität dieser Filter.

\section{Tschebyscheff-Filter}

Als Einstieg betrachent Wir das Tschebyscheff-Filter, welches sehr verwand ist mit dem elliptischen Filter.
Genauer ausgedrückt sind die Tschebyscheff-1 und -2 Fitler ein Spezialfall davon.

Der Name des Filters deutet schon an, dass die Tschebyschff-Polynome $T_N$ relevant sind für das Filter:
\begin{align}
    T_{0}(x)&=1\\
    T_{1}(x)&=x\\
    T_{2}(x)&=2x^{2}-1\\
    T_{3}(x)&=4x^{3}-3x\\
    T_{n+1}(x)&=2x~T_{n}(x)-T_{n-1}(x).
\end{align}
Bemerkenswert ist, dass die Polynome im Intervall $[-1, 1]$ mit der Trigonometrischen Funktion
\begin{equation} \label{ellfilter:eq:chebychef_polynomials}
    T_N(w) = \cos \left( N \cos^{-1}(w) \right)
\end{equation}
übereinstimmt.
Abbildung \ref{ellfilter:fig:chebychef_polynomials} zeigt einige Tschebyscheff-Polynome.
\begin{figure}
    \centering
    \includegraphics[scale=1]{papers/ellfilter/python/F_N_chebychev2.pdf}
    \caption{Die Tschebyscheff-Polynome $C_N$.}
    \label{ellfilter:fig:chebychef_polynomials}
\end{figure}
Da der Kosinus begrenzt zwischen $-1$ und $1$ ist, sind auch die Tschebyscheff-Polynome begrenzt.
Geht man aber über das Intervall $[-1, 1]$ hinaus, divergieren die Funktionen mit zunehmender Ordnung immer steiler gegen $\pm \infty$.
Diese Eigenschaft ist sehr nützlich für ein Filter.
Wenn wir die Tschebyscheff-Polynome quadrieren, passen sie perfekt in die Voraussetzungen für Filterfunktionen, wie es Abbildung \ref{ellfiter:fig:chebychef} demonstriert.
\begin{figure}
    \centering
    \includegraphics[scale=1]{papers/ellfilter/python/F_N_chebychev.pdf}
    \caption{Die Tschebyscheff-Polynome füllen den erlaubten Bereich besser, und erhalten dadurch eine steilere Flanke im Sperrbereich.}
    \label{ellfiter:fig:chebychef}
\end{figure}


Die analytische Fortsetzung von \eqref{ellfilter:eq:chebychef_polynomials} über das Intervall $[-1,1]$ hinaus stimmt mit den Polynomen überein, wie es zu erwarten ist.
Die genauere Betrachtung wird uns dann helfen die elliptischen Filter zu verstehen.

\begin{equation}
    \cos^{-1}(x)
    =
    \int_{0}^{x}
    \frac{
        dz
    }{
        \sqrt{
            1-z^2
        }
    }
\end{equation} %TOdO is it minus dz?

\begin{equation}
    \frac{
        1
    }{
        \sqrt{
            1-z^2
        }
    }
    \in \mathbb{R}
    \quad
    \forall
    \quad
    -1  \leq z \leq 1
\end{equation}
Wenn $|z|$ über 1 hinausgeht, wird der Term unter der Wurzel negativ.
Durch die Quadratwurzel entstehen zwei Reinkomplexe Lösungen
\begin{equation}
    \frac{
        1
    }{
        \sqrt{
            1-z^2
        }
    }
    = i \xi \quad | \quad \xi \in \mathbb{R}
    \quad
    \forall
    \quad
    z \leq -1 \cup z \geq 1
\end{equation}

\begin{figure}
    \centering
    \begin{tikzpicture}[>=stealth', auto, node distance=2cm, scale=1.2]


    \draw[gray, ->] (0,-2) -- (0,2) node[anchor=south]{Im $z$};
    \draw[gray, ->] (-5,0) -- (5,0) node[anchor=west]{Re $z$};

    \begin{scope}
        \draw[thick, ->, orange] (-1, 0) -- (0,0);
        \draw[thick, ->, darkgreen] (0, 0) -- (0,1.5);
        \draw[thick, ->, darkgreen] (0, 0) -- (0,-1.5);
        \draw[thick, ->, orange] (1, 0) -- (0,0);
        \draw[thick, ->, red] (2, 0) -- (1,0);
        \draw[thick, ->, blue] (2,1.5) -- (2, 0);
        \draw[thick, ->, blue] (2,-1.5) -- (2, 0);
        \draw[thick, ->, red] (2, 0) -- (3,0);

        \node[anchor=south west] at (0,1.5) {$\infty$};
        \node[anchor=south west] at (0,-1.5) {$\infty$};
        \node[anchor=south west] at (0,0) {$1$};
        \node[anchor=south] at (1,0) {$0$};
        \node[anchor=south west] at (2,0) {$-1$};
        \node[anchor=south west] at (2,1.5) {$-\infty$};
        \node[anchor=south west] at (2,-1.5) {$-\infty$};
        \node[anchor=south west] at (3,0) {$0$};
    \end{scope}

    \begin{scope}[xshift=4cm]
        \draw[thick, ->, orange] (-1, 0) -- (0,0);
        \draw[thick, ->, darkgreen] (0, 0) -- (0,1.5);
        \draw[thick, ->, darkgreen] (0, 0) -- (0,-1.5);
        % \draw[thick, ->, orange] (1, 0) -- (0,0);
        % \draw[thick, ->, red] (2, 0) -- (1,0);
        % \draw[thick, ->, blue] (2,1.5) -- (2, 0);
        % \draw[thick, ->, blue] (2,-1.5) -- (2, 0);
        % \draw[thick, ->, red] (2, 0) -- (3,0);

        \node[anchor=south west] at (0,1.5) {$\infty$};
        \node[anchor=south west] at (0,-1.5) {$\infty$};
        \node[anchor=south west] at (0,0) {$1$};
        % \node[anchor=south] at (1,0) {$0$};
        % \node[anchor=south west] at (2,0) {$-1$};
        % \node[anchor=south west] at (2,1.5) {$-\infty$};
        % \node[anchor=south west] at (2,-1.5) {$-\infty$};
        % \node[anchor=south west] at (3,0) {$0$};
    \end{scope}

    \begin{scope}[xshift=-4cm]
        % \draw[thick, ->, orange] (-1, 0) -- (0,0);
        \draw[thick, ->, darkgreen] (0, 0) -- (0,1.5);
        \draw[thick, ->, darkgreen] (0, 0) -- (0,-1.5);
        \draw[thick, ->, orange] (1, 0) -- (0,0);
        \draw[thick, ->, red] (2, 0) -- (1,0);
        \draw[thick, ->, blue] (2,1.5) -- (2, 0);
        \draw[thick, ->, blue] (2,-1.5) -- (2, 0);
        \draw[thick, ->, red] (2, 0) -- (3,0);

        \node[anchor=south west] at (0,1.5) {$\infty$};
        \node[anchor=south west] at (0,-1.5) {$\infty$};
        \node[anchor=south west] at (0,0) {$1$};
        \node[anchor=south] at (1,0) {$0$};
        \node[anchor=south west] at (2,0) {$-1$};
        \node[anchor=south west] at (2,1.5) {$-\infty$};
        \node[anchor=south west] at (2,-1.5) {$-\infty$};
        \node[anchor=south west] at (3,0) {$0$};
    \end{scope}

    \node[gray, anchor=north west] at (-4,0) {$-2\pi$};
    \node[gray, anchor=north west] at (-2,0) {$-\pi$};
    \node[gray, anchor=north west] at (0,0) {$0$};
    \node[gray, anchor=north west] at (2,0) {$\pi$};
    \node[gray, anchor=north west] at (4,0) {$2\pi$};


    \node[gray, anchor=south east] at (0,-1.5) {$-\infty$};
    \node[gray, anchor=south east] at (0, 0) {$0$};
    \node[gray, anchor=south east] at (0, 1.5) {$\infty$};



    \begin{scope}[yshift=-2.5cm]

        \draw[gray, ->] (-5,0) -- (5,0) node[anchor=west]{$w$};

        \draw[thick, ->, blue]      (-4, 0) -- (-2, 0);
        \draw[thick, ->, red]       (-2, 0) -- (0, 0);
        \draw[thick, ->, orange]    (0, 0) -- (2, 0);
        \draw[thick, ->, darkgreen] (2, 0) -- (4, 0);

        \node[anchor=south] at (-4,0) {$-\infty$};
        \node[anchor=south] at (-2,0) {$-1$};
        \node[anchor=south] at (0,0) {$0$};
        \node[anchor=south] at (2,0) {$1$};
        \node[anchor=south] at (4,0) {$\infty$};

    \end{scope}

\end{tikzpicture}
    \caption{Die Funktion $z = \cos^{-1}(w)$ dargestellt in der komplexen ebene.}
    \label{ellfilter:fig:arccos}
\end{figure}



\begin{figure}
    \centering
    \begin{tikzpicture}[>=stealth', auto, node distance=2cm, scale=1.2]

    \tikzstyle{zero} = [draw, circle, inner sep =0, minimum height=0.15cm]

    \tikzset{pole/.style={cross out, draw=black, minimum size=(0.15cm-\pgflinewidth), inner sep=0pt, outer sep=0pt}}

    \begin{scope}[xscale=0.5]

        \draw[gray, ->] (0,-2) -- (0,2) node[anchor=south]{Im $z$};
        \draw[gray, ->] (-10,0) -- (10,0) node[anchor=west]{Re $z$};

        \begin{scope}

            \draw[>->, line width=0.05, thick, blue]   (2, 1.5) -- (2,0.05)  -- node[anchor=south, pos=0.5]{$N=1$} (0.1,0.05) -- (0.1,1.5);
            \draw[>->, line width=0.05, thick, orange] (4, 1.5) -- (4,0)     -- node[anchor=south, pos=0.25]{$N=2$} (0,0) -- (0,1.5);
            \draw[>->, line width=0.05, thick, red]    (6, 1.5) -- (6,-0.05) -- node[anchor=south, pos=0.1666]{$N=3$} (-0.1,-0.05) -- (-0.1,1.5);


            \node[zero] at (-7,0) {};
            \node[zero] at (-5,0) {};
            \node[zero] at (-3,0) {};
            \node[zero] at (-1,0) {};
            \node[zero] at (1,0) {};
            \node[zero] at (3,0) {};
            \node[zero] at (5,0) {};
            \node[zero] at (7,0) {};


        \end{scope}

        \node[gray, anchor=north] at (-8,0) {$-4\pi$};
        \node[gray, anchor=north] at (-6,0) {$-3\pi$};
        \node[gray, anchor=north] at (-4,0) {$-2\pi$};
        \node[gray, anchor=north] at (-2,0) {$-\pi$};
        \node[gray, anchor=north] at (2,0) {$\pi$};
        \node[gray, anchor=north] at (4,0) {$2\pi$};
        \node[gray, anchor=north] at (6,0) {$3\pi$};
        \node[gray, anchor=north] at (8,0) {$4\pi$};


        \node[gray, anchor=east] at (0,-1.5) {$-\infty$};
        \node[gray, anchor=east] at (0, 1.5) {$\infty$};

    \end{scope}

\end{tikzpicture}
    \caption{
        $z$-Ebene der Tschebyscheff-Funktion.
        Je grösser die Ordnung $N$ gewählt wird, desto mehr Nullstellen werden hat das Tschebyscheff-Polynom.
    }
    % \label{ellfilter:fig:arccos}
\end{figure}





% Analytische Fortsetzung



\section{Jacobische elliptische Funktionen}


Für das elliptische Filter, wird statt der für das Tschebyscheff-Filter benutzen Kreis-Trigonometrie die elliptischen Funktionen gebraucht.
Der begriff elliptische Funktion wird für sehr viele Funktionen gebraucht, daher ist es hier wichtig zu erwähnen, dass es ausschliesslich um die Jacobischen elliptischen Funktionen geht.

Im Wesentlichen erweitern die Jacobi elliptischen Funktionen die trigonometrische Funktionen für Ellipsen.

%TODO $z$ or $u$ for parameter?

neu zwei parameter
$sn(z, k)$
$z$ ist das winkelargument
Im Kreis ist der Radius für alle Winkel konstant, bei Ellipsen ändert sich das.
Dies hat zur Folge, dass bei einer Ellipse die Kreisbodenstrecke nicht linear zum Winkel verläuft.
Darum kann hier nicht der gewohnte Winkel verwendet werden.
An deren stelle kommt der parameter $k$ zum Einsatz, welcher durch das elliptische Integral erster Art
\begin{equation}
    z
    =
    F(\phi, k)
    =
    \int_{0}^{\phi}
    \frac{
        d\theta
    }{
        \sqrt{
            1-k^2 \sin^2 \theta
        }
    }
\end{equation}
mit dem Winkel $\phi$ in Verbindung liegt.




Dabei wird das vollständige und unvollständige Elliptische integral unterschieden.
Beim vollständigen Integral
\begin{equation}
    K(k)
    =
    \int_{0}^{\pi / 2}
    \frac{
        d\theta
    }{
        \sqrt{
            1-k^2 \sin^2 \theta
        }
    }
\end{equation}
wird über ein viertel Ellipsenbogen integriert also bis $\phi=\pi/2$.

Die Jacobischen elliptischen Funktionen können mit der inversen Funktion
\begin{equation}
    \phi = F^{-1}(z, k)
\end{equation}
definiert werden. Dabei ist zu beachten dass nur das $z$ Argument der Funktion invertiert wird also
\begin{equation}
    z = F(\phi, k)
    \Leftrightarrow
    \phi = F^{-1}(z, k).
\end{equation}
Mithilfe von $F^{-1}$ kann $sn^{-1}$ mit dem Elliptischen integral dargestellt werden:
\begin{equation}
    \sin(\phi)
    =
    \sin \left( F^{-1}(z, k) \right)
    =
    \sn(u, k)
\end{equation}

\begin{align}
    \sn^{-1}(w, k)
        & =
    \int_{0}^{\phi}
    \frac{
        d\theta
    }{
        \sqrt{
            1-k^2 \sin^2 \theta
        }
    },
    \quad
    \phi = \sin^{-1}(w)
    \\
        & =
    \int_{0}^{w}
    \frac{
        dt
    }{
        \sqrt{
            (1-t^2)(1-k^2 t^2)
        }
    }
\end{align}

Beim $\cos^{-1}(x)$ haben wir gesehen, dass die analytische Fortsetzung bei $x < -1$ und $x > 1$ rechtwinklig in die Komplexen zahlen wandert.
Wenn man das gleiche mit $\sn^{-1}(w, k)$ macht, erkennt man zwei interessante Stellen.
Die erste ist die gleiche wie beim $\cos^{-1}(x)$ nämlich bei $t = \pm 1$.
Der erste Term unter der Wurzel wird dann negativ, während der zweite noch positiv ist, da $k \leq 1$.
\begin{equation}
    \frac{
        1
    }{
        \sqrt{
            (1-t^2)(1-k^2 t^2)
        }
    }
    \in \mathbb{R}
    \quad \forall \quad
    -1 \leq t \leq 1
\end{equation}
Die zweite stelle passiert wenn beide Faktoren unter der Wurzel negativ werden, was bei $t = 1/k$ der Fall ist.




Funktion in relle und komplexe Richtung periodisch

In der reellen Richtung ist sie $4K(k)$-periodisch und in der imaginären Richtung $4K^\prime(k)$-periodisch.



%TODO sn^{-1} grafik


\section{Elliptische rationale Funktionen}


\begin{equation}
    R_N(\xi, w) = \cd \left(N~f_1(\xi)~\cd^{-1}(w, 1/\xi), f_2(\xi)\right)
\end{equation}
\begin{equation}
    R_N(\xi, w) = \cd (N~u K_1, k_1), \quad w= \cd(uK, k)
\end{equation}


sieht ähnlich aus wie die trigonometrische darstellung der Tschebyschef-Polynome

der Ordnungszahl $N$ kommt auch als Faktor for 

%TODO cd^{-1} grafik mit 


\subsection{Degree Equation}

Der $cd^{-1}$ Term muss so verzogen werden, dass die umgebene $cd$ funktion die nullstellen und pole trifft.
Dies trifft ein wenn die Degree Equation erfüllt ist.

\begin{equation}
    N \frac{K^\prime}{K} = \frac{K^\prime_1}{K_1}
\end{equation}


Leider ist das lösen dieser Gleichung nicht trivial.
Die Rechnung wird in \ref{ellfilter:bib:orfanidis} im Detail angeschaut.


\subsection{Polynome?}

Bei den Tschebyscheff-Polynomen haben wir gesehen, dass die Trigonometrische Formel zu einfachen Polynomen umgewandelt werden kann.
Im gegensatz zum $\cos^{-1}$ hat der $\cd^{-1}$ nicht nur Nullstellen sondern auch Pole.
Somit entstehen bei den elliptischen rationalen Funktionen, wie es der name auch deutet, rationale Funktionen, also ein Bruch von zwei Polynomen.




\begin{figure}
    \centering
    \includegraphics[scale=1]{papers/ellfilter/python/F_N_elliptic.pdf}
    \caption{$F_N$ für ein elliptischs filter.}
    \label{ellfilter:fig:elliptic}
\end{figure}





%
% teil0.tex -- Definition
%
% (c) 2020 Prof Dr Andreas Müller, Hochschule Rapperswil
%
\section{Definition\label{fresnel:section:teil0}}
\kopfrechts{Definition}
Die Funktion $e^{x^2}$ hat bekanntermassen keine elementare Stammfunktion,
weshalb die Fehlerfunktion als Stammfunktion definiert wurde.
Die Funktionen $\cos x^2$ und $\sin x^2$ sind eng mit $e^{x^2}$
verwandt, es ist daher nicht überraschend, dass sie ebenfalls
keine elementare Stammfunktionen haben.
Dies rechtfertigt die Definition der Fresnel-Integrale als neue spezielle
Funktionen.

\begin{definition}
Die Funktionen 
\begin{align*}
C(x) &= \int_0^x \cos\biggl(\frac{\pi}2 t^2\biggr)\,dt
\\
S(x) &= \int_0^x \sin\biggl(\frac{\pi}2 t^2\biggr)\,dt
\end{align*}
heissen die Fresnel-Integrale.
\end{definition}

Der Faktor $\frac{\pi}2$ ist einigermassen willkürlich, man könnte
daher noch allgemeiner die Funktionen
\begin{align*}
C_a(x) &= \int_0^x \cos(at^2)\,dt
\\
S_a(x) &= \int_0^x \sin(at^2)\,dt
\end{align*}
definieren, so dass die Funktionen $C(x)$ und $S(x)$ der Fall
$a=\frac{\pi}2$ werden, also
\[
\begin{aligned}
C(x) &= C_{\frac{\pi}2}(x),
&
S(x) &= S_{\frac{\pi}2}(x).
\end{aligned}
\]
Durch eine Substitution $t=bs$ erhält man
\begin{align*}
C_a(x)
&=
\int_0^x \cos(at^2)\,dt
=
b
\int_0^{\frac{x}b} \cos(ab^2s^2)\,ds
=
b
C_{ab^2}\biggl(\frac{x}b\biggr)
\\
S_a(x)
&=
\int_0^x \sin(at^2)\,dt
=
b
\int_0^{\frac{x}b} \sin(ab^2s^2)\,ds
=
b
S_{ab^2}\biggl(\frac{x}b\biggr).
\end{align*}
Indem man $ab^2=\frac{\pi}2$ setzt, also
\[
b
=
\sqrt{\frac{\pi}{2a}}
,
\]
kann man die Funktionen $C_a(x)$ und $S_a(x)$ durch $C(x)$ und $S(x)$
ausdrücken:
\begin{align}
C_a(x)
&=
\sqrt{\frac{\pi}{2a}}
C\biggl(x
\sqrt{\frac{2a}{\pi}}
\biggr)
&&\text{und}&
S_a(x)
&=
\sqrt{\frac{\pi}{2a}}
S\biggl(x
\sqrt{\frac{2a}{\pi}}
\biggr).
\label{fresnel:equation:arg}
\end{align}
Im Folgenden werden wir meistens nur den Fall $a=1$, also die Funktionen
$C_1(x)$ und $S_1(x)$ betrachten, da in diesem Fall die Formeln einfacher
werden.
\begin{figure}
\centering
\includegraphics{papers/fresnel/images/fresnelgraph.pdf}
\caption{Graph der Funktionen $C(x)$ ({\color{red}rot}) 
und $S(x)$ ({\color{blue}blau})
\label{fresnel:figure:plot}}
\end{figure}
Die Abbildung~\ref{fresnel:figure:plot} zeigt die Graphen der
Funktion $C(x)$ und $S(x)$.


%
% teil1.tex -- Beispiel-File für das Paper
%
% (c) 2020 Prof Dr Andreas Müller, Hochschule Rapperswil
%
\section{Ordnungsstatistik und Beta-Funktion
\label{dreieck:section:ordnungsstatistik}}
\rhead{}




%
% teil2.tex -- Beispiel-File für teil2 
%
% (c) 2020 Prof Dr Andreas Müller, Hochschule Rapperswil
%
\section{Anwendung in der Physik 
\label{parzyl:section:teil2}}
\rhead{Anwendung in der Physik}

Die parabolischen Zylinderkoordinaten tauchen auf, wenn man das elektrische Feld einer semi-infiniten Platte, wie in Abbildung \ref{parzyl:fig:leiterplatte} gezeigt, finden will.
\begin{figure}
	 \centering
	\includegraphics[width=0.9\textwidth]{papers/parzyl/img/plane.pdf}
	\caption{Semi-infinite Leiterplatte}
	\label{parzyl:fig:leiterplatte}
\end{figure}
Das dies so ist kann im zwei Dimensionalen mit Hilfe von komplexen Funktionen gezeigt werden. Die Platte ist dann nur eine Linie, was man in Abbildung TODO sieht.
Jede komplexe Funktion $F(z)$ kann geschrieben werden als
\begin{equation}
	F(s) = U(x,y) + iV(x,y) \qquad s \in \mathbb{C}; x,y \in \mathbb{R}.
\end{equation}  
Dabei muss gelten, falls die Funktion differenzierbar ist, dass
\begin{equation}
	\frac{\partial U(x,y)}{\partial x} 
	=
	\frac{\partial V(x,y)}{\partial y} 
	\qquad
	\frac{\partial V(x,y)}{\partial x}
	=
	-\frac{\partial U(x,y)}{\partial y}.
\end{equation}
Aus dieser Bedingung folgt 
\begin{equation}
	\label{parzyl_e_feld_zweite_ab}
	\underbrace{
	\frac{\partial^2 U(x,y)}{\partial x^2}
	+ 
	\frac{\partial^2 U(x,y)}{\partial y^2}
	=
	0
	}_{\displaystyle{\nabla^2U(x,y)=0}}
	\qquad
	\underbrace{
	\frac{\partial^2 V(x,y)}{\partial x^2}
	+
	\frac{\partial^2 V(x,y)}{\partial y^2}
	=
	0
	}_{\displaystyle{\nabla^2V(x,y) = 0}}.
\end{equation}
Zusätzlich kann auch gezeigt werden, dass die Funktion $F(z)$ eine winkeltreue Abbildung ist. 
Der Zusammenhang zum elektrischen Feld ist jetzt, dass das Potential an einem quellenfreien Punkt gegeben ist als 
\begin{equation}
	\nabla^2\phi(x,y) = 0.
\end{equation}
Dies ist eine Bedingung welche differenzierbare Funktionen, wie in Gleichung \ref{parzyl_e_feld_zweite_ab} gezeigt wird, bereits besitzen. 
Nun kann zum Beispiel $U(x,y)$ als das Potential angeschaut werden
\begin{equation}
	\phi(x,y) = U(x,y).
\end{equation}
Orthogonal zum Potential ist das elektrische Feld
\begin{equation}
	E(x,y) = V(x,y).
\end{equation}
Um nun zu den parabolische Zylinderkoordinaten zu gelangen muss nur noch eine geeignete komplexe Funktion $F(s)$ gefunden werden, 
welche eine semi-infinite Platte beschreiben kann.
Die gesuchte Funktion in diesem Fall ist
\begin{equation}
	F(s) 
	= 
	\sqrt{s} 
	= 
	\sqrt{x + iy}.
\end{equation}
Dies kann umgeformt werden zu
\begin{equation}
	F(s) 
	= 
	\underbrace{\sqrt{\frac{\sqrt{x^2+y^2} + x}{2}}}_{U(x,y)} 
	+ 
	i\underbrace{\sqrt{\frac{\sqrt{x^2+y^2} - x}{2}}}_{V(x,y)}
	.
\end{equation}
Die Äquipotentialflächen können nun betrachtet werden, indem man die Funktion welche das Potential beschreibt gleich eine Konstante setzt,
\begin{equation}
	\sigma = U(x,y) = \sqrt{\frac{\sqrt{x^2+y^2} + x}{2}},
\end{equation}
und die Flächen mit der gleichen elektrischen Feldstärke können als
\begin{equation}
	\tau = V(x,y) = \sqrt{\frac{\sqrt{x^2+y^2} - x}{2}}
\end{equation}
beschrieben werden. Diese zwei Gleichungen zeigen nun wie man vom kartesischen Koordinatensystem ins parabolische Zylinderkoordinatensystem kommt. Werden diese Formeln nun nach x und y aufgelöst so beschreibe sie, wie man aus dem parabolischen Zylinderkoordinatensystem zurück ins kartesische rechnen kann
\begin{equation}
	x = \sigma \tau,
\end{equation}
\begin{equation}
	y = \frac{1}{2}\left ( \tau^2 - \sigma^2 \right )
\end{equation}






%
% teil3.tex -- Resultate und Ausblick
%
% (c) 2022 Fabian Dünki, Hochschule Rapperswil
%
\section{Auswertung
\label{0f1:section:teil3}}
\rhead{Resultate}
Im Verlauf dieser Arbeit hat sich gezeigt, 
das ein einfacher mathematischer Algorithmus zu implementieren gar nicht so einfach ist.
So haben alle drei umgesetzten Ansätze Probleme mit grossen negativen $z$ in der Funktion $\mathstrut_0F_1(;c;z)$.
Ebenso kann festgestellt werden, dass je grösser der Wert $z$ in $\mathstrut_0F_1(;c;z)$ wird, desto mehr weichen die berechneten Resultate von den Erwarteten \cite{0f1:wolfram-0f1} ab.

\subsection{Konvergenz
\label{0f1:subsection:konvergenz}}
Es zeigt sich in Abbildung \ref{0f1:ausblick:plot:airy:konvergenz}, dass schon nach drei Iterationen ($k = 3$) die Funktionen schon genaue Resultate im Bereich von $-2$ bis $2$ liefert. Ebenso kann festgestellt werden, dass der Kettenbruch schneller konvergiert und im positiven Bereich sogar mit der Referenzfunktion $\operatorname{Ai}(x)$ übereinstimmt. Da die Rekursionsformel eine Abwandlung des Kettenbruches ist, verhalten sich die Funktionen in diesem Fall gleich.

Erst wenn mehrerer Iterationen gemacht werden, um die Genauigkeit zu verbessern, ist der Kettenbruch den anderen zwei Algorithmen, bezüglich Konvergenz überlegen. 
Interessant ist auch, dass die Rekursionsformel nahezu gleich schnell wie die Potenzreihe konvergiert, aber sich danach, wie in Abbildung \ref{0f1:ausblick:plot:konvergenz:positiv} zu beobachten ist, einschwingt. Dieses Verhalten ist auch bei grösseren $z$ zu beobachten, allerdings ist dann die Differenz zwischen dem ersten lokalen Minimum von k bis zum Abbruch kleiner.
Dieses Phänomen ist auf die Lösung der Rekursionsformel \eqref{0f1:math:loesung:eq} zurück zu führen. Da im Gegensatz die ganz kleinen Werte nicht zu einer Konvergenz wie beim Kettenbruch führen, sondern sich noch eine Zeit lang durch die Multiplikation aufschwingen.

Ist $z$ negativ wie im Abbildung \ref{0f1:ausblick:plot:konvergenz:negativ}, führt dies zu einer Gegenseitigen Kompensation von negativen und positiven Termen so bricht die Rekursionsformel hier zusammen mit der Potenzreihe ab.
Die ansteigende Differenz mit anschliessender, ist aufgrund der sich alternierenden Termen mit wechselnden Vorzeichens zu erklären.

\subsection{Stabilität
\label{0f1:subsection:Stabilitaet}}
Verändert sich der Wert von z in $\mathstrut_0F_1(;c;z)$ gegen grössere positive Werte, wie zum Beispiel $c = 800$ liefert die Kettenbruch-Funktion (Listing \ref{0f1:listing:kettenbruchIterativ}) \verb+inf+ zurück. Dies könnte durch ein Abbruchkriterien abgefangen werden. Allerdings würde das, bei grossen Werten zulasten der Genauigkeit gehen. Trotzdem könnte, je nach Anwendung, auf ein paar Nachkommastellen verzichtet werden.

Wohingegen die Potenzreihe (Listing \ref{0f1:listing:potenzreihe}) das Problem hat, dass je mehr Terme berechnet werden, desto schneller wächst die Fakultät und irgendwann gibt es eine Bereichsüberschreitung von \verb+double+. Schlussendlich gibt das Unterprogramm das Resultat \verb+-nan(ind)+ zurück.
Die Rekursionformel \eqref{0f1:listing:kettenbruchRekursion} liefert für sehr grosse positive Werte die genausten Ergebnisse, verglichen mit der GNU Scientific Library. Wie schon vermutet ist die Rekursionsformel, im positivem Bereich, der stabilste Algorithmus. Um die Stabilität zu gewährleisten, muss wie in Abbildung \ref{0f1:ausblick:plot:konvergenz:positiv} dargestellt, die Iterationstiefe $k$ genug gross gewählt werden.

Im negativem Bereich sind alle gewählten und umgesetzten Ansätze instabil. Grund dafür ist die Potenz von z, was zum Phänomen der Auslöschung \cite{0f1:SeminarNumerik} führt. Schön zu beobachten ist dies in der Abbildung \ref{0f1:ausblick:plot:airy:stabilitaet} mit der Airy-Funktion als Test. So sind sowohl der Kettenbruch, als auch die Rekursionsformel bis ungefähr $\frac{-15^3}{9}$ stabil. Dies macht auch Sinn, da beide auf der gleichen mathematischen Grundlage basieren. Danach verhält sich allerdings die Instabilität unterschiedlich. Das unterschiedliche Verhalten kann damit erklärt werden, dass beim Kettenbruch jeweils eine zusätzliche Division stattfindet. Diese Unterschiede sind auch in Abbildung \ref{0f1:ausblick:plot:konvergenz:positiv} festzustellen.



\begin{figure}
    \centering
    \includegraphics[width=0.8\textwidth]{papers/0f1/images/konvergenzAiry.pdf}
    \caption{Konvergenz nach drei Iterationen, dargestellt anhand der Airy Funktion zu den Anfangsbedingungen $\operatorname{Ai}(0)=1$ und $\operatorname{Ai}'(0)=0$.
    \label{0f1:ausblick:plot:airy:konvergenz}}
\end{figure}

\begin{figure}
    \centering
    \includegraphics[width=0.8\textwidth]{papers/0f1/images/konvergenzPositiv.pdf}
    \caption{Konvergenz mit positivem z; Logarithmisch dargestellte Differenz vom erwarteten Endresultat.
    \label{0f1:ausblick:plot:konvergenz:positiv}}
\end{figure}

\begin{figure}
    \centering
    \includegraphics[width=0.8\textwidth]{papers/0f1/images/konvergenzNegativ.pdf}
    \caption{Konvergenz mit negativem z; Logarithmisch dargestellte Differenz vom erwarteten Endresultat.
    \label{0f1:ausblick:plot:konvergenz:negativ}}
\end{figure}

\begin{figure}
    \centering
    \includegraphics[width=1\textwidth]{papers/0f1/images/stabilitaet.pdf}
    \caption{Stabilität der 3 Algorithmen verglichen mit der Referenz Funktion $\operatorname{Ai}(x)$.
    \label{0f1:ausblick:plot:airy:stabilitaet}}
\end{figure}



% \printbibliography[heading=subbibliography]
\end{refsection}
