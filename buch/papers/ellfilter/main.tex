%
% main.tex -- Paper zum Thema Elliptische Filter
%
% (c) 2020 Hochschule Rapperswil
%
\chapter{Elliptische Filter\label{chapter:ellfilter}}
\lhead{Elliptische Filter}
\begin{refsection}
\chapterauthor{Nicolas Tobler}


\section{Einleitung}

Lineare filter

Filter, Signalverarbeitung


Der womöglich wichtigste Filtertyp ist das Tiefpassfilter.
Dieses soll im Durchlassbereich unter der Grenzfrequenz $\Omega_p$ unverstärkt durchlassen und alle anderen Frequenzen vollständig auslöschen.

Bei der Implementierung von Filtern


In der Elektrotechnik führen Schaltungen mit linearen Bauelementen wie Kondensatoren, Spulen und Widerständen immer zu linearen zeitinvarianten Systemen (LTI-System von englich \textit{time-invariant system}).
Die Übertragungsfunktion im Frequenzbereich $|H(\Omega)|$ eines solchen Systems ist dabei immer eine rationale Funktion, also eine Division von zwei Polynomen.
Die Polynome habe dabei immer reelle oder komplex-konjugierte Nullstellen.


\begin{equation} \label{ellfilter:eq:h_omega}
    | H(\Omega)|^2 = \frac{1}{1 + \varepsilon_p^2 F_N^2(w)}, \quad w=\frac{\Omega}{\Omega_p}
\end{equation}

$\Omega = 2 \pi f$ ist die analoge Frequenz


% Linear filter
Damit das Filter implementierbar und stabil ist, muss $H(\Omega)^2$ eine rationale Funktion sein, deren Nullstellen und Pole auf der linken Halbebene liegen.

$N \in \mathbb{N} $ gibt dabei die Ordnung des Filters vor, also die maximale Anzahl Pole oder Nullstellen.

% In \eqref{ellfilter:eq:h_omega} wird $F_N(w)$ so verzogen, dass $F_N(w) \forall |w| < 1$


Damit ein Filter die Passband Kondition erfüllt muss $|F_N(w)| \leq 1 \forall |w| \leq 1$ und für $|w| \geq 1$ sollte die Funktion möglichst schnell divergieren.
Eine einfaches Polynom, dass das erfüllt, erhalten wir wenn $F_N(w) = w^N$.
Tatsächlich erhalten wir damit das Butterworth Filter, wie in Abbildung \ref{ellfilter:fig:butterworth} ersichtlich.
\begin{figure}
    \centering
    \includegraphics[scale=1]{papers/ellfilter/python/F_N_butterworth.pdf}
    \caption{$F_N$ für Butterworth filter. Der grüne Bereich definiert die erlaubten Werte für alle $F_N$-Funktionen.}
    \label{ellfilter:fig:butterworth}
\end{figure}

wenn $F_N(w)$ eine rationale Funktion ist, ist auch $H(\Omega)$ eine rationale Funktion und daher ein lineares Filter. %proof?

\begin{align}
    F_N(w) & =
    \begin{cases}
        w^N                            & \text{Butterworth} \\
        T_N(w)                         & \text{Tschebyscheff, Typ 1}  \\
        [k_1 T_N (k^{-1} w^{-1})]^{-1} & \text{Tschebyscheff, Typ 2}  \\
        R_N(w)                         & \text{Elliptisch (Cauer)}    \\
    \end{cases}
\end{align}

Mit der Ausnahme vom Butterworth filter sind alle Filter nach speziellen Funktionen benannt.
Alle diese Filter sind optimal für unterschiedliche Anwendungsgebiete.
Das Butterworth-Filter, zum Beispiel, ist maximal flach im Durchlassbereich.
Das Tschebyscheff-1 Filter sind maximal steil für eine definierte Welligkeit im Durchlassbereich, währendem es im Sperrbereich monoton abfallend ist.
Es scheint so als sind gewisse Eigenschaften dieser speziellen Funktionen verantwortlich für die Optimalität dieser Filter.

\section{Tschebyscheff-Filter}

Als Einstieg betrachent Wir das Tschebyscheff-Filter, welches sehr verwand ist mit dem elliptischen Filter.
Genauer ausgedrückt sind die Tschebyscheff-1 und -2 Fitler ein Spezialfall davon.

Der Name des Filters deutet schon an, dass die Tschebyschff-Polynome $T_N$ relevant sind für das Filter:
\begin{align}
    T_{0}(x)&=1\\
    T_{1}(x)&=x\\
    T_{2}(x)&=2x^{2}-1\\
    T_{3}(x)&=4x^{3}-3x\\
    T_{n+1}(x)&=2x~T_{n}(x)-T_{n-1}(x).
\end{align}
Bemerkenswert ist, dass die Polynome im Intervall $[-1, 1]$ mit der Trigonometrischen Funktion
\begin{equation} \label{ellfilter:eq:chebychef_polynomials}
    T_N(w) = \cos \left( N \cos^{-1}(w) \right)
\end{equation}
übereinstimmt.
Abbildung \ref{ellfilter:fig:chebychef_polynomials} zeigt einige Tschebyscheff-Polynome.
\begin{figure}
    \centering
    \includegraphics[scale=1]{papers/ellfilter/python/F_N_chebychev2.pdf}
    \caption{Die Tschebyscheff-Polynome $C_N$.}
    \label{ellfilter:fig:chebychef_polynomials}
\end{figure}
Da der Kosinus begrenzt zwischen $-1$ und $1$ ist, sind auch die Tschebyscheff-Polynome begrenzt.
Geht man aber über das Intervall $[-1, 1]$ hinaus, divergieren die Funktionen mit zunehmender Ordnung immer steiler gegen $\pm \infty$.
Diese Eigenschaft ist sehr nützlich für ein Filter.
Wenn wir die Tschebyscheff-Polynome quadrieren, passen sie perfekt in die Voraussetzungen für Filterfunktionen, wie es Abbildung \ref{ellfiter:fig:chebychef} demonstriert.
\begin{figure}
    \centering
    \includegraphics[scale=1]{papers/ellfilter/python/F_N_chebychev.pdf}
    \caption{Die Tschebyscheff-Polynome füllen den erlaubten Bereich besser, und erhalten dadurch eine steilere Flanke im Sperrbereich.}
    \label{ellfiter:fig:chebychef}
\end{figure}


Die analytische Fortsetzung von \eqref{ellfilter:eq:chebychef_polynomials} über das Intervall $[-1,1]$ hinaus stimmt mit den Polynomen überein, wie es zu erwarten ist.
Die genauere Betrachtung wird uns dann helfen die elliptischen Filter zu verstehen.

\begin{equation}
    \cos^{-1}(x)
    =
    \int_{0}^{x}
    \frac{
        dz
    }{
        \sqrt{
            1-z^2
        }
    }
\end{equation} %TOdO is it minus dz?

\begin{equation}
    \frac{
        1
    }{
        \sqrt{
            1-z^2
        }
    }
    \in \mathbb{R}
    \quad
    \forall
    \quad
    -1  \leq z \leq 1
\end{equation}
Wenn $|z|$ über 1 hinausgeht, wird der Term unter der Wurzel negativ.
Durch die Quadratwurzel entstehen zwei Reinkomplexe Lösungen
\begin{equation}
    \frac{
        1
    }{
        \sqrt{
            1-z^2
        }
    }
    = i \xi \quad | \quad \xi \in \mathbb{R}
    \quad
    \forall
    \quad
    z \leq -1 \cup z \geq 1
\end{equation}

\begin{figure}
    \centering
    \begin{tikzpicture}[>=stealth', auto, node distance=2cm, scale=1.2]

    \tikzstyle{zero} = [draw, circle, inner sep =0, minimum height=0.15cm]
    \tikzset{pole/.style={cross out, draw=black, minimum size=(0.15cm-\pgflinewidth), inner sep=0pt, outer sep=0pt}}

    \draw[gray, ->] (0,-2) -- (0,2) node[anchor=south]{$\mathrm{Im}~z$};
    \draw[gray, ->] (-5,0) -- (5,0) node[anchor=west]{$\mathrm{Re}~z$};

    \begin{scope}[xscale=0.6]

        \clip(-7.5,-2) rectangle (7.5,2);

        \draw[thick, ->, darkgreen] (0, 0) -- (0,1.5);
        \draw[thick, ->, orange] (1, 0) -- (0,0);
        \draw[thick, ->, red] (2, 0) -- (1,0);
        \draw[thick, ->, blue] (2,1.5) -- (2, 0);

        \foreach \i in {-2,...,1} {
            \begin{scope}[opacity=0.5, xshift=\i*4cm]
                \draw[->, orange] (-1, 0) -- (0,0);
                \draw[->, darkgreen] (0, 0) -- (0,1.5);
                \draw[->, darkgreen] (0, 0) -- (0,-1.5);
                \draw[->, orange] (1, 0) -- (0,0);
                \draw[->, red] (2, 0) -- (1,0);
                \draw[->, blue] (2,1.5) -- (2, 0);
                \draw[->, blue] (2,-1.5) -- (2, 0);
                \draw[->, red] (2, 0) -- (3,0);

                \node[zero] at (1,0) {};
                \node[zero] at (3,0) {};
            \end{scope}
        }

        \node[gray, anchor=north] at (-6,0) {$-3\pi$};
        \node[gray, anchor=north] at (-4,0) {$-2\pi$};
        \node[gray, anchor=north] at (-2,0) {$-\pi$};
        % \node[gray, anchor=north] at (0,0) {$0$};
        \node[gray, anchor=north] at (2,0) {$\pi$};
        \node[gray, anchor=north] at (4,0) {$2\pi$};
        \node[gray, anchor=north] at (6,0) {$3\pi$};

        \node[gray, anchor=east] at (0,-1.5) {$-\infty$};
        % \node[gray, anchor=south east] at (0, 0) {$0$};
        \node[gray, anchor=east] at (0, 1.5) {$\infty$};

    \end{scope}

    \begin{scope}[yshift=-2.5cm]

        \draw[gray, ->] (-5,0) -- (5,0) node[anchor=west]{$w$};

        \draw[thick, ->, blue]      (-4, 0) -- (-2, 0);
        \draw[thick, ->, red]       (-2, 0) -- (0, 0);
        \draw[thick, ->, orange]    (0, 0) -- (2, 0);
        \draw[thick, ->, darkgreen] (2, 0) -- (4, 0);

        \node[anchor=south] at (-4,0) {$-\infty$};
        \node[anchor=south] at (-2,0) {$-1$};
        \node[anchor=south] at (0,0) {$0$};
        \node[anchor=south] at (2,0) {$1$};
        \node[anchor=south] at (4,0) {$\infty$};

    \end{scope}


\end{tikzpicture}
    \caption{Die Funktion $z = \cos^{-1}(w)$ dargestellt in der komplexen ebene.}
    \label{ellfilter:fig:arccos}
\end{figure}



\begin{figure}
    \centering
    \begin{tikzpicture}[>=stealth', auto, node distance=2cm, scale=1.2]

    \tikzstyle{zero} = [draw, circle, inner sep =0, minimum height=0.15cm]
    \tikzset{pole/.style={cross out, draw=black, minimum size=(0.15cm-\pgflinewidth), inner sep=0pt, outer sep=0pt}}

    \begin{scope}[xscale=0.5]

        \draw[gray, ->] (0,-2) -- (0,2) node[anchor=south]{$\mathrm{Im}~z_1$};
        \draw[gray, ->] (-10,0) -- (10,0) node[anchor=west]{$\mathrm{Re}~z_1$};

        \begin{scope}

            \draw[>->, line width=0.05, thick, blue]   (2, 1.5) -- (2,0.05)  -- node[anchor=south, pos=0.5]{$N=1$} (0.1,0.05) -- (0.1,1.5);
            \draw[>->, line width=0.05, thick, orange] (4, 1.5) -- (4,0)     -- node[anchor=south, pos=0.25]{$N=2$} (0,0) -- (0,1.5);
            \draw[>->, line width=0.05, thick, red]    (6, 1.5) node[anchor=north west]{$-\infty$} -- (6,-0.05) node[anchor=west]{$-1$} -- node[anchor=north]{$0$} node[anchor=south, pos=0.1666]{$N=3$} (-0.1,-0.05) node[anchor=east]{$1$}  -- (-0.1,1.5) node[anchor=north east]{$\infty$};

            \node[zero] at (-7,0) {};
            \node[zero] at (-5,0) {};
            \node[zero] at (-3,0) {};
            \node[zero] at (-1,0) {};
            \node[zero] at (1,0) {};
            \node[zero] at (3,0) {};
            \node[zero] at (5,0) {};
            \node[zero] at (7,0) {};

        \end{scope}

        \node[gray, anchor=north] at (-8,0) {$-4\pi$};
        \node[gray, anchor=north] at (-6,0) {$-3\pi$};
        \node[gray, anchor=north] at (-4,0) {$-2\pi$};
        \node[gray, anchor=north] at (-2,0) {$-\pi$};
        \node[gray, anchor=north] at (2,0) {$\pi$};
        \node[gray, anchor=north] at (4,0) {$2\pi$};
        \node[gray, anchor=north] at (6,0) {$3\pi$};
        \node[gray, anchor=north] at (8,0) {$4\pi$};


        \node[gray, anchor=east] at (0,-1.5) {$-\infty$};
        \node[gray, anchor=east] at (0, 1.5) {$\infty$};

    \end{scope}

    \node[zero] at (4,2) (n) {};
    \node[anchor=west] at (n.east) {Zero};

\end{tikzpicture}
    \caption{
        $z$-Ebene der Tschebyscheff-Funktion.
        Je grösser die Ordnung $N$ gewählt wird, desto mehr Nullstellen werden hat das Tschebyscheff-Polynom.
    }
    % \label{ellfilter:fig:arccos}
\end{figure}





% Analytische Fortsetzung



\section{Jacobische elliptische Funktionen}


Für das elliptische Filter, wird statt der für das Tschebyscheff-Filter benutzen Kreis-Trigonometrie die elliptischen Funktionen gebraucht.
Der begriff elliptische Funktion wird für sehr viele Funktionen gebraucht, daher ist es hier wichtig zu erwähnen, dass es ausschliesslich um die Jacobischen elliptischen Funktionen geht.

Im Wesentlichen erweitern die Jacobi elliptischen Funktionen die trigonometrische Funktionen für Ellipsen.

%TODO $z$ or $u$ for parameter?

neu zwei parameter
$sn(z, k)$
$z$ ist das winkelargument
Im Kreis ist der Radius für alle Winkel konstant, bei Ellipsen ändert sich das.
Dies hat zur Folge, dass bei einer Ellipse die Kreisbodenstrecke nicht linear zum Winkel verläuft.
Darum kann hier nicht der gewohnte Winkel verwendet werden.
An deren stelle kommt der parameter $k$ zum Einsatz, welcher durch das elliptische Integral erster Art
\begin{equation}
    z
    =
    F(\phi, k)
    =
    \int_{0}^{\phi}
    \frac{
        d\theta
    }{
        \sqrt{
            1-k^2 \sin^2 \theta
        }
    }
\end{equation}
mit dem Winkel $\phi$ in Verbindung liegt.




Dabei wird das vollständige und unvollständige Elliptische integral unterschieden.
Beim vollständigen Integral
\begin{equation}
    K(k)
    =
    \int_{0}^{\pi / 2}
    \frac{
        d\theta
    }{
        \sqrt{
            1-k^2 \sin^2 \theta
        }
    }
\end{equation}
wird über ein viertel Ellipsenbogen integriert also bis $\phi=\pi/2$.

Die Jacobischen elliptischen Funktionen können mit der inversen Funktion
\begin{equation}
    \phi = F^{-1}(z, k)
\end{equation}
definiert werden. Dabei ist zu beachten dass nur das $z$ Argument der Funktion invertiert wird also
\begin{equation}
    z = F(\phi, k)
    \Leftrightarrow
    \phi = F^{-1}(z, k).
\end{equation}
Mithilfe von $F^{-1}$ kann $sn^{-1}$ mit dem Elliptischen integral dargestellt werden:
\begin{equation}
    \sin(\phi)
    =
    \sin \left( F^{-1}(z, k) \right)
    =
    \sn(u, k)
\end{equation}

\begin{align}
    \sn^{-1}(w, k)
        & =
    \int_{0}^{\phi}
    \frac{
        d\theta
    }{
        \sqrt{
            1-k^2 \sin^2 \theta
        }
    },
    \quad
    \phi = \sin^{-1}(w)
    \\
        & =
    \int_{0}^{w}
    \frac{
        dt
    }{
        \sqrt{
            (1-t^2)(1-k^2 t^2)
        }
    }
\end{align}

Beim $\cos^{-1}(x)$ haben wir gesehen, dass die analytische Fortsetzung bei $x < -1$ und $x > 1$ rechtwinklig in die Komplexen zahlen wandert.
Wenn man das gleiche mit $\sn^{-1}(w, k)$ macht, erkennt man zwei interessante Stellen.
Die erste ist die gleiche wie beim $\cos^{-1}(x)$ nämlich bei $t = \pm 1$.
Der erste Term unter der Wurzel wird dann negativ, während der zweite noch positiv ist, da $k \leq 1$.
\begin{equation}
    \frac{
        1
    }{
        \sqrt{
            (1-t^2)(1-k^2 t^2)
        }
    }
    \in \mathbb{R}
    \quad \forall \quad
    -1 \leq t \leq 1
\end{equation}
Die zweite stelle passiert wenn beide Faktoren unter der Wurzel negativ werden, was bei $t = 1/k$ der Fall ist.




Funktion in relle und komplexe Richtung periodisch

In der reellen Richtung ist sie $4K(k)$-periodisch und in der imaginären Richtung $4K^\prime(k)$-periodisch.



%TODO sn^{-1} grafik


\section{Elliptische rationale Funktionen}


\begin{equation}
    R_N(\xi, w) = \cd \left(N~f_1(\xi)~\cd^{-1}(w, 1/\xi), f_2(\xi)\right)
\end{equation}
\begin{equation}
    R_N(\xi, w) = \cd (N~u K_1, k_1), \quad w= \cd(uK, k)
\end{equation}


sieht ähnlich aus wie die trigonometrische darstellung der Tschebyschef-Polynome

der Ordnungszahl $N$ kommt auch als Faktor for 

%TODO cd^{-1} grafik mit 


\subsection{Degree Equation}

Der $cd^{-1}$ Term muss so verzogen werden, dass die umgebene $cd$ funktion die nullstellen und pole trifft.
Dies trifft ein wenn die Degree Equation erfüllt ist.

\begin{equation}
    N \frac{K^\prime}{K} = \frac{K^\prime_1}{K_1}
\end{equation}


Leider ist das lösen dieser Gleichung nicht trivial.
Die Rechnung wird in \ref{ellfilter:bib:orfanidis} im Detail angeschaut.


\subsection{Polynome?}

Bei den Tschebyscheff-Polynomen haben wir gesehen, dass die Trigonometrische Formel zu einfachen Polynomen umgewandelt werden kann.
Im gegensatz zum $\cos^{-1}$ hat der $\cd^{-1}$ nicht nur Nullstellen sondern auch Pole.
Somit entstehen bei den elliptischen rationalen Funktionen, wie es der name auch deutet, rationale Funktionen, also ein Bruch von zwei Polynomen.




\begin{figure}
    \centering
    \includegraphics[scale=1]{papers/ellfilter/python/F_N_elliptic.pdf}
    \caption{$F_N$ für ein elliptischs filter.}
    \label{ellfilter:fig:elliptic}
\end{figure}





%
% einleitung.tex -- Beispiel-File für die Einleitung
%
% (c) 2020 Prof Dr Andreas Müller, Hochschule Rapperswil
%
\section{Teil 0\label{fresnel:section:teil0}}
\rhead{Teil 0}
Lorem ipsum dolor sit amet, consetetur sadipscing elitr, sed diam
nonumy eirmod tempor invidunt ut labore et dolore magna aliquyam
erat, sed diam voluptua \cite{fresnel:bibtex}.
At vero eos et accusam et justo duo dolores et ea rebum.
Stet clita kasd gubergren, no sea takimata sanctus est Lorem ipsum
dolor sit amet.

Lorem ipsum dolor sit amet, consetetur sadipscing elitr, sed diam
nonumy eirmod tempor invidunt ut labore et dolore magna aliquyam
erat, sed diam voluptua.
At vero eos et accusam et justo duo dolores et ea rebum.  Stet clita
kasd gubergren, no sea takimata sanctus est Lorem ipsum dolor sit
amet.



%
% teil1.tex -- Beispiel-File für das Paper
%
% (c) 2020 Prof Dr Andreas Müller, Hochschule Rapperswil
%
\section{Lösung
\label{parzyl:section:teil1}}
\rhead{Problemstellung}
Die Differentialgleichungen \eqref{parzyl:sep_dgl_1} und \eqref{parzyl:sep_dgl_2} können mit
Hilfe der Whittaker Gleichung gelöst werden.
\begin{definition}
    Die Funktion 
    \begin{equation*}
        W_{k,m}(z) = 
    e^{-z/2} z^{m+1/2} \,
    {}_{1} F_{1}
    (
        {\textstyle \frac{1}{2}} 
        + m - k, 1 + 2m; z)
    \end{equation*}
    heisst Whittaker Funktion und ist eine Lösung
    von der Whittaker Differentialgleichung
    \begin{equation}
        \frac{d^2W}{d z^2} +
        \left(-\frac{1}{4}  + \frac{k}{z} + \frac{\frac{1}{4} - m^2}{z^2} \right) W = 0.
        \label{parzyl:eq:whitDiffEq}
    \end{equation}
\end{definition}
Es wird nun die Differentialgleichung bestimmt, welche
\begin{equation}
    w = z^{-1/2} W_{k,-1/4} \left({\textstyle \frac{1}{2}} z^2\right)
\end{equation}
als Lösung hat.
Dafür wird $w$ in \eqref{parzyl:eq:whitDiffEq} eingesetzt woraus
\begin{equation}
    \frac{d^2 w}{dz^2} - \left(\frac{1}{4} z^2 - 2k\right) w = 0
\label{parzyl:eq:weberDiffEq}
\end{equation}
resultiert. DIese Differentialgleichung ist dieselbe wie 
\eqref{parzyl:sep_dgl_2} und \eqref{parzyl:sep_dgl_2}, welche somit
$w$ als Lösung haben.
Da es sich um eine Differentialgleichung zweiter Ordnung handelt, hat sie nicht nur
eine sondern zwei Lösungen.
Die zweite Lösung der Whittaker-Gleichung ist $W_{k,-m} (z)$.
Somit hat \eqref{parzyl:eq:weberDiffEq}
\begin{align}
    w_1 & = z^{-1/2} W_{k,-1/4} \left({\textstyle \frac{1}{2}} z^2\right)\\
    w_2 & = z^{-1/2} W_{k,1/4} \left({\textstyle \frac{1}{2}} z^2\right)
\end{align}
als Lösungen.

Ausgeschrieben ergeben sich als Lösungen
\begin{align}
    w_1 &= e^{-z^2/4} \,
    {}_{1} F_{1}
    (
        {\textstyle \frac{1}{4}} 
         - k, {\textstyle \frac{1}{2}} ; {\textstyle \frac{1}{2}}z^2) \\
    w_2 & = z e^{-z^2/4} \,
         {}_{1} F_{1}
         ({\textstyle \frac{3}{4}} 
              - k, {\textstyle \frac{3}{2}} ; {\textstyle \frac{1}{2}}z^2)
\end{align}
%
% teil2.tex -- Umsetzung in C Programmen
%
% (c) 2022 Fabian Dünki, Hochschule Rapperswil
%
\section{Umsetzung
\label{0f1:section:teil2}}
\rhead{Umsetzung}
Zur Umsetzung wurden drei verschiedene Ansätze gewählt, die in
vollständiger Form auf Github \cite{0f1:code} zu finden sind.
Dabei wurde der Schwerpunkt auf die Funktionalität und eine gute
Lesbarkeit des Codes gelegt.
Die Unterprogramme wurde jeweils, wie die GNU Scientific Library,
\index{GNU Scientific Library}%
in C geschrieben.
Die Zwischenresultate wurden vom Hauptprogramm
in einem CSV-File gespeichert.
\index{CSV}%
Anschliessen wurde mit der Matplot-Library
\index{Matplot-Library}%
\index{Python}%
in Python die Resultate geplottet.

\subsection{Potenzreihe
\label{0f1:subsection:potenzreihe}}
Die naheliegendste Lösung ist die Programmierung der Potenzreihe
\begin{align}
    \label{0f1:umsetzung:0f1:eq}
    \mathstrut_0F_1(;c;z)
    &=
    \sum_{k=0}^\infty
    \frac{z^k}{(c)_k \cdot k!}
    &= 
    \frac{1}{c}
    +\frac{z^1}{(c+1) \cdot 1}
    + \cdots
    + \frac{z^{20}}{c(c+1)(c+2)\cdots(c+19) \cdot 2.4 \cdot 10^{18}}.
\end{align}

\lstinputlisting[style=C,float,caption={Potenzreihe.},label={0f1:listing:potenzreihe}, firstline=59]{papers/0f1/listings/potenzreihe.c}

\subsection{Kettenbruch
\label{0f1:subsection:kettenbruch}}
Eine weitere Variante zur Berechnung von $\mathstrut_0F_1(;c;z)$ ist die Umsetzung als Kettenbruch.
\index{Kettenbruch}
Der Vorteil einer Umsetzung als Kettenbruch gegenüber der Potenzreihe ist die schnellere Konvergenz.

\subsubsection{Grundlage}
Ein endlicher Kettenbruch \cite{0f1:wiki-kettenbruch} ist ein Bruch der Form
\begin{equation*}
a_0 + \cfrac{b_1}{a_1+\cfrac{b_2}{a_2+\cfrac{b_3}{a_3+\cdots}}},
\end{equation*}
in welchem $a_0, a_1,\dots,a_n$ und $b_1,b_2,\dots,b_n$ ganze Zahlen sind.

\subsubsection{Rekursionsbeziehungen und Kettenbrüche}

Nimmt man die Gleichung
Wenn es für die analytischen Funktionen $f_i(z)$ eine Relation der Form
\begin{equation}
	f_{i-1}(z) - f_i(z) = k_i z f_{i+1}(z)
\label{0f1:relation}
\end{equation}
für ganzzahlige positive $i$ und Konstanten $k_i$
gibt,
dann gibt es einen Kettenbruch für das Verhältnis
$\frac{f_i(z)}{f_{i-1}(z)}$ \cite{0f1:wiki-fraction}. 
Aus der Relation~\eqref{0f1:relation}
ergibt sich der Zusammenhang
\begin{equation}
	\cfrac{f_i(z)}{f_{i-1}(z)}
	=
	\cfrac{1}{1+k_iz\cfrac{f_{i+1}(z)}{f_i(z)}}.
\label{0f1:bruchrelation}
\end{equation}
Geht man einen Schritt weiter und nimmt für
$g_i(z) = \frac{f_i(z)}{f_{i-1}(z)}$ an, kommt man zur Formel
\begin{equation*}
	g_i(z) = \cfrac{1}{1+k_izg_{i+1}(z)}.
\end{equation*}
Setzt man dies nun für $g_1$ in den Bruch ein, ergibt sich
\begin{equation*}
	g_1(z) = \cfrac{f_1(z)}{f_0(z)} = \cfrac{1}{1+k_izg_2(z)} = \cfrac{1}{1+\cfrac{k_1z}{1+k_2zg_3(z)}} = \cdots
\end{equation*}
Wiederholt man dies unendlich, erhält man einen Kettenbruch in der Form:
\begin{equation}
	\label{0f1:math:rekursion:eq}
	\cfrac{f_1(z)}{f_0(z)}
	=
	\cfrac{1}{1+\cfrac{k_1z}{1+\cfrac{k_2z}{1+\cfrac{k_3z}{\cdots}}}}.
\end{equation}

\subsubsection{Rekursion für $\mathstrut_0F_1$}
Angewendet auf die Potenzreihe
\begin{equation}
	\label{0f1:math:potenzreihe:0f1:eq}
	\mathstrut_0F_1(;c;z)
	=
	1 + \frac{z}{c\cdot1!} + \frac{z^2}{c(c+1)\cdot2!} + \frac{z^3}{c(c+1)(c+2)\cdot3!} + \cdots
\end{equation}
kann durch Substitution bewiesen werden, dass
\begin{equation*}
	\mathstrut_0F_1(;c-1;z) - \mathstrut_0F_1(;c;z)
	=
	\frac{z}{c(c-1)} \cdot \mathstrut_0F_1(;c+1;z)
\end{equation*}
eine Relation der Art \eqref{0f1:relation} dazu ist.
Wenn man für $f_i$ und $k_i$ die Annahme
\begin{align*}
	f_i &= \mathstrut_0F_1(;c+i;z)\\
	k_i	&= \frac{1}{(c+i)(c+i-1)}
\end{align*}
trifft und in die Formel \eqref{0f1:math:rekursion:eq} einsetzt, erhält man:
\begin{equation*}
	\cfrac{\mathstrut_0F_1(;c+1;z)}{\mathstrut_0F_1(;c;z)}
	=
	\cfrac{1}{1+\cfrac{\cfrac{z}{c(c+1)}}{1+\cfrac{\cfrac{z}{(c+1)(c+2)}}{1+\cfrac{\cfrac{z}{(c+2)(c+3)}}{\cdots}}}}.
\end{equation*}

\subsubsection{Algorithmus}
Da mit obigen Formeln nur ein Verhältnis zwischen
$\frac{\mathstrut_0F_1(;c+1;z)}{\mathstrut_0F_1(;c;z)}$
berechnet wurde, braucht es weitere Relationen um $\mathstrut_0F_1(;c;z)$
zu erhalten.
So ergeben ähnliche Relationen nach Wolfram Alpha \cite{0f1:wolfram-0f1} den Kettenbruch
\begin{equation}
	\label{0f1:math:kettenbruch:0f1:eq}
	\mathstrut_0F_1(;c;z) = 1 + \cfrac{\cfrac{z}{c}}{1+\cfrac{-\cfrac{z}{2(c+1)}}{1+\cfrac{z}{2(c+1)}+\cfrac{-\cfrac{z}{3(c+2)}}{1+\cfrac{z}{5(c+4)} + \cdots}}},
\end{equation}
der als Code (Listing \ref{0f1:listing:kettenbruchIterativ})  umgesetzt wurde. 

\lstinputlisting[style=C,float,caption={Iterativ umgesetzter Kettenbruch.},label={0f1:listing:kettenbruchIterativ},  firstline=8]{papers/0f1/listings/kettenbruchIterativ.c}

\subsection{Rekursionsformel
\label{0f1:subsection:rekursionsformel}}
Wesentlich stabiler zur Berechnung eines Kettenbruches ist die
Rekursionsformel.
\index{Rekursionsformel}%
Nachfolgend wird die verkürzte Herleitung vom
Kettenbruch zur Rekursionsformel aufgezeigt.
Eine vollständige Schritt für Schritt Herleitung ist im Seminarbuch Numerik
\cite{0f1:kettenbrueche}
im Kapitel Kettenbrüche zu finden.

\subsubsection{Herleitung}
Ein Näherungsbruch in der Form
\index{Näherungsbruch}%
\begin{align*}
	\cfrac{A_k}{B_k} = a_k + \cfrac{b_{k + 1}}{a_{k + 1} + \cfrac{p}{q}}
\end{align*}
lässt sich zu
\begin{align*}
	\cfrac{A_k}{B_k} = \cfrac{b_{k+1}}{a_{k+1} + \cfrac{p}{q}} = \frac{b_{k+1} \cdot q}{a_{k+1} \cdot q + p}
\end{align*}
umformen.
Dies lässt sich auch durch die Matrizenschreibweise
\index{Matrixschreibeweise eines Kettenbruchs}%
\begin{equation*}
	\begin{pmatrix}
		A_k\\
		B_k
	\end{pmatrix}
	= 		\begin{pmatrix}
		b_{k+1} \cdot q\\
		a_{k+1} \cdot q + p
	\end{pmatrix}
	=\begin{pmatrix}
		0&	b_{k+1}\\
		1&	a_{k+1}
	\end{pmatrix}
	\begin{pmatrix}
		p \\
		q
	\end{pmatrix}
	%\label{0f1:math:rekursionsformel:herleitung}
\end{equation*}
ausdrücken.
Wendet man dies nun auf den Kettenbruch in der Form
\begin{equation*}
	\frac{A_k}{B_k} = a_0 + \cfrac{b_1}{a_1+\cfrac{b_2}{a_2+\cfrac{\cdots}{\cdots+\cfrac{b_{k-1}}{a_{k-1} + \cfrac{b_k}{a_k}}}}}
\end{equation*}
an, ergibt sich die Matrixdarstellungen:
\begin{align*}
	\begin{pmatrix}
		A_k\\
		B_k
	\end{pmatrix}
	&=
	\begin{pmatrix}
		1& a_0\\
		0& 1
	\end{pmatrix}
	\begin{pmatrix}
		0& b_1\\
		1& a_1
	\end{pmatrix}
	\cdots
	\begin{pmatrix}
		0& b_{k-1}\\
		1& a_{k-1}
	\end{pmatrix}
	\begin{pmatrix}
		b_k\\
		a_k
	\end{pmatrix}.
\end{align*}
Nach vollständiger Induktion ergibt sich für den Schritt $k$, die Matrix
\begin{equation}
	\label{0f1:math:matrix:ende:eq}
	 \begin{pmatrix}
		A_{k}\\
		B_{k}			
	\end{pmatrix} 
	=
		\begin{pmatrix}
		A_{k-2}& A_{k-1}\\
		B_{k-2}& B_{k-1}			
	\end{pmatrix}
		\begin{pmatrix}
		b_k\\
		a_k
	\end{pmatrix}.
\end{equation}
Und schlussendlich kann der Näherungsbruch
\[
\frac{A_k}{B_k}
\] 
berechnet werden.

\subsubsection{Algorithmus}
Die Berechnung von $A_k, B_k$ gemäss \eqref{0f1:math:matrix:ende:eq} kann man auch ohne die Matrizenschreibweise \cite{0f1:kettenbrueche} aufschreiben:
\begin{itemize}
\item Startbedingungen:
\begin{align*}
A_{-1} &= 0		&		A_0 &= a_0 \\
B_{-1} &= 1		&		B_0 &= 1 
\end{align*}
\item Schritt $k\to k+1$:
\[
\begin{aligned}
\label{0f1:math:loesung:eq}
A_{k+1} &= A_{k-1} \cdot b_k + A_k \cdot a_k \\
B_{k+1} &= B_{k-1} \cdot b_k + B_k \cdot a_k
\end{aligned}
\]
\item
Näherungsbruch: \qquad$\displaystyle\frac{A_k}{B_k}$.
\end{itemize}
Ein grosser Vorteil dieser Umsetzung
als Rekursionsformel
(Listing~\ref{0f1:listing:kettenbruchRekursion}) ist,
dass im Vergleich zum Code (Listing~\ref{0f1:listing:kettenbruchIterativ})
eine Division gespart werden kann und somit weniger Rundungsfehler
entstehen können.

%Code
\lstinputlisting[style=C,float,caption={Rekursionsformel für Kettenbruch.},label={0f1:listing:kettenbruchRekursion},  firstline=8]{papers/0f1/listings/kettenbruchRekursion.c}

%
% teil3.tex -- Beispiel-File für Teil 3
%
% (c) 2020 Prof Dr Andreas Müller, Hochschule Rapperswil
%
\section{Eigenschaften
\label{parzyl:section:Eigenschaften}}
\rhead{Eigenschaften}

\subsection{Potenzreihenentwicklung
	\label{parzyl:potenz}}
Die parabolischen Zylinderfunktionen, welche in Gleichung \ref{parzyl:eq:solution_dgl} gegeben sind, können auch als Potenzreihen geschrieben werden
\begin{align}
	w_1(k,z)
	&=  
	e^{-z^2/4} \,
	{}_{1} F_{1}
	(
	{\textstyle \frac{1}{4}} 
	- k, {\textstyle \frac{1}{2}} ; {\textstyle \frac{1}{2}}z^2) 
	= 
	e^{-\frac{z^2}{4}}
	\sum^{\infty}_{n=0}
	\frac{\left ( \frac{1}{4} - k \right )_{n}}{\left ( \frac{1}{2}\right )_{n}}
	\frac{\left ( \frac{1}{2} z^2\right )^n}{n!} \\
	&=
	e^{-\frac{z^2}{4}}
	\left ( 
	1 
	+
	\left ( \frac{1}{2} - 2k \right )\frac{z^2}{2!}
	+
	\left ( \frac{1}{2} - 2k \right )\left ( \frac{5}{2} - 2k \right )\frac{z^4}{4!}  
	+
	\dots
	\right )
\end{align}
und
\begin{align}
	w_2(k,z)
	&=  
	ze^{-z^2/4} \,
	{}_{1} F_{1}
	(
	{\textstyle \frac{3}{4}} 
	- k, {\textstyle \frac{3}{2}} ; {\textstyle \frac{1}{2}}z^2) 
	= 
	ze^{-\frac{z^2}{4}}
	\sum^{\infty}_{n=0}
	\frac{\left ( \frac{3}{4} - k \right )_{n}}{\left ( \frac{3}{2}\right )_{n}}
	\frac{\left ( \frac{1}{2} z^2\right )^n}{n!} \\
	&=
	e^{-\frac{z^2}{4}}
	\left ( 
	z 
	+
	\left ( \frac{3}{2} - 2k \right )\frac{z^3}{3!}
	+
	\left ( \frac{3}{2} - 2k \right )\left ( \frac{7}{2} - 2k \right )\frac{z^5}{5!}  
	+
	\dots
	\right ).
\end{align}
Bei den Potenzreihen sieht man gut, dass die Ordnung des Polynoms im generellen ins unendliche geht. Es gibt allerdings die Möglichkeit für bestimmte k das die Terme in der Klammer gleich null werden und das Polynom somit eine endliche Ordnung $n$ hat. Dies geschieht bei $w_1(k,z)$ falls
\begin{equation}
	k = \frac{1}{4} + n \qquad n \in \mathbb{N}_0
\end{equation}
und bei $w_2(k,z)$ falls
\begin{equation}
	k = \frac{3}{4} + n \qquad n \in \mathbb{N}_0.
\end{equation}

\subsection{Ableitung}
Es kann gezeigt werden, dass die Ableitungen $\frac{\partial w_1(z,k)}{\partial z}$ und $\frac{\partial w_2(z,k)}{\partial z}$ einen Zusammenhang zwischen $w_1(z,k)$ und $w_2(z,k)$ zeigen. Die Ableitung von $w_1(z,k)$ nach $z$ kann über die Produktregel berechnet werden und ist gegeben als
\begin{equation}
	\frac{\partial w_1(z,k)}{\partial z} = \left (\frac{1}{2} - 2k \right ) w_2(z, k -\frac{1}{2}) - \frac{1}{2} z w_1(z,k),
\end{equation} 
und die Ableitung von $w_2(z,k)$ als
\begin{equation}
	\frac{\partial w_2(z,k)}{\partial z} = w_1(z, k -\frac{1}{2}) - \frac{1}{2} z w_2(z,k).
\end{equation}
Über diese Eigenschaft können einfach weitere Ableitungen berechnet werden. 



% \printbibliography[heading=subbibliography]
\end{refsection}
