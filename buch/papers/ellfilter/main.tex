%
% main.tex -- Paper zum Thema Elliptische Filter
%
% (c) 2020 Hochschule Rapperswil
%
\chapter{Elliptische Filter\label{chapter:ellfilter}}
\lhead{Elliptische Filter}
\begin{refsection}
\chapterauthor{Nicolas Tobler}


\section{Einleitung}

% Lineare filter

% Filter, Signalverarbeitung


Der womöglich wichtigste Filtertyp ist das Tiefpassfilter.
Dieses soll im Durchlassbereich unter der Grenzfrequenz $\Omega_p$ unverstärkt durchlassen und alle anderen Frequenzen vollständig auslöschen.

% Bei der Implementierung von Filtern

In der Elektrotechnik führen Schaltungen mit linearen Bauelementen wie Kondensatoren, Spulen und Widerständen immer zu linearen zeitinvarianten Systemen (LTI-System von englich \textit{time-invariant system}).
Die Übertragungsfunktion im Frequenzbereich $|H(\Omega)|$ eines solchen Systems ist dabei immer eine rationale Funktion, also eine Division von zwei Polynomen.
Die Polynome habe dabei immer reelle oder komplex-konjugierte Nullstellen.


\begin{equation} \label{ellfilter:eq:h_omega}
    | H(\Omega)|^2 = \frac{1}{1 + \varepsilon_p^2 F_N^2(w)}, \quad w=\frac{\Omega}{\Omega_p}
\end{equation}

$\Omega = 2 \pi f$ ist die analoge Frequenz


% Linear filter
Damit das Filter implementierbar und stabil ist, muss $H(\Omega)^2$ eine rationale Funktion sein, deren Nullstellen und Pole auf der linken Halbebene liegen.

$N \in \mathbb{N} $ gibt dabei die Ordnung des Filters vor, also die maximale Anzahl Pole oder Nullstellen.

Damit ein Filter die Passband Kondition erfüllt muss $|F_N(w)| \leq 1 \forall |w| \leq 1$ und für $|w| \geq 1$ sollte die Funktion möglichst schnell divergieren.
Eine einfaches Polynom, dass das erfüllt, erhalten wir wenn $F_N(w) = w^N$.
Tatsächlich erhalten wir damit das Butterworth Filter, wie in Abbildung \ref{ellfilter:fig:butterworth} ersichtlich.
\begin{figure}
    \centering
    %% Creator: Matplotlib, PGF backend
%%
%% To include the figure in your LaTeX document, write
%%   \input{<filename>.pgf}
%%
%% Make sure the required packages are loaded in your preamble
%%   \usepackage{pgf}
%%
%% Also ensure that all the required font packages are loaded; for instance,
%% the lmodern package is sometimes necessary when using math font.
%%   \usepackage{lmodern}
%%
%% Figures using additional raster images can only be included by \input if
%% they are in the same directory as the main LaTeX file. For loading figures
%% from other directories you can use the `import` package
%%   \usepackage{import}
%%
%% and then include the figures with
%%   \import{<path to file>}{<filename>.pgf}
%%
%% Matplotlib used the following preamble
%%
\begingroup%
\makeatletter%
\begin{pgfpicture}%
\pgfpathrectangle{\pgfpointorigin}{\pgfqpoint{4.000000in}{2.500000in}}%
\pgfusepath{use as bounding box, clip}%
\begin{pgfscope}%
\pgfsetbuttcap%
\pgfsetmiterjoin%
\pgfsetlinewidth{0.000000pt}%
\definecolor{currentstroke}{rgb}{1.000000,1.000000,1.000000}%
\pgfsetstrokecolor{currentstroke}%
\pgfsetstrokeopacity{0.000000}%
\pgfsetdash{}{0pt}%
\pgfpathmoveto{\pgfqpoint{0.000000in}{0.000000in}}%
\pgfpathlineto{\pgfqpoint{4.000000in}{0.000000in}}%
\pgfpathlineto{\pgfqpoint{4.000000in}{2.500000in}}%
\pgfpathlineto{\pgfqpoint{0.000000in}{2.500000in}}%
\pgfpathlineto{\pgfqpoint{0.000000in}{0.000000in}}%
\pgfpathclose%
\pgfusepath{}%
\end{pgfscope}%
\begin{pgfscope}%
\pgfsetbuttcap%
\pgfsetmiterjoin%
\definecolor{currentfill}{rgb}{1.000000,1.000000,1.000000}%
\pgfsetfillcolor{currentfill}%
\pgfsetlinewidth{0.000000pt}%
\definecolor{currentstroke}{rgb}{0.000000,0.000000,0.000000}%
\pgfsetstrokecolor{currentstroke}%
\pgfsetstrokeopacity{0.000000}%
\pgfsetdash{}{0pt}%
\pgfpathmoveto{\pgfqpoint{0.630330in}{0.548769in}}%
\pgfpathlineto{\pgfqpoint{3.727004in}{0.548769in}}%
\pgfpathlineto{\pgfqpoint{3.727004in}{2.301955in}}%
\pgfpathlineto{\pgfqpoint{0.630330in}{2.301955in}}%
\pgfpathlineto{\pgfqpoint{0.630330in}{0.548769in}}%
\pgfpathclose%
\pgfusepath{fill}%
\end{pgfscope}%
\begin{pgfscope}%
\pgfpathrectangle{\pgfqpoint{0.630330in}{0.548769in}}{\pgfqpoint{3.096674in}{1.753186in}}%
\pgfusepath{clip}%
\pgfsetbuttcap%
\pgfsetmiterjoin%
\definecolor{currentfill}{rgb}{0.000000,0.501961,0.000000}%
\pgfsetfillcolor{currentfill}%
\pgfsetfillopacity{0.200000}%
\pgfsetlinewidth{0.000000pt}%
\definecolor{currentstroke}{rgb}{0.000000,0.000000,0.000000}%
\pgfsetstrokecolor{currentstroke}%
\pgfsetstrokeopacity{0.200000}%
\pgfsetdash{}{0pt}%
\pgfpathmoveto{\pgfqpoint{0.630330in}{0.548769in}}%
\pgfpathlineto{\pgfqpoint{2.694779in}{0.548769in}}%
\pgfpathlineto{\pgfqpoint{2.694779in}{1.425362in}}%
\pgfpathlineto{\pgfqpoint{0.630330in}{1.425362in}}%
\pgfpathlineto{\pgfqpoint{0.630330in}{0.548769in}}%
\pgfpathclose%
\pgfusepath{fill}%
\end{pgfscope}%
\begin{pgfscope}%
\pgfpathrectangle{\pgfqpoint{0.630330in}{0.548769in}}{\pgfqpoint{3.096674in}{1.753186in}}%
\pgfusepath{clip}%
\pgfsetbuttcap%
\pgfsetmiterjoin%
\definecolor{currentfill}{rgb}{1.000000,0.647059,0.000000}%
\pgfsetfillcolor{currentfill}%
\pgfsetfillopacity{0.200000}%
\pgfsetlinewidth{0.000000pt}%
\definecolor{currentstroke}{rgb}{0.000000,0.000000,0.000000}%
\pgfsetstrokecolor{currentstroke}%
\pgfsetstrokeopacity{0.200000}%
\pgfsetdash{}{0pt}%
\pgfpathmoveto{\pgfqpoint{2.694779in}{1.425362in}}%
\pgfpathlineto{\pgfqpoint{3.727004in}{1.425362in}}%
\pgfpathlineto{\pgfqpoint{3.727004in}{2.301955in}}%
\pgfpathlineto{\pgfqpoint{2.694779in}{2.301955in}}%
\pgfpathlineto{\pgfqpoint{2.694779in}{1.425362in}}%
\pgfpathclose%
\pgfusepath{fill}%
\end{pgfscope}%
\begin{pgfscope}%
\pgfpathrectangle{\pgfqpoint{0.630330in}{0.548769in}}{\pgfqpoint{3.096674in}{1.753186in}}%
\pgfusepath{clip}%
\pgfsetrectcap%
\pgfsetroundjoin%
\pgfsetlinewidth{0.803000pt}%
\definecolor{currentstroke}{rgb}{0.690196,0.690196,0.690196}%
\pgfsetstrokecolor{currentstroke}%
\pgfsetdash{}{0pt}%
\pgfpathmoveto{\pgfqpoint{0.630330in}{0.548769in}}%
\pgfpathlineto{\pgfqpoint{0.630330in}{2.301955in}}%
\pgfusepath{stroke}%
\end{pgfscope}%
\begin{pgfscope}%
\pgfsetbuttcap%
\pgfsetroundjoin%
\definecolor{currentfill}{rgb}{0.000000,0.000000,0.000000}%
\pgfsetfillcolor{currentfill}%
\pgfsetlinewidth{0.803000pt}%
\definecolor{currentstroke}{rgb}{0.000000,0.000000,0.000000}%
\pgfsetstrokecolor{currentstroke}%
\pgfsetdash{}{0pt}%
\pgfsys@defobject{currentmarker}{\pgfqpoint{0.000000in}{-0.048611in}}{\pgfqpoint{0.000000in}{0.000000in}}{%
\pgfpathmoveto{\pgfqpoint{0.000000in}{0.000000in}}%
\pgfpathlineto{\pgfqpoint{0.000000in}{-0.048611in}}%
\pgfusepath{stroke,fill}%
}%
\begin{pgfscope}%
\pgfsys@transformshift{0.630330in}{0.548769in}%
\pgfsys@useobject{currentmarker}{}%
\end{pgfscope}%
\end{pgfscope}%
\begin{pgfscope}%
\definecolor{textcolor}{rgb}{0.000000,0.000000,0.000000}%
\pgfsetstrokecolor{textcolor}%
\pgfsetfillcolor{textcolor}%
\pgftext[x=0.630330in,y=0.451547in,,top]{\color{textcolor}\rmfamily\fontsize{10.000000}{12.000000}\selectfont \(\displaystyle {0.00}\)}%
\end{pgfscope}%
\begin{pgfscope}%
\pgfpathrectangle{\pgfqpoint{0.630330in}{0.548769in}}{\pgfqpoint{3.096674in}{1.753186in}}%
\pgfusepath{clip}%
\pgfsetrectcap%
\pgfsetroundjoin%
\pgfsetlinewidth{0.803000pt}%
\definecolor{currentstroke}{rgb}{0.690196,0.690196,0.690196}%
\pgfsetstrokecolor{currentstroke}%
\pgfsetdash{}{0pt}%
\pgfpathmoveto{\pgfqpoint{1.146442in}{0.548769in}}%
\pgfpathlineto{\pgfqpoint{1.146442in}{2.301955in}}%
\pgfusepath{stroke}%
\end{pgfscope}%
\begin{pgfscope}%
\pgfsetbuttcap%
\pgfsetroundjoin%
\definecolor{currentfill}{rgb}{0.000000,0.000000,0.000000}%
\pgfsetfillcolor{currentfill}%
\pgfsetlinewidth{0.803000pt}%
\definecolor{currentstroke}{rgb}{0.000000,0.000000,0.000000}%
\pgfsetstrokecolor{currentstroke}%
\pgfsetdash{}{0pt}%
\pgfsys@defobject{currentmarker}{\pgfqpoint{0.000000in}{-0.048611in}}{\pgfqpoint{0.000000in}{0.000000in}}{%
\pgfpathmoveto{\pgfqpoint{0.000000in}{0.000000in}}%
\pgfpathlineto{\pgfqpoint{0.000000in}{-0.048611in}}%
\pgfusepath{stroke,fill}%
}%
\begin{pgfscope}%
\pgfsys@transformshift{1.146442in}{0.548769in}%
\pgfsys@useobject{currentmarker}{}%
\end{pgfscope}%
\end{pgfscope}%
\begin{pgfscope}%
\definecolor{textcolor}{rgb}{0.000000,0.000000,0.000000}%
\pgfsetstrokecolor{textcolor}%
\pgfsetfillcolor{textcolor}%
\pgftext[x=1.146442in,y=0.451547in,,top]{\color{textcolor}\rmfamily\fontsize{10.000000}{12.000000}\selectfont \(\displaystyle {0.25}\)}%
\end{pgfscope}%
\begin{pgfscope}%
\pgfpathrectangle{\pgfqpoint{0.630330in}{0.548769in}}{\pgfqpoint{3.096674in}{1.753186in}}%
\pgfusepath{clip}%
\pgfsetrectcap%
\pgfsetroundjoin%
\pgfsetlinewidth{0.803000pt}%
\definecolor{currentstroke}{rgb}{0.690196,0.690196,0.690196}%
\pgfsetstrokecolor{currentstroke}%
\pgfsetdash{}{0pt}%
\pgfpathmoveto{\pgfqpoint{1.662555in}{0.548769in}}%
\pgfpathlineto{\pgfqpoint{1.662555in}{2.301955in}}%
\pgfusepath{stroke}%
\end{pgfscope}%
\begin{pgfscope}%
\pgfsetbuttcap%
\pgfsetroundjoin%
\definecolor{currentfill}{rgb}{0.000000,0.000000,0.000000}%
\pgfsetfillcolor{currentfill}%
\pgfsetlinewidth{0.803000pt}%
\definecolor{currentstroke}{rgb}{0.000000,0.000000,0.000000}%
\pgfsetstrokecolor{currentstroke}%
\pgfsetdash{}{0pt}%
\pgfsys@defobject{currentmarker}{\pgfqpoint{0.000000in}{-0.048611in}}{\pgfqpoint{0.000000in}{0.000000in}}{%
\pgfpathmoveto{\pgfqpoint{0.000000in}{0.000000in}}%
\pgfpathlineto{\pgfqpoint{0.000000in}{-0.048611in}}%
\pgfusepath{stroke,fill}%
}%
\begin{pgfscope}%
\pgfsys@transformshift{1.662555in}{0.548769in}%
\pgfsys@useobject{currentmarker}{}%
\end{pgfscope}%
\end{pgfscope}%
\begin{pgfscope}%
\definecolor{textcolor}{rgb}{0.000000,0.000000,0.000000}%
\pgfsetstrokecolor{textcolor}%
\pgfsetfillcolor{textcolor}%
\pgftext[x=1.662555in,y=0.451547in,,top]{\color{textcolor}\rmfamily\fontsize{10.000000}{12.000000}\selectfont \(\displaystyle {0.50}\)}%
\end{pgfscope}%
\begin{pgfscope}%
\pgfpathrectangle{\pgfqpoint{0.630330in}{0.548769in}}{\pgfqpoint{3.096674in}{1.753186in}}%
\pgfusepath{clip}%
\pgfsetrectcap%
\pgfsetroundjoin%
\pgfsetlinewidth{0.803000pt}%
\definecolor{currentstroke}{rgb}{0.690196,0.690196,0.690196}%
\pgfsetstrokecolor{currentstroke}%
\pgfsetdash{}{0pt}%
\pgfpathmoveto{\pgfqpoint{2.178667in}{0.548769in}}%
\pgfpathlineto{\pgfqpoint{2.178667in}{2.301955in}}%
\pgfusepath{stroke}%
\end{pgfscope}%
\begin{pgfscope}%
\pgfsetbuttcap%
\pgfsetroundjoin%
\definecolor{currentfill}{rgb}{0.000000,0.000000,0.000000}%
\pgfsetfillcolor{currentfill}%
\pgfsetlinewidth{0.803000pt}%
\definecolor{currentstroke}{rgb}{0.000000,0.000000,0.000000}%
\pgfsetstrokecolor{currentstroke}%
\pgfsetdash{}{0pt}%
\pgfsys@defobject{currentmarker}{\pgfqpoint{0.000000in}{-0.048611in}}{\pgfqpoint{0.000000in}{0.000000in}}{%
\pgfpathmoveto{\pgfqpoint{0.000000in}{0.000000in}}%
\pgfpathlineto{\pgfqpoint{0.000000in}{-0.048611in}}%
\pgfusepath{stroke,fill}%
}%
\begin{pgfscope}%
\pgfsys@transformshift{2.178667in}{0.548769in}%
\pgfsys@useobject{currentmarker}{}%
\end{pgfscope}%
\end{pgfscope}%
\begin{pgfscope}%
\definecolor{textcolor}{rgb}{0.000000,0.000000,0.000000}%
\pgfsetstrokecolor{textcolor}%
\pgfsetfillcolor{textcolor}%
\pgftext[x=2.178667in,y=0.451547in,,top]{\color{textcolor}\rmfamily\fontsize{10.000000}{12.000000}\selectfont \(\displaystyle {0.75}\)}%
\end{pgfscope}%
\begin{pgfscope}%
\pgfpathrectangle{\pgfqpoint{0.630330in}{0.548769in}}{\pgfqpoint{3.096674in}{1.753186in}}%
\pgfusepath{clip}%
\pgfsetrectcap%
\pgfsetroundjoin%
\pgfsetlinewidth{0.803000pt}%
\definecolor{currentstroke}{rgb}{0.690196,0.690196,0.690196}%
\pgfsetstrokecolor{currentstroke}%
\pgfsetdash{}{0pt}%
\pgfpathmoveto{\pgfqpoint{2.694779in}{0.548769in}}%
\pgfpathlineto{\pgfqpoint{2.694779in}{2.301955in}}%
\pgfusepath{stroke}%
\end{pgfscope}%
\begin{pgfscope}%
\pgfsetbuttcap%
\pgfsetroundjoin%
\definecolor{currentfill}{rgb}{0.000000,0.000000,0.000000}%
\pgfsetfillcolor{currentfill}%
\pgfsetlinewidth{0.803000pt}%
\definecolor{currentstroke}{rgb}{0.000000,0.000000,0.000000}%
\pgfsetstrokecolor{currentstroke}%
\pgfsetdash{}{0pt}%
\pgfsys@defobject{currentmarker}{\pgfqpoint{0.000000in}{-0.048611in}}{\pgfqpoint{0.000000in}{0.000000in}}{%
\pgfpathmoveto{\pgfqpoint{0.000000in}{0.000000in}}%
\pgfpathlineto{\pgfqpoint{0.000000in}{-0.048611in}}%
\pgfusepath{stroke,fill}%
}%
\begin{pgfscope}%
\pgfsys@transformshift{2.694779in}{0.548769in}%
\pgfsys@useobject{currentmarker}{}%
\end{pgfscope}%
\end{pgfscope}%
\begin{pgfscope}%
\definecolor{textcolor}{rgb}{0.000000,0.000000,0.000000}%
\pgfsetstrokecolor{textcolor}%
\pgfsetfillcolor{textcolor}%
\pgftext[x=2.694779in,y=0.451547in,,top]{\color{textcolor}\rmfamily\fontsize{10.000000}{12.000000}\selectfont \(\displaystyle {1.00}\)}%
\end{pgfscope}%
\begin{pgfscope}%
\pgfpathrectangle{\pgfqpoint{0.630330in}{0.548769in}}{\pgfqpoint{3.096674in}{1.753186in}}%
\pgfusepath{clip}%
\pgfsetrectcap%
\pgfsetroundjoin%
\pgfsetlinewidth{0.803000pt}%
\definecolor{currentstroke}{rgb}{0.690196,0.690196,0.690196}%
\pgfsetstrokecolor{currentstroke}%
\pgfsetdash{}{0pt}%
\pgfpathmoveto{\pgfqpoint{3.210892in}{0.548769in}}%
\pgfpathlineto{\pgfqpoint{3.210892in}{2.301955in}}%
\pgfusepath{stroke}%
\end{pgfscope}%
\begin{pgfscope}%
\pgfsetbuttcap%
\pgfsetroundjoin%
\definecolor{currentfill}{rgb}{0.000000,0.000000,0.000000}%
\pgfsetfillcolor{currentfill}%
\pgfsetlinewidth{0.803000pt}%
\definecolor{currentstroke}{rgb}{0.000000,0.000000,0.000000}%
\pgfsetstrokecolor{currentstroke}%
\pgfsetdash{}{0pt}%
\pgfsys@defobject{currentmarker}{\pgfqpoint{0.000000in}{-0.048611in}}{\pgfqpoint{0.000000in}{0.000000in}}{%
\pgfpathmoveto{\pgfqpoint{0.000000in}{0.000000in}}%
\pgfpathlineto{\pgfqpoint{0.000000in}{-0.048611in}}%
\pgfusepath{stroke,fill}%
}%
\begin{pgfscope}%
\pgfsys@transformshift{3.210892in}{0.548769in}%
\pgfsys@useobject{currentmarker}{}%
\end{pgfscope}%
\end{pgfscope}%
\begin{pgfscope}%
\definecolor{textcolor}{rgb}{0.000000,0.000000,0.000000}%
\pgfsetstrokecolor{textcolor}%
\pgfsetfillcolor{textcolor}%
\pgftext[x=3.210892in,y=0.451547in,,top]{\color{textcolor}\rmfamily\fontsize{10.000000}{12.000000}\selectfont \(\displaystyle {1.25}\)}%
\end{pgfscope}%
\begin{pgfscope}%
\pgfpathrectangle{\pgfqpoint{0.630330in}{0.548769in}}{\pgfqpoint{3.096674in}{1.753186in}}%
\pgfusepath{clip}%
\pgfsetrectcap%
\pgfsetroundjoin%
\pgfsetlinewidth{0.803000pt}%
\definecolor{currentstroke}{rgb}{0.690196,0.690196,0.690196}%
\pgfsetstrokecolor{currentstroke}%
\pgfsetdash{}{0pt}%
\pgfpathmoveto{\pgfqpoint{3.727004in}{0.548769in}}%
\pgfpathlineto{\pgfqpoint{3.727004in}{2.301955in}}%
\pgfusepath{stroke}%
\end{pgfscope}%
\begin{pgfscope}%
\pgfsetbuttcap%
\pgfsetroundjoin%
\definecolor{currentfill}{rgb}{0.000000,0.000000,0.000000}%
\pgfsetfillcolor{currentfill}%
\pgfsetlinewidth{0.803000pt}%
\definecolor{currentstroke}{rgb}{0.000000,0.000000,0.000000}%
\pgfsetstrokecolor{currentstroke}%
\pgfsetdash{}{0pt}%
\pgfsys@defobject{currentmarker}{\pgfqpoint{0.000000in}{-0.048611in}}{\pgfqpoint{0.000000in}{0.000000in}}{%
\pgfpathmoveto{\pgfqpoint{0.000000in}{0.000000in}}%
\pgfpathlineto{\pgfqpoint{0.000000in}{-0.048611in}}%
\pgfusepath{stroke,fill}%
}%
\begin{pgfscope}%
\pgfsys@transformshift{3.727004in}{0.548769in}%
\pgfsys@useobject{currentmarker}{}%
\end{pgfscope}%
\end{pgfscope}%
\begin{pgfscope}%
\definecolor{textcolor}{rgb}{0.000000,0.000000,0.000000}%
\pgfsetstrokecolor{textcolor}%
\pgfsetfillcolor{textcolor}%
\pgftext[x=3.727004in,y=0.451547in,,top]{\color{textcolor}\rmfamily\fontsize{10.000000}{12.000000}\selectfont \(\displaystyle {1.50}\)}%
\end{pgfscope}%
\begin{pgfscope}%
\definecolor{textcolor}{rgb}{0.000000,0.000000,0.000000}%
\pgfsetstrokecolor{textcolor}%
\pgfsetfillcolor{textcolor}%
\pgftext[x=2.178667in,y=0.272534in,,top]{\color{textcolor}\rmfamily\fontsize{10.000000}{12.000000}\selectfont \(\displaystyle w\)}%
\end{pgfscope}%
\begin{pgfscope}%
\pgfpathrectangle{\pgfqpoint{0.630330in}{0.548769in}}{\pgfqpoint{3.096674in}{1.753186in}}%
\pgfusepath{clip}%
\pgfsetrectcap%
\pgfsetroundjoin%
\pgfsetlinewidth{0.803000pt}%
\definecolor{currentstroke}{rgb}{0.690196,0.690196,0.690196}%
\pgfsetstrokecolor{currentstroke}%
\pgfsetdash{}{0pt}%
\pgfpathmoveto{\pgfqpoint{0.630330in}{0.548769in}}%
\pgfpathlineto{\pgfqpoint{3.727004in}{0.548769in}}%
\pgfusepath{stroke}%
\end{pgfscope}%
\begin{pgfscope}%
\pgfsetbuttcap%
\pgfsetroundjoin%
\definecolor{currentfill}{rgb}{0.000000,0.000000,0.000000}%
\pgfsetfillcolor{currentfill}%
\pgfsetlinewidth{0.803000pt}%
\definecolor{currentstroke}{rgb}{0.000000,0.000000,0.000000}%
\pgfsetstrokecolor{currentstroke}%
\pgfsetdash{}{0pt}%
\pgfsys@defobject{currentmarker}{\pgfqpoint{-0.048611in}{0.000000in}}{\pgfqpoint{-0.000000in}{0.000000in}}{%
\pgfpathmoveto{\pgfqpoint{-0.000000in}{0.000000in}}%
\pgfpathlineto{\pgfqpoint{-0.048611in}{0.000000in}}%
\pgfusepath{stroke,fill}%
}%
\begin{pgfscope}%
\pgfsys@transformshift{0.630330in}{0.548769in}%
\pgfsys@useobject{currentmarker}{}%
\end{pgfscope}%
\end{pgfscope}%
\begin{pgfscope}%
\definecolor{textcolor}{rgb}{0.000000,0.000000,0.000000}%
\pgfsetstrokecolor{textcolor}%
\pgfsetfillcolor{textcolor}%
\pgftext[x=0.355638in, y=0.500544in, left, base]{\color{textcolor}\rmfamily\fontsize{10.000000}{12.000000}\selectfont \(\displaystyle {0.0}\)}%
\end{pgfscope}%
\begin{pgfscope}%
\pgfpathrectangle{\pgfqpoint{0.630330in}{0.548769in}}{\pgfqpoint{3.096674in}{1.753186in}}%
\pgfusepath{clip}%
\pgfsetrectcap%
\pgfsetroundjoin%
\pgfsetlinewidth{0.803000pt}%
\definecolor{currentstroke}{rgb}{0.690196,0.690196,0.690196}%
\pgfsetstrokecolor{currentstroke}%
\pgfsetdash{}{0pt}%
\pgfpathmoveto{\pgfqpoint{0.630330in}{0.987065in}}%
\pgfpathlineto{\pgfqpoint{3.727004in}{0.987065in}}%
\pgfusepath{stroke}%
\end{pgfscope}%
\begin{pgfscope}%
\pgfsetbuttcap%
\pgfsetroundjoin%
\definecolor{currentfill}{rgb}{0.000000,0.000000,0.000000}%
\pgfsetfillcolor{currentfill}%
\pgfsetlinewidth{0.803000pt}%
\definecolor{currentstroke}{rgb}{0.000000,0.000000,0.000000}%
\pgfsetstrokecolor{currentstroke}%
\pgfsetdash{}{0pt}%
\pgfsys@defobject{currentmarker}{\pgfqpoint{-0.048611in}{0.000000in}}{\pgfqpoint{-0.000000in}{0.000000in}}{%
\pgfpathmoveto{\pgfqpoint{-0.000000in}{0.000000in}}%
\pgfpathlineto{\pgfqpoint{-0.048611in}{0.000000in}}%
\pgfusepath{stroke,fill}%
}%
\begin{pgfscope}%
\pgfsys@transformshift{0.630330in}{0.987065in}%
\pgfsys@useobject{currentmarker}{}%
\end{pgfscope}%
\end{pgfscope}%
\begin{pgfscope}%
\definecolor{textcolor}{rgb}{0.000000,0.000000,0.000000}%
\pgfsetstrokecolor{textcolor}%
\pgfsetfillcolor{textcolor}%
\pgftext[x=0.355638in, y=0.938840in, left, base]{\color{textcolor}\rmfamily\fontsize{10.000000}{12.000000}\selectfont \(\displaystyle {0.5}\)}%
\end{pgfscope}%
\begin{pgfscope}%
\pgfpathrectangle{\pgfqpoint{0.630330in}{0.548769in}}{\pgfqpoint{3.096674in}{1.753186in}}%
\pgfusepath{clip}%
\pgfsetrectcap%
\pgfsetroundjoin%
\pgfsetlinewidth{0.803000pt}%
\definecolor{currentstroke}{rgb}{0.690196,0.690196,0.690196}%
\pgfsetstrokecolor{currentstroke}%
\pgfsetdash{}{0pt}%
\pgfpathmoveto{\pgfqpoint{0.630330in}{1.425362in}}%
\pgfpathlineto{\pgfqpoint{3.727004in}{1.425362in}}%
\pgfusepath{stroke}%
\end{pgfscope}%
\begin{pgfscope}%
\pgfsetbuttcap%
\pgfsetroundjoin%
\definecolor{currentfill}{rgb}{0.000000,0.000000,0.000000}%
\pgfsetfillcolor{currentfill}%
\pgfsetlinewidth{0.803000pt}%
\definecolor{currentstroke}{rgb}{0.000000,0.000000,0.000000}%
\pgfsetstrokecolor{currentstroke}%
\pgfsetdash{}{0pt}%
\pgfsys@defobject{currentmarker}{\pgfqpoint{-0.048611in}{0.000000in}}{\pgfqpoint{-0.000000in}{0.000000in}}{%
\pgfpathmoveto{\pgfqpoint{-0.000000in}{0.000000in}}%
\pgfpathlineto{\pgfqpoint{-0.048611in}{0.000000in}}%
\pgfusepath{stroke,fill}%
}%
\begin{pgfscope}%
\pgfsys@transformshift{0.630330in}{1.425362in}%
\pgfsys@useobject{currentmarker}{}%
\end{pgfscope}%
\end{pgfscope}%
\begin{pgfscope}%
\definecolor{textcolor}{rgb}{0.000000,0.000000,0.000000}%
\pgfsetstrokecolor{textcolor}%
\pgfsetfillcolor{textcolor}%
\pgftext[x=0.355638in, y=1.377137in, left, base]{\color{textcolor}\rmfamily\fontsize{10.000000}{12.000000}\selectfont \(\displaystyle {1.0}\)}%
\end{pgfscope}%
\begin{pgfscope}%
\pgfpathrectangle{\pgfqpoint{0.630330in}{0.548769in}}{\pgfqpoint{3.096674in}{1.753186in}}%
\pgfusepath{clip}%
\pgfsetrectcap%
\pgfsetroundjoin%
\pgfsetlinewidth{0.803000pt}%
\definecolor{currentstroke}{rgb}{0.690196,0.690196,0.690196}%
\pgfsetstrokecolor{currentstroke}%
\pgfsetdash{}{0pt}%
\pgfpathmoveto{\pgfqpoint{0.630330in}{1.863658in}}%
\pgfpathlineto{\pgfqpoint{3.727004in}{1.863658in}}%
\pgfusepath{stroke}%
\end{pgfscope}%
\begin{pgfscope}%
\pgfsetbuttcap%
\pgfsetroundjoin%
\definecolor{currentfill}{rgb}{0.000000,0.000000,0.000000}%
\pgfsetfillcolor{currentfill}%
\pgfsetlinewidth{0.803000pt}%
\definecolor{currentstroke}{rgb}{0.000000,0.000000,0.000000}%
\pgfsetstrokecolor{currentstroke}%
\pgfsetdash{}{0pt}%
\pgfsys@defobject{currentmarker}{\pgfqpoint{-0.048611in}{0.000000in}}{\pgfqpoint{-0.000000in}{0.000000in}}{%
\pgfpathmoveto{\pgfqpoint{-0.000000in}{0.000000in}}%
\pgfpathlineto{\pgfqpoint{-0.048611in}{0.000000in}}%
\pgfusepath{stroke,fill}%
}%
\begin{pgfscope}%
\pgfsys@transformshift{0.630330in}{1.863658in}%
\pgfsys@useobject{currentmarker}{}%
\end{pgfscope}%
\end{pgfscope}%
\begin{pgfscope}%
\definecolor{textcolor}{rgb}{0.000000,0.000000,0.000000}%
\pgfsetstrokecolor{textcolor}%
\pgfsetfillcolor{textcolor}%
\pgftext[x=0.355638in, y=1.815433in, left, base]{\color{textcolor}\rmfamily\fontsize{10.000000}{12.000000}\selectfont \(\displaystyle {1.5}\)}%
\end{pgfscope}%
\begin{pgfscope}%
\pgfpathrectangle{\pgfqpoint{0.630330in}{0.548769in}}{\pgfqpoint{3.096674in}{1.753186in}}%
\pgfusepath{clip}%
\pgfsetrectcap%
\pgfsetroundjoin%
\pgfsetlinewidth{0.803000pt}%
\definecolor{currentstroke}{rgb}{0.690196,0.690196,0.690196}%
\pgfsetstrokecolor{currentstroke}%
\pgfsetdash{}{0pt}%
\pgfpathmoveto{\pgfqpoint{0.630330in}{2.301955in}}%
\pgfpathlineto{\pgfqpoint{3.727004in}{2.301955in}}%
\pgfusepath{stroke}%
\end{pgfscope}%
\begin{pgfscope}%
\pgfsetbuttcap%
\pgfsetroundjoin%
\definecolor{currentfill}{rgb}{0.000000,0.000000,0.000000}%
\pgfsetfillcolor{currentfill}%
\pgfsetlinewidth{0.803000pt}%
\definecolor{currentstroke}{rgb}{0.000000,0.000000,0.000000}%
\pgfsetstrokecolor{currentstroke}%
\pgfsetdash{}{0pt}%
\pgfsys@defobject{currentmarker}{\pgfqpoint{-0.048611in}{0.000000in}}{\pgfqpoint{-0.000000in}{0.000000in}}{%
\pgfpathmoveto{\pgfqpoint{-0.000000in}{0.000000in}}%
\pgfpathlineto{\pgfqpoint{-0.048611in}{0.000000in}}%
\pgfusepath{stroke,fill}%
}%
\begin{pgfscope}%
\pgfsys@transformshift{0.630330in}{2.301955in}%
\pgfsys@useobject{currentmarker}{}%
\end{pgfscope}%
\end{pgfscope}%
\begin{pgfscope}%
\definecolor{textcolor}{rgb}{0.000000,0.000000,0.000000}%
\pgfsetstrokecolor{textcolor}%
\pgfsetfillcolor{textcolor}%
\pgftext[x=0.355638in, y=2.253730in, left, base]{\color{textcolor}\rmfamily\fontsize{10.000000}{12.000000}\selectfont \(\displaystyle {2.0}\)}%
\end{pgfscope}%
\begin{pgfscope}%
\definecolor{textcolor}{rgb}{0.000000,0.000000,0.000000}%
\pgfsetstrokecolor{textcolor}%
\pgfsetfillcolor{textcolor}%
\pgftext[x=0.300082in,y=1.425362in,,bottom,rotate=90.000000]{\color{textcolor}\rmfamily\fontsize{10.000000}{12.000000}\selectfont \(\displaystyle F^2_N(w)\)}%
\end{pgfscope}%
\begin{pgfscope}%
\pgfpathrectangle{\pgfqpoint{0.630330in}{0.548769in}}{\pgfqpoint{3.096674in}{1.753186in}}%
\pgfusepath{clip}%
\pgfsetrectcap%
\pgfsetroundjoin%
\pgfsetlinewidth{1.505625pt}%
\definecolor{currentstroke}{rgb}{0.121569,0.466667,0.705882}%
\pgfsetstrokecolor{currentstroke}%
\pgfsetdash{}{0pt}%
\pgfpathmoveto{\pgfqpoint{0.630330in}{0.548769in}}%
\pgfpathlineto{\pgfqpoint{0.661609in}{0.548970in}}%
\pgfpathlineto{\pgfqpoint{0.692889in}{0.549574in}}%
\pgfpathlineto{\pgfqpoint{0.724168in}{0.550580in}}%
\pgfpathlineto{\pgfqpoint{0.755448in}{0.551989in}}%
\pgfpathlineto{\pgfqpoint{0.786727in}{0.553800in}}%
\pgfpathlineto{\pgfqpoint{0.818007in}{0.556013in}}%
\pgfpathlineto{\pgfqpoint{0.849287in}{0.558629in}}%
\pgfpathlineto{\pgfqpoint{0.880566in}{0.561648in}}%
\pgfpathlineto{\pgfqpoint{0.911846in}{0.565069in}}%
\pgfpathlineto{\pgfqpoint{0.943125in}{0.568893in}}%
\pgfpathlineto{\pgfqpoint{0.974405in}{0.573119in}}%
\pgfpathlineto{\pgfqpoint{1.005684in}{0.577747in}}%
\pgfpathlineto{\pgfqpoint{1.036964in}{0.582778in}}%
\pgfpathlineto{\pgfqpoint{1.068243in}{0.588211in}}%
\pgfpathlineto{\pgfqpoint{1.099523in}{0.594047in}}%
\pgfpathlineto{\pgfqpoint{1.130802in}{0.600286in}}%
\pgfpathlineto{\pgfqpoint{1.162082in}{0.606927in}}%
\pgfpathlineto{\pgfqpoint{1.193361in}{0.613970in}}%
\pgfpathlineto{\pgfqpoint{1.224641in}{0.621416in}}%
\pgfpathlineto{\pgfqpoint{1.255921in}{0.629264in}}%
\pgfpathlineto{\pgfqpoint{1.287200in}{0.637515in}}%
\pgfpathlineto{\pgfqpoint{1.318480in}{0.646168in}}%
\pgfpathlineto{\pgfqpoint{1.349759in}{0.655224in}}%
\pgfpathlineto{\pgfqpoint{1.381039in}{0.664682in}}%
\pgfpathlineto{\pgfqpoint{1.412318in}{0.674543in}}%
\pgfpathlineto{\pgfqpoint{1.443598in}{0.684806in}}%
\pgfpathlineto{\pgfqpoint{1.474877in}{0.695471in}}%
\pgfpathlineto{\pgfqpoint{1.506157in}{0.706539in}}%
\pgfpathlineto{\pgfqpoint{1.537436in}{0.718010in}}%
\pgfpathlineto{\pgfqpoint{1.568716in}{0.729883in}}%
\pgfpathlineto{\pgfqpoint{1.599995in}{0.742159in}}%
\pgfpathlineto{\pgfqpoint{1.631275in}{0.754837in}}%
\pgfpathlineto{\pgfqpoint{1.662555in}{0.767917in}}%
\pgfpathlineto{\pgfqpoint{1.693834in}{0.781400in}}%
\pgfpathlineto{\pgfqpoint{1.725114in}{0.795285in}}%
\pgfpathlineto{\pgfqpoint{1.756393in}{0.809573in}}%
\pgfpathlineto{\pgfqpoint{1.787673in}{0.824264in}}%
\pgfpathlineto{\pgfqpoint{1.818952in}{0.839357in}}%
\pgfpathlineto{\pgfqpoint{1.850232in}{0.854852in}}%
\pgfpathlineto{\pgfqpoint{1.881511in}{0.870750in}}%
\pgfpathlineto{\pgfqpoint{1.912791in}{0.887050in}}%
\pgfpathlineto{\pgfqpoint{1.944070in}{0.903753in}}%
\pgfpathlineto{\pgfqpoint{1.975350in}{0.920858in}}%
\pgfpathlineto{\pgfqpoint{2.006629in}{0.938366in}}%
\pgfpathlineto{\pgfqpoint{2.037909in}{0.956276in}}%
\pgfpathlineto{\pgfqpoint{2.069189in}{0.974589in}}%
\pgfpathlineto{\pgfqpoint{2.100468in}{0.993304in}}%
\pgfpathlineto{\pgfqpoint{2.131748in}{1.012421in}}%
\pgfpathlineto{\pgfqpoint{2.163027in}{1.031941in}}%
\pgfpathlineto{\pgfqpoint{2.194307in}{1.051864in}}%
\pgfpathlineto{\pgfqpoint{2.225586in}{1.072189in}}%
\pgfpathlineto{\pgfqpoint{2.256866in}{1.092917in}}%
\pgfpathlineto{\pgfqpoint{2.288145in}{1.114047in}}%
\pgfpathlineto{\pgfqpoint{2.319425in}{1.135579in}}%
\pgfpathlineto{\pgfqpoint{2.350704in}{1.157514in}}%
\pgfpathlineto{\pgfqpoint{2.381984in}{1.179851in}}%
\pgfpathlineto{\pgfqpoint{2.413263in}{1.202591in}}%
\pgfpathlineto{\pgfqpoint{2.444543in}{1.225734in}}%
\pgfpathlineto{\pgfqpoint{2.475823in}{1.249279in}}%
\pgfpathlineto{\pgfqpoint{2.507102in}{1.273226in}}%
\pgfpathlineto{\pgfqpoint{2.538382in}{1.297576in}}%
\pgfpathlineto{\pgfqpoint{2.569661in}{1.322328in}}%
\pgfpathlineto{\pgfqpoint{2.600941in}{1.347483in}}%
\pgfpathlineto{\pgfqpoint{2.632220in}{1.373040in}}%
\pgfpathlineto{\pgfqpoint{2.663500in}{1.399000in}}%
\pgfpathlineto{\pgfqpoint{2.694779in}{1.425362in}}%
\pgfpathlineto{\pgfqpoint{2.726059in}{1.452126in}}%
\pgfpathlineto{\pgfqpoint{2.757338in}{1.479294in}}%
\pgfpathlineto{\pgfqpoint{2.788618in}{1.506863in}}%
\pgfpathlineto{\pgfqpoint{2.819897in}{1.534835in}}%
\pgfpathlineto{\pgfqpoint{2.851177in}{1.563210in}}%
\pgfpathlineto{\pgfqpoint{2.882457in}{1.591987in}}%
\pgfpathlineto{\pgfqpoint{2.913736in}{1.621166in}}%
\pgfpathlineto{\pgfqpoint{2.945016in}{1.650748in}}%
\pgfpathlineto{\pgfqpoint{2.976295in}{1.680733in}}%
\pgfpathlineto{\pgfqpoint{3.007575in}{1.711120in}}%
\pgfpathlineto{\pgfqpoint{3.038854in}{1.741909in}}%
\pgfpathlineto{\pgfqpoint{3.070134in}{1.773101in}}%
\pgfpathlineto{\pgfqpoint{3.101413in}{1.804696in}}%
\pgfpathlineto{\pgfqpoint{3.132693in}{1.836692in}}%
\pgfpathlineto{\pgfqpoint{3.163972in}{1.869092in}}%
\pgfpathlineto{\pgfqpoint{3.195252in}{1.901894in}}%
\pgfpathlineto{\pgfqpoint{3.226531in}{1.935098in}}%
\pgfpathlineto{\pgfqpoint{3.257811in}{1.968705in}}%
\pgfpathlineto{\pgfqpoint{3.289091in}{2.002714in}}%
\pgfpathlineto{\pgfqpoint{3.320370in}{2.037126in}}%
\pgfpathlineto{\pgfqpoint{3.351650in}{2.071940in}}%
\pgfpathlineto{\pgfqpoint{3.382929in}{2.107156in}}%
\pgfpathlineto{\pgfqpoint{3.414209in}{2.142776in}}%
\pgfpathlineto{\pgfqpoint{3.445488in}{2.178797in}}%
\pgfpathlineto{\pgfqpoint{3.476768in}{2.215221in}}%
\pgfpathlineto{\pgfqpoint{3.508047in}{2.252048in}}%
\pgfpathlineto{\pgfqpoint{3.539327in}{2.289277in}}%
\pgfpathlineto{\pgfqpoint{3.561409in}{2.315844in}}%
\pgfusepath{stroke}%
\end{pgfscope}%
\begin{pgfscope}%
\pgfpathrectangle{\pgfqpoint{0.630330in}{0.548769in}}{\pgfqpoint{3.096674in}{1.753186in}}%
\pgfusepath{clip}%
\pgfsetrectcap%
\pgfsetroundjoin%
\pgfsetlinewidth{1.505625pt}%
\definecolor{currentstroke}{rgb}{1.000000,0.498039,0.054902}%
\pgfsetstrokecolor{currentstroke}%
\pgfsetdash{}{0pt}%
\pgfpathmoveto{\pgfqpoint{0.630330in}{0.548769in}}%
\pgfpathlineto{\pgfqpoint{0.661609in}{0.548769in}}%
\pgfpathlineto{\pgfqpoint{0.692889in}{0.548770in}}%
\pgfpathlineto{\pgfqpoint{0.724168in}{0.548773in}}%
\pgfpathlineto{\pgfqpoint{0.755448in}{0.548781in}}%
\pgfpathlineto{\pgfqpoint{0.786727in}{0.548798in}}%
\pgfpathlineto{\pgfqpoint{0.818007in}{0.548829in}}%
\pgfpathlineto{\pgfqpoint{0.849287in}{0.548880in}}%
\pgfpathlineto{\pgfqpoint{0.880566in}{0.548958in}}%
\pgfpathlineto{\pgfqpoint{0.911846in}{0.549072in}}%
\pgfpathlineto{\pgfqpoint{0.943125in}{0.549231in}}%
\pgfpathlineto{\pgfqpoint{0.974405in}{0.549445in}}%
\pgfpathlineto{\pgfqpoint{1.005684in}{0.549727in}}%
\pgfpathlineto{\pgfqpoint{1.036964in}{0.550088in}}%
\pgfpathlineto{\pgfqpoint{1.068243in}{0.550544in}}%
\pgfpathlineto{\pgfqpoint{1.099523in}{0.551108in}}%
\pgfpathlineto{\pgfqpoint{1.130802in}{0.551796in}}%
\pgfpathlineto{\pgfqpoint{1.162082in}{0.552627in}}%
\pgfpathlineto{\pgfqpoint{1.193361in}{0.553618in}}%
\pgfpathlineto{\pgfqpoint{1.224641in}{0.554789in}}%
\pgfpathlineto{\pgfqpoint{1.255921in}{0.556160in}}%
\pgfpathlineto{\pgfqpoint{1.287200in}{0.557753in}}%
\pgfpathlineto{\pgfqpoint{1.318480in}{0.559591in}}%
\pgfpathlineto{\pgfqpoint{1.349759in}{0.561697in}}%
\pgfpathlineto{\pgfqpoint{1.381039in}{0.564096in}}%
\pgfpathlineto{\pgfqpoint{1.412318in}{0.566815in}}%
\pgfpathlineto{\pgfqpoint{1.443598in}{0.569880in}}%
\pgfpathlineto{\pgfqpoint{1.474877in}{0.573320in}}%
\pgfpathlineto{\pgfqpoint{1.506157in}{0.577165in}}%
\pgfpathlineto{\pgfqpoint{1.537436in}{0.581444in}}%
\pgfpathlineto{\pgfqpoint{1.568716in}{0.586189in}}%
\pgfpathlineto{\pgfqpoint{1.599995in}{0.591434in}}%
\pgfpathlineto{\pgfqpoint{1.631275in}{0.597211in}}%
\pgfpathlineto{\pgfqpoint{1.662555in}{0.603556in}}%
\pgfpathlineto{\pgfqpoint{1.693834in}{0.610505in}}%
\pgfpathlineto{\pgfqpoint{1.725114in}{0.618095in}}%
\pgfpathlineto{\pgfqpoint{1.756393in}{0.626364in}}%
\pgfpathlineto{\pgfqpoint{1.787673in}{0.635351in}}%
\pgfpathlineto{\pgfqpoint{1.818952in}{0.645098in}}%
\pgfpathlineto{\pgfqpoint{1.850232in}{0.655645in}}%
\pgfpathlineto{\pgfqpoint{1.881511in}{0.667035in}}%
\pgfpathlineto{\pgfqpoint{1.912791in}{0.679313in}}%
\pgfpathlineto{\pgfqpoint{1.944070in}{0.692523in}}%
\pgfpathlineto{\pgfqpoint{1.975350in}{0.706710in}}%
\pgfpathlineto{\pgfqpoint{2.006629in}{0.721923in}}%
\pgfpathlineto{\pgfqpoint{2.037909in}{0.738209in}}%
\pgfpathlineto{\pgfqpoint{2.069189in}{0.755618in}}%
\pgfpathlineto{\pgfqpoint{2.100468in}{0.774200in}}%
\pgfpathlineto{\pgfqpoint{2.131748in}{0.794006in}}%
\pgfpathlineto{\pgfqpoint{2.163027in}{0.815091in}}%
\pgfpathlineto{\pgfqpoint{2.194307in}{0.837506in}}%
\pgfpathlineto{\pgfqpoint{2.225586in}{0.861307in}}%
\pgfpathlineto{\pgfqpoint{2.256866in}{0.886550in}}%
\pgfpathlineto{\pgfqpoint{2.288145in}{0.913292in}}%
\pgfpathlineto{\pgfqpoint{2.319425in}{0.941592in}}%
\pgfpathlineto{\pgfqpoint{2.350704in}{0.971508in}}%
\pgfpathlineto{\pgfqpoint{2.381984in}{1.003102in}}%
\pgfpathlineto{\pgfqpoint{2.413263in}{1.036434in}}%
\pgfpathlineto{\pgfqpoint{2.444543in}{1.071567in}}%
\pgfpathlineto{\pgfqpoint{2.475823in}{1.108565in}}%
\pgfpathlineto{\pgfqpoint{2.507102in}{1.147494in}}%
\pgfpathlineto{\pgfqpoint{2.538382in}{1.188418in}}%
\pgfpathlineto{\pgfqpoint{2.569661in}{1.231405in}}%
\pgfpathlineto{\pgfqpoint{2.600941in}{1.276523in}}%
\pgfpathlineto{\pgfqpoint{2.632220in}{1.323841in}}%
\pgfpathlineto{\pgfqpoint{2.663500in}{1.373430in}}%
\pgfpathlineto{\pgfqpoint{2.694779in}{1.425362in}}%
\pgfpathlineto{\pgfqpoint{2.726059in}{1.479708in}}%
\pgfpathlineto{\pgfqpoint{2.757338in}{1.536544in}}%
\pgfpathlineto{\pgfqpoint{2.788618in}{1.595942in}}%
\pgfpathlineto{\pgfqpoint{2.819897in}{1.657980in}}%
\pgfpathlineto{\pgfqpoint{2.851177in}{1.722735in}}%
\pgfpathlineto{\pgfqpoint{2.882457in}{1.790285in}}%
\pgfpathlineto{\pgfqpoint{2.913736in}{1.860708in}}%
\pgfpathlineto{\pgfqpoint{2.945016in}{1.934086in}}%
\pgfpathlineto{\pgfqpoint{2.976295in}{2.010499in}}%
\pgfpathlineto{\pgfqpoint{3.007575in}{2.090031in}}%
\pgfpathlineto{\pgfqpoint{3.038854in}{2.172766in}}%
\pgfpathlineto{\pgfqpoint{3.070134in}{2.258787in}}%
\pgfpathlineto{\pgfqpoint{3.090098in}{2.315844in}}%
\pgfusepath{stroke}%
\end{pgfscope}%
\begin{pgfscope}%
\pgfpathrectangle{\pgfqpoint{0.630330in}{0.548769in}}{\pgfqpoint{3.096674in}{1.753186in}}%
\pgfusepath{clip}%
\pgfsetrectcap%
\pgfsetroundjoin%
\pgfsetlinewidth{1.505625pt}%
\definecolor{currentstroke}{rgb}{0.172549,0.627451,0.172549}%
\pgfsetstrokecolor{currentstroke}%
\pgfsetdash{}{0pt}%
\pgfpathmoveto{\pgfqpoint{0.630330in}{0.548769in}}%
\pgfpathlineto{\pgfqpoint{0.661609in}{0.548769in}}%
\pgfpathlineto{\pgfqpoint{0.692889in}{0.548769in}}%
\pgfpathlineto{\pgfqpoint{0.724168in}{0.548769in}}%
\pgfpathlineto{\pgfqpoint{0.755448in}{0.548769in}}%
\pgfpathlineto{\pgfqpoint{0.786727in}{0.548769in}}%
\pgfpathlineto{\pgfqpoint{0.818007in}{0.548769in}}%
\pgfpathlineto{\pgfqpoint{0.849287in}{0.548770in}}%
\pgfpathlineto{\pgfqpoint{0.880566in}{0.548772in}}%
\pgfpathlineto{\pgfqpoint{0.911846in}{0.548774in}}%
\pgfpathlineto{\pgfqpoint{0.943125in}{0.548779in}}%
\pgfpathlineto{\pgfqpoint{0.974405in}{0.548788in}}%
\pgfpathlineto{\pgfqpoint{1.005684in}{0.548800in}}%
\pgfpathlineto{\pgfqpoint{1.036964in}{0.548820in}}%
\pgfpathlineto{\pgfqpoint{1.068243in}{0.548849in}}%
\pgfpathlineto{\pgfqpoint{1.099523in}{0.548890in}}%
\pgfpathlineto{\pgfqpoint{1.130802in}{0.548947in}}%
\pgfpathlineto{\pgfqpoint{1.162082in}{0.549025in}}%
\pgfpathlineto{\pgfqpoint{1.193361in}{0.549130in}}%
\pgfpathlineto{\pgfqpoint{1.224641in}{0.549268in}}%
\pgfpathlineto{\pgfqpoint{1.255921in}{0.549448in}}%
\pgfpathlineto{\pgfqpoint{1.287200in}{0.549678in}}%
\pgfpathlineto{\pgfqpoint{1.318480in}{0.549971in}}%
\pgfpathlineto{\pgfqpoint{1.349759in}{0.550339in}}%
\pgfpathlineto{\pgfqpoint{1.381039in}{0.550796in}}%
\pgfpathlineto{\pgfqpoint{1.412318in}{0.551358in}}%
\pgfpathlineto{\pgfqpoint{1.443598in}{0.552045in}}%
\pgfpathlineto{\pgfqpoint{1.474877in}{0.552878in}}%
\pgfpathlineto{\pgfqpoint{1.506157in}{0.553880in}}%
\pgfpathlineto{\pgfqpoint{1.537436in}{0.555077in}}%
\pgfpathlineto{\pgfqpoint{1.568716in}{0.556500in}}%
\pgfpathlineto{\pgfqpoint{1.599995in}{0.558181in}}%
\pgfpathlineto{\pgfqpoint{1.631275in}{0.560156in}}%
\pgfpathlineto{\pgfqpoint{1.662555in}{0.562466in}}%
\pgfpathlineto{\pgfqpoint{1.693834in}{0.565152in}}%
\pgfpathlineto{\pgfqpoint{1.725114in}{0.568265in}}%
\pgfpathlineto{\pgfqpoint{1.756393in}{0.571855in}}%
\pgfpathlineto{\pgfqpoint{1.787673in}{0.575980in}}%
\pgfpathlineto{\pgfqpoint{1.818952in}{0.580702in}}%
\pgfpathlineto{\pgfqpoint{1.850232in}{0.586087in}}%
\pgfpathlineto{\pgfqpoint{1.881511in}{0.592209in}}%
\pgfpathlineto{\pgfqpoint{1.912791in}{0.599146in}}%
\pgfpathlineto{\pgfqpoint{1.944070in}{0.606983in}}%
\pgfpathlineto{\pgfqpoint{1.975350in}{0.615811in}}%
\pgfpathlineto{\pgfqpoint{2.006629in}{0.625726in}}%
\pgfpathlineto{\pgfqpoint{2.037909in}{0.636835in}}%
\pgfpathlineto{\pgfqpoint{2.069189in}{0.649249in}}%
\pgfpathlineto{\pgfqpoint{2.100468in}{0.663089in}}%
\pgfpathlineto{\pgfqpoint{2.131748in}{0.678481in}}%
\pgfpathlineto{\pgfqpoint{2.163027in}{0.695564in}}%
\pgfpathlineto{\pgfqpoint{2.194307in}{0.714481in}}%
\pgfpathlineto{\pgfqpoint{2.225586in}{0.735388in}}%
\pgfpathlineto{\pgfqpoint{2.256866in}{0.758448in}}%
\pgfpathlineto{\pgfqpoint{2.288145in}{0.783835in}}%
\pgfpathlineto{\pgfqpoint{2.319425in}{0.811733in}}%
\pgfpathlineto{\pgfqpoint{2.350704in}{0.842338in}}%
\pgfpathlineto{\pgfqpoint{2.381984in}{0.875855in}}%
\pgfpathlineto{\pgfqpoint{2.413263in}{0.912502in}}%
\pgfpathlineto{\pgfqpoint{2.444543in}{0.952509in}}%
\pgfpathlineto{\pgfqpoint{2.475823in}{0.996118in}}%
\pgfpathlineto{\pgfqpoint{2.507102in}{1.043583in}}%
\pgfpathlineto{\pgfqpoint{2.538382in}{1.095172in}}%
\pgfpathlineto{\pgfqpoint{2.569661in}{1.151168in}}%
\pgfpathlineto{\pgfqpoint{2.600941in}{1.211867in}}%
\pgfpathlineto{\pgfqpoint{2.632220in}{1.277579in}}%
\pgfpathlineto{\pgfqpoint{2.663500in}{1.348630in}}%
\pgfpathlineto{\pgfqpoint{2.694779in}{1.425362in}}%
\pgfpathlineto{\pgfqpoint{2.726059in}{1.508132in}}%
\pgfpathlineto{\pgfqpoint{2.757338in}{1.597316in}}%
\pgfpathlineto{\pgfqpoint{2.788618in}{1.693303in}}%
\pgfpathlineto{\pgfqpoint{2.819897in}{1.796505in}}%
\pgfpathlineto{\pgfqpoint{2.851177in}{1.907347in}}%
\pgfpathlineto{\pgfqpoint{2.882457in}{2.026275in}}%
\pgfpathlineto{\pgfqpoint{2.913736in}{2.153756in}}%
\pgfpathlineto{\pgfqpoint{2.945016in}{2.290274in}}%
\pgfpathlineto{\pgfqpoint{2.950492in}{2.315844in}}%
\pgfusepath{stroke}%
\end{pgfscope}%
\begin{pgfscope}%
\pgfpathrectangle{\pgfqpoint{0.630330in}{0.548769in}}{\pgfqpoint{3.096674in}{1.753186in}}%
\pgfusepath{clip}%
\pgfsetrectcap%
\pgfsetroundjoin%
\pgfsetlinewidth{1.505625pt}%
\definecolor{currentstroke}{rgb}{0.839216,0.152941,0.156863}%
\pgfsetstrokecolor{currentstroke}%
\pgfsetdash{}{0pt}%
\pgfpathmoveto{\pgfqpoint{0.630330in}{0.548769in}}%
\pgfpathlineto{\pgfqpoint{0.661609in}{0.548769in}}%
\pgfpathlineto{\pgfqpoint{0.692889in}{0.548769in}}%
\pgfpathlineto{\pgfqpoint{0.724168in}{0.548769in}}%
\pgfpathlineto{\pgfqpoint{0.755448in}{0.548769in}}%
\pgfpathlineto{\pgfqpoint{0.786727in}{0.548769in}}%
\pgfpathlineto{\pgfqpoint{0.818007in}{0.548769in}}%
\pgfpathlineto{\pgfqpoint{0.849287in}{0.548769in}}%
\pgfpathlineto{\pgfqpoint{0.880566in}{0.548769in}}%
\pgfpathlineto{\pgfqpoint{0.911846in}{0.548769in}}%
\pgfpathlineto{\pgfqpoint{0.943125in}{0.548769in}}%
\pgfpathlineto{\pgfqpoint{0.974405in}{0.548769in}}%
\pgfpathlineto{\pgfqpoint{1.005684in}{0.548770in}}%
\pgfpathlineto{\pgfqpoint{1.036964in}{0.548771in}}%
\pgfpathlineto{\pgfqpoint{1.068243in}{0.548772in}}%
\pgfpathlineto{\pgfqpoint{1.099523in}{0.548775in}}%
\pgfpathlineto{\pgfqpoint{1.130802in}{0.548779in}}%
\pgfpathlineto{\pgfqpoint{1.162082in}{0.548786in}}%
\pgfpathlineto{\pgfqpoint{1.193361in}{0.548796in}}%
\pgfpathlineto{\pgfqpoint{1.224641in}{0.548810in}}%
\pgfpathlineto{\pgfqpoint{1.255921in}{0.548831in}}%
\pgfpathlineto{\pgfqpoint{1.287200in}{0.548861in}}%
\pgfpathlineto{\pgfqpoint{1.318480in}{0.548902in}}%
\pgfpathlineto{\pgfqpoint{1.349759in}{0.548959in}}%
\pgfpathlineto{\pgfqpoint{1.381039in}{0.549037in}}%
\pgfpathlineto{\pgfqpoint{1.412318in}{0.549140in}}%
\pgfpathlineto{\pgfqpoint{1.443598in}{0.549277in}}%
\pgfpathlineto{\pgfqpoint{1.474877in}{0.549456in}}%
\pgfpathlineto{\pgfqpoint{1.506157in}{0.549689in}}%
\pgfpathlineto{\pgfqpoint{1.537436in}{0.549987in}}%
\pgfpathlineto{\pgfqpoint{1.568716in}{0.550366in}}%
\pgfpathlineto{\pgfqpoint{1.599995in}{0.550845in}}%
\pgfpathlineto{\pgfqpoint{1.631275in}{0.551446in}}%
\pgfpathlineto{\pgfqpoint{1.662555in}{0.552193in}}%
\pgfpathlineto{\pgfqpoint{1.693834in}{0.553117in}}%
\pgfpathlineto{\pgfqpoint{1.725114in}{0.554251in}}%
\pgfpathlineto{\pgfqpoint{1.756393in}{0.555637in}}%
\pgfpathlineto{\pgfqpoint{1.787673in}{0.557321in}}%
\pgfpathlineto{\pgfqpoint{1.818952in}{0.559354in}}%
\pgfpathlineto{\pgfqpoint{1.850232in}{0.561799in}}%
\pgfpathlineto{\pgfqpoint{1.881511in}{0.564725in}}%
\pgfpathlineto{\pgfqpoint{1.912791in}{0.568210in}}%
\pgfpathlineto{\pgfqpoint{1.944070in}{0.572343in}}%
\pgfpathlineto{\pgfqpoint{1.975350in}{0.577226in}}%
\pgfpathlineto{\pgfqpoint{2.006629in}{0.582972in}}%
\pgfpathlineto{\pgfqpoint{2.037909in}{0.589709in}}%
\pgfpathlineto{\pgfqpoint{2.069189in}{0.597579in}}%
\pgfpathlineto{\pgfqpoint{2.100468in}{0.606742in}}%
\pgfpathlineto{\pgfqpoint{2.131748in}{0.617377in}}%
\pgfpathlineto{\pgfqpoint{2.163027in}{0.629681in}}%
\pgfpathlineto{\pgfqpoint{2.194307in}{0.643875in}}%
\pgfpathlineto{\pgfqpoint{2.225586in}{0.660200in}}%
\pgfpathlineto{\pgfqpoint{2.256866in}{0.678928in}}%
\pgfpathlineto{\pgfqpoint{2.288145in}{0.700353in}}%
\pgfpathlineto{\pgfqpoint{2.319425in}{0.724803in}}%
\pgfpathlineto{\pgfqpoint{2.350704in}{0.752636in}}%
\pgfpathlineto{\pgfqpoint{2.381984in}{0.784247in}}%
\pgfpathlineto{\pgfqpoint{2.413263in}{0.820066in}}%
\pgfpathlineto{\pgfqpoint{2.444543in}{0.860565in}}%
\pgfpathlineto{\pgfqpoint{2.475823in}{0.906258in}}%
\pgfpathlineto{\pgfqpoint{2.507102in}{0.957706in}}%
\pgfpathlineto{\pgfqpoint{2.538382in}{1.015520in}}%
\pgfpathlineto{\pgfqpoint{2.569661in}{1.080363in}}%
\pgfpathlineto{\pgfqpoint{2.600941in}{1.152955in}}%
\pgfpathlineto{\pgfqpoint{2.632220in}{1.234078in}}%
\pgfpathlineto{\pgfqpoint{2.663500in}{1.324575in}}%
\pgfpathlineto{\pgfqpoint{2.694779in}{1.425362in}}%
\pgfpathlineto{\pgfqpoint{2.726059in}{1.537424in}}%
\pgfpathlineto{\pgfqpoint{2.757338in}{1.661827in}}%
\pgfpathlineto{\pgfqpoint{2.788618in}{1.799717in}}%
\pgfpathlineto{\pgfqpoint{2.819897in}{1.952328in}}%
\pgfpathlineto{\pgfqpoint{2.851177in}{2.120989in}}%
\pgfpathlineto{\pgfqpoint{2.882457in}{2.307124in}}%
\pgfpathlineto{\pgfqpoint{2.883786in}{2.315844in}}%
\pgfusepath{stroke}%
\end{pgfscope}%
\begin{pgfscope}%
\pgfsetrectcap%
\pgfsetmiterjoin%
\pgfsetlinewidth{0.803000pt}%
\definecolor{currentstroke}{rgb}{0.000000,0.000000,0.000000}%
\pgfsetstrokecolor{currentstroke}%
\pgfsetdash{}{0pt}%
\pgfpathmoveto{\pgfqpoint{0.630330in}{0.548769in}}%
\pgfpathlineto{\pgfqpoint{0.630330in}{2.301955in}}%
\pgfusepath{stroke}%
\end{pgfscope}%
\begin{pgfscope}%
\pgfsetrectcap%
\pgfsetmiterjoin%
\pgfsetlinewidth{0.803000pt}%
\definecolor{currentstroke}{rgb}{0.000000,0.000000,0.000000}%
\pgfsetstrokecolor{currentstroke}%
\pgfsetdash{}{0pt}%
\pgfpathmoveto{\pgfqpoint{3.727004in}{0.548769in}}%
\pgfpathlineto{\pgfqpoint{3.727004in}{2.301955in}}%
\pgfusepath{stroke}%
\end{pgfscope}%
\begin{pgfscope}%
\pgfsetrectcap%
\pgfsetmiterjoin%
\pgfsetlinewidth{0.803000pt}%
\definecolor{currentstroke}{rgb}{0.000000,0.000000,0.000000}%
\pgfsetstrokecolor{currentstroke}%
\pgfsetdash{}{0pt}%
\pgfpathmoveto{\pgfqpoint{0.630330in}{0.548769in}}%
\pgfpathlineto{\pgfqpoint{3.727004in}{0.548769in}}%
\pgfusepath{stroke}%
\end{pgfscope}%
\begin{pgfscope}%
\pgfsetrectcap%
\pgfsetmiterjoin%
\pgfsetlinewidth{0.803000pt}%
\definecolor{currentstroke}{rgb}{0.000000,0.000000,0.000000}%
\pgfsetstrokecolor{currentstroke}%
\pgfsetdash{}{0pt}%
\pgfpathmoveto{\pgfqpoint{0.630330in}{2.301955in}}%
\pgfpathlineto{\pgfqpoint{3.727004in}{2.301955in}}%
\pgfusepath{stroke}%
\end{pgfscope}%
\begin{pgfscope}%
\pgfsetbuttcap%
\pgfsetmiterjoin%
\definecolor{currentfill}{rgb}{1.000000,1.000000,1.000000}%
\pgfsetfillcolor{currentfill}%
\pgfsetfillopacity{0.800000}%
\pgfsetlinewidth{1.003750pt}%
\definecolor{currentstroke}{rgb}{0.800000,0.800000,0.800000}%
\pgfsetstrokecolor{currentstroke}%
\pgfsetstrokeopacity{0.800000}%
\pgfsetdash{}{0pt}%
\pgfpathmoveto{\pgfqpoint{0.727552in}{1.416153in}}%
\pgfpathlineto{\pgfqpoint{1.553360in}{1.416153in}}%
\pgfpathquadraticcurveto{\pgfqpoint{1.581138in}{1.416153in}}{\pgfqpoint{1.581138in}{1.443930in}}%
\pgfpathlineto{\pgfqpoint{1.581138in}{2.204733in}}%
\pgfpathquadraticcurveto{\pgfqpoint{1.581138in}{2.232510in}}{\pgfqpoint{1.553360in}{2.232510in}}%
\pgfpathlineto{\pgfqpoint{0.727552in}{2.232510in}}%
\pgfpathquadraticcurveto{\pgfqpoint{0.699774in}{2.232510in}}{\pgfqpoint{0.699774in}{2.204733in}}%
\pgfpathlineto{\pgfqpoint{0.699774in}{1.443930in}}%
\pgfpathquadraticcurveto{\pgfqpoint{0.699774in}{1.416153in}}{\pgfqpoint{0.727552in}{1.416153in}}%
\pgfpathlineto{\pgfqpoint{0.727552in}{1.416153in}}%
\pgfpathclose%
\pgfusepath{stroke,fill}%
\end{pgfscope}%
\begin{pgfscope}%
\pgfsetrectcap%
\pgfsetroundjoin%
\pgfsetlinewidth{1.505625pt}%
\definecolor{currentstroke}{rgb}{0.121569,0.466667,0.705882}%
\pgfsetstrokecolor{currentstroke}%
\pgfsetdash{}{0pt}%
\pgfpathmoveto{\pgfqpoint{0.755330in}{2.128344in}}%
\pgfpathlineto{\pgfqpoint{0.894219in}{2.128344in}}%
\pgfpathlineto{\pgfqpoint{1.033108in}{2.128344in}}%
\pgfusepath{stroke}%
\end{pgfscope}%
\begin{pgfscope}%
\definecolor{textcolor}{rgb}{0.000000,0.000000,0.000000}%
\pgfsetstrokecolor{textcolor}%
\pgfsetfillcolor{textcolor}%
\pgftext[x=1.144219in,y=2.079733in,left,base]{\color{textcolor}\rmfamily\fontsize{10.000000}{12.000000}\selectfont \(\displaystyle N=1\)}%
\end{pgfscope}%
\begin{pgfscope}%
\pgfsetrectcap%
\pgfsetroundjoin%
\pgfsetlinewidth{1.505625pt}%
\definecolor{currentstroke}{rgb}{1.000000,0.498039,0.054902}%
\pgfsetstrokecolor{currentstroke}%
\pgfsetdash{}{0pt}%
\pgfpathmoveto{\pgfqpoint{0.755330in}{1.934671in}}%
\pgfpathlineto{\pgfqpoint{0.894219in}{1.934671in}}%
\pgfpathlineto{\pgfqpoint{1.033108in}{1.934671in}}%
\pgfusepath{stroke}%
\end{pgfscope}%
\begin{pgfscope}%
\definecolor{textcolor}{rgb}{0.000000,0.000000,0.000000}%
\pgfsetstrokecolor{textcolor}%
\pgfsetfillcolor{textcolor}%
\pgftext[x=1.144219in,y=1.886060in,left,base]{\color{textcolor}\rmfamily\fontsize{10.000000}{12.000000}\selectfont \(\displaystyle N=2\)}%
\end{pgfscope}%
\begin{pgfscope}%
\pgfsetrectcap%
\pgfsetroundjoin%
\pgfsetlinewidth{1.505625pt}%
\definecolor{currentstroke}{rgb}{0.172549,0.627451,0.172549}%
\pgfsetstrokecolor{currentstroke}%
\pgfsetdash{}{0pt}%
\pgfpathmoveto{\pgfqpoint{0.755330in}{1.740998in}}%
\pgfpathlineto{\pgfqpoint{0.894219in}{1.740998in}}%
\pgfpathlineto{\pgfqpoint{1.033108in}{1.740998in}}%
\pgfusepath{stroke}%
\end{pgfscope}%
\begin{pgfscope}%
\definecolor{textcolor}{rgb}{0.000000,0.000000,0.000000}%
\pgfsetstrokecolor{textcolor}%
\pgfsetfillcolor{textcolor}%
\pgftext[x=1.144219in,y=1.692387in,left,base]{\color{textcolor}\rmfamily\fontsize{10.000000}{12.000000}\selectfont \(\displaystyle N=3\)}%
\end{pgfscope}%
\begin{pgfscope}%
\pgfsetrectcap%
\pgfsetroundjoin%
\pgfsetlinewidth{1.505625pt}%
\definecolor{currentstroke}{rgb}{0.839216,0.152941,0.156863}%
\pgfsetstrokecolor{currentstroke}%
\pgfsetdash{}{0pt}%
\pgfpathmoveto{\pgfqpoint{0.755330in}{1.547325in}}%
\pgfpathlineto{\pgfqpoint{0.894219in}{1.547325in}}%
\pgfpathlineto{\pgfqpoint{1.033108in}{1.547325in}}%
\pgfusepath{stroke}%
\end{pgfscope}%
\begin{pgfscope}%
\definecolor{textcolor}{rgb}{0.000000,0.000000,0.000000}%
\pgfsetstrokecolor{textcolor}%
\pgfsetfillcolor{textcolor}%
\pgftext[x=1.144219in,y=1.498714in,left,base]{\color{textcolor}\rmfamily\fontsize{10.000000}{12.000000}\selectfont \(\displaystyle N=4\)}%
\end{pgfscope}%
\end{pgfpicture}%
\makeatother%
\endgroup%

    \caption{$F_N$ für Butterworth filter. Der grüne Bereich definiert die erlaubten Werte für alle $F_N$-Funktionen.}
    \label{ellfilter:fig:butterworth}
\end{figure}

wenn $F_N(w)$ eine rationale Funktion ist, ist auch $H(\Omega)$ eine rationale Funktion und daher ein lineares Filter. %proof?

\begin{align}
    F_N(w) & =
    \begin{cases}
        w^N                            & \text{Butterworth} \\
        T_N(w)                         & \text{Tschebyscheff, Typ 1}  \\
        [k_1 T_N (k^{-1} w^{-1})]^{-1} & \text{Tschebyscheff, Typ 2}  \\
        R_N(w, \xi)                    & \text{Elliptisch (Cauer)}    \\
    \end{cases}
\end{align}

Mit der Ausnahme vom Butterworth filter sind alle Filter nach speziellen Funktionen benannt.
Alle diese Filter sind optimal für unterschiedliche Anwendungsgebiete.
Das Butterworth-Filter, zum Beispiel, ist maximal flach im Durchlassbereich.
Das Tschebyscheff-1 Filter sind maximal steil für eine definierte Welligkeit im Durchlassbereich, währendem es im Sperrbereich monoton abfallend ist.
Es scheint so als sind gewisse Eigenschaften dieser speziellen Funktionen verantwortlich für die Optimalität dieser Filter.

\section{Tschebyscheff-Filter}

Als Einstieg betrachent Wir das Tschebyscheff-Filter, welches sehr verwand ist mit dem elliptischen Filter.
Genauer ausgedrückt sind die Tschebyscheff-1 und -2 Filter Spezialfälle davon.

Der Name des Filters deutet schon an, dass die Tschebyscheff-Polynome $T_N$ für das Filter relevant sind:
\begin{align}
    T_{0}(x)&=1\\
    T_{1}(x)&=x\\
    T_{2}(x)&=2x^{2}-1\\
    T_{3}(x)&=4x^{3}-3x\\
    T_{n+1}(x)&=2x~T_{n}(x)-T_{n-1}(x).
\end{align}
Bemerkenswert ist, dass die Polynome im Intervall $[-1, 1]$ mit der trigonometrischen Funktion
\begin{align} \label{ellfilter:eq:chebychef_polynomials}
    T_N(w) &= \cos \left( N \cos^{-1}(w) \right) \\
           &= \cos \left(N~z \right), \quad w= \cos(z)
\end{align}
übereinstimmt.
Der Zusammenhang lässt sich mit den Doppel- und Mehrfachwinkelfunktionen der trigonometrischen Funktionen erklären.
Abbildung \ref{ellfilter:fig:chebychef_polynomials} zeigt einige Tschebyscheff-Polynome.
\begin{figure}
    \centering
    %% Creator: Matplotlib, PGF backend
%%
%% To include the figure in your LaTeX document, write
%%   \input{<filename>.pgf}
%%
%% Make sure the required packages are loaded in your preamble
%%   \usepackage{pgf}
%%
%% Also ensure that all the required font packages are loaded; for instance,
%% the lmodern package is sometimes necessary when using math font.
%%   \usepackage{lmodern}
%%
%% Figures using additional raster images can only be included by \input if
%% they are in the same directory as the main LaTeX file. For loading figures
%% from other directories you can use the `import` package
%%   \usepackage{import}
%%
%% and then include the figures with
%%   \import{<path to file>}{<filename>.pgf}
%%
%% Matplotlib used the following preamble
%%
\begingroup%
\makeatletter%
\begin{pgfpicture}%
\pgfpathrectangle{\pgfpointorigin}{\pgfqpoint{5.500000in}{2.500000in}}%
\pgfusepath{use as bounding box, clip}%
\begin{pgfscope}%
\pgfsetbuttcap%
\pgfsetmiterjoin%
\pgfsetlinewidth{0.000000pt}%
\definecolor{currentstroke}{rgb}{1.000000,1.000000,1.000000}%
\pgfsetstrokecolor{currentstroke}%
\pgfsetstrokeopacity{0.000000}%
\pgfsetdash{}{0pt}%
\pgfpathmoveto{\pgfqpoint{0.000000in}{0.000000in}}%
\pgfpathlineto{\pgfqpoint{5.500000in}{0.000000in}}%
\pgfpathlineto{\pgfqpoint{5.500000in}{2.500000in}}%
\pgfpathlineto{\pgfqpoint{0.000000in}{2.500000in}}%
\pgfpathlineto{\pgfqpoint{0.000000in}{0.000000in}}%
\pgfpathclose%
\pgfusepath{}%
\end{pgfscope}%
\begin{pgfscope}%
\pgfsetbuttcap%
\pgfsetmiterjoin%
\definecolor{currentfill}{rgb}{1.000000,1.000000,1.000000}%
\pgfsetfillcolor{currentfill}%
\pgfsetlinewidth{0.000000pt}%
\definecolor{currentstroke}{rgb}{0.000000,0.000000,0.000000}%
\pgfsetstrokecolor{currentstroke}%
\pgfsetstrokeopacity{0.000000}%
\pgfsetdash{}{0pt}%
\pgfpathmoveto{\pgfqpoint{0.617954in}{0.548769in}}%
\pgfpathlineto{\pgfqpoint{5.350000in}{0.548769in}}%
\pgfpathlineto{\pgfqpoint{5.350000in}{2.301955in}}%
\pgfpathlineto{\pgfqpoint{0.617954in}{2.301955in}}%
\pgfpathlineto{\pgfqpoint{0.617954in}{0.548769in}}%
\pgfpathclose%
\pgfusepath{fill}%
\end{pgfscope}%
\begin{pgfscope}%
\pgfpathrectangle{\pgfqpoint{0.617954in}{0.548769in}}{\pgfqpoint{4.732046in}{1.753186in}}%
\pgfusepath{clip}%
\pgfsetrectcap%
\pgfsetroundjoin%
\pgfsetlinewidth{0.803000pt}%
\definecolor{currentstroke}{rgb}{0.690196,0.690196,0.690196}%
\pgfsetstrokecolor{currentstroke}%
\pgfsetdash{}{0pt}%
\pgfpathmoveto{\pgfqpoint{1.012292in}{0.548769in}}%
\pgfpathlineto{\pgfqpoint{1.012292in}{2.301955in}}%
\pgfusepath{stroke}%
\end{pgfscope}%
\begin{pgfscope}%
\pgfsetbuttcap%
\pgfsetroundjoin%
\definecolor{currentfill}{rgb}{0.000000,0.000000,0.000000}%
\pgfsetfillcolor{currentfill}%
\pgfsetlinewidth{0.803000pt}%
\definecolor{currentstroke}{rgb}{0.000000,0.000000,0.000000}%
\pgfsetstrokecolor{currentstroke}%
\pgfsetdash{}{0pt}%
\pgfsys@defobject{currentmarker}{\pgfqpoint{0.000000in}{-0.048611in}}{\pgfqpoint{0.000000in}{0.000000in}}{%
\pgfpathmoveto{\pgfqpoint{0.000000in}{0.000000in}}%
\pgfpathlineto{\pgfqpoint{0.000000in}{-0.048611in}}%
\pgfusepath{stroke,fill}%
}%
\begin{pgfscope}%
\pgfsys@transformshift{1.012292in}{0.548769in}%
\pgfsys@useobject{currentmarker}{}%
\end{pgfscope}%
\end{pgfscope}%
\begin{pgfscope}%
\definecolor{textcolor}{rgb}{0.000000,0.000000,0.000000}%
\pgfsetstrokecolor{textcolor}%
\pgfsetfillcolor{textcolor}%
\pgftext[x=1.012292in,y=0.451547in,,top]{\color{textcolor}\rmfamily\fontsize{10.000000}{12.000000}\selectfont \(\displaystyle {\ensuremath{-}1.0}\)}%
\end{pgfscope}%
\begin{pgfscope}%
\pgfpathrectangle{\pgfqpoint{0.617954in}{0.548769in}}{\pgfqpoint{4.732046in}{1.753186in}}%
\pgfusepath{clip}%
\pgfsetrectcap%
\pgfsetroundjoin%
\pgfsetlinewidth{0.803000pt}%
\definecolor{currentstroke}{rgb}{0.690196,0.690196,0.690196}%
\pgfsetstrokecolor{currentstroke}%
\pgfsetdash{}{0pt}%
\pgfpathmoveto{\pgfqpoint{1.998134in}{0.548769in}}%
\pgfpathlineto{\pgfqpoint{1.998134in}{2.301955in}}%
\pgfusepath{stroke}%
\end{pgfscope}%
\begin{pgfscope}%
\pgfsetbuttcap%
\pgfsetroundjoin%
\definecolor{currentfill}{rgb}{0.000000,0.000000,0.000000}%
\pgfsetfillcolor{currentfill}%
\pgfsetlinewidth{0.803000pt}%
\definecolor{currentstroke}{rgb}{0.000000,0.000000,0.000000}%
\pgfsetstrokecolor{currentstroke}%
\pgfsetdash{}{0pt}%
\pgfsys@defobject{currentmarker}{\pgfqpoint{0.000000in}{-0.048611in}}{\pgfqpoint{0.000000in}{0.000000in}}{%
\pgfpathmoveto{\pgfqpoint{0.000000in}{0.000000in}}%
\pgfpathlineto{\pgfqpoint{0.000000in}{-0.048611in}}%
\pgfusepath{stroke,fill}%
}%
\begin{pgfscope}%
\pgfsys@transformshift{1.998134in}{0.548769in}%
\pgfsys@useobject{currentmarker}{}%
\end{pgfscope}%
\end{pgfscope}%
\begin{pgfscope}%
\definecolor{textcolor}{rgb}{0.000000,0.000000,0.000000}%
\pgfsetstrokecolor{textcolor}%
\pgfsetfillcolor{textcolor}%
\pgftext[x=1.998134in,y=0.451547in,,top]{\color{textcolor}\rmfamily\fontsize{10.000000}{12.000000}\selectfont \(\displaystyle {\ensuremath{-}0.5}\)}%
\end{pgfscope}%
\begin{pgfscope}%
\pgfpathrectangle{\pgfqpoint{0.617954in}{0.548769in}}{\pgfqpoint{4.732046in}{1.753186in}}%
\pgfusepath{clip}%
\pgfsetrectcap%
\pgfsetroundjoin%
\pgfsetlinewidth{0.803000pt}%
\definecolor{currentstroke}{rgb}{0.690196,0.690196,0.690196}%
\pgfsetstrokecolor{currentstroke}%
\pgfsetdash{}{0pt}%
\pgfpathmoveto{\pgfqpoint{2.983977in}{0.548769in}}%
\pgfpathlineto{\pgfqpoint{2.983977in}{2.301955in}}%
\pgfusepath{stroke}%
\end{pgfscope}%
\begin{pgfscope}%
\pgfsetbuttcap%
\pgfsetroundjoin%
\definecolor{currentfill}{rgb}{0.000000,0.000000,0.000000}%
\pgfsetfillcolor{currentfill}%
\pgfsetlinewidth{0.803000pt}%
\definecolor{currentstroke}{rgb}{0.000000,0.000000,0.000000}%
\pgfsetstrokecolor{currentstroke}%
\pgfsetdash{}{0pt}%
\pgfsys@defobject{currentmarker}{\pgfqpoint{0.000000in}{-0.048611in}}{\pgfqpoint{0.000000in}{0.000000in}}{%
\pgfpathmoveto{\pgfqpoint{0.000000in}{0.000000in}}%
\pgfpathlineto{\pgfqpoint{0.000000in}{-0.048611in}}%
\pgfusepath{stroke,fill}%
}%
\begin{pgfscope}%
\pgfsys@transformshift{2.983977in}{0.548769in}%
\pgfsys@useobject{currentmarker}{}%
\end{pgfscope}%
\end{pgfscope}%
\begin{pgfscope}%
\definecolor{textcolor}{rgb}{0.000000,0.000000,0.000000}%
\pgfsetstrokecolor{textcolor}%
\pgfsetfillcolor{textcolor}%
\pgftext[x=2.983977in,y=0.451547in,,top]{\color{textcolor}\rmfamily\fontsize{10.000000}{12.000000}\selectfont \(\displaystyle {0.0}\)}%
\end{pgfscope}%
\begin{pgfscope}%
\pgfpathrectangle{\pgfqpoint{0.617954in}{0.548769in}}{\pgfqpoint{4.732046in}{1.753186in}}%
\pgfusepath{clip}%
\pgfsetrectcap%
\pgfsetroundjoin%
\pgfsetlinewidth{0.803000pt}%
\definecolor{currentstroke}{rgb}{0.690196,0.690196,0.690196}%
\pgfsetstrokecolor{currentstroke}%
\pgfsetdash{}{0pt}%
\pgfpathmoveto{\pgfqpoint{3.969820in}{0.548769in}}%
\pgfpathlineto{\pgfqpoint{3.969820in}{2.301955in}}%
\pgfusepath{stroke}%
\end{pgfscope}%
\begin{pgfscope}%
\pgfsetbuttcap%
\pgfsetroundjoin%
\definecolor{currentfill}{rgb}{0.000000,0.000000,0.000000}%
\pgfsetfillcolor{currentfill}%
\pgfsetlinewidth{0.803000pt}%
\definecolor{currentstroke}{rgb}{0.000000,0.000000,0.000000}%
\pgfsetstrokecolor{currentstroke}%
\pgfsetdash{}{0pt}%
\pgfsys@defobject{currentmarker}{\pgfqpoint{0.000000in}{-0.048611in}}{\pgfqpoint{0.000000in}{0.000000in}}{%
\pgfpathmoveto{\pgfqpoint{0.000000in}{0.000000in}}%
\pgfpathlineto{\pgfqpoint{0.000000in}{-0.048611in}}%
\pgfusepath{stroke,fill}%
}%
\begin{pgfscope}%
\pgfsys@transformshift{3.969820in}{0.548769in}%
\pgfsys@useobject{currentmarker}{}%
\end{pgfscope}%
\end{pgfscope}%
\begin{pgfscope}%
\definecolor{textcolor}{rgb}{0.000000,0.000000,0.000000}%
\pgfsetstrokecolor{textcolor}%
\pgfsetfillcolor{textcolor}%
\pgftext[x=3.969820in,y=0.451547in,,top]{\color{textcolor}\rmfamily\fontsize{10.000000}{12.000000}\selectfont \(\displaystyle {0.5}\)}%
\end{pgfscope}%
\begin{pgfscope}%
\pgfpathrectangle{\pgfqpoint{0.617954in}{0.548769in}}{\pgfqpoint{4.732046in}{1.753186in}}%
\pgfusepath{clip}%
\pgfsetrectcap%
\pgfsetroundjoin%
\pgfsetlinewidth{0.803000pt}%
\definecolor{currentstroke}{rgb}{0.690196,0.690196,0.690196}%
\pgfsetstrokecolor{currentstroke}%
\pgfsetdash{}{0pt}%
\pgfpathmoveto{\pgfqpoint{4.955663in}{0.548769in}}%
\pgfpathlineto{\pgfqpoint{4.955663in}{2.301955in}}%
\pgfusepath{stroke}%
\end{pgfscope}%
\begin{pgfscope}%
\pgfsetbuttcap%
\pgfsetroundjoin%
\definecolor{currentfill}{rgb}{0.000000,0.000000,0.000000}%
\pgfsetfillcolor{currentfill}%
\pgfsetlinewidth{0.803000pt}%
\definecolor{currentstroke}{rgb}{0.000000,0.000000,0.000000}%
\pgfsetstrokecolor{currentstroke}%
\pgfsetdash{}{0pt}%
\pgfsys@defobject{currentmarker}{\pgfqpoint{0.000000in}{-0.048611in}}{\pgfqpoint{0.000000in}{0.000000in}}{%
\pgfpathmoveto{\pgfqpoint{0.000000in}{0.000000in}}%
\pgfpathlineto{\pgfqpoint{0.000000in}{-0.048611in}}%
\pgfusepath{stroke,fill}%
}%
\begin{pgfscope}%
\pgfsys@transformshift{4.955663in}{0.548769in}%
\pgfsys@useobject{currentmarker}{}%
\end{pgfscope}%
\end{pgfscope}%
\begin{pgfscope}%
\definecolor{textcolor}{rgb}{0.000000,0.000000,0.000000}%
\pgfsetstrokecolor{textcolor}%
\pgfsetfillcolor{textcolor}%
\pgftext[x=4.955663in,y=0.451547in,,top]{\color{textcolor}\rmfamily\fontsize{10.000000}{12.000000}\selectfont \(\displaystyle {1.0}\)}%
\end{pgfscope}%
\begin{pgfscope}%
\definecolor{textcolor}{rgb}{0.000000,0.000000,0.000000}%
\pgfsetstrokecolor{textcolor}%
\pgfsetfillcolor{textcolor}%
\pgftext[x=2.983977in,y=0.272534in,,top]{\color{textcolor}\rmfamily\fontsize{10.000000}{12.000000}\selectfont \(\displaystyle w\)}%
\end{pgfscope}%
\begin{pgfscope}%
\pgfpathrectangle{\pgfqpoint{0.617954in}{0.548769in}}{\pgfqpoint{4.732046in}{1.753186in}}%
\pgfusepath{clip}%
\pgfsetrectcap%
\pgfsetroundjoin%
\pgfsetlinewidth{0.803000pt}%
\definecolor{currentstroke}{rgb}{0.690196,0.690196,0.690196}%
\pgfsetstrokecolor{currentstroke}%
\pgfsetdash{}{0pt}%
\pgfpathmoveto{\pgfqpoint{0.617954in}{0.548769in}}%
\pgfpathlineto{\pgfqpoint{5.350000in}{0.548769in}}%
\pgfusepath{stroke}%
\end{pgfscope}%
\begin{pgfscope}%
\pgfsetbuttcap%
\pgfsetroundjoin%
\definecolor{currentfill}{rgb}{0.000000,0.000000,0.000000}%
\pgfsetfillcolor{currentfill}%
\pgfsetlinewidth{0.803000pt}%
\definecolor{currentstroke}{rgb}{0.000000,0.000000,0.000000}%
\pgfsetstrokecolor{currentstroke}%
\pgfsetdash{}{0pt}%
\pgfsys@defobject{currentmarker}{\pgfqpoint{-0.048611in}{0.000000in}}{\pgfqpoint{-0.000000in}{0.000000in}}{%
\pgfpathmoveto{\pgfqpoint{-0.000000in}{0.000000in}}%
\pgfpathlineto{\pgfqpoint{-0.048611in}{0.000000in}}%
\pgfusepath{stroke,fill}%
}%
\begin{pgfscope}%
\pgfsys@transformshift{0.617954in}{0.548769in}%
\pgfsys@useobject{currentmarker}{}%
\end{pgfscope}%
\end{pgfscope}%
\begin{pgfscope}%
\definecolor{textcolor}{rgb}{0.000000,0.000000,0.000000}%
\pgfsetstrokecolor{textcolor}%
\pgfsetfillcolor{textcolor}%
\pgftext[x=0.343262in, y=0.500544in, left, base]{\color{textcolor}\rmfamily\fontsize{10.000000}{12.000000}\selectfont \(\displaystyle {\ensuremath{-}2}\)}%
\end{pgfscope}%
\begin{pgfscope}%
\pgfpathrectangle{\pgfqpoint{0.617954in}{0.548769in}}{\pgfqpoint{4.732046in}{1.753186in}}%
\pgfusepath{clip}%
\pgfsetrectcap%
\pgfsetroundjoin%
\pgfsetlinewidth{0.803000pt}%
\definecolor{currentstroke}{rgb}{0.690196,0.690196,0.690196}%
\pgfsetstrokecolor{currentstroke}%
\pgfsetdash{}{0pt}%
\pgfpathmoveto{\pgfqpoint{0.617954in}{0.987065in}}%
\pgfpathlineto{\pgfqpoint{5.350000in}{0.987065in}}%
\pgfusepath{stroke}%
\end{pgfscope}%
\begin{pgfscope}%
\pgfsetbuttcap%
\pgfsetroundjoin%
\definecolor{currentfill}{rgb}{0.000000,0.000000,0.000000}%
\pgfsetfillcolor{currentfill}%
\pgfsetlinewidth{0.803000pt}%
\definecolor{currentstroke}{rgb}{0.000000,0.000000,0.000000}%
\pgfsetstrokecolor{currentstroke}%
\pgfsetdash{}{0pt}%
\pgfsys@defobject{currentmarker}{\pgfqpoint{-0.048611in}{0.000000in}}{\pgfqpoint{-0.000000in}{0.000000in}}{%
\pgfpathmoveto{\pgfqpoint{-0.000000in}{0.000000in}}%
\pgfpathlineto{\pgfqpoint{-0.048611in}{0.000000in}}%
\pgfusepath{stroke,fill}%
}%
\begin{pgfscope}%
\pgfsys@transformshift{0.617954in}{0.987065in}%
\pgfsys@useobject{currentmarker}{}%
\end{pgfscope}%
\end{pgfscope}%
\begin{pgfscope}%
\definecolor{textcolor}{rgb}{0.000000,0.000000,0.000000}%
\pgfsetstrokecolor{textcolor}%
\pgfsetfillcolor{textcolor}%
\pgftext[x=0.343262in, y=0.938840in, left, base]{\color{textcolor}\rmfamily\fontsize{10.000000}{12.000000}\selectfont \(\displaystyle {\ensuremath{-}1}\)}%
\end{pgfscope}%
\begin{pgfscope}%
\pgfpathrectangle{\pgfqpoint{0.617954in}{0.548769in}}{\pgfqpoint{4.732046in}{1.753186in}}%
\pgfusepath{clip}%
\pgfsetrectcap%
\pgfsetroundjoin%
\pgfsetlinewidth{0.803000pt}%
\definecolor{currentstroke}{rgb}{0.690196,0.690196,0.690196}%
\pgfsetstrokecolor{currentstroke}%
\pgfsetdash{}{0pt}%
\pgfpathmoveto{\pgfqpoint{0.617954in}{1.425362in}}%
\pgfpathlineto{\pgfqpoint{5.350000in}{1.425362in}}%
\pgfusepath{stroke}%
\end{pgfscope}%
\begin{pgfscope}%
\pgfsetbuttcap%
\pgfsetroundjoin%
\definecolor{currentfill}{rgb}{0.000000,0.000000,0.000000}%
\pgfsetfillcolor{currentfill}%
\pgfsetlinewidth{0.803000pt}%
\definecolor{currentstroke}{rgb}{0.000000,0.000000,0.000000}%
\pgfsetstrokecolor{currentstroke}%
\pgfsetdash{}{0pt}%
\pgfsys@defobject{currentmarker}{\pgfqpoint{-0.048611in}{0.000000in}}{\pgfqpoint{-0.000000in}{0.000000in}}{%
\pgfpathmoveto{\pgfqpoint{-0.000000in}{0.000000in}}%
\pgfpathlineto{\pgfqpoint{-0.048611in}{0.000000in}}%
\pgfusepath{stroke,fill}%
}%
\begin{pgfscope}%
\pgfsys@transformshift{0.617954in}{1.425362in}%
\pgfsys@useobject{currentmarker}{}%
\end{pgfscope}%
\end{pgfscope}%
\begin{pgfscope}%
\definecolor{textcolor}{rgb}{0.000000,0.000000,0.000000}%
\pgfsetstrokecolor{textcolor}%
\pgfsetfillcolor{textcolor}%
\pgftext[x=0.451287in, y=1.377137in, left, base]{\color{textcolor}\rmfamily\fontsize{10.000000}{12.000000}\selectfont \(\displaystyle {0}\)}%
\end{pgfscope}%
\begin{pgfscope}%
\pgfpathrectangle{\pgfqpoint{0.617954in}{0.548769in}}{\pgfqpoint{4.732046in}{1.753186in}}%
\pgfusepath{clip}%
\pgfsetrectcap%
\pgfsetroundjoin%
\pgfsetlinewidth{0.803000pt}%
\definecolor{currentstroke}{rgb}{0.690196,0.690196,0.690196}%
\pgfsetstrokecolor{currentstroke}%
\pgfsetdash{}{0pt}%
\pgfpathmoveto{\pgfqpoint{0.617954in}{1.863658in}}%
\pgfpathlineto{\pgfqpoint{5.350000in}{1.863658in}}%
\pgfusepath{stroke}%
\end{pgfscope}%
\begin{pgfscope}%
\pgfsetbuttcap%
\pgfsetroundjoin%
\definecolor{currentfill}{rgb}{0.000000,0.000000,0.000000}%
\pgfsetfillcolor{currentfill}%
\pgfsetlinewidth{0.803000pt}%
\definecolor{currentstroke}{rgb}{0.000000,0.000000,0.000000}%
\pgfsetstrokecolor{currentstroke}%
\pgfsetdash{}{0pt}%
\pgfsys@defobject{currentmarker}{\pgfqpoint{-0.048611in}{0.000000in}}{\pgfqpoint{-0.000000in}{0.000000in}}{%
\pgfpathmoveto{\pgfqpoint{-0.000000in}{0.000000in}}%
\pgfpathlineto{\pgfqpoint{-0.048611in}{0.000000in}}%
\pgfusepath{stroke,fill}%
}%
\begin{pgfscope}%
\pgfsys@transformshift{0.617954in}{1.863658in}%
\pgfsys@useobject{currentmarker}{}%
\end{pgfscope}%
\end{pgfscope}%
\begin{pgfscope}%
\definecolor{textcolor}{rgb}{0.000000,0.000000,0.000000}%
\pgfsetstrokecolor{textcolor}%
\pgfsetfillcolor{textcolor}%
\pgftext[x=0.451287in, y=1.815433in, left, base]{\color{textcolor}\rmfamily\fontsize{10.000000}{12.000000}\selectfont \(\displaystyle {1}\)}%
\end{pgfscope}%
\begin{pgfscope}%
\pgfpathrectangle{\pgfqpoint{0.617954in}{0.548769in}}{\pgfqpoint{4.732046in}{1.753186in}}%
\pgfusepath{clip}%
\pgfsetrectcap%
\pgfsetroundjoin%
\pgfsetlinewidth{0.803000pt}%
\definecolor{currentstroke}{rgb}{0.690196,0.690196,0.690196}%
\pgfsetstrokecolor{currentstroke}%
\pgfsetdash{}{0pt}%
\pgfpathmoveto{\pgfqpoint{0.617954in}{2.301955in}}%
\pgfpathlineto{\pgfqpoint{5.350000in}{2.301955in}}%
\pgfusepath{stroke}%
\end{pgfscope}%
\begin{pgfscope}%
\pgfsetbuttcap%
\pgfsetroundjoin%
\definecolor{currentfill}{rgb}{0.000000,0.000000,0.000000}%
\pgfsetfillcolor{currentfill}%
\pgfsetlinewidth{0.803000pt}%
\definecolor{currentstroke}{rgb}{0.000000,0.000000,0.000000}%
\pgfsetstrokecolor{currentstroke}%
\pgfsetdash{}{0pt}%
\pgfsys@defobject{currentmarker}{\pgfqpoint{-0.048611in}{0.000000in}}{\pgfqpoint{-0.000000in}{0.000000in}}{%
\pgfpathmoveto{\pgfqpoint{-0.000000in}{0.000000in}}%
\pgfpathlineto{\pgfqpoint{-0.048611in}{0.000000in}}%
\pgfusepath{stroke,fill}%
}%
\begin{pgfscope}%
\pgfsys@transformshift{0.617954in}{2.301955in}%
\pgfsys@useobject{currentmarker}{}%
\end{pgfscope}%
\end{pgfscope}%
\begin{pgfscope}%
\definecolor{textcolor}{rgb}{0.000000,0.000000,0.000000}%
\pgfsetstrokecolor{textcolor}%
\pgfsetfillcolor{textcolor}%
\pgftext[x=0.451287in, y=2.253730in, left, base]{\color{textcolor}\rmfamily\fontsize{10.000000}{12.000000}\selectfont \(\displaystyle {2}\)}%
\end{pgfscope}%
\begin{pgfscope}%
\definecolor{textcolor}{rgb}{0.000000,0.000000,0.000000}%
\pgfsetstrokecolor{textcolor}%
\pgfsetfillcolor{textcolor}%
\pgftext[x=0.287707in,y=1.425362in,,bottom,rotate=90.000000]{\color{textcolor}\rmfamily\fontsize{10.000000}{12.000000}\selectfont \(\displaystyle T_N(w)\)}%
\end{pgfscope}%
\begin{pgfscope}%
\pgfpathrectangle{\pgfqpoint{0.617954in}{0.548769in}}{\pgfqpoint{4.732046in}{1.753186in}}%
\pgfusepath{clip}%
\pgfsetrectcap%
\pgfsetroundjoin%
\pgfsetlinewidth{1.505625pt}%
\definecolor{currentstroke}{rgb}{0.121569,0.466667,0.705882}%
\pgfsetstrokecolor{currentstroke}%
\pgfsetdash{}{0pt}%
\pgfpathmoveto{\pgfqpoint{0.815123in}{0.538250in}}%
\pgfpathlineto{\pgfqpoint{0.867228in}{0.667673in}}%
\pgfpathlineto{\pgfqpoint{0.919332in}{0.789210in}}%
\pgfpathlineto{\pgfqpoint{0.971437in}{0.903055in}}%
\pgfpathlineto{\pgfqpoint{1.023541in}{1.009402in}}%
\pgfpathlineto{\pgfqpoint{1.075646in}{1.108444in}}%
\pgfpathlineto{\pgfqpoint{1.123409in}{1.192982in}}%
\pgfpathlineto{\pgfqpoint{1.171171in}{1.271695in}}%
\pgfpathlineto{\pgfqpoint{1.218934in}{1.344733in}}%
\pgfpathlineto{\pgfqpoint{1.266696in}{1.412244in}}%
\pgfpathlineto{\pgfqpoint{1.314459in}{1.474380in}}%
\pgfpathlineto{\pgfqpoint{1.362221in}{1.531289in}}%
\pgfpathlineto{\pgfqpoint{1.409984in}{1.583121in}}%
\pgfpathlineto{\pgfqpoint{1.453404in}{1.625960in}}%
\pgfpathlineto{\pgfqpoint{1.496825in}{1.664840in}}%
\pgfpathlineto{\pgfqpoint{1.540245in}{1.699871in}}%
\pgfpathlineto{\pgfqpoint{1.583666in}{1.731168in}}%
\pgfpathlineto{\pgfqpoint{1.627086in}{1.758841in}}%
\pgfpathlineto{\pgfqpoint{1.670507in}{1.783003in}}%
\pgfpathlineto{\pgfqpoint{1.713927in}{1.803767in}}%
\pgfpathlineto{\pgfqpoint{1.757348in}{1.821245in}}%
\pgfpathlineto{\pgfqpoint{1.800768in}{1.835549in}}%
\pgfpathlineto{\pgfqpoint{1.844189in}{1.846792in}}%
\pgfpathlineto{\pgfqpoint{1.887609in}{1.855086in}}%
\pgfpathlineto{\pgfqpoint{1.935372in}{1.860937in}}%
\pgfpathlineto{\pgfqpoint{1.983135in}{1.863505in}}%
\pgfpathlineto{\pgfqpoint{2.030897in}{1.862940in}}%
\pgfpathlineto{\pgfqpoint{2.078660in}{1.859391in}}%
\pgfpathlineto{\pgfqpoint{2.130764in}{1.852293in}}%
\pgfpathlineto{\pgfqpoint{2.182869in}{1.842015in}}%
\pgfpathlineto{\pgfqpoint{2.239316in}{1.827518in}}%
\pgfpathlineto{\pgfqpoint{2.295762in}{1.809766in}}%
\pgfpathlineto{\pgfqpoint{2.356551in}{1.787290in}}%
\pgfpathlineto{\pgfqpoint{2.421682in}{1.759685in}}%
\pgfpathlineto{\pgfqpoint{2.491154in}{1.726641in}}%
\pgfpathlineto{\pgfqpoint{2.564969in}{1.687966in}}%
\pgfpathlineto{\pgfqpoint{2.647468in}{1.641059in}}%
\pgfpathlineto{\pgfqpoint{2.738651in}{1.585589in}}%
\pgfpathlineto{\pgfqpoint{2.851545in}{1.513148in}}%
\pgfpathlineto{\pgfqpoint{3.064305in}{1.371911in}}%
\pgfpathlineto{\pgfqpoint{3.207593in}{1.278793in}}%
\pgfpathlineto{\pgfqpoint{3.307460in}{1.217378in}}%
\pgfpathlineto{\pgfqpoint{3.394301in}{1.167524in}}%
\pgfpathlineto{\pgfqpoint{3.472458in}{1.126261in}}%
\pgfpathlineto{\pgfqpoint{3.541931in}{1.093000in}}%
\pgfpathlineto{\pgfqpoint{3.607061in}{1.065165in}}%
\pgfpathlineto{\pgfqpoint{3.667850in}{1.042451in}}%
\pgfpathlineto{\pgfqpoint{3.724297in}{1.024458in}}%
\pgfpathlineto{\pgfqpoint{3.780743in}{1.009703in}}%
\pgfpathlineto{\pgfqpoint{3.832848in}{0.999169in}}%
\pgfpathlineto{\pgfqpoint{3.884953in}{0.991798in}}%
\pgfpathlineto{\pgfqpoint{3.932715in}{0.987985in}}%
\pgfpathlineto{\pgfqpoint{3.980478in}{0.987142in}}%
\pgfpathlineto{\pgfqpoint{4.028240in}{0.989420in}}%
\pgfpathlineto{\pgfqpoint{4.076003in}{0.994966in}}%
\pgfpathlineto{\pgfqpoint{4.119423in}{1.002971in}}%
\pgfpathlineto{\pgfqpoint{4.162844in}{1.013914in}}%
\pgfpathlineto{\pgfqpoint{4.206264in}{1.027907in}}%
\pgfpathlineto{\pgfqpoint{4.249685in}{1.045063in}}%
\pgfpathlineto{\pgfqpoint{4.293105in}{1.065493in}}%
\pgfpathlineto{\pgfqpoint{4.336526in}{1.089310in}}%
\pgfpathlineto{\pgfqpoint{4.379946in}{1.116627in}}%
\pgfpathlineto{\pgfqpoint{4.423367in}{1.147556in}}%
\pgfpathlineto{\pgfqpoint{4.466787in}{1.182209in}}%
\pgfpathlineto{\pgfqpoint{4.510208in}{1.220699in}}%
\pgfpathlineto{\pgfqpoint{4.553628in}{1.263138in}}%
\pgfpathlineto{\pgfqpoint{4.597049in}{1.309638in}}%
\pgfpathlineto{\pgfqpoint{4.644812in}{1.365612in}}%
\pgfpathlineto{\pgfqpoint{4.692574in}{1.426786in}}%
\pgfpathlineto{\pgfqpoint{4.740337in}{1.493310in}}%
\pgfpathlineto{\pgfqpoint{4.788099in}{1.565332in}}%
\pgfpathlineto{\pgfqpoint{4.835862in}{1.643002in}}%
\pgfpathlineto{\pgfqpoint{4.883624in}{1.726469in}}%
\pgfpathlineto{\pgfqpoint{4.931387in}{1.815884in}}%
\pgfpathlineto{\pgfqpoint{4.983491in}{1.920387in}}%
\pgfpathlineto{\pgfqpoint{5.035596in}{2.032338in}}%
\pgfpathlineto{\pgfqpoint{5.087701in}{2.151934in}}%
\pgfpathlineto{\pgfqpoint{5.139805in}{2.279368in}}%
\pgfpathlineto{\pgfqpoint{5.152831in}{2.312474in}}%
\pgfpathlineto{\pgfqpoint{5.152831in}{2.312474in}}%
\pgfusepath{stroke}%
\end{pgfscope}%
\begin{pgfscope}%
\pgfpathrectangle{\pgfqpoint{0.617954in}{0.548769in}}{\pgfqpoint{4.732046in}{1.753186in}}%
\pgfusepath{clip}%
\pgfsetrectcap%
\pgfsetroundjoin%
\pgfsetlinewidth{1.505625pt}%
\definecolor{currentstroke}{rgb}{1.000000,0.498039,0.054902}%
\pgfsetstrokecolor{currentstroke}%
\pgfsetdash{}{0pt}%
\pgfpathmoveto{\pgfqpoint{0.963285in}{2.315844in}}%
\pgfpathlineto{\pgfqpoint{0.988805in}{2.065008in}}%
\pgfpathlineto{\pgfqpoint{1.014857in}{1.843281in}}%
\pgfpathlineto{\pgfqpoint{1.036568in}{1.682977in}}%
\pgfpathlineto{\pgfqpoint{1.058278in}{1.543280in}}%
\pgfpathlineto{\pgfqpoint{1.079988in}{1.422723in}}%
\pgfpathlineto{\pgfqpoint{1.101698in}{1.319903in}}%
\pgfpathlineto{\pgfqpoint{1.123409in}{1.233476in}}%
\pgfpathlineto{\pgfqpoint{1.140777in}{1.175276in}}%
\pgfpathlineto{\pgfqpoint{1.158145in}{1.126117in}}%
\pgfpathlineto{\pgfqpoint{1.175513in}{1.085395in}}%
\pgfpathlineto{\pgfqpoint{1.192881in}{1.052526in}}%
\pgfpathlineto{\pgfqpoint{1.210250in}{1.026952in}}%
\pgfpathlineto{\pgfqpoint{1.223276in}{1.012233in}}%
\pgfpathlineto{\pgfqpoint{1.236302in}{1.001095in}}%
\pgfpathlineto{\pgfqpoint{1.249328in}{0.993325in}}%
\pgfpathlineto{\pgfqpoint{1.262354in}{0.988718in}}%
\pgfpathlineto{\pgfqpoint{1.275380in}{0.987075in}}%
\pgfpathlineto{\pgfqpoint{1.288407in}{0.988202in}}%
\pgfpathlineto{\pgfqpoint{1.301433in}{0.991914in}}%
\pgfpathlineto{\pgfqpoint{1.318801in}{1.000572in}}%
\pgfpathlineto{\pgfqpoint{1.336169in}{1.013093in}}%
\pgfpathlineto{\pgfqpoint{1.353537in}{1.029086in}}%
\pgfpathlineto{\pgfqpoint{1.375248in}{1.053389in}}%
\pgfpathlineto{\pgfqpoint{1.401300in}{1.087971in}}%
\pgfpathlineto{\pgfqpoint{1.431694in}{1.134386in}}%
\pgfpathlineto{\pgfqpoint{1.466431in}{1.193454in}}%
\pgfpathlineto{\pgfqpoint{1.514193in}{1.281335in}}%
\pgfpathlineto{\pgfqpoint{1.648797in}{1.533319in}}%
\pgfpathlineto{\pgfqpoint{1.692217in}{1.606504in}}%
\pgfpathlineto{\pgfqpoint{1.731296in}{1.666194in}}%
\pgfpathlineto{\pgfqpoint{1.766032in}{1.713499in}}%
\pgfpathlineto{\pgfqpoint{1.796426in}{1.750004in}}%
\pgfpathlineto{\pgfqpoint{1.826821in}{1.781659in}}%
\pgfpathlineto{\pgfqpoint{1.852873in}{1.804782in}}%
\pgfpathlineto{\pgfqpoint{1.878925in}{1.824113in}}%
\pgfpathlineto{\pgfqpoint{1.904978in}{1.839604in}}%
\pgfpathlineto{\pgfqpoint{1.931030in}{1.851242in}}%
\pgfpathlineto{\pgfqpoint{1.957082in}{1.859041in}}%
\pgfpathlineto{\pgfqpoint{1.983135in}{1.863047in}}%
\pgfpathlineto{\pgfqpoint{2.009187in}{1.863329in}}%
\pgfpathlineto{\pgfqpoint{2.035239in}{1.859984in}}%
\pgfpathlineto{\pgfqpoint{2.061291in}{1.853128in}}%
\pgfpathlineto{\pgfqpoint{2.087344in}{1.842898in}}%
\pgfpathlineto{\pgfqpoint{2.113396in}{1.829452in}}%
\pgfpathlineto{\pgfqpoint{2.143790in}{1.809929in}}%
\pgfpathlineto{\pgfqpoint{2.174185in}{1.786560in}}%
\pgfpathlineto{\pgfqpoint{2.208921in}{1.755558in}}%
\pgfpathlineto{\pgfqpoint{2.243658in}{1.720467in}}%
\pgfpathlineto{\pgfqpoint{2.282736in}{1.676770in}}%
\pgfpathlineto{\pgfqpoint{2.330499in}{1.618420in}}%
\pgfpathlineto{\pgfqpoint{2.386945in}{1.544378in}}%
\pgfpathlineto{\pgfqpoint{2.491154in}{1.401257in}}%
\pgfpathlineto{\pgfqpoint{2.573653in}{1.290423in}}%
\pgfpathlineto{\pgfqpoint{2.630100in}{1.219842in}}%
\pgfpathlineto{\pgfqpoint{2.677863in}{1.165204in}}%
\pgfpathlineto{\pgfqpoint{2.716941in}{1.124786in}}%
\pgfpathlineto{\pgfqpoint{2.756020in}{1.088796in}}%
\pgfpathlineto{\pgfqpoint{2.790756in}{1.060903in}}%
\pgfpathlineto{\pgfqpoint{2.825492in}{1.037164in}}%
\pgfpathlineto{\pgfqpoint{2.855887in}{1.019988in}}%
\pgfpathlineto{\pgfqpoint{2.886281in}{1.006308in}}%
\pgfpathlineto{\pgfqpoint{2.916675in}{0.996229in}}%
\pgfpathlineto{\pgfqpoint{2.947070in}{0.989827in}}%
\pgfpathlineto{\pgfqpoint{2.973122in}{0.987304in}}%
\pgfpathlineto{\pgfqpoint{2.999174in}{0.987534in}}%
\pgfpathlineto{\pgfqpoint{3.025227in}{0.990514in}}%
\pgfpathlineto{\pgfqpoint{3.051279in}{0.996229in}}%
\pgfpathlineto{\pgfqpoint{3.081673in}{1.006308in}}%
\pgfpathlineto{\pgfqpoint{3.112068in}{1.019988in}}%
\pgfpathlineto{\pgfqpoint{3.142462in}{1.037164in}}%
\pgfpathlineto{\pgfqpoint{3.172856in}{1.057704in}}%
\pgfpathlineto{\pgfqpoint{3.207593in}{1.085092in}}%
\pgfpathlineto{\pgfqpoint{3.242329in}{1.116388in}}%
\pgfpathlineto{\pgfqpoint{3.281408in}{1.155866in}}%
\pgfpathlineto{\pgfqpoint{3.324828in}{1.204423in}}%
\pgfpathlineto{\pgfqpoint{3.372591in}{1.262618in}}%
\pgfpathlineto{\pgfqpoint{3.433379in}{1.342111in}}%
\pgfpathlineto{\pgfqpoint{3.533247in}{1.479174in}}%
\pgfpathlineto{\pgfqpoint{3.615746in}{1.590473in}}%
\pgfpathlineto{\pgfqpoint{3.667850in}{1.656113in}}%
\pgfpathlineto{\pgfqpoint{3.711271in}{1.706362in}}%
\pgfpathlineto{\pgfqpoint{3.750349in}{1.747148in}}%
\pgfpathlineto{\pgfqpoint{3.785086in}{1.779222in}}%
\pgfpathlineto{\pgfqpoint{3.815480in}{1.803631in}}%
\pgfpathlineto{\pgfqpoint{3.845874in}{1.824284in}}%
\pgfpathlineto{\pgfqpoint{3.876269in}{1.840876in}}%
\pgfpathlineto{\pgfqpoint{3.902321in}{1.851653in}}%
\pgfpathlineto{\pgfqpoint{3.928373in}{1.859081in}}%
\pgfpathlineto{\pgfqpoint{3.954425in}{1.863021in}}%
\pgfpathlineto{\pgfqpoint{3.980478in}{1.863350in}}%
\pgfpathlineto{\pgfqpoint{4.002188in}{1.860795in}}%
\pgfpathlineto{\pgfqpoint{4.023898in}{1.855619in}}%
\pgfpathlineto{\pgfqpoint{4.049951in}{1.845904in}}%
\pgfpathlineto{\pgfqpoint{4.076003in}{1.832340in}}%
\pgfpathlineto{\pgfqpoint{4.102055in}{1.814925in}}%
\pgfpathlineto{\pgfqpoint{4.128108in}{1.793690in}}%
\pgfpathlineto{\pgfqpoint{4.154160in}{1.768700in}}%
\pgfpathlineto{\pgfqpoint{4.184554in}{1.734938in}}%
\pgfpathlineto{\pgfqpoint{4.214949in}{1.696434in}}%
\pgfpathlineto{\pgfqpoint{4.249685in}{1.647018in}}%
\pgfpathlineto{\pgfqpoint{4.288763in}{1.585242in}}%
\pgfpathlineto{\pgfqpoint{4.332184in}{1.510192in}}%
\pgfpathlineto{\pgfqpoint{4.388631in}{1.405562in}}%
\pgfpathlineto{\pgfqpoint{4.505866in}{1.185791in}}%
\pgfpathlineto{\pgfqpoint{4.544944in}{1.120547in}}%
\pgfpathlineto{\pgfqpoint{4.575339in}{1.075851in}}%
\pgfpathlineto{\pgfqpoint{4.601391in}{1.043132in}}%
\pgfpathlineto{\pgfqpoint{4.623101in}{1.020680in}}%
\pgfpathlineto{\pgfqpoint{4.640469in}{1.006375in}}%
\pgfpathlineto{\pgfqpoint{4.657838in}{0.995735in}}%
\pgfpathlineto{\pgfqpoint{4.675206in}{0.989161in}}%
\pgfpathlineto{\pgfqpoint{4.688232in}{0.987152in}}%
\pgfpathlineto{\pgfqpoint{4.701258in}{0.987850in}}%
\pgfpathlineto{\pgfqpoint{4.714284in}{0.991448in}}%
\pgfpathlineto{\pgfqpoint{4.727310in}{0.998141in}}%
\pgfpathlineto{\pgfqpoint{4.740337in}{1.008133in}}%
\pgfpathlineto{\pgfqpoint{4.753363in}{1.021634in}}%
\pgfpathlineto{\pgfqpoint{4.766389in}{1.038861in}}%
\pgfpathlineto{\pgfqpoint{4.783757in}{1.068014in}}%
\pgfpathlineto{\pgfqpoint{4.801125in}{1.104738in}}%
\pgfpathlineto{\pgfqpoint{4.818494in}{1.149605in}}%
\pgfpathlineto{\pgfqpoint{4.835862in}{1.203207in}}%
\pgfpathlineto{\pgfqpoint{4.853230in}{1.266163in}}%
\pgfpathlineto{\pgfqpoint{4.870598in}{1.339113in}}%
\pgfpathlineto{\pgfqpoint{4.892308in}{1.445372in}}%
\pgfpathlineto{\pgfqpoint{4.914019in}{1.569642in}}%
\pgfpathlineto{\pgfqpoint{4.935729in}{1.713341in}}%
\pgfpathlineto{\pgfqpoint{4.957439in}{1.877948in}}%
\pgfpathlineto{\pgfqpoint{4.979149in}{2.065008in}}%
\pgfpathlineto{\pgfqpoint{5.004669in}{2.315844in}}%
\pgfpathlineto{\pgfqpoint{5.004669in}{2.315844in}}%
\pgfusepath{stroke}%
\end{pgfscope}%
\begin{pgfscope}%
\pgfpathrectangle{\pgfqpoint{0.617954in}{0.548769in}}{\pgfqpoint{4.732046in}{1.753186in}}%
\pgfusepath{clip}%
\pgfsetrectcap%
\pgfsetroundjoin%
\pgfsetlinewidth{1.505625pt}%
\definecolor{currentstroke}{rgb}{0.172549,0.627451,0.172549}%
\pgfsetstrokecolor{currentstroke}%
\pgfsetdash{}{0pt}%
\pgfpathmoveto{\pgfqpoint{0.997762in}{0.534880in}}%
\pgfpathlineto{\pgfqpoint{1.010515in}{0.938420in}}%
\pgfpathlineto{\pgfqpoint{1.023541in}{1.256633in}}%
\pgfpathlineto{\pgfqpoint{1.036568in}{1.493946in}}%
\pgfpathlineto{\pgfqpoint{1.049594in}{1.662731in}}%
\pgfpathlineto{\pgfqpoint{1.058278in}{1.742696in}}%
\pgfpathlineto{\pgfqpoint{1.066962in}{1.800120in}}%
\pgfpathlineto{\pgfqpoint{1.075646in}{1.837754in}}%
\pgfpathlineto{\pgfqpoint{1.084330in}{1.858134in}}%
\pgfpathlineto{\pgfqpoint{1.088672in}{1.862591in}}%
\pgfpathlineto{\pgfqpoint{1.093014in}{1.863596in}}%
\pgfpathlineto{\pgfqpoint{1.097356in}{1.861410in}}%
\pgfpathlineto{\pgfqpoint{1.101698in}{1.856284in}}%
\pgfpathlineto{\pgfqpoint{1.110382in}{1.838165in}}%
\pgfpathlineto{\pgfqpoint{1.119067in}{1.811034in}}%
\pgfpathlineto{\pgfqpoint{1.132093in}{1.756980in}}%
\pgfpathlineto{\pgfqpoint{1.149461in}{1.667415in}}%
\pgfpathlineto{\pgfqpoint{1.179855in}{1.488530in}}%
\pgfpathlineto{\pgfqpoint{1.214592in}{1.289910in}}%
\pgfpathlineto{\pgfqpoint{1.236302in}{1.184400in}}%
\pgfpathlineto{\pgfqpoint{1.253670in}{1.114719in}}%
\pgfpathlineto{\pgfqpoint{1.266696in}{1.072074in}}%
\pgfpathlineto{\pgfqpoint{1.279722in}{1.038054in}}%
\pgfpathlineto{\pgfqpoint{1.292749in}{1.012779in}}%
\pgfpathlineto{\pgfqpoint{1.301433in}{1.000762in}}%
\pgfpathlineto{\pgfqpoint{1.310117in}{0.992549in}}%
\pgfpathlineto{\pgfqpoint{1.318801in}{0.988057in}}%
\pgfpathlineto{\pgfqpoint{1.327485in}{0.987177in}}%
\pgfpathlineto{\pgfqpoint{1.336169in}{0.989782in}}%
\pgfpathlineto{\pgfqpoint{1.344853in}{0.995723in}}%
\pgfpathlineto{\pgfqpoint{1.353537in}{1.004838in}}%
\pgfpathlineto{\pgfqpoint{1.366563in}{1.024074in}}%
\pgfpathlineto{\pgfqpoint{1.379590in}{1.049416in}}%
\pgfpathlineto{\pgfqpoint{1.396958in}{1.091524in}}%
\pgfpathlineto{\pgfqpoint{1.418668in}{1.155158in}}%
\pgfpathlineto{\pgfqpoint{1.444720in}{1.243239in}}%
\pgfpathlineto{\pgfqpoint{1.488141in}{1.403898in}}%
\pgfpathlineto{\pgfqpoint{1.531561in}{1.561511in}}%
\pgfpathlineto{\pgfqpoint{1.557614in}{1.646386in}}%
\pgfpathlineto{\pgfqpoint{1.579324in}{1.708574in}}%
\pgfpathlineto{\pgfqpoint{1.601034in}{1.761436in}}%
\pgfpathlineto{\pgfqpoint{1.618402in}{1.796276in}}%
\pgfpathlineto{\pgfqpoint{1.635771in}{1.824067in}}%
\pgfpathlineto{\pgfqpoint{1.648797in}{1.840132in}}%
\pgfpathlineto{\pgfqpoint{1.661823in}{1.852039in}}%
\pgfpathlineto{\pgfqpoint{1.674849in}{1.859776in}}%
\pgfpathlineto{\pgfqpoint{1.687875in}{1.863368in}}%
\pgfpathlineto{\pgfqpoint{1.700901in}{1.862878in}}%
\pgfpathlineto{\pgfqpoint{1.713927in}{1.858401in}}%
\pgfpathlineto{\pgfqpoint{1.726954in}{1.850065in}}%
\pgfpathlineto{\pgfqpoint{1.739980in}{1.838024in}}%
\pgfpathlineto{\pgfqpoint{1.757348in}{1.816524in}}%
\pgfpathlineto{\pgfqpoint{1.774716in}{1.789256in}}%
\pgfpathlineto{\pgfqpoint{1.796426in}{1.747911in}}%
\pgfpathlineto{\pgfqpoint{1.818137in}{1.699616in}}%
\pgfpathlineto{\pgfqpoint{1.844189in}{1.634299in}}%
\pgfpathlineto{\pgfqpoint{1.878925in}{1.538536in}}%
\pgfpathlineto{\pgfqpoint{1.983135in}{1.243956in}}%
\pgfpathlineto{\pgfqpoint{2.013529in}{1.169800in}}%
\pgfpathlineto{\pgfqpoint{2.039581in}{1.114392in}}%
\pgfpathlineto{\pgfqpoint{2.061291in}{1.074988in}}%
\pgfpathlineto{\pgfqpoint{2.083002in}{1.042391in}}%
\pgfpathlineto{\pgfqpoint{2.100370in}{1.021533in}}%
\pgfpathlineto{\pgfqpoint{2.117738in}{1.005506in}}%
\pgfpathlineto{\pgfqpoint{2.135106in}{0.994424in}}%
\pgfpathlineto{\pgfqpoint{2.148132in}{0.989394in}}%
\pgfpathlineto{\pgfqpoint{2.161159in}{0.987181in}}%
\pgfpathlineto{\pgfqpoint{2.174185in}{0.987773in}}%
\pgfpathlineto{\pgfqpoint{2.187211in}{0.991136in}}%
\pgfpathlineto{\pgfqpoint{2.200237in}{0.997224in}}%
\pgfpathlineto{\pgfqpoint{2.217605in}{1.009467in}}%
\pgfpathlineto{\pgfqpoint{2.234974in}{1.026248in}}%
\pgfpathlineto{\pgfqpoint{2.252342in}{1.047331in}}%
\pgfpathlineto{\pgfqpoint{2.274052in}{1.079308in}}%
\pgfpathlineto{\pgfqpoint{2.295762in}{1.116953in}}%
\pgfpathlineto{\pgfqpoint{2.321815in}{1.168623in}}%
\pgfpathlineto{\pgfqpoint{2.352209in}{1.236174in}}%
\pgfpathlineto{\pgfqpoint{2.391287in}{1.331016in}}%
\pgfpathlineto{\pgfqpoint{2.504181in}{1.611124in}}%
\pgfpathlineto{\pgfqpoint{2.534575in}{1.677221in}}%
\pgfpathlineto{\pgfqpoint{2.560627in}{1.727693in}}%
\pgfpathlineto{\pgfqpoint{2.586680in}{1.771344in}}%
\pgfpathlineto{\pgfqpoint{2.608390in}{1.801859in}}%
\pgfpathlineto{\pgfqpoint{2.630100in}{1.826589in}}%
\pgfpathlineto{\pgfqpoint{2.647468in}{1.841982in}}%
\pgfpathlineto{\pgfqpoint{2.664837in}{1.853327in}}%
\pgfpathlineto{\pgfqpoint{2.682205in}{1.860536in}}%
\pgfpathlineto{\pgfqpoint{2.699573in}{1.863558in}}%
\pgfpathlineto{\pgfqpoint{2.712599in}{1.863067in}}%
\pgfpathlineto{\pgfqpoint{2.725625in}{1.860222in}}%
\pgfpathlineto{\pgfqpoint{2.742993in}{1.852807in}}%
\pgfpathlineto{\pgfqpoint{2.760362in}{1.841331in}}%
\pgfpathlineto{\pgfqpoint{2.777730in}{1.825916in}}%
\pgfpathlineto{\pgfqpoint{2.795098in}{1.806716in}}%
\pgfpathlineto{\pgfqpoint{2.816808in}{1.777690in}}%
\pgfpathlineto{\pgfqpoint{2.838519in}{1.743493in}}%
\pgfpathlineto{\pgfqpoint{2.864571in}{1.696359in}}%
\pgfpathlineto{\pgfqpoint{2.894965in}{1.634247in}}%
\pgfpathlineto{\pgfqpoint{2.929702in}{1.556076in}}%
\pgfpathlineto{\pgfqpoint{2.986148in}{1.420053in}}%
\pgfpathlineto{\pgfqpoint{3.051279in}{1.264602in}}%
\pgfpathlineto{\pgfqpoint{3.086015in}{1.189019in}}%
\pgfpathlineto{\pgfqpoint{3.116410in}{1.130020in}}%
\pgfpathlineto{\pgfqpoint{3.142462in}{1.086119in}}%
\pgfpathlineto{\pgfqpoint{3.164172in}{1.054967in}}%
\pgfpathlineto{\pgfqpoint{3.185883in}{1.029260in}}%
\pgfpathlineto{\pgfqpoint{3.203251in}{1.012882in}}%
\pgfpathlineto{\pgfqpoint{3.220619in}{1.000409in}}%
\pgfpathlineto{\pgfqpoint{3.237987in}{0.991969in}}%
\pgfpathlineto{\pgfqpoint{3.255355in}{0.987657in}}%
\pgfpathlineto{\pgfqpoint{3.268382in}{0.987166in}}%
\pgfpathlineto{\pgfqpoint{3.281408in}{0.989038in}}%
\pgfpathlineto{\pgfqpoint{3.298776in}{0.995204in}}%
\pgfpathlineto{\pgfqpoint{3.316144in}{1.005522in}}%
\pgfpathlineto{\pgfqpoint{3.333512in}{1.019913in}}%
\pgfpathlineto{\pgfqpoint{3.350880in}{1.038258in}}%
\pgfpathlineto{\pgfqpoint{3.372591in}{1.066505in}}%
\pgfpathlineto{\pgfqpoint{3.394301in}{1.100293in}}%
\pgfpathlineto{\pgfqpoint{3.420353in}{1.147476in}}%
\pgfpathlineto{\pgfqpoint{3.446406in}{1.200975in}}%
\pgfpathlineto{\pgfqpoint{3.481142in}{1.280166in}}%
\pgfpathlineto{\pgfqpoint{3.528905in}{1.398487in}}%
\pgfpathlineto{\pgfqpoint{3.611404in}{1.604382in}}%
\pgfpathlineto{\pgfqpoint{3.646140in}{1.682101in}}%
\pgfpathlineto{\pgfqpoint{3.672192in}{1.733770in}}%
\pgfpathlineto{\pgfqpoint{3.698245in}{1.778285in}}%
\pgfpathlineto{\pgfqpoint{3.719955in}{1.809052in}}%
\pgfpathlineto{\pgfqpoint{3.737323in}{1.829084in}}%
\pgfpathlineto{\pgfqpoint{3.754691in}{1.844751in}}%
\pgfpathlineto{\pgfqpoint{3.772059in}{1.855828in}}%
\pgfpathlineto{\pgfqpoint{3.785086in}{1.861014in}}%
\pgfpathlineto{\pgfqpoint{3.798112in}{1.863458in}}%
\pgfpathlineto{\pgfqpoint{3.811138in}{1.863117in}}%
\pgfpathlineto{\pgfqpoint{3.824164in}{1.859967in}}%
\pgfpathlineto{\pgfqpoint{3.837190in}{1.853997in}}%
\pgfpathlineto{\pgfqpoint{3.850216in}{1.845218in}}%
\pgfpathlineto{\pgfqpoint{3.867584in}{1.829191in}}%
\pgfpathlineto{\pgfqpoint{3.884953in}{1.808333in}}%
\pgfpathlineto{\pgfqpoint{3.902321in}{1.782817in}}%
\pgfpathlineto{\pgfqpoint{3.924031in}{1.744730in}}%
\pgfpathlineto{\pgfqpoint{3.945741in}{1.700309in}}%
\pgfpathlineto{\pgfqpoint{3.971794in}{1.639665in}}%
\pgfpathlineto{\pgfqpoint{4.002188in}{1.560701in}}%
\pgfpathlineto{\pgfqpoint{4.045609in}{1.437971in}}%
\pgfpathlineto{\pgfqpoint{4.119423in}{1.227942in}}%
\pgfpathlineto{\pgfqpoint{4.149818in}{1.151107in}}%
\pgfpathlineto{\pgfqpoint{4.175870in}{1.093948in}}%
\pgfpathlineto{\pgfqpoint{4.197580in}{1.054139in}}%
\pgfpathlineto{\pgfqpoint{4.214949in}{1.028264in}}%
\pgfpathlineto{\pgfqpoint{4.232317in}{1.008285in}}%
\pgfpathlineto{\pgfqpoint{4.245343in}{0.997460in}}%
\pgfpathlineto{\pgfqpoint{4.258369in}{0.990394in}}%
\pgfpathlineto{\pgfqpoint{4.271395in}{0.987234in}}%
\pgfpathlineto{\pgfqpoint{4.284421in}{0.988096in}}%
\pgfpathlineto{\pgfqpoint{4.297447in}{0.993064in}}%
\pgfpathlineto{\pgfqpoint{4.310474in}{1.002190in}}%
\pgfpathlineto{\pgfqpoint{4.323500in}{1.015487in}}%
\pgfpathlineto{\pgfqpoint{4.336526in}{1.032928in}}%
\pgfpathlineto{\pgfqpoint{4.353894in}{1.062510in}}%
\pgfpathlineto{\pgfqpoint{4.371262in}{1.099057in}}%
\pgfpathlineto{\pgfqpoint{4.392973in}{1.153882in}}%
\pgfpathlineto{\pgfqpoint{4.414683in}{1.217773in}}%
\pgfpathlineto{\pgfqpoint{4.440735in}{1.304246in}}%
\pgfpathlineto{\pgfqpoint{4.479814in}{1.446826in}}%
\pgfpathlineto{\pgfqpoint{4.536260in}{1.652803in}}%
\pgfpathlineto{\pgfqpoint{4.562313in}{1.735052in}}%
\pgfpathlineto{\pgfqpoint{4.579681in}{1.781356in}}%
\pgfpathlineto{\pgfqpoint{4.597049in}{1.818848in}}%
\pgfpathlineto{\pgfqpoint{4.610075in}{1.840192in}}%
\pgfpathlineto{\pgfqpoint{4.623101in}{1.855001in}}%
\pgfpathlineto{\pgfqpoint{4.631785in}{1.860942in}}%
\pgfpathlineto{\pgfqpoint{4.640469in}{1.863546in}}%
\pgfpathlineto{\pgfqpoint{4.649154in}{1.862667in}}%
\pgfpathlineto{\pgfqpoint{4.657838in}{1.858175in}}%
\pgfpathlineto{\pgfqpoint{4.666522in}{1.849962in}}%
\pgfpathlineto{\pgfqpoint{4.675206in}{1.837945in}}%
\pgfpathlineto{\pgfqpoint{4.688232in}{1.812669in}}%
\pgfpathlineto{\pgfqpoint{4.701258in}{1.778650in}}%
\pgfpathlineto{\pgfqpoint{4.714284in}{1.736005in}}%
\pgfpathlineto{\pgfqpoint{4.731653in}{1.666324in}}%
\pgfpathlineto{\pgfqpoint{4.749021in}{1.583371in}}%
\pgfpathlineto{\pgfqpoint{4.770731in}{1.464716in}}%
\pgfpathlineto{\pgfqpoint{4.835862in}{1.093743in}}%
\pgfpathlineto{\pgfqpoint{4.848888in}{1.039690in}}%
\pgfpathlineto{\pgfqpoint{4.857572in}{1.012559in}}%
\pgfpathlineto{\pgfqpoint{4.866256in}{0.994439in}}%
\pgfpathlineto{\pgfqpoint{4.870598in}{0.989314in}}%
\pgfpathlineto{\pgfqpoint{4.874940in}{0.987128in}}%
\pgfpathlineto{\pgfqpoint{4.879282in}{0.988132in}}%
\pgfpathlineto{\pgfqpoint{4.883624in}{0.992590in}}%
\pgfpathlineto{\pgfqpoint{4.887966in}{1.000774in}}%
\pgfpathlineto{\pgfqpoint{4.896650in}{1.029477in}}%
\pgfpathlineto{\pgfqpoint{4.905335in}{1.076675in}}%
\pgfpathlineto{\pgfqpoint{4.914019in}{1.145012in}}%
\pgfpathlineto{\pgfqpoint{4.922703in}{1.237348in}}%
\pgfpathlineto{\pgfqpoint{4.931387in}{1.356777in}}%
\pgfpathlineto{\pgfqpoint{4.944413in}{1.594091in}}%
\pgfpathlineto{\pgfqpoint{4.957439in}{1.912304in}}%
\pgfpathlineto{\pgfqpoint{4.970193in}{2.315844in}}%
\pgfpathlineto{\pgfqpoint{4.970193in}{2.315844in}}%
\pgfusepath{stroke}%
\end{pgfscope}%
\begin{pgfscope}%
\pgfsetrectcap%
\pgfsetmiterjoin%
\pgfsetlinewidth{0.803000pt}%
\definecolor{currentstroke}{rgb}{0.000000,0.000000,0.000000}%
\pgfsetstrokecolor{currentstroke}%
\pgfsetdash{}{0pt}%
\pgfpathmoveto{\pgfqpoint{0.617954in}{0.548769in}}%
\pgfpathlineto{\pgfqpoint{0.617954in}{2.301955in}}%
\pgfusepath{stroke}%
\end{pgfscope}%
\begin{pgfscope}%
\pgfsetrectcap%
\pgfsetmiterjoin%
\pgfsetlinewidth{0.803000pt}%
\definecolor{currentstroke}{rgb}{0.000000,0.000000,0.000000}%
\pgfsetstrokecolor{currentstroke}%
\pgfsetdash{}{0pt}%
\pgfpathmoveto{\pgfqpoint{5.350000in}{0.548769in}}%
\pgfpathlineto{\pgfqpoint{5.350000in}{2.301955in}}%
\pgfusepath{stroke}%
\end{pgfscope}%
\begin{pgfscope}%
\pgfsetrectcap%
\pgfsetmiterjoin%
\pgfsetlinewidth{0.803000pt}%
\definecolor{currentstroke}{rgb}{0.000000,0.000000,0.000000}%
\pgfsetstrokecolor{currentstroke}%
\pgfsetdash{}{0pt}%
\pgfpathmoveto{\pgfqpoint{0.617954in}{0.548769in}}%
\pgfpathlineto{\pgfqpoint{5.350000in}{0.548769in}}%
\pgfusepath{stroke}%
\end{pgfscope}%
\begin{pgfscope}%
\pgfsetrectcap%
\pgfsetmiterjoin%
\pgfsetlinewidth{0.803000pt}%
\definecolor{currentstroke}{rgb}{0.000000,0.000000,0.000000}%
\pgfsetstrokecolor{currentstroke}%
\pgfsetdash{}{0pt}%
\pgfpathmoveto{\pgfqpoint{0.617954in}{2.301955in}}%
\pgfpathlineto{\pgfqpoint{5.350000in}{2.301955in}}%
\pgfusepath{stroke}%
\end{pgfscope}%
\begin{pgfscope}%
\pgfsetbuttcap%
\pgfsetmiterjoin%
\definecolor{currentfill}{rgb}{1.000000,1.000000,1.000000}%
\pgfsetfillcolor{currentfill}%
\pgfsetfillopacity{0.800000}%
\pgfsetlinewidth{1.003750pt}%
\definecolor{currentstroke}{rgb}{0.800000,0.800000,0.800000}%
\pgfsetstrokecolor{currentstroke}%
\pgfsetstrokeopacity{0.800000}%
\pgfsetdash{}{0pt}%
\pgfpathmoveto{\pgfqpoint{0.715177in}{1.609825in}}%
\pgfpathlineto{\pgfqpoint{1.610430in}{1.609825in}}%
\pgfpathquadraticcurveto{\pgfqpoint{1.638207in}{1.609825in}}{\pgfqpoint{1.638207in}{1.637603in}}%
\pgfpathlineto{\pgfqpoint{1.638207in}{2.204733in}}%
\pgfpathquadraticcurveto{\pgfqpoint{1.638207in}{2.232510in}}{\pgfqpoint{1.610430in}{2.232510in}}%
\pgfpathlineto{\pgfqpoint{0.715177in}{2.232510in}}%
\pgfpathquadraticcurveto{\pgfqpoint{0.687399in}{2.232510in}}{\pgfqpoint{0.687399in}{2.204733in}}%
\pgfpathlineto{\pgfqpoint{0.687399in}{1.637603in}}%
\pgfpathquadraticcurveto{\pgfqpoint{0.687399in}{1.609825in}}{\pgfqpoint{0.715177in}{1.609825in}}%
\pgfpathlineto{\pgfqpoint{0.715177in}{1.609825in}}%
\pgfpathclose%
\pgfusepath{stroke,fill}%
\end{pgfscope}%
\begin{pgfscope}%
\pgfsetrectcap%
\pgfsetroundjoin%
\pgfsetlinewidth{1.505625pt}%
\definecolor{currentstroke}{rgb}{0.121569,0.466667,0.705882}%
\pgfsetstrokecolor{currentstroke}%
\pgfsetdash{}{0pt}%
\pgfpathmoveto{\pgfqpoint{0.742954in}{2.128344in}}%
\pgfpathlineto{\pgfqpoint{0.881843in}{2.128344in}}%
\pgfpathlineto{\pgfqpoint{1.020732in}{2.128344in}}%
\pgfusepath{stroke}%
\end{pgfscope}%
\begin{pgfscope}%
\definecolor{textcolor}{rgb}{0.000000,0.000000,0.000000}%
\pgfsetstrokecolor{textcolor}%
\pgfsetfillcolor{textcolor}%
\pgftext[x=1.131843in,y=2.079733in,left,base]{\color{textcolor}\rmfamily\fontsize{10.000000}{12.000000}\selectfont \(\displaystyle N=3\)}%
\end{pgfscope}%
\begin{pgfscope}%
\pgfsetrectcap%
\pgfsetroundjoin%
\pgfsetlinewidth{1.505625pt}%
\definecolor{currentstroke}{rgb}{1.000000,0.498039,0.054902}%
\pgfsetstrokecolor{currentstroke}%
\pgfsetdash{}{0pt}%
\pgfpathmoveto{\pgfqpoint{0.742954in}{1.934671in}}%
\pgfpathlineto{\pgfqpoint{0.881843in}{1.934671in}}%
\pgfpathlineto{\pgfqpoint{1.020732in}{1.934671in}}%
\pgfusepath{stroke}%
\end{pgfscope}%
\begin{pgfscope}%
\definecolor{textcolor}{rgb}{0.000000,0.000000,0.000000}%
\pgfsetstrokecolor{textcolor}%
\pgfsetfillcolor{textcolor}%
\pgftext[x=1.131843in,y=1.886060in,left,base]{\color{textcolor}\rmfamily\fontsize{10.000000}{12.000000}\selectfont \(\displaystyle N=6\)}%
\end{pgfscope}%
\begin{pgfscope}%
\pgfsetrectcap%
\pgfsetroundjoin%
\pgfsetlinewidth{1.505625pt}%
\definecolor{currentstroke}{rgb}{0.172549,0.627451,0.172549}%
\pgfsetstrokecolor{currentstroke}%
\pgfsetdash{}{0pt}%
\pgfpathmoveto{\pgfqpoint{0.742954in}{1.740998in}}%
\pgfpathlineto{\pgfqpoint{0.881843in}{1.740998in}}%
\pgfpathlineto{\pgfqpoint{1.020732in}{1.740998in}}%
\pgfusepath{stroke}%
\end{pgfscope}%
\begin{pgfscope}%
\definecolor{textcolor}{rgb}{0.000000,0.000000,0.000000}%
\pgfsetstrokecolor{textcolor}%
\pgfsetfillcolor{textcolor}%
\pgftext[x=1.131843in,y=1.692387in,left,base]{\color{textcolor}\rmfamily\fontsize{10.000000}{12.000000}\selectfont \(\displaystyle N=11\)}%
\end{pgfscope}%
\end{pgfpicture}%
\makeatother%
\endgroup%

    \caption{Die Tschebyscheff-Polynome $C_N$.}
    \label{ellfilter:fig:chebychef_polynomials}
\end{figure}
Da der Kosinus begrenzt zwischen $-1$ und $1$ ist, sind auch die Tschebyscheff-Polynome begrenzt.
Geht man aber über das Intervall $[-1, 1]$ hinaus, divergieren die Funktionen mit zunehmender Ordnung immer steiler gegen $\pm \infty$.
Diese Eigenschaft ist sehr nützlich für ein Filter.
Wenn wir die Tschebyscheff-Polynome quadrieren, passen sie perfekt in die Voraussetzungen für Filterfunktionen, wie es Abbildung \ref{ellfiter:fig:chebychef} demonstriert.
\begin{figure}
    \centering
    %% Creator: Matplotlib, PGF backend
%%
%% To include the figure in your LaTeX document, write
%%   \input{<filename>.pgf}
%%
%% Make sure the required packages are loaded in your preamble
%%   \usepackage{pgf}
%%
%% Also ensure that all the required font packages are loaded; for instance,
%% the lmodern package is sometimes necessary when using math font.
%%   \usepackage{lmodern}
%%
%% Figures using additional raster images can only be included by \input if
%% they are in the same directory as the main LaTeX file. For loading figures
%% from other directories you can use the `import` package
%%   \usepackage{import}
%%
%% and then include the figures with
%%   \import{<path to file>}{<filename>.pgf}
%%
%% Matplotlib used the following preamble
%%
\begingroup%
\makeatletter%
\begin{pgfpicture}%
\pgfpathrectangle{\pgfpointorigin}{\pgfqpoint{4.000000in}{2.500000in}}%
\pgfusepath{use as bounding box, clip}%
\begin{pgfscope}%
\pgfsetbuttcap%
\pgfsetmiterjoin%
\pgfsetlinewidth{0.000000pt}%
\definecolor{currentstroke}{rgb}{1.000000,1.000000,1.000000}%
\pgfsetstrokecolor{currentstroke}%
\pgfsetstrokeopacity{0.000000}%
\pgfsetdash{}{0pt}%
\pgfpathmoveto{\pgfqpoint{0.000000in}{0.000000in}}%
\pgfpathlineto{\pgfqpoint{4.000000in}{0.000000in}}%
\pgfpathlineto{\pgfqpoint{4.000000in}{2.500000in}}%
\pgfpathlineto{\pgfqpoint{0.000000in}{2.500000in}}%
\pgfpathlineto{\pgfqpoint{0.000000in}{0.000000in}}%
\pgfpathclose%
\pgfusepath{}%
\end{pgfscope}%
\begin{pgfscope}%
\pgfsetbuttcap%
\pgfsetmiterjoin%
\definecolor{currentfill}{rgb}{1.000000,1.000000,1.000000}%
\pgfsetfillcolor{currentfill}%
\pgfsetlinewidth{0.000000pt}%
\definecolor{currentstroke}{rgb}{0.000000,0.000000,0.000000}%
\pgfsetstrokecolor{currentstroke}%
\pgfsetstrokeopacity{0.000000}%
\pgfsetdash{}{0pt}%
\pgfpathmoveto{\pgfqpoint{0.630330in}{0.548769in}}%
\pgfpathlineto{\pgfqpoint{3.727004in}{0.548769in}}%
\pgfpathlineto{\pgfqpoint{3.727004in}{2.301955in}}%
\pgfpathlineto{\pgfqpoint{0.630330in}{2.301955in}}%
\pgfpathlineto{\pgfqpoint{0.630330in}{0.548769in}}%
\pgfpathclose%
\pgfusepath{fill}%
\end{pgfscope}%
\begin{pgfscope}%
\pgfpathrectangle{\pgfqpoint{0.630330in}{0.548769in}}{\pgfqpoint{3.096674in}{1.753186in}}%
\pgfusepath{clip}%
\pgfsetbuttcap%
\pgfsetmiterjoin%
\definecolor{currentfill}{rgb}{0.000000,0.501961,0.000000}%
\pgfsetfillcolor{currentfill}%
\pgfsetfillopacity{0.200000}%
\pgfsetlinewidth{0.000000pt}%
\definecolor{currentstroke}{rgb}{0.000000,0.000000,0.000000}%
\pgfsetstrokecolor{currentstroke}%
\pgfsetstrokeopacity{0.200000}%
\pgfsetdash{}{0pt}%
\pgfpathmoveto{\pgfqpoint{0.630330in}{0.548769in}}%
\pgfpathlineto{\pgfqpoint{2.694779in}{0.548769in}}%
\pgfpathlineto{\pgfqpoint{2.694779in}{1.425362in}}%
\pgfpathlineto{\pgfqpoint{0.630330in}{1.425362in}}%
\pgfpathlineto{\pgfqpoint{0.630330in}{0.548769in}}%
\pgfpathclose%
\pgfusepath{fill}%
\end{pgfscope}%
\begin{pgfscope}%
\pgfpathrectangle{\pgfqpoint{0.630330in}{0.548769in}}{\pgfqpoint{3.096674in}{1.753186in}}%
\pgfusepath{clip}%
\pgfsetbuttcap%
\pgfsetmiterjoin%
\definecolor{currentfill}{rgb}{1.000000,0.647059,0.000000}%
\pgfsetfillcolor{currentfill}%
\pgfsetfillopacity{0.200000}%
\pgfsetlinewidth{0.000000pt}%
\definecolor{currentstroke}{rgb}{0.000000,0.000000,0.000000}%
\pgfsetstrokecolor{currentstroke}%
\pgfsetstrokeopacity{0.200000}%
\pgfsetdash{}{0pt}%
\pgfpathmoveto{\pgfqpoint{2.694779in}{1.425362in}}%
\pgfpathlineto{\pgfqpoint{3.727004in}{1.425362in}}%
\pgfpathlineto{\pgfqpoint{3.727004in}{2.301955in}}%
\pgfpathlineto{\pgfqpoint{2.694779in}{2.301955in}}%
\pgfpathlineto{\pgfqpoint{2.694779in}{1.425362in}}%
\pgfpathclose%
\pgfusepath{fill}%
\end{pgfscope}%
\begin{pgfscope}%
\pgfpathrectangle{\pgfqpoint{0.630330in}{0.548769in}}{\pgfqpoint{3.096674in}{1.753186in}}%
\pgfusepath{clip}%
\pgfsetrectcap%
\pgfsetroundjoin%
\pgfsetlinewidth{0.803000pt}%
\definecolor{currentstroke}{rgb}{0.690196,0.690196,0.690196}%
\pgfsetstrokecolor{currentstroke}%
\pgfsetdash{}{0pt}%
\pgfpathmoveto{\pgfqpoint{0.630330in}{0.548769in}}%
\pgfpathlineto{\pgfqpoint{0.630330in}{2.301955in}}%
\pgfusepath{stroke}%
\end{pgfscope}%
\begin{pgfscope}%
\pgfsetbuttcap%
\pgfsetroundjoin%
\definecolor{currentfill}{rgb}{0.000000,0.000000,0.000000}%
\pgfsetfillcolor{currentfill}%
\pgfsetlinewidth{0.803000pt}%
\definecolor{currentstroke}{rgb}{0.000000,0.000000,0.000000}%
\pgfsetstrokecolor{currentstroke}%
\pgfsetdash{}{0pt}%
\pgfsys@defobject{currentmarker}{\pgfqpoint{0.000000in}{-0.048611in}}{\pgfqpoint{0.000000in}{0.000000in}}{%
\pgfpathmoveto{\pgfqpoint{0.000000in}{0.000000in}}%
\pgfpathlineto{\pgfqpoint{0.000000in}{-0.048611in}}%
\pgfusepath{stroke,fill}%
}%
\begin{pgfscope}%
\pgfsys@transformshift{0.630330in}{0.548769in}%
\pgfsys@useobject{currentmarker}{}%
\end{pgfscope}%
\end{pgfscope}%
\begin{pgfscope}%
\definecolor{textcolor}{rgb}{0.000000,0.000000,0.000000}%
\pgfsetstrokecolor{textcolor}%
\pgfsetfillcolor{textcolor}%
\pgftext[x=0.630330in,y=0.451547in,,top]{\color{textcolor}\rmfamily\fontsize{10.000000}{12.000000}\selectfont \(\displaystyle {0.00}\)}%
\end{pgfscope}%
\begin{pgfscope}%
\pgfpathrectangle{\pgfqpoint{0.630330in}{0.548769in}}{\pgfqpoint{3.096674in}{1.753186in}}%
\pgfusepath{clip}%
\pgfsetrectcap%
\pgfsetroundjoin%
\pgfsetlinewidth{0.803000pt}%
\definecolor{currentstroke}{rgb}{0.690196,0.690196,0.690196}%
\pgfsetstrokecolor{currentstroke}%
\pgfsetdash{}{0pt}%
\pgfpathmoveto{\pgfqpoint{1.146442in}{0.548769in}}%
\pgfpathlineto{\pgfqpoint{1.146442in}{2.301955in}}%
\pgfusepath{stroke}%
\end{pgfscope}%
\begin{pgfscope}%
\pgfsetbuttcap%
\pgfsetroundjoin%
\definecolor{currentfill}{rgb}{0.000000,0.000000,0.000000}%
\pgfsetfillcolor{currentfill}%
\pgfsetlinewidth{0.803000pt}%
\definecolor{currentstroke}{rgb}{0.000000,0.000000,0.000000}%
\pgfsetstrokecolor{currentstroke}%
\pgfsetdash{}{0pt}%
\pgfsys@defobject{currentmarker}{\pgfqpoint{0.000000in}{-0.048611in}}{\pgfqpoint{0.000000in}{0.000000in}}{%
\pgfpathmoveto{\pgfqpoint{0.000000in}{0.000000in}}%
\pgfpathlineto{\pgfqpoint{0.000000in}{-0.048611in}}%
\pgfusepath{stroke,fill}%
}%
\begin{pgfscope}%
\pgfsys@transformshift{1.146442in}{0.548769in}%
\pgfsys@useobject{currentmarker}{}%
\end{pgfscope}%
\end{pgfscope}%
\begin{pgfscope}%
\definecolor{textcolor}{rgb}{0.000000,0.000000,0.000000}%
\pgfsetstrokecolor{textcolor}%
\pgfsetfillcolor{textcolor}%
\pgftext[x=1.146442in,y=0.451547in,,top]{\color{textcolor}\rmfamily\fontsize{10.000000}{12.000000}\selectfont \(\displaystyle {0.25}\)}%
\end{pgfscope}%
\begin{pgfscope}%
\pgfpathrectangle{\pgfqpoint{0.630330in}{0.548769in}}{\pgfqpoint{3.096674in}{1.753186in}}%
\pgfusepath{clip}%
\pgfsetrectcap%
\pgfsetroundjoin%
\pgfsetlinewidth{0.803000pt}%
\definecolor{currentstroke}{rgb}{0.690196,0.690196,0.690196}%
\pgfsetstrokecolor{currentstroke}%
\pgfsetdash{}{0pt}%
\pgfpathmoveto{\pgfqpoint{1.662555in}{0.548769in}}%
\pgfpathlineto{\pgfqpoint{1.662555in}{2.301955in}}%
\pgfusepath{stroke}%
\end{pgfscope}%
\begin{pgfscope}%
\pgfsetbuttcap%
\pgfsetroundjoin%
\definecolor{currentfill}{rgb}{0.000000,0.000000,0.000000}%
\pgfsetfillcolor{currentfill}%
\pgfsetlinewidth{0.803000pt}%
\definecolor{currentstroke}{rgb}{0.000000,0.000000,0.000000}%
\pgfsetstrokecolor{currentstroke}%
\pgfsetdash{}{0pt}%
\pgfsys@defobject{currentmarker}{\pgfqpoint{0.000000in}{-0.048611in}}{\pgfqpoint{0.000000in}{0.000000in}}{%
\pgfpathmoveto{\pgfqpoint{0.000000in}{0.000000in}}%
\pgfpathlineto{\pgfqpoint{0.000000in}{-0.048611in}}%
\pgfusepath{stroke,fill}%
}%
\begin{pgfscope}%
\pgfsys@transformshift{1.662555in}{0.548769in}%
\pgfsys@useobject{currentmarker}{}%
\end{pgfscope}%
\end{pgfscope}%
\begin{pgfscope}%
\definecolor{textcolor}{rgb}{0.000000,0.000000,0.000000}%
\pgfsetstrokecolor{textcolor}%
\pgfsetfillcolor{textcolor}%
\pgftext[x=1.662555in,y=0.451547in,,top]{\color{textcolor}\rmfamily\fontsize{10.000000}{12.000000}\selectfont \(\displaystyle {0.50}\)}%
\end{pgfscope}%
\begin{pgfscope}%
\pgfpathrectangle{\pgfqpoint{0.630330in}{0.548769in}}{\pgfqpoint{3.096674in}{1.753186in}}%
\pgfusepath{clip}%
\pgfsetrectcap%
\pgfsetroundjoin%
\pgfsetlinewidth{0.803000pt}%
\definecolor{currentstroke}{rgb}{0.690196,0.690196,0.690196}%
\pgfsetstrokecolor{currentstroke}%
\pgfsetdash{}{0pt}%
\pgfpathmoveto{\pgfqpoint{2.178667in}{0.548769in}}%
\pgfpathlineto{\pgfqpoint{2.178667in}{2.301955in}}%
\pgfusepath{stroke}%
\end{pgfscope}%
\begin{pgfscope}%
\pgfsetbuttcap%
\pgfsetroundjoin%
\definecolor{currentfill}{rgb}{0.000000,0.000000,0.000000}%
\pgfsetfillcolor{currentfill}%
\pgfsetlinewidth{0.803000pt}%
\definecolor{currentstroke}{rgb}{0.000000,0.000000,0.000000}%
\pgfsetstrokecolor{currentstroke}%
\pgfsetdash{}{0pt}%
\pgfsys@defobject{currentmarker}{\pgfqpoint{0.000000in}{-0.048611in}}{\pgfqpoint{0.000000in}{0.000000in}}{%
\pgfpathmoveto{\pgfqpoint{0.000000in}{0.000000in}}%
\pgfpathlineto{\pgfqpoint{0.000000in}{-0.048611in}}%
\pgfusepath{stroke,fill}%
}%
\begin{pgfscope}%
\pgfsys@transformshift{2.178667in}{0.548769in}%
\pgfsys@useobject{currentmarker}{}%
\end{pgfscope}%
\end{pgfscope}%
\begin{pgfscope}%
\definecolor{textcolor}{rgb}{0.000000,0.000000,0.000000}%
\pgfsetstrokecolor{textcolor}%
\pgfsetfillcolor{textcolor}%
\pgftext[x=2.178667in,y=0.451547in,,top]{\color{textcolor}\rmfamily\fontsize{10.000000}{12.000000}\selectfont \(\displaystyle {0.75}\)}%
\end{pgfscope}%
\begin{pgfscope}%
\pgfpathrectangle{\pgfqpoint{0.630330in}{0.548769in}}{\pgfqpoint{3.096674in}{1.753186in}}%
\pgfusepath{clip}%
\pgfsetrectcap%
\pgfsetroundjoin%
\pgfsetlinewidth{0.803000pt}%
\definecolor{currentstroke}{rgb}{0.690196,0.690196,0.690196}%
\pgfsetstrokecolor{currentstroke}%
\pgfsetdash{}{0pt}%
\pgfpathmoveto{\pgfqpoint{2.694779in}{0.548769in}}%
\pgfpathlineto{\pgfqpoint{2.694779in}{2.301955in}}%
\pgfusepath{stroke}%
\end{pgfscope}%
\begin{pgfscope}%
\pgfsetbuttcap%
\pgfsetroundjoin%
\definecolor{currentfill}{rgb}{0.000000,0.000000,0.000000}%
\pgfsetfillcolor{currentfill}%
\pgfsetlinewidth{0.803000pt}%
\definecolor{currentstroke}{rgb}{0.000000,0.000000,0.000000}%
\pgfsetstrokecolor{currentstroke}%
\pgfsetdash{}{0pt}%
\pgfsys@defobject{currentmarker}{\pgfqpoint{0.000000in}{-0.048611in}}{\pgfqpoint{0.000000in}{0.000000in}}{%
\pgfpathmoveto{\pgfqpoint{0.000000in}{0.000000in}}%
\pgfpathlineto{\pgfqpoint{0.000000in}{-0.048611in}}%
\pgfusepath{stroke,fill}%
}%
\begin{pgfscope}%
\pgfsys@transformshift{2.694779in}{0.548769in}%
\pgfsys@useobject{currentmarker}{}%
\end{pgfscope}%
\end{pgfscope}%
\begin{pgfscope}%
\definecolor{textcolor}{rgb}{0.000000,0.000000,0.000000}%
\pgfsetstrokecolor{textcolor}%
\pgfsetfillcolor{textcolor}%
\pgftext[x=2.694779in,y=0.451547in,,top]{\color{textcolor}\rmfamily\fontsize{10.000000}{12.000000}\selectfont \(\displaystyle {1.00}\)}%
\end{pgfscope}%
\begin{pgfscope}%
\pgfpathrectangle{\pgfqpoint{0.630330in}{0.548769in}}{\pgfqpoint{3.096674in}{1.753186in}}%
\pgfusepath{clip}%
\pgfsetrectcap%
\pgfsetroundjoin%
\pgfsetlinewidth{0.803000pt}%
\definecolor{currentstroke}{rgb}{0.690196,0.690196,0.690196}%
\pgfsetstrokecolor{currentstroke}%
\pgfsetdash{}{0pt}%
\pgfpathmoveto{\pgfqpoint{3.210892in}{0.548769in}}%
\pgfpathlineto{\pgfqpoint{3.210892in}{2.301955in}}%
\pgfusepath{stroke}%
\end{pgfscope}%
\begin{pgfscope}%
\pgfsetbuttcap%
\pgfsetroundjoin%
\definecolor{currentfill}{rgb}{0.000000,0.000000,0.000000}%
\pgfsetfillcolor{currentfill}%
\pgfsetlinewidth{0.803000pt}%
\definecolor{currentstroke}{rgb}{0.000000,0.000000,0.000000}%
\pgfsetstrokecolor{currentstroke}%
\pgfsetdash{}{0pt}%
\pgfsys@defobject{currentmarker}{\pgfqpoint{0.000000in}{-0.048611in}}{\pgfqpoint{0.000000in}{0.000000in}}{%
\pgfpathmoveto{\pgfqpoint{0.000000in}{0.000000in}}%
\pgfpathlineto{\pgfqpoint{0.000000in}{-0.048611in}}%
\pgfusepath{stroke,fill}%
}%
\begin{pgfscope}%
\pgfsys@transformshift{3.210892in}{0.548769in}%
\pgfsys@useobject{currentmarker}{}%
\end{pgfscope}%
\end{pgfscope}%
\begin{pgfscope}%
\definecolor{textcolor}{rgb}{0.000000,0.000000,0.000000}%
\pgfsetstrokecolor{textcolor}%
\pgfsetfillcolor{textcolor}%
\pgftext[x=3.210892in,y=0.451547in,,top]{\color{textcolor}\rmfamily\fontsize{10.000000}{12.000000}\selectfont \(\displaystyle {1.25}\)}%
\end{pgfscope}%
\begin{pgfscope}%
\pgfpathrectangle{\pgfqpoint{0.630330in}{0.548769in}}{\pgfqpoint{3.096674in}{1.753186in}}%
\pgfusepath{clip}%
\pgfsetrectcap%
\pgfsetroundjoin%
\pgfsetlinewidth{0.803000pt}%
\definecolor{currentstroke}{rgb}{0.690196,0.690196,0.690196}%
\pgfsetstrokecolor{currentstroke}%
\pgfsetdash{}{0pt}%
\pgfpathmoveto{\pgfqpoint{3.727004in}{0.548769in}}%
\pgfpathlineto{\pgfqpoint{3.727004in}{2.301955in}}%
\pgfusepath{stroke}%
\end{pgfscope}%
\begin{pgfscope}%
\pgfsetbuttcap%
\pgfsetroundjoin%
\definecolor{currentfill}{rgb}{0.000000,0.000000,0.000000}%
\pgfsetfillcolor{currentfill}%
\pgfsetlinewidth{0.803000pt}%
\definecolor{currentstroke}{rgb}{0.000000,0.000000,0.000000}%
\pgfsetstrokecolor{currentstroke}%
\pgfsetdash{}{0pt}%
\pgfsys@defobject{currentmarker}{\pgfqpoint{0.000000in}{-0.048611in}}{\pgfqpoint{0.000000in}{0.000000in}}{%
\pgfpathmoveto{\pgfqpoint{0.000000in}{0.000000in}}%
\pgfpathlineto{\pgfqpoint{0.000000in}{-0.048611in}}%
\pgfusepath{stroke,fill}%
}%
\begin{pgfscope}%
\pgfsys@transformshift{3.727004in}{0.548769in}%
\pgfsys@useobject{currentmarker}{}%
\end{pgfscope}%
\end{pgfscope}%
\begin{pgfscope}%
\definecolor{textcolor}{rgb}{0.000000,0.000000,0.000000}%
\pgfsetstrokecolor{textcolor}%
\pgfsetfillcolor{textcolor}%
\pgftext[x=3.727004in,y=0.451547in,,top]{\color{textcolor}\rmfamily\fontsize{10.000000}{12.000000}\selectfont \(\displaystyle {1.50}\)}%
\end{pgfscope}%
\begin{pgfscope}%
\definecolor{textcolor}{rgb}{0.000000,0.000000,0.000000}%
\pgfsetstrokecolor{textcolor}%
\pgfsetfillcolor{textcolor}%
\pgftext[x=2.178667in,y=0.272534in,,top]{\color{textcolor}\rmfamily\fontsize{10.000000}{12.000000}\selectfont \(\displaystyle w\)}%
\end{pgfscope}%
\begin{pgfscope}%
\pgfpathrectangle{\pgfqpoint{0.630330in}{0.548769in}}{\pgfqpoint{3.096674in}{1.753186in}}%
\pgfusepath{clip}%
\pgfsetrectcap%
\pgfsetroundjoin%
\pgfsetlinewidth{0.803000pt}%
\definecolor{currentstroke}{rgb}{0.690196,0.690196,0.690196}%
\pgfsetstrokecolor{currentstroke}%
\pgfsetdash{}{0pt}%
\pgfpathmoveto{\pgfqpoint{0.630330in}{0.548769in}}%
\pgfpathlineto{\pgfqpoint{3.727004in}{0.548769in}}%
\pgfusepath{stroke}%
\end{pgfscope}%
\begin{pgfscope}%
\pgfsetbuttcap%
\pgfsetroundjoin%
\definecolor{currentfill}{rgb}{0.000000,0.000000,0.000000}%
\pgfsetfillcolor{currentfill}%
\pgfsetlinewidth{0.803000pt}%
\definecolor{currentstroke}{rgb}{0.000000,0.000000,0.000000}%
\pgfsetstrokecolor{currentstroke}%
\pgfsetdash{}{0pt}%
\pgfsys@defobject{currentmarker}{\pgfqpoint{-0.048611in}{0.000000in}}{\pgfqpoint{-0.000000in}{0.000000in}}{%
\pgfpathmoveto{\pgfqpoint{-0.000000in}{0.000000in}}%
\pgfpathlineto{\pgfqpoint{-0.048611in}{0.000000in}}%
\pgfusepath{stroke,fill}%
}%
\begin{pgfscope}%
\pgfsys@transformshift{0.630330in}{0.548769in}%
\pgfsys@useobject{currentmarker}{}%
\end{pgfscope}%
\end{pgfscope}%
\begin{pgfscope}%
\definecolor{textcolor}{rgb}{0.000000,0.000000,0.000000}%
\pgfsetstrokecolor{textcolor}%
\pgfsetfillcolor{textcolor}%
\pgftext[x=0.355638in, y=0.500544in, left, base]{\color{textcolor}\rmfamily\fontsize{10.000000}{12.000000}\selectfont \(\displaystyle {0.0}\)}%
\end{pgfscope}%
\begin{pgfscope}%
\pgfpathrectangle{\pgfqpoint{0.630330in}{0.548769in}}{\pgfqpoint{3.096674in}{1.753186in}}%
\pgfusepath{clip}%
\pgfsetrectcap%
\pgfsetroundjoin%
\pgfsetlinewidth{0.803000pt}%
\definecolor{currentstroke}{rgb}{0.690196,0.690196,0.690196}%
\pgfsetstrokecolor{currentstroke}%
\pgfsetdash{}{0pt}%
\pgfpathmoveto{\pgfqpoint{0.630330in}{0.987065in}}%
\pgfpathlineto{\pgfqpoint{3.727004in}{0.987065in}}%
\pgfusepath{stroke}%
\end{pgfscope}%
\begin{pgfscope}%
\pgfsetbuttcap%
\pgfsetroundjoin%
\definecolor{currentfill}{rgb}{0.000000,0.000000,0.000000}%
\pgfsetfillcolor{currentfill}%
\pgfsetlinewidth{0.803000pt}%
\definecolor{currentstroke}{rgb}{0.000000,0.000000,0.000000}%
\pgfsetstrokecolor{currentstroke}%
\pgfsetdash{}{0pt}%
\pgfsys@defobject{currentmarker}{\pgfqpoint{-0.048611in}{0.000000in}}{\pgfqpoint{-0.000000in}{0.000000in}}{%
\pgfpathmoveto{\pgfqpoint{-0.000000in}{0.000000in}}%
\pgfpathlineto{\pgfqpoint{-0.048611in}{0.000000in}}%
\pgfusepath{stroke,fill}%
}%
\begin{pgfscope}%
\pgfsys@transformshift{0.630330in}{0.987065in}%
\pgfsys@useobject{currentmarker}{}%
\end{pgfscope}%
\end{pgfscope}%
\begin{pgfscope}%
\definecolor{textcolor}{rgb}{0.000000,0.000000,0.000000}%
\pgfsetstrokecolor{textcolor}%
\pgfsetfillcolor{textcolor}%
\pgftext[x=0.355638in, y=0.938840in, left, base]{\color{textcolor}\rmfamily\fontsize{10.000000}{12.000000}\selectfont \(\displaystyle {0.5}\)}%
\end{pgfscope}%
\begin{pgfscope}%
\pgfpathrectangle{\pgfqpoint{0.630330in}{0.548769in}}{\pgfqpoint{3.096674in}{1.753186in}}%
\pgfusepath{clip}%
\pgfsetrectcap%
\pgfsetroundjoin%
\pgfsetlinewidth{0.803000pt}%
\definecolor{currentstroke}{rgb}{0.690196,0.690196,0.690196}%
\pgfsetstrokecolor{currentstroke}%
\pgfsetdash{}{0pt}%
\pgfpathmoveto{\pgfqpoint{0.630330in}{1.425362in}}%
\pgfpathlineto{\pgfqpoint{3.727004in}{1.425362in}}%
\pgfusepath{stroke}%
\end{pgfscope}%
\begin{pgfscope}%
\pgfsetbuttcap%
\pgfsetroundjoin%
\definecolor{currentfill}{rgb}{0.000000,0.000000,0.000000}%
\pgfsetfillcolor{currentfill}%
\pgfsetlinewidth{0.803000pt}%
\definecolor{currentstroke}{rgb}{0.000000,0.000000,0.000000}%
\pgfsetstrokecolor{currentstroke}%
\pgfsetdash{}{0pt}%
\pgfsys@defobject{currentmarker}{\pgfqpoint{-0.048611in}{0.000000in}}{\pgfqpoint{-0.000000in}{0.000000in}}{%
\pgfpathmoveto{\pgfqpoint{-0.000000in}{0.000000in}}%
\pgfpathlineto{\pgfqpoint{-0.048611in}{0.000000in}}%
\pgfusepath{stroke,fill}%
}%
\begin{pgfscope}%
\pgfsys@transformshift{0.630330in}{1.425362in}%
\pgfsys@useobject{currentmarker}{}%
\end{pgfscope}%
\end{pgfscope}%
\begin{pgfscope}%
\definecolor{textcolor}{rgb}{0.000000,0.000000,0.000000}%
\pgfsetstrokecolor{textcolor}%
\pgfsetfillcolor{textcolor}%
\pgftext[x=0.355638in, y=1.377137in, left, base]{\color{textcolor}\rmfamily\fontsize{10.000000}{12.000000}\selectfont \(\displaystyle {1.0}\)}%
\end{pgfscope}%
\begin{pgfscope}%
\pgfpathrectangle{\pgfqpoint{0.630330in}{0.548769in}}{\pgfqpoint{3.096674in}{1.753186in}}%
\pgfusepath{clip}%
\pgfsetrectcap%
\pgfsetroundjoin%
\pgfsetlinewidth{0.803000pt}%
\definecolor{currentstroke}{rgb}{0.690196,0.690196,0.690196}%
\pgfsetstrokecolor{currentstroke}%
\pgfsetdash{}{0pt}%
\pgfpathmoveto{\pgfqpoint{0.630330in}{1.863658in}}%
\pgfpathlineto{\pgfqpoint{3.727004in}{1.863658in}}%
\pgfusepath{stroke}%
\end{pgfscope}%
\begin{pgfscope}%
\pgfsetbuttcap%
\pgfsetroundjoin%
\definecolor{currentfill}{rgb}{0.000000,0.000000,0.000000}%
\pgfsetfillcolor{currentfill}%
\pgfsetlinewidth{0.803000pt}%
\definecolor{currentstroke}{rgb}{0.000000,0.000000,0.000000}%
\pgfsetstrokecolor{currentstroke}%
\pgfsetdash{}{0pt}%
\pgfsys@defobject{currentmarker}{\pgfqpoint{-0.048611in}{0.000000in}}{\pgfqpoint{-0.000000in}{0.000000in}}{%
\pgfpathmoveto{\pgfqpoint{-0.000000in}{0.000000in}}%
\pgfpathlineto{\pgfqpoint{-0.048611in}{0.000000in}}%
\pgfusepath{stroke,fill}%
}%
\begin{pgfscope}%
\pgfsys@transformshift{0.630330in}{1.863658in}%
\pgfsys@useobject{currentmarker}{}%
\end{pgfscope}%
\end{pgfscope}%
\begin{pgfscope}%
\definecolor{textcolor}{rgb}{0.000000,0.000000,0.000000}%
\pgfsetstrokecolor{textcolor}%
\pgfsetfillcolor{textcolor}%
\pgftext[x=0.355638in, y=1.815433in, left, base]{\color{textcolor}\rmfamily\fontsize{10.000000}{12.000000}\selectfont \(\displaystyle {1.5}\)}%
\end{pgfscope}%
\begin{pgfscope}%
\pgfpathrectangle{\pgfqpoint{0.630330in}{0.548769in}}{\pgfqpoint{3.096674in}{1.753186in}}%
\pgfusepath{clip}%
\pgfsetrectcap%
\pgfsetroundjoin%
\pgfsetlinewidth{0.803000pt}%
\definecolor{currentstroke}{rgb}{0.690196,0.690196,0.690196}%
\pgfsetstrokecolor{currentstroke}%
\pgfsetdash{}{0pt}%
\pgfpathmoveto{\pgfqpoint{0.630330in}{2.301955in}}%
\pgfpathlineto{\pgfqpoint{3.727004in}{2.301955in}}%
\pgfusepath{stroke}%
\end{pgfscope}%
\begin{pgfscope}%
\pgfsetbuttcap%
\pgfsetroundjoin%
\definecolor{currentfill}{rgb}{0.000000,0.000000,0.000000}%
\pgfsetfillcolor{currentfill}%
\pgfsetlinewidth{0.803000pt}%
\definecolor{currentstroke}{rgb}{0.000000,0.000000,0.000000}%
\pgfsetstrokecolor{currentstroke}%
\pgfsetdash{}{0pt}%
\pgfsys@defobject{currentmarker}{\pgfqpoint{-0.048611in}{0.000000in}}{\pgfqpoint{-0.000000in}{0.000000in}}{%
\pgfpathmoveto{\pgfqpoint{-0.000000in}{0.000000in}}%
\pgfpathlineto{\pgfqpoint{-0.048611in}{0.000000in}}%
\pgfusepath{stroke,fill}%
}%
\begin{pgfscope}%
\pgfsys@transformshift{0.630330in}{2.301955in}%
\pgfsys@useobject{currentmarker}{}%
\end{pgfscope}%
\end{pgfscope}%
\begin{pgfscope}%
\definecolor{textcolor}{rgb}{0.000000,0.000000,0.000000}%
\pgfsetstrokecolor{textcolor}%
\pgfsetfillcolor{textcolor}%
\pgftext[x=0.355638in, y=2.253730in, left, base]{\color{textcolor}\rmfamily\fontsize{10.000000}{12.000000}\selectfont \(\displaystyle {2.0}\)}%
\end{pgfscope}%
\begin{pgfscope}%
\definecolor{textcolor}{rgb}{0.000000,0.000000,0.000000}%
\pgfsetstrokecolor{textcolor}%
\pgfsetfillcolor{textcolor}%
\pgftext[x=0.300082in,y=1.425362in,,bottom,rotate=90.000000]{\color{textcolor}\rmfamily\fontsize{10.000000}{12.000000}\selectfont \(\displaystyle F^2_N(w)\)}%
\end{pgfscope}%
\begin{pgfscope}%
\pgfpathrectangle{\pgfqpoint{0.630330in}{0.548769in}}{\pgfqpoint{3.096674in}{1.753186in}}%
\pgfusepath{clip}%
\pgfsetrectcap%
\pgfsetroundjoin%
\pgfsetlinewidth{1.505625pt}%
\definecolor{currentstroke}{rgb}{0.121569,0.466667,0.705882}%
\pgfsetstrokecolor{currentstroke}%
\pgfsetdash{}{0pt}%
\pgfpathmoveto{\pgfqpoint{0.630330in}{0.548769in}}%
\pgfpathlineto{\pgfqpoint{0.661609in}{0.548970in}}%
\pgfpathlineto{\pgfqpoint{0.692889in}{0.549574in}}%
\pgfpathlineto{\pgfqpoint{0.724168in}{0.550580in}}%
\pgfpathlineto{\pgfqpoint{0.755448in}{0.551989in}}%
\pgfpathlineto{\pgfqpoint{0.786727in}{0.553800in}}%
\pgfpathlineto{\pgfqpoint{0.818007in}{0.556013in}}%
\pgfpathlineto{\pgfqpoint{0.849287in}{0.558629in}}%
\pgfpathlineto{\pgfqpoint{0.880566in}{0.561648in}}%
\pgfpathlineto{\pgfqpoint{0.911846in}{0.565069in}}%
\pgfpathlineto{\pgfqpoint{0.943125in}{0.568893in}}%
\pgfpathlineto{\pgfqpoint{0.974405in}{0.573119in}}%
\pgfpathlineto{\pgfqpoint{1.005684in}{0.577747in}}%
\pgfpathlineto{\pgfqpoint{1.036964in}{0.582778in}}%
\pgfpathlineto{\pgfqpoint{1.068243in}{0.588211in}}%
\pgfpathlineto{\pgfqpoint{1.099523in}{0.594047in}}%
\pgfpathlineto{\pgfqpoint{1.130802in}{0.600286in}}%
\pgfpathlineto{\pgfqpoint{1.162082in}{0.606927in}}%
\pgfpathlineto{\pgfqpoint{1.193361in}{0.613970in}}%
\pgfpathlineto{\pgfqpoint{1.224641in}{0.621416in}}%
\pgfpathlineto{\pgfqpoint{1.255921in}{0.629264in}}%
\pgfpathlineto{\pgfqpoint{1.287200in}{0.637515in}}%
\pgfpathlineto{\pgfqpoint{1.318480in}{0.646168in}}%
\pgfpathlineto{\pgfqpoint{1.349759in}{0.655224in}}%
\pgfpathlineto{\pgfqpoint{1.381039in}{0.664682in}}%
\pgfpathlineto{\pgfqpoint{1.412318in}{0.674543in}}%
\pgfpathlineto{\pgfqpoint{1.443598in}{0.684806in}}%
\pgfpathlineto{\pgfqpoint{1.474877in}{0.695471in}}%
\pgfpathlineto{\pgfqpoint{1.506157in}{0.706539in}}%
\pgfpathlineto{\pgfqpoint{1.537436in}{0.718010in}}%
\pgfpathlineto{\pgfqpoint{1.568716in}{0.729883in}}%
\pgfpathlineto{\pgfqpoint{1.599995in}{0.742159in}}%
\pgfpathlineto{\pgfqpoint{1.631275in}{0.754837in}}%
\pgfpathlineto{\pgfqpoint{1.662555in}{0.767917in}}%
\pgfpathlineto{\pgfqpoint{1.693834in}{0.781400in}}%
\pgfpathlineto{\pgfqpoint{1.725114in}{0.795285in}}%
\pgfpathlineto{\pgfqpoint{1.756393in}{0.809573in}}%
\pgfpathlineto{\pgfqpoint{1.787673in}{0.824264in}}%
\pgfpathlineto{\pgfqpoint{1.818952in}{0.839357in}}%
\pgfpathlineto{\pgfqpoint{1.850232in}{0.854852in}}%
\pgfpathlineto{\pgfqpoint{1.881511in}{0.870750in}}%
\pgfpathlineto{\pgfqpoint{1.912791in}{0.887050in}}%
\pgfpathlineto{\pgfqpoint{1.944070in}{0.903753in}}%
\pgfpathlineto{\pgfqpoint{1.975350in}{0.920858in}}%
\pgfpathlineto{\pgfqpoint{2.006629in}{0.938366in}}%
\pgfpathlineto{\pgfqpoint{2.037909in}{0.956276in}}%
\pgfpathlineto{\pgfqpoint{2.069189in}{0.974589in}}%
\pgfpathlineto{\pgfqpoint{2.100468in}{0.993304in}}%
\pgfpathlineto{\pgfqpoint{2.131748in}{1.012421in}}%
\pgfpathlineto{\pgfqpoint{2.163027in}{1.031941in}}%
\pgfpathlineto{\pgfqpoint{2.194307in}{1.051864in}}%
\pgfpathlineto{\pgfqpoint{2.225586in}{1.072189in}}%
\pgfpathlineto{\pgfqpoint{2.256866in}{1.092917in}}%
\pgfpathlineto{\pgfqpoint{2.288145in}{1.114047in}}%
\pgfpathlineto{\pgfqpoint{2.319425in}{1.135579in}}%
\pgfpathlineto{\pgfqpoint{2.350704in}{1.157514in}}%
\pgfpathlineto{\pgfqpoint{2.381984in}{1.179851in}}%
\pgfpathlineto{\pgfqpoint{2.413263in}{1.202591in}}%
\pgfpathlineto{\pgfqpoint{2.444543in}{1.225734in}}%
\pgfpathlineto{\pgfqpoint{2.475823in}{1.249279in}}%
\pgfpathlineto{\pgfqpoint{2.507102in}{1.273226in}}%
\pgfpathlineto{\pgfqpoint{2.538382in}{1.297576in}}%
\pgfpathlineto{\pgfqpoint{2.569661in}{1.322328in}}%
\pgfpathlineto{\pgfqpoint{2.600941in}{1.347483in}}%
\pgfpathlineto{\pgfqpoint{2.632220in}{1.373040in}}%
\pgfpathlineto{\pgfqpoint{2.663500in}{1.399000in}}%
\pgfpathlineto{\pgfqpoint{2.694779in}{1.425362in}}%
\pgfpathlineto{\pgfqpoint{2.726059in}{1.452126in}}%
\pgfpathlineto{\pgfqpoint{2.757338in}{1.479294in}}%
\pgfpathlineto{\pgfqpoint{2.788618in}{1.506863in}}%
\pgfpathlineto{\pgfqpoint{2.819897in}{1.534835in}}%
\pgfpathlineto{\pgfqpoint{2.851177in}{1.563210in}}%
\pgfpathlineto{\pgfqpoint{2.882457in}{1.591987in}}%
\pgfpathlineto{\pgfqpoint{2.913736in}{1.621166in}}%
\pgfpathlineto{\pgfqpoint{2.945016in}{1.650748in}}%
\pgfpathlineto{\pgfqpoint{2.976295in}{1.680733in}}%
\pgfpathlineto{\pgfqpoint{3.007575in}{1.711120in}}%
\pgfpathlineto{\pgfqpoint{3.038854in}{1.741909in}}%
\pgfpathlineto{\pgfqpoint{3.070134in}{1.773101in}}%
\pgfpathlineto{\pgfqpoint{3.101413in}{1.804696in}}%
\pgfpathlineto{\pgfqpoint{3.132693in}{1.836692in}}%
\pgfpathlineto{\pgfqpoint{3.163972in}{1.869092in}}%
\pgfpathlineto{\pgfqpoint{3.195252in}{1.901894in}}%
\pgfpathlineto{\pgfqpoint{3.226531in}{1.935098in}}%
\pgfpathlineto{\pgfqpoint{3.257811in}{1.968705in}}%
\pgfpathlineto{\pgfqpoint{3.289091in}{2.002714in}}%
\pgfpathlineto{\pgfqpoint{3.320370in}{2.037126in}}%
\pgfpathlineto{\pgfqpoint{3.351650in}{2.071940in}}%
\pgfpathlineto{\pgfqpoint{3.382929in}{2.107156in}}%
\pgfpathlineto{\pgfqpoint{3.414209in}{2.142776in}}%
\pgfpathlineto{\pgfqpoint{3.445488in}{2.178797in}}%
\pgfpathlineto{\pgfqpoint{3.476768in}{2.215221in}}%
\pgfpathlineto{\pgfqpoint{3.508047in}{2.252048in}}%
\pgfpathlineto{\pgfqpoint{3.539327in}{2.289277in}}%
\pgfpathlineto{\pgfqpoint{3.561409in}{2.315844in}}%
\pgfusepath{stroke}%
\end{pgfscope}%
\begin{pgfscope}%
\pgfpathrectangle{\pgfqpoint{0.630330in}{0.548769in}}{\pgfqpoint{3.096674in}{1.753186in}}%
\pgfusepath{clip}%
\pgfsetrectcap%
\pgfsetroundjoin%
\pgfsetlinewidth{1.505625pt}%
\definecolor{currentstroke}{rgb}{1.000000,0.498039,0.054902}%
\pgfsetstrokecolor{currentstroke}%
\pgfsetdash{}{0pt}%
\pgfpathmoveto{\pgfqpoint{0.630330in}{1.425362in}}%
\pgfpathlineto{\pgfqpoint{0.661609in}{1.424557in}}%
\pgfpathlineto{\pgfqpoint{0.692889in}{1.422145in}}%
\pgfpathlineto{\pgfqpoint{0.724168in}{1.418132in}}%
\pgfpathlineto{\pgfqpoint{0.755448in}{1.412530in}}%
\pgfpathlineto{\pgfqpoint{0.786727in}{1.405354in}}%
\pgfpathlineto{\pgfqpoint{0.818007in}{1.396623in}}%
\pgfpathlineto{\pgfqpoint{0.849287in}{1.386363in}}%
\pgfpathlineto{\pgfqpoint{0.880566in}{1.374602in}}%
\pgfpathlineto{\pgfqpoint{0.911846in}{1.361373in}}%
\pgfpathlineto{\pgfqpoint{0.943125in}{1.346715in}}%
\pgfpathlineto{\pgfqpoint{0.974405in}{1.330668in}}%
\pgfpathlineto{\pgfqpoint{1.005684in}{1.313281in}}%
\pgfpathlineto{\pgfqpoint{1.036964in}{1.294603in}}%
\pgfpathlineto{\pgfqpoint{1.068243in}{1.274690in}}%
\pgfpathlineto{\pgfqpoint{1.099523in}{1.253603in}}%
\pgfpathlineto{\pgfqpoint{1.130802in}{1.231405in}}%
\pgfpathlineto{\pgfqpoint{1.162082in}{1.208165in}}%
\pgfpathlineto{\pgfqpoint{1.193361in}{1.183956in}}%
\pgfpathlineto{\pgfqpoint{1.224641in}{1.158856in}}%
\pgfpathlineto{\pgfqpoint{1.255921in}{1.132948in}}%
\pgfpathlineto{\pgfqpoint{1.287200in}{1.106316in}}%
\pgfpathlineto{\pgfqpoint{1.318480in}{1.079053in}}%
\pgfpathlineto{\pgfqpoint{1.349759in}{1.051254in}}%
\pgfpathlineto{\pgfqpoint{1.381039in}{1.023019in}}%
\pgfpathlineto{\pgfqpoint{1.412318in}{0.994451in}}%
\pgfpathlineto{\pgfqpoint{1.443598in}{0.965659in}}%
\pgfpathlineto{\pgfqpoint{1.474877in}{0.936757in}}%
\pgfpathlineto{\pgfqpoint{1.506157in}{0.907863in}}%
\pgfpathlineto{\pgfqpoint{1.537436in}{0.879097in}}%
\pgfpathlineto{\pgfqpoint{1.568716in}{0.850586in}}%
\pgfpathlineto{\pgfqpoint{1.599995in}{0.822462in}}%
\pgfpathlineto{\pgfqpoint{1.631275in}{0.794859in}}%
\pgfpathlineto{\pgfqpoint{1.662555in}{0.767917in}}%
\pgfpathlineto{\pgfqpoint{1.693834in}{0.741781in}}%
\pgfpathlineto{\pgfqpoint{1.725114in}{0.716598in}}%
\pgfpathlineto{\pgfqpoint{1.756393in}{0.692523in}}%
\pgfpathlineto{\pgfqpoint{1.787673in}{0.669711in}}%
\pgfpathlineto{\pgfqpoint{1.818952in}{0.648326in}}%
\pgfpathlineto{\pgfqpoint{1.850232in}{0.628534in}}%
\pgfpathlineto{\pgfqpoint{1.881511in}{0.610505in}}%
\pgfpathlineto{\pgfqpoint{1.912791in}{0.594414in}}%
\pgfpathlineto{\pgfqpoint{1.944070in}{0.580441in}}%
\pgfpathlineto{\pgfqpoint{1.975350in}{0.568771in}}%
\pgfpathlineto{\pgfqpoint{2.006629in}{0.559591in}}%
\pgfpathlineto{\pgfqpoint{2.037909in}{0.553095in}}%
\pgfpathlineto{\pgfqpoint{2.069189in}{0.549479in}}%
\pgfpathlineto{\pgfqpoint{2.100468in}{0.548946in}}%
\pgfpathlineto{\pgfqpoint{2.131748in}{0.551703in}}%
\pgfpathlineto{\pgfqpoint{2.163027in}{0.557958in}}%
\pgfpathlineto{\pgfqpoint{2.194307in}{0.567929in}}%
\pgfpathlineto{\pgfqpoint{2.225586in}{0.581833in}}%
\pgfpathlineto{\pgfqpoint{2.256866in}{0.599896in}}%
\pgfpathlineto{\pgfqpoint{2.288145in}{0.622346in}}%
\pgfpathlineto{\pgfqpoint{2.319425in}{0.649414in}}%
\pgfpathlineto{\pgfqpoint{2.350704in}{0.681340in}}%
\pgfpathlineto{\pgfqpoint{2.381984in}{0.718364in}}%
\pgfpathlineto{\pgfqpoint{2.413263in}{0.760732in}}%
\pgfpathlineto{\pgfqpoint{2.444543in}{0.808696in}}%
\pgfpathlineto{\pgfqpoint{2.475823in}{0.862510in}}%
\pgfpathlineto{\pgfqpoint{2.507102in}{0.922433in}}%
\pgfpathlineto{\pgfqpoint{2.538382in}{0.988730in}}%
\pgfpathlineto{\pgfqpoint{2.569661in}{1.061668in}}%
\pgfpathlineto{\pgfqpoint{2.600941in}{1.141521in}}%
\pgfpathlineto{\pgfqpoint{2.632220in}{1.228566in}}%
\pgfpathlineto{\pgfqpoint{2.663500in}{1.323084in}}%
\pgfpathlineto{\pgfqpoint{2.694779in}{1.425362in}}%
\pgfpathlineto{\pgfqpoint{2.726059in}{1.535689in}}%
\pgfpathlineto{\pgfqpoint{2.757338in}{1.654362in}}%
\pgfpathlineto{\pgfqpoint{2.788618in}{1.781678in}}%
\pgfpathlineto{\pgfqpoint{2.819897in}{1.917942in}}%
\pgfpathlineto{\pgfqpoint{2.851177in}{2.063463in}}%
\pgfpathlineto{\pgfqpoint{2.882457in}{2.218553in}}%
\pgfpathlineto{\pgfqpoint{2.900903in}{2.315844in}}%
\pgfusepath{stroke}%
\end{pgfscope}%
\begin{pgfscope}%
\pgfpathrectangle{\pgfqpoint{0.630330in}{0.548769in}}{\pgfqpoint{3.096674in}{1.753186in}}%
\pgfusepath{clip}%
\pgfsetrectcap%
\pgfsetroundjoin%
\pgfsetlinewidth{1.505625pt}%
\definecolor{currentstroke}{rgb}{0.172549,0.627451,0.172549}%
\pgfsetstrokecolor{currentstroke}%
\pgfsetdash{}{0pt}%
\pgfpathmoveto{\pgfqpoint{0.630330in}{0.548769in}}%
\pgfpathlineto{\pgfqpoint{0.661609in}{0.550579in}}%
\pgfpathlineto{\pgfqpoint{0.692889in}{0.555996in}}%
\pgfpathlineto{\pgfqpoint{0.724168in}{0.564979in}}%
\pgfpathlineto{\pgfqpoint{0.755448in}{0.577464in}}%
\pgfpathlineto{\pgfqpoint{0.786727in}{0.593357in}}%
\pgfpathlineto{\pgfqpoint{0.818007in}{0.612541in}}%
\pgfpathlineto{\pgfqpoint{0.849287in}{0.634873in}}%
\pgfpathlineto{\pgfqpoint{0.880566in}{0.660185in}}%
\pgfpathlineto{\pgfqpoint{0.911846in}{0.688287in}}%
\pgfpathlineto{\pgfqpoint{0.943125in}{0.718965in}}%
\pgfpathlineto{\pgfqpoint{0.974405in}{0.751984in}}%
\pgfpathlineto{\pgfqpoint{1.005684in}{0.787089in}}%
\pgfpathlineto{\pgfqpoint{1.036964in}{0.824004in}}%
\pgfpathlineto{\pgfqpoint{1.068243in}{0.862437in}}%
\pgfpathlineto{\pgfqpoint{1.099523in}{0.902078in}}%
\pgfpathlineto{\pgfqpoint{1.130802in}{0.942605in}}%
\pgfpathlineto{\pgfqpoint{1.162082in}{0.983681in}}%
\pgfpathlineto{\pgfqpoint{1.193361in}{1.024958in}}%
\pgfpathlineto{\pgfqpoint{1.224641in}{1.066081in}}%
\pgfpathlineto{\pgfqpoint{1.255921in}{1.106686in}}%
\pgfpathlineto{\pgfqpoint{1.287200in}{1.146406in}}%
\pgfpathlineto{\pgfqpoint{1.318480in}{1.184870in}}%
\pgfpathlineto{\pgfqpoint{1.349759in}{1.221710in}}%
\pgfpathlineto{\pgfqpoint{1.381039in}{1.256559in}}%
\pgfpathlineto{\pgfqpoint{1.412318in}{1.289056in}}%
\pgfpathlineto{\pgfqpoint{1.443598in}{1.318849in}}%
\pgfpathlineto{\pgfqpoint{1.474877in}{1.345598in}}%
\pgfpathlineto{\pgfqpoint{1.506157in}{1.368977in}}%
\pgfpathlineto{\pgfqpoint{1.537436in}{1.388677in}}%
\pgfpathlineto{\pgfqpoint{1.568716in}{1.404413in}}%
\pgfpathlineto{\pgfqpoint{1.599995in}{1.415923in}}%
\pgfpathlineto{\pgfqpoint{1.631275in}{1.422973in}}%
\pgfpathlineto{\pgfqpoint{1.662555in}{1.425362in}}%
\pgfpathlineto{\pgfqpoint{1.693834in}{1.422924in}}%
\pgfpathlineto{\pgfqpoint{1.725114in}{1.415535in}}%
\pgfpathlineto{\pgfqpoint{1.756393in}{1.403113in}}%
\pgfpathlineto{\pgfqpoint{1.787673in}{1.385624in}}%
\pgfpathlineto{\pgfqpoint{1.818952in}{1.363088in}}%
\pgfpathlineto{\pgfqpoint{1.850232in}{1.335583in}}%
\pgfpathlineto{\pgfqpoint{1.881511in}{1.303245in}}%
\pgfpathlineto{\pgfqpoint{1.912791in}{1.266280in}}%
\pgfpathlineto{\pgfqpoint{1.944070in}{1.224962in}}%
\pgfpathlineto{\pgfqpoint{1.975350in}{1.179644in}}%
\pgfpathlineto{\pgfqpoint{2.006629in}{1.130759in}}%
\pgfpathlineto{\pgfqpoint{2.037909in}{1.078826in}}%
\pgfpathlineto{\pgfqpoint{2.069189in}{1.024455in}}%
\pgfpathlineto{\pgfqpoint{2.100468in}{0.968355in}}%
\pgfpathlineto{\pgfqpoint{2.131748in}{0.911337in}}%
\pgfpathlineto{\pgfqpoint{2.163027in}{0.854319in}}%
\pgfpathlineto{\pgfqpoint{2.194307in}{0.798335in}}%
\pgfpathlineto{\pgfqpoint{2.225586in}{0.744537in}}%
\pgfpathlineto{\pgfqpoint{2.256866in}{0.694207in}}%
\pgfpathlineto{\pgfqpoint{2.288145in}{0.648754in}}%
\pgfpathlineto{\pgfqpoint{2.319425in}{0.609730in}}%
\pgfpathlineto{\pgfqpoint{2.350704in}{0.578830in}}%
\pgfpathlineto{\pgfqpoint{2.381984in}{0.557901in}}%
\pgfpathlineto{\pgfqpoint{2.413263in}{0.548947in}}%
\pgfpathlineto{\pgfqpoint{2.444543in}{0.554140in}}%
\pgfpathlineto{\pgfqpoint{2.475823in}{0.575820in}}%
\pgfpathlineto{\pgfqpoint{2.507102in}{0.616509in}}%
\pgfpathlineto{\pgfqpoint{2.538382in}{0.678913in}}%
\pgfpathlineto{\pgfqpoint{2.569661in}{0.765934in}}%
\pgfpathlineto{\pgfqpoint{2.600941in}{0.880671in}}%
\pgfpathlineto{\pgfqpoint{2.632220in}{1.026434in}}%
\pgfpathlineto{\pgfqpoint{2.663500in}{1.206748in}}%
\pgfpathlineto{\pgfqpoint{2.694779in}{1.425362in}}%
\pgfpathlineto{\pgfqpoint{2.726059in}{1.686256in}}%
\pgfpathlineto{\pgfqpoint{2.757338in}{1.993649in}}%
\pgfpathlineto{\pgfqpoint{2.785461in}{2.315844in}}%
\pgfusepath{stroke}%
\end{pgfscope}%
\begin{pgfscope}%
\pgfpathrectangle{\pgfqpoint{0.630330in}{0.548769in}}{\pgfqpoint{3.096674in}{1.753186in}}%
\pgfusepath{clip}%
\pgfsetrectcap%
\pgfsetroundjoin%
\pgfsetlinewidth{1.505625pt}%
\definecolor{currentstroke}{rgb}{0.839216,0.152941,0.156863}%
\pgfsetstrokecolor{currentstroke}%
\pgfsetdash{}{0pt}%
\pgfpathmoveto{\pgfqpoint{0.630330in}{1.425362in}}%
\pgfpathlineto{\pgfqpoint{0.661609in}{1.422146in}}%
\pgfpathlineto{\pgfqpoint{0.692889in}{1.412542in}}%
\pgfpathlineto{\pgfqpoint{0.724168in}{1.396682in}}%
\pgfpathlineto{\pgfqpoint{0.755448in}{1.374785in}}%
\pgfpathlineto{\pgfqpoint{0.786727in}{1.347155in}}%
\pgfpathlineto{\pgfqpoint{0.818007in}{1.314175in}}%
\pgfpathlineto{\pgfqpoint{0.849287in}{1.276306in}}%
\pgfpathlineto{\pgfqpoint{0.880566in}{1.234079in}}%
\pgfpathlineto{\pgfqpoint{0.911846in}{1.188091in}}%
\pgfpathlineto{\pgfqpoint{0.943125in}{1.138997in}}%
\pgfpathlineto{\pgfqpoint{0.974405in}{1.087504in}}%
\pgfpathlineto{\pgfqpoint{1.005684in}{1.034360in}}%
\pgfpathlineto{\pgfqpoint{1.036964in}{0.980345in}}%
\pgfpathlineto{\pgfqpoint{1.068243in}{0.926267in}}%
\pgfpathlineto{\pgfqpoint{1.099523in}{0.872943in}}%
\pgfpathlineto{\pgfqpoint{1.130802in}{0.821195in}}%
\pgfpathlineto{\pgfqpoint{1.162082in}{0.771836in}}%
\pgfpathlineto{\pgfqpoint{1.193361in}{0.725662in}}%
\pgfpathlineto{\pgfqpoint{1.224641in}{0.683436in}}%
\pgfpathlineto{\pgfqpoint{1.255921in}{0.645879in}}%
\pgfpathlineto{\pgfqpoint{1.287200in}{0.613660in}}%
\pgfpathlineto{\pgfqpoint{1.318480in}{0.587381in}}%
\pgfpathlineto{\pgfqpoint{1.349759in}{0.567570in}}%
\pgfpathlineto{\pgfqpoint{1.381039in}{0.554667in}}%
\pgfpathlineto{\pgfqpoint{1.412318in}{0.549018in}}%
\pgfpathlineto{\pgfqpoint{1.443598in}{0.550860in}}%
\pgfpathlineto{\pgfqpoint{1.474877in}{0.560318in}}%
\pgfpathlineto{\pgfqpoint{1.506157in}{0.577394in}}%
\pgfpathlineto{\pgfqpoint{1.537436in}{0.601962in}}%
\pgfpathlineto{\pgfqpoint{1.568716in}{0.633764in}}%
\pgfpathlineto{\pgfqpoint{1.599995in}{0.672404in}}%
\pgfpathlineto{\pgfqpoint{1.631275in}{0.717346in}}%
\pgfpathlineto{\pgfqpoint{1.662555in}{0.767917in}}%
\pgfpathlineto{\pgfqpoint{1.693834in}{0.823307in}}%
\pgfpathlineto{\pgfqpoint{1.725114in}{0.882572in}}%
\pgfpathlineto{\pgfqpoint{1.756393in}{0.944644in}}%
\pgfpathlineto{\pgfqpoint{1.787673in}{1.008337in}}%
\pgfpathlineto{\pgfqpoint{1.818952in}{1.072360in}}%
\pgfpathlineto{\pgfqpoint{1.850232in}{1.135334in}}%
\pgfpathlineto{\pgfqpoint{1.881511in}{1.195810in}}%
\pgfpathlineto{\pgfqpoint{1.912791in}{1.252288in}}%
\pgfpathlineto{\pgfqpoint{1.944070in}{1.303249in}}%
\pgfpathlineto{\pgfqpoint{1.975350in}{1.347179in}}%
\pgfpathlineto{\pgfqpoint{2.006629in}{1.382608in}}%
\pgfpathlineto{\pgfqpoint{2.037909in}{1.408144in}}%
\pgfpathlineto{\pgfqpoint{2.069189in}{1.422523in}}%
\pgfpathlineto{\pgfqpoint{2.100468in}{1.424652in}}%
\pgfpathlineto{\pgfqpoint{2.131748in}{1.413666in}}%
\pgfpathlineto{\pgfqpoint{2.163027in}{1.388989in}}%
\pgfpathlineto{\pgfqpoint{2.194307in}{1.350397in}}%
\pgfpathlineto{\pgfqpoint{2.225586in}{1.298092in}}%
\pgfpathlineto{\pgfqpoint{2.256866in}{1.232780in}}%
\pgfpathlineto{\pgfqpoint{2.288145in}{1.155757in}}%
\pgfpathlineto{\pgfqpoint{2.319425in}{1.069002in}}%
\pgfpathlineto{\pgfqpoint{2.350704in}{0.975275in}}%
\pgfpathlineto{\pgfqpoint{2.381984in}{0.878228in}}%
\pgfpathlineto{\pgfqpoint{2.413263in}{0.782522in}}%
\pgfpathlineto{\pgfqpoint{2.444543in}{0.693947in}}%
\pgfpathlineto{\pgfqpoint{2.475823in}{0.619562in}}%
\pgfpathlineto{\pgfqpoint{2.507102in}{0.567831in}}%
\pgfpathlineto{\pgfqpoint{2.538382in}{0.548781in}}%
\pgfpathlineto{\pgfqpoint{2.569661in}{0.574165in}}%
\pgfpathlineto{\pgfqpoint{2.600941in}{0.657630in}}%
\pgfpathlineto{\pgfqpoint{2.632220in}{0.814902in}}%
\pgfpathlineto{\pgfqpoint{2.663500in}{1.063985in}}%
\pgfpathlineto{\pgfqpoint{2.694779in}{1.425362in}}%
\pgfpathlineto{\pgfqpoint{2.726059in}{1.922215in}}%
\pgfpathlineto{\pgfqpoint{2.744758in}{2.315844in}}%
\pgfusepath{stroke}%
\end{pgfscope}%
\begin{pgfscope}%
\pgfsetrectcap%
\pgfsetmiterjoin%
\pgfsetlinewidth{0.803000pt}%
\definecolor{currentstroke}{rgb}{0.000000,0.000000,0.000000}%
\pgfsetstrokecolor{currentstroke}%
\pgfsetdash{}{0pt}%
\pgfpathmoveto{\pgfqpoint{0.630330in}{0.548769in}}%
\pgfpathlineto{\pgfqpoint{0.630330in}{2.301955in}}%
\pgfusepath{stroke}%
\end{pgfscope}%
\begin{pgfscope}%
\pgfsetrectcap%
\pgfsetmiterjoin%
\pgfsetlinewidth{0.803000pt}%
\definecolor{currentstroke}{rgb}{0.000000,0.000000,0.000000}%
\pgfsetstrokecolor{currentstroke}%
\pgfsetdash{}{0pt}%
\pgfpathmoveto{\pgfqpoint{3.727004in}{0.548769in}}%
\pgfpathlineto{\pgfqpoint{3.727004in}{2.301955in}}%
\pgfusepath{stroke}%
\end{pgfscope}%
\begin{pgfscope}%
\pgfsetrectcap%
\pgfsetmiterjoin%
\pgfsetlinewidth{0.803000pt}%
\definecolor{currentstroke}{rgb}{0.000000,0.000000,0.000000}%
\pgfsetstrokecolor{currentstroke}%
\pgfsetdash{}{0pt}%
\pgfpathmoveto{\pgfqpoint{0.630330in}{0.548769in}}%
\pgfpathlineto{\pgfqpoint{3.727004in}{0.548769in}}%
\pgfusepath{stroke}%
\end{pgfscope}%
\begin{pgfscope}%
\pgfsetrectcap%
\pgfsetmiterjoin%
\pgfsetlinewidth{0.803000pt}%
\definecolor{currentstroke}{rgb}{0.000000,0.000000,0.000000}%
\pgfsetstrokecolor{currentstroke}%
\pgfsetdash{}{0pt}%
\pgfpathmoveto{\pgfqpoint{0.630330in}{2.301955in}}%
\pgfpathlineto{\pgfqpoint{3.727004in}{2.301955in}}%
\pgfusepath{stroke}%
\end{pgfscope}%
\begin{pgfscope}%
\pgfsetbuttcap%
\pgfsetmiterjoin%
\definecolor{currentfill}{rgb}{1.000000,1.000000,1.000000}%
\pgfsetfillcolor{currentfill}%
\pgfsetfillopacity{0.800000}%
\pgfsetlinewidth{1.003750pt}%
\definecolor{currentstroke}{rgb}{0.800000,0.800000,0.800000}%
\pgfsetstrokecolor{currentstroke}%
\pgfsetstrokeopacity{0.800000}%
\pgfsetdash{}{0pt}%
\pgfpathmoveto{\pgfqpoint{2.803974in}{0.618213in}}%
\pgfpathlineto{\pgfqpoint{3.629782in}{0.618213in}}%
\pgfpathquadraticcurveto{\pgfqpoint{3.657560in}{0.618213in}}{\pgfqpoint{3.657560in}{0.645991in}}%
\pgfpathlineto{\pgfqpoint{3.657560in}{1.406793in}}%
\pgfpathquadraticcurveto{\pgfqpoint{3.657560in}{1.434571in}}{\pgfqpoint{3.629782in}{1.434571in}}%
\pgfpathlineto{\pgfqpoint{2.803974in}{1.434571in}}%
\pgfpathquadraticcurveto{\pgfqpoint{2.776196in}{1.434571in}}{\pgfqpoint{2.776196in}{1.406793in}}%
\pgfpathlineto{\pgfqpoint{2.776196in}{0.645991in}}%
\pgfpathquadraticcurveto{\pgfqpoint{2.776196in}{0.618213in}}{\pgfqpoint{2.803974in}{0.618213in}}%
\pgfpathlineto{\pgfqpoint{2.803974in}{0.618213in}}%
\pgfpathclose%
\pgfusepath{stroke,fill}%
\end{pgfscope}%
\begin{pgfscope}%
\pgfsetrectcap%
\pgfsetroundjoin%
\pgfsetlinewidth{1.505625pt}%
\definecolor{currentstroke}{rgb}{0.121569,0.466667,0.705882}%
\pgfsetstrokecolor{currentstroke}%
\pgfsetdash{}{0pt}%
\pgfpathmoveto{\pgfqpoint{2.831751in}{1.330404in}}%
\pgfpathlineto{\pgfqpoint{2.970640in}{1.330404in}}%
\pgfpathlineto{\pgfqpoint{3.109529in}{1.330404in}}%
\pgfusepath{stroke}%
\end{pgfscope}%
\begin{pgfscope}%
\definecolor{textcolor}{rgb}{0.000000,0.000000,0.000000}%
\pgfsetstrokecolor{textcolor}%
\pgfsetfillcolor{textcolor}%
\pgftext[x=3.220640in,y=1.281793in,left,base]{\color{textcolor}\rmfamily\fontsize{10.000000}{12.000000}\selectfont \(\displaystyle N=1\)}%
\end{pgfscope}%
\begin{pgfscope}%
\pgfsetrectcap%
\pgfsetroundjoin%
\pgfsetlinewidth{1.505625pt}%
\definecolor{currentstroke}{rgb}{1.000000,0.498039,0.054902}%
\pgfsetstrokecolor{currentstroke}%
\pgfsetdash{}{0pt}%
\pgfpathmoveto{\pgfqpoint{2.831751in}{1.136732in}}%
\pgfpathlineto{\pgfqpoint{2.970640in}{1.136732in}}%
\pgfpathlineto{\pgfqpoint{3.109529in}{1.136732in}}%
\pgfusepath{stroke}%
\end{pgfscope}%
\begin{pgfscope}%
\definecolor{textcolor}{rgb}{0.000000,0.000000,0.000000}%
\pgfsetstrokecolor{textcolor}%
\pgfsetfillcolor{textcolor}%
\pgftext[x=3.220640in,y=1.088120in,left,base]{\color{textcolor}\rmfamily\fontsize{10.000000}{12.000000}\selectfont \(\displaystyle N=2\)}%
\end{pgfscope}%
\begin{pgfscope}%
\pgfsetrectcap%
\pgfsetroundjoin%
\pgfsetlinewidth{1.505625pt}%
\definecolor{currentstroke}{rgb}{0.172549,0.627451,0.172549}%
\pgfsetstrokecolor{currentstroke}%
\pgfsetdash{}{0pt}%
\pgfpathmoveto{\pgfqpoint{2.831751in}{0.943059in}}%
\pgfpathlineto{\pgfqpoint{2.970640in}{0.943059in}}%
\pgfpathlineto{\pgfqpoint{3.109529in}{0.943059in}}%
\pgfusepath{stroke}%
\end{pgfscope}%
\begin{pgfscope}%
\definecolor{textcolor}{rgb}{0.000000,0.000000,0.000000}%
\pgfsetstrokecolor{textcolor}%
\pgfsetfillcolor{textcolor}%
\pgftext[x=3.220640in,y=0.894448in,left,base]{\color{textcolor}\rmfamily\fontsize{10.000000}{12.000000}\selectfont \(\displaystyle N=3\)}%
\end{pgfscope}%
\begin{pgfscope}%
\pgfsetrectcap%
\pgfsetroundjoin%
\pgfsetlinewidth{1.505625pt}%
\definecolor{currentstroke}{rgb}{0.839216,0.152941,0.156863}%
\pgfsetstrokecolor{currentstroke}%
\pgfsetdash{}{0pt}%
\pgfpathmoveto{\pgfqpoint{2.831751in}{0.749386in}}%
\pgfpathlineto{\pgfqpoint{2.970640in}{0.749386in}}%
\pgfpathlineto{\pgfqpoint{3.109529in}{0.749386in}}%
\pgfusepath{stroke}%
\end{pgfscope}%
\begin{pgfscope}%
\definecolor{textcolor}{rgb}{0.000000,0.000000,0.000000}%
\pgfsetstrokecolor{textcolor}%
\pgfsetfillcolor{textcolor}%
\pgftext[x=3.220640in,y=0.700775in,left,base]{\color{textcolor}\rmfamily\fontsize{10.000000}{12.000000}\selectfont \(\displaystyle N=4\)}%
\end{pgfscope}%
\end{pgfpicture}%
\makeatother%
\endgroup%

    \caption{Die Tschebyscheff-Polynome füllen den erlaubten Bereich besser, und erhalten dadurch eine steilere Flanke im Sperrbereich.}
    \label{ellfiter:fig:chebychef}
\end{figure}


Die analytische Fortsetzung von \eqref{ellfilter:eq:chebychef_polynomials} über das Intervall $[-1,1]$ hinaus stimmt mit den Polynomen überein, wie es zu erwarten ist.
Die genauere Betrachtung wird uns dann helfen die elliptischen Filter besser zu verstehen.

Starten wir mit der Funktion, die als erstes auf $w$ angewendet wird, dem Arcuscosinus.
Die invertierte Funktion des Kosinus kann als definites Integral dargestellt werden:
\begin{align}
    \cos^{-1}(x)
    &=
    \int_{x}^{1}
    \frac{
        dz
    }{
        \sqrt{
            1-z^2
        }
    }\\
    &=
    \int_{0}^{x}
    \frac{
        -1
    }{
        \sqrt{
            1-z^2
        }
    }
    ~dz
    + \frac{\pi}{2}
\end{align}
Der Integrand oder auch die Ableitung
\begin{equation}
    \frac{
        -1
    }{
        \sqrt{
            1-z^2
        }
    }
\end{equation}
bestimmt dabei die Richtung, in der die Funktion verläuft.
Der reelle Arcuscosinus is bekanntlich nur für $|z| \leq 1$ definiert.
Hier bleibt der Wert unter der Wurzel positiv und das Integral liefert reelle Werte.
Doch wenn $|z|$ über 1 hinausgeht, wird der Term unter der Wurzel negativ.
Durch die Quadratwurzel entstehen für den Integranden zwei rein komplexe Lösungen.
Der Wert des Arcuscosinus verlässt also bei $z= \pm 1$ den reellen Zahlenstrahl und knickt in die komplexe Ebene ab.
Abbildung \ref{ellfilter:fig:arccos} zeigt den $\arccos$ in der komplexen Ebene.
\begin{figure}
    \centering
    \begin{tikzpicture}[>=stealth', auto, node distance=2cm, scale=1.2]

    \tikzstyle{zero} = [draw, circle, inner sep =0, minimum height=0.15cm]
    \tikzset{pole/.style={cross out, draw=black, minimum size=(0.15cm-\pgflinewidth), inner sep=0pt, outer sep=0pt}}

    \draw[gray, ->] (0,-2) -- (0,2) node[anchor=south]{$\mathrm{Im}~z$};
    \draw[gray, ->] (-5,0) -- (5,0) node[anchor=west]{$\mathrm{Re}~z$};

    \begin{scope}[xscale=0.6]

        \clip(-7.5,-2) rectangle (7.5,2);

        \draw[thick, ->, darkgreen] (0, 0) -- (0,1.5);
        \draw[thick, ->, orange] (1, 0) -- (0,0);
        \draw[thick, ->, red] (2, 0) -- (1,0);
        \draw[thick, ->, blue] (2,1.5) -- (2, 0);

        \foreach \i in {-2,...,1} {
            \begin{scope}[opacity=0.5, xshift=\i*4cm]
                \draw[->, orange] (-1, 0) -- (0,0);
                \draw[->, darkgreen] (0, 0) -- (0,1.5);
                \draw[->, darkgreen] (0, 0) -- (0,-1.5);
                \draw[->, orange] (1, 0) -- (0,0);
                \draw[->, red] (2, 0) -- (1,0);
                \draw[->, blue] (2,1.5) -- (2, 0);
                \draw[->, blue] (2,-1.5) -- (2, 0);
                \draw[->, red] (2, 0) -- (3,0);

                \node[zero] at (1,0) {};
                \node[zero] at (3,0) {};
            \end{scope}
        }

        \node[gray, anchor=north] at (-6,0) {$-3\pi$};
        \node[gray, anchor=north] at (-4,0) {$-2\pi$};
        \node[gray, anchor=north] at (-2,0) {$-\pi$};
        % \node[gray, anchor=north] at (0,0) {$0$};
        \node[gray, anchor=north] at (2,0) {$\pi$};
        \node[gray, anchor=north] at (4,0) {$2\pi$};
        \node[gray, anchor=north] at (6,0) {$3\pi$};

        \node[gray, anchor=east] at (0,-1.5) {$-\infty$};
        % \node[gray, anchor=south east] at (0, 0) {$0$};
        \node[gray, anchor=east] at (0, 1.5) {$\infty$};

    \end{scope}

    \begin{scope}[yshift=-2.5cm]

        \draw[gray, ->] (-5,0) -- (5,0) node[anchor=west]{$w$};

        \draw[thick, ->, blue]      (-4, 0) -- (-2, 0);
        \draw[thick, ->, red]       (-2, 0) -- (0, 0);
        \draw[thick, ->, orange]    (0, 0) -- (2, 0);
        \draw[thick, ->, darkgreen] (2, 0) -- (4, 0);

        \node[anchor=south] at (-4,0) {$-\infty$};
        \node[anchor=south] at (-2,0) {$-1$};
        \node[anchor=south] at (0,0) {$0$};
        \node[anchor=south] at (2,0) {$1$};
        \node[anchor=south] at (4,0) {$\infty$};

    \end{scope}


\end{tikzpicture}
    \caption{Die Funktion $z = \cos^{-1}(w)$ dargestellt in der komplexen ebene.}
    \label{ellfilter:fig:arccos}
\end{figure}
Wegen der Periodizität des Kosinus ist auch der Arcuscosinus $2\pi$-periodisch und es entstehen periodische Nullstellen.
% \begin{equation}
%     \frac{
%         1
%     }{
%         \sqrt{
%             1-z^2
%         }
%     }
%     \in \mathbb{R}
%     \quad
%     \forall
%     \quad
%     -1  \leq z \leq 1
% \end{equation}
% \begin{equation}
%     \frac{
%         1
%     }{
%         \sqrt{
%             1-z^2
%         }
%     }
%     = i \xi \quad | \quad \xi \in \mathbb{R}
%     \quad
%     \forall
%     \quad
%     z \leq -1 \cup z \geq 1
% \end{equation}

Die Tschebyscheff-Polynome skalieren diese Nullstellen mit dem Ordnungsfaktor $N$, wie dargestellt in Abbildung \ref{ellfilter:fig:arccos2}.
\begin{figure}
    \centering
    \begin{tikzpicture}[>=stealth', auto, node distance=2cm, scale=1.2]

    \tikzstyle{zero} = [draw, circle, inner sep =0, minimum height=0.15cm]
    \tikzset{pole/.style={cross out, draw=black, minimum size=(0.15cm-\pgflinewidth), inner sep=0pt, outer sep=0pt}}

    \begin{scope}[xscale=0.5]

        \draw[gray, ->] (0,-2) -- (0,2) node[anchor=south]{$\mathrm{Im}~z_1$};
        \draw[gray, ->] (-10,0) -- (10,0) node[anchor=west]{$\mathrm{Re}~z_1$};

        \begin{scope}

            \draw[>->, line width=0.05, thick, blue]   (2, 1.5) -- (2,0.05)  -- node[anchor=south, pos=0.5]{$N=1$} (0.1,0.05) -- (0.1,1.5);
            \draw[>->, line width=0.05, thick, orange] (4, 1.5) -- (4,0)     -- node[anchor=south, pos=0.25]{$N=2$} (0,0) -- (0,1.5);
            \draw[>->, line width=0.05, thick, red]    (6, 1.5) node[anchor=north west]{$-\infty$} -- (6,-0.05) node[anchor=west]{$-1$} -- node[anchor=north]{$0$} node[anchor=south, pos=0.1666]{$N=3$} (-0.1,-0.05) node[anchor=east]{$1$}  -- (-0.1,1.5) node[anchor=north east]{$\infty$};

            \node[zero] at (-7,0) {};
            \node[zero] at (-5,0) {};
            \node[zero] at (-3,0) {};
            \node[zero] at (-1,0) {};
            \node[zero] at (1,0) {};
            \node[zero] at (3,0) {};
            \node[zero] at (5,0) {};
            \node[zero] at (7,0) {};

        \end{scope}

        \node[gray, anchor=north] at (-8,0) {$-4\pi$};
        \node[gray, anchor=north] at (-6,0) {$-3\pi$};
        \node[gray, anchor=north] at (-4,0) {$-2\pi$};
        \node[gray, anchor=north] at (-2,0) {$-\pi$};
        \node[gray, anchor=north] at (2,0) {$\pi$};
        \node[gray, anchor=north] at (4,0) {$2\pi$};
        \node[gray, anchor=north] at (6,0) {$3\pi$};
        \node[gray, anchor=north] at (8,0) {$4\pi$};


        \node[gray, anchor=east] at (0,-1.5) {$-\infty$};
        \node[gray, anchor=east] at (0, 1.5) {$\infty$};

    \end{scope}

    \node[zero] at (4,2) (n) {};
    \node[anchor=west] at (n.east) {Zero};

\end{tikzpicture}
    \caption{
        $z_1=N \cos^{-1}(w)$-Ebene der Tschebyscheff-Funktion.
        Die eingefärbten Pfade sind Verläufe von $w~\forall~[-\infty, \infty]$ für verschiedene Ordnungen $N$.
        Je grösser die Ordnung $N$ gewählt wird, desto mehr Nullstellen werden passiert.
    }
    \label{ellfilter:fig:arccos2}
\end{figure}
Somit passert $\cos( N~\cos^{-1}(w))$ im Intervall $[-1, 1]$ $N$ Nullstellen.
Durch die spezielle Anordnung der Nullstellen hat die Funktion Equirippel-Verhalten und ist dennoch ein Polynom, was sich perfekt für linear Filter eignet.

\section{Jacobische elliptische Funktionen}

%TODO $z$ or $u$ for parameter?

Für das elliptische Filter wird statt der, für das Tschebyscheff-Filter benutzen Kreis-Trigonometrie die elliptischen Funktionen gebraucht.
Der Begriff elliptische Funktion wird für sehr viele Funktionen gebraucht, daher ist es hier wichtig zu erwähnen, dass es ausschliesslich um die Jacobischen elliptischen Funktionen geht.

Im Wesentlichen erweitern die Jacobi elliptischen Funktionen die trigonometrische Funktionen für Ellipsen.
Zum Beispiel gibt es analog zum Sinus den elliptischen $\sn(z, k)$.
Im Gegensatz zum den trigonometrischen Funktionen haben die elliptischen Funktionen zwei parameter.
Zum einen gibt es den \textit{elliptische Modul} $k$, der die Exzentrizität der Ellipse parametrisiert.
Zum andern das Winkelargument $z$.
Im Kreis ist der Radius für alle Winkel konstant, bei Ellipsen ändert sich das.
Dies hat zur Folge, dass bei einer Ellipse die Kreisbodenstrecke nicht linear zum Winkel verläuft.
Darum kann hier nicht der gewohnte Winkel verwendet werden.
Das Winkelargument $z$ kann durch das elliptische Integral erster Art
\begin{equation}
    z
    =
    F(\phi, k)
    =
    \int_{0}^{\phi}
    \frac{
        d\theta
    }{
        \sqrt{
            1-k^2 \sin^2 \theta
        }
    }
    =
    \int_{0}^{\phi}
    \frac{
        dt
    }{
        \sqrt{
            (1-t^2)(1-k^2 t^2)
        }
    } %TODO which is right? are both functions from phi?
\end{equation}
mit dem Winkel $\phi$ in Verbindung liegt.

Dabei wird das vollständige und unvollständige Elliptische integral unterschieden.
Beim vollständigen Integral
\begin{equation}
    K(k)
    =
    \int_{0}^{\pi / 2}
    \frac{
        d\theta
    }{
        \sqrt{
            1-k^2 \sin^2 \theta
        }
    }
\end{equation}
wird über ein viertel Ellipsenbogen integriert also bis $\phi=\pi/2$ und liefert das Winkelargument für eine Vierteldrehung.
Die Zahl wird oft auch abgekürzt mit $K = K(k)$ und ist für das elliptische Filter sehr relevant.
Alle elliptishen Funktionen sind somit $4K$-periodisch.

Neben dem $\sn$ gibt es zwei weitere basis-elliptische Funktionen $\cn$ und $\dn$.
Dazu kommen noch weitere abgeleitete Funktionen, die durch Quotienten und Kehrwerte dieser Funktionen zustande kommen.
Insgesamt sind es die zwölf Funktionen
\begin{equation*}
    \sn \quad
    \ns \quad
    \scelliptic \quad
    \sd \quad
    \cn \quad
    \nc \quad
    \cs \quad
    \cd \quad
    \dn \quad
    \nd \quad
    \ds \quad
    \dc.
\end{equation*}

Die Jacobischen elliptischen Funktionen können mit der inversen Funktion des kompletten elliptischen Integrals erster Art
\begin{equation}
    \phi = F^{-1}(z, k)
\end{equation}
definiert werden. Dabei ist zu beachten dass nur das $z$ Argument der Funktion invertiert wird, also
\begin{equation}
    z = F(\phi, k)
    \Leftrightarrow
    \phi = F^{-1}(z, k).
\end{equation}
Mithilfe von $F^{-1}$ kann zum Beispiel $sn^{-1}$ mit dem Elliptischen integral dargestellt werden:
\begin{equation}
    \sin(\phi)
    =
    \sin \left( F^{-1}(z, k) \right)
    =
    \sn(z, k)
    =
    w
\end{equation}

\begin{equation}
    \phi
    =
     F^{-1}(z, k)
     =
     \sin^{-1} \big( \sn (z, k ) \big)
     =
    \sin^{-1} ( w )
\end{equation}

\begin{equation}
    F(\phi, k)
    =
    z
    =
    F( \sin^{-1} \big( \sn (z, k ) \big) , k)
    =
    F( \sin^{-1} ( w ), k)
\end{equation}

\begin{equation}
    \sn^{-1}(w, k)
    =
    F(\phi, k),
    \quad
    \phi = \sin^{-1}(w)
\end{equation}

\begin{align}
    \sn^{-1}(w, k)
        & =
    \int_{0}^{\phi}
    \frac{
        d\theta
    }{
        \sqrt{
            1-k^2 \sin^2 \theta
        }
    },
    \quad
    \phi = \sin^{-1}(w)
    \\
        & =
    \int_{0}^{w}
    \frac{
        dt
    }{
        \sqrt{
            (1-t^2)(1-k^2 t^2)
        }
    }
\end{align}

Beim $\cos^{-1}(x)$ haben wir gesehen, dass die analytische Fortsetzung bei $x < -1$ und $x > 1$ rechtwinklig in die Komplexen zahlen wandert.
Wenn man das gleiche mit $\sn^{-1}(w, k)$ macht, erkennt man zwei interessante Stellen.
Die erste ist die gleiche wie beim $\cos^{-1}(x)$ nämlich bei $t = \pm 1$.
Der erste Term unter der Wurzel wird dann negativ, während der zweite noch positiv ist, da $k \leq 1$.
\begin{equation}
    \frac{
        1
    }{
        \sqrt{
            (1-t^2)(1-k^2 t^2)
        }
    }
    \in \mathbb{R}
    \quad \forall \quad
    -1 \leq t \leq 1
\end{equation}
Die zweite stelle passiert wenn beide Faktoren unter der Wurzel negativ werden, was bei $t = 1/k$ der Fall ist.




Funktion in relle und komplexe Richtung periodisch

In der reellen Richtung ist sie $4K(k)$-periodisch und in der imaginären Richtung $4K^\prime(k)$-periodisch.



%TODO sn^{-1} grafik

\begin{figure}
    \centering
    \begin{tikzpicture}[>=stealth', auto, node distance=2cm, scale=1.2]

    \tikzstyle{zero} = [draw, circle, inner sep =0, minimum height=0.15cm]

    \tikzset{pole/.style={cross out, draw=black, minimum size=(0.15cm-\pgflinewidth), inner sep=0pt, outer sep=0pt}}

    \begin{scope}[xscale=0.9, yscale=1.8]

        \draw[gray, ->] (0,-1.5) -- (0,1.5) node[anchor=south]{$\mathrm{Im}~z$};
        \draw[gray, ->] (-5,0) -- (5,0) node[anchor=west]{$\mathrm{Re}~z$};

        \begin{scope}

            \clip(-4.5,-1.25) rectangle (4.5,1.25);

            \fill[yellow!30] (0,0) rectangle (1, 0.5);

            \begin{scope}[xshift=-1cm]

                \foreach \i in {-2,...,2} {
                    \foreach \j in {-2,...,1} {
                        \begin{scope}[xshift=\i*4cm, yshift=\j*1cm]
                            \draw[<-, blue!50] (0, 0) -- (0,0.5);
                            \draw[<-, cyan!50] (1, 0) -- (0,0);
                            \draw[<-, darkgreen!50] (2, 0) -- (1,0);
                            \draw[<-, orange!50] (2,0.5) -- (2, 0);
                            \draw[<-, red!50] (1, 0.5) -- (2,0.5);
                            \draw[<-, purple!50] (0, 0.5) -- (1,0.5);
                            \draw[<-, blue!50] (0,1) -- (0,0.5);
                            \draw[<-, orange!50] (2,0.5) -- (2, 1);
                            \draw[<-, red!50] (3, 0.5) -- (2,0.5);
                            \draw[<-, purple!50] (4, 0.5) -- (3,0.5);
                            \draw[<-, darkgreen!50] (2, 0) -- (3,0);
                            \draw[<-, cyan!50] (3, 0) -- (4,0);
                        \end{scope}
                    }
                }

                % \pause
                \draw[ultra thick, <-, darkgreen] (2, 0) -- (1,0);
                % \pause
                \draw[ultra thick, <-, orange] (2,0.5) -- (2, 0);
                % \pause
                \draw[ultra thick, <-, red] (1, 0.5) -- (2,0.5);
                % \pause
                \draw[ultra thick, <-, blue] (0, 0) -- (0,0.5);
                \draw[ultra thick, <-, purple] (0, 0.5) -- (1,0.5);
                \draw[ultra thick, <-, cyan] (1, 0) -- (0,0);
                % \pause


                \foreach \i in {-2,...,2} {
                    \foreach \j in {-2,...,1} {
                        \begin{scope}[xshift=\i*4cm, yshift=\j*1cm]
                            \node[zero] at ( 1, 0) {};
                            \node[zero] at ( 3, 0) {};
                            \node[pole] at ( 1,0.5) {};
                            \node[pole] at ( 3,0.5) {};
                        \end{scope}
                    }
                }

            \end{scope}

        \end{scope}

        \draw[gray] ( 1,0) +(0,0.1) -- +(0, -0.1) node[inner sep=0, anchor=north] {\small $K$};
        \draw[gray]  (0, 0.5) +(0.1, 0) -- +(-0.1, 0) node[inner sep=0, anchor=east]{\small $jK^\prime$};

    \end{scope}

    \node[zero] at (4,3) (n) {};
    \node[anchor=west] at (n.east) {Zero};
    \node[pole, below=0.25cm of n] (n) {};
    \node[anchor=west] at (n.east) {Pole};

    \begin{scope}[yshift=-4cm, xscale=0.75]

        \draw[gray, ->] (-6,0) -- (6,0) node[anchor=west]{$w$};

        \draw[ultra thick, ->, purple] (-5, 0) -- (-3, 0);
        \draw[ultra thick, ->, blue]      (-3, 0) -- (-2, 0);
        \draw[ultra thick, ->, cyan]       (-2, 0) -- (0, 0);
        \draw[ultra thick, ->, darkgreen]    (0, 0) -- (2, 0);
        \draw[ultra thick, ->, orange] (2, 0) -- (3, 0);
        \draw[ultra thick, ->, red] (3, 0) -- (5, 0);

        \node[anchor=south] at (-5,0) {$-\infty$};
        \node[anchor=south] at (-3,0) {$-1/k$};
        \node[anchor=south] at (-2,0) {$-1$};
        \node[anchor=south] at (0,0) {$0$};
        \node[anchor=south] at (2,0) {$1$};
        \node[anchor=south] at (3,0) {$1/k$};
        \node[anchor=south] at (5,0) {$\infty$};

    \end{scope}


\end{tikzpicture}
    \caption{
        $z$-Ebene der Funktion $z = \sn^{-1}(w, k)$.
        Die Funktion ist in der realen Achse $4K$-periodisch und in der imaginären Achse $2jK^\prime$-periodisch.
    }
    % \label{ellfilter:fig:cd2}
\end{figure}

\section{Elliptische rationale Funktionen}

Kommen wir nun zum eigentlichen Teil dieses Papers, den elliptischen rationalen Funktionen
\begin{align}
    R_N(\xi, w) &= \cd \left(N~f_1(\xi)~\cd^{-1}(w, 1/\xi), f_2(\xi)\right) \\
                &= \cd \left(N~\frac{K_1}{K}~\cd^{-1}(w, k), k_1)\right) , \quad k= 1/\xi, k_1 = 1/f(\xi) \\
                &= \cd \left(N~K_1~z , k_1 \right), \quad w= \cd(z K, k)
\end{align}


sieht ähnlich aus wie die trigonometrische Darstellung der Tschebyschef-Polynome \eqref{ellfilter:eq:chebychef_polynomials}
Anstelle vom Kosinus kommt hier die $\cd$-Funktion zum Einsatz.
Die Ordnungszahl $N$ kommt auch als Faktor for.
Zusätzlich werden noch zwei verschiedene elliptische Module $k$ und $k_1$ gebraucht.



Sinus entspricht $\sn$

Damit die Nullstellen an ähnlichen Positionen zu liegen kommen wie bei den Tschebyscheff-Polynomen, muss die $\cd$-Funktion gewählt werden.

Die $\cd^{-1}(w, k)$-Funktion ist um $K$ verschoben zur $\sn^{-1}(w, k)$-Funktion, wie ersichtlich in Abbildung \ref{ellfilter:fig:cd}.
\begin{figure}
    \centering
    \begin{tikzpicture}[>=stealth', auto, node distance=2cm, scale=1.2, thick]

    \tikzstyle{zero} = [draw, circle, inner sep =0, minimum height=0.15cm]

    \tikzset{pole/.style={cross out, draw=black, minimum size=(0.15cm-\pgflinewidth), inner sep=0pt, outer sep=0pt}}

    \begin{scope}[xscale=0.9, yscale=1.8]

        \draw[gray, ->] (0,-1.5) -- (0,1.5) node[anchor=south]{$\mathrm{Im}~z$};
        \draw[gray, ->] (-5,0) -- (5,0) node[anchor=west]{$\mathrm{Re}~z$};


        \begin{scope}[xshift=0cm]

            \clip(-4.5,-1.25) rectangle (4.5,1.25);

            \fill[yellow!30] (0,0) rectangle (1, 0.5);

            \foreach \i in {-2,...,1} {
                \foreach \j in {-2,...,1} {
                    \begin{scope}[xshift=\i*4cm, yshift=\j*1cm]
                        \draw[->, thick, orange!50] (0, 0) -- (0,0.5);
                        \draw[->, thick, darkgreen!50] (1, 0) -- (0,0);
                        \draw[->, thick, cyan!50] (2, 0) -- (1,0);
                        \draw[->, thick, blue!50] (2,0.5) -- (2, 0);
                        \draw[->, thick, purple!50] (1, 0.5) -- (2,0.5);
                        \draw[->, thick, red!50] (0, 0.5) -- (1,0.5);
                        \draw[->, thick, orange!50] (0,1) -- (0,0.5);
                        \draw[->, thick, blue!50] (2,0.5) -- (2, 1);
                        \draw[->, thick, purple!50] (3, 0.5) -- (2,0.5);
                        \draw[->, thick, red!50] (4, 0.5) -- (3,0.5);
                        \draw[->, thick, cyan!50] (2, 0) -- (3,0);
                        \draw[->, thick, darkgreen!50] (3, 0) -- (4,0);
                    \end{scope}
                }
            }

            \draw[ultra thick, ->, orange] (0, 0) -- (0,0.5);
            \draw[ultra thick, ->, darkgreen] (1, 0) -- (0,0);
            \draw[ultra thick, ->, cyan] (2, 0) -- (1,0);
            \draw[ultra thick, ->, blue] (2,0.5) -- (2, 0);
            \draw[ultra thick, ->, purple] (1, 0.5) -- (2,0.5);
            \draw[ultra thick, ->, red] (0, 0.5) -- (1,0.5);

            \foreach \i in {-2,...,1} {
                \foreach \j in {-2,...,1} {
                    \begin{scope}[xshift=\i*4cm, yshift=\j*1cm]
                        \node[zero] at ( 1, 0) {};
                        \node[zero] at ( 3, 0) {};
                        \node[pole] at ( 1,0.5) {};
                        \node[pole] at ( 3,0.5) {};

                    \end{scope}
                }
            }

        \end{scope}

        \draw[gray] ( 1,0) +(0,0.05) -- +(0, -0.05) node[inner sep=0, anchor=north west] {\small $K$};
        \draw[gray]  (0, 0.5) +(0.1, 0) -- +(-0.1, 0) node[inner sep=0, anchor=south east]{\small $jK^\prime$};

    \end{scope}


    \node[zero] at (4,3) (n) {};
    \node[anchor=west] at (n.east) {Nullstelle};
    \node[pole, below=0.25cm of n] (n) {};
    \node[anchor=west] at (n.east) {Polstelle};

    \begin{scope}[yshift=-4cm, xscale=0.75]

        \draw[gray, ->] (-6,0) -- (6,0) node[anchor=west]{$w$};

        \draw[ultra thick, ->, purple] (-5, 0) -- (-3, 0);
        \draw[ultra thick, ->, blue]      (-3, 0) -- (-2, 0);
        \draw[ultra thick, ->, cyan]       (-2, 0) -- (0, 0);
        \draw[ultra thick, ->, darkgreen]    (0, 0) -- (2, 0);
        \draw[ultra thick, ->, orange] (2, 0) -- (3, 0);
        \draw[ultra thick, ->, red] (3, 0) -- (5, 0);

        \node[anchor=south] at (-5,0) {$-\infty$};
        \node[anchor=south] at (-3,0) {$-1/k$};
        \node[anchor=south] at (-2,0) {$-1$};
        \node[anchor=south] at (0,0) {$0$};
        \node[anchor=south] at (2,0) {$1$};
        \node[anchor=south] at (3,0) {$1/k$};
        \node[anchor=south] at (5,0) {$\infty$};

    \end{scope}

\end{tikzpicture}
    \caption{
        $z$-Ebene der Funktion $z = \sn^{-1}(w, k)$.
        Die Funktion ist in der realen Achse $4K$-periodisch und in der imaginären Achse $2jK^\prime$-periodisch.
    }
    \label{ellfilter:fig:cd}
\end{figure}
Auffallend ist, dass sich alle Nullstellen und Polstellen um $K$ verschoben haben.

Durch das Konzept vom fundamentalen Rechteck, siehe Abbildung \ref{ellfilter:fig:fundamental_rectangle} können für alle inversen Jaccobi elliptischen Funktionen die Positionen der Null- und Polstellen anhand eines Diagramms ermittelt werden.
Der erste Buchstabe bestimmt die Position der Nullstelle und der zweite Buchstabe die Polstelle.
\begin{figure}
    \centering
    \begin{tikzpicture}[>=stealth', auto, node distance=2cm, scale=1.2]

    \tikzstyle{zero} = [draw, circle, inner sep =0, minimum height=0.15cm]

    \tikzset{pole/.style={cross out, draw=black, minimum size=(0.15cm-\pgflinewidth), inner sep=0pt, outer sep=0pt}}

    \begin{scope}[xscale=2, yscale=2]

        \draw[gray, ->] (0,-0.25) -- (0,1.25) node[anchor=south]{$\mathrm{Im}~z$};
        \draw[gray, ->] (-0.25,0) -- (1.5,0) node[anchor=west]{$\mathrm{Re}~z$};

        \draw[gray] ( 1,0) +(0,0.05) -- +(0, -0.05) node[inner sep=0, anchor=north] {\small $K$};

        \draw[gray]  (0, 1) +(0.05, 0) -- +(-0.05, 0) node[inner sep=0, anchor=east]{\small $jK^\prime$};

        \fill[yellow!50] (0,0) rectangle (1, 1);

        \node[anchor=south east] at ( 1,0) {$c$};
        \node[anchor=north east] at ( 1,1) {$d$};
        \node[anchor=north west] at ( 0,1) {$n$};
        \node[anchor=south west] at ( 0,0) {$s$};

    \end{scope}


\end{tikzpicture}
    \caption{
        Fundamentales Rechteck der inversen Jaccobi elliptischen Funktionen.
    }
    \label{ellfilter:fig:fundamental_rectangle}
\end{figure}

Auffallend an der $w = \sn(z, k)$-Funktion ist, dass sich $w$ auf der reellen Achse wie der Kosinus immer zwischen $-1$ und $1$ bewegt, während bei $\mathrm{Im(z) = K^\prime}$ die Werte zwischen $\pm 1/k$ und $\pm \infty$ verlaufen.
Die Funktion hat also Equirippel-Verhalten um $w=0$ und um $w=\pm \infty$.
Falls es möglich ist diese Werte abzufahren im Sti der Tschebyscheff-Polynome, kann ein Filter gebaut werden, dass Equirippel-Verhalten im Durchlass- und Sperrbereich aufweist.



Analog zu Abbildung \ref{ellfilter:fig:arccos2} können wir auch bei den elliptisch rationalen Funktionen die komplexe $z$-Ebene betrachten, wie ersichtlich in Abbildung \ref{ellfilter:fig:cd2}, um die besser zu verstehen.
\begin{figure}
    \centering
    \begin{tikzpicture}[>=stealth', auto, node distance=2cm, scale=1.2]

    \tikzstyle{zero} = [draw, circle, inner sep =0, minimum height=0.15cm]
    \tikzstyle{dot} = [fill, circle, inner sep =0, minimum height=0.1cm]

    \tikzset{pole/.style={cross out, draw=black, minimum size=(0.15cm-\pgflinewidth), inner sep=0pt, outer sep=0pt}}

    \begin{scope}[xscale=1.25, yscale=3.5]

        \draw[gray, ->] (0,-0.55) -- (0,1.05) node[anchor=south]{$\mathrm{Im}~z_1$};
        \draw[gray, ->] (-1.5,0) -- (6,0) node[anchor=west]{$\mathrm{Re}~z_1$};

        \draw[gray] ( 1,0) +(0,0.05) -- +(0, -0.05) node[inner sep=0, anchor=north] {\small $K_1$};
        \draw[gray] ( 5,0) +(0,0.05) -- +(0, -0.05) node[inner sep=0, anchor=north] {\small $5K_1$};
        \draw[gray]  (0, 0.5) +(0.1, 0) -- +(-0.1, 0) node[inner sep=0, anchor=east]{\small $jK^\prime_1$};

        \begin{scope}

            \clip(-1.5,-0.75) rectangle (6.8,1.25);

            % \draw[>->, line width=0.05, thick, blue]   (1, 0.45) -- (2, 0.45) -- (2, 0.05) -- ( 0.1, 0.05) -- ( 0.1,0.45) -- (1, 0.45);
            % \draw[>->, line width=0.05, thick, orange] (2, 0.5 ) -- (4, 0.5 ) -- (4, 0   ) -- ( 0  , 0   ) -- ( 0  ,0.5 ) -- (2, 0.5 );
            % \draw[>->, line width=0.05, thick, red]    (3, 0.55) -- (6, 0.55) -- (6,-0.05) -- (-0.1,-0.05) -- (-0.1,0.55) -- (3, 0.55);
            % \node[blue] at (1, 0.25) {$N=1$};
            % \node[orange] at (3, 0.25) {$N=2$};
            % \node[red] at (5, 0.25) {$N=3$};



            % \draw[line width=0.1cm, fill, red!50] (0,0) rectangle (3, 0.5);
            % \draw[line width=0.05cm, fill, orange!50] (0,0) rectangle (2, 0.5);
            % \fill[yellow!50] (0,0) rectangle (1, 0.5);
            % \node[] at (0.5, 0.25) {\small $N=1$};
            % \node[] at (1.5, 0.25) {\small $N=2$};
            % \node[] at (2.5, 0.25) {\small $N=3$};

            \fill[orange!30] (0,0) rectangle (5, 0.5);
            % \fill[yellow!30] (0,0) rectangle (1, 0.1);
            \node[] at (2.5, 0.25) {\small $N=5$};


            \draw[decorate,decoration={brace,amplitude=3pt,mirror}, yshift=0.05cm]
                (5,0.5) node(t_k_unten){} -- node[above, yshift=0.1cm]{$NK_1$}
                (0,0.5) node(t_k_opt_unten){};

            \draw[decorate,decoration={brace,amplitude=3pt,mirror}, xshift=0.1cm]
                (5,0) node(t_k_unten){} -- node[right, xshift=0.1cm]{$K^\prime \frac{K_1N}{K} = K^\prime_1$}
                (5,0.5) node(t_k_opt_unten){};


            \draw[ultra thick, ->, darkgreen] (5, 0) -- node[yshift=-0.5cm]{Durchlassbereich} (0,0);
            \draw[ultra thick, ->, orange] (-0, 0) --  node[align=center]{Übergangs-\\berech} (0,0.5);
            \draw[ultra thick, ->, red] (0,0.5) -- node[align=center, yshift=0.7cm]{Sperrbereich} (5, 0.5);

            \draw (4,0  )  node[dot]{} node[anchor=south]      {\small $1$};
            \draw (2,0  )  node[dot]{} node[anchor=south]      {\small $-1$};
            \draw (0,0  )  node[dot]{} node[anchor=south west] {\small $1$};
            \draw (0,0.5)  node[dot]{} node[anchor=north west] {\small $1/k$};
            \draw (2,0.5)  node[dot]{} node[anchor=north]      {\small $-1/k$};
            \draw (4,0.5)  node[dot]{} node[anchor=north]      {\small $1/k$};

            \foreach \i in {-2,...,1} {
                \foreach \j in {-2,...,1} {
                    \begin{scope}[xshift=\i*4cm, yshift=\j*1cm]

                        \node[zero] at ( 1, 0) {};
                        \node[zero] at ( 3, 0) {};
                        \node[pole] at ( 1,0.5) {};
                        \node[pole] at ( 3,0.5) {};

                    \end{scope}
                }
            }

        \end{scope}

    \end{scope}

\end{tikzpicture}
    \caption{
        $z_1$-Ebene der elliptischen rationalen Funktionen.
        Je grösser die Ordnung $N$ gewählt wird, desto mehr Nullstellen passiert.
    }
    \label{ellfilter:fig:cd2}
\end{figure}
% Da die $\cd^{-1}$-Funktion 



\begin{figure}
    \centering
    %% Creator: Matplotlib, PGF backend
%%
%% To include the figure in your LaTeX document, write
%%   \input{<filename>.pgf}
%%
%% Make sure the required packages are loaded in your preamble
%%   \usepackage{pgf}
%%
%% Also ensure that all the required font packages are loaded; for instance,
%% the lmodern package is sometimes necessary when using math font.
%%   \usepackage{lmodern}
%%
%% Figures using additional raster images can only be included by \input if
%% they are in the same directory as the main LaTeX file. For loading figures
%% from other directories you can use the `import` package
%%   \usepackage{import}
%%
%% and then include the figures with
%%   \import{<path to file>}{<filename>.pgf}
%%
%% Matplotlib used the following preamble
%%
\begingroup%
\makeatletter%
\begin{pgfpicture}%
\pgfpathrectangle{\pgfpointorigin}{\pgfqpoint{4.000000in}{2.500000in}}%
\pgfusepath{use as bounding box, clip}%
\begin{pgfscope}%
\pgfsetbuttcap%
\pgfsetmiterjoin%
\pgfsetlinewidth{0.000000pt}%
\definecolor{currentstroke}{rgb}{1.000000,1.000000,1.000000}%
\pgfsetstrokecolor{currentstroke}%
\pgfsetstrokeopacity{0.000000}%
\pgfsetdash{}{0pt}%
\pgfpathmoveto{\pgfqpoint{0.000000in}{0.000000in}}%
\pgfpathlineto{\pgfqpoint{4.000000in}{0.000000in}}%
\pgfpathlineto{\pgfqpoint{4.000000in}{2.500000in}}%
\pgfpathlineto{\pgfqpoint{0.000000in}{2.500000in}}%
\pgfpathlineto{\pgfqpoint{0.000000in}{0.000000in}}%
\pgfpathclose%
\pgfusepath{}%
\end{pgfscope}%
\begin{pgfscope}%
\pgfsetbuttcap%
\pgfsetmiterjoin%
\definecolor{currentfill}{rgb}{1.000000,1.000000,1.000000}%
\pgfsetfillcolor{currentfill}%
\pgfsetlinewidth{0.000000pt}%
\definecolor{currentstroke}{rgb}{0.000000,0.000000,0.000000}%
\pgfsetstrokecolor{currentstroke}%
\pgfsetstrokeopacity{0.000000}%
\pgfsetdash{}{0pt}%
\pgfpathmoveto{\pgfqpoint{0.733531in}{0.548769in}}%
\pgfpathlineto{\pgfqpoint{3.761597in}{0.548769in}}%
\pgfpathlineto{\pgfqpoint{3.761597in}{2.301955in}}%
\pgfpathlineto{\pgfqpoint{0.733531in}{2.301955in}}%
\pgfpathlineto{\pgfqpoint{0.733531in}{0.548769in}}%
\pgfpathclose%
\pgfusepath{fill}%
\end{pgfscope}%
\begin{pgfscope}%
\pgfpathrectangle{\pgfqpoint{0.733531in}{0.548769in}}{\pgfqpoint{3.028066in}{1.753186in}}%
\pgfusepath{clip}%
\pgfsetbuttcap%
\pgfsetmiterjoin%
\definecolor{currentfill}{rgb}{0.000000,0.501961,0.000000}%
\pgfsetfillcolor{currentfill}%
\pgfsetfillopacity{0.200000}%
\pgfsetlinewidth{0.000000pt}%
\definecolor{currentstroke}{rgb}{0.000000,0.000000,0.000000}%
\pgfsetstrokecolor{currentstroke}%
\pgfsetstrokeopacity{0.200000}%
\pgfsetdash{}{0pt}%
\pgfpathmoveto{\pgfqpoint{0.733531in}{-174.068564in}}%
\pgfpathlineto{\pgfqpoint{2.247564in}{-174.068564in}}%
\pgfpathlineto{\pgfqpoint{2.247564in}{1.250043in}}%
\pgfpathlineto{\pgfqpoint{0.733531in}{1.250043in}}%
\pgfpathlineto{\pgfqpoint{0.733531in}{-174.068564in}}%
\pgfpathclose%
\pgfusepath{fill}%
\end{pgfscope}%
\begin{pgfscope}%
\pgfpathrectangle{\pgfqpoint{0.733531in}{0.548769in}}{\pgfqpoint{3.028066in}{1.753186in}}%
\pgfusepath{clip}%
\pgfsetbuttcap%
\pgfsetmiterjoin%
\definecolor{currentfill}{rgb}{1.000000,0.647059,0.000000}%
\pgfsetfillcolor{currentfill}%
\pgfsetfillopacity{0.200000}%
\pgfsetlinewidth{0.000000pt}%
\definecolor{currentstroke}{rgb}{0.000000,0.000000,0.000000}%
\pgfsetstrokecolor{currentstroke}%
\pgfsetstrokeopacity{0.200000}%
\pgfsetdash{}{0pt}%
\pgfpathmoveto{\pgfqpoint{2.247564in}{1.250043in}}%
\pgfpathlineto{\pgfqpoint{2.262583in}{1.250043in}}%
\pgfpathlineto{\pgfqpoint{2.262583in}{1.600680in}}%
\pgfpathlineto{\pgfqpoint{2.247564in}{1.600680in}}%
\pgfpathlineto{\pgfqpoint{2.247564in}{1.250043in}}%
\pgfpathclose%
\pgfusepath{fill}%
\end{pgfscope}%
\begin{pgfscope}%
\pgfpathrectangle{\pgfqpoint{0.733531in}{0.548769in}}{\pgfqpoint{3.028066in}{1.753186in}}%
\pgfusepath{clip}%
\pgfsetbuttcap%
\pgfsetmiterjoin%
\definecolor{currentfill}{rgb}{1.000000,0.000000,0.000000}%
\pgfsetfillcolor{currentfill}%
\pgfsetfillopacity{0.200000}%
\pgfsetlinewidth{0.000000pt}%
\definecolor{currentstroke}{rgb}{0.000000,0.000000,0.000000}%
\pgfsetstrokecolor{currentstroke}%
\pgfsetstrokeopacity{0.200000}%
\pgfsetdash{}{0pt}%
\pgfpathmoveto{\pgfqpoint{2.262583in}{1.600680in}}%
\pgfpathlineto{\pgfqpoint{3.776616in}{1.600680in}}%
\pgfpathlineto{\pgfqpoint{3.776616in}{2.301962in}}%
\pgfpathlineto{\pgfqpoint{2.262583in}{2.301962in}}%
\pgfpathlineto{\pgfqpoint{2.262583in}{1.600680in}}%
\pgfpathclose%
\pgfusepath{fill}%
\end{pgfscope}%
\begin{pgfscope}%
\pgfpathrectangle{\pgfqpoint{0.733531in}{0.548769in}}{\pgfqpoint{3.028066in}{1.753186in}}%
\pgfusepath{clip}%
\pgfsetrectcap%
\pgfsetroundjoin%
\pgfsetlinewidth{0.803000pt}%
\definecolor{currentstroke}{rgb}{0.690196,0.690196,0.690196}%
\pgfsetstrokecolor{currentstroke}%
\pgfsetdash{}{0pt}%
\pgfpathmoveto{\pgfqpoint{0.733531in}{0.548769in}}%
\pgfpathlineto{\pgfqpoint{0.733531in}{2.301955in}}%
\pgfusepath{stroke}%
\end{pgfscope}%
\begin{pgfscope}%
\pgfsetbuttcap%
\pgfsetroundjoin%
\definecolor{currentfill}{rgb}{0.000000,0.000000,0.000000}%
\pgfsetfillcolor{currentfill}%
\pgfsetlinewidth{0.803000pt}%
\definecolor{currentstroke}{rgb}{0.000000,0.000000,0.000000}%
\pgfsetstrokecolor{currentstroke}%
\pgfsetdash{}{0pt}%
\pgfsys@defobject{currentmarker}{\pgfqpoint{0.000000in}{-0.048611in}}{\pgfqpoint{0.000000in}{0.000000in}}{%
\pgfpathmoveto{\pgfqpoint{0.000000in}{0.000000in}}%
\pgfpathlineto{\pgfqpoint{0.000000in}{-0.048611in}}%
\pgfusepath{stroke,fill}%
}%
\begin{pgfscope}%
\pgfsys@transformshift{0.733531in}{0.548769in}%
\pgfsys@useobject{currentmarker}{}%
\end{pgfscope}%
\end{pgfscope}%
\begin{pgfscope}%
\definecolor{textcolor}{rgb}{0.000000,0.000000,0.000000}%
\pgfsetstrokecolor{textcolor}%
\pgfsetfillcolor{textcolor}%
\pgftext[x=0.733531in,y=0.451547in,,top]{\color{textcolor}\rmfamily\fontsize{10.000000}{12.000000}\selectfont \(\displaystyle {0.0}\)}%
\end{pgfscope}%
\begin{pgfscope}%
\pgfpathrectangle{\pgfqpoint{0.733531in}{0.548769in}}{\pgfqpoint{3.028066in}{1.753186in}}%
\pgfusepath{clip}%
\pgfsetrectcap%
\pgfsetroundjoin%
\pgfsetlinewidth{0.803000pt}%
\definecolor{currentstroke}{rgb}{0.690196,0.690196,0.690196}%
\pgfsetstrokecolor{currentstroke}%
\pgfsetdash{}{0pt}%
\pgfpathmoveto{\pgfqpoint{1.490547in}{0.548769in}}%
\pgfpathlineto{\pgfqpoint{1.490547in}{2.301955in}}%
\pgfusepath{stroke}%
\end{pgfscope}%
\begin{pgfscope}%
\pgfsetbuttcap%
\pgfsetroundjoin%
\definecolor{currentfill}{rgb}{0.000000,0.000000,0.000000}%
\pgfsetfillcolor{currentfill}%
\pgfsetlinewidth{0.803000pt}%
\definecolor{currentstroke}{rgb}{0.000000,0.000000,0.000000}%
\pgfsetstrokecolor{currentstroke}%
\pgfsetdash{}{0pt}%
\pgfsys@defobject{currentmarker}{\pgfqpoint{0.000000in}{-0.048611in}}{\pgfqpoint{0.000000in}{0.000000in}}{%
\pgfpathmoveto{\pgfqpoint{0.000000in}{0.000000in}}%
\pgfpathlineto{\pgfqpoint{0.000000in}{-0.048611in}}%
\pgfusepath{stroke,fill}%
}%
\begin{pgfscope}%
\pgfsys@transformshift{1.490547in}{0.548769in}%
\pgfsys@useobject{currentmarker}{}%
\end{pgfscope}%
\end{pgfscope}%
\begin{pgfscope}%
\definecolor{textcolor}{rgb}{0.000000,0.000000,0.000000}%
\pgfsetstrokecolor{textcolor}%
\pgfsetfillcolor{textcolor}%
\pgftext[x=1.490547in,y=0.451547in,,top]{\color{textcolor}\rmfamily\fontsize{10.000000}{12.000000}\selectfont \(\displaystyle {0.5}\)}%
\end{pgfscope}%
\begin{pgfscope}%
\pgfpathrectangle{\pgfqpoint{0.733531in}{0.548769in}}{\pgfqpoint{3.028066in}{1.753186in}}%
\pgfusepath{clip}%
\pgfsetrectcap%
\pgfsetroundjoin%
\pgfsetlinewidth{0.803000pt}%
\definecolor{currentstroke}{rgb}{0.690196,0.690196,0.690196}%
\pgfsetstrokecolor{currentstroke}%
\pgfsetdash{}{0pt}%
\pgfpathmoveto{\pgfqpoint{2.247564in}{0.548769in}}%
\pgfpathlineto{\pgfqpoint{2.247564in}{2.301955in}}%
\pgfusepath{stroke}%
\end{pgfscope}%
\begin{pgfscope}%
\pgfsetbuttcap%
\pgfsetroundjoin%
\definecolor{currentfill}{rgb}{0.000000,0.000000,0.000000}%
\pgfsetfillcolor{currentfill}%
\pgfsetlinewidth{0.803000pt}%
\definecolor{currentstroke}{rgb}{0.000000,0.000000,0.000000}%
\pgfsetstrokecolor{currentstroke}%
\pgfsetdash{}{0pt}%
\pgfsys@defobject{currentmarker}{\pgfqpoint{0.000000in}{-0.048611in}}{\pgfqpoint{0.000000in}{0.000000in}}{%
\pgfpathmoveto{\pgfqpoint{0.000000in}{0.000000in}}%
\pgfpathlineto{\pgfqpoint{0.000000in}{-0.048611in}}%
\pgfusepath{stroke,fill}%
}%
\begin{pgfscope}%
\pgfsys@transformshift{2.247564in}{0.548769in}%
\pgfsys@useobject{currentmarker}{}%
\end{pgfscope}%
\end{pgfscope}%
\begin{pgfscope}%
\definecolor{textcolor}{rgb}{0.000000,0.000000,0.000000}%
\pgfsetstrokecolor{textcolor}%
\pgfsetfillcolor{textcolor}%
\pgftext[x=2.247564in,y=0.451547in,,top]{\color{textcolor}\rmfamily\fontsize{10.000000}{12.000000}\selectfont \(\displaystyle {1.0}\)}%
\end{pgfscope}%
\begin{pgfscope}%
\pgfpathrectangle{\pgfqpoint{0.733531in}{0.548769in}}{\pgfqpoint{3.028066in}{1.753186in}}%
\pgfusepath{clip}%
\pgfsetrectcap%
\pgfsetroundjoin%
\pgfsetlinewidth{0.803000pt}%
\definecolor{currentstroke}{rgb}{0.690196,0.690196,0.690196}%
\pgfsetstrokecolor{currentstroke}%
\pgfsetdash{}{0pt}%
\pgfpathmoveto{\pgfqpoint{3.004580in}{0.548769in}}%
\pgfpathlineto{\pgfqpoint{3.004580in}{2.301955in}}%
\pgfusepath{stroke}%
\end{pgfscope}%
\begin{pgfscope}%
\pgfsetbuttcap%
\pgfsetroundjoin%
\definecolor{currentfill}{rgb}{0.000000,0.000000,0.000000}%
\pgfsetfillcolor{currentfill}%
\pgfsetlinewidth{0.803000pt}%
\definecolor{currentstroke}{rgb}{0.000000,0.000000,0.000000}%
\pgfsetstrokecolor{currentstroke}%
\pgfsetdash{}{0pt}%
\pgfsys@defobject{currentmarker}{\pgfqpoint{0.000000in}{-0.048611in}}{\pgfqpoint{0.000000in}{0.000000in}}{%
\pgfpathmoveto{\pgfqpoint{0.000000in}{0.000000in}}%
\pgfpathlineto{\pgfqpoint{0.000000in}{-0.048611in}}%
\pgfusepath{stroke,fill}%
}%
\begin{pgfscope}%
\pgfsys@transformshift{3.004580in}{0.548769in}%
\pgfsys@useobject{currentmarker}{}%
\end{pgfscope}%
\end{pgfscope}%
\begin{pgfscope}%
\definecolor{textcolor}{rgb}{0.000000,0.000000,0.000000}%
\pgfsetstrokecolor{textcolor}%
\pgfsetfillcolor{textcolor}%
\pgftext[x=3.004580in,y=0.451547in,,top]{\color{textcolor}\rmfamily\fontsize{10.000000}{12.000000}\selectfont \(\displaystyle {1.5}\)}%
\end{pgfscope}%
\begin{pgfscope}%
\pgfpathrectangle{\pgfqpoint{0.733531in}{0.548769in}}{\pgfqpoint{3.028066in}{1.753186in}}%
\pgfusepath{clip}%
\pgfsetrectcap%
\pgfsetroundjoin%
\pgfsetlinewidth{0.803000pt}%
\definecolor{currentstroke}{rgb}{0.690196,0.690196,0.690196}%
\pgfsetstrokecolor{currentstroke}%
\pgfsetdash{}{0pt}%
\pgfpathmoveto{\pgfqpoint{3.761597in}{0.548769in}}%
\pgfpathlineto{\pgfqpoint{3.761597in}{2.301955in}}%
\pgfusepath{stroke}%
\end{pgfscope}%
\begin{pgfscope}%
\pgfsetbuttcap%
\pgfsetroundjoin%
\definecolor{currentfill}{rgb}{0.000000,0.000000,0.000000}%
\pgfsetfillcolor{currentfill}%
\pgfsetlinewidth{0.803000pt}%
\definecolor{currentstroke}{rgb}{0.000000,0.000000,0.000000}%
\pgfsetstrokecolor{currentstroke}%
\pgfsetdash{}{0pt}%
\pgfsys@defobject{currentmarker}{\pgfqpoint{0.000000in}{-0.048611in}}{\pgfqpoint{0.000000in}{0.000000in}}{%
\pgfpathmoveto{\pgfqpoint{0.000000in}{0.000000in}}%
\pgfpathlineto{\pgfqpoint{0.000000in}{-0.048611in}}%
\pgfusepath{stroke,fill}%
}%
\begin{pgfscope}%
\pgfsys@transformshift{3.761597in}{0.548769in}%
\pgfsys@useobject{currentmarker}{}%
\end{pgfscope}%
\end{pgfscope}%
\begin{pgfscope}%
\definecolor{textcolor}{rgb}{0.000000,0.000000,0.000000}%
\pgfsetstrokecolor{textcolor}%
\pgfsetfillcolor{textcolor}%
\pgftext[x=3.761597in,y=0.451547in,,top]{\color{textcolor}\rmfamily\fontsize{10.000000}{12.000000}\selectfont \(\displaystyle {2.0}\)}%
\end{pgfscope}%
\begin{pgfscope}%
\definecolor{textcolor}{rgb}{0.000000,0.000000,0.000000}%
\pgfsetstrokecolor{textcolor}%
\pgfsetfillcolor{textcolor}%
\pgftext[x=2.247564in,y=0.272534in,,top]{\color{textcolor}\rmfamily\fontsize{10.000000}{12.000000}\selectfont \(\displaystyle w\)}%
\end{pgfscope}%
\begin{pgfscope}%
\pgfpathrectangle{\pgfqpoint{0.733531in}{0.548769in}}{\pgfqpoint{3.028066in}{1.753186in}}%
\pgfusepath{clip}%
\pgfsetrectcap%
\pgfsetroundjoin%
\pgfsetlinewidth{0.803000pt}%
\definecolor{currentstroke}{rgb}{0.690196,0.690196,0.690196}%
\pgfsetstrokecolor{currentstroke}%
\pgfsetdash{}{0pt}%
\pgfpathmoveto{\pgfqpoint{0.733531in}{0.548769in}}%
\pgfpathlineto{\pgfqpoint{3.761597in}{0.548769in}}%
\pgfusepath{stroke}%
\end{pgfscope}%
\begin{pgfscope}%
\pgfsetbuttcap%
\pgfsetroundjoin%
\definecolor{currentfill}{rgb}{0.000000,0.000000,0.000000}%
\pgfsetfillcolor{currentfill}%
\pgfsetlinewidth{0.803000pt}%
\definecolor{currentstroke}{rgb}{0.000000,0.000000,0.000000}%
\pgfsetstrokecolor{currentstroke}%
\pgfsetdash{}{0pt}%
\pgfsys@defobject{currentmarker}{\pgfqpoint{-0.048611in}{0.000000in}}{\pgfqpoint{-0.000000in}{0.000000in}}{%
\pgfpathmoveto{\pgfqpoint{-0.000000in}{0.000000in}}%
\pgfpathlineto{\pgfqpoint{-0.048611in}{0.000000in}}%
\pgfusepath{stroke,fill}%
}%
\begin{pgfscope}%
\pgfsys@transformshift{0.733531in}{0.548769in}%
\pgfsys@useobject{currentmarker}{}%
\end{pgfscope}%
\end{pgfscope}%
\begin{pgfscope}%
\definecolor{textcolor}{rgb}{0.000000,0.000000,0.000000}%
\pgfsetstrokecolor{textcolor}%
\pgfsetfillcolor{textcolor}%
\pgftext[x=0.348306in, y=0.500544in, left, base]{\color{textcolor}\rmfamily\fontsize{10.000000}{12.000000}\selectfont \(\displaystyle {10^{-4}}\)}%
\end{pgfscope}%
\begin{pgfscope}%
\pgfpathrectangle{\pgfqpoint{0.733531in}{0.548769in}}{\pgfqpoint{3.028066in}{1.753186in}}%
\pgfusepath{clip}%
\pgfsetrectcap%
\pgfsetroundjoin%
\pgfsetlinewidth{0.803000pt}%
\definecolor{currentstroke}{rgb}{0.690196,0.690196,0.690196}%
\pgfsetstrokecolor{currentstroke}%
\pgfsetdash{}{0pt}%
\pgfpathmoveto{\pgfqpoint{0.733531in}{0.899406in}}%
\pgfpathlineto{\pgfqpoint{3.761597in}{0.899406in}}%
\pgfusepath{stroke}%
\end{pgfscope}%
\begin{pgfscope}%
\pgfsetbuttcap%
\pgfsetroundjoin%
\definecolor{currentfill}{rgb}{0.000000,0.000000,0.000000}%
\pgfsetfillcolor{currentfill}%
\pgfsetlinewidth{0.803000pt}%
\definecolor{currentstroke}{rgb}{0.000000,0.000000,0.000000}%
\pgfsetstrokecolor{currentstroke}%
\pgfsetdash{}{0pt}%
\pgfsys@defobject{currentmarker}{\pgfqpoint{-0.048611in}{0.000000in}}{\pgfqpoint{-0.000000in}{0.000000in}}{%
\pgfpathmoveto{\pgfqpoint{-0.000000in}{0.000000in}}%
\pgfpathlineto{\pgfqpoint{-0.048611in}{0.000000in}}%
\pgfusepath{stroke,fill}%
}%
\begin{pgfscope}%
\pgfsys@transformshift{0.733531in}{0.899406in}%
\pgfsys@useobject{currentmarker}{}%
\end{pgfscope}%
\end{pgfscope}%
\begin{pgfscope}%
\definecolor{textcolor}{rgb}{0.000000,0.000000,0.000000}%
\pgfsetstrokecolor{textcolor}%
\pgfsetfillcolor{textcolor}%
\pgftext[x=0.348306in, y=0.851181in, left, base]{\color{textcolor}\rmfamily\fontsize{10.000000}{12.000000}\selectfont \(\displaystyle {10^{-2}}\)}%
\end{pgfscope}%
\begin{pgfscope}%
\pgfpathrectangle{\pgfqpoint{0.733531in}{0.548769in}}{\pgfqpoint{3.028066in}{1.753186in}}%
\pgfusepath{clip}%
\pgfsetrectcap%
\pgfsetroundjoin%
\pgfsetlinewidth{0.803000pt}%
\definecolor{currentstroke}{rgb}{0.690196,0.690196,0.690196}%
\pgfsetstrokecolor{currentstroke}%
\pgfsetdash{}{0pt}%
\pgfpathmoveto{\pgfqpoint{0.733531in}{1.250043in}}%
\pgfpathlineto{\pgfqpoint{3.761597in}{1.250043in}}%
\pgfusepath{stroke}%
\end{pgfscope}%
\begin{pgfscope}%
\pgfsetbuttcap%
\pgfsetroundjoin%
\definecolor{currentfill}{rgb}{0.000000,0.000000,0.000000}%
\pgfsetfillcolor{currentfill}%
\pgfsetlinewidth{0.803000pt}%
\definecolor{currentstroke}{rgb}{0.000000,0.000000,0.000000}%
\pgfsetstrokecolor{currentstroke}%
\pgfsetdash{}{0pt}%
\pgfsys@defobject{currentmarker}{\pgfqpoint{-0.048611in}{0.000000in}}{\pgfqpoint{-0.000000in}{0.000000in}}{%
\pgfpathmoveto{\pgfqpoint{-0.000000in}{0.000000in}}%
\pgfpathlineto{\pgfqpoint{-0.048611in}{0.000000in}}%
\pgfusepath{stroke,fill}%
}%
\begin{pgfscope}%
\pgfsys@transformshift{0.733531in}{1.250043in}%
\pgfsys@useobject{currentmarker}{}%
\end{pgfscope}%
\end{pgfscope}%
\begin{pgfscope}%
\definecolor{textcolor}{rgb}{0.000000,0.000000,0.000000}%
\pgfsetstrokecolor{textcolor}%
\pgfsetfillcolor{textcolor}%
\pgftext[x=0.435112in, y=1.201818in, left, base]{\color{textcolor}\rmfamily\fontsize{10.000000}{12.000000}\selectfont \(\displaystyle {10^{0}}\)}%
\end{pgfscope}%
\begin{pgfscope}%
\pgfpathrectangle{\pgfqpoint{0.733531in}{0.548769in}}{\pgfqpoint{3.028066in}{1.753186in}}%
\pgfusepath{clip}%
\pgfsetrectcap%
\pgfsetroundjoin%
\pgfsetlinewidth{0.803000pt}%
\definecolor{currentstroke}{rgb}{0.690196,0.690196,0.690196}%
\pgfsetstrokecolor{currentstroke}%
\pgfsetdash{}{0pt}%
\pgfpathmoveto{\pgfqpoint{0.733531in}{1.600680in}}%
\pgfpathlineto{\pgfqpoint{3.761597in}{1.600680in}}%
\pgfusepath{stroke}%
\end{pgfscope}%
\begin{pgfscope}%
\pgfsetbuttcap%
\pgfsetroundjoin%
\definecolor{currentfill}{rgb}{0.000000,0.000000,0.000000}%
\pgfsetfillcolor{currentfill}%
\pgfsetlinewidth{0.803000pt}%
\definecolor{currentstroke}{rgb}{0.000000,0.000000,0.000000}%
\pgfsetstrokecolor{currentstroke}%
\pgfsetdash{}{0pt}%
\pgfsys@defobject{currentmarker}{\pgfqpoint{-0.048611in}{0.000000in}}{\pgfqpoint{-0.000000in}{0.000000in}}{%
\pgfpathmoveto{\pgfqpoint{-0.000000in}{0.000000in}}%
\pgfpathlineto{\pgfqpoint{-0.048611in}{0.000000in}}%
\pgfusepath{stroke,fill}%
}%
\begin{pgfscope}%
\pgfsys@transformshift{0.733531in}{1.600680in}%
\pgfsys@useobject{currentmarker}{}%
\end{pgfscope}%
\end{pgfscope}%
\begin{pgfscope}%
\definecolor{textcolor}{rgb}{0.000000,0.000000,0.000000}%
\pgfsetstrokecolor{textcolor}%
\pgfsetfillcolor{textcolor}%
\pgftext[x=0.435112in, y=1.552455in, left, base]{\color{textcolor}\rmfamily\fontsize{10.000000}{12.000000}\selectfont \(\displaystyle {10^{2}}\)}%
\end{pgfscope}%
\begin{pgfscope}%
\pgfpathrectangle{\pgfqpoint{0.733531in}{0.548769in}}{\pgfqpoint{3.028066in}{1.753186in}}%
\pgfusepath{clip}%
\pgfsetrectcap%
\pgfsetroundjoin%
\pgfsetlinewidth{0.803000pt}%
\definecolor{currentstroke}{rgb}{0.690196,0.690196,0.690196}%
\pgfsetstrokecolor{currentstroke}%
\pgfsetdash{}{0pt}%
\pgfpathmoveto{\pgfqpoint{0.733531in}{1.951318in}}%
\pgfpathlineto{\pgfqpoint{3.761597in}{1.951318in}}%
\pgfusepath{stroke}%
\end{pgfscope}%
\begin{pgfscope}%
\pgfsetbuttcap%
\pgfsetroundjoin%
\definecolor{currentfill}{rgb}{0.000000,0.000000,0.000000}%
\pgfsetfillcolor{currentfill}%
\pgfsetlinewidth{0.803000pt}%
\definecolor{currentstroke}{rgb}{0.000000,0.000000,0.000000}%
\pgfsetstrokecolor{currentstroke}%
\pgfsetdash{}{0pt}%
\pgfsys@defobject{currentmarker}{\pgfqpoint{-0.048611in}{0.000000in}}{\pgfqpoint{-0.000000in}{0.000000in}}{%
\pgfpathmoveto{\pgfqpoint{-0.000000in}{0.000000in}}%
\pgfpathlineto{\pgfqpoint{-0.048611in}{0.000000in}}%
\pgfusepath{stroke,fill}%
}%
\begin{pgfscope}%
\pgfsys@transformshift{0.733531in}{1.951318in}%
\pgfsys@useobject{currentmarker}{}%
\end{pgfscope}%
\end{pgfscope}%
\begin{pgfscope}%
\definecolor{textcolor}{rgb}{0.000000,0.000000,0.000000}%
\pgfsetstrokecolor{textcolor}%
\pgfsetfillcolor{textcolor}%
\pgftext[x=0.435112in, y=1.903092in, left, base]{\color{textcolor}\rmfamily\fontsize{10.000000}{12.000000}\selectfont \(\displaystyle {10^{4}}\)}%
\end{pgfscope}%
\begin{pgfscope}%
\pgfpathrectangle{\pgfqpoint{0.733531in}{0.548769in}}{\pgfqpoint{3.028066in}{1.753186in}}%
\pgfusepath{clip}%
\pgfsetrectcap%
\pgfsetroundjoin%
\pgfsetlinewidth{0.803000pt}%
\definecolor{currentstroke}{rgb}{0.690196,0.690196,0.690196}%
\pgfsetstrokecolor{currentstroke}%
\pgfsetdash{}{0pt}%
\pgfpathmoveto{\pgfqpoint{0.733531in}{2.301955in}}%
\pgfpathlineto{\pgfqpoint{3.761597in}{2.301955in}}%
\pgfusepath{stroke}%
\end{pgfscope}%
\begin{pgfscope}%
\pgfsetbuttcap%
\pgfsetroundjoin%
\definecolor{currentfill}{rgb}{0.000000,0.000000,0.000000}%
\pgfsetfillcolor{currentfill}%
\pgfsetlinewidth{0.803000pt}%
\definecolor{currentstroke}{rgb}{0.000000,0.000000,0.000000}%
\pgfsetstrokecolor{currentstroke}%
\pgfsetdash{}{0pt}%
\pgfsys@defobject{currentmarker}{\pgfqpoint{-0.048611in}{0.000000in}}{\pgfqpoint{-0.000000in}{0.000000in}}{%
\pgfpathmoveto{\pgfqpoint{-0.000000in}{0.000000in}}%
\pgfpathlineto{\pgfqpoint{-0.048611in}{0.000000in}}%
\pgfusepath{stroke,fill}%
}%
\begin{pgfscope}%
\pgfsys@transformshift{0.733531in}{2.301955in}%
\pgfsys@useobject{currentmarker}{}%
\end{pgfscope}%
\end{pgfscope}%
\begin{pgfscope}%
\definecolor{textcolor}{rgb}{0.000000,0.000000,0.000000}%
\pgfsetstrokecolor{textcolor}%
\pgfsetfillcolor{textcolor}%
\pgftext[x=0.435112in, y=2.253730in, left, base]{\color{textcolor}\rmfamily\fontsize{10.000000}{12.000000}\selectfont \(\displaystyle {10^{6}}\)}%
\end{pgfscope}%
\begin{pgfscope}%
\definecolor{textcolor}{rgb}{0.000000,0.000000,0.000000}%
\pgfsetstrokecolor{textcolor}%
\pgfsetfillcolor{textcolor}%
\pgftext[x=0.292751in,y=1.425362in,,bottom,rotate=90.000000]{\color{textcolor}\rmfamily\fontsize{10.000000}{12.000000}\selectfont \(\displaystyle F^2_N(w)\)}%
\end{pgfscope}%
\begin{pgfscope}%
\pgfpathrectangle{\pgfqpoint{0.733531in}{0.548769in}}{\pgfqpoint{3.028066in}{1.753186in}}%
\pgfusepath{clip}%
\pgfsetrectcap%
\pgfsetroundjoin%
\pgfsetlinewidth{1.003750pt}%
\definecolor{currentstroke}{rgb}{0.000000,0.501961,0.000000}%
\pgfsetstrokecolor{currentstroke}%
\pgfsetdash{}{0pt}%
\pgfpathmoveto{\pgfqpoint{0.739446in}{0.534880in}}%
\pgfpathlineto{\pgfqpoint{0.744135in}{0.623954in}}%
\pgfpathlineto{\pgfqpoint{0.750951in}{0.699544in}}%
\pgfpathlineto{\pgfqpoint{0.759282in}{0.759051in}}%
\pgfpathlineto{\pgfqpoint{0.769129in}{0.808333in}}%
\pgfpathlineto{\pgfqpoint{0.781247in}{0.852909in}}%
\pgfpathlineto{\pgfqpoint{0.794880in}{0.891121in}}%
\pgfpathlineto{\pgfqpoint{0.810028in}{0.924642in}}%
\pgfpathlineto{\pgfqpoint{0.827448in}{0.955767in}}%
\pgfpathlineto{\pgfqpoint{0.847140in}{0.984592in}}%
\pgfpathlineto{\pgfqpoint{0.869105in}{1.011289in}}%
\pgfpathlineto{\pgfqpoint{0.894099in}{1.036759in}}%
\pgfpathlineto{\pgfqpoint{0.922122in}{1.060860in}}%
\pgfpathlineto{\pgfqpoint{0.953933in}{1.084065in}}%
\pgfpathlineto{\pgfqpoint{0.989531in}{1.106163in}}%
\pgfpathlineto{\pgfqpoint{1.029673in}{1.127411in}}%
\pgfpathlineto{\pgfqpoint{1.075116in}{1.147900in}}%
\pgfpathlineto{\pgfqpoint{1.125862in}{1.167334in}}%
\pgfpathlineto{\pgfqpoint{1.182666in}{1.185708in}}%
\pgfpathlineto{\pgfqpoint{1.244773in}{1.202512in}}%
\pgfpathlineto{\pgfqpoint{1.312181in}{1.217523in}}%
\pgfpathlineto{\pgfqpoint{1.383376in}{1.230197in}}%
\pgfpathlineto{\pgfqpoint{1.456086in}{1.240011in}}%
\pgfpathlineto{\pgfqpoint{1.527281in}{1.246554in}}%
\pgfpathlineto{\pgfqpoint{1.594689in}{1.249711in}}%
\pgfpathlineto{\pgfqpoint{1.656038in}{1.249582in}}%
\pgfpathlineto{\pgfqpoint{1.710571in}{1.246487in}}%
\pgfpathlineto{\pgfqpoint{1.759044in}{1.240693in}}%
\pgfpathlineto{\pgfqpoint{1.800701in}{1.232678in}}%
\pgfpathlineto{\pgfqpoint{1.837056in}{1.222595in}}%
\pgfpathlineto{\pgfqpoint{1.868109in}{1.210890in}}%
\pgfpathlineto{\pgfqpoint{1.895375in}{1.197412in}}%
\pgfpathlineto{\pgfqpoint{1.919612in}{1.181986in}}%
\pgfpathlineto{\pgfqpoint{1.940819in}{1.164808in}}%
\pgfpathlineto{\pgfqpoint{1.959754in}{1.145405in}}%
\pgfpathlineto{\pgfqpoint{1.976416in}{1.123825in}}%
\pgfpathlineto{\pgfqpoint{1.990807in}{1.100289in}}%
\pgfpathlineto{\pgfqpoint{2.003683in}{1.073560in}}%
\pgfpathlineto{\pgfqpoint{2.015044in}{1.043233in}}%
\pgfpathlineto{\pgfqpoint{2.025647in}{1.005813in}}%
\pgfpathlineto{\pgfqpoint{2.034736in}{0.961762in}}%
\pgfpathlineto{\pgfqpoint{2.042310in}{0.909257in}}%
\pgfpathlineto{\pgfqpoint{2.048369in}{0.845806in}}%
\pgfpathlineto{\pgfqpoint{2.052913in}{0.768537in}}%
\pgfpathlineto{\pgfqpoint{2.056700in}{0.642785in}}%
\pgfpathlineto{\pgfqpoint{2.058170in}{0.534880in}}%
\pgfpathmoveto{\pgfqpoint{2.061062in}{0.534880in}}%
\pgfpathlineto{\pgfqpoint{2.063517in}{0.690755in}}%
\pgfpathlineto{\pgfqpoint{2.068061in}{0.809745in}}%
\pgfpathlineto{\pgfqpoint{2.074120in}{0.894170in}}%
\pgfpathlineto{\pgfqpoint{2.082452in}{0.966161in}}%
\pgfpathlineto{\pgfqpoint{2.092298in}{1.024095in}}%
\pgfpathlineto{\pgfqpoint{2.103659in}{1.073241in}}%
\pgfpathlineto{\pgfqpoint{2.117292in}{1.118465in}}%
\pgfpathlineto{\pgfqpoint{2.132440in}{1.158024in}}%
\pgfpathlineto{\pgfqpoint{2.148345in}{1.191332in}}%
\pgfpathlineto{\pgfqpoint{2.164250in}{1.217932in}}%
\pgfpathlineto{\pgfqpoint{2.178641in}{1.236315in}}%
\pgfpathlineto{\pgfqpoint{2.190002in}{1.246177in}}%
\pgfpathlineto{\pgfqpoint{2.198333in}{1.249773in}}%
\pgfpathlineto{\pgfqpoint{2.205150in}{1.249364in}}%
\pgfpathlineto{\pgfqpoint{2.210452in}{1.245998in}}%
\pgfpathlineto{\pgfqpoint{2.215753in}{1.238648in}}%
\pgfpathlineto{\pgfqpoint{2.221055in}{1.225044in}}%
\pgfpathlineto{\pgfqpoint{2.225599in}{1.204942in}}%
\pgfpathlineto{\pgfqpoint{2.230144in}{1.169937in}}%
\pgfpathlineto{\pgfqpoint{2.233931in}{1.115425in}}%
\pgfpathlineto{\pgfqpoint{2.236960in}{1.023401in}}%
\pgfpathlineto{\pgfqpoint{2.238475in}{0.917874in}}%
\pgfpathlineto{\pgfqpoint{2.239990in}{0.614529in}}%
\pgfpathlineto{\pgfqpoint{2.242262in}{1.034265in}}%
\pgfpathlineto{\pgfqpoint{2.246806in}{1.228220in}}%
\pgfpathlineto{\pgfqpoint{2.247564in}{1.250043in}}%
\pgfpathlineto{\pgfqpoint{2.247564in}{1.250043in}}%
\pgfusepath{stroke}%
\end{pgfscope}%
\begin{pgfscope}%
\pgfpathrectangle{\pgfqpoint{0.733531in}{0.548769in}}{\pgfqpoint{3.028066in}{1.753186in}}%
\pgfusepath{clip}%
\pgfsetrectcap%
\pgfsetroundjoin%
\pgfsetlinewidth{1.003750pt}%
\definecolor{currentstroke}{rgb}{1.000000,0.647059,0.000000}%
\pgfsetstrokecolor{currentstroke}%
\pgfsetdash{}{0pt}%
\pgfpathmoveto{\pgfqpoint{2.247564in}{1.250043in}}%
\pgfpathlineto{\pgfqpoint{2.256527in}{1.456923in}}%
\pgfpathlineto{\pgfqpoint{2.262583in}{1.600512in}}%
\pgfpathlineto{\pgfqpoint{2.262583in}{1.600512in}}%
\pgfusepath{stroke}%
\end{pgfscope}%
\begin{pgfscope}%
\pgfpathrectangle{\pgfqpoint{0.733531in}{0.548769in}}{\pgfqpoint{3.028066in}{1.753186in}}%
\pgfusepath{clip}%
\pgfsetrectcap%
\pgfsetroundjoin%
\pgfsetlinewidth{1.003750pt}%
\definecolor{currentstroke}{rgb}{1.000000,0.000000,0.000000}%
\pgfsetstrokecolor{currentstroke}%
\pgfsetdash{}{0pt}%
\pgfpathmoveto{\pgfqpoint{2.262583in}{1.600512in}}%
\pgfpathlineto{\pgfqpoint{2.267082in}{1.764522in}}%
\pgfpathlineto{\pgfqpoint{2.269332in}{1.949179in}}%
\pgfpathlineto{\pgfqpoint{2.270082in}{2.126941in}}%
\pgfpathlineto{\pgfqpoint{2.270832in}{2.115965in}}%
\pgfpathlineto{\pgfqpoint{2.273831in}{1.806954in}}%
\pgfpathlineto{\pgfqpoint{2.278331in}{1.704361in}}%
\pgfpathlineto{\pgfqpoint{2.283580in}{1.654377in}}%
\pgfpathlineto{\pgfqpoint{2.289579in}{1.626078in}}%
\pgfpathlineto{\pgfqpoint{2.295578in}{1.611340in}}%
\pgfpathlineto{\pgfqpoint{2.301577in}{1.603818in}}%
\pgfpathlineto{\pgfqpoint{2.307576in}{1.600689in}}%
\pgfpathlineto{\pgfqpoint{2.314325in}{1.600615in}}%
\pgfpathlineto{\pgfqpoint{2.322574in}{1.603892in}}%
\pgfpathlineto{\pgfqpoint{2.333072in}{1.611738in}}%
\pgfpathlineto{\pgfqpoint{2.346570in}{1.626063in}}%
\pgfpathlineto{\pgfqpoint{2.363067in}{1.648464in}}%
\pgfpathlineto{\pgfqpoint{2.381064in}{1.678367in}}%
\pgfpathlineto{\pgfqpoint{2.398312in}{1.712908in}}%
\pgfpathlineto{\pgfqpoint{2.414809in}{1.753046in}}%
\pgfpathlineto{\pgfqpoint{2.429057in}{1.796016in}}%
\pgfpathlineto{\pgfqpoint{2.441055in}{1.841839in}}%
\pgfpathlineto{\pgfqpoint{2.451553in}{1.894485in}}%
\pgfpathlineto{\pgfqpoint{2.459802in}{1.951191in}}%
\pgfpathlineto{\pgfqpoint{2.466551in}{2.018321in}}%
\pgfpathlineto{\pgfqpoint{2.471800in}{2.100984in}}%
\pgfpathlineto{\pgfqpoint{2.475549in}{2.207779in}}%
\pgfpathlineto{\pgfqpoint{2.477412in}{2.315844in}}%
\pgfpathmoveto{\pgfqpoint{2.481215in}{2.315844in}}%
\pgfpathlineto{\pgfqpoint{2.484548in}{2.160223in}}%
\pgfpathlineto{\pgfqpoint{2.489797in}{2.056487in}}%
\pgfpathlineto{\pgfqpoint{2.496546in}{1.983295in}}%
\pgfpathlineto{\pgfqpoint{2.504795in}{1.926727in}}%
\pgfpathlineto{\pgfqpoint{2.514543in}{1.880902in}}%
\pgfpathlineto{\pgfqpoint{2.525792in}{1.842715in}}%
\pgfpathlineto{\pgfqpoint{2.538540in}{1.810301in}}%
\pgfpathlineto{\pgfqpoint{2.552787in}{1.782438in}}%
\pgfpathlineto{\pgfqpoint{2.568535in}{1.758275in}}%
\pgfpathlineto{\pgfqpoint{2.586532in}{1.736373in}}%
\pgfpathlineto{\pgfqpoint{2.606779in}{1.716740in}}%
\pgfpathlineto{\pgfqpoint{2.629275in}{1.699278in}}%
\pgfpathlineto{\pgfqpoint{2.654771in}{1.683422in}}%
\pgfpathlineto{\pgfqpoint{2.684017in}{1.668923in}}%
\pgfpathlineto{\pgfqpoint{2.717761in}{1.655709in}}%
\pgfpathlineto{\pgfqpoint{2.756755in}{1.643804in}}%
\pgfpathlineto{\pgfqpoint{2.802498in}{1.633115in}}%
\pgfpathlineto{\pgfqpoint{2.857239in}{1.623602in}}%
\pgfpathlineto{\pgfqpoint{2.922479in}{1.615504in}}%
\pgfpathlineto{\pgfqpoint{3.001966in}{1.608875in}}%
\pgfpathlineto{\pgfqpoint{3.099451in}{1.603960in}}%
\pgfpathlineto{\pgfqpoint{3.222432in}{1.600986in}}%
\pgfpathlineto{\pgfqpoint{3.382157in}{1.600379in}}%
\pgfpathlineto{\pgfqpoint{3.598872in}{1.602843in}}%
\pgfpathlineto{\pgfqpoint{3.761597in}{1.606074in}}%
\pgfpathlineto{\pgfqpoint{3.761597in}{1.606074in}}%
\pgfusepath{stroke}%
\end{pgfscope}%
\begin{pgfscope}%
\pgfpathrectangle{\pgfqpoint{0.733531in}{0.548769in}}{\pgfqpoint{3.028066in}{1.753186in}}%
\pgfusepath{clip}%
\pgfsetbuttcap%
\pgfsetroundjoin%
\definecolor{currentfill}{rgb}{0.000000,0.000000,0.000000}%
\pgfsetfillcolor{currentfill}%
\pgfsetfillopacity{0.000000}%
\pgfsetlinewidth{1.003750pt}%
\definecolor{currentstroke}{rgb}{0.000000,0.000000,0.000000}%
\pgfsetstrokecolor{currentstroke}%
\pgfsetdash{}{0pt}%
\pgfsys@defobject{currentmarker}{\pgfqpoint{-0.041667in}{-0.041667in}}{\pgfqpoint{0.041667in}{0.041667in}}{%
\pgfpathmoveto{\pgfqpoint{0.000000in}{-0.041667in}}%
\pgfpathcurveto{\pgfqpoint{0.011050in}{-0.041667in}}{\pgfqpoint{0.021649in}{-0.037276in}}{\pgfqpoint{0.029463in}{-0.029463in}}%
\pgfpathcurveto{\pgfqpoint{0.037276in}{-0.021649in}}{\pgfqpoint{0.041667in}{-0.011050in}}{\pgfqpoint{0.041667in}{0.000000in}}%
\pgfpathcurveto{\pgfqpoint{0.041667in}{0.011050in}}{\pgfqpoint{0.037276in}{0.021649in}}{\pgfqpoint{0.029463in}{0.029463in}}%
\pgfpathcurveto{\pgfqpoint{0.021649in}{0.037276in}}{\pgfqpoint{0.011050in}{0.041667in}}{\pgfqpoint{0.000000in}{0.041667in}}%
\pgfpathcurveto{\pgfqpoint{-0.011050in}{0.041667in}}{\pgfqpoint{-0.021649in}{0.037276in}}{\pgfqpoint{-0.029463in}{0.029463in}}%
\pgfpathcurveto{\pgfqpoint{-0.037276in}{0.021649in}}{\pgfqpoint{-0.041667in}{0.011050in}}{\pgfqpoint{-0.041667in}{0.000000in}}%
\pgfpathcurveto{\pgfqpoint{-0.041667in}{-0.011050in}}{\pgfqpoint{-0.037276in}{-0.021649in}}{\pgfqpoint{-0.029463in}{-0.029463in}}%
\pgfpathcurveto{\pgfqpoint{-0.021649in}{-0.037276in}}{\pgfqpoint{-0.011050in}{-0.041667in}}{\pgfqpoint{0.000000in}{-0.041667in}}%
\pgfpathlineto{\pgfqpoint{0.000000in}{-0.041667in}}%
\pgfpathclose%
\pgfusepath{stroke,fill}%
}%
\begin{pgfscope}%
\pgfsys@transformshift{0.733531in}{0.548769in}%
\pgfsys@useobject{currentmarker}{}%
\end{pgfscope}%
\begin{pgfscope}%
\pgfsys@transformshift{2.050740in}{0.548769in}%
\pgfsys@useobject{currentmarker}{}%
\end{pgfscope}%
\begin{pgfscope}%
\pgfsys@transformshift{2.239994in}{0.548769in}%
\pgfsys@useobject{currentmarker}{}%
\end{pgfscope}%
\end{pgfscope}%
\begin{pgfscope}%
\pgfpathrectangle{\pgfqpoint{0.733531in}{0.548769in}}{\pgfqpoint{3.028066in}{1.753186in}}%
\pgfusepath{clip}%
\pgfsetbuttcap%
\pgfsetroundjoin%
\definecolor{currentfill}{rgb}{0.000000,0.000000,0.000000}%
\pgfsetfillcolor{currentfill}%
\pgfsetfillopacity{0.000000}%
\pgfsetlinewidth{1.003750pt}%
\definecolor{currentstroke}{rgb}{0.000000,0.000000,0.000000}%
\pgfsetstrokecolor{currentstroke}%
\pgfsetdash{}{0pt}%
\pgfsys@defobject{currentmarker}{\pgfqpoint{-0.041667in}{-0.041667in}}{\pgfqpoint{0.041667in}{0.041667in}}{%
\pgfpathmoveto{\pgfqpoint{-0.041667in}{-0.041667in}}%
\pgfpathlineto{\pgfqpoint{0.041667in}{0.041667in}}%
\pgfpathmoveto{\pgfqpoint{-0.041667in}{0.041667in}}%
\pgfpathlineto{\pgfqpoint{0.041667in}{-0.041667in}}%
\pgfusepath{stroke,fill}%
}%
\begin{pgfscope}%
\pgfsys@transformshift{2.262704in}{2.301955in}%
\pgfsys@useobject{currentmarker}{}%
\end{pgfscope}%
\begin{pgfscope}%
\pgfsys@transformshift{2.482239in}{2.301955in}%
\pgfsys@useobject{currentmarker}{}%
\end{pgfscope}%
\end{pgfscope}%
\begin{pgfscope}%
\pgfsetrectcap%
\pgfsetmiterjoin%
\pgfsetlinewidth{0.803000pt}%
\definecolor{currentstroke}{rgb}{0.000000,0.000000,0.000000}%
\pgfsetstrokecolor{currentstroke}%
\pgfsetdash{}{0pt}%
\pgfpathmoveto{\pgfqpoint{0.733531in}{0.548769in}}%
\pgfpathlineto{\pgfqpoint{0.733531in}{2.301955in}}%
\pgfusepath{stroke}%
\end{pgfscope}%
\begin{pgfscope}%
\pgfsetrectcap%
\pgfsetmiterjoin%
\pgfsetlinewidth{0.803000pt}%
\definecolor{currentstroke}{rgb}{0.000000,0.000000,0.000000}%
\pgfsetstrokecolor{currentstroke}%
\pgfsetdash{}{0pt}%
\pgfpathmoveto{\pgfqpoint{3.761597in}{0.548769in}}%
\pgfpathlineto{\pgfqpoint{3.761597in}{2.301955in}}%
\pgfusepath{stroke}%
\end{pgfscope}%
\begin{pgfscope}%
\pgfsetrectcap%
\pgfsetmiterjoin%
\pgfsetlinewidth{0.803000pt}%
\definecolor{currentstroke}{rgb}{0.000000,0.000000,0.000000}%
\pgfsetstrokecolor{currentstroke}%
\pgfsetdash{}{0pt}%
\pgfpathmoveto{\pgfqpoint{0.733531in}{0.548769in}}%
\pgfpathlineto{\pgfqpoint{3.761597in}{0.548769in}}%
\pgfusepath{stroke}%
\end{pgfscope}%
\begin{pgfscope}%
\pgfsetrectcap%
\pgfsetmiterjoin%
\pgfsetlinewidth{0.803000pt}%
\definecolor{currentstroke}{rgb}{0.000000,0.000000,0.000000}%
\pgfsetstrokecolor{currentstroke}%
\pgfsetdash{}{0pt}%
\pgfpathmoveto{\pgfqpoint{0.733531in}{2.301955in}}%
\pgfpathlineto{\pgfqpoint{3.761597in}{2.301955in}}%
\pgfusepath{stroke}%
\end{pgfscope}%
\end{pgfpicture}%
\makeatother%
\endgroup%

    \caption{$F_N$ für ein elliptischs filter.}
    \label{ellfilter:fig:elliptic}
\end{figure}

\subsection{Degree Equation}

Der $\cd^{-1}$ Term muss so verzogen werden, dass die umgebene $\cd$-Funktion die Nullstellen und Pole trifft.
Dies trifft ein wenn die Degree Equation erfüllt ist.

\begin{equation}
    N \frac{K^\prime}{K} = \frac{K^\prime_1}{K_1}
\end{equation}


Leider ist das lösen dieser Gleichung nicht trivial.
Die Rechnung wird in \ref{ellfilter:bib:orfanidis} im Detail angeschaut.


\subsection{Polynome?}

Bei den Tschebyscheff-Polynomen haben wir gesehen, dass die Trigonometrische Formel zu einfachen Polynomen umgewandelt werden kann.
Im gegensatz zum $\cos^{-1}$ hat der $\cd^{-1}$ nicht nur Nullstellen sondern auch Pole.
Somit entstehen bei den elliptischen rationalen Funktionen, wie es der name auch deutet, rationale Funktionen, also ein Bruch von zwei Polynomen.

Da Transformationen einer rationalen Funktionen mit Grundrechenarten, wie es in \eqref{ellfilter:eq:h_omega} der Fall ist, immer noch rationale Funktionen ergeben, stellt dies kein Problem für die Implementierung dar.

%
% einleitung.tex -- Beispiel-File für die Einleitung
%
% (c) 2020 Prof Dr Andreas Müller, Hochschule Rapperswil
%
\section{Teil 0\label{fresnel:section:teil0}}
\rhead{Teil 0}
Lorem ipsum dolor sit amet, consetetur sadipscing elitr, sed diam
nonumy eirmod tempor invidunt ut labore et dolore magna aliquyam
erat, sed diam voluptua \cite{fresnel:bibtex}.
At vero eos et accusam et justo duo dolores et ea rebum.
Stet clita kasd gubergren, no sea takimata sanctus est Lorem ipsum
dolor sit amet.

Lorem ipsum dolor sit amet, consetetur sadipscing elitr, sed diam
nonumy eirmod tempor invidunt ut labore et dolore magna aliquyam
erat, sed diam voluptua.
At vero eos et accusam et justo duo dolores et ea rebum.  Stet clita
kasd gubergren, no sea takimata sanctus est Lorem ipsum dolor sit
amet.



%
% teil1.tex -- Beispiel-File für das Paper
%
% (c) 2020 Prof Dr Andreas Müller, Hochschule Rapperswil
%
\section{Lösung
\label{parzyl:section:teil1}}
\rhead{Problemstellung}
Die Differentialgleichungen \eqref{parzyl:sep_dgl_1} und \eqref{parzyl:sep_dgl_2} können mit
Hilfe der Whittaker Gleichung gelöst werden.
\begin{definition}
    Die Funktion 
    \begin{equation*}
        W_{k,m}(z) = 
    e^{-z/2} z^{m+1/2} \,
    {}_{1} F_{1}
    (
        {\textstyle \frac{1}{2}} 
        + m - k, 1 + 2m; z)
    \end{equation*}
    heisst Whittaker Funktion und ist eine Lösung
    von der Whittaker Differentialgleichung
    \begin{equation}
        \frac{d^2W}{d z^2} +
        \left(-\frac{1}{4}  + \frac{k}{z} + \frac{\frac{1}{4} - m^2}{z^2} \right) W = 0.
        \label{parzyl:eq:whitDiffEq}
    \end{equation}
\end{definition}
Es wird nun die Differentialgleichung bestimmt, welche
\begin{equation}
    w = z^{-1/2} W_{k,-1/4} \left({\textstyle \frac{1}{2}} z^2\right)
\end{equation}
als Lösung hat.
Dafür wird $w$ in \eqref{parzyl:eq:whitDiffEq} eingesetzt woraus
\begin{equation}
    \frac{d^2 w}{dz^2} - \left(\frac{1}{4} z^2 - 2k\right) w = 0
\label{parzyl:eq:weberDiffEq}
\end{equation}
resultiert. DIese Differentialgleichung ist dieselbe wie 
\eqref{parzyl:sep_dgl_2} und \eqref{parzyl:sep_dgl_2}, welche somit
$w$ als Lösung haben.
Da es sich um eine Differentialgleichung zweiter Ordnung handelt, hat sie nicht nur
eine sondern zwei Lösungen.
Die zweite Lösung der Whittaker-Gleichung ist $W_{k,-m} (z)$.
Somit hat \eqref{parzyl:eq:weberDiffEq}
\begin{align}
    w_1 & = z^{-1/2} W_{k,-1/4} \left({\textstyle \frac{1}{2}} z^2\right)\\
    w_2 & = z^{-1/2} W_{k,1/4} \left({\textstyle \frac{1}{2}} z^2\right)
\end{align}
als Lösungen.

Ausgeschrieben ergeben sich als Lösungen
\begin{align}
    w_1 &= e^{-z^2/4} \,
    {}_{1} F_{1}
    (
        {\textstyle \frac{1}{4}} 
         - k, {\textstyle \frac{1}{2}} ; {\textstyle \frac{1}{2}}z^2) \\
    w_2 & = z e^{-z^2/4} \,
         {}_{1} F_{1}
         ({\textstyle \frac{3}{4}} 
              - k, {\textstyle \frac{3}{2}} ; {\textstyle \frac{1}{2}}z^2)
\end{align}
%
% teil2.tex -- Umsetzung in C Programmen
%
% (c) 2022 Fabian Dünki, Hochschule Rapperswil
%
\section{Umsetzung
\label{0f1:section:teil2}}
\rhead{Umsetzung}
Zur Umsetzung wurden drei verschiedene Ansätze gewählt, die in
vollständiger Form auf Github \cite{0f1:code} zu finden sind.
Dabei wurde der Schwerpunkt auf die Funktionalität und eine gute
Lesbarkeit des Codes gelegt.
Die Unterprogramme wurde jeweils, wie die GNU Scientific Library,
\index{GNU Scientific Library}%
in C geschrieben.
Die Zwischenresultate wurden vom Hauptprogramm
in einem CSV-File gespeichert.
\index{CSV}%
Anschliessen wurde mit der Matplot-Library
\index{Matplot-Library}%
\index{Python}%
in Python die Resultate geplottet.

\subsection{Potenzreihe
\label{0f1:subsection:potenzreihe}}
Die naheliegendste Lösung ist die Programmierung der Potenzreihe
\begin{align}
    \label{0f1:umsetzung:0f1:eq}
    \mathstrut_0F_1(;c;z)
    &=
    \sum_{k=0}^\infty
    \frac{z^k}{(c)_k \cdot k!}
    &= 
    \frac{1}{c}
    +\frac{z^1}{(c+1) \cdot 1}
    + \cdots
    + \frac{z^{20}}{c(c+1)(c+2)\cdots(c+19) \cdot 2.4 \cdot 10^{18}}.
\end{align}

\lstinputlisting[style=C,float,caption={Potenzreihe.},label={0f1:listing:potenzreihe}, firstline=59]{papers/0f1/listings/potenzreihe.c}

\subsection{Kettenbruch
\label{0f1:subsection:kettenbruch}}
Eine weitere Variante zur Berechnung von $\mathstrut_0F_1(;c;z)$ ist die Umsetzung als Kettenbruch.
\index{Kettenbruch}
Der Vorteil einer Umsetzung als Kettenbruch gegenüber der Potenzreihe ist die schnellere Konvergenz.

\subsubsection{Grundlage}
Ein endlicher Kettenbruch \cite{0f1:wiki-kettenbruch} ist ein Bruch der Form
\begin{equation*}
a_0 + \cfrac{b_1}{a_1+\cfrac{b_2}{a_2+\cfrac{b_3}{a_3+\cdots}}},
\end{equation*}
in welchem $a_0, a_1,\dots,a_n$ und $b_1,b_2,\dots,b_n$ ganze Zahlen sind.

\subsubsection{Rekursionsbeziehungen und Kettenbrüche}

Nimmt man die Gleichung
Wenn es für die analytischen Funktionen $f_i(z)$ eine Relation der Form
\begin{equation}
	f_{i-1}(z) - f_i(z) = k_i z f_{i+1}(z)
\label{0f1:relation}
\end{equation}
für ganzzahlige positive $i$ und Konstanten $k_i$
gibt,
dann gibt es einen Kettenbruch für das Verhältnis
$\frac{f_i(z)}{f_{i-1}(z)}$ \cite{0f1:wiki-fraction}. 
Aus der Relation~\eqref{0f1:relation}
ergibt sich der Zusammenhang
\begin{equation}
	\cfrac{f_i(z)}{f_{i-1}(z)}
	=
	\cfrac{1}{1+k_iz\cfrac{f_{i+1}(z)}{f_i(z)}}.
\label{0f1:bruchrelation}
\end{equation}
Geht man einen Schritt weiter und nimmt für
$g_i(z) = \frac{f_i(z)}{f_{i-1}(z)}$ an, kommt man zur Formel
\begin{equation*}
	g_i(z) = \cfrac{1}{1+k_izg_{i+1}(z)}.
\end{equation*}
Setzt man dies nun für $g_1$ in den Bruch ein, ergibt sich
\begin{equation*}
	g_1(z) = \cfrac{f_1(z)}{f_0(z)} = \cfrac{1}{1+k_izg_2(z)} = \cfrac{1}{1+\cfrac{k_1z}{1+k_2zg_3(z)}} = \cdots
\end{equation*}
Wiederholt man dies unendlich, erhält man einen Kettenbruch in der Form:
\begin{equation}
	\label{0f1:math:rekursion:eq}
	\cfrac{f_1(z)}{f_0(z)}
	=
	\cfrac{1}{1+\cfrac{k_1z}{1+\cfrac{k_2z}{1+\cfrac{k_3z}{\cdots}}}}.
\end{equation}

\subsubsection{Rekursion für $\mathstrut_0F_1$}
Angewendet auf die Potenzreihe
\begin{equation}
	\label{0f1:math:potenzreihe:0f1:eq}
	\mathstrut_0F_1(;c;z)
	=
	1 + \frac{z}{c\cdot1!} + \frac{z^2}{c(c+1)\cdot2!} + \frac{z^3}{c(c+1)(c+2)\cdot3!} + \cdots
\end{equation}
kann durch Substitution bewiesen werden, dass
\begin{equation*}
	\mathstrut_0F_1(;c-1;z) - \mathstrut_0F_1(;c;z)
	=
	\frac{z}{c(c-1)} \cdot \mathstrut_0F_1(;c+1;z)
\end{equation*}
eine Relation der Art \eqref{0f1:relation} dazu ist.
Wenn man für $f_i$ und $k_i$ die Annahme
\begin{align*}
	f_i &= \mathstrut_0F_1(;c+i;z)\\
	k_i	&= \frac{1}{(c+i)(c+i-1)}
\end{align*}
trifft und in die Formel \eqref{0f1:math:rekursion:eq} einsetzt, erhält man:
\begin{equation*}
	\cfrac{\mathstrut_0F_1(;c+1;z)}{\mathstrut_0F_1(;c;z)}
	=
	\cfrac{1}{1+\cfrac{\cfrac{z}{c(c+1)}}{1+\cfrac{\cfrac{z}{(c+1)(c+2)}}{1+\cfrac{\cfrac{z}{(c+2)(c+3)}}{\cdots}}}}.
\end{equation*}

\subsubsection{Algorithmus}
Da mit obigen Formeln nur ein Verhältnis zwischen
$\frac{\mathstrut_0F_1(;c+1;z)}{\mathstrut_0F_1(;c;z)}$
berechnet wurde, braucht es weitere Relationen um $\mathstrut_0F_1(;c;z)$
zu erhalten.
So ergeben ähnliche Relationen nach Wolfram Alpha \cite{0f1:wolfram-0f1} den Kettenbruch
\begin{equation}
	\label{0f1:math:kettenbruch:0f1:eq}
	\mathstrut_0F_1(;c;z) = 1 + \cfrac{\cfrac{z}{c}}{1+\cfrac{-\cfrac{z}{2(c+1)}}{1+\cfrac{z}{2(c+1)}+\cfrac{-\cfrac{z}{3(c+2)}}{1+\cfrac{z}{5(c+4)} + \cdots}}},
\end{equation}
der als Code (Listing \ref{0f1:listing:kettenbruchIterativ})  umgesetzt wurde. 

\lstinputlisting[style=C,float,caption={Iterativ umgesetzter Kettenbruch.},label={0f1:listing:kettenbruchIterativ},  firstline=8]{papers/0f1/listings/kettenbruchIterativ.c}

\subsection{Rekursionsformel
\label{0f1:subsection:rekursionsformel}}
Wesentlich stabiler zur Berechnung eines Kettenbruches ist die
Rekursionsformel.
\index{Rekursionsformel}%
Nachfolgend wird die verkürzte Herleitung vom
Kettenbruch zur Rekursionsformel aufgezeigt.
Eine vollständige Schritt für Schritt Herleitung ist im Seminarbuch Numerik
\cite{0f1:kettenbrueche}
im Kapitel Kettenbrüche zu finden.

\subsubsection{Herleitung}
Ein Näherungsbruch in der Form
\index{Näherungsbruch}%
\begin{align*}
	\cfrac{A_k}{B_k} = a_k + \cfrac{b_{k + 1}}{a_{k + 1} + \cfrac{p}{q}}
\end{align*}
lässt sich zu
\begin{align*}
	\cfrac{A_k}{B_k} = \cfrac{b_{k+1}}{a_{k+1} + \cfrac{p}{q}} = \frac{b_{k+1} \cdot q}{a_{k+1} \cdot q + p}
\end{align*}
umformen.
Dies lässt sich auch durch die Matrizenschreibweise
\index{Matrixschreibeweise eines Kettenbruchs}%
\begin{equation*}
	\begin{pmatrix}
		A_k\\
		B_k
	\end{pmatrix}
	= 		\begin{pmatrix}
		b_{k+1} \cdot q\\
		a_{k+1} \cdot q + p
	\end{pmatrix}
	=\begin{pmatrix}
		0&	b_{k+1}\\
		1&	a_{k+1}
	\end{pmatrix}
	\begin{pmatrix}
		p \\
		q
	\end{pmatrix}
	%\label{0f1:math:rekursionsformel:herleitung}
\end{equation*}
ausdrücken.
Wendet man dies nun auf den Kettenbruch in der Form
\begin{equation*}
	\frac{A_k}{B_k} = a_0 + \cfrac{b_1}{a_1+\cfrac{b_2}{a_2+\cfrac{\cdots}{\cdots+\cfrac{b_{k-1}}{a_{k-1} + \cfrac{b_k}{a_k}}}}}
\end{equation*}
an, ergibt sich die Matrixdarstellungen:
\begin{align*}
	\begin{pmatrix}
		A_k\\
		B_k
	\end{pmatrix}
	&=
	\begin{pmatrix}
		1& a_0\\
		0& 1
	\end{pmatrix}
	\begin{pmatrix}
		0& b_1\\
		1& a_1
	\end{pmatrix}
	\cdots
	\begin{pmatrix}
		0& b_{k-1}\\
		1& a_{k-1}
	\end{pmatrix}
	\begin{pmatrix}
		b_k\\
		a_k
	\end{pmatrix}.
\end{align*}
Nach vollständiger Induktion ergibt sich für den Schritt $k$, die Matrix
\begin{equation}
	\label{0f1:math:matrix:ende:eq}
	 \begin{pmatrix}
		A_{k}\\
		B_{k}			
	\end{pmatrix} 
	=
		\begin{pmatrix}
		A_{k-2}& A_{k-1}\\
		B_{k-2}& B_{k-1}			
	\end{pmatrix}
		\begin{pmatrix}
		b_k\\
		a_k
	\end{pmatrix}.
\end{equation}
Und schlussendlich kann der Näherungsbruch
\[
\frac{A_k}{B_k}
\] 
berechnet werden.

\subsubsection{Algorithmus}
Die Berechnung von $A_k, B_k$ gemäss \eqref{0f1:math:matrix:ende:eq} kann man auch ohne die Matrizenschreibweise \cite{0f1:kettenbrueche} aufschreiben:
\begin{itemize}
\item Startbedingungen:
\begin{align*}
A_{-1} &= 0		&		A_0 &= a_0 \\
B_{-1} &= 1		&		B_0 &= 1 
\end{align*}
\item Schritt $k\to k+1$:
\[
\begin{aligned}
\label{0f1:math:loesung:eq}
A_{k+1} &= A_{k-1} \cdot b_k + A_k \cdot a_k \\
B_{k+1} &= B_{k-1} \cdot b_k + B_k \cdot a_k
\end{aligned}
\]
\item
Näherungsbruch: \qquad$\displaystyle\frac{A_k}{B_k}$.
\end{itemize}
Ein grosser Vorteil dieser Umsetzung
als Rekursionsformel
(Listing~\ref{0f1:listing:kettenbruchRekursion}) ist,
dass im Vergleich zum Code (Listing~\ref{0f1:listing:kettenbruchIterativ})
eine Division gespart werden kann und somit weniger Rundungsfehler
entstehen können.

%Code
\lstinputlisting[style=C,float,caption={Rekursionsformel für Kettenbruch.},label={0f1:listing:kettenbruchRekursion},  firstline=8]{papers/0f1/listings/kettenbruchRekursion.c}

%
% teil3.tex -- Beispiel-File für Teil 3
%
% (c) 2020 Prof Dr Andreas Müller, Hochschule Rapperswil
%
\section{Eigenschaften
\label{parzyl:section:Eigenschaften}}
\rhead{Eigenschaften}

\subsection{Potenzreihenentwicklung
	\label{parzyl:potenz}}
Die parabolischen Zylinderfunktionen, welche in Gleichung \ref{parzyl:eq:solution_dgl} gegeben sind, können auch als Potenzreihen geschrieben werden
\begin{align}
	w_1(k,z)
	&=  
	e^{-z^2/4} \,
	{}_{1} F_{1}
	(
	{\textstyle \frac{1}{4}} 
	- k, {\textstyle \frac{1}{2}} ; {\textstyle \frac{1}{2}}z^2) 
	= 
	e^{-\frac{z^2}{4}}
	\sum^{\infty}_{n=0}
	\frac{\left ( \frac{1}{4} - k \right )_{n}}{\left ( \frac{1}{2}\right )_{n}}
	\frac{\left ( \frac{1}{2} z^2\right )^n}{n!} \\
	&=
	e^{-\frac{z^2}{4}}
	\left ( 
	1 
	+
	\left ( \frac{1}{2} - 2k \right )\frac{z^2}{2!}
	+
	\left ( \frac{1}{2} - 2k \right )\left ( \frac{5}{2} - 2k \right )\frac{z^4}{4!}  
	+
	\dots
	\right )
\end{align}
und
\begin{align}
	w_2(k,z)
	&=  
	ze^{-z^2/4} \,
	{}_{1} F_{1}
	(
	{\textstyle \frac{3}{4}} 
	- k, {\textstyle \frac{3}{2}} ; {\textstyle \frac{1}{2}}z^2) 
	= 
	ze^{-\frac{z^2}{4}}
	\sum^{\infty}_{n=0}
	\frac{\left ( \frac{3}{4} - k \right )_{n}}{\left ( \frac{3}{2}\right )_{n}}
	\frac{\left ( \frac{1}{2} z^2\right )^n}{n!} \\
	&=
	e^{-\frac{z^2}{4}}
	\left ( 
	z 
	+
	\left ( \frac{3}{2} - 2k \right )\frac{z^3}{3!}
	+
	\left ( \frac{3}{2} - 2k \right )\left ( \frac{7}{2} - 2k \right )\frac{z^5}{5!}  
	+
	\dots
	\right ).
\end{align}
Bei den Potenzreihen sieht man gut, dass die Ordnung des Polynoms im generellen ins unendliche geht. Es gibt allerdings die Möglichkeit für bestimmte k das die Terme in der Klammer gleich null werden und das Polynom somit eine endliche Ordnung $n$ hat. Dies geschieht bei $w_1(k,z)$ falls
\begin{equation}
	k = \frac{1}{4} + n \qquad n \in \mathbb{N}_0
\end{equation}
und bei $w_2(k,z)$ falls
\begin{equation}
	k = \frac{3}{4} + n \qquad n \in \mathbb{N}_0.
\end{equation}

\subsection{Ableitung}
Es kann gezeigt werden, dass die Ableitungen $\frac{\partial w_1(z,k)}{\partial z}$ und $\frac{\partial w_2(z,k)}{\partial z}$ einen Zusammenhang zwischen $w_1(z,k)$ und $w_2(z,k)$ zeigen. Die Ableitung von $w_1(z,k)$ nach $z$ kann über die Produktregel berechnet werden und ist gegeben als
\begin{equation}
	\frac{\partial w_1(z,k)}{\partial z} = \left (\frac{1}{2} - 2k \right ) w_2(z, k -\frac{1}{2}) - \frac{1}{2} z w_1(z,k),
\end{equation} 
und die Ableitung von $w_2(z,k)$ als
\begin{equation}
	\frac{\partial w_2(z,k)}{\partial z} = w_1(z, k -\frac{1}{2}) - \frac{1}{2} z w_2(z,k).
\end{equation}
Über diese Eigenschaft können einfach weitere Ableitungen berechnet werden. 



% \printbibliography[heading=subbibliography]
\end{refsection}
