\section{Jacobische elliptische Funktionen}

Für das elliptische Filter werden, wie es der Name bereits deutet, elliptische Funktionen gebraucht.
Wie die trigonometrischen Funktionen Zusammenhänge eines Kreises darlegen, beschreiben die elliptischen Funktionen Ellipsen.
Es ist daher naheliegend, dass der Kosinus des Tschebyscheff-Filters gegen ein elliptisches Pendant ausgetauscht werden könnte.
Der Begriff elliptische Funktion wird für sehr viele Funktionen gebraucht, daher ist es hier wichtig zu erwähnen, dass es hier ausschliesslich um die Jacobischen elliptischen Funktionen geht.

\subsection{Grundlegende Eigenschaften}

Die Jacobi elliptischen Funktionen werden ausführlich im Kapitel \ref{buch:elliptisch:section:jacobi} behandelt.
Im Wesentlichen erweitern die Jacobi elliptischen Funktionen die trigonometrische Funktionen für Ellipsen.
Zum Beispiel gibt es analog zum Sinus den elliptischen $\sn(z, k)$.
Im Gegensatz zum den trigonometrischen Funktionen haben die elliptischen Funktionen zwei Parameter.
Den elliptischen Modul $k$, der die Exzentrizität der Ellipse parametrisiert und das Winkelargument $z$.
Im Kreis ist der Radius für alle Winkel konstant, bei Ellipsen ändert sich das.
Dies hat zur Folge, dass bei einer Ellipse die Kreisbogenlänge nicht linear zum Winkel verläuft.
Darum kann hier nicht der gewohnte Winkel verwendet werden.
Das Winkelargument $z$ kann durch das elliptische Integral erster Art
\begin{equation}
    z
    =
    F(\phi, k)
    =
    \int_{0}^{\phi}
    \frac{
        d\theta
    }{
        \sqrt{
            1-k^2 \sin^2 \theta
        }
    }
\end{equation}
mit dem Winkel $\phi$ in Verbindung gebracht werden.

Dabei wird das vollständige und unvollständige elliptische integral unterschieden.
Beim vollständigen Integral
\begin{equation}
    K(k)
    =
    \int_{0}^{\pi / 2}
    \frac{
        d\theta
    }{
        \sqrt{
            1-k^2 \sin^2 \theta
        }
    }
\end{equation}
wird über ein viertel Ellipsenbogen integriert, also bis $\phi=\pi/2$ und liefert das Winkelargument für eine Vierteldrehung.
Die Zahl wird oft auch abgekürzt mit $K = K(k)$ und ist für das elliptische Filter sehr relevant.
Alle elliptischen Funktionen sind somit $4K$-periodisch.

Neben dem $\sn$ gibt es zwei weitere elliptische Basisfunktionen $\cn$ und $\dn$.
Dazu kommen noch weitere abgeleitete Funktionen, die durch Quotienten und Kehrwerte dieser Funktionen zustande kommen.
Insgesamt sind es die zwölf Funktionen
\begin{equation*}
    \sn \quad
    \ns \quad
    \scelliptic \quad
    \sd \quad
    \cn \quad
    \nc \quad
    \cs \quad
    \cd \quad
    \dn \quad
    \nd \quad
    \ds \quad
    \dc.
\end{equation*}

Die Jacobischen elliptischen Funktionen können mit der inversen Funktion des vollständigen elliptischen Integrals erster Art
\begin{equation}
    \phi = F^{-1}(z, k)
\end{equation}
definiert werden. Dabei ist zu beachten dass nur das $z$ Argument der Funktion invertiert wird, also
\begin{equation}
    z = F(\phi, k)
    \Leftrightarrow
    \phi = F^{-1}(z, k).
\end{equation}
Mithilfe von $F^{-1}$ kann zum Beispiel $sn^{-1}$ mit dem elliptischen Integral dargestellt werden:
\begin{equation}
    \sin(\phi)
    =
    \sin \left( F^{-1}(z, k) \right)
    =
    \sn(z, k)
    =
    w.
\end{equation}

% \begin{equation} %TODO remove unnecessary equations
%     \phi
%     =
%      F^{-1}(z, k)
%      =
%      \sin^{-1} \big( \sn (z, k ) \big)
%      =
%     \sin^{-1} ( w )
% \end{equation}

% \begin{equation}
%     F(\phi, k)
%     =
%     z
%     =
%     F( \sin^{-1} \big( \sn (z, k ) \big) , k)
%     =
%     F( \sin^{-1} ( w ), k)
% \end{equation}

% \begin{equation}
%     \sn^{-1}(w, k)
%     =
%     F(\phi, k),
%     \quad
%     \phi = \sin^{-1}(w)
% \end{equation}

\subsection{Die Funktion $\sn^{-1}$}

Beim Tschebyscheff-Filter konnten wir mit Betrachten des Arcuscosinus die Funktionalität erklären.
Für das Elliptische Filter machen wir die gleiche Betrachtung mit der $\sn^{-1}$-Funktion.
Der $\sn^{-1}$ ist durch das elliptische Integral
\begin{align}
    \sn^{-1}(w, k)
        & =
    \int_{0}^{\phi}
    \frac{
        d\theta
    }{
        \sqrt{
            1-k^2 \sin^2 \theta
        }
    },
    \quad
    \phi = \sin^{-1}(w)
    \\
        & =
    \int_{0}^{w}
    \frac{
        dt
    }{
        \sqrt{
            (1-t^2)(1-k^2 t^2)
        }
    }
\end{align}
beschrieben.
Dazu betrachten wir wieder den Integranden
\begin{equation}
    \frac{
        1
    }{
        \sqrt{
            (1-t^2)(1-k^2 t^2)
        }
    }.
\end{equation}
Beim $\cos^{-1}(x)$ haben wir gesehen, dass die analytische Fortsetzung bei $x < -1$ und $x > 1$ rechtwinklig in die komplexen Zahlen wandert.
Wenn man das Gleiche mit $\sn^{-1}(w, k)$ macht, erkennt man zwei interessante Stellen.
Die erste ist die gleiche wie beim $\cos^{-1}(x)$ nämlich bei $t = \pm 1$.
Der erste Term unter der Wurzel wird dann negativ, während der zweite noch positiv ist, da $k \leq 1$.
Ab diesem Punkt knickt die Funktion in die imaginäre Richtung ab.
Bei $t = 1/k$ ist auch der zweite Term negativ und die Funktion verläuft in die negative reelle Richtung.
Abbildung \ref{ellfilter:fig:sn} zeigt den Verlauf der Funktion in der komplexen Ebene.
\begin{figure}
    \centering
    \begin{tikzpicture}[>=stealth', auto, node distance=2cm, scale=1.2]

    \tikzstyle{zero} = [draw, circle, inner sep =0, minimum height=0.15cm]

    \tikzset{pole/.style={cross out, draw=black, minimum size=(0.15cm-\pgflinewidth), inner sep=0pt, outer sep=0pt}}

    \begin{scope}[xscale=0.9, yscale=1.8]

        \draw[gray, ->] (0,-1.5) -- (0,1.5) node[anchor=south]{$\mathrm{Im}~z$};
        \draw[gray, ->] (-5,0) -- (5,0) node[anchor=west]{$\mathrm{Re}~z$};

        \begin{scope}

            \clip(-4.5,-1.25) rectangle (4.5,1.25);

            \fill[yellow!30] (0,0) rectangle (1, 0.5);

            \begin{scope}[xshift=-1cm]

                \foreach \i in {-2,...,2} {
                    \foreach \j in {-2,...,1} {
                        \begin{scope}[xshift=\i*4cm, yshift=\j*1cm]
                            \draw[<-, blue!50] (0, 0) -- (0,0.5);
                            \draw[<-, cyan!50] (1, 0) -- (0,0);
                            \draw[<-, darkgreen!50] (2, 0) -- (1,0);
                            \draw[<-, orange!50] (2,0.5) -- (2, 0);
                            \draw[<-, red!50] (1, 0.5) -- (2,0.5);
                            \draw[<-, purple!50] (0, 0.5) -- (1,0.5);
                            \draw[<-, blue!50] (0,1) -- (0,0.5);
                            \draw[<-, orange!50] (2,0.5) -- (2, 1);
                            \draw[<-, red!50] (3, 0.5) -- (2,0.5);
                            \draw[<-, purple!50] (4, 0.5) -- (3,0.5);
                            \draw[<-, darkgreen!50] (2, 0) -- (3,0);
                            \draw[<-, cyan!50] (3, 0) -- (4,0);
                        \end{scope}
                    }
                }

                % \pause
                \draw[ultra thick, <-, darkgreen] (2, 0) -- (1,0);
                % \pause
                \draw[ultra thick, <-, orange] (2,0.5) -- (2, 0);
                % \pause
                \draw[ultra thick, <-, red] (1, 0.5) -- (2,0.5);
                % \pause
                \draw[ultra thick, <-, blue] (0, 0) -- (0,0.5);
                \draw[ultra thick, <-, purple] (0, 0.5) -- (1,0.5);
                \draw[ultra thick, <-, cyan] (1, 0) -- (0,0);
                % \pause


                \foreach \i in {-2,...,2} {
                    \foreach \j in {-2,...,1} {
                        \begin{scope}[xshift=\i*4cm, yshift=\j*1cm]
                            \node[zero] at ( 1, 0) {};
                            \node[zero] at ( 3, 0) {};
                            \node[pole] at ( 1,0.5) {};
                            \node[pole] at ( 3,0.5) {};
                        \end{scope}
                    }
                }

            \end{scope}

        \end{scope}

        \draw[gray] ( 1,0) +(0,0.1) -- +(0, -0.1) node[inner sep=0, anchor=north] {\small $K$};
        \draw[gray]  (0, 0.5) +(0.1, 0) -- +(-0.1, 0) node[inner sep=0, anchor=east]{\small $jK^\prime$};

    \end{scope}

    \node[zero] at (4,3) (n) {};
    \node[anchor=west] at (n.east) {Zero};
    \node[pole, below=0.25cm of n] (n) {};
    \node[anchor=west] at (n.east) {Pole};

    \begin{scope}[yshift=-4cm, xscale=0.75]

        \draw[gray, ->] (-6,0) -- (6,0) node[anchor=west]{$w$};

        \draw[ultra thick, ->, purple] (-5, 0) -- (-3, 0);
        \draw[ultra thick, ->, blue]      (-3, 0) -- (-2, 0);
        \draw[ultra thick, ->, cyan]       (-2, 0) -- (0, 0);
        \draw[ultra thick, ->, darkgreen]    (0, 0) -- (2, 0);
        \draw[ultra thick, ->, orange] (2, 0) -- (3, 0);
        \draw[ultra thick, ->, red] (3, 0) -- (5, 0);

        \node[anchor=south] at (-5,0) {$-\infty$};
        \node[anchor=south] at (-3,0) {$-1/k$};
        \node[anchor=south] at (-2,0) {$-1$};
        \node[anchor=south] at (0,0) {$0$};
        \node[anchor=south] at (2,0) {$1$};
        \node[anchor=south] at (3,0) {$1/k$};
        \node[anchor=south] at (5,0) {$\infty$};

    \end{scope}


\end{tikzpicture}
    \caption{
        $z$-Ebene der Funktion $z = \sn^{-1}(w, k)$.
        Die Funktion ist in der realen Achse $4K$-periodisch und in der imaginären Achse $2jK^\prime$-periodisch.
    }
    \label{ellfilter:fig:sn}
\end{figure}
In der reellen Richtung ist sie $4K(k)$-periodisch und in der imaginären Richtung $4K^\prime$-periodisch, wobei $K^\prime$ das komplementäre vollständige Elliptische Integral ist:
\begin{equation}
    K^\prime(k)
    =
    \int_{0}^{\pi / 2}
    \frac{
        d\theta
    }{
        \sqrt{
            1-{k^\prime}^2 \sin^2 \theta
        }
    },
    \quad
    k^\prime = \sqrt{1-k^2}.
\end{equation}
