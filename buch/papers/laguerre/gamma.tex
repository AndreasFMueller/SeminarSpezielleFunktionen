%
% gamma.tex 
%
% (c) 2022 Patrik Müller, Ostschweizer Fachhochschule
%
\section{Anwendung: Berechnung der Gamma-Funktion
  \label{laguerre:section:quad-gamma}}
Die Gauss-Laguerre-Quadratur kann nun verwendet werden,
um exponentiell abfallende Funktionen im Definitionsbereich $(0, \infty)$ zu
berechnen.
Dabei bietet sich z.B. die Gamma-Funkion bestens an, wie wir in den folgenden
Abschnitten sehen werden.

\subsection{Gamma-Funktion}
Die Gamma-Funktion ist eine Erweiterung der Fakultät auf die reale und komplexe
Zahlenmenge.
Die Definition~\ref{buch:rekursion:def:gamma} beschreibt die Gamma-Funktion als
Integral der Form
\begin{align}
\Gamma(z)
 & =
\int_0^\infty t^{z-1} e^{-t} dt
,
\quad
\text{wobei Realteil von $z$ grösser als $0$}
,
\label{laguerre:gamma}
\end{align}
welches alle Eigenschaften erfüllt, um mit der Gauss-Laguerre-Quadratur
berechnet zu werden.

\subsubsection{Funktionalgleichung}
Die Funktionalgleichung besagt
\begin{align}
z \Gamma(z) = \Gamma(z+1).
\label{laguerre:gamma_funktional}
\end{align}
Mittels dieser Gleichung kann der Wert an einer bestimmten,
geeigneten Stelle evaluiert werden und dann zurückverschoben werden,
um das gewünschte Resultat zu erhalten.

\subsection{Berechnung mittels Gauss-Laguerre-Quadratur}

Fehlerterm:
\begin{align*}
R_n
=
(z - 2n)_{2n} \frac{(n!)^2}{(2n)!} \xi^{z-2n-1}
\end{align*}

\subsubsection{Finden der optimalen Berechnungsstelle}
Nun stellt sich die Frage,
ob die Approximation mittels Gauss-Laguerre-Quadratur verbessert werden kann,
wenn man das Problem an einer geeigneten Stelle evaluiert und
dann zurückverschiebt mit der Funktionalgleichung.
Dazu wollen wir den Fehlerterm in
Gleichung~\eqref{laguerre:lagurre:lag_error} anpassen und dann minimieren.
Zunächst wollen wir dies nur für $z\in \mathbb{R}$ und $0<z<1$ definieren.
Zudem nehmen wir an, dass die optimale Stelle $x^* \in \mathbb{R}$, $z < x^*$
ist.
Dann fügen wir einen Verschiebungsterm um $m$ Stellen ein, daraus folgt
\begin{align*}
R_n
=
\frac{(z - 2n)_{2n}}{(z - m)_m} \frac{(n!)^2}{(2n)!} \xi^{z + m - 2n - 1}
.
\end{align*}

{
\large \color{red}
TODO:
Geeignete Minimierung für Fehler finden, so dass sie mit den emprisich
bestimmen optimalen Punkten übereinstimmen.
}

\subsection{Resultate}
