%
% wasserstoff.tex 
%
% (c) 2022 Patrik Müller, Ostschweizer Fachhochschule
%
\section{Radialer Schwingungsanteil eines Wasserstoffatoms
\label{laguerre:section:radial_h_atom}}

\begin{align}
    \nonumber
    - \frac{\hbar^2}{2m} 
    &
    \left( 
        \frac{1}{r^2} \pdv{}{r}
        \left( r^2 \pdv{}{r} \right)
        +
        \frac{1}{r^2 \sin \vartheta} \pdv{}{\vartheta}
        \left( \sin \vartheta \pdv{}{\vartheta} \right)
        +
        \frac{1}{r^2 \sin^2 \vartheta} \pdv[2]{}{\varphi}
    \right)
    u(r, \vartheta, \varphi)
    \\
    & -
    \frac{e^2}{4 \pi \epsilon_0 r} u(r, \vartheta, \varphi)
    =
    E u(r, \vartheta, \varphi)
    \label{laguerre:pdg_h_atom}
\end{align}
