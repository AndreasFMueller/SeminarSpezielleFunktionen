%
% wasserstoff.tex 
%
% (c) 2022 Patrik Müller, Ostschweizer Fachhochschule
%
\section{Radialer Schwingungsanteil eines Wasserstoffatoms
\label{laguerre:section:radial_h_atom}}

Das Wasserstoffatom besteht aus einem Proton im Kern 
mit Masse $M$ und Ladung $+e$.
Ein Elektron mit Masse $m$ und Ladung $-e$ umkreist das Proton
(vgl. Abbildung~\ref{laguerre:fig:wasserstoff_model}).
Für das folgende Model werden folgende Annahmen getroffen:

\begin{figure}
\centering
\includegraphics{papers/laguerre/images/wasserstoff_model.pdf}
\caption{Skizze eines Wasserstoffatoms.
Kartesische, wie auch Kugelkoordinaten sind eingezeichnet.
}
\label{laguerre:fig:wasserstoff_model}
\end{figure}

\begin{enumerate}
\item 
Das Elektron wird als nicht-relativistisches Teilchen betrachtet, 
das heisst,
relativistische Effekte sind vernachlässigbar.
\item        
Der Spin des Elektrons und des Protons
und das damit verbundene magnetische Moment
wird vernachlässigt.
\item
Fluktuationen des Vakuums werden nicht berücksichtigt.
\item
Wechselwirkung zwischen Elektron und Proton
ist durch die Coulombwechselwirkung gegeben. 
Somit entspricht die potentielle Energie der Coulombenergie $V_C(r)$
und nimmt damit die folgende Form an
\begin{align}
    V_C(r) 
    = 
    -\frac{e^2}{4 \pi \epsilon_0 r}
    \text{ mit }
    r
    =
    \lvert\vec{r}\rvert
    = 
    \sqrt{x^2 + y^2 + z^2}
    .
    \label{laguerre:coulombenergie}
\end{align}
Im Falle das der Kern einen endlichen Radius $r_0$ besitzt,
ist die $1/r$-Abhängigkeit in Gleichung \eqref{laguerre:coulombenergie}
als Näherung zu betrachten.
Diese Näherung darf nur angewendet werden, wenn die 
Aufenthaltswahrscheinlicheit des Elektrons
innerhalb $r_0$ vernachlässigbar ist.
Für das Wasserstoffatom ist diese Näherung für alle Zustände gerechtfertigt.
\item
Da $M \gg m$, kann das Proton als in Ruhe angenommen werden.
\end{enumerate}

\subsection{Herleitung zeitunabhängige Schrödinger-Gleichung}
\label{laguerre:subsection:herleitung_schroedinger}
Das Problem ist kugelsymmetrisch, 
darum transformieren wir das Problem in Kugelkoordinaten.
Somit gilt:

\begin{align*}
    r
    & =
    \sqrt{x^2 + y^2 + z^2}\\
    \vartheta
    & =
    \arccos\left(\frac{z}{r}\right)\\
    \varphi
    & =
    \arctan\left(\frac{y}{x}\right)
\end{align*}

Die potentielle Energie $V_C(r)$ hat keine direkte Zeitabhängigkeit.
Daraus folgt, dass die konstant ist Gesamtenergie $E$
und es existieren stationäre Zustände

\begin{align}
    \psi(r, \vartheta, \varphi, t)
    =
    u(r, \vartheta, \varphi) e^{-i E t / h},
\end{align}
wobei $u(r, \vartheta, \varphi)$ 
die zeitunabhängige Schrödinger-Gleichung erfüllt.

\begin{align}
    -\frac{\hbar^2}{2m} \Delta u(r, \vartheta, \varphi)
    + V_C(r) u(r, \vartheta, \varphi)
    =
    E u(r, \vartheta, \varphi)
    \label{laguerre:schroedinger}
\end{align}

Für Kugelkoordinaten hat der Laplace-Operator $\Delta$ die Form

\begin{align}
    \Delta
    =
    \frac{1}{r^2} \pdv{}{r} \left( r^2 \pdv{}{r} \right)
    + \frac{1}{r^2 \sin\vartheta} \pdv{}{\vartheta} 
    \left(\sin\vartheta \pdv{}{\vartheta}\right)
    + \frac{1}{r^2 \sin^2\vartheta} \pdv[2]{}{\varphi}
    \label{laguerre:laplace_kugel}
\end{align}

Setzt man nun 
\eqref{laguerre:coulombenergie} und \eqref{laguerre:laplace_kugel} 
in \eqref{laguerre:schroedinger} ein,
erhält man die zeitunabhängige Schrödinger-Gleichung für Kugelkoordinaten

\begin{align}
\nonumber
- \frac{\hbar^2}{2m} 
&
\left( 
\frac{1}{r^2} \pdv{}{r}
\left( r^2 \pdv{}{r} \right)
+
\frac{1}{r^2 \sin \vartheta} \pdv{}{\vartheta}
\left( \sin \vartheta \pdv{}{\vartheta} \right)
+
\frac{1}{r^2 \sin^2 \vartheta} \pdv[2]{}{\varphi}
\right)
u(r, \vartheta, \varphi)
\\
& -
\frac{e^2}{4 \pi \epsilon_0 r} u(r, \vartheta, \varphi)
=
E u(r, \vartheta, \varphi).
\label{laguerre:pdg_h_atom}
\end{align}

\subsection{Separation der Schrödinger-Gleichung}
\label{laguerre:subsection:seperation_schroedinger}
