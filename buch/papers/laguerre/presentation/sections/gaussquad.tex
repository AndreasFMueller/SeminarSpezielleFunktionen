\section{Gauss-Quadratur}

\begin{frame}{Gauss-Quadratur}
\textbf{Idee}
\begin{itemize}[<+->]
\item Polynome können viele Funktionen approximieren
\item Wenn Verfahren gut für Polynome funktioniert,
sollte es auch für andere Funktionen funktionieren
\item Integrieren eines Interpolationspolynom
\item Interpolationspolynom ist durch Funktionswerte $f(x_i)$ bestimmt
$\Rightarrow$ Integral kann durch Funktionswerte berechnet werden
\item Evaluation der Funktionswerte an geeigneten Stellen
\end{itemize}
\end{frame}

\begin{frame}{Gauss-Quadratur}
\begin{align*}
\int_{-1}^{1} f(x) \, dx
\approx
\sum_{i=1}^n f(x_i) A_i
\end{align*}

\begin{itemize}[<+->]
\item Exakt für Polynome mit Grad $2n-1$
\item Interpolationspolynome müssen orthogonal sein
\item Stützstellen $x_i$ sind Nullstellen des Polynoms
\item Fehler:
\begin{align*}
E
=
\frac{f^{(2n)}(\xi)}{(2n)!} \int_{-1}^{1} l(x)^2 \, dx
,\quad
\text{wobei }
l(x) = \prod_{i=1}^n (x-x_i)
\end{align*}
\end{itemize}
\end{frame}

\begin{frame}{Gauss-Laguerre-Quadratur}
\begin{itemize}[<+->]
\item Erweiterung des Integrationsintervall von $[-1, 1]$ auf $(a, b)$
\item Hinzufügen einer Gewichtsfunktion
\item Bei uneigentlichen Integralen muss Gewichtsfunktion schneller als jedes
Integrationspolynom gegen $0$ gehen
\item[$\Rightarrow$] Für Laguerre-Polynome haben wir den Definitionsbereich
$(0, \infty)$ und die Gewichtsfunktion $w(x) = e^{-x}$
\begin{align*}
\int_0^\infty & f(x) e^{-x} \, dx
\approx
\sum_{i=1}^n f(x_i) A_i
\\
                & \text{wobei }
A_i = \frac{x_i}{(n+1)^2 \left[ L_{n+1}(x_i) \right]^2}
\text{ und $x_i$ die Nullstellen von $L_n(x)$}
\end{align*}
\end{itemize}
\end{frame}

\begin{frame}{Fehler der Gauss-Laguerre-Quadratur}
\begin{align*}
R_n
=
\frac{(n!)^2}{(2n)!} f^{(2n)}(\xi)
,\quad
0 < \xi < \infty
\end{align*}
\end{frame}