%
% definition.tex 
%
% (c) 2022 Patrik Müller, Ostschweizer Fachhochschule
%
\section{Definition
\label{laguerre:section:definition}}
\rhead{Definition}
Die Laguerre-Differentialgleichung ist gegeben durch
\begin{align}
x y''(x) + (1 - x) y'(x) + n y(x)
=
0
, \quad
n \in \mathbb{N}_0
, \quad
x \in \mathbb{R}
.
\label{laguerre:dgl}
\end{align}
Zur Lösung der Gleichung \eqref{laguerre:dgl}
verwenden wir einen Potenzreihenansatz.
Setzt man nun den Ansatz
\begin{align*}
y(x) 
&= 
\sum_{k=0}^\infty a_k x^k
\\
y'(x)
& = 
\sum_{k=1}^\infty k a_k x^{k-1}
=
\sum_{k=0}^\infty (k+1) a_{k+1} x^k
\\
y''(x)
&=
\sum_{k=2}^\infty k (k-1) a_k x^{k-2}
=
\sum_{k=1}^\infty (k+1) k a_{k+1} x^{k-1}
\end{align*}
in die Differentialgleichung ein, erhält man:
\begin{align*}
\sum_{k=1}^\infty (k+1) k a_{k+1} x^k
+ \sum_{k=0}^\infty (k+1) a_{k+1} x^k
- \sum_{k=0}^\infty k a_k x^k
+ n \sum_{k=0}^\infty a_k x^k 
&= 
0\\
\sum_{k=0}^\infty
\left[ (k+1) k a_{k+1} + (k+1) a_{k+1} - k a_k + n a_k \right] x^k
&=
0.
\end{align*}
Daraus lässt sich die Rekursionsbeziehung
\begin{align*}
a_{k+1}
&= 
\frac{k-n}{(k+1) ^ 2} a_k
\end{align*}
ableiten.
Für ein konstantes $n$ erhalten wir als Potenzreihenlösung ein Polynom vom Grad $n$,
denn für $k=n$ wird $a_{n+1} = 0$ und damit auch $a_{n+2}=a_{n+3}=\ldots=0$.
Aus der Rekursionsbeziehung ist zudem ersichtlich, 
dass $a_0 \neq 0$ beliebig gewählt werden kann.
Wählen wir nun $c_0 = 1$, dann folgt für die Koeffizienten $a_1, a_2, a_3$
\begin{align*}
a_1 
= 
-\frac{n}{1^2}
,&&
a_2 
= 
\frac{(n-1)n}{1^2 2^2}
,&&
a_3
=
-\frac{(n-2)(n-1)n}{1^2 2^2 3^2}
\end{align*}
und allgemein
\begin{align*}
k&\leq n:
&
a_k 
&=
(-1)^k \frac{n!}{(n-k)!} \frac{1}{(k!)^2} 
= 
\frac{(-1)^k}{k!}
\begin{pmatrix}
n
\\
k
\end{pmatrix}
\\
k&>n:
&
a_k
&=
0.
\end{align*}
Somit haben wir die Laguerre-Polynome $L_n(x)$ erhalten:
\begin{align}
L_n(x)
=
\sum_{k=0}^{n} 
\frac{(-1)^k}{k!}
\begin{pmatrix}
n \\
k
\end{pmatrix}
x^k
\label{laguerre:polynom}
\end{align}

\subsection{Assoziierte Laguerre-Polynome
\label{laguerre:subsection:assoz_laguerre}
}
\begin{align}
x y''(x) + (\alpha + 1 - x) y'(x) + n y(x)
=
0 
\label{laguerre:generell_dgl}
\end{align}

\begin{align}
L_n^\alpha (x)
=
\sum_{k=0}^{n} 
\frac{(-1)^k}{k!}
\begin{pmatrix}
n + \alpha \\
n - k
\end{pmatrix}
x^k
\label{laguerre:polynom}
\end{align}

% https://www.math.kit.edu/iana1/lehre/hm3phys2012w/media/laguerre.pdf
% http://www.physics.okayama-u.ac.jp/jeschke_homepage/E4/kapitel4.pdf
