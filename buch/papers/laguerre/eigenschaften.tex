%
% eigenschaften.tex 
%
% (c) 2022 Patrik Müller, Ostschweizer Fachhochschule
%
\subsection{Orthogonalität%
\label{laguerre:subsection:orthogonal}}
\rhead{Orthogonalität}%
Im Abschnitt~\ref{laguerre:subsection:potenzreihenansatz}
haben wir die Behauptung aufgestellt,
dass die Laguerre-Polynome orthogonal sind.
Zu dieser Behauptung möchten wir nun einen Beweis liefern.
%
Um die Orthogonalität von Funktionen zu zeigen,
bieten sich folgende Möglichkeiten an:
\begin{enumerate}
\item Identifizieren der Funktion als Eigenfunktion eines Skalarproduktes
mit einem selbstadjungierten Operator.
Dafür muss aber zuerst bewiesen werden,
dass der verwendete Operator selbstadjungiert ist.
Die Theorie dazu findet sich in den
Abschnitten~\ref{buch:orthogonal:section:orthogonale-polynome-und-dgl} und
\ref{buch:orthogonalitaet:section:bessel}.
\item Umformen der Differentialgleichung in die Form der
Sturm-Liouville-Differentialgleichung,
denn für dieses verallgemeinerte Problem
ist die Orthogonalität bereits bewiesen.
Die Theorie dazu findet sich im Abschnitt~\ref{buch:integrale:subsection:sturm-liouville-problem}.
\end{enumerate}

% \subsubsection{Plan}
\subsubsection{Idee}
Für den Beweis der Orthogonalität der Laguerre-Polynome möchten
wir den zweiten Ansatz über das Sturm-Liouville-Problem verwenden.
% Dazu müssen wir die Laguerre-Differentialgleichung~\eqref{laguerre:dgl}
% in die Form der Sturm-Liouville-Differentialgleichung bringen.
Allerdings möchten wir nicht die Laguerre-Differentialgleichung
in die richtige Form bringen,
sondern den Laguerre-Operator
\begin{align}
\Lambda
=
x \frac{d}{dx^2} + (\nu + 1 -x) \frac{d}{dx}
\label{laguerre:lagop}
.
\end{align}
Da es sich beim Sturm-Liouville-Problem um ein Eigenwertproblem handelt,
kann die Orthogonalität äquivalent über denn Sturm-Liouville-Operator
\begin{align}
S
=
\frac{1}{w(x)} \left(-\frac{d}{dx}p(x) \frac{d}{dx} + q(x) \right).
\label{laguerre:slop}
\end{align}
bewiesen werden.
Dazu müssen wir die Operatoren einander gleichsetzen.

% Wenn wir \eqref{laguerre:dgl} in ein
% Sturm-Liouville-Problem umwandeln können, haben wir bewiesen, dass es sich
% bei den Laguerre-Polynomen um orthogonale Polynome handelt (siehe
% Abschnitt~\ref{buch:integrale:subsection:sturm-liouville-problem}).
% Der Beweis kann äquivalent auch über den Sturm-Liouville-Operator
% \begin{align}
% S
% =
% \frac{1}{w(x)} \left(-\frac{d}{dx}p(x) \frac{d}{dx} + q(x) \right).
% \label{laguerre:slop}
% \end{align}
% und den Laguerre-Operator
% \begin{align}
% \Lambda
% =
% x \frac{d}{dx^2} + (\nu + 1 -x) \frac{d}{dx}
% \end{align}
% erhalten werden,
% indem wir diese Operatoren einander gleichsetzen.

\subsubsection{Umformen in Sturm-Liouville-Operator}
% Aus der Beziehung von
Setzen wir nun 
\eqref{laguerre:lagop} und \eqref{laguerre:slop}
einander gleich
\begin{align}
S
 & =
\Lambda
\nonumber
\\
\frac{1}{w(x)} \left(-\frac{d}{dx}p(x) \frac{d}{dx} + q(x) \right)
 & =
x \frac{d^2}{dx^2} + (\nu + 1 - x) \frac{d}{dx}
\label{laguerre:sl-lag}
,
\end{align}
lässt sich sofort erkennen, dass $q(x) = 0$.
Ausserdem ist ersichtlich, dass $p(x)$ die Differentialgleichung
\begin{align*}
x \frac{dp}{dx}
=
(\nu + 1 - x) p
\end{align*}
erfüllen muss.
Durch Separation erhalten wir dann
\begin{align*}
\int \frac{dp}{p}
 & =
\int \frac{\nu + 1 - x}{x} \, dx
=
\int \frac{\nu + 1}{x} \, dx - \int 1\, dx
\\
\log p
 & =
(\nu + 1)\log x - x + c
\\
p(x)
 & =
C x^{\nu + 1} e^{-x}
.
\end{align*}
Eingefügt in Gleichung~\eqref{laguerre:sl-lag} ergibt sich
\begin{align*}
\frac{C}{w(x)}
\left(
-x^{\nu+1} e^{-x} \frac{d^2}{dx^2} -
(\nu + 1 - x) x^{\nu} e^{-x} \frac{d}{dx}
\right)
=
x \frac{d^2}{dx^2} + (\nu + 1 - x) \frac{d}{dx}.
\end{align*}
Mittels Koeffizientenvergleich kann nun abgelesen werden,
dass $w(x) = x^\nu e^{-x}$ und $C=-1$. %mit $\nu \geq 0$.
Die Gewichtsfunktion $w(x)$ wächst für $x\rightarrow-\infty$ sehr schnell an.
Ausserdem hat die Gewichtsfunktion $w(x)$ für negative $\nu$ einen Pol bei $x=0$,
daher ist die Laguerre-Gewichtsfunktion nur für den
Definitionsbereich $(0, \infty)$ geeignet.

\subsubsection{Randbedingungen}
Bleibt nur noch sicherzustellen, dass die Randbedingungen
\begin{align}
k_0 y(0) + h_0 p(0)y'(0)
 & =
0
\label{laguerre:sllag_randa}
\\
k_\infty y(\infty) + h_\infty p(\infty) y'(\infty)
 & =
0
\label{laguerre:sllag_randb}
\end{align}
mit $|k_i|^2 + |h_i|^2 \neq 0,\,\forall i \in \{0, \infty\}$, erfüllt sind.
%
Am linken Rand \eqref{laguerre:sllag_randa} kann $y(0) = 1$, $k_0 = 0$ und
$h_0 = 1$ verwendet werden,
was auch die Laguerre-Polynome ergeben haben.

Für den rechten Rand ist die Bedingung \eqref{laguerre:sllag_randb}
\begin{align*}
\lim_{x \rightarrow \infty} p(x) y'(x)
 & =
\lim_{x \rightarrow \infty} -x^{\nu + 1} e^{-x} y'(x)
=
0
\end{align*}
für beliebige Polynomlösungen erfüllt für $k_\infty=0$ und $h_\infty=1$.

% Somit können wir schlussfolgern:
\begin{satz}
Die Laguerre-Polynome %($\nu=0$)
\eqref{laguerre:polynom}
% \begin{align*}
% L_n(x)
% =
% \sum_{k=0}^{n} \frac{(-1)^k}{k!} \binom{n}{k} x^k
% \end{align*}
sind orthognale Polynome bezüglich des Skalarproduktes
im Intervall~$(0, \infty)$ mit der Gewichts\-funktion~$w(x)=e^{-x}$.
\end{satz}

\begin{satz}
Die assoziierten Laguerre-Polynome \eqref{laguerre:allg_polynom}
% \begin{align*}
% L_n^\nu(x)
% =
% \sum_{k=0}^{n} \frac{(-1)^k}{(\nu + 1)_k} \binom{n}{k} x^k.
% \end{align*}
sind orthogonale Polynome bezüglich des Skalarproduktes 
im Intervall~$(0, \infty)$ mit der Gewichts\-funktion~$w(x)=x^\nu e^{-x}$.
\end{satz}
