%
% eigenschaften.tex 
%
% (c) 2022 Patrik Müller, Ostschweizer Fachhochschule
%
\section{Eigenschaften
  \label{laguerre:section:eigenschaften}}
{
\large \color{red}
TODO:
Evtl. nur Orthogonalität hier behandeln, da nur diese für die Gauss-Quadratur
benötigt wird.
}

Die Laguerre-Polynome besitzen einige interessante Eigenschaften
\rhead{Eigenschaften}

\subsection{Orthogonalität
  \label{laguerre:subsection:orthogonal}}
Im Abschnitt~\ref{laguerre:section:definition} haben wir behauptet,
dass die Laguerre-Polynome orthogonale Polynome sind.
Zu dieser Behauptung möchten wir nun einen Beweis liefern.
Wenn wir die Laguerre\--Differentialgleichung in ein
Sturm\--Liouville\--Problem umwandeln können, haben wir bewiesen, dass es sich
bei
den Laguerre\--Polynomen um orthogonale Polynome handelt (siehe
Abschnitt~\ref{buch:integrale:subsection:sturm-liouville-problem}).
Der Sturm-Liouville-Operator hat die Form
\begin{align}
S
=
\frac{1}{w(x)} \left(-\frac{d}{dx}p(x) \frac{d}{dx} + q(x) \right).
\label{laguerre:slop}
\end{align}
Aus der Beziehung
\begin{align}
S
 & =
\Lambda
\nonumber
\\
\frac{1}{w(x)} \left(-\frac{d}{dx}p(x) \frac{d}{dx} + q(x) \right)
 & =
x \frac{d^2}{dx^2} + (\nu + 1 - x) \frac{d}{dx}
\label{laguerre:sl-lag}
\end{align}
lässt sich sofort erkennen, dass $q(x) = 0$.
Ausserdem ist ersichtlich, dass $p(x)$ die Differentialgleichung
\begin{align*}
x \frac{dp}{dx}
=
-(\nu + 1 - x) p,
\end{align*}
erfüllen muss.
Durch Separation erhalten wir dann
\begin{align*}
\int \frac{dp}{p}
 & =
-\int \frac{\nu + 1 - x}{x}dx
\\
\log p
 & =
-\log \nu + 1 - x + C
\\
p(x)
 & =
-C x^{\nu + 1} e^{-x}
\end{align*}
Eingefügt in Gleichung~\eqref{laguerre:sl-lag} erhalten wir
\begin{align*}
\frac{C}{w(x)}
\left(
x^{\nu+1} e^{-x} \frac{d^2}{dx^2} +
(\nu + 1 - x) x^{\nu} e^{-x} \frac{d}{dx}
\right)
=
x \frac{d^2}{dx^2} + (\nu + 1 - x) \frac{d}{dx}.
\end{align*}
Mittels Koeffizientenvergleich kann nun abgelesen werden, dass $w(x) = x^\nu
e^{-x}$ und $C=1$ mit $\nu > -1$.
Die Gewichtsfunktion $w(x)$ wächst für $x\rightarrow-\infty$ sehr schnell an,
deshalb ist die Laguerre-Gewichtsfunktion nur geeignet für den
Definitionsbereich $(0, \infty)$.
Bleibt nur noch sicherzustellen, dass die Randbedingungen,
\begin{align}
k_0 y(0) + h_0 p(0)y'(0)
 & =
0
\label{laguerre:sllag_randa}
\\
k_\infty y(\infty) + h_\infty p(\infty) y'(\infty)
 & =
0
\label{laguerre:sllag_randb}
\end{align}
mit $|k_i|^2 + |h_i|^2 \neq 0,\,\forall i \in \{0, \infty\}$, erfüllt sind.
Am linken Rand (Gleichung~\eqref{laguerre:sllag_randa}) kann $y(0) = 1$, $k_0 =
0$ und $h_0 = 1$ verwendet werden,
was auch die Laguerre-Polynome ergeben haben.
Für den rechten Rand ist die Bedingung (Gleichung~\eqref{laguerre:sllag_randb})
\begin{align*}
\lim_{x \rightarrow \infty} p(x) y'(x)
 & =
\lim_{x \rightarrow \infty} -x^{\nu + 1} e^{-x} y'(x)
=
0
\end{align*}
für beliebige Polynomlösungen erfüllt für $k_\infty=0$ und $h_\infty=1$.
Damit können wir schlussfolgern, dass die Laguerre-Polynome orthogonal
bezüglich des Skalarproduktes auf dem Intervall $(0, \infty)$ mit der Laguerre\--Gewichtsfunktion 
$w(x)=x^\nu e^{-x}$ sind.


\subsection{Rodrigues-Formel}

\subsection{Drei-Terme Rekursion}

\subsection{Beziehung mit der Hypergeometrischen Funktion}

