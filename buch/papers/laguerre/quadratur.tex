%
% quadratur.tex 
%
% (c) 2022 Patrik Müller, Ostschweizer Fachhochschule
%
\section{Gauss-Quadratur%
  \label{laguerre:section:quadratur}}
\rhead{Gauss-Quadratur}%
Die Gauss-Quadratur ist ein numerisches Integrationsverfahren,
\index{Gauss-Quadratur}%
welches die Eigenschaften von orthogonalen Polynomen verwendet.
Herleitungen und Analysen der Gauss-Quadratur können im
Abschnitt~\ref{buch:orthogonal:section:gauss-quadratur} gefunden werden.
Als grundlegende Idee wird die Beobachtung,
dass viele Funktionen sich gut mit Polynomen approximieren lassen,
verwendet.
Stellt man also sicher,
dass ein Verfahren gut für Polynome funktioniert,
sollte es auch für andere Funktionen angemessene Resultate liefern.
Es wird ein Interpolationspolynom verwendet,
welches an den Punkten $x_0 < x_1 < \ldots < x_n$
die Funktionwerte~$f(x_i)$ annimmt.
Als Resultat kann das Integral via einer gewichteten Summe der Form
\begin{align}
\int_a^b f(x) w(x) \, dx
\approx
\sum_{i=1}^n f(x_i) A_i
\label{laguerre:gaussquadratur}
\end{align}
berechnet werden.
Die Gauss-Quadratur ist exakt für Polynome mit Grad höchstens $2n -1$,
wenn ein Interpolationspolynom von Grad $n$ gewählt wurde.

\subsection{Gauss-Laguerre-Quadratur%
\label{laguerre:subsection:gausslag-quadratur}}
Wir möchten nun die Gauss-Quadratur auf die Berechnung
von uneigentlichen Integralen erweitern,
spezifisch auf das Intervall~$(0, \infty)$.
Mit dem vorher beschriebenen Verfahren ist dies nicht direkt möglich.
% Mit einer Transformation
% \begin{align*}
% x
% =
% % a + 
% \frac{1 - t}{t}
% \end{align*}
% die das unendliche Intervall~$(0, \infty)$ 
% auf das Intervall~$[0, 1]$ transformiert,
% kann dies behoben werden.
% % Für unseren Fall gilt $a = 0$.
Das Integral eines Polynomes in diesem Intervall ist immer divergent.
Es ist also nötig,
den Integranden durch Funktionen zu approximieren,
die genügend schnell gegen $0$ gehen.
Man kann Polynome beliebigen Grades verwenden,
wenn sie mit einer Funktion multipliziert werden,
die schneller gegen $0$ geht als jedes Polynom.
Damit stellen wir sicher,
dass das Integral immer noch konvergiert.
% Darum müssen wir das Polynom mit einer Funktion multiplizieren,
% die schneller als jedes Polynom gegen $0$ geht,
% damit das Integral immer noch konvergiert.
Die Laguerre-Polynome $L_n$ schaffen hier Abhilfe,
da ihre Gewichtsfunktion $w(x) = e^{-x}$ schneller
gegen $0$ konvergiert als jedes Polynom.
% In unserem Falle möchten wir die Gauss Quadratur auf die Laguerre-Polynome
% $L_n$ ausweiten.
% Diese sind orthogonal im Intervall $(0, \infty)$ bezüglich
% der Gewichtsfunktion $e^{-x}$.
Um also das Integral einer Funktion $g(x)$ im Intervall~$(0,\infty)$ zu
berechen,
formt man das Integral wie folgt um:
\begin{align*}
\int_0^\infty g(x) \, dx
=
\int_0^\infty f(x) e^{-x} \, dx
\end{align*}
Wir approximieren dann $f(x)$ durch ein Interpolationspolynom
wie bei der Gauss-Quadratur.
% Die Gleichung~\eqref{laguerre:gaussquadratur} lässt sich daher wie folgt
% umformulieren:
Die Gleichung~\eqref{laguerre:gaussquadratur} wird also
für die Gauss-Laguerre-Quadratur zu
\index{Gauss-Laguerre-Quadratur}%
\begin{align}
\int_{0}^{\infty} f(x) e^{-x} dx
\approx
\sum_{i=1}^{n} f(x_i) A_i
\label{laguerre:laguerrequadratur}
.
\end{align}

\subsubsection{Stützstellen und Gewichte}
Nach der Definition der Gauss-Quadratur müssen als Stützstellen die Nullstellen
\index{Gauss-Quadratur}%
\index{Stützstellen}%
\index{Nullstellen}%
des Approximationspolynoms genommen werden.
Für das Laguerre-Polynom $L_n(x)$ müssen demnach dessen Nullstellen $x_i$ und
als Gewichte $A_i$ die Integrale von $l_i(x) e^{-x}$ verwendet werden.
\index{Gewicht}%
Dabei sind
\begin{align*}
l_i(x_j)
=
\delta_{ij}
=
\begin{cases}
1 & i=j          \\
0 & \text{sonst}
\end{cases}
% .
\end{align*}
die Lagrangeschen Interpolationspolynome.
\index{Lagrange-Interpolationspolynom}%
Laut \cite{laguerre:hildebrand2013introduction} können die Gewichte mit
\begin{align*}
A_i
 & =
-\frac{C_{n+1} \gamma_n}{C_n \phi'_n(x_i) \phi_{n+1} (x_i)}
\end{align*}
berechnet werden.
$C_i$ entspricht dabei dem Koeffizienten von $x^i$
des orthogonalen Polynoms $\phi_n(x)$, $\forall i =0,\ldots,n$ und
\begin{align*}
\gamma_n
=
\int_0^\infty w(x) \phi_n^2(x)\,dx
\end{align*}
als Normalierungsfaktor.
\index{Normierungsfaktor}%

Wir setzen nun $\phi_n(x) = L_n(x)$ und
nutzen den Vorzeichenwechsel der Laguerre-Koeffizienten
(ersichtlich am Term $(-1)^k$ in \eqref{laguerre:polynom})
aus,
damit erhalten wir
\begin{align*}
A_i
 & =
-\frac{C_{n+1} \gamma_n}{C_n L'_n(x_i) L_{n+1} (x_i)}
\\
 & = \frac{C_n}{C_{n-1}} \frac{\gamma_{n-1}}{L_{n-1}(x_i) L'_n(x_i)}
.
\end{align*}
Für Laguerre-Polynome gilt
\begin{align*}
\frac{C_n}{C_{n-1}}
=
-\frac{1}{n}
\quad \text{und} \quad
\gamma_n
=
1
.
\end{align*}
Daraus folgt
\begin{align}
A_i
 & =
- \frac{1}{n L_{n-1}(x_i) L'_n(x_i)}
\label{laguerre:gewichte_lag_temp}
.
\end{align}
Nun kann die Rekursionseigenschaft der Laguerre-Polynome
\index{Rekursionseigenschaft}%
\cite{laguerre:hildebrand2013introduction}
% (siehe \cite{laguerre:hildebrand2013introduction})
\begin{align*}
x L'_n(x)
 & =
n L_n(x) - n L_{n-1}(x)
\\
 & = (x - n - 1) L_n(x) + (n + 1) L_{n+1}(x)
\end{align*}
umgeformt werden und da $x_i$ die Nullstellen von $L_n(x)$ sind,
vereinfacht sich die Gleichung zu
\begin{align*}
x_i L'_n(x_i)
 & =
- n L_{n-1}(x_i)
\\
 & =
(n + 1) L_{n+1}(x_i)
.
\end{align*}
Setzen wir diese Beziehung nun in \eqref{laguerre:gewichte_lag_temp} ein,
ergibt sich
\begin{align}
\nonumber
A_i
 & =
\frac{1}{x_i \left[ L'_n(x_i) \right]^2}
\\
 & =
\frac{x_i}{(n+1)^2 \left[ L_{n+1}(x_i) \right]^2}
.
\label{laguerre:quadratur_gewichte}
\end{align}

\subsubsection{Fehlerterm}
\index{Fehler}
Die Gauss-Laguerre-Quadratur mit $n$ Stützstellen berechnet Integrale
von Polynomen bis zum Grad $2n - 1$ exakt.
Für beliebige Funktionen kann eine Fehlerabschätzung angegeben werden.
Der Fehlerterm $R_n$ folgt direkt aus der Approximation
\begin{align*}
\int_0^{\infty} f(x) e^{-x} \, dx
=
\sum_{i=1}^n f(x_i) A_i + R_n
\end{align*}
und \cite{laguerre:abramowitz+stegun} gibt ihn als
\begin{align}
R_n
 & =
\frac{f^{(2n)}(\xi)}{(2n)!} \int_0^\infty l(x)^2 e^{-x}\,dx
\\
 & =
\frac{(n!)^2}{(2n)!} f^{(2n)}(\xi)
,\quad
0 < \xi < \infty
\label{laguerre:lag_error}
\end{align}
an.
Der Fehler ist also abhängig von der $2n$-ten Ableitung
der zu integrierenden Funktion.
