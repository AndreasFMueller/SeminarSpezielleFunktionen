%
% quadratur.tex 
%
% (c) 2022 Patrik Müller, Ostschweizer Fachhochschule
%
\section{Gauss-Quadratur
  \label{laguerre:section:quadratur}}
 {\large \color{red} TODO: Einleitung und kurze Beschreibung Gauss-Quadratur}
 
Siehe Abschnitt~\ref{buch:orthogonalitaet:section:gauss-quadratur}
\begin{align}
\int_a^b f(x) w(x) \, dx
\approx
\sum_{i=1}^n f(x_i) A_i
\label{laguerre:gaussquadratur}
\end{align}

\subsection{Gauss-Laguerre-Quadratur
\label{laguerre:subsection:gausslag-quadratur}}
Die Gauss-Quadratur kann auch auf Skalarprodukte mit Gewichtsfunktionen
ausgeweitet werden.
In unserem Falle möchten wir die Gauss Quadratur auf die Laguerre-Polynome
$L_n$ ausweiten.
Diese sind orthogonal im Intervall $(0, \infty)$ bezüglich
der Gewichtsfunktion $e^{-x}$.
Gleichung~\eqref{laguerre:laguerrequadratur} lässt sich wie folgt umformulieren:
\begin{align}
\int_{0}^{\infty} f(x) e^{-x} dx
\approx
\sum_{i=1}^{n} f(x_i) A_i
\label{laguerre:laguerrequadratur}
\end{align}

\subsubsection{Stützstellen und Gewichte}
Nach der Definition der Gauss-Quadratur müssen als Stützstellen die Nullstellen
des verwendeten Polynoms genommen werden.
Das heisst für das Laguerre-Polynom $L_n$ müssen dessen Nullstellen $x_i$ und
als Gewichte $A_i$ die Integrale $l_i(x)e^{-x}$ verwendet werden.
Dabei sind
\begin{align*}
l_i(x_j)
=
\delta_{ij}
=
\begin{cases}
1 & i=j           \\
0 & \text{sonst.}
\end{cases}
\end{align*}
Laut \cite{abramowitz+stegun} sind die Gewichte
\begin{align}
A_i
=
\frac{x_i}{(n + 1)^2 \left[ L_{n + 1}(x_i)\right]^2}
.
\label{laguerre:quadratur_gewichte}
\end{align}

\subsubsection{Fehlerterm}
Der Fehlerterm $R_n$ folgt direkt aus der Approximation
\begin{align*}
\int_0^{\infty} f(x) e^{-x} \, dx
=
\sum_{i=1}^n f(x_i) A_i + R_n
\end{align*}
und \cite{abramowitz+stegun} gibt ihn als
\begin{align}
R_n
=
\frac{(n!)^2}{(2n)!} f^{(2n)}(\xi)
,\quad
0 < \xi < \infty
\label{laguerre:lag_error}
\end{align}
an.

{
\large \color{red}
TODO:
Noch mehr Text / bessere Beschreibungen in allen Abschnitten
}
