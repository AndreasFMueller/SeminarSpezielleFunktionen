\def\cas{\operatorname{cas}}
Die Funktion $\cas$ definiert durch
$\cas x = \cos x + \sin x$ hat einige interessante Eigenschaften.
Wie die gewöhnlichen trigonometrischen Funktionen $\sin x$ und $\cos x$
ist $\cas x$ $2\pi$-periodisch.
Die Ableitung und das Additionstheorem benötigen bei den gewöhnlichen
trigonometrischen Funktionen aber beide Funktionen, im Gegensatz zu den
im folgenden hergeleiteten Formeln, die nur die Funktion $\cas x$ brauchen.
\begin{teilaufgaben}
\item
Drücken Sie die Ableitung von $\cas x$  allein durch Werte der
$\cas$-Funktion aus.
\item
Zeigen Sie, dass 
\[
\cas x
=
\sqrt{2} \sin\biggl(x+\frac{\pi}4\biggr)
=
\sqrt{2} \cos\biggl(x-\frac{\pi}4\biggr).
\]
\item
Beweisen Sie das Additionstheorem für die $\cas$-Funktion
\begin{equation}
\cas(x+y)
=
\frac12\bigl(
\cas(x)\cas(y) + \cas x\cas (-y) + \cas(-x)\cas(y) -\cas(-x)\cas(-y)
\bigr)
\label{buch:geometrie:uebung3:eqn:addition}
\end{equation}
\end{teilaufgaben}
Youtuber Dr Barker hat die Funktion $\cas$ im Video
{\small\url{https://www.youtube.com/watch?v=bn38o3u0lDc}} vorgestellt.

\begin{loesung}
\begin{teilaufgaben}
\item
Die Ableitung ist
\[
\frac{d}{dx}\cas x
=
\frac{d}{dx}(\cos x + \sin x)
=
-\sin x + \cos x
=
\sin(-x) + \cos(-x)
=
\cas(x).
\]
\item
Die Additionstheoreme angewendet auf die trigonometrischen Funktionen
auf der rechten Seite ergibt
\begin{align*}
\sin\biggl(x+\frac{\pi}4\biggr)
&=
\sin x \cos\frac{\pi}4  + \cos x \sin\frac{\pi}4
&&&
\cos\biggl(x-\frac{\pi}4\biggr)
&=
\cos(x)\cos\frac{\pi}4 -\sin x \sin\biggl(-\frac{\pi}4\biggr)
\\
&=
\frac{1}{\sqrt{2}} \sin x
+
\frac{1}{\sqrt{2}} \cos x
&&&
&=
\frac{1}{\sqrt{2}} \cos x
+
\frac{1}{\sqrt{2}} \sin x
\\
&=\frac{1}{\sqrt{2}} \cas x
&&&
&=
\frac{1}{\sqrt{2}} \cas x.
\end{align*}
Multiplikation mit $\sqrt{2}$ ergibt die behaupteten Relationen.
\item
Substituiert man die Definition von $\cas(x)$ auf der rechten Seite von
\eqref{buch:geometrie:uebung3:eqn:addition} und multipliziert aus,
erhält man
\begin{align*}
\eqref{buch:geometrie:uebung3:eqn:addition}
&=
{\textstyle\frac12}\bigl(
(\cos x + \sin x)
(\cos y + \sin y)
+
(\cos x + \sin x)
(\cos y - \sin y)
\\
&\qquad
+
(\cos x - \sin x)
(\cos y + \sin y)
-
(\cos x - \sin x)
(\cos y - \sin y)
\bigr)
\\
&=
\phantom{-\mathstrut}
{\textstyle\frac12}\bigl(
\cos x\cos y
+
\cos x\sin y
+
\sin x\cos y
+
\sin x\sin y
\\
&
\phantom{=-\mathstrut{\textstyle\frac12}\bigl(}\llap{$\mathstrut +\mathstrut$}
\cos x\cos y
-
\cos x\sin y
+
\sin x\cos y
-
\sin x\sin y
\\
&
\phantom{=-\mathstrut{\textstyle\frac12}\bigl(}\llap{$\mathstrut +\mathstrut$}
\cos x\cos y
+
\cos x\sin y
-
\sin x\cos y
-
\sin x\sin y
\bigr)
\\
&
\phantom{=}
-\mathstrut{\textstyle\frac12}\bigl(
\cos x\cos y
-
\cos x\sin y
-
\sin x\cos y
+
\sin x\sin y
\bigr)
\\
&= \cos x \cos y
+
\cos x \sin y
+
\sin x \cos y
-
\sin x \sin y.
\intertext{Die äussersten zwei Terme passen zum Additionstheorem für den
Kosinus, die beiden inneren Terme dagegen zum Sinus.
Fasst man sie zusammen, erhält man}
&=
(\sin x\cos y + \cos x \sin y)
+
(\cos x\cos y - \sin x \sin y)
\\
&=
\sin (x+y) + \cos(x+y)
=
\cas(x+y).
\end{align*}
Damit ist das Additionstheorem für die Funktion $\cas$ bewiesen.
\qedhere
\end{teilaufgaben}
\end{loesung}
