Berechnen Sie $\sin\alpha$ und $\cos\alpha$ für den Winkel
$\alpha=5\nicefrac{5}{8}^\circ$ exakt.

\begin{loesung}
Der Winkel $\alpha=5\nicefrac{5}{8}^\circ$ ist
$\alpha=\bigl(\frac{45}{8}\bigr)^\circ$, also ein Sechzehntel eines rechten
Winkels.
Den Wert von $\sin\alpha$ und $\cos\alpha$ erhält man also, indem
man dreimal die Halbwinkelformeln
\begin{align*}
\sin\frac{\alpha}2
&=
\sqrt{\frac{1-\cos\alpha}{2}}
&
\cos\frac{\alpha}2
&=
\sqrt{\frac{1+\cos\alpha}{2}}
\intertext{auf $\sin 45^\circ=\cos 45^\circ=1/\sqrt{2}$ anwendet:}
\sin\biggl(\frac{45}2\biggr)^\circ
&=
\sqrt{\frac{2-\sqrt{2\mathstrut}}{4}}
&
\cos\biggl(\frac{45}2\biggr)^\circ
&=
\sqrt{\frac{2+\sqrt{2\mathstrut}}{4}}
\\
&=
0.382683432365090
&
&=
0.923879532511287
\\
\intertext{auf $\sin 45^\circ=\cos 45^\circ=1/\sqrt{2}$ anwendet:}
\sin\biggl(\frac{45}4\biggr)^\circ
&=
\sqrt{\frac12-\frac12
\sqrt{\frac{2+\sqrt{2\mathstrut}}{4}}
}
&
\cos\biggl(\frac{45}4\biggr)^\circ
&=
\sqrt{\frac12+\frac12
\sqrt{\frac{2+\sqrt{2\mathstrut}}{4}}
}
\\
&= 0.195090322016128
&
&= 0.980785280403230
\\
\sin\biggl(\frac{45}8\biggr)^\circ
&=
\sqrt{\frac12-\frac12
\sqrt{\frac12+\frac12
\sqrt{\frac{2+\sqrt{2\mathstrut}}{4}}
}
}
&
\cos\biggl(\frac{45}8\biggr)^\circ
&=
\sqrt{\frac12+\frac12
\sqrt{\frac12+\frac12
\sqrt{\frac{2+\sqrt{2\mathstrut}}{4}}
}
}
\\
&= 0.098017140329560
&
&= 0.995184726672197
\qedhere
\end{align*}
\end{loesung}
