Finden Sie $x$ so, dass $\sin x = 2$.

\begin{loesung}
Es ist klar, dass die Lösung nicht reell sein kann, da reelle
Argumente immer nur Sinus-Werte zwischen $-1$ und $1$ ergeben kann.
Die Darstellung der Sinus-Funktion als Linearkombination von
Exponentialfunktionen ergibt
\[
\sin x = \frac{e^{ix}-e^{-ix}}{2i} = 2.
\]
Wir schreiben $y=e^{ix}$ und multiplizieren die Gleichung mit $y$,
so entsteht die quadratische Gleichung
\[
y^2-4iy-1=0
\]
mit den Lösungen
\[
y_{\pm}
=
2i\pm \sqrt{-4+1}
=
2i\pm \sqrt{-3}
=
(2\pm \sqrt{3})i
=
(2\pm\sqrt{3})e^{\frac{i\pi}2}.
\]
Davon muss jetzt der Logarithmus bestimmt werden.
Der Realteil des Logarithmus ist der Betrag von $y_\pm$:
\begin{align*}
|y_\pm| &= 2\pm \sqrt{3}
\\
\operatorname{arg} y_\pm &= \frac{\pi}2.
\end{align*}
Daraus bekommt man
\[
x_\pm
=
\frac{1}{i}
\log y_\pm
=
\frac{1}{i}
(
\log |y_\pm|
+
i\arg y_\pm
)
=
\frac{\pi}2
-i
\log(2\pm\sqrt{3})
\approx
\begin{cases}
1.5707963 - 1.3169579i\\
1.5707963 + 1.3169579i
\end{cases}
\]
Weitere Lösungen erhält man natürlich durch Addition von ganzzahligen
Vielfachen von $2\pi$.
\end{loesung}
