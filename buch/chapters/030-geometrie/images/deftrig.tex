%
% deftrig.tex -- Definition der trigonometrischen Funktionen
%
% (c) 2021 Prof Dr Andreas Müller, OST Ostschweizer Fachhochschule
%
\documentclass[tikz]{standalone}
\usepackage{amsmath}
\usepackage{times}
\usepackage{txfonts}
\usepackage{pgfplots}
\usepackage{csvsimple}
\usetikzlibrary{arrows,intersections,math,calc}
\begin{document}
\def\skala{1}
\begin{tikzpicture}[>=latex,thick,scale=\skala]

\coordinate (B) at (0,0);
\coordinate (C) at (4,0);
\coordinate (A) at (4,3);
\fill[color=gray] (B) -- (0:1.5) arc (0:{atan(3/4)}:1.5) -- cycle;
\node at ({0.5*atan(3/4)}:1) {$\alpha$};
\draw (A) -- (B) -- (C) -- cycle;
\fill (A) circle[radius=0.04];
\fill (B) circle[radius=0.04];
\fill (C) circle[radius=0.04];
\node at ($0.5*(A)+0.5*(B)$) [above left] {$c$};
\node at ($0.5*(B)+0.5*(C)$) [below] {$a$};
\node at ($0.5*(C)+0.5*(A)$) [right] {$b$};
\node at (A) [above right] {$A$};
\node at (B) [below left] {$B$};
\node at (C) [below right] {$C$};

\end{tikzpicture}
\end{document}

---
