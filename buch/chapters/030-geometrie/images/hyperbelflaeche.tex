%
% hyperbelflaeche.tex -- Argument der Hyperbelfunktionen
%
% (c) 2021 Prof Dr Andreas Müller, OST Ostschweizer Fachhochschule
%
\documentclass[tikz]{standalone}
\usepackage{amsmath}
\usepackage{times}
\usepackage{txfonts}
\usepackage{pgfplots}
\usepackage{csvsimple}
\usetikzlibrary{arrows,intersections,math}
\begin{document}
\def\skala{2}
\begin{tikzpicture}[>=latex,thick,scale=\skala]

\fill[color=blue!20]
	(0,0)
	--
	plot[domain=0:1,samples=100] ({cosh(\x)},{sinh(\x)})
	-- cycle;
\draw[color=blue] 
	(0,0)
	--
	plot[domain=0:1,samples=100] ({cosh(\x)},{sinh(\x)})
	-- cycle;

\begin{scope}
\clip (-1.8,-2) rectangle (2.5,2);
\draw[color=red,line width=1.4pt] plot[domain=-2:2,samples=100]
	({cosh(\x)},{sinh(\x)});
\draw[color=red,line width=1.4pt] plot[domain=-2:2,samples=100]
	({-cosh(\x)},{sinh(\x)});
\end{scope}

\fill[color=white] ({cosh(1)},{sinh(1)}) circle[radius=0.03];
\draw ({cosh(1)},{sinh(1)}) circle[radius=0.03];

\node at ({cosh(1)},{sinh(1)}) [right] {$\gamma(t)=(\cosh t,\sinh t)$};

\draw[->] (-1.8,0) -- (3.1,0) coordinate[label={$x$}];
\draw[->] (0,-2) -- (0,2.1) coordinate[label={right:$y$}];

\node[color=blue] at (0.8,0.3) {$t$};
\node at (0,0) [below left] {$O$};

\end{tikzpicture}
\end{document}

