%
% Was ist zu erwarten
%
\subsection*{Was ist zu erwarten?}
Spezielle Funktionen wie die eben angedeuteten werden also zu
Bausteinen, die in der Lösung algebraischer oder auch analytischer
Probleme verwendet werden können.
Die Erfahrung zeigt, dass diese Funktionen immer wieder nützlich
sind, es lohnt sich also, ihre Berechnung zum Beispiel in einer
Bibliothek zu implementieren.
Spezielle Funktionen sind in diesem Sinn eine mathematische Form
des informatischen Prinzips des ``code reuse''.

Die nachstehenden Kapitel sollen die vielfältigen Arten illustrieren,
wie diese Prinzipien zu neuen und nützlichen speziellen Funktionen
und ihren Anwendungen führen können.
Hier eine kurze Übersicht über ihren Inhalt.
\begin{enumerate}
\item
Potenzen und Wurzeln: Potenzen und Polynome sind die einfachsten
Funktionen, die sich unmittelbar aus den arithmetischen Operationen
konstruieren lassen.
Die zugehörigen Umkehrfunktionen sind die Wurzelfunktionen,
sie lösen gewisse algebraische Gleichungen.
Aus den Polynomen lassen sich weiter rationale Funktionen und
Potenzreihen konstruieren, die als wichtige Werkzeuge zur Konstruktion
spezieller Funktionen in späteren Kapiteln sind.
\item
Exponentialfunktion und Exponentialgleichungen.
Die Exponentialfunktion entsteht aus dem Zinsproblem durch Grenzwertbildung.
Jost Bürgi hat sie zur Berechnung seiner Logarithmentabelle verwendet.
Hier zeigt sich die Nützlichkeit spezieller Funktionen als Grundlage
für die numerische Rechnung: Logarithmentafeln waren über Jahrhunderte
das zentrale Werkzeug für die Durchführung von Berechnungen.
Besonders nützlich ist aber auch die Potenzreihendarstellung der
Exponentialfunktion, die meist für die Implementation in Bibliotheken
verwendet wird.
Die Lambert-$W$-schliesslich löst gewisse Exponentialgleichungen.
\item
Spezielle Funktionen aus der Geometrie.
Dieses Kapitel startet mit der langen Geschichte der trigonometrischen
Funktionen, den wahrscheinlich wichtigsten speziellen Funktionen für
geometrische Anwendungen.
Es führt aber auch die Kegelschnitte, die hyperbolischen Funktionen
und andere Parametrisierungen der Kegelschnitte ein, die später
wichtig werden.
Es beginnt auch die Diskussion einiger geometrischer Fragestellungen,
die sich oft nur durch Definition neuer spezieller Funktionen lösen
lassen, wie zum Beispiel das Problem der Kurvenlänge auf einer
Ellipse.
\item
Spezielle Funktionen und Rekursion.
Viele Probleme lassen eine Lösung in rekursiver Form zu.
Zum Beispiel lässt sich die Fakultät durch eine Rekursionsbeziehung
vollständig definieren.
Dieses Kapitel zeigt, wie sich die Fakultät zur Gamma-Funktion
$\Gamma(x)$ erweitern lässt, die für beliebige reelle $x$
definiert ist.
Sie ist aber nur die Spitze eines Eisbergs von weiteren wichtigen
Funktionen.
Die Beta-Integrale sind ebenfalls durch Rekursionsbeziehungen
charakterisiert, lassen sich durch Gamma-Funktionen ausdrücken und 
haben als Anwendung die Verteilungsfunktionen der Ordnungsstatistiken.
Lineare Differenzengleichungen sind Rekursionsgleichungen, die sich
besonders leicht mit Potenzfunktionen lösen lassen.
Alle diese Funktionen sind Speziallfälle einer sehr viel grösseren
Klasse von Funktionen, den hypergeometrischen Funktionen, die sich
durch eine Rekursionsbeziehung der Koeffizienten ihrer
Potenzreihenentwicklung auszeichnen.
Es wird sich im nächsten Kapitel zeigen, dass sie besonders gut
geeignet sind, Lösungen von linearen gewöhnlichen Differentialgleichungen zu
beschreiben.
\item
Differentialgleichungen.
Lösungsfunktionen von Differentialgleichungen sind meistens die
erste Anwendung, in der man die klassischen speziellen Funktionen
kennenlernt.
Sie entstehen mit Hilfe der Potenzreihenmethode und können daher
als hypergeometrische Funktionen geschrieben werden.
Sie sind aber von derart grosser Bedeutung für die Anwendung,
dass viele dieser Funktionen als eigenständige Funktionenfamilien
definiert worden sind.
Die Bessel-Funktionen werden in diesem Zusammenhang eingehend
behandelt.
\item
Integrale können als Lösungen sehr spezieller Differentialgleichungen
betrachtet werden.
Eine Stammfunktion $F(x)$ der Funktion $f(x)$ hat als Ableitung die
ursprüngliche Funktion: $F'(x)=f(x)$.
Während Ableiten ein einfacher, algebraischer Prozess ist, 
scheint das Finden einer Stammfunktion sehr viel anspruchsvoller
zu sein.
Spezielle Funktionen sinnvoll sein, wenn eine Stammfunktion sich nicht
mit den bereits definierten Funktionen ausdrücken lässt.
Es gibt eine systematische Methode zu entscheiden, ob eine Stammfunktion
sich durch ``elementare Funktionen'' ausdrücken lässt, sie wird oft
der Risch-Algorithmus genannt.
\item
Orthogonalität.
Mit dem Integral lassen sich auch für Funktionen Skalarprodukte
definieren.
Orthogonalität zwischen Funktionen zeichnet dann Funktionen aus, die
sich besonders gut zur Darstellung beliebiger stetiger oder
integrierbarer Funktionen eignen.
Die Fourier-Theorie und ihre vielen Varianten sind ein Resultat.
Besonders einfache orthogonale Funktionenfamilien sind die orthogonalen
Polynome, die ausserdem zu ausserordentlich genauen numerischen
Integrationsverfahren führen.
\item
Funktionentheorie.
Einige Eigenschaften der Lösungen gewöhnlicher Differentialgleichungen
sind allein mit der reellen Analysis nicht zu bewältigen.
In der Welt der speziellen Funktionen hat man aber strengere
Anforderungen an Funktionen, sie lassen sich immer auch als Funktionen
einer komplexen Variablen verstehen.
Dieses Kapitel stellt die wichtigsten Eigenschaften komplex
differenzierbarer Funktionen zusammen und wendet sie zum Beispiel
auf das Problem an, weitere Lösungen der Bessel-Differentialgleichung
zu finden.
\item
Partielle Differentialgleichungen sind eine der wichtigsten Quellen
der gewöhnlichen Differentialgleichungen, die nur mit speziellen
Funktionen gelöst werden können.
So führen rotationssymmetrische Wellenprobleme in der Ebene
ganz natürlich auf die Besselsche Differentialgleichung und damit
auf die Bessel-Funktionen als Lösungsfunktionen.
\item
Integraltransformationen.
Die trigonometrischen Funktionen sind die Grundlage der Fourier-Theorie.
Doch auch andere spezielle Funktionenfamilien können ähnlich
nützliche Integraltransformationen hergeben.
Die Bessel-Funktionen stellen sich in diesem Zusammenhang als die
Polarkoordinaten-Variante der Fourier-Theorie in der Ebene heraus.
\item
Elliptische Funktionen.
Einige der in Kapitel~\ref{buch:chapter:geometrie} angesprochenen
Fragestellungen wie der Berechnung der Bogenlänge auf einer Ellipse
lassen sich mit keiner der bisher vorgestellten Technik lösen.
In diesem Kapitel werden die elliptischen Integrale und die
zugehörigen Umkehrfunktionen vorgestellt.
Die Jacobischen elliptischen Funktionen verallgemeinern
die trigonometrischen Funktionen und können gewisse nichtlineare 
Differentialgleichungen lösen.
Sie finden auch Anwendungen im Design elliptischer Filter
(siehe Kapitel~\ref{chapter:ellfilter}).
\end{enumerate}

Natürlich ist damit das weite Gebiet der speziellen Funktionen
nur ganz grob umrissen.
Weitere Aspekte und Anwendungen werden in den Artikeln im zweiten
Teil vorgestellt.
Eine Übersicht dazu findet der Leser auf Seite~\pageref{buch:uebersicht}.

