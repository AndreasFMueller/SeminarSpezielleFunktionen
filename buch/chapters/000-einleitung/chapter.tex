%
% einleitung.tex
%
% (c) 2022 Prof Dr Andreas Müller
%
\chapter*{Einleitung\label{chapter:einleitung}}
\lhead{Einleitung}
\rhead{}
\addcontentsline{toc}{chapter}{Einleitung}
Eine Polynomgleichung wie etwa
\begin{equation}
p(x) = ax^2+bx+c = 0
\label{buch:einleitung:quadratisch}
\end{equation}
kann manchmal dadurch gelöst werden, dass man die Nullstellen errät
und damit eine Faktorisierung $p(x)=a(x-x_1)(x-x_2)$ konstruiert.
Doch im Allgemeinen wird man die Lösungsformel für quadratische 
Gleichungen verwenden, die auf quadratischem Ergänzen basiert.
Es erlaubt die Gleichung~\eqref{buch:einleitung:quadratisch} umzwandeln in
\[
\biggl(x + \frac{b}{2a}\biggr)^2
=
-\frac{c}{a} + \frac{b^2}{4a^2}
=
\frac{b^2-4ac}{4a^2}.
\]
Um diese Gleichung nach $x$ aufzulösen, muss man die inverse Funktion
der Quadratfunktion zur Verfügung haben, die Wurzelfunktion.
Dies ist wohl das älteste Beispiel einer speziellen Funktion,
die man zu dem Zweck eingeführt hat, spezielle algebraische Gleichungen
lösen zu können.
Sie liefert die bekannte Lösungsformel
\[
x=\frac{-b\pm\sqrt{b^2-4ac}}{2a}
\]
für die quadratische Gleichung.

Durch die Definition der Wurzelfunktion ist das Problem der numerischen
Berechnung der Nullstelle natürlich noch nicht gelöst, aber man hat
ein handliches mathematisches Symbol gewonnen, mit dem man die Lösungen
übersichtlich beschreiben und algebraisch manipulieren kann.
Diese Idee steht hinter allen weiteren in diesem Buch diskutierten
Funktionen: wann immer ein wichtiges mathematisches Konzept sich nicht
direkt durch die bereits entwickelten Funktionen ausdrücken lässt,
erfindet man dafür eine neue Funktion oder Familie von Funktionen.
Beispielsweise hat sich die Darstellung von Zahlen $x$ als Potenzen
einer gemeinsamen Basis, zum Beispiel $x=10^y$, als sehr nützlich
herausgestellt, um Multiplikationen auf die von Hand leichter
ausführbaren Additionen zurückzuführen.
Man braucht also die Fähigkeit, die Abhängigkeit des Exponenten $y$
von $x$ auszudrücken, mit anderen Worten, man braucht die Logarithmusfunktion.

Spezielle Funktionen wie die Wurzelfunktion und die Logarithmusfunktion
werden also zu Bausteinen, die in der Lösung algebraischer oder auch
analytischer Probleme verwendet werden können.
Die Erfahrung zeigt, dass diese Funktionen immer wieder nützlich
sind, es lohnt sich also, ihre Berechnung zum Beispiel in einer
Bibliothek zu implementieren.
Spezielle Funktionen sind in diesem Sinn eine mathematische Form
des informatischen Prinzips des ``code reuse''.

Die trigonometrischen Funktionen kann man als Lösungen des geometrischen 
Problems der Parametrisierung eines Kreises verstehen.
Alternativ kann man $\sin x$ und $\cos x$ als spezielle Lösungen der
Differentialgleichung $y''=-y$ verstehen.
Viele andere Funktionen wie die hyperbolischen Funktionen oder die
Bessel-Funktionen sind ebenfalls Lösungen spezieller Differentialgleichungen.
Auch die Theorie der partiellen Differentialgleichungen gibt Anlass
zu interessanten Lösungsfunktionen.
Die Separation des Poisson-Problems in Kugelkoordinaten führt zum Beispiel
auf die Kugelfunktionen, mit denen sich beliebige Funktionen auf einer
Kugeloberfläche analysieren und synthetisieren lassen.

Die Lösungen einer linearer gewöhnlicher Differentialgleichung können
oft mit Hilfe von Potenzreihen dargestellt werden.
So kann man zum Beispiel die Potenzreihenentwicklung der Exponentialfunktion
und der trigonometrischen Funktionen finden.
Die Konvergenz einer Potenzreihe wird aber durch Singularitäten
eingeschränkt.
Komplexe Potenzreihen ermöglichen aber, solche Stellen zu ``umgehen''.
Die Theorie der komplex differenzierbaren Funktionen bildet einen
allgemeinen Rahmen, mit solchen Funktionen umzugehen und ist zum 
Beispiel nötig, um die Bessel-Funktionen der zweiten Art zu konstruieren,
die ebenfalls Lösungen ger Bessel-Gleichung sind, aber bei $x=0$
eine Singularität aufweisen.

Die Stammfunktion $F(x)$ einer gegebenen Funktion $f(x)$ ist natürlich
auch die Lösung der besonders einfachen Differentialgleichung $F'=f$.
Ein bekanntes Beispiel ist die Stammfunktion der Wahrscheinlichkeitsdichte
\[
\varphi(x)
=
\frac{1}{\sqrt{2\pi}\sigma} e^{-\frac{(x-\mu)^2}{2\sigma^2}},
\]
der Normalverteilung, für die aber keine geschlossene Darstellung
mit bekannten Funktionen bekannt ist.
Sie kann aber durch die Fehlerfunktion
\[
\operatorname{erf}(x)
=
\frac{2}{\sqrt{\pi}} \int_0^x e^{-t^2}\,dt
\]
dargestellt werden.
Mit dem Risch-Algorithmus kann man nachweisen, dass es tatsächlich
keine Möglichkeit gibt, die Stammfunktion in geschlossener Form durch
die bereits bekannten Funktionen darzustellen, die Definition einer
neuen speziellen Funktion ist also der einzige Ausweg.
Die Fehlerfunktion ist heute in der Standardbibliothek enthalten auf
gleicher Stufe wie Wurzeln, trigonometrische Funktionen,
Exponentialfunktionen oder Logarithmen.

Die nachstehenden Kapitel sollen die vielfältigen Arten illustrieren,
wie diese Prinzipien zu neuen und nützlichen speziellen Funktionen
und ihren Anwendungen führen können.


