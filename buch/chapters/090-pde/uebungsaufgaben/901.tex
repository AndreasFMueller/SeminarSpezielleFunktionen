Die Differentialgleichung
\begin{equation}
\frac{\partial u}{\partial t} = \kappa \frac{\partial^2 u}{\partial x^2}
\qquad
\text{im Gebiet}
\qquad
(t,x)\in \Omega=\mathbb{R}^+\times (0,l)
\label{505:waermeleitungsgleichung}
\end{equation}
beschreibt die Änderung der Temperatur eines Stabes der Länge $l$.
Die homogene Randbedingung
\begin{equation}
u(t,0)=
u(t,l)=0
\label{505:homogene-randbedingung}
\end{equation}
besagt, dass der Stab an seinen Enden auf Temperatur $0$ gehalten.
Zur Lösung dieser Differentialgleichung muss auch die Temperatur
zur Zeit $t=0$ in Form einer Randbedingung
\[
u(0,x) = T_0(x)
\]
gegeben sein.
Führen Sie Separation für die
Differentialgleichung~\eqref{505:waermeleitungsgleichung}
durch und bestimmen Sie die zulässigen Werte der Separationskonstanten.

\begin{loesung}
Man verwendet den Ansatz $u(t,x)= T(t)\cdot X(x)$ und setzt diesen 
in die Differentialgleichung ein, die dadurch zu
\[
T'(t)X(x) = \kappa T(t) X''(x)
\]
wird.
Division durch $T(t)X(x)$ wird dies zu
\[
\frac{T'(t)}{T(t)}
=
\kappa
\frac{X''(x)}{X(x)}.
\]
Da die linke Seite nur von $t$ abhängt, die rechte aber nur von $x$, müssen
beide Seiten konstant sein.
Wir bezeichnen die Konstante mit $-\lambda^2$, so dass wir die beiden
gewöhnlichen Differentialgleichungen
\begin{align*}
\frac{1}{\kappa}
\frac{T'(t)}{T(t)}&=-\lambda^2
&
\frac{X''(x)}{X(x)}&=-\lambda^2
\\
T'(t)&=-\lambda^2\kappa T(t)
&
X''(x) &= -\lambda^2 X(x)
\intertext{welche die Lösungen}
T(t)&=Ce^{-\lambda^2\kappa t}
&
X(x)&= A\cos\lambda x + B\sin\lambda x
\end{align*}
haben.
Die Lösung $X(x)$ muss aber auch die homogene Randbedingung 
\eqref{505:homogene-randbedingung} erfüllen.
Setzt man $x=0$ und $x=l$ ein, folgt
\begin{align*}
0 = X(0)&=A\cos 0 + B\sin 0 = A
&
0 = X(l)&=B\sin \lambda l,
\end{align*}
woraus man schliessen kann, dass $\lambda l$ ein ganzzahliges
Vielfaches von $\pi$ ist, wir schreiben $\lambda l = k\pi$ oder
\[
\lambda = \frac{k\pi}{l}.
\]
Damit sind die möglichen Werte $\lambda$ bestimmt und man kann jetzt
auch die möglichen Lösungen aufschreiben, sie sind
\[
u(t,x)
=
\sum_{k=1}^\infty b_k e^{-k^2\pi^2\kappa t/l^2}\sin\frac{k\pi x}{l}.
\qedhere
\]
\end{loesung}
