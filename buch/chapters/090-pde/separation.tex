%
% separation.tex
%
% (c) 2021 Prof Dr Andreas Müller, OST Ostschweizer Fachhochschule
%
\section{Separationsmethode
\label{buch:pde:section:separation}}
Die Existenz der Lösung einer gewöhnlichen Differentialgleichung
ist unter einigermassen milden Bedingungen in der Nähe der
Anfangsbedingung garantiert.
Ausserdem steht eine ganze Reihe von Lösungsverfahren zur
Verfügung, nicht zuletzt das Potenzreihenverfahren, welches in
Kapitel~\ref{buch:chapter:differential} beschrieben wurde.
Das Ziel dieses Abschnitts ist eine Methode vorzustellen, mit
der partielle Differentialgleichungen auf gewöhnliche
Differentialgleichungen zurückgeführt werden können.

%
% Ansatz
%
\subsection{Separationsansatz}
Die Separationsmethode ist motiviert durch die Beobachtung, dass in
vielen partiellen Differentialgleichungen die Ableitungen nach
verschiedenen Variablen sich in verschiedenen Termen befinden und
sich daher algebraisch trennen lassen.
Für eine beliebige Funktion bringt das nicht viel, aber für
Funktionen mit einer speziellen Form kann man daraus eine Vereinfachung
ableiten.

%
% Prinzip der Separation
%
\subsubsection{Prinzip}
Die Grundlage der Separationsmethode ist die Idee, die Differentialgleichung
in zwei Teile aufzuteilen, die keine gemeinsamen Variablen enthalten.
Eine partielle Differentialgleichungen in einem zweidimensionalen
Gebiet mit den Koordinaten $x$ und $y$ soll so umgeformt
werden, dass auf der linken Seite des Gleichheitszeichens nur
die Variable $x$ vorkommt und auf der rechten nur die Variable $y$.
Es entsteht also eine Gleichung der Form
\begin{equation}
F(x) = G(y).
\label{buch:pde:ansatz:eqn:F=G}
\end{equation}
Wie so etwas gehen gehen kann wird weiter unten untersucht.

Betrachtet hält man in der Gleichung~\eqref{buch:pde:ansatz:eqn:F=G}
die Variable $x$ fest, steht links eine fest Zahl, schreiben wir 
sie $\lambda$.
Die Gleichung wird also zu
\[
\lambda = G(y),
\]
sie muss für alle $y$ gelten.
Es folgt dann, dass die rechte Seite gar nicht von $y$ abhängen kann.
Für jeden Wert von $y$ muss $G$ den gleichen Wert $\lambda$ geben.

Wenn aber $G$ konstant ist und immer den Wert $\lambda$ ergibt, dann
ist die Gleichung~\eqref{buch:pde:ansatz:eqn:F=G} auch gleichbedeutend
mit der Gleichung
\[
F(x) = \lambda,
\]
$F$ muss also auch konstant sein.

Die algebraische Trennung der beiden Variablen $x$ und $y$ hat also 
zur Folge, dass die beiden Seiten der Gleichung gar nicht varieren
können, beide Seiten müssen konstant sein.
Die Konstante ist allerdings nicht bekannt und muss im Laufe der
weiteren Lösungsschritte der Gleichung bestimmt werden.

Die Überlegungen funktionieren auch für eine grössere Zahl von
Variablen.
Entscheidend ist nur, dass die einen Variablen, zum Beispiel
$x_1,\dots,x_k$, nur auf der linken Seite vorkommen und die anderen,
wir nennen sie $x_{k+1},\dots,x_n$ nur auf der rechten.
Die Gleichung hat dann die Form
\begin{equation}
F(x_1,\dots,x_k)
=
G(x_{k+1},\dots,x_n).
\label{buch:pde:ansatz:eqn:FF=GG}
\end{equation}
Setzt man feste Werte von $x_1,\dots,x_k$ ein, ist die linke Seite
eine Zahl, die wir wieder $\lambda$ nennen können.
Es muss also für alle $x_{k+1},\dots,x_n$ gelten, dass
$G(x_{k+1},\dots,x_n)=\lambda$ ist.
Daher ist $G$ eine Konstante, sie ist gar nicht von den Variablen
abhängig.
Wenn aber die rechte Seite konstant ist, dann muss auch für alle
$x_1,\dots,x_k$ gelten, dass $F(x_1,\dots,x_k)=\lambda$ ist,
die linke Seite kann also auch nicht varieren.

\begin{prinzip}
In einer Gleichung
\[
F(x_1,\dots,x_k) = G(x_{k+1},\dots,x_n),
\]
in der die linke Seite nur von $x_1,\dots,x_k$ abhängt und die
rechte nur von $x_{k+1},\dots,x_n$ müssen beide Seiten konstant sein.
\end{prinzip}

%
% Beispiel zur Erklärung des Separationsvorgehens
%
\subsubsection{Ein Beispiel}
In der Differentialgleichung
\[
x\frac{\partial u}{\partial x}
-
y^2\frac{\partial^2 u}{\partial y^2}
=
y^4
\]
kommen die Ableitungen nach $x$ und $y$ in verschiedenen Termen vor.
Wir versuchen daher, auch die Lösungsfunktion als Summe
\[
u(x,y) = X(x) + Y(y)
\]
von Termen zu schreiben, die nur von jeweils einer Variablen abhängen.
Setzt man dies in die Differentialgleichung ein, erhält man
\[
x\frac{\partial}{\partial x}(X(x)+Y(y))
-y^2\frac{\partial}{\partial y}(X(x)+Y(y))
=
xX'(x) -y^2Y'(y)
=
y^4.
\]
Indem man den Term mit $y$ auf die rechte Seite schafft, findet man
die Gleichung
\[
xX'(x) = y^2Y'(y) + y^4,
\]
in der die Variablen $x$ und $y$ separiert sind.
Es folgt, dass beide Seiten konstant sein müssen, es gibt also eine
Konstante $\lambda$ derart, dass
\[
xX'(x) = \lambda
\qquad\text{und}\qquad
y^2Y''(y) +y^4 = \lambda.
\]
Diese beiden Gleichungen lassen sich als Differentialgleichungen in
der üblicheren Form als
\begin{align*}
X'(x) &= \frac{\lambda}{x}
&&\Rightarrow&
X(x) &= \int \frac{\lambda}{x}\,dx = \lambda \log x + C
\\
Y''(y) &= \frac{\lambda - y^4}{y^2}
&&\Rightarrow&
Y'(y)
&=
\int \frac{\lambda-y^4}{y^2}\,dy
=
-\frac{\lambda}{y}-\frac{y^3}3 + D
\\
&
&&\Rightarrow&
Y(y)
&=
\int Y'(y)\,dy
=
-\lambda \log y - \frac{y^4}{12} +Dy +E
\end{align*}
schreiben und im Falle von $X(x)$ mit einem Integral lösen.
$Y(y)$ benötigt zwei Integrationen, ist aber ansonsten nicht
schwieriger zu bestimmen.

Das Beispiel zeigt, dass ein Separationsansatz ermöglicht, eine
partielle Differntialgleichung in mehrere gewöhnliche Differentialgleichungen
zu zerlegen, eine für jede Variable, und zu lösen.

%
% Anpassung des Ansatzes an die Randbedingungen
%
\subsubsection{Separationsansatz und Randbedingungen}
Die im Beispiel gewählte Aufteilung der Lösungsfunktion in eine
Summe macht es sehr schwierig, Randbedingungen der partiellen
Differentialgleichungen in Randbedingungen der gewöhnlichen
Differentialgleichungen zu übersetzen.

Als Beispiel dieser Schwierigkeit betrachten wir die Differentialgleichung
\[
\Delta u
=
\frac{\partial^2 u}{\partial x^2}
+
\frac{\partial^2 u}{\partial y^2}
=
a u
\]
auf dem Gebiet
$\Omega = [0,a]\times [0,b] = \{(x,y)\in\mathbb{R}^2\mid 0<x<a\wedge 0<y<b\}$
mit den Randwerten $u(x,y)=0$ für Punkte auf dem Rand von $\Omega$.
Genauer:
\[
\begin{aligned}
u(0,y) &= 0,& u(a,y) &= 0&&\text{für $0<y<b$} \\
u(x,0) &= 0,& u(x,b) &= 0&&\text{für $0<x<a$}.
\end{aligned}
\]
Ein Ansatz der Form $u(x,y)=X(x) + Y(y)$ bedeutet für die
Randwerte $u(x,y)=0$, dass auf dem Rand $X(x)=-Y(y)$ gelten muss.
Das bedeutet aber, dass $X(0) = -Y(y)$, $Y$ müsste also konstant
sein.

Ein Produktansatz löst das Problem.
Wir verwenden stattdessen einen Produktansatz
$u(x,y) = X(x)\cdot Y(y)$, wobei die Funktionen $X(x)$ und $Y(y)$
nicht konstant sein sollen.
Die Randbedingungen sind
\[
\begin{aligned}
u(0,y) &= X(0) Y(y) = 0&&\Rightarrow& X(0)&=0\\
u(a,y) &= X(a) Y(y) = 0&&\Rightarrow& X(a)&=0\\
u(x,0) &= X(x) Y(0) = 0&&\Rightarrow& Y(0)&=0\\
u(x,b) &= X(x) Y(b) = 0&&\Rightarrow& Y(b)&=0.
\end{aligned}
\]
Der Produktansatz ermöglicht also, die Randbedingungen für die Funktion
$u(x,y)$ in Randbedingungen für die Funktionen $X(x)$ oder $Y(y)$
umzuwandeln.

%
% Eigenwertprobleme
%
\subsection{Eigenwertproblem}
Viele partielle Differentialgleichungen der mathematischen Physik
sind zeitabhängig, aber das räumliche Gebiet, in dem sie 
definiert sind, ist nicht von der Zeit abhängig.
Dies 

\subsubsection{Wellengleichung}
Die Schwingung einer ebenen Membran, die in ein emGebiet
$G\subset\mathbb{R}^n$ eingespannt ist, wird durch die
Wellengleichung
\begin{equation}
\frac{1}{c^2} \frac{\partial^2 u}{\partial t^2} = \Delta u,
\label{buch:pde:separation:wellengleichung}
\end{equation}
beschrieben.
Darin ist $u(t,x)$ die Auslenkung der Membran zur Zeit $t>0$ in einem
Punkt $x\in G$ des Gebietes $G$ ist.
Die Randbedingungen zerfallen in zwei Teile:
\begin{itemize}
\item
Bedingungen, die wiedergeben, dass die Membran in einen 
Rahmen eingespannt und damit unbeweglich ist.
Dies bedeutet, dass $u(t,x)=0$ für alle Zeiten $t>0$ und für 
Randpunkte $x\in\partial G$ von $G$ ist.
\item
Bedingungen, die Auslenkung und Geschwindigkeit der Membran zur
Zeit $t=0$ beschreiben, typischerweise ind er Form
\begin{align*}
u(0,x) = f(x),
\frac{\partial u}{\partial t}(0,x) = g(x)
\end{align*}
wobei $f(x)$ und $g(x)$ Funktionen auf dem Gebiet $G$ sind.
\end{itemize}

In der Zeitableitung auf der linken Seite
von~\eqref{buch:pde:separation:wellengleichung}
kommen die Ortskoordinaten nicht vor und im Laplace-Operator
auf der rechten Seite tritt die Zeit nicht auf.
Es ist daher naheliegend zu versuchen, die Lösung der Differntialgleichung
als Produkt
\[
u(t,x) = T(t) \cdot U(x)
\]
zu schreiben.
Wendet man die Differentialgleichung darauf an, wird daraus die Gleichung
\[
\frac{1}{c^2}
T''(t)\cdot U(x)
=
T(t) \cdot \Delta U(x).
\]
Indem man druch $T(t)$ und $U(x)$ teilt, entsteht die separierte Gleichung
\[
\frac{1}{c^2} \frac{T''(t)}{T(t)}
=
\frac{\Delta U(x)}{U(x)}.
\]
Die linke Seite ist nur von der Zeit abhängig, die rechte nur von den
Ortskoordinaten.
Damit ist die Differentialgleichung separiert und das Problem darauf
reduziert, die gewöhnliche Differentialgleichung 
\[
T''(t) = \lambda T(t)
\]
und die partielle Differentialgleichung
\[
\Delta U(x) = \lambda U(x)
\]
niedrigerer Dimension zu lösen.

\subsubsection{Allgemeine Situation}
Das Definitionsgebiet der partiellen Differentialgleichung ist 
also von der Form $\mathbb{R}^+\times G$, wobei $G\subset\mathbb{R}^n$
ein räumliches Gebiet ist und $\mathbb{R}^+$ die Zeitachse.
Auch die Randbedingungen zerfallen in zwei Arten:
\begin{itemize}
\item
Bedingungen über die Lösungsfunktion zur Zeit $t=0$ im inneren des
räumliche Gebietes $G$, zum Beispiel
die Anfangsauslenkung und/oder Anfangsgeschwindigkeit einer schwingenden
Saite oder Membran.
\item
Bedingungen über die Lösungsfunktion auf dem Rand $\partial G$ von
$G$ für alle Zeiten $t>0$, zum Beispiel die Bedingung, dass die
Membran fest eingespannt ist.
\end{itemize}
Oft zerfällt auch der Differentialoperator in Zeitableitungen
und einen zeitunabhängigen Teil der nur Ableitungen nach den
Ortsvariablen enthält.
Die Wellengleichung
\[
\frac{1}{c^2}
\frac{\partial^2}{\partial t^2} u
=
\Delta u
\qquad\Leftrightarrow\qquad
\biggl(
\frac{1}{c^2}\frac{\partial^2}{\partial t^2} - \Delta
\biggr) u = 0
\]
enthält Ableitungen nach der Zeit, die nicht von Ortskoordinaten
abhängig sind.
Der Laplace-Operator $\Delta$ ist nicht von der Zeitabhängig und das
Gebiet $G$ hängt ebenfalls nicht von der Zeit ab.

\subsubsection{Separation der Zeit}
Unter den gegeben Voraussetzungen ist es naheliegend, die Lösungsfunktion
$u(t,x)$ als Produkt
\[
u(t,x) = T(t) \cdot U(x),\qquad t\in\mathbb{R}^+, x\in G
\]
anzusetezen.
Die Wellengleichung wird dann
\[
\frac{1}{c^2}
T''(t)\cdot U(x)
=
T(t)\cdot\Delta U(x)
\]
und nach Separation
\[
\frac{1}{c^2} \frac{T''(t)}{T(t)}
=
\frac{\Delta U(x)}{U(x)}.
\]
Es gibt also eine gemeinsame Konstante.
Da wir Schwingungslösungen erwarten, für die $T''(t) = -\omega^2 T(t)$
ist, schreiben wir die gemeinsame Konstante als $-\lambda^2$, was
später die Formeln vereinfachen wird.
Die separierten Differentialgleichungen werden jetzt
\begin{align*}
\frac{1}{c^2}
\frac{T''(t)}{T(t)}
&=
-\lambda^2
&&\Rightarrow&
T''(t)-c^2\lambda T(t)&=0
&&\Rightarrow&
T''(t) &= A \cos(c\sqrt\lambda t) + B \sin(c \lambda t)
\\
&&&&&&&&
       &= C \cos(c \lambda t+\delta)
\\
\frac{\Delta U(x)}{U(x)}&=-\lambda^2
&&\Rightarrow&
\Delta U &= -\lambda^2 U
\end{align*}
Die letzte Gleichung für die Funktion $U(x)$ hat die Form
eines Eigenwertproblems mit dem Eigenwert $-\lambda^2$.

\begin{definition}
Eine Eigenfunktion eines Operators $L$ zum Eigenwert $\lambda$
ist eine Funktion $U$ derart, dass $LU=\lambda U$.
\end{definition}

Die Separation ermöglich also, das ursprüngliche Problem aufzuspalten
in ein Eigenwertproblem für eine nur ortsabhängige Funktion $U(x)$
und eine Schwingungsgleichung für $T(t)$.
Die Schwingungsfrequenz $c \lambda $ hängt direkt mit dem
Eigenwert zusammen.
Die Funktion $U(x)$ beschreibt die Form der Membran, die Amplitude
in jedem Punkt, der Faktor $T(t)$ beschreibt die Schwingung.


