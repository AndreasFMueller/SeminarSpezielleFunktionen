%
% kugel.tex
%
% (c) 2021 Prof Dr Andreas Müller, OST Ostschweizer Fachhochschule
%
\section{Kugelfunktionen
\label{buch:pde:section:kugel}}
Kugelsymmetrische Probleme können oft vorteilhaft in Kugelkoordinaten
beschrieben werden.
Die Separationsmethode kann auf partielle Differentialgleichungen
mit dem Laplace-Operator angewendet werden.
Die daraus resultierenden gewöhnlichen Differentialgleichungen führen
einerseits auf die Laguerre-Differentialgleichung für den radialen
Anteil sowie auf Kugelfunktionen für die Koordinaten der
geographischen Länge und Breite.

\subsection{Kugelkoordinaten}
Wir verwenden Kugelkoordinaten $(r,\vartheta,\varphi)$, wobei $r$
der Radius ist, $\vartheta$ die geographische Breite gemessen vom
Nordpol der Kugel und $\varphi$ die geographische Breite.
Der Definitionsbereich für Kugelkoordinaten ist
\[
\Omega
=
\{(r,\vartheta,\varphi)
\;|\;
r\ge 0\wedge 
0\le \vartheta\le \pi\wedge
0\le \varphi< 2\pi
\}.
\]
Die Entfernung eines Punktes von der $z$-Achse ist $r\sin\vartheta$.
Daraus lassen sich die karteischen Koordinaten eines Punktes mit Hilfe
von
\[
\begin{pmatrix}x\\y\\z\end{pmatrix}
=
\begin{pmatrix}
r\cos\vartheta\\
r\sin\vartheta\cos\varphi\\
r\sin\vartheta\sin\varphi
\end{pmatrix}.
\]
Man beachte, dass die Punkte auf der $z$-Achse keine eindeutigen
Kugelkoordinaten haben.
Sie sind charakterisiert durch $r\sin\vartheta=0$, was $\cos\vartheta=\pm1$
impliziert.
Entsprechend führen alle Werte von $\varphi$ auf den gleichen Punkt
$(0,0,\pm r)$.

\subsection{Der Laplace-Operator in Kugelkoordinaten}
Der Laplace-Operator in Kugelkoordinaten lautet
\begin{align}
\Delta
&=
\frac{1}{r^2} \frac{\partial}{\partial r}r^2\frac{\partial}{\partial r}
+
\frac{1}{r^2\sin\vartheta}\frac{\partial}{\partial\vartheta}
\sin\vartheta\frac{\partial}{\partial\vartheta}
+
\frac{1}{r^2\sin^2\vartheta}\frac{\partial^2}{\partial\varphi^2}.
\label{buch:pde:kugel:laplace1}
\intertext{Dies kann auch geschrieben werden als}
&=
\frac{\partial^2}{\partial r^2}
+
\frac{2}{r}\frac{\partial}{\partial r}
+
\frac{1}{r^2\sin\vartheta}\frac{\partial}{\partial\vartheta}
\sin\vartheta\frac{\partial}{\partial\vartheta}
+
\frac{1}{r^2\sin^2\vartheta}\frac{\partial^2}{\partial\varphi^2}
\label{buch:pde:kugel:laplace2}
\intertext{oder}
&=
\frac{1}{r}
\frac{\partial^2}{\partial r^2} r
+
\frac{1}{r^2\sin\vartheta}\frac{\partial}{\partial\vartheta}
\sin\vartheta\frac{\partial}{\partial\vartheta}
+
\frac{1}{r^2\sin^2\vartheta}\frac{\partial^2}{\partial\varphi^2}.
\label{buch:pde:kugel:laplace3}
\end{align}
Dabei ist zu berücksichtigen, dass mit der Notation gemeint ist,
dass ein Ableitungsoperator auf alles wirkt, was rechts im gleichen
Term steht.
Der Operator
\[
\frac{1}{r}
\frac{\partial^2}{\partial r^2}r
\quad\text{wirkt daher als}\quad
\frac{1}{r}
\frac{\partial^2}{\partial r^2}rf
=
\frac{1}{r}
\frac{\partial}{\partial r}\biggl(f + r\frac{\partial f}{\partial r}\biggr)
=
\frac{1}{r}
\frac{\partial f}{\partial r}
+
\frac{1}{r}
\frac{\partial f}{\partial r}
+
\frac{\partial^2f}{\partial r^2}.
=
\frac{2}{r}\frac{\partial f}{\partial r}
+
\frac{\partial^2f}{\partial r^2},
\]
was die Äquivalenz der beiden Formen
\eqref{buch:pde:kugel:laplace2}
und
\eqref{buch:pde:kugel:laplace3}
rechtfertigt.
Auch die Äquivalenz mit
\eqref{buch:pde:kugel:laplace1}
kann auf ähnliche Weise verstanden werden.

Die Herleitung dieser Formel ist ziemlich aufwendig und soll hier
nicht dargestellt werden.
Es sei aber darauf hingewiesen, dass sich für $\vartheta=\frac{\pi}2$ 
wegen $\sin\vartheta=\sin\frac{\pi}2=1$
der eingeschränkte Operator
\[
\Delta
= 
\frac{1}{r^2}\frac{\partial }{\partial r} r^2\frac{\partial}{\partial r}
+
\frac{1}{r^2}\frac{\partial^2}{\partial\varphi^2}
\]
ergibt.
Wendet man wie oben die Produktregel auf den ersten Term an, entsteht die
Form
\[
\frac{\partial^2}{\partial r^2}
+
\frac{2}{r}
\frac{\partial}{\partial r}
+
\frac{1}{r^2}\frac{\partial^2}{\partial\varphi^2}
\]
die {\em nicht} übereinstimmt mit dem Laplace-Operator in 
Polarkoordinaten~\eqref{buch:pde:kreis:laplace}.
Der Unterschied rührt daher, dass der Laplace-Operator die Krümmung
der Koordinatenlinien berücksichtigt, in diesem Fall der Meridiane.


\subsection{Separation}





