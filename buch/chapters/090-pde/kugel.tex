%
% kugel.tex
%
% (c) 2021 Prof Dr Andreas Müller, OST Ostschweizer Fachhochschule
%
\section{Kugelfunktionen
\label{buch:pde:section:kugel}}
\rhead{Kugelfunktionen}
Kugelsymmetrische Probleme können oft vorteilhaft in Kugelkoordinaten
beschrieben werden.
Die Separationsmethode kann auf partielle Differentialgleichungen
mit dem Laplace-Operator angewendet werden.
Die daraus resultierenden gewöhnlichen Differentialgleichungen führen
einerseits auf die Laguerre-Differentialgleichung für den radialen
Anteil sowie auf Kugelfunktionen für die Koordinaten der
geographischen Länge und Breite.

\subsection{Kugelkoordinaten}
Wir verwenden Kugelkoordinaten $(r,\vartheta,\varphi)$, wobei $r$
der Radius ist, $\vartheta$ die geographische Breite gemessen vom
Nordpol der Kugel und $\varphi$ die geographische Breite.
Der Definitionsbereich für Kugelkoordinaten ist
\[
\Omega
=
\{(r,\vartheta,\varphi)
\;|\;
r\ge 0\wedge 
0\le \vartheta\le \pi\wedge
0\le \varphi< 2\pi
\}.
\]
Die Entfernung eines Punktes von der $z$-Achse ist $r\sin\vartheta$.
Daraus lassen sich die karteischen Koordinaten eines Punktes mit Hilfe
von
\[
\begin{pmatrix}x\\y\\z\end{pmatrix}
=
\begin{pmatrix}
r\cos\vartheta\\
r\sin\vartheta\cos\varphi\\
r\sin\vartheta\sin\varphi
\end{pmatrix}.
\]
Man beachte, dass die Punkte auf der $z$-Achse keine eindeutigen
Kugelkoordinaten haben.
Sie sind charakterisiert durch $r\sin\vartheta=0$, was $\cos\vartheta=\pm1$
impliziert.
Entsprechend führen alle Werte von $\varphi$ auf den gleichen Punkt
$(0,0,\pm r)$.

\subsection{Der Laplace-Operator in Kugelkoordinaten}
Der Laplace-Operator in Kugelkoordinaten lautet
\begin{align}
\Delta
&=
\frac{1}{r^2} \frac{\partial}{\partial r}r^2\frac{\partial}{\partial r}
+
\frac{1}{r^2\sin\vartheta}\frac{\partial}{\partial\vartheta}
\sin\vartheta\frac{\partial}{\partial\vartheta}
+
\frac{1}{r^2\sin^2\vartheta}\frac{\partial^2}{\partial\varphi^2}.
\label{buch:pde:kugel:laplace1}
\intertext{Dies kann auch geschrieben werden als}
&=
\frac{\partial^2}{\partial r^2}
+
\frac{2}{r}\frac{\partial}{\partial r}
+
\frac{1}{r^2\sin\vartheta}\frac{\partial}{\partial\vartheta}
\sin\vartheta\frac{\partial}{\partial\vartheta}
+
\frac{1}{r^2\sin^2\vartheta}\frac{\partial^2}{\partial\varphi^2}
\label{buch:pde:kugel:laplace2}
\intertext{oder}
&=
\frac{1}{r}
\frac{\partial^2}{\partial r^2} r
+
\frac{1}{r^2\sin\vartheta}\frac{\partial}{\partial\vartheta}
\sin\vartheta\frac{\partial}{\partial\vartheta}
+
\frac{1}{r^2\sin^2\vartheta}\frac{\partial^2}{\partial\varphi^2}.
\label{buch:pde:kugel:laplace3}
\end{align}
Dabei ist zu berücksichtigen, dass mit der Notation gemeint ist,
dass ein Ableitungsoperator auf alles wirkt, was rechts im gleichen
Term steht.
Der Operator
\[
\frac{1}{r}
\frac{\partial^2}{\partial r^2}r
\quad\text{wirkt daher als}\quad
\frac{1}{r}
\frac{\partial^2}{\partial r^2}rf
=
\frac{1}{r}
\frac{\partial}{\partial r}\biggl(f + r\frac{\partial f}{\partial r}\biggr)
=
\frac{1}{r}
\frac{\partial f}{\partial r}
+
\frac{1}{r}
\frac{\partial f}{\partial r}
+
\frac{\partial^2f}{\partial r^2}.
=
\frac{2}{r}\frac{\partial f}{\partial r}
+
\frac{\partial^2f}{\partial r^2},
\]
was die Äquivalenz der beiden Formen
\eqref{buch:pde:kugel:laplace2}
und
\eqref{buch:pde:kugel:laplace3}
rechtfertigt.
Auch die Äquivalenz mit
\eqref{buch:pde:kugel:laplace1}
kann auf ähnliche Weise verstanden werden.

Die Herleitung dieser Formel ist ziemlich aufwendig und soll hier
nicht dargestellt werden.
Es sei aber darauf hingewiesen, dass sich für $\vartheta=\frac{\pi}2$ 
wegen $\sin\vartheta=\sin\frac{\pi}2=1$
der eingeschränkte Operator
\[
\Delta
= 
\frac{1}{r^2}\frac{\partial }{\partial r} r^2\frac{\partial}{\partial r}
+
\frac{1}{r^2}\frac{\partial^2}{\partial\varphi^2}
\]
ergibt.
Wendet man wie oben die Produktregel auf den ersten Term an, entsteht die
Form
\[
\frac{\partial^2}{\partial r^2}
+
\frac{2}{r}
\frac{\partial}{\partial r}
+
\frac{1}{r^2}\frac{\partial^2}{\partial\varphi^2}
\]
die {\em nicht} übereinstimmt mit dem Laplace-Operator in 
Polarkoordinaten~\eqref{buch:pde:kreis:laplace}.
Der Unterschied rührt daher, dass der Laplace-Operator die Krümmung
der Koordinatenlinien berücksichtigt, in diesem Fall der Meridiane.

\subsection{Separation}
In Abschnitt~\ref{buch:pde:subsection:eigenwertproblem}
wurde bereits gzeigt, wie die Wellengleichung
\[
\frac{1}{c^2}
\frac{\partial^2 U}{\partial t^2}
-\Delta U
=
0
\]
durch Separation der Zeit auf ein Eigenwertproblem für eine
Funktion $u$ reduziert werden kann, die nur von den Ortskoordinaten
abhängt.
Es geht also nur noch darum, dass Eigenwertproblem
\[
\Delta u = -\lambda^2 u
\]
mit geeigneten Randbedingungen zu lösen.
Dazu gehören einerseits eventuelle Gebietsränder, die im Moment
nicht interessieren.
Andererseits muss sichergestellt sein, dass die Lösungsfunktionen
stetig und differentierbar sind an den Orten, wo das Koordinatensystem
singulär ist.
So müssen $u(r,\vartheta,\varphi)$ $2\pi$-periodisch in $\varphi$ sein.
% XXX Ableitungen

\subsubsection{Separation des radialen Anteils}
Für das Eigenwertproblem verwenden wir den Ansatz
\[
u(r,\vartheta,\varphi)
=
R(r) \Theta(\vartheta) \Phi(\varphi),
\]
den wir in die Differentialgleichung einsetzen.
So erhalten wir
\[
\biggl(\frac{1}{r^2}R''(r)+\frac{2}{r}R'(r) \biggr)
\Theta(\vartheta)\Phi(\varphi)
+
R(r)
\frac{1}{r^2\sin\vartheta}
\frac{\partial}{\partial\vartheta}(\sin\vartheta \Theta'(\vartheta))
\Phi(\varphi)
+
R(r)\Theta(\vartheta)
\frac{1}{r^2\sin\vartheta} \Phi''(\varphi)
=
-\lambda^2 R(r)\Theta(\vartheta)\Phi(\varphi).
\]
Die Gleichung lässt sich nach Multiplikation mit $r^2$ und
Division durch $u$ separieren in 
\begin{equation}
\frac{R''(r)+2rR'(r)+\lambda^2r^2}{R(r)}
+
\frac{1}{\Theta(\vartheta) \sin\vartheta}
\frac{\partial}{\partial\vartheta}\sin\vartheta\Theta'(\vartheta)
+
\frac{1}{\sin^2\vartheta}\frac{\Phi''(\varphi)}{\Phi(\varphi)}
=
0
\label{buch:pde:kugel:separiert2}
\end{equation}
Der erste Term hängt nur von $r$ ab, die anderen nur von $\vartheta$ und
$\varphi$, daher muss der erste Term konstant sein.
Damit ergbit sich für den Radialanteil die gewöhnliche Differentialgleichung
\[
R''(r) + 2rR'(r) +\lambda^2 r^2 = \mu^2 R(r),
\]
die zum Beispiel mit der Potenzreihenmethode gelöst werden kann.
Sie kann aber durch eine geeignete Substition nochmals auf die
Laguerre-Differentialgleichung reduziert werden, wie in
Kapitel~\ref{chapter:laguerre} dargelegt wird.

\subsubsection{Kugelflächenanteil}
Für die Separation der verbleibenden winkelabhängigen Teile muss die
Gleichung
\[
\frac{1}{\Theta(\vartheta) \sin\vartheta}
\frac{\partial}{\partial\vartheta}\sin\vartheta\Theta'(\vartheta)
+
\frac{1}{\sin^2\vartheta}\frac{\Phi''(\varphi)}{\Phi(\varphi)}
=
-\mu^2
\]
mit $\sin^2\vartheta$ multipliziert werden, was auf
\[
\frac{\sin\vartheta}{\Theta(\vartheta)}
\frac{\partial}{\partial\vartheta}\sin\vartheta\Theta'(\vartheta)
+
\frac{\Phi''(\varphi)}{\Phi(\varphi)}
=
-\mu^2\sin^2\vartheta
\quad\Rightarrow\quad
\frac{\sin\vartheta}{\Theta(\vartheta)}
\frac{\partial}{\partial\vartheta}\sin\vartheta\Theta'(\vartheta)
+
\mu^2\sin^2\vartheta
=
-
\frac{\Phi''(\varphi)}{\Phi(\varphi)}
\]
führt.
Die linke Seite der letzten Gleichung hängt nur von $\vartheta$
ab, die rechte nur von $\varphi$, beide Seiten müssen daher
konstant sein, wir bezeichnen diese Konstante mit $\alpha^2$.
So ergibt sich die Differentialgleichung
\[
\alpha^2
=
-\frac{\Phi''(\varphi)}{\Phi(\varphi)}
\]
für die Abhängigkeit von $\varphi$, mit der allgemeinen Lösung
\[
\Phi(\varphi)
=
A\cos\alpha \varphi
+
B\sin\alpha \varphi.
\]
Die Randbedingungen verlangen, dass $\Phi(\varphi)$ eine $2\pi$-periodische
Funktion ist, was genau dann möglich ist, wenn $\alpha=m$ ganzzahlig ist.
Damit ergibt sich für die $\vartheta$-Abhängigkeit die Differentialgleichung
\begin{equation}
\frac{\sin\vartheta}{\Theta(\vartheta)}
\frac{\partial}{\partial\vartheta}\sin\vartheta\Theta'(\vartheta)
+
\mu^2\sin^2\vartheta
=
m^2.
\label{buch:pde:kugel:eqn:thetaanteil}
\end{equation}

\subsubsection{Abhängigkeit von $\vartheta$}
Die Differentialgleichung~\eqref{buch:pde:kugel:eqn:thetaanteil}
ist etwas unhandlich, daher verwenden wir die Substitution $z=\cos\vartheta$,
um die trigonometrischen Funktionen los zu werden.
Wegen
\[
\frac{dz}{d\vartheta} = -\sin\vartheta =-\sqrt{1-z^2}
\]
können die Ableitungen nach $\vartheta$ auch durch Ableitungen nach $z$
ausgedrückt werden.
Wir schreiben dazu $Z(z)=\Theta(\vartheta)$ und berechnen 
\[
\Theta'(\vartheta)
=
\frac{d\Theta}{d\vartheta}
=
\frac{dZ}{dz}\frac{dz}{d\vartheta}
=
-
\sqrt{1-z^2}
Z'(z).
\]
Dies bedeutet auch, dass
\[
\sin\vartheta\frac{d}{d\vartheta}
=
-
(1-z^2)\frac{d}{dz},
\]
damit lässt sich die Differentialgleichung für $\Theta(\vartheta)$ umschreiben
in eine Differentialgleichung für $Z(z)$, nämlich
\[
(1-z^2)\frac{d}{dz}(1-z^2)\frac{d}{dz} Z(z)
+
\mu^2
(1-z^2)
Z(z)
=
m^2
Z(z).
\]
Indem man die Ableitung im ersten Term mit Hilfe der Produktregel
ausführt, kann man die Gleichung
\[
(1-z^2)\biggl(
-2zZ'(z) + (1-z^2)Z''(z)
\biggr)
+
\mu^2(1-z^2)Z(z)
=
-m^2 Z(z)
\]
bekommen.
Division durch $1-z^2$ ergibt die
{\em Legendre-Differentialgleichung}
\begin{equation}
(1-z^2)Z''(z)
-2zZ'(z)
+
\biggl(
\mu^2 - \frac{m^2}{1-z^2}
\biggr)
Z(z)
=
0.
\label{buch:pde:kugel:eqn:legendre-dgl}
\end{equation}
Eine Diskussion der Lösungen dieser Differentialgleichung erfolgt im
Kapitel~\ref{chapter:kugel}.

\subsection{Kugelfunktionen}
Die Legendre-Differentialgleichung~\eqref{buch:pde:kugel:eqn:legendre-dgl}
hat Lösungen für Werte von $\mu$ derart, dass $\mu^2=l(l+1)$ für natürliche
Zahlen $l$.
Die Lösungen sind sogar Polynome, die wir mit $P_l^{(m)}(z)$ 
bezeichnen, dabei ist $m$ eine ganze Zahl mit $-l\le m\le l$.
Die Funktionen $P_l^{(m)}(\cos\vartheta)e^{im\varphi}$ 
sind daher alle Lösungen des von $\vartheta$ und $\varphi$
abhängigen Teils der Lösungen des Eigenwertproblems.
Mit einer geeigneten Normierung kann man zudem eine Familie von
bezüglich des Skalarproduktes
\[
\langle f,g\rangle_{S^2}
=
\int_{-\pi}^{\pi}
\int_{0}^{\pi}
\overline{f(\vartheta,\varphi)}
g(\vartheta,\varphi)
\sin\vartheta
\,d\vartheta
\,d\varphi
\]
orthonormiete Funktionen auf der Kugeloberfläche erhalten, die
man normalerweise als 
\[
Y_{lm}(\vartheta,\varphi)
=
\frac{1}{\sqrt{2\pi}}
\sqrt{
\frac{2l+1}{2}\cdot
\frac{(l-m)!}{(l+m)!}
}
P_{l}^{(m)}(\cos\vartheta)e^{im\varphi}
\]
bezeichnet.





