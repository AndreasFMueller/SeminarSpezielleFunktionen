%
% gleichung.tex
%
% (c) 2021 Prof Dr Andreas Müller, OST Ostschweizer Fachhochschule
%
\section{Gleichungen und Randbedingungen
\label{buch:pde:section:gleichungen-und-randbedingungen}}
\rhead{Gebiete, Gleichungen und Randbedingungen}

\subsection{Gebiete, Differentialoperatoren, Randbedingungen}


\subsubsection{Gebiete}
Gewöhnliche Differentialgleichungen haben nur eine unabhängige
Variable, die gesuchte Lösungsfunktion ist auf eine 
Intervall in $\mathbb{R}$ definiert.
Die Lösungsfunktion einer partiellen Differentialgleichung
ist auf einer Teilmenge von $\mathbb{R}^n$ definiert, des 
ermöglicht wesentlich vielfältigere und kompliziertere
Situationen.

\begin{definition}
Ein Gebiet $G\subset\mathbb{R}^n$ ist eine offene Teilmenge
von $\mathbb{R}^n$, d.~h.~für jeden Punkt $x\in G$ gibt es
eine kleine Umgebung
\(
U_{\varepsilon}(x)
=
\{y\in\mathbb{R}^n\mid |x-y|<\varepsilon\}
\), die ebenfalls in $G$ in enthalten ist,
also $U_{\varepsilon}(x)\subset G$.
\end{definition}

\subsubsection{Differentialoperatoren}
Eine gewöhnliche Differentialgleichung für eine Funktion
ist eine Beziehung zwischen den Werten der Funktion und ihrer
Ableitung in jedem Punkt des Definitionsintervalls.
Eine partielle Differentialgleichung ist entsprechend eine
Beziehung zwischen den Werten einer Funktion und ihren partiellen
Ableitungen.
Eine Funktion von mehreren Variablen hat sehr viel mehr partielle
Ableitungen, bereits partielle Differentialgleichungen erster
Ordnung sind daher sehr viel vielfältiger.
Bei höheren partiellen Ableitungen kommen noch die zusätzliche Bedingungen
\[
\frac{\partial^2 u}{\partial x_i\,\partial x_j}
=
\frac{\partial^2 u}{\partial x_j\,\partial x_i}
\]
hinzu, die für jedes Paar von Indizes $i,j$ ebenfalls erfüllt sein
müssen.

In diesem Kapitel betrachten wir ausschliesslich lineare
Differentialgleichungen.
Die Funktionswerte und partiellen Ableitungen lassen sich daher
in der Form eines Operators
\[
L 
=
a
+ \sum_{i=1}^n b_i \frac{\partial}{\partial x_i}
+ \sum_{i,j=1}^n c_{ij} \frac{\partial^2}{\partial x_i\,\partial x_j}
+ \dots
\]
schreiben.
Die Koeffizienten $a$, $b_i$, $c_{ij}$ können dabei durchaus auch
Funktionen der unabhängigen Variablen sein.

\subsubsection{Laplace-Operator}
Der Laplace-Operator hat in einem karteischen Koordinatensystem die
Form
\[
\Delta
=
\frac{\partial^2}{\partial x_1^2}
+
\frac{\partial^2}{\partial x_2^2}
+
\dots
+
\frac{\partial^2}{\partial x_n^2}.
\]
Er zeichnet sich durch die Eigenschaft aus, dass eine beliebige 
Translation oder Drehung des Koordinatensystems den Wert von $\Delta u$
nicht ändert.
Man könnte sagen, der Laplace-Operator ist symmetrisch bezüglich
aller Bewegungen des Raumes.

\subsubsection{Wellengleichung}

\subsubsection{Eigenfunktionen}
Eine besonders einfache 

\subsubsection{Trigonometrische Funktionen}
Die trigonometrischen Funktionen 

\subsection{Orthogonalität}
In der linearen Algebra lernt man, dass die Eigenvektoren einer
symmetrischen Matrix zu verschiedenen Eigenwerten orthgonal sind.
Dies hat zur Folge, dass die Transformation in eine Eigenbasis
mit einer orthogonalen Matrix möglich ist, was wiederum die Basis
von Diagonalisierungsverfahren wie dem Jacobi-Verfahren ist.

Das Separationsverfahren wird zeigen, wie sich das Finden einer
Lösung der Wellengleichung auf Lösungen des Eigenwertproblems
$\Delta u = \lambda u$ zurückführen lässt.
Damit stellt sich die Frage, welche Eigenschaften 


\subsubsection{Gewöhnliche Differentialglichung}


\subsubsection{$n$-dimensionaler Fall}
