%
% gleichung.tex
%
% (c) 2021 Prof Dr Andreas Müller, OST Ostschweizer Fachhochschule
%
\section{Gleichungen und Randbedingungen
\label{buch:pde:section:gleichungen-und-randbedingungen}}
\rhead{Gebiete, Gleichungen und Randbedingungen}
Gewöhnliche Differentialgleichungen sind immer auf einem
Intervall als Definitionsgebiet definiert.
Partielle Differentialgleichungen sind Gleichungen, die verschiedene
partielle Ableitungen einer Funktion mehrerer Variablen involvieren,
das Definitionsgebiet ist daher immer eine höherdimensionale Teilmenge 
von $\mathbb{R}^n$.
Sowohl das Gebiet wie auch dessen Rand können wesentlich komplexer sein.
Eine sorgfältige Definition ist unabdingbar, um Widersprüchen vorzubeugen.

%
% Gebiete, Differentialoperatoren, Randbedingungen
%
\subsection{Gebiete, Differentialoperatoren, Randbedingungen}
In diesem Abschnitt sollen die Begriffe geklärt werden, die zur
korrekten Formulierung eines partiellen Differentialgleichungsproblems
notwendig sind.

%
% Gebiete
%
\subsubsection{Gebiete}
Gewöhnliche Differentialgleichungen haben nur eine unabhängige
Variable, die gesuchte Lösungsfunktion ist auf eine 
Intervall in $\mathbb{R}$ definiert.
Die Lösungsfunktion einer partiellen Differentialgleichung
ist auf einer Teilmenge von $\mathbb{R}^n$ definiert, des 
ermöglicht wesentlich vielfältigere und kompliziertere
Situationen.

\begin{definition}
\label{buch:pde:definition:gebiet}
Ein Gebiet $G\subset\mathbb{R}^n$ ist eine offene Teilmenge
von $\mathbb{R}^n$, d.~h.~für jeden Punkt $x\in G$ gibt es
eine kleine Umgebung
\(
U_{\varepsilon}(x)
=
\{y\in\mathbb{R}^n\mid |x-y|<\varepsilon\}
\), die ebenfalls in $G$ in enthalten ist,
also $U_{\varepsilon}(x)\subset G$.
\index{Gebiet}%
\end{definition}

%
% Differentialoperatoren
%
\subsubsection{Differentialoperatoren}
Eine gewöhnliche Differentialgleichung für eine Funktion
ist eine Beziehung zwischen den Werten der Funktion und ihrer
Ableitung in jedem Punkt des Definitionsintervalls.
Eine partielle Differentialgleichung ist entsprechend eine
Beziehung zwischen den Werten einer Funktion und ihren partiellen
Ableitungen.
Eine Funktion von mehreren Variablen hat sehr viel mehr partielle
Ableitungen, bereits partielle Differentialgleichungen erster
Ordnung sind daher sehr viel vielfältiger.
Bei höheren partiellen Ableitungen kommen noch die zusätzliche Bedingungen
\[
\frac{\partial^2 u}{\partial x_i\,\partial x_j}
=
\frac{\partial^2 u}{\partial x_j\,\partial x_i}
\]
hinzu, die für jedes Paar von Indizes $i,j$ ebenfalls erfüllt sein
müssen.

In diesem Kapitel betrachten wir ausschliesslich lineare
Differentialgleichungen.
Die Funktionswerte und partiellen Ableitungen lassen sich daher
in der Form eines Operators
\[
L 
=
a
+ \sum_{i=1}^n b_i \frac{\partial}{\partial x_i}
+ \sum_{i,j=1}^n c_{ij} \frac{\partial^2}{\partial x_i\,\partial x_j}
+ \dots
\]
schreiben.
Die Koeffizienten $a$, $b_i$, $c_{ij}$ können dabei durchaus auch
Funktionen der unabhängigen Variablen sein.

%
% Laplace-Operator
%
\subsubsection{Laplace-Operator}
Der {\em Laplace-Operator} hat in einem karteischen Koordinatensystem die
Form
\index{Laplace-Operator}%
\[
\Delta
=
\frac{\partial^2}{\partial x_1^2}
+
\frac{\partial^2}{\partial x_2^2}
+
\dots
+
\frac{\partial^2}{\partial x_n^2}.
\]
Er zeichnet sich durch die Eigenschaft aus, dass eine beliebige 
Translation oder Drehung des Koordinatensystems den Wert von $\Delta u$
nicht ändert.
Man könnte sagen, der Laplace-Operator ist symmetrisch bezüglich
aller Bewegungen des Raumes.

%
% Wellengleichung
%
\subsubsection{Wellengleichung}
Da die physikalischen Gesetze invariant sein müssen unter solchen
Bewegungen, ist zu erwarten, dass der Laplace-Operator in partiellen
Differentialgleichungen 
Als Beispiel betrachten wir die Ausbreitung einer Welle, welche sich
in einem Medium mit der Geschwindigkeit $c$ ausbreitet.
Ist $u(x,t)$ die Auslenkung der Welle im Punkt $x\in\mathbb{R}^n$
zur Zeit $t\in\mathbb{R}$, dann erfüllt die Funktion $u(x,t)$
die partielle Differentialgleichung
\begin{equation}
\frac{1}{c^2}
\frac{\partial^2 u}{\partial t^2}
=
\Delta u.
\label{buch:pde:eqn:waveequation}
\end{equation}
In dieser Gleichung treten nicht nur die partiellen Ableitungen
nach den Ortskoordinaten auf, die der Laplace-Operator miteinander
verknüpft.
Die Funktion $u(x,t)$ ist definiert auf einem Gebiet in 
$\mathbb{R}^{n}\times\mathbb{R}=\mathbb{R}^{n+1}$ mit den Koordinaten
$(x_1,\dots,x_n,t)$.
Der Gleichung~\eqref{buch:pde:eqn:waveequation} ist daher eigentlich
die Gleichung
\[
\square u = 0
\qquad\text{mit}\quad
\square
=
\frac{1}{c^2}\frac{^2}{\partial t^2}
-
\Delta
=
\frac{1}{c^2}\frac{\partial^2}{\partial t^2}
-
\frac{\partial^2}{\partial x_1^2}
-
\frac{\partial^2}{\partial x_2^2}
-\dots- 
\frac{\partial^2}{\partial x_n^2}
\]
wird.
Der Operator $\square$ heisst auch d'Alembert-Operator.
\index{dAlembertoperator@d'Alembert-Operator}%

%
% Randbedingungen
%
\subsubsection{Randbedingungen}
Die Differentialgleichung oder der Differentialoperator legen die
Lösung nicht fest.
Wie bei gewöhnlichen Differentialgleichungen ist dazu die Spezifikation
geeigneter Randbedingungen nötig.

\begin{definition}
\label{buch:pde:definition:randbedingungen}
Eine {\em Randbedingung} für das Gebiet $\Omega$ ist eine Teilmenge
$F\subset\partial\Omega$ sowie eine auf $F$ definierte Funktion
$f\colon F\to\mathbb{R}$.
Eine Funktion $u\colon \overline{\Omega} \to\mathbb{R}$ erfüllt eine
{\em Dirichlet-Randbedingung}, wenn
\index{Dirichlet-Randbedingung}%
\index{Randbedingung!Dirichlet-}%
\(
u(x) = f(x)
\)
für $x\in F$.
Sie erfüllt eine {\em Neumann-Randbedingung}, wenn
\index{Neumann-Randbedingung}%
\index{Randbedingung!Neumann-}%
\[
\frac{\partial u}{\partial n}
=
f(x)\qquad\text{für $x\in F$}.
\]
Dabei ist
\[
\frac{\partial u}{\partial n}
=
\frac{d}{dt}
u(x+tn)
\bigg|_{t=0}
=
\operatorname{grad}u\cdot n
\]
\index{Normalableitung}%
die {\em Normalableitung}, die Richtungsableitung in Richtung des
Vektors $n$, der senkrecht ist auf dem Rand $\partial\Omega$ von
$\Omega$.
\end{definition}

Die Vorgabe nur von Ableitungen kann natürlich die Lösung $u(x)$
einer linearen partiellen Differentialgleichung nicht eindeutig
festlegen, dazu ist noch mindestens ein Funktionswert notwendig.
Die Vorgabe von anderen Ableitungen in Richtungen tangential an den
Rand liefert keine neue Information, denn ausgehend von dem einen
Funktionswert auf dem Rand kann man durch Integration entlang
einer Kurve auf dem Rand eine Neumann-Randbedingung konstruieren,
die die gleiche Information beinhaltet wie Anforderungen an die
tangentialen Ableitungen.
Dirichlet- und Neumann-Randbedingungen sind daher die einzigen
sinnvollen linearen Randbedingungen.

