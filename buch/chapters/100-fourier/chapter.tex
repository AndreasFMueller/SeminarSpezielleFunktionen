%
% chapter.tex -- Spezielle Funktionen definiert durch Fourier-Transformation
%
% (c) 2021 Prof Dr Andreas Müller, Hochschule Rapperswil
%
% !TeX spellcheck = de_CH
\chapter{Integraltransformationen
\label{buch:chapter:fourier}}
\lhead{Integraltransformationen}
\rhead{}
Die Fourier-Transformation und andere Integraltransformationen
führen zu neuen speziellen Funktionen.
In diesem Kapitel soll als Beispiel die Fourier-Transformation
der Bessel-Funktionen untersucht werden.

%
% 1d.tex
%
% (c) 2022 Prof Dr Andreas Müller
%
\section{Fourier-Transformation
\label{buch:fourier:section:fourier}}
\rhead{Fourier-Transformation}
Im Kapitel~\ref{buch:chapter:orthogonalitaet}
wurde gezeigt, wie orthogonale Funktionenfamilien ermöglichen,
beliebige Funktionen auf besonders einfache Art und Weise
als Linearkombinationen oder Reihen von Funktionen der Familie
darzustellen.

%
% Fourier-Reihen
%
\subsection{Fourier-Reihen}
Das berühmteste solche Beispiel ist die Fourier-Reihe, die
Joseph Fourier im Rahmen seiner Studien zur Wärmeleitungsgleichung
entdeckt hat.
Sie besagt, dass unter geeigneten Regularitätsvoraussetzungen
jede $2\pi$-periodische Funktion $f(x)$ als konvergente Reihe von
trigonometrischen Funktionen in der Form
\begin{equation}
f(x)
=
\frac{a_0}2
+
\sum_{k=1}^\infty (a_k \cos kx + b_k \sin kx)
\label{buch:fourier:eqn:fourierreihe}
\end{equation}
geschrieben werden kann.
Die Koeffizienten $a_k$ und $b_k$ können mit Hilfe von Integralen
\[
\begin{aligned}
a_k
&=
\frac{1}{\pi}
\int_{-\pi}^\pi f(x)\, \cos kx\,dx
&&\text{für $k\ge 0$}
\\
b_k
&=
\frac{1}{\pi}
\int_{-\pi}^\pi f(x)\, \sin kx\,dx
&&\text{für $k\ge 1$}
\end{aligned}
\]
berechnet werden.

%
% Komplexes Skalarprodukt
%
\subsubsection{Komplexes Skalarprodukt}
Die Reihe~\eqref{buch:fourier:eqn:fourierreihe} hat zwar den Vorteil,
nur reellwertige Funktionen zu verwenden.
Die trigonometrischen Funktionen lassen sich aber mit Hilfe der
eulerschen Formel mittels
\[
\cos x = \frac{e^{ix}+e^{-ix}}{2}
\qquad
\sin x = \frac{e^{ix}-e^{-ix}}{2i}
\]
als Linearkombinationen von Exponentialfunktionen ausdrücken.
Wendet man diese Transformation auf
die Fourier-Reihe~\eqref{buch:fourier:eqn:fourierreihe}
an, erhält man eine symmetrischere, komplexe Darstellung.
Dazu ist zunächst die Definition eines Skalarproduktes für komplexe
Funktionen nötig.

\begin{definition}
\label{buch:fourier:def:cskalar}
Die bilineare Funktion
\[
\langle f,g\rangle
=
\frac{1}{2\pi}
\int_{-\pi}^\pi \overline{f(x)}g(x)\,dx
\]
ist {\em sesquilinear}, d.~h. sie ist
\begin{enumerate}
\item
linear im zweiten und konjugiert linear im ersten Argument:
\begin{align*}
\langle\lambda_1f_1+\lambda_2f_2,g\rangle
&=
\overline{\lambda_1}\langle f_1,g\rangle
+
\overline{\lambda_2}\langle f_2,g\rangle
&
\langle f,\mu_1g_1+\mu_2g_2\rangle
&=
\mu_1
\langle f,g_1\rangle
+
\mu_2
\langle f,g_2\rangle,
\end{align*}
für $\lambda_i,\mu_i\in\mathbb{C}$.
\item
Konjugiert-symmetrisch:
$\langle f,g\rangle=\overline{\langle g,f\rangle}$.
\item
Positiv definit: Falls $f\ne 0$ folgt
$\langle f,f\rangle > 0$.
\end{enumerate}
Man nennt
$\langle\;\,,\;\rangle$
ein {\em Skalarprodukt} für komplexe Funktionen.
Es liefert eine
{\em Norm}
\[
\|f\|^2
=
\langle f,f\rangle
=
\int_{-\pi}^{\pi} |f(x)|^2\,dx.
\]
\end{definition}

%
% Komlexe Fourier-Reihe
%
\subsubsection{Komplexe Fourier-Reihe}
Bezüglich des Skalarprodukts
der Definition~\ref{buch:fourier:def:cskalar}
sind die Funktionen
\[
e_k
:
\mathbb{R} \to \mathbb{C}
\colon
x\mapsto e_k(x)=e^{ikx}
\]
eine orthonormierte Familie.
Die komplexe Fourier-Theorie besagt dann,
dass eine beliebige komplexe $2\pi$-periodische Funktion $f(x)$ 
in der Form einer bezüglich der Norm $\|\,\cdot\,\|$ konvergenten Reihe
\begin{equation}
f(x)
=
\sum_{k=-\infty}^\infty c_ke^{ikx}
\label{buch:fourier:eqn:cfourierreihe}
\end{equation}
mit den Koeffizienten
\[
c_k
=
\frac{1}{2\pi}
\int_{-\pi}^\pi
f(x) e^{-ikx}\,dx.
\]

\subsubsection{Parseval-Identität}
Aus der geometrischen Motivation lässt sich auch auch die
Parseval-Identität ableiten.
Sie besagt, dass Skalarprodukt und Norm zweier Funktionen
$f(x)$ und $g(x)$ auch aus den Fourier-Koeffizienten
berechnet werden können.
Sind $d_k$ die Fourier-Koeffizienten der Funktion $g(x)$, dann gilt
\begin{align*}
\langle f,g\rangle
&=
\int_{-\pi}^\pi \overline{f(x)} g(x)\,dx
=
\sum_{k=-\infty}^\infty \overline{c_k}d_k,
\\
\|f\|^2
&=
\int_{-\pi}^\pi |f(x)|^2\,dx
=
\sum_{k=-\infty}^{\infty} |c_k|^2.
\end{align*}
Die rechten Seiten können als Skalarprodukt und Norm der Folgen
der Fourier-Koeffizienten interpretiert werden.
Die Formeln besagen dann, dass der Übergang zu den Fourier-Koeffizienten
das Skalarprodukt erhält.
Man kann die Formeln auch als die Fourier-Version des Satzes
von Pythagoras sehen.

%
% Das Fourier-Integral
%
\subsection{Fourier-Integral und Fourier-Transformation}
Gibt man die Einschränkung auf $2\pi$-periodische Funktionen
auf, gibt es keine abzählbare Familie von Basisfunktionen, mit denen
man eine beliebige Funktion als Summe darstellen können.
Die im folgenden zusammengefasste Fourier-Transformation zeigt
aber, dass eine Darstellung als Integral möglich ist.

\subsubsection{Fourier-Integral}
Die Fourier-Transformierte
\begin{equation}
\hat{f}(k)
=
(\mathscr{F}f)(k)
=
\frac{1}{\sqrt{2\pi}}
\int_{-\infty}^\infty f(x)e^{-ikx}\,dx
\label{buch:fourier:eqn:fouriertrafo}
\end{equation}
verallgemeinert die Berechnung der Fourier-Koeffizienten
auf beliebige Funktionen $f\colon\mathbb{R}\to\mathbb{R}$.
Wegen des unendlich langen Integrationsintervals sind zusätzliche
Einschränkungen nötig, damit das Integral überhaupt definiert ist.
Es genügt zu verlangen, dass
\[
\int_{-\infty}^\infty |f(x)|\,dx < \infty
\]
ist.
Die Funktionen mit dieser Eigenschaft bilden den Funktionenraum
$L^1(\mathbb{R})$.
Man beachte, dass die Funktionen
\(
e_k(x) = e^{ikx}
\)
nicht in $L^1(\mathbb{R})$ sind, die Betrachtungsweise auf dem
Intervall $[-\pi,\pi]$ lässt sich daher nicht auf den Fall des
Definitionsbereichs $\mathbb{R}$ übertragen.

Die allgemeine Theorie zeigt, dass die
Transformation~\eqref{buch:fourier:eqn:fouriertrafo}
durch das Integral
\[
f(x)
=
\frac{1}{\sqrt{2\pi}}
\int_{-\infty}^\infty (\mathscr{F}f)(k)e^{ikx}\,dk.
\]
umgekehrt wird.
Dies funktioniert wieder nur, wenn auch $\mathscr{F}f$ eine Funktion
in $L^1(\mathbb{R})$ ist.

%
% Skalarprodukt
%
\subsubsection{Skalarprodukt}
Auch die Fourier-Transformation kann im Rahmen eines Funktionenraums
mit einem Skalarprodukt beschrieben werden.
Das Skalarprodukt in diesem Fall ist
\[
\langle f,g\rangle
=
\int_{-\infty}^\infty \overline{f(x)} g(x)\,dx.
\]
Das Skalarprodukt ist wieder nur für eine eingeschränkte Menge von
Funktionen definiert, nämlich für Funktionen aus dem Funktionenraum
\[
L^2(\mathbb{R})
=
\left\{
f\colon\mathbb{R}\to\mathbb{R}
\,\left|\,
\int_{-\infty}^\infty |f(x)|^2\,dx < \infty
\right.
\right\}.
\]
In der Praxis hat man oft mit Funktionen zu tun, die nur in einem
beschränkten Intervall von $0$ verschieden sind und zudem stetig.
Solche Funktionen sind immer in $L^2(\mathbb{R})\cap L^1(\mathbb{R})$.

%
% Parseval-Identität
%
\subsubsection{Parseval-Identität}
Wie im Falle der Fourier-Reihe, ändert die Fourier-Transformation
die Länge des Vektors nicht.
Für die Fourier-Transformation lautet die Parseval-Identität
\begin{align*}
\langle f,g\rangle
&=
\int_{-\infty}^\infty
\overline{f(x)} g(x)\,dx
=
\int_{-\infty}^\infty
\overline{ \mathscr{F}f(k) }
\mathscr{F}g(k)\,dk
=
\langle \mathscr{F}f,\mathscr{F}g\rangle
=
\langle\hat{f},\hat{g}\rangle
\\
\|f\|^2
&=
\int_{-\infty}^\infty
|f(x)|^2\,dx
=
\int_{-\infty}^\infty
|\mathscr{F}f(k)|^2\,dk
=
\|\mathscr{F}f\|^2
=
\|\hat{f}\|^2,
\end{align*}
für Funktionen, für die alles definiert ist.
Die Fourier-Transformation erhält also das Skalarprodukt,
sie ist eine unitäre Abbildung.

%
% 2d.tex
%
% (c) 2022 Prof Dr Andreas Müller, OST Ostschweizer Fachhochschule
%
\section{Zweidimensionale Fourier-Transformation
\label{buch:fourier:section:2d}}
\rhead{Zweidimensionale Fourier-Transformation}

\subsection{Fourier-Transformation und partielle Differentialgleichungen}

\subsection{Fourier-Transformation in kartesischen Koordinaten}

\subsection{Basisfunktionen in Polarkoordinaten}






%
% bessel.tex
%
% (c) 2021 Prof Dr Andreas Müller, OST Ostschweizer Fachhochschule
%
\section{Bessel-Funktionen
\label{buch:differntialgleichungen:section:bessel}}
\rhead{Bessel-Funktionen}
Die Besselsche Differentialgleichung
erlaubt Wellen mit zylindrischer
Symmetrie und die Strömung in einem zylindrischen Rohr zu beschreiben.
Die Auflösung eines optischen Systems wird durch die Beugung limitiert,
die Helligkeitskverteilung des Bildes einer Punktquelle ist
zylindersymmetrisch und kann mit Hilfe von Lösungen der Besselschen
Differentialgleichung beschrieben werden.
Das Kapitel~\ref{chapter:kreismembran} zeigt, wie die Bessel-Funktionen
bei der Lösung der Wellengleichung für eine kreisförmige Membran
auftreten.
Die Besselsche Differentialgleichung hat im Allgemeinen keine Lösung,
die sich durch bekannte Funktionen ausdrücken lassen, es ist also
nötig, eine neue Familie von speziellen Funktionen zu definieren,
die Bessel-Funktionen.

%
% Besselsche Differentialgleichung
%
\subsection{Die Besselsche Differentialgleichung}
% XXX Wo taucht diese Gleichung auf
Die Besselsche Differentialgleichung ist die Differentialgleichung
\begin{equation}
x^2\frac{d^2y}{dx^2} + x\frac{dy}{dx} + (x^2-\alpha^2)y = 0
\label{buch:differentialgleichungen:eqn:bessel}
\end{equation}
\index{Differentialgleichung!Besselsche}%
\index{Besselsche Differentialgleichung}%
zweiter Ordnung
für eine auf dem Interval $[0,\infty)$ definierte Funktion $y(x)$.
Der Parameter $\alpha$ ist eine beliebige reelle oder sogar komplexe
Zahl $\alpha\in \mathbb{C}$,
die Lösungsfunktionen hängen daher von $\alpha$ ab.

%
% Eigenwertproblem
%
\subsubsection{Eigenwertproblem}
Die Besselsche Differentialgleichung
\eqref{buch:differentialgleichungen:eqn:bessel}
kann man auch als Eigenwertproblem für den Bessel-Operator
\index{Bessel-Operator}%
\index{Operator!Bessel-}%
\begin{equation}
B = x^2\frac{d^2}{dx^2} + x\frac{d}{dx} + x^2
\label{buch:differentialgleichungen:bessel-operator}
\end{equation}
schreiben.
Eine Lösung $y(x)$ der Gleichung
\eqref{buch:differentialgleichungen:eqn:bessel}
erfüllt
\[
By
=
x^2y''+xy'+x^2y
=\alpha^2 y,
\]
ist also eine Eigenfunktion des Bessel-Operators zum Eigenwert
$\alpha^2$.

%
% Indexgleichung
%
\subsubsection{Indexgleichung}
Die Besselsche Differentialgleichung ist eine Differentialgleichung
der Art~\eqref{buch:differentialgleichungen:eqn:dglverallg} mit
\[
p(x) = 1
\qquad\text{und}\qquad
q(x) = x^2-\alpha^2.
\]
Nach den Ausführungen von
Abschnitt~\ref{buch:differentialgleichungen:subsection:verallgemeinrt},
muss die Lösung in der Form einer verallgemeinerten Potenzreihe 
gesucht werden.
Dazu muss zunächst die Indexgleichung
\[
0
=
X(X-1) + Xp_0 + q_0
=
X(X-1) + X - \alpha^2
=
X^2-\alpha^2
=
(X-\alpha)(X+\alpha)
\]
gelöst werden.
Die Nullstellen sind offenbar $\varrho_1=\alpha$ und $\varrho_2=-\alpha$.

Die beiden Vorzeichen der Nullstellen der Indexgleichung führen
auf die gleiche Differentialgleichung.
Der Lösungsraum der Differentialgleichung ist natürlich trotzdem
zweidimensional, so dass es immer noch möglich ist, den
beiden Nullstellen der Indexgleichung verschiedene Lösungen
zuzuordnen.
Die Diskussion in
Abschnitt~\ref{buch:differentialgleichungen:subsection:verallgemeinrt}
hat Kriterien ergeben, unter denen zwei linear unabhängige Lösungen
mit Hilfe einer verallgemeinerten Potenzreihe gefunden werden können.
Falls nur eine solche Lösung gefunden werden kann, wird sie der grösseren
der beiden Zahlen $\alpha$ und $-\alpha$ zugeordnet
(oder $0$, falls $\alpha=-\alpha=0$).
Eine weitere Lösung kann mit Hilfe analytischer Fortsetzung gefunden werden,
wie später in Kapitel~\ref{buch:chapter:funktionentheorie} gezeigt wird.

Für nicht reelles $\alpha$ kann $\varrho_1-\varrho_2=2\alpha$ keine 
Ganzzahl sein, es ist also garantiert, dass zwei linear unabhängig
Lösungen der Form
\begin{equation}
y_1(x) = x^\alpha\sum_{k=0}^\infty a_kx^k
\qquad\text{und}\qquad
y_2(x) = x^{-\alpha}\sum_{k=0}^\infty b_kx^k.
\label{buch:differentialgleichungen:eqn:besselloesungen}
\end{equation}
existieren.

Für reelles $\alpha\in\mathbb{R}$ gibt es zwei Lösungen der
Form~\eqref{buch:differentialgleichungen:eqn:besselloesungen}
genau dann, wenn $\varrho_1-\varrho_2$ keine Ganzzahl ist.
Nur eine Lösung kann man finden, wenn 
\[
\alpha-(-\alpha)=2\alpha \in \mathbb{Z}
\qquad\Rightarrow\qquad
\alpha = \frac{k}{2},\quad k\in\mathbb{Z}
\]
ist.

%
% Bessel-Funktionen erster Art
%
\subsection{Bessel-Funktionen erster Art
\label{buch:differentialgleichungen:subsection:bessel1steart}}
Für $\alpha \ge 0$ gibt es immer mindestens eine Lösung der Bessel-Gleichung
als verallgemeinerte Potenzreihe mit $\varrho=\alpha$.
Die Funktion $q(x)=x^2-\alpha^2$ ist ein Polynom, die einzigen
von $0$ verschiedenen Koeffizienten sind $q_0=-\alpha^2$
und $q_2=1$.
Für den ersten Koeffizienten $a_0$ gibt es keine Einschränkungen,
wir wählen $a_0=1$.

Die Rekursionsformel für $n=1$ ist
\[
F(\varrho+1) a_1 = (\varrho p_1+q_1)a_0,
\]
aber die Koeffizienten $p_1$ und $q_1$ verschwinden beide und damit
die ganze rechte Seite.
Da $F(\varrho+1)\ne 0$ ist, folgt dass $a_1=0$ sein muss.

% Fall n=1 gesondert behandeln

%
% Der allgemeine Fall
%
\subsubsection{Der allgemeine Fall}
Für die höheren Potenzen $n>1$ wird die Rekursionsformel für die
Koeffizienten $a_n$ der verallgemeinerten Potenzreihe
\[
a_{n} =
-\frac{ q_2 a_{n-2} }{F(\varrho+n)}
=
-\frac{a_{n-2}}{(\varrho+n)^2-\alpha^2}
=
-\frac{a_{n-2}}{\varrho^2 + 2\varrho n+n^2-\alpha^2}
=
-\frac{a_{n-2}}{n(n+2\varrho)}.
\]
Im letzten Schritt haben wir verwendet, dass $\varrho=\pm\alpha$
und damit $\varrho^2=\alpha^2$ ist.
Daraus folgt wegen $a_1=0$, dass auch $a_{2k+1}=0$ für alle $k$.
Damit können wir jetzt die Reihe hinschreiben:
\begin{align*}
y(x)
&=
x^{\varrho}\biggl(
1
-
\frac{1}{2(2+2\varrho)} x^2
+
\frac{1}{2(2+2\varrho)4(4+2\varrho)} x^4
-
\frac{1}{2(2+2\varrho)4(4+2\varrho)6(6+2\varrho)} x^6
+
\dots
\biggr)
\\
&=
x^{\varrho}
\biggl(
1
+
\frac{(-x^2/4)}{1\cdot (1+\varrho)}
+
\frac{(-x^2/4)^2}{1\cdot 2\cdot (1+\varrho)\cdot(2-\varrho)}
+
\frac{(-x^2/4)^3}{1\cdot 2\cdot 3\cdot (1+\varrho)\cdot(2+\varrho)\cdot(3+\varrho)}
+
\dots
\biggr)
\\
&=
x^\varrho\biggl(
1
+
\frac{1}{(\varrho+1)}\frac{(-x^2/4)}{1!}
+
\frac{1}{(\varrho+1)(\varrho+2)}\frac{(-x^2/4)^2}{2!}
+
\frac{1}{(\varrho+1)(\varrho+2)(\varrho+3)}\frac{(-x^2/4)^3}{3!}
+
\dots
\biggr)
\\
&=
x^\varrho \sum_{k=0}^\infty
\frac{1}{(\varrho+1)_k} \frac{(-x^2/4)}{k!}
=
x^\varrho
\cdot
\mathstrut_0F_1\biggl(;\varrho+1;-\frac{x^2}{4}\biggr)
\end{align*}
Wir finden also zwei Lösungsfunktionen
\begin{align}
y_1(x)
%J_\alpha(x)
&=
x^{\alpha\phantom{-}}
\sum_{k=0}^\infty
\frac{1}{(\alpha+1)_k}
\frac{(-x^2/4)^k}{k!}
=
x^\alpha
\cdot
\mathstrut_0F_1\biggl(;\alpha+1;-\frac{x^2}{4}\biggr),
\label{buch:differentialgleichunge:bessel:eqn:erste}
\\
y_2(x)
%J_{-\alpha}(x)
&=
x^{-\alpha} \sum_{k=0}^\infty
\frac{1}{(-\alpha+1)_k} \frac{(-x^2/4)^k}{k!}
=
x^{-\alpha}
\cdot
\mathstrut_0F_1\biggl(;-\alpha+1;-\frac{x^2}{4}\biggr).
\label{buch:differentialgleichunge:bessel:eqn:zweite}
\end{align}
Man beachte, dass die zweite Lösung für ganzzahliges $\alpha>0$ nicht
definiert ist.
Man kann auch direkt nachrechnen, dass diese Funktionen Lösungen
der Besselschen Differentialgleichung sind.

%
% Bessel-Funktionen
%
\subsubsection{Bessel-Funktionen}
Da die Besselsche Differentialgleichung linear ist, ist auch
jede Linearkombination der Funktionen
\eqref{buch:differentialgleichunge:bessel:eqn:erste}
und
\eqref{buch:differentialgleichunge:bessel:eqn:zweite}
eine Lösung.
Satz~\ref{buch:rekursion:gamma:satz:gamma-pochhammer}
ermöglicht, das Pochhammer-Symbol durch Werte der Gamma-Funktion
wie in
\[
(\alpha+1)_n = \frac{\Gamma(\alpha+k+1)}{\Gamma(\alpha+1)}
\]
auszudrücken.
Damit wird
\begin{align}
y_1(x)
&=
x^\alpha
\sum_{k=0}^\infty
\frac{\Gamma(\alpha+1)}{\Gamma(\alpha+k+1)}
\frac{(-x^2/4)^k}{k!}
=
\Gamma(\alpha+1) 2^{\alpha}
\biggl(\frac{x}{2}\biggr)^\alpha
\sum_{k=0}^\infty
\frac{(-1)^k}{k!\,\Gamma(\alpha+k+1)} \biggl(\frac{x}{2}\biggr)^{2k}
\label{buch:differentialgleichungen:bessel:normierungsgleichung}
\end{align}
Nur gerade der Faktor $2^\alpha\Gamma(\alpha+1)$ ist von $k$ und $x$ 
unabhängig, daher ist die folgende Definition sinnvoll:

\begin{definition}
\label{buch:differentialgleichungen:bessel:definition}
Die Funktion
\[
J_{\alpha}(x)
=
\biggl(\frac{x}{2}\biggr)^\alpha
\sum_{k=0}^\infty
\frac{(-1)^k}{k!\,\Gamma(\alpha+k+1)}
\biggl(\frac{x}{2}\biggr)^{2k}
\]
heisst {\em Bessel-Funktion erster Art der Ordnung $\alpha$}.
\index{Bessel-Funktion!erster Art}%
\end{definition}

Die Bessel-Funktion $J_\alpha(x)$ der Ordnung $\alpha$ unterscheidet sich
nur durch einen Normierungsfaktor von der Lösung $y_1(x)$.
Dasselbe gilt für $J_{-\alpha}(x)$ und $y_2(x)$:
\begin{align*}
J_{\alpha}(x)
&=
\frac{1}{2^\alpha\Gamma(\alpha+1)}
\cdot
y_1(x)
\\
J_{-\alpha}(x)
&=
\frac{1}{2^{-\alpha}\Gamma(-\alpha+1)}
\cdot
y_2(x).
\end{align*}

%
% Ganzzahlige Ordnung
%
\subsubsection{Bessel-Funktionen ganzzahliger Ordnung}
Man beachte, dass diese Definition für beliebige ganzzahlige 
$\alpha$ funktioniert.
Ist $\alpha=-n<0$, $n\in\mathbb{N}$, dann hat der Nenner Pole 
an den Stellen $k=0,1,\dots,n-1$.
Die Summe beginnt also erst bei $k=n$ oder
\begin{align*}
J_{-n}(x)
&=
\sum_{k=n}^\infty \frac{(-1)^k}{m!\,k!}\biggl(\frac{x}{2}\biggr)^{2k-n}
=
\sum_{l=0}^\infty
\frac{(-1)^{l+n}}{m!\,(l+n)!}\biggl(\frac{x}{2}\biggr)^{2(l+n)-n}
=
(-1)^n
\sum_{l=0}^\infty
\frac{(-1)^l}{m!\,\Gamma(l+n+1)}\biggl(\frac{x}{2}\biggr)^{2l+n}
\\
&=
(-1)^n
J_{n}(x).
\end{align*}
Insbesondere unterscheiden sich $J_n(x)$ und $J_{-n}(x)$ nur durch
ein Vorzeichen.

Als lineare Differentialgleichung zweiter Ordnung erwarten wir noch
eine zweite, linear unabhängige Lösung.
Diese kann jedoch nicht allein mit der Potenzreihenmethode
bestimmt werden,
dazu sind die Methoden der Funktionentheorie nötig.
Im Abschnitt~\ref{buch:funktionentheorie:subsection:dglsing}
wird gezeigt, wie dies möglich ist und auf
Seite~\pageref{buch:funktionentheorie:subsubsection:bessel2art}
werden die damit zu findenden Bessel-Funktionen 0-ter Ordnung und
zweiter Art vorgestellt.

%
% Erzeugende Funktione
%
\subsubsection{Erzeugende Funktion}
\begin{figure}
\centering
\includegraphics{chapters/050-differential/images/besselgrid.pdf}
\caption{Indexmenge für Herleitung der erzeugenden Funktion der
Bessel-Funktionen.
Die rote Summe in \eqref{buch:differentialgleichungen:bessel:eqn:rotesumme}
entspricht den vertikalen roten Streifen oben,
die blaue Summe in
\eqref{buch:differentialgleichungen:bessel:eqn:blauesumme}
den horizontalen Streifen in der Abbildung unten.
Alle Terme enthalten $\Gamma(n+k+1)$ im Nenner,
im grau hinterlegten Gebiet verschwinden sie.
\label{buch:differentialgleichungen:bessel:fig:indexmenge}}
\end{figure}
Die erzeugende Funktion der Bessel-Funktionen ist die Summe
\begin{align}
\sum_{n\in\mathbb{Z}} J_n(x)z^n
&=
\sum_{n\in\mathbb{Z}}
{\color{darkred}
\sum_{k=0}^\infty
\frac{(-1)^k}{k!\,\Gamma(k+n+1)}
\biggl(\frac{x}{2}\biggr)^{2k+n}
}
z^n.
\label{buch:differentialgleichungen:bessel:eqn:rotesumme}
\intertext{Die rote Summe entspricht den vertikalen roten Streifen in
Abbildung~\ref{buch:differentialgleichungen:bessel:fig:indexmenge} oben.
Die grau hinterlegten Punkte in der Abbildung gehören zu verschwindenden
Termen.
Wir schreiben $m=k+n$ und drücken alle Terme durch $k$ und $m$ aus:}
&=
\sum_{n\in \mathbb{Z}}
\sum_{k=0}^\infty
\frac{(-1)^k}{k!\,\Gamma(n+k+1)}
\biggl(\frac{x}{2}\biggr)^k
\biggl(\frac{x}{2}\biggr)^{n+k}
z^{n+k}
z^{-k}
\notag
\\
&=
\sum_{m\in \mathbb{Z}}
\sum_{k=0}^\infty \frac{(-1)^k}{k!}
\biggl(\frac{x}{2}\biggr)^k
z^{-k}
\frac{1}{\Gamma(m+1)}
\biggl(\frac{x}{2}\biggr)^{m}
z^{n+k}.
\notag
\intertext{Auch in dieser Summe fallen wieder die Terme mit $m<0$
wegen $\Gamma(m+1)=\infty$ weg.
Die Grenzen der Summation über $k$ hängen nicht von $m$ ab, daher
können wir die Summationsreihenfolge vertauschen.
Die Summation über $m$ entspricht den horizontalen blauen Streifen
in 
Abbildung~\ref{buch:differentialgleichungen:bessel:fig:indexmenge}
unten.
Es ergibt sich die Summe}
&=
\sum_{k=0}^\infty
\sum_{m=0}^\infty
\frac{(-1)^k}{k!}
\biggl(\frac{x}{2}\biggr)^k
z^{-k}
\frac{1}{\Gamma(m+1)}
\biggl(\frac{x}{2}\biggr)^{m}
z^{m}
\notag
\\
&=
\sum_{k=0}^\infty \frac{(-1)^k}{k!}
\biggl(\frac{x}{2}\biggr)^k
z^{-k}
\cdot
{\color{blue}
\sum_{m=0}^\infty
\frac{1}{\Gamma(m+1)}
\biggl(\frac{x}{2}\biggr)^{m}
z^{m}
}.
\label{buch:differentialgleichungen:bessel:eqn:blauesumme}
\intertext{Beide Reihen sind Exponentialreihen, was man besser sehen kann,
wenn man die Gamma-Funktion in der zweiten Summe wieder als die
Fakultät $\Gamma(m+1)=m!$ schreibt.
Die beiden Exponentialreihen sind
}
&=
\sum_{k=0}^\infty \frac{\bigl(-\frac{x}2\cdot\frac1z\bigr)}{k!}
\cdot
\sum_{m=0}^\infty
\frac{\bigl(z\frac{x}2\bigr)^m}{m!}
=
\exp\biggl(\frac{x}2\cdot\biggl(-\frac1z\biggr)\biggr)
\cdot
\exp\biggl(\frac{x}2\cdot z\biggr)
=
\exp\biggl(\frac{x}2\cdot\biggl(z-\frac1z\biggr)\biggr).
\notag
\end{align}
Wir fassen das Resultat im folgenden Satz zusammen.

\begin{satz}[Erzeugende Funktion der Bessel-Funktionen]
Die erzeugende Funktion der Besselfunktionen ist
\[
\sum_{k=-\infty}^\infty
J_k(x)
=
\exp\biggl(
\frac{x}2\cdot\biggl(1-\frac1z\biggr)
\biggr)
\]
\end{satz}
\index{erzeugende Funktion}%

%
% Additionstheorem
%
\subsubsection{Additionstheorem}
Die erzeugende Funktion kann dazu verwendet werden, das Additionstheorem
für die Bessel-Funktionen zu beweisen.

\begin{satz}
\index{Satz!Additionstheorem für Bessel-Funktionen}%
Für $l\in\mathbb{Z}$ und $x,y\in\mathbb{R}$ gilt
\[
J_l(x+y) = \sum_{m=-\infty}^\infty J_m(x)J_{l-m}(y).
\]
\end{satz}

\begin{proof}[Beweis]
Die Koeffizienten der erzeugenden Funktion der Bessel-Funktionen für
das Argument $x+y$ ist
\begin{align*}
\exp\biggl(\frac{x+y}2\biggl(z+\frac1z\biggr)\biggr)
&=
\sum_{n=-\infty}^\infty J_n(x+y)z^n.
\intertext{%
Wir verwenden die Exponentialgesetze auf der linken Seite und 
erhalten}
&=
\exp\biggl(\frac{x}2\biggl(z+\frac1z\biggr)\biggr)
\cdot
\exp\biggl(\frac{y}2\biggl(z+\frac1z\biggr)\biggr).
\intertext{Beide Faktoren sind erzeugende Funktionen von Bessel-Funktionen,
wir können sie also als}
&=
\sum_{m=-\infty}^\infty J_m(x)z^m
\cdot
\sum_{k=-\infty}^\infty J_k(y)z^k
\intertext{schreiben.
Durch Ausmultiplizieren und Zusammenfassen von Termen mit gleichem
Exponenten finden wir
}
&=
\sum_{m,k} J_m(x)J_k(y) z^{k+m}
=
\sum_{l=-\infty}^\infty
\biggl(
\sum_{m=-\infty}^\infty J_m(x)J_{l-m}(y)
\biggr)
z^l.
\intertext{Daraus folgt schliesslich mit Koeffizientenvergleich das
Additionstheorem}
J_l(x+y) &= \sum_{m=-\infty}^\infty J_m(x)J_{l-m}(y)
\end{align*}
für alle $l$.
\end{proof}

%
% Der Fall \alpha=0
% 
\subsubsection{Der Fall $\alpha=0$}
Im Fall $\alpha=0$ hat das Indexpolynom eine doppelte Nullstelle, wir
können daher nur eine Lösung erwarten.
Im Fall $\alpha=0$ wird das Produkt im Nenner zu $n!$, so dass die
Lösungsfunktion
\[
J_0(x)
=
\sum_{k=0}^\infty
\frac{(-1)^k}{(k!)^2}
\biggl(\frac{x}{2}\biggr)^{2k}
\]
geschrieben werden kann.


%
% Der Fall \alpha=p, p\in \mathbb{N}
%
\subsubsection{Der Fall $\alpha=p$, $p\in\mathbb{N}, p > 0$}
In diesem Fall kann nur die erste
Lösung~\eqref{buch:differentialgleichunge:bessel:erste}
verwendet werden.
Damit erhält die Lösungsfunktion die Form
\[
J_p(x)
=
\sum_{k=0}^\infty
\frac{(-1)^k}{k!(p+k)!}\biggl(\frac{x}{2}\biggr)^{p+2k}.
\]

%
% Der Fall $\alpha=n+\frac12$
%
\subsubsection{Der Fall $\alpha=n+\frac12$, $n\in\mathbb{N}$}
Obwohl $2\alpha$ eine Ganzzahl ist, sind die beiden Lösungen
\label{buch:differentialgleichunge:bessel:erste}
und
\label{buch:differentialgleichunge:bessel:zweite}
linear unabhängig.

Man kann zeigen, dass sich die Lösungsfunktionen in diesem Fall
durch bereits bekannte elementare Funktionen ausdrücken lassen.
Wir rechnen dies für $n=0$ nach.
Zunächst drücken wir die Pochhammer-Symbole im Nenner anders aus.
Es ist
\begin{align*}
\biggl(\frac12 + 1\biggr)_k
&=
\biggl(\frac12 + 1\biggr)
\biggl(\frac12 + 2\biggr)
\cdots
\biggl(\frac12 + k\biggr)
=
\frac{1}{2^k}\bigl(3\cdot 5\cdot\ldots\cdot (2k+1)\bigr)
=
\frac{(2k+1)!}{2^{2k}\cdot k!}
\\
\biggl(-\frac12 + 1\biggr)_k
&=
\biggl(-\frac12 + 1\biggr)
\biggl(-\frac12 + 2\biggr)
\cdots
\biggl(-\frac12 + k\biggr)
\\
&=
\biggl(\frac12 + 0\biggr)
\biggl(\frac12 + 1\biggr)
\cdots
\biggl(\frac12 + k-1\biggr)
=
\frac{1}{2^k}\bigl(1\cdot 3 \cdot\ldots\cdot (2(k-1)+1)\bigr)
=
\frac{(2k-1)!}{2^{2k-1}\cdot (k-1)!}
\end{align*}
Damit können jetzt die Reihenentwicklungen der Lösung wie folgt
umgeformt werden
\begin{align*}
y_1(x)
&=
x^\alpha
\sum_{k=0}^\infty
\frac{1}{(\alpha+1)_k}
\frac{(-x^2/4)^k}{k!}
=
\sqrt{x}
\sum_{k=0}^\infty
\frac{2^{2k}k!}{(2k+1)!}
\frac{(-x^2/4)^k}{k!}
=
\sqrt{x}
\sum_{k=0}^\infty
(-1)^k
\frac{x^{2k}}{(2k+1)!}
\\
&=
\frac{1}{\sqrt{x}}
\sum_{k=0}^\infty
(-1)^k
\frac{x^{2k+1}}{(2k+1)!}
=
\frac{1}{\sqrt{x}} \sin x
\\
y_2(x)
&=
x^{-\alpha}
\sum_{k=0}^\infty
\frac{1}{(-\alpha+1)_k}
\frac{(-x^2/4)^k}{k!}
=
x^{-\frac12}
\sum_{k=0}^\infty
\frac{2^{2k-1}\cdot (k-1)!}{(2k-1)!}
\frac{(-x^2/4)^k}{k!}
\\
&=
\frac{1}{\sqrt{x}}
\sum_{k=0}^\infty
(-1)^k
\frac{x^{2k}}{(2k-1)!\cdot 2k}
=
\frac{1}{\sqrt{x}} \cos x.
\end{align*}

Die Bessel-Funktionen verwenden aber eine andere Normierung. 
Die Gleichung~\eqref{buch:differentialgleichungen:bessel:normierungsgleichung}
zeigt, dass die Bessel-Funktionen durch Division
der Funktion $y_1(x)$ und $y_2(x)$ durch $2^\alpha \Gamma(\alpha+1)$ 
erhalten werden können.
Dies ergibt
\begin{equation*}
\renewcommand{\arraycolsep}{1pt}
\begin{array}{rclclclcl}
J_{\frac12}(x)
&=&
\displaystyle\frac{1}{2^{\frac12}\Gamma(\frac12+1)}
y_1(x)
&=&
\displaystyle\frac{1}{2^{\frac12}\frac12\Gamma(\frac12)}
y_1(x)
&=&
\displaystyle\frac{\sqrt{2}}{\Gamma(\frac12)}
y_1(x)
&=&
\displaystyle\frac{1}{\Gamma(\frac12)}
\sqrt{ \frac{2}{x}}
\sin x,
\\
J_{-\frac12}(x)
&=&
\displaystyle\frac{1}{2^{-\frac12}\Gamma(-\frac12+1)}
y_2(x)
&=&
\displaystyle\frac{2^{\frac12}}{\Gamma(\frac12)}
y_2(x)
&=&
\displaystyle\frac{\sqrt{2}}{\Gamma(\frac12)}
y_2(x)
&=&
\displaystyle\frac{1}{\Gamma(\frac12)}
\sqrt{\frac{2}{x}}
\cos x.
\end{array}
\end{equation*}
Wegen $\Gamma(\frac12)=\sqrt{\pi}$ sind die
halbzahligen Bessel-Funktionen daher
\begin{align*}
J_{\frac12}(x)
&=
\sqrt{\frac{2}{\pi x}} \sin x
=
\sqrt{\frac{2}{\pi}} x^{-\frac12}\sin x
&
&\text{und}&
J_{-\frac12}(x)
&=
\sqrt{\frac{2}{\pi x}} \cos x
=
\sqrt{\frac{2}{\pi}} x^{-\frac12}\cos x.
\end{align*}

%
% Direkte Verifikation der Lösungen
%
\subsubsection{Direkte Verifikation der Lösungen für $\alpha=\pm\frac12$}
Tatsächlich führt die Anwendung des Bessel-Operators auf die beiden
Funktionen auf
\begin{align*}
\sqrt{\frac{\pi}2}
BJ_{\frac12}(x)
&=
\sqrt{\frac{\pi}2}
\biggl(
x^2J_{\frac12}''(x) + xJ_{\frac12}'(x) + x^2J_{\frac12}(x)
\biggr)
\\
&=
x^2(x^{-\frac12}\sin x)''
+
x(x^{-\frac12}\sin x)'
+
x^2(x^{-\frac12}\sin x)
\\
&=
x^2(
x^{-{\textstyle\frac12}}\cos x
-{\textstyle\frac12}x^{-\frac32}\sin x
)'
+
x(
x^{-\frac12}\cos x
-{\textstyle\frac12}x^{-\frac32}\sin x
)
+
x^{\frac32}\sin x
\\
&=
x^2(
-x^{-\frac12}\sin x
-{\textstyle\frac12}x^{-\frac32}\cos x
-{\textstyle\frac12}x^{-\frac32}\cos x
+{\textstyle\frac{3}{4}}x^{-\frac52}\sin x
)
+
x^{\frac12}\cos x
+
x^{-\frac12}(x-{\textstyle\frac12})\sin x
\\
&=
(
-x^{\frac32}
+{\textstyle\frac34}x^{-\frac12}
+x^{\frac32}
-{\textstyle\frac12}x^{-\frac12}
)
\sin x
=
\frac14x^{-\frac12}\sin x
=
\frac14
\sqrt{\frac{\pi}2}
J_{\frac12}(x)
\\
BJ_{\frac12}(x)
&=
\biggl(\frac12\biggr)^2 J_{\frac12}(x).
\end{align*}
Dies zeigt, dass $J_{\frac12}(x)$ tatsächlich eine Eigenfunktion
des Bessel-Operators zum Eigenwert $\alpha^2 = \frac14$ ist.
Analog kann man die Lösung $y_2(x)$ für $-\frac12$ verifizieren.



%\section{TODO}
%\begin{itemize}
%\end{itemize}

%\section*{Übungsaufgaben}
%\rhead{Übungsaufgaben}
%\aufgabetoplevel{chapters/070-orthogonalitaet/uebungsaufgaben}
%\begin{uebungsaufgaben}
%\uebungsaufgabe{0}
%\uebungsaufgabe{1}
%\end{uebungsaufgaben}

