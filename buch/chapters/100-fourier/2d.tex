%
% 2d.tex
%
% (c) 2022 Prof Dr Andreas Müller, OST Ostschweizer Fachhochschule
%
\section{Zweidimensionale Fourier-Transformation
\label{buch:fourier:section:2d}}
\rhead{Zweidimensionale Fourier-Transformation}
Die Schwingung einer rechteckigen Membran, wie sie im
Abschnitt~\ref{buch:pde:section:rechteck} berechnet wird,
verlangt nach Funktionen, die von zwei Variablen abhängen.
Das Separationsverfahren hat gezeigt, dass dazu Produkte
von trigonometrischen Funktionen verwendet werden müssen.
Analog zur eindimensionalen Fourier-Theorie ergibt sich
ein Darstellung beleibiger zweifach periodischer Funktionen.
Auch die Theorie der Fourier-Transformation kann auf
den Definnitionsbereich $\mathbb{R}^2$ ausgedehnt werden.
Hier werden nur die wichtigsten Resultate zusammengestellt,
die für die Herleitungen von
Abschnitt~\ref{buch:fourier:section:fourier-und-bessel}
benöigt werden.

%
% Zweidimensionale Fourier-Reihe
%
\subsection{Zweidimensionale Fourier-Reihen}
Seien $f(x,y)$ und $g(x,y)$ in beiden Variablen $x$ und $y$
$2\pi$-periodische Funktionen.
Mit dem Skalarprodukt
\[
\langle f,g\rangle
=
\frac{1}{(2\pi)^2}
\int_{-\pi}^\pi
\int_{-\pi}^\pi
\overline{f(x,y)} g(x,y)
\,dx
\,dy
\]
kann man aus der orthogonalen Funktionenfamilie
\[
e_{kl}(x,y)
=
e^{i(kx + ly)}
\]
für ganzzahlige $k$ und $l$ eine Reihenentwicklung
\[
f(x,y)
=
\sum_{k=-\infty}^\infty
\sum_{l=-\infty}^\infty
a_{kl}
e_{kl}(x,y)
\qquad\text{mit}\qquad
a_{kl}
=
\langle e_{kl},f\rangle.
\]
Unter geeigneten Voraussetzungen an die Funktion $f$ kann sichergestellt
werden, dass die Reihe konvergiert.

%
% Zweidimensionale Fourier-Transformation
%
\subsection{Zweidimensionale Fourier-Transformation}
Für Funktionen $f,g\colon \mathbb{R}^2 \to \mathbb{C}$, kann das komplexe
Skalarprodukt
\[
\langle f,g\rangle
=
\int_{-\infty}^\infty
\int_{-\infty}^\infty
\overline{f(x,y)} g(x,y)
\,dx\,dy
\]
definiert werden.
Mit den Funktionen
\[
e_{k,l}(x,y)
=
e^{i(kx+ly)}
\]
kann man jetzt die Fourier-Transformation
\[
\hat{f}(k,l)
=
\mathscr{F}f(k,l)
=
\frac{1}{2\pi}
\int_{-\infty}^\infty
\int_{-\infty}^\infty
f(x,y) e^{i(kx+ly)}
\,dx\,dy
\]
mit der Umkehrung
\[
f(x,y)
=
\frac{1}{2\pi}
\int_{-\infty}^\infty
\int_{-\infty}^\infty
\mathscr{F}f(k,l) e^{-i(kx+ly)}
\,dk\,dl
\]
Ausserdem gilt die Parseval-Identität
\[
\langle f,g\rangle
=
\langle \mathscr{F}f,\mathscr{F}g\rangle.
\]



