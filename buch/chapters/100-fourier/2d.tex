%
% 2d.tex
%
% (c) 2022 Prof Dr Andreas Müller, OST Ostschweizer Fachhochschule
%
\section{Zweidimensionale Fourier-Transformation
\label{buch:fourier:section:2d}}
\kopfrechts{Zweidimensionale Fourier-Transformation}
Die Schwingung einer rechteckigen Membran, wie sie im
Abschnitt~\ref{buch:pde:section:rechteck} berechnet wird,
verlangt nach Funktionen, die von zwei Variablen abhängen.
Das Separationsverfahren hat gezeigt, dass dazu Produkte
von trigonometrischen Funktionen verwendet werden müssen.
Analog zur eindimensionalen Fourier-Theorie ergibt sich
ein Darstellung beliebiger zweifach periodischer Funktionen.
Auch die Theorie der Fourier-Transformation kann auf
den Definnitionsbereich $\mathbb{R}^2$ ausgedehnt werden.
Hier werden nur die wichtigsten Resultate zusammengestellt,
die für die Herleitungen von
Abschnitt~\ref{buch:fourier:section:fourier-und-bessel}
benötigt werden.

%
% Zweidimensionale Fourier-Reihe
%
\subsection{Zweidimensionale Fourier-Reihen}
Seien $f(x,y)$ und $g(x,y)$ in beiden Variablen $x$ und $y$
$2\pi$-periodische Funktionen.
Mit dem Skalarprodukt
\[
\langle f,g\rangle
=
\frac{1}{(2\pi)^2}
\int_{-\pi}^\pi
\int_{-\pi}^\pi
\overline{f(x,y)} g(x,y)
\,dx
\,dy
\]
kann man aus der orthogonalen Funktionenfamilie
\[
e_{kl}(x,y)
=
e^{i(kx + ly)}
\]
für ganzzahlige $k$ und $l$ eine Reihenentwicklung
\[
f(x,y)
=
\sum_{k=-\infty}^\infty
\sum_{l=-\infty}^\infty
a_{kl}
e_{kl}(x,y)
\qquad\text{mit}\qquad
a_{kl}
=
\langle e_{kl},f\rangle.
\]
Unter geeigneten Voraussetzungen an die Funktion $f$ kann sichergestellt
werden, dass die Reihe konvergiert.

%
% Zweidimensionale Fourier-Transformation
%
\subsection{Zweidimensionale Fourier-Transformation}
Für Funktionen $f,g\colon \mathbb{R}^2 \to \mathbb{C}$ kann das komplexe
Skalarprodukt
\[
\langle f,g\rangle
=
\int_{-\infty}^\infty
\int_{-\infty}^\infty
\overline{f(x,y)} g(x,y)
\,dx\,dy
\]
definiert werden, sofern sie quadratintegrierbar sind, also in
\[
L^2(\mathbb{R}^2)
=
\left\{
h\colon \mathbb{R}^2\to \mathbb{C}
\;\left|\;
\int_{-\infty}^\infty |h(x,y)|^2\,dx\,dy
\right.
\right\}
\]
\index{L2R2@$L^2(\mathbb{R}^2)$}%
liegen.

Mit den Funktionen
\(
e_{kl}(x,y)
=
e^{i(kx+ly)}
\),
die aber selbst nicht in $L^2(\mathbb{R}^2)$ sind,
kann man jetzt die Fourier-Transformation
\begin{equation}
\hat{f}(k,l)
=
\mathscr{F}f(k,l)
=
\frac{1}{2\pi}
\int_{-\infty}^\infty
\int_{-\infty}^\infty
f(x,y) e^{i(kx+ly)}
\,dx\,dy
\label{buch:fourier:eqn:2dtransform}
\end{equation}
mit der Umkehrung
\[
f(x,y)
=
\frac{1}{2\pi}
\int_{-\infty}^\infty
\int_{-\infty}^\infty
\mathscr{F}f(k,l) e^{-i(kx+ly)}
\,dk\,dl
\]
konstruieren.
Ausserdem gilt die Parseval-Identität
\index{Parseval-Identität}%
\begin{align*}
\langle f,g\rangle
&=
\langle \mathscr{F}f,\mathscr{F}g\rangle
\\
\|f\|
&=
\|\mathscr{F}f\|.
\end{align*}



