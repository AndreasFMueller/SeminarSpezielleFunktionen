%
% ellintegral.tex
%
% (c) 2021 Prof Dr Andreas Müller, OST Ostschweizer Fachhochschule
%
\section{Elliptische Integrale
\label{buch:elliptisch:section:integral}}
\rhead{Elliptisches Integral}
Bei der Berechnung des Ellipsenbogens in 
Abschnitt~\ref{buch:geometrie:subsection:hyperbeln-und-ellipsen}
sind wir auf ein Integral gestossen, welches sich nicht in geschlossener
Form ausdrücken liess.
Um solche Integrale in den Griff zu bekommen, ist es nötig, sie als
neue spezielle Funktionen zu definieren.

\subsection{Definition
\label{buch:elliptisch:subsection:definition}}
Ein {\em elliptisches Integral} ist ein Integral der Form
\index{elliptishes Integral}%
\index{Integral, elliptisch}%
\begin{equation}
\int R\left( x, \sqrt{p(x)}\right)\,dx
\label{buch:elliptisch:def:allgemein}
\end{equation}
wobei $R(x,y)$ eine rationale Funktion von zwei Variablen ist und
$p(x)$ ein Polynom dritten oder vierten Grades.
Hätte $p(x)$ ein mehrfache Nullstelle $x_0$, müsste es durch $(x-x_0)^2$
teilbar sein, man könnte also einen Faktor $(x-x_0)$ aus der
Wurzel im Integraneden von \eqref{buch:elliptisch:def:allgemein}
ausklammern und damit das Integral in eine Form bringen, wo $p(x)$
höchstens zweiten Grades ist.
Solche Integrale lassen sich meistens mit trigonometrischen Substitutionen
berechnen.
Wir verlangen daher, dass $p(x)$ keine mehrfachen Nullstellen hat.

Man kann zeigen, dass sich elliptische Integrale in Summen von
elementaren Funktionen und speziellen elliptischen Integralen 
der folgenden Form überführen lassen
\cite[Abschnitt 164, p.~506]{buch:smirnov32}.

\begin{definition}
\label{buch:elliptisch:def:integrale123}
Die elliptischen Integrale erster, zweiter und dritter Art sind die
Integrale
\[
\begin{aligned}
\text{1.~Art:}&&&
\int \frac{dx}{\sqrt{(1-x^2)(1-k^2x^2)}}
\\
\text{2.~Art:}&&&
\int \sqrt{\frac{1-k^2x^2}{1-x^2}}\,dx
\\
\text{3.~Art:}&&&
\int \frac{dx}{(1-nx^2)\sqrt{(1-x^2)(1-k^2x^2)}}
\end{aligned}
\]
mit $0<k<1$.
Es ist auch üblich, den Parameter $m=k^2$ zu verwenden.
\end{definition}

Wie gesagt lassen sich für diese unbestimmten Integrale keine 
geschlossenen Formen finden.
Es bleibt uns daher nichts anderes übrig, als die Integralgrenzen
festzulegen und damit eine Stammfunktion auszuwählen.

%
% Elliptisches Integral
%
\subsection{Vollständige elliptische Integrale
\label{buch:elliptisch:subsection:vollstaendig}}
In diesem Abschnitt legen wir beide Integrationsgrenzen fest und
untersuchen die entstehenenden Funktionen von den Parametern
$k$ und $n$.

\subsubsection{Definition der vollständigen elliptischen Integrale}
Da der Nenner in allen drei elliptischen Integralen eine Nullstelle
bei $\pm1$ hat, kann das Integral nur von $0$ bis $1$ erstreckt werden.

\begin{definition}
\label{buch:elliptisch:def:vollstintegrale123}
Die vollständigen elliptischen Integrale erster, zweiter und dritter
Art sind
\[
\begin{aligned}
\text{1.~Art:}&&
K(k)&=\int_0^1 \frac{dt}{\sqrt{(1-t^2)(1-k^2t^2)}} \\
\text{2.~Art:}&&
E(k)&=\int_0^1 \sqrt{\frac{1-k^2t^2}{1-t^2}}\,dt \\
\text{3.~Art:}&&
\Pi(n, k)&=\int_0^1\frac{dt}{(1-nt^2)\sqrt{(1-t^2)(1-k^2t^2)}} 
\end{aligned}
\]
mit $0<k<1$.
\end{definition}

Die Funktionen hängen stetig von $k$ ab.
Die Nullstellen des Faktors $1-k^2x^2$ liegen ausserhalb des
Integrationsintervalls und spielen daher keine Rolle.
Die Werte von $K(k)$ und $E(k)$ für $k=0$ können direkt berechnet
werden:
\begin{align*}
K(0)
=
E(0)
&=
\int_0^1 \frac{dt}{\sqrt{1-t^2}}=\frac{\pi}2.
\end{align*}
Das Integral $\Pi(n,0)$ ist etwas komplizierter.

Für $k\to 1$ ist $E(k)=1$, die Integrale $K(1)$ und $\Pi(n,1)$
sind dagegen divergent.

\subsubsection{Jacobi- und Legendre-Normalform}
Die Integrationsvariable $t$ der vollständigen elliptischen Integrale
kann durch die Substitution $t=\sin\varphi$ durch die Variable
$\varphi$ und das Integral über das Intervall $[0,1]$ durch ein
Integral über das Intervall $[0,\frac{\pi}2]$ ersetzt werden.
Mit
\[
\frac{dt}{d\varphi} = \cos\varphi = \sqrt{1-\sin^2\varphi}
\]
können die Funktionen $K(k)$, $E(k)$ und $\Pi(n,k)$ auch als
\begin{align*}
K(k)
&=
\int_0^{\frac{\pi}2}
\frac{
\sqrt{1-\sin^2\varphi}\,d\varphi
}{
\sqrt{(1-\sin^2\varphi)(1-k^2\sin^2\varphi)}
}
=
\int_0^{\frac{\pi}2}
\frac{d\varphi}{\sqrt{1-k^2\sin^2\varphi}}
,
\\
E(k)
&=
\int_0^{\frac{\pi}2}
\sqrt{\frac{1-k^2\sin^2\varphi}{1-\sin^2\varphi}}\sqrt{1-\sin^2\varphi}\,d\varphi
=
\int_0^{\frac{\pi}2}
\sqrt{1-k^2\sin^2\varphi}\,d\varphi
,
\\
\Pi(n,k)
&=
\int_0^{\frac{\pi}2}
\frac{
\sqrt{1-\sin^2\varphi}\,d\varphi
}{
(1-n\sin^2\varphi)\sqrt{(1-\sin^2\varphi)(1-k^2\sin^2\varphi)}
}
=
\int_0^{\frac{\pi}2}
\frac{
d\varphi
}{
(1-n\sin^2\varphi)\sqrt{1-k^2\sin^2\varphi}
}
.
\end{align*}
Diese Form wird auch die {\em Legendre-Normalform} der vollständigen 
\index{Legendre-Normalform}%
elliptischen Integrale genannt, während die Form von
Definition~\ref{buch:elliptisch:def:vollstintegrale123}
die {\em Jacobi-Normalform} heisst.
\index{Jacobi-Normalform}%

\subsubsection{Umfang einer Ellipse}
\begin{figure}
\centering
\includegraphics{chapters/110-elliptisch/images/ellipsenumfang.pdf}
\caption{Bogenlänge eines Viertels einer Ellipse mit Exzentrizität
$\varepsilon$.
Eine solche Ellipse hat Halbachsen $1$ und $\sqrt{1-\varepsilon^2}$,
ein entsprechender Ellipsenbogen ist für ausgewählte Werte in blau
eingezeichnet.
\label{buch:elliptisch:fig:ellipsenumfang}}
\end{figure}
Wir zeigen, wie sich die Berechnung des Umfangs $U$ einer Ellipse
mit Halbachsen $a$ und $b$, $a\le b$, auf ein volltändiges elliptisches
Integral zurückführen lässt.
Der Fall $a>b$ kann behandelt werden, indem die $x$- und $y$-Koordinaten
vertauscht werden.

Die Parametrisierung
\[
t\mapsto \begin{pmatrix}a\cos t\\ b\sin t\end{pmatrix}
\]
einer Ellipse führt auf das Integral
\begin{align}
U
&=
\int_0^{2\pi} \sqrt{a^2\sin^2t + b^2\cos^2 t}\,dt
\notag
\\
&=
4\int_0^{\frac{\pi}2}
\sqrt{a^2\sin^2t + b^2(1-\sin^2 t)}
\,dt
\notag
\\
&=
4b \int_0^{\frac{\pi}2} \sqrt{1-(b^2-a^2)/b^2\cdot \sin^2t}\,dt
\label{buch:elliptisch:eqn:umfangellipse}
\end{align}
für den Umfang der Ellipse.
Bei einem Kreis ist $a=b$ und der zweite Term unter der Wurzel fällt weg,
der Umfang wird $4b\frac{\pi}2=2\pi b$.
Die Differenz $e^2=b^2-a^2$ ist die {\em lineare Exzentrizität} der Ellipse,
\index{lineare Exzentrizität}%
der Quotient $e/b$ wird die {\em numerische Exzentrizität} der Ellipse
genannt.
Insbesondere ist $k = \varepsilon$.

Das Integral~\eqref{buch:elliptisch:eqn:umfangellipse} erhält jetzt die
Form
\[
U
=
4b\int_0^{\frac{\pi}2} \sqrt{1-k^2\sin^2t}\,dt
\]
und ist damit als elliptisches Integral zweiter Art erkannt.
Für den Umfang der Ellipse finden wir damit die Formel
\[
U
=
4b E(k)
=
4b E(\varepsilon).
\]
Das vollständige elliptische Integral zweiter Art $E(\varepsilon)$
liefert also genau den Umfang eines Viertels der Ellipse mit
numerischer Exzentrizität $\varepsilon$ und kleiner Halbachse $1$.
Für den extremen Wert $\varepsilon=0$ entsteht der Umfang einer Ellipse,
also $E(0)=\frac{\pi}2$.
Für $\varepsilon=1$ ist $a=0$, es entsteht eine Strecke mit Länge $E(1)=1$.

\subsubsection{Komplementäre Integrale}

\subsubsection{Ableitung}
XXX Ableitung \\
XXX Stammfunktion \\

\subsection{Unvollständige elliptische Integrale}
Die Funktionen $K(k)$ und $E(k)$ sind als bestimmte Integrale über ein
festes Intervall definiert.
Die {\em unvollständigen elliptischen Integrale} entstehen, indem die
\index{unvollständiges elliptisches Integral}%
obere Grenze des Integrals variabel wird:
\[
\begin{aligned}
\text{1.~Art:}&&
F(x,k)
&=
\int_0^x \frac{dt}{\sqrt{(1-t^2)(1-k^2t^2)}}
&&=
\int_0^\varphi \frac{d\vartheta}{\sqrt{1-k^2\sin^2\vartheta}}
\\
\text{2.~Art:}&&
E(x,k)
&=
\int_0^x \sqrt{\frac{1-k^2t^2}{1-t^2}}\,dt
&&=
\int_0^\varphi \sqrt{1-k^2\sin^2\vartheta}\,d\vartheta
\\
\text{3.~Art:}&&
\Pi(n,x,k)
&=
\int_0^x \frac{dt}{(1-nt^2)\sqrt{(1-t^2)(1-k^2t^2)}}
&&=
\int_0^\varphi
\frac{d\vartheta}{(1-n\sin^2\vartheta)\sqrt{1-k^2\sin^2\vartheta}},
\end{aligned}
\]
die erste Formel ist jeweils die Jacobi-Form, die zweite die Legrendre-Form
\index{Jacobi-Form}%
\index{Legendre-Form}%
mit dem Parameter $\varphi$, gegeben durch
$\sin \vartheta=x$.
Wie bei den vollständigen elliptischen Integralen ist auch hier in manchen
Referenzen die Parameterkonvention mit dem Parameter $m=k^2$ üblich.

Die vollständigen elliptischen Integrale sind die Werte der 
unvollständigen elliptischen Integrale mit $x=1$, also
\begin{align*}
K(k) &= F(1,k),
&
E(k) &= E(1,k),
&
\Pi(n,k) &=\Pi(n,x,k).
\end{align*}
Man beachte auch, dass $F(x,0) = E(x,0)$ gilt.

\begin{figure}
\centering
\includegraphics{chapters/110-elliptisch/images/unvollstaendig.pdf}
\caption{Unvollständige elliptische Integrale $F(x,k)$ und $E(x,k)$
für verschiedene Werte des Parameters $k$.
Für $k=0$ stimmen die Integrale erster und zweiter Art überein,
$F(x,0)=E(x,0)$.
\label{buch:elliptisch:fig:unvollstaendigeintegrale}}
\end{figure}
Wegen $k<1$ sind alle drei Integranden als reelle Funktionen nicht
mehr definiert, wenn $|x|>1$ ist.
Die Abbildung~\ref{buch:elliptisch:fig:unvollstaendigeintegrale}
zeigt Graphen der unvollständigen elliptischen Integrale für verschiedene
Werte des Parameters.

\subsubsection{Symmetrieeigenschaften}
Die Integranden aller drei unvollständigen elliptischen Integrale
sind gerade Funktionen der reellen Variablen $t$.
Die Funktionen $F(x,k)$, $E(x,k)$ und $\Pi(n,x,k)$ sind daher
ungeraden Funktionen von $x$.

\subsubsection{Elliptische Integrale als komplexe Funktionen}
Die unvollständigen elliptischen Integrale $F(x,k)$, $F(x,k)$ und $\Pi(n,x,k)$
in Jacobi-Form lassen sich auch für komplexe Argumente interpretieren.
Dazu muss für die Berechnung des Integrals ein Pfad in der komplexen
Ebene gewählt werden, der die Singulariätten des Integranden vermeidet.

Die Faktoren, die in den Integranden der unvollständigen elliptischen
Integrale vorkommen, haben Nullstellen bei $\pm1$, $\pm1/k$ und
$\pm 1/\sqrt{n}$

XXX Additionstheoreme \\
XXX Parameterkonventionen \\
XXX Wertebereich (Rechtecke) \\
XXX Komplementäre Integrale \\

\subsection{Potenzreihe}
XXX Potenzreihen \\
XXX Als hypergeometrische Funktionen \\
XXX Berechnung mit der Landen-Transformation https://en.wikipedia.org/wiki/Landen%27s_transformation
