%
% ellselection.tex -- Wahl einer elliptischen Funktion
%
% (c) 2021 Prof Dr Andreas Müller, OST Ostschweizer Fachhochschule
%
\documentclass[tikz]{standalone}
\usepackage{amsmath}
\usepackage{times}
\usepackage{txfonts}
\usepackage{pgfplots}
\usepackage{csvsimple}
\usetikzlibrary{arrows,intersections,math}
\begin{document}
\def\skala{1}
\begin{tikzpicture}[>=latex,thick,scale=\skala]

\input{rechteckpfade2.tex}

\def\l{0.45}
\pgfmathparse{\l*72/2.54}
\xdef\L{\pgfmathresult}

\pgfmathparse{4.1/\xmax}
\xdef\dx{\pgfmathresult}
\xdef\dy{\dx}

\def\sx{4.1}
\pgfmathparse{\sx*72/2.54}
\xdef\Sx{\pgfmathresult}

\pgfmathparse{\dx*\ymax}
\xdef\sy{\pgfmathresult}
\pgfmathparse{\sy*72/2.54}
\xdef\Sy{\pgfmathresult}

\pgfmathparse{\sx/2-\l}
\xdef\linksx{\pgfmathresult}
\pgfmathparse{\sy/2-\l}
\xdef\linksy{\pgfmathresult}

\pgfmathparse{\sx/2+2*\l}
\xdef\rechtsx{\pgfmathresult}
\pgfmathparse{\sy/2}
\xdef\rechtsy{\pgfmathresult}

\begin{scope}
	\clip (-\sx,-\sy) rectangle (\sx,\sy);
	\begin{scope}[xshift={-\Sx}]
		\hintergrund
		\netz{0.7pt}
	\end{scope}
	\begin{scope}[xshift={\Sx}]
		\hintergrund
		\netz{0.7pt}
	\end{scope}
\end{scope}

\fill[color=red!14,opacity=0.7] ({-\sx},0) rectangle (\sx,\sy);
\fill[color=blue!14,opacity=0.7] ({-\sx},{-\sy}) rectangle (\sx,0);
\fill[color=yellow!40,opacity=0.5] (0,0) rectangle (\sx,\sy);

\draw (-\sx,-\sy) rectangle (\sx,\sy);

\draw[->] ({-1.4*\sx},0) -- ({1.4*\sx},0) coordinate[label={$\Re u$}];
\draw[->] (0,{-\sy-1}) -- (0,{\sy+1}) coordinate[label={right:$\Im u$}];

\definecolor{darkgreen}{rgb}{0,0.6,0}

\draw[->,line width=1.9pt,color=darkgreen]
	(\sx,0) to[out=180,in=-79] (\linksx,\linksy);
\draw[->,line width=1.9pt,color=darkgreen]
	(\sx,{\sy-\l}) to[out=-90,in=0] (\rechtsx,\rechtsy);

\def\rect#1#2{
	\fill[color=white] (-\l,-\l) rectangle (\l,\l);
	#2
	\draw (-\l,-\l) rectangle (\l,\l);
	\node at (0,0) {\Huge #1\strut};
}

\def\kreuz{
	\begin{scope}
	\clip ({-\l},{-\l}) rectangle ({\l},{\l});
	\fill[color=white] ({-2*\l},{-2*\l}) rectangle ({2*\l},{2*\l});
	\draw[color=darkgreen!30,line width=3pt] (-\l,-\l) -- (\l,\l);
	\draw[color=darkgreen!30,line width=3pt] (-\l,\l) -- (\l,-\l);
	\end{scope}
}

\def\kreis{
	\begin{scope}
	\clip ({-\l},{-\l}) rectangle ({\l},{\l});
	\fill[color=white] ({-2*\l},{-2*\l}) rectangle ({2*\l},{2*\l});
	\draw[color=darkgreen!30,line width=3pt]
		(0,0) circle[radius={\l*(\L-1.5)/\L}];
	\end{scope}
}

\begin{scope}[xshift={0},yshift={0}]
	\rect{s}{}
\end{scope}

\begin{scope}[xshift={\Sx},yshift={0}]
	\rect{c}{\kreis}
\end{scope}

\begin{scope}[xshift={\Sx},yshift={\Sy}]
	\rect{d}{\kreuz}
\end{scope}

\begin{scope}[xshift={0},yshift={\Sy}]
	\rect{n}{}
\end{scope}

\node at ({-\l+0.1},{\sy+\l-0.1})    [above left]   {$iK'\mathstrut$};
\node at ({-\l+0.1},{-\l+0.1})       [below left]   {$0\mathstrut$};
\node at ({\sx+\l-0.1},{-\l+0.1})    [below right]  {$K\mathstrut$};
\node at ({\sx+\l-0.1},{\sy+\l-0.1}) [above right]  {$K+iK'\mathstrut$};
\node at ({-\sx},0)                  [below left]   {$-K\mathstrut$};
\node at (0,{-\sy+0.05})             [below left]   {$-iK'\mathstrut$};
\node at ({\sx-0.1},{-\sy+0.1})      [below right]  {$K-iK'\mathstrut$};
\node at ({-\sx+0.1},{-\sy+0.1})     [below left]   {$-K-iK'\mathstrut$};
\node at ({-\sx+0.1},{\sy-0.1})      [above left]   {$-K+iK'\mathstrut$};

\begin{scope}[xshift={-\L+0.5*\Sx},yshift={0.5*\Sy}]
	\node at ({-\l},{\l-0.1}) [above] {Nullstelle\strut};
	\kreis
	\node[color=darkgreen] at (0,0) {\Huge c\strut};
	\draw[line width=0.2pt] (-\l,-\l) rectangle (\l,\l);
\end{scope}

\begin{scope}[xshift={\L+0.5*\Sx},yshift={0.5*\Sy}]
	\node at ({\l},{\l-0.1}) [above] {Pol\strut};
	\kreuz
	\node[color=darkgreen] at (0,0) {\Huge d\strut};
	\draw[line width=0.2pt] (-\l,-\l) rectangle (\l,\l);
\end{scope}

\end{tikzpicture}
\end{document}

