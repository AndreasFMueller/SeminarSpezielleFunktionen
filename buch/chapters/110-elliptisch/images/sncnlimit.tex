%
% sncnlimit.tex -- template for standalon tikz images
%
% (c) 2021 Prof Dr Andreas Müller, OST Ostschweizer Fachhochschule
%
\documentclass[tikz]{standalone}
\usepackage{amsmath}
\usepackage{times}
\usepackage{txfonts}
\usepackage{pgfplots}
\usepackage{csvsimple}
\definecolor{darkgreen}{rgb}{0,0.6,0}
\usetikzlibrary{arrows,intersections,math,calc}
\begin{document}
\def\skala{2}
\begin{tikzpicture}[>=latex,thick,scale=\skala]

\def\pitick#1#2{
	\draw (#1,{-0.1/\skala}) -- (#1,{0.1/\skala});
	\node at (#1,{-0.1/\skala}) [below] {$\mathstrut #2$};
}

\def\achsen{
	\draw[->] ({-0.1/\skala},0) -- ({6.4+0.3/\skala},0)
		coordinate[label=$u$];
	\pitick{3.1415}{\pi}
	\pitick{2*3.1415}{2\pi}
	\draw ({-0.1/\skala},1) -- ({0.1/\skala},1);
	\node at ({-0.1/\skala},1) [left] {$1$};
	\node at ({-0.1/\skala},0) [left] {$0$};
}

\begin{scope}
\achsen
\draw[->] (0,{-1-0.1/\skala}) -- (0,{1+0.3/\skala})
	coordinate[label={right:$y$}];
\draw[color=red!50,line width=1.4pt]
	plot[domain=0:365,samples=100] ({3.1415*\x/180},{sin(\x)});
\draw[color=red,line width=1.4pt]
	plot[domain=0:6.4,samples=100]
		({\x},{(exp(\x)-exp(-\x))/(exp(\x)+exp(-\x))});
\node[color=red] at ({1.5*3.1415},1)
	[above] {$\operatorname{sn}(u,1)=\tanh u$};
\node[color=red] at ({1.5*3.1415},{-1+0.14})
	[above] {$\operatorname{sn}(u,0)=\sin u$};
\end{scope}

\begin{scope}[yshift=-2.4cm]
\achsen
\draw[->] (0,{-1-0.1/\skala}) -- (0,{1+0.3/\skala})
	coordinate[label={right:$y$}];

\draw[color=blue!50,line width=1.4pt]
	plot[domain=0:365,samples=100] ({3.1415*\x/180},{cos(\x)});
\draw[color=blue,line width=1.4pt]
	plot[domain=0:6.4,samples=100] ({\x},{2/(exp(\x)+exp(-\x))});
\node[color=blue] at (3.1415,{-1+0.15})
	[above] {$\operatorname{cn}(u,0)=\cos u$};
\node[color=blue] at (3.1415,{2/(exp(3.1415)+exp(-3.1415))+0.05})
	[above] {$\operatorname{sech}u$};
\end{scope}

\begin{scope}[yshift=-4.8cm]
\achsen
\draw[->] (0,{-0-0.1/\skala}) -- (0,{1+0.3/\skala})
	coordinate[label={right:$y$}];
\node[color=darkgreen] at (3.1415,1) [above] {$\operatorname{dn}(u,0) = 1$};
\draw[color=darkgreen,line width=1.4pt]
	plot[domain=0:6.4,samples=100] ({\x},{2/(exp(\x)+exp(-\x))});
\draw[color=darkgreen!50,line width=1.4pt]
	(0,1) -- (6.4,1);
\node[color=darkgreen] at (3.1415,{2/(exp(3.1415)+exp(-3.1415))+0.05})
	[above] {$\operatorname{dn}(u,1)=\operatorname{sech}u$};
\node at (0,{-0.1/\skala}) [below] {$\mathstrut 0$};
\end{scope}

\end{tikzpicture}
\end{document}

