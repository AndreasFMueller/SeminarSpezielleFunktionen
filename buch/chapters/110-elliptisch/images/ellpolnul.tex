%
% ellpolnul.tex -- template for standalon tikz images
%
% (c) 2021 Prof Dr Andreas Müller, OST Ostschweizer Fachhochschule
%
\documentclass[tikz]{standalone}
\usepackage{amsmath}
\usepackage{times}
\usepackage{txfonts}
\usepackage{pgfplots}
\usepackage{csvsimple}
\usetikzlibrary{arrows,intersections,math,calc}
\begin{document}
%
% ellcommon.tex -- common macros/definitions for elliptic function
%                  values display
%
% (c) 2022 Prof Dr Andreas Müller, OST Ostschweizer Fachhochschule
%
\definecolor{rot}{rgb}{0.8,0,0}
\definecolor{blau}{rgb}{0,0,1}
\definecolor{gruen}{rgb}{0,0.6,0}
\def\l{0.2}

\def\pol#1#2{
	\draw[color=#2!40,line width=2.4pt]
		($#1+(-\l,-\l)$) -- ($#1+(\l,\l)$);
	\draw[color=#2!40,line width=2.4pt]
		($#1+(-\l,\l)$) -- ($#1+(\l,-\l)$);
}
\def\nullstelle#1#2{
	\draw[color=#2!40,line width=2.4pt] #1 circle[radius=\l];
}
\def\rechteck#1#2{
	\fill[color=#1!20] (-1,-1) rectangle (1,1);
	\node[color=#1] at (0,0) {$#2$};
}

\def\skala{1}
\begin{tikzpicture}[>=latex,thick,scale=\skala]

\input rechteckpfade3.tex

\pgfmathparse{2/\xmax}
\xdef\dx{\pgfmathresult}
\xdef\dy{\dx}

\begin{scope}[xshift=-1cm,yshift=-1cm]
\clip (0,0) rectangle (2,2);
\netz{0.4pt}
\draw[line width=0.4pt] (-1,0) -- (1,0);
\end{scope}
\fill[color=white,opacity=0.7] (-1,-1) rectangle (1,1);
\draw (-1,-1) rectangle (1,1);
\node at (-1,-1) [below left] {$0$};
\node at (1,-1) [below right] {$K$};
\node at (1,1) [above right] {$K+iK'$};
\node at (-1,1) [above left] {$iK'$};
\node at (0,0) {$u$};

\begin{scope}[xshift=4cm]
\rechteck{rot}{\operatorname{sn}(u,k)}
\nullstelle{(-1,-1)}{rot}
\pol{(-1,1)}{rot}
\node at (-1,-1) {$0$};
\node at (1,-1) {$1$};
\node at (1,1) {$\frac1k$};
\node at (-1,1) {$\infty$};
\end{scope}

\begin{scope}[xshift=7cm]
\rechteck{blau}{\operatorname{cn}(u,k)}
\nullstelle{(1,-1)}{blau}
\pol{(-1,1)}{blau}
\node at (-1,-1) {$1$};
\node at (1,-1) {$0$};
\node at (1,1) {$\frac{k'}{ik}$};
\node at (-1,1) {$\infty$};
\end{scope}

\begin{scope}[xshift=10cm]
\rechteck{gruen}{\operatorname{dn}(u,k)}
\nullstelle{(1,1)}{gruen}
\pol{(-1,1)}{gruen}
\node at (-1,-1) {$1$};
\node at (1,-1) {$k'$};
\node at (1,1) {$0$};
\node at (-1,1) {$\infty$};
\end{scope}

\end{tikzpicture}
\end{document}

