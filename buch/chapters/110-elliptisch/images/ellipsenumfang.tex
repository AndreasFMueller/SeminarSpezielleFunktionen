%
% ellipsenumfang.tex -- template for standalon tikz images
%
% (c) 2021 Prof Dr Andreas Müller, OST Ostschweizer Fachhochschule
%
\documentclass[tikz]{standalone}
\usepackage{amsmath}
\usepackage{times}
\usepackage{txfonts}
\usepackage{pgfplots}
\usepackage{csvsimple}
\usetikzlibrary{arrows,intersections,math}
\begin{document}
\input{ekplot.tex}
\def\skala{1}
\begin{tikzpicture}[>=latex,thick,scale=\skala]

\def\dx{10}
\def\dy{4}

\draw[->] (0,-0.1) -- (0,6.8) coordinate[label={right:$E(\varepsilon)$}];
\draw[->] (-0.1,0) -- (10.5,0) coordinate[label={$\varepsilon$}];
\draw[line width=0.4pt] (0,\dy) -- (10,\dy);
\draw[line width=0.4pt] (\dx,0) -- (10,\dy);

\draw[color=red,line width=1.4pt] \ekpath;
\fill[color=red] (\dx,\dy) circle[radius=0.05];

\foreach \y in {2,4,...,16}{
	\draw (-0.1,{\dy*\y/10})  -- (0.1,{\dy*\y/10});
	\pgfmathparse{\y/10}
	\xdef\v{\pgfmathresult}
	\node at (0,{\dy*\y/10}) [left] {$\v$};
}
\foreach \i in {1,...,9}{
	\draw (\i,-0.1) -- (\i,0.1);
	\node at (\i,0) [below] {$0.\i$};
}
\draw (10,-0.1) -- (10,0.1);
\node at (10,0) [below] {$1.0$};

\end{tikzpicture}
\end{document}

