%
% jacobi.tex
%
% (c) 2021 Prof Dr Andreas Müller, OST Ostschweizer Fachhochschule
%
\section{Jacobische elliptische Funktionen
\label{buch:elliptisch:section:jacobi}}
\rhead{Jacobische elliptische Funktionen}
Die elliptischen Integrale von
Abschnitt~\ref{buch:elliptisch:section:integral}
können dazu verwendet werden, die Länge eines Ellipsenbogens aus
den Koordinaten der Endpunkte zu berechnen.
Die trigonometrischen Funktionen drücken dagegen umgekehrt die
Koordinaten eines Punktes auf einem Kreis aus der Länge des
Kreisbogens aus.
Das elliptische Integral, welches die Bogenlänge auf einer Ellipse zwischen
den Punkten $(1,0)$ und $(x,y)$ entsprecht also eher der Funktion
$\arcsin y=\sin^{-1}y$.
Möchte man Funktionen konstruieren, die die Eigenschaften der 
trigonometrischen Funktionen auf die Geometrie von Ellipsen erweitern,
dann muss man die Umkehrfunktionen der elliptischen Integrale dafür ins
Auge fassen.


%%
%% elliptische Funktionen als Trigonometrie
%%
%\subsection{Elliptische Funktionen als Trigonometrie}
%\begin{figure}
%\centering
%\includegraphics{chapters/110-elliptisch/images/ellipse.pdf}
%\caption{Kreis und Ellipse zum Vergleich und zur Herleitung der 
%elliptischen Funktionen von Jacobi als ``trigonometrische'' Funktionen
%auf einer Ellipse.
%\label{buch:elliptisch:fig:ellipse}}
%\end{figure}
%% based on Willliam Schwalm, Elliptic functions and elliptic integrals
%% https://youtu.be/DCXItCajCyo
%
%%
%% Geometrie einer Ellipse
%%
%\subsubsection{Geometrie einer Ellipse}
%Eine {\em Ellipse} ist die Menge der Punkte der Ebene, für die die Summe
%\index{Ellipse}%
%der Entfernungen von zwei festen Punkten $F_1$ und $F_2$,
%den {\em Brennpunkten}, konstant ist.
%\index{Brennpunkt}%
%In Abbildung~\ref{buch:elliptisch:fig:ellipse} eine Ellipse
%mit Brennpunkten in $F_1=(-e,0)$ und $F_2=(e,0)$ dargestellt,
%die durch die Punkte $(\pm a,0)$ und $(0,\pm b)$ auf den Achsen geht.
%Der Punkt $(a,0)$ hat die Entfernungen $a+e$ und $a-e$ von den beiden
%Brennpunkten, also die Entfernungssumme $a+e+a-e=2a$.
%Jeder andere Punkt auf der Ellipse muss ebenfalls diese Entfernungssumme
%haben, insbesondere auch der Punkt $(0,b)$.
%Seine Entfernung zu jedem Brennpunkt muss aus Symmetriegründen gleich gross,
%also $a$ sein.
%Aus dem Satz von Pythagoras liest man daher ab, dass
%\[
%b^2+e^2=a^2
%\qquad\Rightarrow\qquad
%e^2 = a^2-b^2
%\]
%sein muss.
%Die Strecke $e$ heisst auch {\em (lineare) Exzentrizität} der Ellipse.
%Das Verhältnis $\varepsilon= e/a$  heisst die {\em numerische Exzentrizität}
%der Ellipse.
%
%%
%% Die Ellipsengleichung
%%
%\subsubsection{Ellipsengleichung}
%Der Punkt $P=(x,y)$ auf der Ellipse hat die Entfernungen
%\begin{equation}
%\begin{aligned}
%\overline{PF_1}^2
%&=
%y^2 + (x+e)^2
%\\
%\overline{PF_2}^2
%&=
%y^2 + (x-e)^2
%\end{aligned}
%\label{buch:elliptisch:eqn:wurzelausdruecke}
%\end{equation}
%von den Brennpunkten, für die 
%\begin{equation}
%\overline{PF_1}+\overline{PF_2}
%=
%2a
%\label{buch:elliptisch:eqn:pf1pf2a}
%\end{equation}
%gelten muss.
%Man kann nachrechnen, dass ein Punkt $P$, der die Gleichung
%\[
%\frac{x^2}{a^2} + \frac{y^2}{b^2}=1
%\]
%erfüllt, auch die Eigenschaft~\eqref{buch:elliptisch:eqn:pf1pf2a}
%erfüllt.
%Zur Vereinfachung setzen wir $l_1=\overline{PF_1}$ und $l_2=\overline{PF_2}$.
%$l_1$ und $l_2$ sind Wurzeln aus der rechten Seite von
%\eqref{buch:elliptisch:eqn:wurzelausdruecke}.
%Das Quadrat von $l_1+l_2$ ist
%\[
%l_1^2 + 2l_1l_2 + l_2^2 = 4a^2.
%\]
%Um die Wurzeln ganz zu eliminieren, bringt man das Produkt $l_1l_2$ alleine
%auf die rechte Seite und quadriert.
%Man muss also verifizieren, dass
%\[
%(l_1^2 + l_2^2 -4a^2)^2 = 4l_1^2l_2^2.
%\]
%In den entstehenden Ausdrücken muss man ausserdem $e=\sqrt{a^2-b^2}$ und
%\[
%y=b\sqrt{1-\frac{x^2}{a^2}}
%\]
%substituieren.
%Diese Rechnung führt man am einfachsten mit Hilfe eines
%Computeralgebraprogramms durch, welches obige Behauptung bestätigt.
%
%%
%% Normierung
%%
%\subsubsection{Normierung}
%Die trigonometrischen Funktionen sind definiert als Verhältnisse 
%von Seiten rechtwinkliger Dreiecke.
%Dadurch, dass man den die Hypothenuse auf Länge $1$ normiert, 
%kann man die Sinus- und Kosinus-Funktion als Koordinaten eines
%Punktes auf dem Einheitskreis interpretieren.
%
%Für die Koordinaten eines Punktes auf der Ellipse ist dies nicht so einfach,
%weil es nicht nur eine Ellipse gibt, sondern für jede numerische Exzentrizität
%mindestens eine mit Halbeachse $1$.
%Wir wählen die Ellipsen so, dass $a$ die grosse Halbachse ist, also $a>b$.
%Als Normierungsbedingung verwenden wir, dass $b=1$ sein soll, wie in
%Abbildung~\ref{buch:elliptisch:fig:jacobidef}.
%Dann ist $a=1/\varepsilon>1$.
%In dieser Normierung haben Punkte $(x,y)$ auf der Ellipse $y$-Koordinaten
%zwischen $-1$ und $1$ und $x$-Koordinaten zwischen $-a$ und $a$.
%
%Im Zusammenhang mit elliptischen Funktionen wird die numerische Exzentrizität
%$\varepsilon$ auch mit
%\[
%k
%=
%\varepsilon
%=
%\frac{e}{a}
%=
%\frac{\sqrt{a^2-b^2}}{a}
%=
%\frac{\sqrt{a^2-1}}{a},
%\]
%die Zahl $k$ heisst auch der {\em Modulus}.
%Man kann $a$ auch durch $k$ ausdrücken, durch Quadrieren und Umstellen
%findet man
%\[
%k^2a^2 = a^2-1
%\quad\Rightarrow\quad
%1=a^2(k^2-1)
%\quad\Rightarrow\quad
%a=\frac{1}{\sqrt{k^2-1}}.
%\]
%
%Die Gleichung der ``Einheitsellipse'' zu diesem Modulus ist
%\[
%\frac{x^2}{a^2}+y^2=1
%\qquad\text{oder}\qquad
%x^2(k^2-1) + y^2 = 1.
%\]
%
%%
%% Definition der elliptischen Funktionen
%%
%\begin{figure}
%\centering
%\includegraphics{chapters/110-elliptisch/images/jacobidef.pdf}
%\caption{Definition der elliptischen Funktionen als Trigonometrie
%an einer Ellipse mit Halbachsen $a$ und $1$.
%\label{buch:elliptisch:fig:jacobidef}}
%\end{figure}
%\subsubsection{Definition der elliptischen Funktionen}
%Die elliptischen Funktionen für einen Punkt $P$ auf der Ellipse mit Modulus $k$
%können jetzt als Verhältnisse der Koordinaten des Punktes definieren.
%Es stellt sich aber die Frage, was man als Argument verwenden soll.
%Es soll so etwas wie den Winkel $\varphi$ zwischen der $x$-Achse und dem
%Radiusvektor zum Punkt $P$
%darstellen, aber wir haben hier noch eine Wahlfreiheit, die wir später
%ausnützen möchten.
%Im Moment müssen wir die Frage noch nicht beantworten und nennen das
%noch unbestimmte Argument $u$.
%Wir kümmern uns später um die Frage, wie $u$ von $\varphi$ abhängt.
%
%Die Funktionen, die wir definieren wollen, hängen ausserdem auch 
%vom Modulus ab.
%Falls der verwendete Modulus aus dem Zusammenhang klar ist, lassen
%wir das $k$-Argument weg.
%
%Die Punkte auf dem Einheitskreis haben alle den gleichen Abstand vom
%Nullpunkt, dies ist gleichzeitig die definierende Gleichung $r^2=x^2+y^2=1$
%des Kreises.
%Die Punkte auf der Ellipse erfüllen die Gleichung $x^2/a^2+y^2=1$,
%die Entfernung der Punkte $r=\sqrt{x^2+y^2}$ vom Nullpunkt variert aber.
%
%In Analogie zu den trigonometrischen Funktionen setzen wir jetzt für 
%die Funktionen
%\[
%\begin{aligned}
%&\text{sinus amplitudinis:}&
%{\color{red}\operatorname{sn}(u,k)}&= y \\
%&\text{cosinus amplitudinis:}&
%{\color{blue}\operatorname{cn}(u,k)}&= \frac{x}{a} \\
%&\text{delta amplitudinis:}&
%{\color{darkgreen}\operatorname{dn}(u,k)}&=\frac{r}{a},
%\end{aligned}
%\]
%die auch in Abbildung~\ref{buch:elliptisch:fig:jacobidef}
%dargestellt sind.
%Aus der Gleichung der Ellipse folgt sofort, dass
%\[
%\operatorname{sn}(u,k)^2 + \operatorname{cn}(u,k)^2 = 1
%\]
%ist.
%Der Satz von Pythagoras kann verwendet werden, um die Entfernung zu
%berechnen, also gilt
%\begin{equation}
%r^2
%=
%a^2 \operatorname{dn}(u,k)^2
%=
%x^2 + y^2
%=
%a^2\operatorname{cn}(u,k)^2 + \operatorname{sn}(u,k)^2
%\quad
%\Rightarrow
%\quad
%a^2 \operatorname{dn}(u,k)^2
%=
%a^2\operatorname{cn}(u,k)^2 + \operatorname{sn}(u,k)^2.
%\label{buch:elliptisch:eqn:sncndnrelation}
%\end{equation}
%Ersetzt man
%$
%a^2\operatorname{cn}(u,k)^2
%=
%a^2-a^2\operatorname{sn}(u,k)^2
%$, ergibt sich
%\[
%a^2 \operatorname{dn}(u,k)^2
%=
%a^2-a^2\operatorname{sn}(u,k)^2
%+
%\operatorname{sn}(u,k)^2
%\quad
%\Rightarrow
%\quad
%\operatorname{dn}(u,k)^2
%+
%\frac{a^2-1}{a^2}\operatorname{sn}(u,k)^2
%=
%1,
%\]
%woraus sich die Identität
%\[
%\operatorname{dn}(u,k)^2 + k^2 \operatorname{sn}(u,k)^2 = 1
%\]
%ergibt.
%Ebenso kann man aus~\eqref{buch:elliptisch:eqn:sncndnrelation}
%die Funktion $\operatorname{cn}(u,k)$ eliminieren, was auf
%\[
%a^2\operatorname{dn}(u,k)^2
%=
%a^2\operatorname{cn}(u,k)^2
%+1-\operatorname{cn}(u,k)^2
%=
%(a^2-1)\operatorname{cn}(u,k)^2
%+1.
%\]
%Nach Division durch $a^2$ ergibt sich
%\begin{align*}
%\operatorname{dn}(u,k)^2
%-
%k^2\operatorname{cn}(u,k)^2
%&=
%\frac{1}{a^2}
%=
%\frac{a^2-a^2+1}{a^2}
%=
%1-k^2 =: k^{\prime 2}.
%\end{align*}
%Wir stellen die hiermit gefundenen Relationen zwischen den grundlegenden
%Jacobischen elliptischen Funktionen für später zusammen in den Formeln
%\begin{equation}
%\begin{aligned}
%\operatorname{sn}^2(u,k)
%+
%\operatorname{cn}^2(u,k)
%&=
%1
%\\
%\operatorname{dn}^2(u,k) + k^2\operatorname{sn}^2(u,k)
%&=
%1
%\\
%\operatorname{dn}^2(u,k)  -k^2\operatorname{cn}^2(u,k)
%&=
%k^{\prime 2}.
%\end{aligned}
%\label{buch:elliptisch:eqn:jacobi-relationen}
%\end{equation}
%zusammen.
%So wie es möglich ist, $\sin\alpha$ durch $\cos\alpha$ auszudrücken,
%ist es mit
%\eqref{buch:elliptisch:eqn:jacobi-relationen}
%jetzt auch möglich jede grundlegende elliptische Funktion durch
%jede anderen auszudrücken.
%Die Resultate sind in der Tabelle~\ref{buch:elliptisch:fig:jacobi-relationen}
%zusammengestellt.
%
%\begin{table}
%\centering
%\renewcommand{\arraystretch}{2.1}
%\begin{tabular}{|>{$\displaystyle}c<{$}|>{$\displaystyle}c<{$}>{$\displaystyle}c<{$}>{$\displaystyle}c<{$}|}
%\hline
%&\operatorname{sn}(u,k)
%&\operatorname{cn}(u,k)
%&\operatorname{dn}(u,k)\\
%\hline
%\operatorname{sn}(u,k)
%&\operatorname{sn}(u,k)
%&\sqrt{1-\operatorname{cn}^2(u,k)}
%&\frac1k\sqrt{1-\operatorname{dn}^2(u,k)}
%\\
%\operatorname{cn}(u,k)
%&\sqrt{1-\operatorname{sn}^2(u,k)}
%&\operatorname{cn}(u,k)
%&\frac{1}{k}\sqrt{\operatorname{dn}^2(u,k)-k^{\prime2}}
%\\
%\operatorname{dn}(u,k)
%&\sqrt{1-k^2\operatorname{sn}^2(u,k)}
%&\sqrt{k^{\prime2}+k^2\operatorname{cn}^2(u,k)}
%&\operatorname{dn}(u,k)
%\\
%\hline
%\end{tabular}
%\caption{Jede der Jacobischen elliptischen Funktionen lässt sich
%unter Verwendung der Relationen~\eqref{buch:elliptisch:eqn:jacobi-relationen}
%durch jede andere ausdrücken.
%\label{buch:elliptisch:fig:jacobi-relationen}}
%\end{table}
%
%%
%% Ableitungen der Jacobi-ellpitischen Funktionen
%% 
%\subsubsection{Ableitung}
%Die trigonometrischen Funktionen sind deshalb so besonders nützlich 
%für die Lösung von Schwingungsdifferentialgleichungen, weil sie die
%Beziehungen
%\[
%\frac{d}{d\varphi}  \cos\varphi = -\sin\varphi
%\qquad\text{und}\qquad
%\frac{d}{d\varphi}  \sin\varphi = \cos\varphi
%\]
%erfüllen.
%So einfach können die Beziehungen natürlich nicht sein, sonst würde sich
%durch Integration ja wieder nur die trigonometrischen Funktionen ergeben.
%Durch geschickte Wahl des Arguments $u$ kann man aber erreichen, dass
%sie ähnlich nützliche Beziehungen zwischen den Ableitungen ergeben.
%
%Gesucht ist jetzt also eine Wahl für das Argument $u$ zum Beispiel in
%Abhängigkeit von $\varphi$, dass sich einfache und nützliche
%Ableitungsformeln ergeben.
%Wir setzen daher $u(\varphi)$ voraus und beachten, dass $x$ und $y$
%ebenfalls von $\varphi$ abhängen, es ist
%$y=\sin\varphi$ und $x=a\cos\varphi$.
%Die Ableitungen von $x$ und $y$ nach $\varphi$ sind
%\begin{align*}
%\frac{dy}{d\varphi}
%&=
%\cos\varphi
%=
%\frac{1}{a} x
%=
%\operatorname{cn}(u,k)
%\\
%\frac{dx}{d\varphi}
%&=
%-a\sin\varphi
%=
%-a y
%=
%-a\operatorname{sn}(u,k).
%\end{align*}
%Daraus kann man jetzt die folgenden Ausdrücke für die Ableitungen der
%elliptischen Funktionen nach $\varphi$ ableiten:
%\begin{align*}
%\frac{d}{d\varphi} \operatorname{sn}(u,z)
%&=
%\frac{d}{d\varphi} y(\varphi)
%=
%\cos\varphi
%=
%\frac{x}{a}
%=
%\operatorname{cn}(u,k)
%&&\Rightarrow&
%\frac{d}{du}
%\operatorname{sn}(u,k)
%&=
%\operatorname{cn}(u,k) \frac{d\varphi}{du}
%\\
%\frac{d}{d\varphi} \operatorname{cn}(u,z)
%&=
%\frac{d}{d\varphi} \frac{x(\varphi)}{a}
%=
%-\sin\varphi
%=
%-\operatorname{sn}(u,k)
%&&\Rightarrow&
%\frac{d}{du}\operatorname{cn}(u,k)
%&=
%-\operatorname{sn}(u,k) \frac{d\varphi}{du}
%\\
%\frac{d}{d\varphi} \operatorname{dn}(u,z)
%&=
%\frac{1}{a}\frac{dr}{d\varphi}
%=
%\frac{1}{a}\frac{d\sqrt{x^2+y^2}}{d\varphi}
%%\\
%%&
%\rlap{$\displaystyle\mathstrut
%=
%\frac{x}{ar} \frac{dx}{d\varphi}
%+
%\frac{y}{ar} \frac{dy}{d\varphi}
%%\\
%%&
%=
%\frac{x}{ar} (-a\operatorname{sn}(u,k))
%+
%\frac{y}{ar} \operatorname{cn}(u,k)
%$}
%\\
%&
%\rlap{$\displaystyle\mathstrut
%=
%\frac{x}{ar}(-ay)
%+
%\frac{y}{ar} \frac{x}{a}
%%\rlap{$\displaystyle
%=
%\frac{xy(-1+\frac{1}{a^2})}{r} 
%%$}
%%\\
%%&
%=
%-\frac{xy(a^2-1)}{a^2r} 
%$}
%\\
%&=
%-\frac{a^2-1}{ar}
%\operatorname{cn}(u,k) \operatorname{sn}(u,k)
%%\\
%%&
%\rlap{$\displaystyle\mathstrut
%=
%-k^2
%\frac{a}{r}
%\operatorname{cn}(u,k) \operatorname{sn}(u,k)
%$}
%\\
%&=
%-k^2\frac{\operatorname{cn}(u,k)\operatorname{sn}(u,k)}{\operatorname{dn}(u,k)}
%&&\Rightarrow&
%\frac{d}{du} \operatorname{dn}(u,k)
%&=
%-k^2\frac{\operatorname{cn}(u,k)
%\operatorname{sn}(u,k)}{\operatorname{dn}(u,k)}
%\frac{d\varphi}{du}.
%\end{align*}
%Die einfachsten Beziehungen ergeben sich offenbar, wenn man $u$ so
%wählt, dass
%\[
%\frac{d\varphi}{du}
%=
%\operatorname{dn}(u,k)
%=
%\frac{r}{a}.
%\]
%Damit haben wir die grundlegenden Ableitungsregeln
%
%\begin{satz}
%\label{buch:elliptisch:satz:ableitungen}
%Die Jacobischen elliptischen Funktionen haben die Ableitungen
%\begin{equation}
%\begin{aligned}
%\frac{d}{du}\operatorname{sn}(u,k)
%&=
%\phantom{-}\operatorname{cn}(u,k)\operatorname{dn}(u,k)
%\\
%\frac{d}{du}\operatorname{cn}(u,k)
%&=
%-\operatorname{sn}(u,k)\operatorname{dn}(u,k)
%\\
%\frac{d}{du}\operatorname{dn}(u,k)
%&=
%-k^2\operatorname{sn}(u,k)\operatorname{cn}(u,k).
%\end{aligned}
%\label{buch:elliptisch:eqn:ableitungsregeln}
%\end{equation}
%\end{satz}
%
%%
%% Der Grenzfall $k=1$
%%
%\subsubsection{Der Grenzwert $k\to1$}
%\begin{figure}
%\centering
%\includegraphics{chapters/110-elliptisch/images/sncnlimit.pdf}
%\caption{Grenzfälle der Jacobischen elliptischen Funktionen 
%für die Werte $0$ und $1$ des Parameters $k$.
%\label{buch:elliptisch:fig:sncnlimit}}
%\end{figure}
%Für $k=1$ ist $k^{\prime2}=1-k^2=$ und es folgt aus den
%Relationen~\eqref{buch:elliptisch:eqn:jacobi-relationen}
%\[
%\operatorname{cn}^2(u,k)
%-
%k^2
%\operatorname{dn}^2(u,k)
%=
%k^{\prime2}
%=
%0
%\qquad\Rightarrow\qquad
%\operatorname{cn}^2(u,1)
%=
%\operatorname{dn}^2(u,1),
%\]
%die beiden Funktionen
%$\operatorname{cn}(u,k)$
%und
%$\operatorname{dn}(u,k)$
%fallen also zusammen.
%Die Ableitungsregeln werden dadurch vereinfacht:
%\begin{align*}
%\operatorname{sn}'(u,1)
%&=
%\operatorname{cn}(u,1)
%\operatorname{dn}(u,1)
%=
%\operatorname{cn}^2(u,1)
%=
%1-\operatorname{sn}^2(u,1)
%&&\Rightarrow& y'&=1-y^2
%\\
%\operatorname{cn}'(u,1)
%&=
%-
%\operatorname{sn}(u,1)
%\operatorname{dn}(u,1)
%=
%-
%\operatorname{sn}(u,1)\operatorname{cn}(u,1)
%&&\Rightarrow&
%\frac{z'}{z}&=(\log z)' = -y
%\end{align*}
%Die erste Differentialgleichung für $y$ lässt sich separieren, man findet
%die Lösung
%\[
%\frac{y'}{1-y^2}
%=
%1
%\quad\Rightarrow\quad
%\int \frac{dy}{1-y^2} = \int \,du
%\quad\Rightarrow\quad
%\operatorname{artanh}(y) = u
%\quad\Rightarrow\quad
%\operatorname{sn}(u,1)=\tanh u.
%\]
%Damit kann man jetzt auch $z$ berechnen:
%\begin{align*}
%(\log \operatorname{cn}(u,1))'
%&=
%\tanh u
%&&\Rightarrow&
%\log\operatorname{cn}(u,1)
%&=
%-\int\tanh u\,du
%=
%-\log\cosh u
%\\
%&
%&&\Rightarrow&
%\operatorname{cn}(u,1)
%&=
%\frac{1}{\cosh u}
%=
%\operatorname{sech}u.
%\end{align*}
%Die Grenzfunktionen sind in Abbildung~\ref{buch:elliptisch:fig:sncnlimit}
%dargestellt.
%
%%
%% Das Argument u
%%
%\subsubsection{Das Argument $u$}
%Die Gleichung 
%\begin{equation}
%\frac{d\varphi}{du}
%=
%\operatorname{dn}(u,k)
%\label{buch:elliptisch:eqn:uableitung}
%\end{equation}
%ermöglicht, $\varphi$ in Abhängigkeit von $u$ zu berechnen, ohne jedoch
%die geometrische Bedeutung zu klären.
%Das beginnt bereits damit, dass der Winkel $\varphi$ nicht nicht der
%Polarwinkel des Punktes $P$ in Abbildung~\ref{buch:elliptisch:fig:jacobidef}
%ist, diesen nennen wir $\vartheta$.
%Der Zusammenhang zwischen $\varphi$ und $\vartheta$ ist
%\begin{equation}
%\frac1{a}\tan\varphi = \tan\vartheta
%\label{buch:elliptisch:eqn:phitheta}
%\end{equation}
%
%Um die geometrische Bedeutung besser zu verstehen, nehmen wir jetzt an,
%dass die Ellipse mit einem Parameter $t$ parametrisiert ist, dass also
%$\varphi(t)$, $\vartheta(t)$ und $u(t)$ Funktionen von $t$ sind.
%Die Ableitung von~\eqref{buch:elliptisch:eqn:phitheta} ist
%\[
%\frac1{a}\cdot \frac{1}{\cos^2\varphi}\cdot \dot{\varphi}
%=
%\frac{1}{\cos^2\vartheta}\cdot \dot{\vartheta}.
%\]
%Daraus kann die Ableitung von $\vartheta$ nach $\varphi$ bestimmt
%werden, sie ist
%\[
%\frac{d\vartheta}{d\varphi}
%=
%\frac{\dot{\vartheta}}{\dot{\varphi}}
%=
%\frac{1}{a}
%\cdot
%\frac{\cos^2\vartheta}{\cos^2\varphi}
%=
%\frac{1}{a}
%\cdot
%\frac{(x/r)^2}{(x/a)^2}
%=
%\frac{1}{a}\cdot
%\frac{a^2}{r^2}
%=
%\frac{1}{a}\cdot\frac{1}{\operatorname{dn}^2(u,k)}.
%\]
%Damit kann man jetzt mit Hilfe von~\eqref{buch:elliptisch:eqn:uableitung} 
%Die Ableitung von $\vartheta$ nach $u$ ermitteln, sie ist
%\[
%\frac{d\vartheta}{du}
%=
%\frac{d\vartheta}{d\varphi}
%\cdot
%\frac{d\varphi}{du}
%=
%\frac{1}{a}\cdot\frac{1}{\operatorname{dn}^2(u,k)}
%\cdot
%\operatorname{dn}(u,k)
%=
%\frac{1}{a}
%\cdot
%\frac{1}{\operatorname{dn}(u,k)}
%=
%\frac{1}{a}
%\cdot\frac{a}{r}
%=
%\frac{1}{r},
%\]
%wobei wir auch die Definition der Funktion $\operatorname{dn}(u,k)$
%verwendet haben.
%
%In der Parametrisierung mit dem Parameter $t$ kann man jetzt die Ableitung
%von $u$ nach $t$ berechnen als
%\[
%\frac{du}{dt}
%=
%\frac{du}{d\vartheta}
%\frac{d\vartheta}{dt}
%=
%r
%\dot{\vartheta}.
%\]
%Darin ist $\dot{\vartheta}$ die Winkelgeschwindigkeit des Punktes um
%das Zentrum $O$ und $r$ ist die aktuelle Entfernung des Punktes $P$
%von $O$.
%$r\dot{\vartheta}$ ist also die Geschwindigkeitskomponenten des Punktes
%$P$ senkrecht auf den aktuellen Radiusvektor.
%Der Parameter $u$, der zum Punkt $P$ gehört, ist also das Integral
%\[
%u(P) = \int_0^P r\,d\vartheta.
%\]
%Für einen Kreis ist die Geschwindigkeit von $P$ immer senkrecht
%auf dem Radiusvektor und der Radius ist konstant, so dass
%$u(P)=\vartheta(P)$ ist.
%
%%
%% Die abgeleiteten elliptischen Funktionen
%%
%\begin{figure}
%\centering
%\includegraphics[width=\textwidth]{chapters/110-elliptisch/images/jacobi12.pdf}
%\caption{Die Verhältnisse der Funktionen
%$\operatorname{sn}(u,k)$,
%$\operatorname{cn}(u,k)$
%udn
%$\operatorname{dn}(u,k)$
%geben Anlass zu neun weitere Funktionen, die sich mit Hilfe
%des Strahlensatzes geometrisch interpretieren lassen.
%\label{buch:elliptisch:fig:jacobi12}}
%\end{figure}
%\begin{table}
%\centering
%\renewcommand{\arraystretch}{2.5}
%\begin{tabular}{|>{$\displaystyle}c<{$}|>{$\displaystyle}c<{$}>{$\displaystyle}c<{$}>{$\displaystyle}c<{$}>{$\displaystyle}c<{$}|}
%\hline
%\cdot &
%\frac{1}{1} &
%\frac{1}{\operatorname{sn}(u,k)} &
%\frac{1}{\operatorname{cn}(u,k)} &
%\frac{1}{\operatorname{dn}(u,k)} 
%\\[5pt]
%\hline
%1&
%&%\operatorname{nn}(u,k)=\frac{1}{1} &
%\operatorname{ns}(u,k)=\frac{1}{\operatorname{sn}(u,k)} &
%\operatorname{nc}(u,k)=\frac{1}{\operatorname{cn}(u,k)} &
%\operatorname{nd}(u,k)=\frac{1}{\operatorname{dn}(u,k)}
%\\
%\operatorname{sn}(u,k) &
%\operatorname{sn}(u,k)=\frac{\operatorname{sn}(u,k)}{1}&
%&%\operatorname{ss}(u,k)=\frac{\operatorname{sn}(u,k)}{\operatorname{sn}(u,k)}&
%\operatorname{sc}(u,k)=\frac{\operatorname{sn}(u,k)}{\operatorname{cn}(u,k)}&
%\operatorname{sd}(u,k)=\frac{\operatorname{sn}(u,k)}{\operatorname{dn}(u,k)}
%\\
%\operatorname{cn}(u,k) &
%\operatorname{cn}(u,k)=\frac{\operatorname{cn}(u,k)}{1} &
%\operatorname{cs}(u,k)=\frac{\operatorname{cn}(u,k)}{\operatorname{sn}(u,k)}&
%&%\operatorname{cc}(u,k)=\frac{\operatorname{cn}(u,k)}{\operatorname{cn}(u,k)}&
%\operatorname{cd}(u,k)=\frac{\operatorname{cn}(u,k)}{\operatorname{dn}(u,k)}
%\\
%\operatorname{dn}(u,k) &
%\operatorname{dn}(u,k)=\frac{\operatorname{dn}(u,k)}{1} &
%\operatorname{ds}(u,k)=\frac{\operatorname{dn}(u,k)}{\operatorname{sn}(u,k)}&
%\operatorname{dc}(u,k)=\frac{\operatorname{dn}(u,k)}{\operatorname{cn}(u,k)}&
%%\operatorname{dd}(u,k)=\frac{\operatorname{dn}(u,k)}{\operatorname{dn}(u,k)}
%\\[5pt]
%\hline
%\end{tabular}
%\caption{Zusammenstellung der abgeleiteten Jacobischen elliptischen
%Funktionen in hinteren drei Spalten als Quotienten der grundlegenden
%Jacobischen elliptischen Funktionen.
%Die erste Spalte zum Nenner $1$ enthält die grundlegenden
%Jacobischen elliptischen Funktionen.
%\label{buch:elliptisch:table:abgeleitetjacobi}}
%\end{table}
%\subsubsection{Die abgeleiteten elliptischen Funktionen}
%Zusätzlich zu den grundlegenden Jacobischen elliptischen Funktioenn
%lassen sich weitere elliptische Funktionen bilden, die unglücklicherweise
%die {\em abgeleiteten elliptischen Funktionen} genannt werden.
%Ähnlich wie die trigonometrischen Funktionen $\tan\alpha$, $\cot\alpha$,
%$\sec\alpha$ und $\csc\alpha$ als Quotienten von $\sin\alpha$ und
%$\cos\alpha$ definiert sind, sind die abgeleiteten elliptischen Funktionen
%die in Tabelle~\ref{buch:elliptisch:table:abgeleitetjacobi} zusammengestellten
%Quotienten der grundlegenden Jacobischen elliptischen Funktionen.
%Die Bezeichnungskonvention ist, dass die Funktion $\operatorname{pq}(u,k)$
%ein Quotient ist, dessen Zähler durch den Buchstaben p bestimmt ist,
%der Nenner durch den Buchstaben q.
%Der Buchstabe n steht für eine $1$, die Buchstaben s, c und d stehen für
%die Anfangsbuchstaben der grundlegenden Jacobischen elliptischen
%Funktionen.
%Meint man irgend eine der Jacobischen elliptischen Funktionen, schreibt
%man manchmal auch $\operatorname{zn}(u,k)$.
%
%In Abbildung~\ref{buch:elliptisch:fig:jacobi12} sind die Quotienten auch
%geometrisch interpretiert.
%Der Wert der Funktion $\operatorname{nq}(u,k)$ ist die auf dem Strahl
%mit Polarwinkel $\varphi$ abgetragene Länge bis zu den vertikalen
%Geraden, die den verschiedenen möglichen Nennern entsprechen.
%Entsprechend ist der Wert der Funktion $\operatorname{dq}(u,k)$ die
%Länge auf dem Strahl mit Polarwinkel $\vartheta$.
%
%Die Relationen~\ref{buch:elliptisch:eqn:jacobi-relationen}
%ermöglichen, jede Funktion $\operatorname{zn}(u,k)$ durch jede
%andere auszudrücken.
%Die schiere Anzahl solcher Beziehungen macht es unmöglich, sie 
%übersichtlich in einer Tabelle zusammenzustellen, daher soll hier
%nur an einem Beispiel das Vorgehen gezeigt werden:
%
%\begin{beispiel}
%Die Funktion $\operatorname{sc}(u,k)$ soll durch $\operatorname{cd}(u,k)$
%ausgedrückt werden.
%Zunächst ist 
%\[
%\operatorname{sc}(u,k)
%=
%\frac{\operatorname{sn}(u,k)}{\operatorname{cn}(u,k)}
%\]
%nach Definition.
%Im Resultat sollen nur noch $\operatorname{cn}(u,k)$ und
%$\operatorname{dn}(u,k)$ vorkommen.
%Daher eliminieren wir zunächst die Funktion $\operatorname{sn}(u,k)$
%mit Hilfe von \eqref{buch:elliptisch:eqn:jacobi-relationen} und erhalten
%\begin{equation}
%\operatorname{sc}(u,k)
%=
%\frac{\sqrt{1-\operatorname{cn}^2(u,k)}}{\operatorname{cn}(u,k)}.
%\label{buch:elliptisch:eqn:allgausdruecken}
%\end{equation}
%Nun genügt es, die Funktion $\operatorname{cn}(u,k)$ durch
%$\operatorname{cd}(u,k)$ auszudrücken.
%Aus der Definition und der
%dritten Relation in \eqref{buch:elliptisch:eqn:jacobi-relationen} 
%erhält man
%\begin{align*}
%\operatorname{cd}^2(u,k)
%&=
%\frac{\operatorname{cn}^2(u,k)}{\operatorname{dn}^2(u,k)}
%=
%\frac{\operatorname{cn}^2(u,k)}{k^{\prime2}+k^2\operatorname{cn}^2(u,k)}
%\\
%\Rightarrow
%\qquad
%k^{\prime 2}
%\operatorname{cd}^2(u,k)
%+
%k^2\operatorname{cd}^2(u,k)\operatorname{cn}^2(u,k)
%&=
%\operatorname{cn}^2(u,k)
%\\
%\operatorname{cn}^2(u,k)
%-
%k^2\operatorname{cd}^2(u,k)\operatorname{cn}^2(u,k)
%&=
%k^{\prime 2}
%\operatorname{cd}^2(u,k)
%\\
%\operatorname{cn}^2(u,k)
%&=
%\frac{
%k^{\prime 2}
%\operatorname{cd}^2(u,k)
%}{
%1 - k^2\operatorname{cd}^2(u,k)
%}
%\end{align*}
%Für den Zähler brauchen wir $1-\operatorname{cn}^2(u,k)$, also
%\[
%1-\operatorname{cn}^2(u,k)
%=
%\frac{
%1
%-
%k^2\operatorname{cd}^2(u,k)
%-
%k^{\prime 2}
%\operatorname{cd}^2(u,k)
%}{
%1
%-
%k^2\operatorname{cd}^2(u,k)
%}
%=
%\frac{1-\operatorname{cd}^2(u,k)}{1-k^2\operatorname{cd}^2(u,k)}
%\]
%Einsetzen in~\eqref{buch:elliptisch:eqn:allgausdruecken} gibt
%\begin{align*}
%\operatorname{sc}(u,k)
%&=
%\frac{
%\sqrt{1-\operatorname{cd}^2(u,k)}
%}{\sqrt{1-k^2\operatorname{cd}^2(u,k)}}
%\cdot
%\frac{
%\sqrt{1 - k^2\operatorname{cd}^2(u,k)}
%}{
%k'
%\operatorname{cd}(u,k)
%}
%=
%\frac{
%\sqrt{1-\operatorname{cd}^2(u,k)}
%}{
%k'
%\operatorname{cd}(u,k)
%}.
%\qedhere
%\end{align*}
%\end{beispiel}
%
%\subsubsection{Ableitung der abgeleiteten elliptischen Funktionen}
%Aus den Ableitungen der grundlegenden Jacobischen elliptischen Funktionen
%können mit der Quotientenregel nun auch beliebige Ableitungen der
%abgeleiteten Jacobischen elliptischen Funktionen gefunden werden.
%Als Beispiel berechnen wir die Ableitung von $\operatorname{sc}(u,k)$.
%Sie ist
%\begin{align*}
%\frac{d}{du}
%\operatorname{sc}(u,k)
%&=
%\frac{d}{du}
%\frac{\operatorname{sn}(u,k)}{\operatorname{cn}(u,k)}
%=
%\frac{
%\operatorname{sn}'(u,k)\operatorname{cn}(u,k)
%-
%\operatorname{sn}(u,k)\operatorname{cn}'(u,k)}{
%\operatorname{cn}^2(u,k)
%}
%\\
%&=
%\frac{
%\operatorname{cn}^2(u,k)\operatorname{dn}(u,k)
%+
%\operatorname{sn}^2(u,k)\operatorname{dn}(u,k)
%}{
%\operatorname{cn}^2(u,k)
%}
%=
%\frac{(
%\operatorname{sn}^2(u,k)
%+
%\operatorname{cn}^2(u,k)
%)\operatorname{dn}(u,k)}{
%\operatorname{cn}^2(u,k)
%}
%\\
%&=
%\frac{1}{\operatorname{cn}(u,k)}
%\cdot
%\frac{\operatorname{dn}(u,k)}{\operatorname{cn}(u,k)}
%=
%\operatorname{nc}(u,k)
%\operatorname{dc}(u,k).
%\end{align*}
%Man beachte, dass das Quadrat der Nennerfunktion im Resultat
%der Quotientenregel zur Folge hat, dass die
%beiden Funktionen im Resultat beide den gleichen Nenner haben wie
%die Funktion, die abgeleitet wird.
%
%Mit etwas Fleiss kann man nach diesem Muster alle Ableitungen
%\begin{equation}
%%\small
%\begin{aligned}
%\operatorname{sn}'(u,k)
%&= 
%\phantom{-}
%\operatorname{cn}(u,k)\,\operatorname{dn}(u,k)
%&&\qquad&
%\operatorname{ns}'(u,k)
%&=
%-
%\operatorname{cs}(u,k)\,\operatorname{ds}(u,k)
%\\
%\operatorname{cn}'(u,k)
%&= 
%-
%\operatorname{sn}(u,k)\,\operatorname{dn}(u,k)
%&&&
%\operatorname{nc}'(u,k)
%&=
%\phantom{-}
%\operatorname{sc}(u,k)\,\operatorname{dc}(u,k)
%\\
%\operatorname{dn}'(u,k)
%&= 
%-k^2
%\operatorname{sn}(u,k)\,\operatorname{cn}(u,k)
%&&&
%\operatorname{nd}'(u,k)
%&=
%\phantom{-}
%k^2
%\operatorname{sd}(u,k)\,\operatorname{cd}(u,k)
%\\
%\operatorname{sc}'(u,k)
%&=
%\phantom{-}
%\operatorname{dc}(u,k)\,\operatorname{nc}(u,k)
%&&&
%\operatorname{cs}'(u,k)
%&=
%-
%\operatorname{ds}(u,k)\,\operatorname{ns}(u,k)
%\\
%\operatorname{cd}'(u,k)
%&=
%-k^{\prime2}
%\operatorname{sd}(u,k)\,\operatorname{nd}(u,k)
%&&&
%\operatorname{dc}'(u,k)
%&=
%\phantom{-}
%k^{\prime2}
%\operatorname{dc}(u,k)\,\operatorname{nc}(u,k)
%\\
%\operatorname{ds}'(d,k)
%&=
%-
%\operatorname{cs}(u,k)\,\operatorname{ns}(u,k)
%&&&
%\operatorname{sd}'(d,k)
%&=
%\phantom{-}
%\operatorname{cd}(u,k)\,\operatorname{nd}(u,k)
%\end{aligned}
%\label{buch:elliptisch:eqn:alleableitungen}
%\end{equation}
%finden.
%Man beachte, dass in jeder Identität alle Funktionen den gleichen
%zweiten Buchstaben haben.
%
%\subsubsection{TODO}
%XXX algebraische Beziehungen \\
%XXX Additionstheoreme \\
%XXX Perioden
%% use https://math.stackexchange.com/questions/3013692/how-to-show-that-jacobi-sine-function-is-doubly-periodic
%
%
%XXX Ableitungen \\
%XXX Werte \\

%%
%% Lösung von Differentialgleichungen
%%
%\subsection{Lösungen von Differentialgleichungen
%\label{buch:elliptisch:subsection:differentialgleichungen}}
%Die elliptischen Funktionen ermöglichen die Lösung gewisser nichtlinearer
%Differentialgleichungen in geschlossener Form.
%Ziel dieses Abschnitts ist, Differentialgleichungen der Form
%\(
%\dot{x}(t)^2
%=
%P(x(t))
%\)
%mit einem Polynom $P$ vierten Grades oder
%\(
%\ddot{x}(t)
%=
%p(x(t))
%\)
%mit einem Polynom dritten Grades als rechter Seite lösen zu können.
%
%%
%% Die Differentialgleichung der elliptischen Funktionen
%%
%\subsubsection{Die Differentialgleichungen der elliptischen Funktionen}
%Um Differentialgleichungen mit elliptischen Funktion lösen zu
%können, muss man als erstes die Differentialgleichungen derselben
%finden.
%Quadriert man die Ableitungsregel für $\operatorname{sn}(u,k)$, erhält
%man
%\[
%\biggl(\frac{d}{du}\operatorname{sn}(u,k)\biggr)^2
%=
%\operatorname{cn}(u,k)^2 \operatorname{dn}(u,k)^2.
%\]
%Die Funktionen auf der rechten Seite können durch $\operatorname{sn}(u,k)$
%ausgedrückt werden, dies führt auf die Differentialgleichung
%\begin{align*}
%\biggl(\frac{d}{du}\operatorname{sn}(u,k)\biggr)^2
%&=
%\bigl(
%1-\operatorname{sn}(u,k)^2
%\bigr)
%\bigl(
%1-k^2 \operatorname{sn}(u,k)^2
%\bigr)
%\\
%&=
%k^2\operatorname{sn}(u,k)^4 
%-(1+k^2)
%\operatorname{sn}(u,k)^2 
%+1.
%\end{align*}
%Für die Funktion $\operatorname{cn}(u,k)$ ergibt die analoge Rechnung
%\begin{align*}
%\frac{d}{du}\operatorname{cn}(u,k)
%&=
%-\operatorname{sn}(u,k) \operatorname{dn}(u,k)
%\\
%\biggl(\frac{d}{du}\operatorname{cn}(u,k)\biggr)^2
%&=
%\operatorname{sn}(u,k)^2 \operatorname{dn}(u,k)^2
%\\
%&=
%\bigl(1-\operatorname{cn}(u,k)^2\bigr)
%\bigl(k^{\prime 2}+k^2 \operatorname{cn}(u,k)^2\bigr)
%\\
%&=
%-k^2\operatorname{cn}(u,k)^4
%+
%(k^2-k^{\prime 2})\operatorname{cn}(u,k)^2
%+
%k^{\prime 2}
%\intertext{und weiter für $\operatorname{dn}(u,k)$:}
%\frac{d}{du}\operatorname{dn}(u,k)
%&=
%-k^2\operatorname{sn}(u,k)\operatorname{cn}(u,k)
%\\
%\biggl(
%\frac{d}{du}\operatorname{dn}(u,k)
%\biggr)^2
%&=
%\bigl(k^2 \operatorname{sn}(u,k)^2\bigr)
%\bigl(k^2 \operatorname{cn}(u,k)^2\bigr)
%\\
%&=
%\bigl(
%1-\operatorname{dn}(u,k)^2
%\bigr)
%\bigl(
%\operatorname{dn}(u,k)^2-k^{\prime 2}
%\bigr)
%\\
%&=
%-\operatorname{dn}(u,k)^4
%+
%(1+k^{\prime 2})\operatorname{dn}(u,k)^2
%-k^{\prime 2}.
%\end{align*}
%
%\begin{table}
%\centering
%\renewcommand{\arraystretch}{1.7}
%\begin{tabular}{|>{$}l<{$}|>{$}l<{$}|>{$}c<{$}|>{$}c<{$}|>{$}c<{$}|}
%\hline
%\text{Funktion $y=$}&\text{Differentialgleichung}&\alpha&\beta&\gamma\\
%\hline
%\operatorname{sn}(u,k)
%	& y'^2 = \phantom{-}(1-y^2)(1-k^2y^2)
%		&k^2&1+k^2&1
%\\
%\operatorname{cn}(u,k) &y'^2 = \phantom{-}(1-y^2)(k^{\prime2}+k^2y^2)
%		&-k^2	&k^2-k^{\prime 2}=2k^2-1&k^{\prime2}
%\\
%\operatorname{dn}(u,k)
%	& y'^2 = -(1-y^2)(k^{\prime 2}-y^2)
%		&-1	&1+k^{\prime 2}=2-k^2	&-k^{\prime2}
%\\
%\hline
%\end{tabular}
%\caption{Elliptische Funktionen als Lösungsfunktionen für verschiedene
%nichtlineare Differentialgleichungen der Art
%\eqref{buch:elliptisch:eqn:1storderdglell}.
%Die Vorzeichen der Koeffizienten $\alpha$, $\beta$ und $\gamma$
%entscheidet darüber, welche Funktion für die Lösung verwendet werden
%muss.
%\label{buch:elliptisch:tabelle:loesungsfunktionen}}
%\end{table}
%
%Die drei grundlegenden Jacobischen elliptischen Funktionen genügen also alle
%einer nichtlinearen Differentialgleichung erster Ordnung der selben Art.
%Das Quadrat der Ableitung ist ein Polynom vierten Grades der Funktion.
%Die Differentialgleichungen sind in der
%Tabelle~\ref{buch:elliptisch:tabelle:loesungsfunktionen} zusammengefasst.
%
%%
%% Differentialgleichung der abgeleiteten elliptischen Funktionen
%%
%\subsubsection{Die Differentialgleichung der abgeleiteten elliptischen
%Funktionen}
%Da auch die Ableitungen der abgeleiteten Jacobischen elliptischen
%Funktionen Produkte von genau zwei Funktionen sind, die sich wieder
%durch die ursprüngliche Funktion ausdrücken lassen, darf man erwarten,
%dass alle elliptischen Funktionen einer ähnlichen Differentialgleichung
%genügen.
%Um dies besser einzufangen, schreiben wir $\operatorname{pq}(u,k)$,
%wenn wir eine beliebige abgeleitete Jacobische elliptische Funktion.
%Für 
%$\operatorname{pq}=\operatorname{sn}$
%$\operatorname{pq}=\operatorname{cn}$
%und
%$\operatorname{pq}=\operatorname{dn}$
%wissen wir bereits und erwarten für jede andere Funktion dass
%$\operatorname{pq}(u,k)$ auch, dass sie Lösung einer Differentialgleichung
%der Form
%\begin{equation}
%\operatorname{pq}'(u,k)^2
%=
%\alpha \operatorname{pq}(u,k)^4 + \beta \operatorname{pq}(u,k)^2 + \gamma
%\label{buch:elliptisch:eqn:1storderdglell}
%\end{equation}
%erfüllt,
%wobei wir mit $\operatorname{pq}'(u,k)$ die Ableitung von
%$\operatorname{pq}(u,k)$ nach dem ersten Argument meinen.
%Die Koeffizienten $\alpha$, $\beta$ und $\gamma$ hängen von $k$ ab,
%ihre Werte für die grundlegenden Jacobischen elliptischen
%sind in Tabelle~\ref{buch:elliptisch:table:differentialgleichungen}
%zusammengestellt.
%
%Die Koeffizienten müssen nicht für jede Funktion wieder neu bestimmt
%werden, denn für den Kehrwert einer Funktion lässt sich die
%Differentialgleichung aus der Differentialgleichung der ursprünglichen
%Funktion ermitteln.
%
%%
%% Differentialgleichung der Kehrwertfunktion
%%
%\subsubsection{Differentialgleichung für den Kehrwert einer elliptischen Funktion}
%Aus der Differentialgleichung~\eqref{buch:elliptisch:eqn:1storderdglell}
%für die Funktion $\operatorname{pq}(u,k)$ kann auch eine
%Differentialgleichung für den Kehrwert
%$\operatorname{qp}(u,k)=\operatorname{pq}(u,k)^{-1}$
%ableiten.
%Dazu rechnet man
%\[
%\operatorname{qp}'(u,k)
%=
%\frac{d}{du}\frac{1}{\operatorname{pq}(u,k)}
%=
%\frac{\operatorname{pq}'(u,k)}{\operatorname{pq}(u,k)^2}
%\qquad\Rightarrow\qquad
%\left\{
%\quad
%\begin{aligned}
%\operatorname{pq}(u,k)
%&=
%\frac{1}{\operatorname{qp}(u,k)}
%\\
%\operatorname{pq}'(u,k)
%&=
%\frac{\operatorname{qp}'(u,k)}{\operatorname{qp}(u,k)^2}
%\end{aligned}
%\right.
%\]
%und setzt in die Differentialgleichung ein:
%\begin{align*}
%\biggl(
%\frac{
%\operatorname{qp}'(u,k)
%}{
%\operatorname{qp}(u,k)
%}
%\biggr)^2
%&=
%\alpha \frac{1}{\operatorname{qp}(u,k)^4}
%+
%\beta \frac{1}{\operatorname{qp}(u,k)^2}
%+
%\gamma.
%\end{align*}
%Nach Multiplikation mit $\operatorname{qp}(u,k)^4$ erhält man den
%folgenden Satz.
%
%\begin{satz}
%Wenn die Jacobische elliptische Funktion $\operatorname{pq}(u,k)$
%der Differentialgleichung genügt, dann genügt der Kehrwert
%$\operatorname{qp}(u,k) = 1/\operatorname{pq}(u,k)$ der Differentialgleichung
%\begin{equation}
%(\operatorname{qp}'(u,k))^2
%= 
%\gamma \operatorname{qp}(u,k)^4
%+
%\beta \operatorname{qp}(u,k)^2
%+
%\alpha
%\label{buch:elliptisch:eqn:kehrwertdgl}
%\end{equation}
%\end{satz}
%
%\begin{table}
%\centering
%\def\lfn#1{\multicolumn{1}{|l|}{#1}}
%\def\rfn#1{\multicolumn{1}{r|}{#1}}
%\renewcommand{\arraystretch}{1.3}
%\begin{tabular}{l|>{$}c<{$}>{$}c<{$}>{$}c<{$}|r}
%\cline{1-4}
%\lfn{Funktion}
%         &  \alpha   & \beta     &    \gamma  &\\
%\hline
%\lfn{sn}&    k^2    & -(1+k^2)  &      1     &\rfn{ns}\\
%\lfn{cn}&   -k^2    & -(1-2k^2) &    1-k^2   &\rfn{nc}\\
%\lfn{dn}&     1     &  2-k^2    &   -(1-k^2) &\rfn{nd}\\
%\hline
%\lfn{sc}&   1-k^2   &  2-k^2    &      1     &\rfn{cs}\\
%\lfn{sd}&-k^2(1-k^2)&-(1-2k^2)  &       1    &\rfn{ds}\\
%\lfn{cd}&    k^2    &-(1+k^2)   &      1     &\rfn{dc}\\
%\hline 
%                 &   \gamma  & \beta     &   \alpha   &\rfn{Reziproke}\\
%\cline{2-5}
%\end{tabular}
%\caption{Koeffizienten der Differentialgleichungen für die Jacobischen
%elliptischen Funktionen.
%Der Kehrwert einer Funktion hat jeweils die Differentialgleichung der
%ursprünglichen Funktion, in der die Koeffizienten $\alpha$ und $\gamma$
%vertauscht worden sind.
%\label{buch:elliptisch:table:differentialgleichungen}}
%\end{table}
%
%%
%% Differentialgleichung zweiter Ordnung
%%
%\subsubsection{Differentialgleichung zweiter Ordnung}
%Leitet die Differentialgleichung~\eqref{buch:elliptisch:eqn:1storderdglell}
%man dies nochmals nach $u$ ab, erhält man die Differentialgleichung
%\[
%2\operatorname{pq}''(u,k)\operatorname{pq}'(u,k)
%=
%4\alpha \operatorname{pq}(u,k)^3\operatorname{pq}'(u,k) + 2\beta \operatorname{pq}'(u,k)\operatorname{pq}(u,k).
%\]
%Teilt man auf beiden Seiten durch $2\operatorname{pq}'(u,k)$,
%bleibt die nichtlineare
%Differentialgleichung
%\[
%\frac{d^2\operatorname{pq}}{du^2}
%=
%\beta \operatorname{pq} + 2\alpha \operatorname{pq}^3.
%\]
%Dies ist die Gleichung eines harmonischen Oszillators mit einer 
%Anharmonizität der Form $2\alpha z^3$.
%
%
%
%%
%% Jacobischen elliptische Funktionen und elliptische Integrale
%%
%\subsubsection{Jacobische elliptische Funktionen als elliptische Integrale}
%Die in Tabelle~\ref{buch:elliptisch:tabelle:loesungsfunktionen}
%zusammengestellten Differentialgleichungen ermöglichen nun, den
%Zusammenhang zwischen den Funktionen 
%$\operatorname{sn}(u,k)$, $\operatorname{cn}(u,k)$ und $\operatorname{dn}(u,k)$
%und den unvollständigen elliptischen Integralen herzustellen.
%Die Differentialgleichungen sind alle von der Form
%\begin{equation}
%\biggl(
%\frac{d y}{d u}
%\biggr)^2
%=
%p(u),
%\label{buch:elliptisch:eqn:allgdgl}
%\end{equation}
%wobei $p(u)$ ein Polynom vierten Grades in $y$ ist.
%Diese Differentialgleichung lässt sich mit Separation lösen.
%Dazu zieht man aus~\eqref{buch:elliptisch:eqn:allgdgl} die
%Wurzel
%\begin{align}
%\frac{dy}{du}
%=
%\sqrt{p(y)}
%\notag
%\intertext{und trennt die Variablen. Man erhält}
%\int\frac{dy}{\sqrt{p(y)}} = u+C.
%\label{buch:elliptisch:eqn:yintegral}
%\end{align}
%Solange $p(y)>0$ ist, ist der Integrand auf der linken Seite
%von~\eqref{buch:elliptisch:eqn:yintegral} ebenfalls positiv und
%das Integral ist eine monoton wachsende Funktion $F(y)$.
%Insbesondere ist $F(y)$ invertierbar.
%Die Lösung $y(u)$ der Differentialgleichung~\eqref{buch:elliptisch:eqn:allgdgl}
%ist daher 
%\[
%y(u) = F^{-1}(u+C).
%\]
%Die Jacobischen elliptischen Funktionen sind daher inverse Funktionen
%der unvollständigen elliptischen Integrale.
%
%
%%
%% Differentialgleichung des anharmonischen Oszillators
%%
%\subsubsection{Differentialgleichung des anharmonischen Oszillators}
%Wir möchten die nichtlineare Differentialgleichung
%\begin{equation}
%\biggl(
%\frac{dx}{dt}
%\biggr)^2
%=
%Ax^4+Bx^2 + C
%\label{buch:elliptisch:eqn:allgdgl}
%\end{equation}
%mit Hilfe elliptischer Funktionen lösen.
%Wir nehmen also an, dass die gesuchte Lösung eine Funktion der Form
%\begin{equation}
%x(t) = a\operatorname{zn}(bt,k)
%\label{buch:elliptisch:eqn:loesungsansatz}
%\end{equation}
%ist.
%Die erste Ableitung von $x(t)$ ist
%\[
%\dot{x}(t) 
%=
%a\operatorname{zn}'(bt,k).
%\]
%
%Indem wir diesen Lösungsansatz in die
%Differentialgleichung~\eqref{buch:elliptisch:eqn:allgdgl}
%einsetzen, erhalten wir
%\begin{equation}
%a^2b^2 \operatorname{zn}'(bt,k)^2
%=
%a^4A\operatorname{zn}(bt,k)^4
%+
%a^2B\operatorname{zn}(bt,k)^2
%+C
%\label{buch:elliptisch:eqn:dglx}
%\end{equation}
%Andererseits wissen wir, dass $\operatorname{zn}(u,k)$ einer
%Differentilgleichung der Form~\eqref{buch:elliptisch:eqn:1storderdglell}
%erfüllt.
%Wenn wir \eqref{buch:elliptisch:eqn:dglx} durch $a^2b^2$ teilen, können wir
%die rechte Seite von \eqref{buch:elliptisch:eqn:dglx} mit der rechten
%Seite von \eqref{buch:elliptisch:eqn:1storderdglell} vergleichen:
%\[
%\frac{a^2A}{b^2}\operatorname{zn}(bt,k)^4
%+
%\frac{B}{b^2}\operatorname{zn}(bt,k)^2
%+\frac{C}{a^2b^2}
%=
%\alpha\operatorname{zn}(bt,k)^4
%+
%\beta\operatorname{zn}(bt,k)^2
%+
%\gamma\operatorname{zn}(bt,k).
%\]
%Daraus ergeben sich die Gleichungen
%\begin{align}
%\alpha &= \frac{a^2A}{b^2},
%&
%\beta &= \frac{B}{b^2}
%&&\text{und}
%&
%\gamma &= \frac{C}{a^2b^2}
%\label{buch:elliptisch:eqn:koeffvergl}
%\intertext{oder aufgelöst nach den Koeffizienten der ursprünglichen
%Differentialgleichung}
%A&=\frac{\alpha b^2}{a^2}
%&
%B&=\beta b^2
%&&\text{und}&
%C &= \gamma a^2b^2
%\label{buch:elliptisch:eqn:koeffABC}
%\end{align}
%für die Koeffizienten der Differentialgleichung der zu verwendenden
%Funktion.
%
%Man beachte, dass nach \eqref{buch:elliptisch:eqn:koeffvergl} die 
%Koeffizienten $A$, $B$ und $C$ die gleichen Vorzeichen haben wie
%$\alpha$, $\beta$ und $\gamma$, da in 
%\eqref{buch:elliptisch:eqn:koeffvergl} nur mit Quadraten multipliziert
%wird, die immer positiv sind.
%Diese Vorzeichen bestimmen, welche der Funktionen gewählt werden muss.
%
%In den Differentialgleichungen für die elliptischen Funktionen gibt
%es nur den Parameter $k$, der angepasst werden kann.
%Es folgt, dass die Gleichungen
%\eqref{buch:elliptisch:eqn:koeffvergl} 
%auch $a$ und $b$ bestimmen.
%Zum Beispiel folgt aus der letzten Gleichung, dass
%\[
%b = \pm\sqrt{\frac{B}{\beta}}.
%\]
%Damit folgt dann aus der zweiten
%\[
%a=\pm\sqrt{\frac{\beta C}{\gamma B}}.
%\]
%Die verbleibende Gleichung legt $k$ fest.
%Das folgende Beispiel illustriert das Vorgehen am Beispiel einer
%Gleichung, die Lösungsfunktion $\operatorname{sn}(u,k)$ verlangt.
%
%\begin{beispiel}
%Wir nehmen an, dass die Vorzeichen von $A$, $B$ und $C$ gemäss
%Tabelle~\ref{buch:elliptische:tabelle:loesungsfunktionen} verlangen,
%dass die Funktion $\operatorname{sn}(u,k)$ für die Lösung verwendet
%werden muss.
%Die Tabelle sagt dann auch, dass 
%$\alpha=k^2$, $\beta=1$ und $\gamma=1$ gewählt werden müssen.
%Aus dem Koeffizientenvergleich~\eqref{buch:elliptisch:eqn:koeffvergl}
%folgt dann der Reihe nach
%\begin{align*}
%b&=\pm \sqrt{B}
%\\
%a&=\pm \sqrt{\frac{C}{B}}
%\\
%k^2
%&=
%\frac{AC}{B^2}.
%\end{align*}
%Man beachte, dass man $k^2$ durch Einsetzen von
%\eqref{buch:elliptisch:eqn:koeffABC}
%auch direkt aus den Koeffizienten $\alpha$, $\beta$ und $\gamma$
%erhalten kann, nämlich
%\[
%\frac{AC}{B^2}
%=
%\frac{\frac{\alpha b^2}{a^2} \gamma a^2b^2}{\beta^2 b^4}
%=
%\frac{\alpha\gamma}{\beta^2}.
%\qedhere
%\]
%\end{beispiel}
%
%Da alle Parameter im 
%Lösungsansatz~\eqref{buch:elliptisch:eqn:loesungsansatz} bereits
%festgelegt sind stellt sich die Frage, woher man einen weiteren
%Parameter nehmen kann, mit dem Anfangsbedingungen erfüllen kann.
%Die Differentialgleichung~\eqref{buch:elliptisch:eqn:allgdgl} ist
%autonom, die Koeffizienten der rechten Seite der Differentialgleichung
%sind nicht von der Zeit abhängig. 
%Damit ist eine zeitverschobene Funktion $x(t-t_0)$ ebenfalls eine
%Lösung der Differentialgleichung.
%Die allgmeine Lösung der 
%Differentialgleichung~\eqref{buch:elliptisch:eqn:allgdgl} hat
%also die Form
%\[
%x(t) = a\operatorname{zn}(b(t-t_0)),
%\]
%wobei die Funktion $\operatorname{zn}(u,k)$ auf Grund der Vorzeichen
%von $A$, $B$ und $C$ gewählt werden müssen.

%%
%% Das mathematische Pendel
%%
%\subsection{Das mathematische Pendel
%\label{buch:elliptisch:subsection:mathpendel}}
%\begin{figure}
%\centering
%\includegraphics{chapters/110-elliptisch/images/pendel.pdf}
%\caption{Mathematisches Pendel
%\label{buch:elliptisch:fig:mathpendel}}
%\end{figure}
%Das in Abbildung~\ref{buch:elliptisch:fig:mathpendel} dargestellte
%Mathematische Pendel besteht aus einem Massepunkt der Masse $m$
%im Punkt $P$,
%der über eine masselose Stange der Länge $l$ mit dem Drehpunkt $O$
%verbunden ist.
%Das Pendel bewegt sich unter dem Einfluss der Schwerebeschleunigung $g$.
%
%Das Trägheitsmoment des Massepunktes um den Drehpunkt $O$ ist
%\(
%I=ml^2
%\).
%Das Drehmoment der Schwerkraft ist
%\(M=gl\sin\vartheta\).
%Die Bewegungsgleichung wird daher
%\[
%\begin{aligned}
%\frac{d}{dt} I\dot{\vartheta}
%&=
%M
%=
%gl\sin\vartheta
%\\
%ml^2\ddot{\vartheta}
%&=
%gl\sin\vartheta
%&&\Rightarrow&
%\ddot{\vartheta}
%&=\frac{g}{l}\sin\vartheta
%\end{aligned}
%\]
%Dies ist eine nichtlineare Differentialgleichung zweiter Ordnung, die
%wir nicht unmittelbar mit den Differentialgleichungen erster Ordnung
%der elliptischen Funktionen vergleichen können.
%
%Die Differentialgleichungen erster Ordnung der elliptischen Funktionen
%enthalten das Quadrat der ersten Ableitung.
%In unserem Fall entspricht das einer Gleichung, die $\dot{\vartheta}^2$
%enthält.
%Der Energieerhaltungssatz kann uns eine solche Gleichung geben.
%Die Summe von kinetischer und potentieller Energie muss konstant sein.
%Dies führt auf
%\[
%E_{\text{kinetisch}}
%+
%E_{\text{potentiell}}
%=
%\frac12I\dot{\vartheta}^2
%+
%mgl(1-\cos\vartheta)
%=
%\frac12ml^2\dot{\vartheta}^2
%+
%mgl(1-\cos\vartheta)
%=
%E
%\]
%Durch Auflösen nach $\dot{\vartheta}$ kann man jetzt die
%Differentialgleichung
%\[
%\dot{\vartheta}^2
%=
%-
%\frac{2g}{l}(1-\cos\vartheta)
%+\frac{2E}{ml^2}
%\]
%finden.
%In erster Näherung, d.h. wenn man die rechte Seite bis zu vierten
%Potenzen in eine Taylor-Reihe in $\vartheta$ entwickelt,  ist dies
%tatsächlich eine Differentialgleichung der Art, wie wir sie für
%elliptische Funktionen gefunden haben, wir möchten aber eine exakte
%Lösung konstruieren.
%
%Die maximale Energie für eine Bewegung, bei der sich das Pendel gerade
%über den höchsten Punkt hinweg zu bewegen vermag, ist 
%$E=2lmg$.
%Falls $E<2mgl$ ist, erwarten wir Schwingungslösungen, bei denen 
%der Winkel $\vartheta$ immer im offenen Interval $(-\pi,\pi)$
%bleibt.
%Für $E>2mgl$ wird sich das Pendel im Kreis bewegen, für sehr grosse
%Energie ist die kinetische Energie dominant, die Verlangsamung im
%höchsten Punkt wird immer weniger ausgeprägt sein.
%
%%
%% Koordinatentransformation auf elliptische Funktionen
%%
%\subsubsection{Koordinatentransformation auf elliptische Funktionen}
%Wir verwenden als neue Variable 
%\[
%y = \sin\frac{\vartheta}2
%\]
%mit der Ableitung
%\[
%\dot{y}=\frac12\cos\frac{\vartheta}{2}\cdot \dot{\vartheta}.
%\]
%Man beachte, dass $y$ nicht eine Koordinate in
%Abbildung~\ref{buch:elliptisch:fig:mathpendel} ist.
%
%Aus den Halbwinkelformeln finden wir
%\[
%\cos\vartheta
%=
%1-2\sin^2 \frac{\vartheta}2
%=
%1-2y^2.
%\]
%Dies können wir zusammen mit der
%Identität $\cos^2\vartheta/2 = 1-\sin^2\vartheta/2 = 1-y^2$
%in die Energiegleichung einsetzen und erhalten
%\[
%\frac12ml^2\dot{\vartheta}^2 + mgly^2 = E
%\qquad\Rightarrow\qquad
%\frac14 \dot{\vartheta}^2 = \frac{E}{2ml^2} - \frac{g}{2l}y^2.
%\]
%Der konstante Term auf der rechten Seite ist grösser oder kleiner als
%$1$ je nachdem, ob das Pendel sich im Kreis bewegt oder nicht.
%
%Durch Multiplizieren mit $\cos^2\frac{\vartheta}{2}=1-y^2$
%erhalten wir auf der linken Seite einen Ausdruck, den wir
%als Funktion von $\dot{y}$ ausdrücken können.
%Wir erhalten
%\begin{align*}
%\frac14
%\cos^2\frac{\vartheta}2
%\cdot
%\dot{\vartheta}^2
%&=
%\frac14
%(1-y^2)
%\biggl(\frac{E}{2ml^2} -\frac{g}{2l}y^2\biggr)
%\\
%\dot{y}^2
%&=
%\frac{1}{4}
%(1-y^2)
%\biggl(\frac{E}{2ml^2} -\frac{g}{2l}y^2\biggr)
%\end{align*}
%Die letzte Gleichung hat die Form einer Differentialgleichung
%für elliptische Funktionen.
%Welche Funktion verwendet werden muss, hängt von der Grösse der
%Koeffizienten in der zweiten Klammer ab.
%Die Tabelle~\ref{buch:elliptisch:tabelle:loesungsfunktionen}
%zeigt, dass in der zweiten Klammer jeweils einer der Terme
%$1$ sein muss.
%
%%
%% Der Fall E < 2mgl
%%
%\subsubsection{Der Fall $E<2mgl$}
%\begin{figure}
%\centering
%\includegraphics[width=\textwidth]{chapters/110-elliptisch/images/jacobiplots.pdf}
%\caption{%
%Abhängigkeit der elliptischen Funktionen von $u$ für
%verschiedene Werte von $k^2=m$.
%Für $m=0$ ist $\operatorname{sn}(u,0)=\sin u$, 
%$\operatorname{cn}(u,0)=\cos u$ und $\operatorname{dn}(u,0)=1$, diese
%sind in allen Plots in einer helleren Farbe eingezeichnet.
%Für kleine Werte von $m$ weichen die elliptischen Funktionen nur wenig
%von den trigonometrischen Funktionen ab,
%es ist aber klar erkennbar, dass die anharmonischen Terme in der
%Differentialgleichung die Periode mit steigender Amplitude verlängern.
%Sehr grosse Werte von $m$ nahe bei $1$ entsprechen der Situation, dass
%die Energie des Pendels fast ausreicht, dass es den höchsten Punkt
%erreichen kann, was es für $m$ macht.
%\label{buch:elliptisch:fig:jacobiplots}}
%\end{figure}
%
%
%Wir verwenden als neue Variable 
%\[
%y = \sin\frac{\vartheta}2
%\]
%mit der Ableitung
%\[
%\dot{y}=\frac12\cos\frac{\vartheta}{2}\cdot \dot{\vartheta}.
%\]
%Man beachte, dass $y$ nicht eine Koordinate in
%Abbildung~\ref{buch:elliptisch:fig:mathpendel} ist.
%
%Aus den Halbwinkelformeln finden wir
%\[
%\cos\vartheta
%=
%1-2\sin^2 \frac{\vartheta}2
%=
%1-2y^2.
%\]
%Dies können wir zusammen mit der
%Identität $\cos^2\vartheta/2 = 1-\sin^2\vartheta/2 = 1-y^2$
%in die Energiegleichung einsetzen und erhalten
%\[
%\frac12ml^2\dot{\vartheta}^2 + mgly^2 = E.
%\]
%Durch Multiplizieren mit $\cos^2\frac{\vartheta}{2}=1-y^2$
%erhalten wir auf der linken Seite einen Ausdruck, den wir
%als Funktion von $\dot{y}$ ausdrücken können.
%Wir erhalten
%\begin{align*}
%\frac12ml^2
%\cos^2\frac{\vartheta}2
%\dot{\vartheta}^2
%&=
%(1-y^2)
%(E -mgly^2)
%\\
%\frac{1}{4}\cos^2\frac{\vartheta}{2}\dot{\vartheta}^2
%&=
%\frac{1}{2}
%(1-y^2)
%\biggl(\frac{E}{ml^2} -\frac{g}{l}y^2\biggr)
%\\
%\dot{y}^2
%&=
%\frac{E}{2ml^2}
%(1-y^2)\biggl(
%1-\frac{2gml}{E}y^2
%\biggr).
%\end{align*}
%Dies ist genau die Form der Differentialgleichung für die elliptische
%Funktion $\operatorname{sn}(u,k)$
%mit $k^2 = 2gml/E< 1$.
%
%%%
%%% Der Fall E > 2mgl
%%%
%%\subsection{Der Fall $E > 2mgl$}
%%In diesem Fall hat das Pendel im höchsten Punkte immer noch genügend
%%kinetische Energie, so dass es sich im Kreise dreht.
%%Indem wir die Gleichung
%
%
%%\subsection{Soliton-Lösungen der Sinus-Gordon-Gleichung}
%
%%\subsection{Nichtlineare Differentialgleichung vierter Ordnung}
%%XXX Möbius-Transformation \\
%%XXX Reduktion auf die Differentialgleichung elliptischer Funktionen
