%
% jacobi.tex
%
% (c) 2021 Prof Dr Andreas Müller, OST Ostschweizer Fachhochschule
%
\section{Jacobische elliptische Funktionen
\label{buch:elliptisch:section:jacobi}}
\rhead{Jacobische elliptische Funktionen}
Die elliptischen Integrale von
Abschnitt~\ref{buch:elliptisch:section:integral}
können dazu verwendet werden, die Länge eines Ellipsenbogens aus
den Koordinaten der Endpunkte zu berechnen.
Die trigonometrischen Funktionen drücken dagegen umgekehrt die
Koordinaten eines Punktes auf einem Kreis aus der Länge des
Kreisbogens aus.
Das elliptische Integral, welches die Bogenlänge auf einer Ellipse zwischen
den Punkten $(1,0)$ und $(x,y)$ entsprecht also eher der Funktion
$\arcsin y=\sin^{-1}y$.
Möchte man Funktionen konstruieren, die die Eigenschaften der 
trigonometrischen Funktionen auf die Geometrie von Ellipsen erweitern,
dann muss man die Umkehrfunktionen der elliptischen Integrale dafür ins
Auge fassen.




