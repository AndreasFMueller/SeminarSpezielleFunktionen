\label{buch:elliptisch:aufgabe:5}
Die sehr schnelle Konvergenz des arithmetisch-geometrische Mittels
kann auch dazu ausgenutzt werden, eine grosse Zahl von Stellen der
Kreiszahl $\pi$ zu berechnen.
Almkvist und Berndt haben gezeigt \cite{buch:almkvist-berndt}, dass
\[
\pi
=
\frac{4 M(1,\!\sqrt{2}/2)^2}{
\displaystyle 1-\sum_{n=1}^\infty 2^{n+1}(a_n^2-b_n^2)
}.
\]
Verwenden Sie diese Formel, um Approximationen von $\pi$ zu berechnen.

\begin{loesung}
\begin{table}
\centering
\begin{tabular}{|>{$}c<{$}|>{$}c<{$}|>{$}c<{$}|>{$}c<{$}|}
\hline
n & a_n               & b_n               & \pi_n%
\mathstrut\text{\vrule height12pt depth6pt width0pt}\\
\hline
\mathstrut\text{\vrule height12pt depth0pt width0pt}%
0 & 1.000000000000000 & 0.707106781186548 &                  
\mathstrut\text{\vrule height12pt depth0pt width0pt}\\
1 & 0.853553390593274 & 0.840896415253715 & 3.\underline{1}87672642712106 \\
2 & 0.847224902923494 & 0.847201266746892 & 3.\underline{141}680293297648 \\
3 & 0.847213084835193 & 0.847213084752765 & 3.\underline{141592653}895451 \\
4 & 0.847213084793979 & 0.847213084793979 & 3.\underline{141592653589}822 \\
5 & 0.847213084793979 & 0.847213084793979 & 3.\underline{141592653589}871%
\mathstrut\text{\vrule height0pt depth6pt width0pt}\\
\hline
\infty &              &                   & 3.141592653589793%
\mathstrut\text{\vrule height12pt depth6pt width0pt}\\
\hline
\end{tabular}
\caption{Approximationen der Kreiszahl $\pi$ mit Hilfe des Algorithmus
des arithmetisch-geometrischen Mittels.
In nur 4 Schritten werden 12 Stellen Genauigkeit erreicht.
\label{buch:elliptisch:aufgabe:5:table}}
\end{table}
Wir schreiben
\[
\pi_n
=
\frac{4 a_k^2}{
\displaystyle
1-\sum_{k=1}^\infty 2^{k+1}(a_k^2-b_k^2)
}
\]
für die Approximationen von $\pi$,
wobei $a_k$ und $b_k$ die Folgen der arithmetischen und geometrischen
Mittel von $1$ und $\!\sqrt{2}/2$ sind.
Die Tabelle~\ref{buch:elliptisch:aufgabe:5:table} zeigt die Resultat.
In nur 4 Schritten können 12 Stellen Genauigkeit erreicht werden,
dann beginnen jedoch bereits Rundungsfehler das Resultat zu beinträchtigen.
Für die Berechnung einer grösseren Zahl von Stellen muss daher mit
grösserer Präzision gerechnet werden.
\end{loesung}
