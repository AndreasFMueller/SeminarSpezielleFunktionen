\label{buch:elliptisch:aufgabe:4}
Es ist bekannt, dass $\operatorname{sn}(K+iK', k) = 1/k$ gilt.
Verwenden Sie den Algorithmus von Aufgabe~\ref{buch:elliptisch:aufgabe:3},
um dies für $k=\frac12$ nachzurechnen.

\begin{loesung}
\begin{table}
\centering
\renewcommand{\tabcolsep}{5pt}
\begin{tabular}{|>{$}c<{$}|>{$}c<{$}|>{$}c<{$}|>{$}c<{$}|}
\hline
 n & k_n               & u_n                                    & \operatorname{sn}(u_n,k_n)%
\mathstrut\text{\vrule height12pt depth6pt width0pt}%
\\
\hline
\mathstrut\text{\vrule height12pt depth0pt width0pt}%
 0 & 0.500000000000000 & 1.685750354812596 + 2.156515647499643i & 2.000000000000000 \\
 1 & 0.071796769724491 & 1.572826493259468 + 2.012056490946491i & 3.732050807568877 \\
 2 & 0.001292026239995 & 1.570796982340579 + 2.009460215619685i & 3.796651109009551 \\
 3 & 0.000000417333300 & 1.570796326794965 + 2.009459377005374i & 3.796672364209438 \\
 4 & 0.000000000000044 & 1.570796326794897 + 2.009459377005286i & 3.796672364211658 \\
 N & 0.000000000000000 & 1.570796326794897 + 2.009459377005286i & 3.796672364211658%
\mathstrut\text{\vrule height12pt depth6pt width0pt}%
\\
\hline
\end{tabular}
\caption{Berechnung von $\operatorname{sn}(K+iK',k)=1/k$ mit Hilfe der Landen-Transformation.
Konvergenz der Folge $k_n$ ist bei $N=5$ eintegreten.
\label{buch:elliptisch:aufgabe:4:table}}
\end{table}
Zunächst müssen wir mit dem Algorithmus des arithmetisch-geometrischen 
Mittels
\[
K(k)
\approx
1.685750354812596
\qquad\text{und}\qquad
K(k')
\approx
2.156515647499643
\]
berechnen.
Aus $k=\frac12$ kann man jetzt die Folgen $k_n$ und $u_n$ berechnen, die innert
$N=5$ Iterationen konvergiert.
Sie führt auf 
\[
u_N 
=
\frac{\pi}2 + 2.009459377005286i
=
\frac{\pi}2 + bi.
\]
Jetzt muss der Sinus von $u_N$ berechnet werden.
Dazu verwenden wir die komplexe Darstellung:
\[
\sin u_N
=
\frac{e^{i\frac{\pi}2-b} - e^{-i\frac{\pi}2+b}}{2i}
=
\frac{ie^{-b}+ie^{b}}{2i}
=
\cosh b
=
3.796672364211658.
\]
Da der Wert $\operatorname{sn}(u_N,k_N) = \sin u_N$ reell ist, wird auch
die daraus wie in Aufgabe~\ref{buch:elliptisch:aufgabe:3}
konstruierte Folge $\operatorname{sn}(u_n,k_n)$ reell sein.
Die Werte von $\operatorname{cn}(u_n,k_n)$ und $\operatorname{dn}(u_n,k_n)$
werden für die Iterationsformeln~\eqref{buch:elliptisch:aufgabe:3:gauss}
für $\operatorname{sn}(u_n,k_n)$ nicht benötigt.
Die Berechnung ist in Tabelle~\ref{buch:elliptisch:aufgabe:4:table}
zusammengefasst.
Man liest ab, dass $\operatorname{sn}(K+iK',k)=2 = 1/k$, wie erwartet.
\end{loesung}
