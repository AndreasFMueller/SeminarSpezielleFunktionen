\label{buch:elliptisch:aufgabe:2}%
Die Landen-Transformation basiert auf der Iteration
\begin{equation}
\begin{aligned}
k_{n+1}
&=
\frac{1-k_n'}{1+k_n'}
&
&\text{und}&
k_{n+1}'
&=
\sqrt{1-k_{n+1}^2}
\end{aligned}
\label{buch:elliptisch:aufgabe:2:iteration}
\end{equation}
mit den Startwerten $k_0 = k$ und $k_0' = \sqrt{1-k_0^2}$.
Zeigen Sie, dass $k_n\to 0$ und $k_n'\to 1$ mit quadratischer Konvergenz.

\begin{loesung}
\begin{table}
\centering
\begin{tabular}{|>{$}c<{$}|>{$}c<{$}|>{$}c<{$}|}
\hline
n &         k         &         k'%
\mathstrut\text{\vrule height12pt depth6pt width0pt}%
\\
\hline
\mathstrut\text{\vrule height12pt depth0pt width0pt}%
0 & 0.200000000000000 & 0.979795897113271 \\
1 & 0.010205144336438 & 0.999947926158694 \\
2 & 0.000026037598592 & 0.999999999661022 \\
3 & 0.000000000169489 & 1.000000000000000 \\
4 & 0.000000000000000 & 1.000000000000000%
\mathstrut\text{\vrule height0pt depth6pt width0pt}\\
\hline
\end{tabular}
\caption{Numerisches Experiment zur Folge $(k_n,k_n')$ 
gemäss \eqref{buch:elliptisch:aufgabe:2:iteration}
mit $k_0=0.2$
\label{buch:ellptisch:aufgabe:2:numerisch}}
\end{table}
Es ist klar, dass $k'_n\to 1$ folgt, wenn man zeigen kann, dass 
$k_n\to 0$ gilt.
Wir berechnen daher 
\begin{align*}
k_{n+1}
&=
\frac{1-k_n'}{1+k_n'}
=
\frac{1-\sqrt{1-k_n^2}}{1+\sqrt{1-k_n^2}}
\intertext{und erweitern mit dem Nenner $1+\sqrt{1-k_n^2}$, um}
&=
\frac{1-(1-k_n^2)}{(1+\sqrt{1-k_n^2})^2}
=
\frac{ k_n^2 }{(1+\sqrt{1-k_n^2})^2}
\le
k_n^2
\end{align*}
zu erhalten.
Daraus folgt jetzt sofort die quadratische Konvergenz von $k_n$ gegen $0$.

Ein einfaches numerisches Experiment (siehe
Tabelle~\ref{buch:ellptisch:aufgabe:2:numerisch})
bestätigt die quadratische Konvergenz der Folgen.
\end{loesung}
