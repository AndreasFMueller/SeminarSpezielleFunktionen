%
% part1.tex
%
% (c) 2018 Prof Dr Andreas Müller, Hochschule Rapperswil
%
\begin{refsection}
\input{chapters/vorwort.tex}
\part{Grundlagen}
%
% chapter.tex -- Beschreibung des Inhaltes
%
% (c) 2021 Prof Dr Andreas Müller, Hochschule Rapperswil
%
% !TeX spellcheck = de_CH
\chapter{Spezielle Funktionen und Rekursion
\label{buch:chapter:rekursion}}
\lhead{Spezielle Funktionen und Rekursion}
\rhead{}

%
% gamma.tex -- Abschnitt über die Gamma-funktion
%
% (c) 2021 Prof Dr Andreas Müller, OST Ostschweizer Fachhochschule
%
\section{Die Gamma-Funktion
\label{buch:rekursion:section:gamma}}
Die Fakultät $x!$ kann rekursiv durch 
\[
	x! = x\cdot (x-1)! \qquad\text{und}\qquad 0!=1
\]
für alle natürlichen Zahlen $x\in\mathbb{N}$ definiert werden.
Äquivalent damit ist eine Funktion 
\begin{equation}
\Gamma(x+1) = x\Gamma(x)
\qquad\text{und}\qquad 
\Gamma(1)=1.
\label{buch:rekursion:eqn:gammadef}
\end{equation}
Kann man eine reelle oder komplexe Funktion finden, die die
Funktionalgleichung~\eqref{buch:rekursion:eqn:gammadef}
erfüllt und damit die Fakultät auf beliebige Argumente ausdehnt?

\subsection{Integralformel für die Gamma-Funktion}
Euler hat die folgende Integraldefinition der Gamma-Funktion gegeben.

\begin{definition}
\label{buch:rekursion:def:gamma}
Die Gamma-Funktion ist die Funktion 
\[
\Gamma
\colon
\{z\in\mathbb{C} \mid \operatorname{Re}z>0\}
\to \mathbb{C}
:
z
\mapsto
\Gamma(z) = \int_0^\infty t^{x-1}e^{-t}\,dt
\]
\end{definition}

Man beachte, dass das Integral für $x=0$ nicht definiert ist, eine
Potenzreihenentwicklung um einen Punkt $x_0$ auf der positiven reellen
Achse kann also höchstens den Konvergenzradius $\varrho=|x_0|$ haben.

\begin{figure}
\centering
\includegraphics{chapters/040-rekursion/images/gammaplot.pdf}
\caption{Graph der Gamma-Funktion $z\mapsto\Gamma(z)$ und der alternativen
Funktion $\Gamma(z)+\sin(\pi z)$, die für ganzzahlige Argumente ebenfalls
die Werte der Fakultät annimmt.
\label{buch:rekursion:fig:gamma}}
\end{figure}

\subsubsection{Alternative Lösungen}
Die Funktion $\Gamma(z)$ ist nicht die einzige Funktion, die natürlichen
Zahlen die Werte $\Gamma(n+1) = n!$ der Fakultät annimmt.
Indem man eine beliebige Funktion $f(z)$ addiert, die auf alle
natürlichen Zahlen verschwindet, also $f(n)=0$ für $n\in\mathbb{N}$,
erhält man eine weitere Funktion, die auf natürlichen Zahlen
die Werte der Fakultät annimmt.
Ein Beispiel einer solchen Funktion ist
\begin{equation}
z\mapsto f(z)=\Gamma(z) + \sin \pi z,
\label{buch:rekursion:eqn:gammaalternative}
\end{equation}
die Funktion $f(z)=\sin\pi z$ verschwindet sogar auf allen ganzen
Zahlen.

In Abbildung~\ref{buch:rekursion:fig:gamma} ist die Gamma-Funktion
in rot geplotet, die Funktion~\eqref{buch:rekursion:eqn:gammaalternative}
in grün.
Die Punkte $(n,(n-1)!)$ sind in blau bezeichnet, sie sind beiden Graphen
gemeinsam.

\subsubsection{Pol erster Ordnung bei $z=0$}
Wir haben zu prüfen, dass sowohl der Wert $\Gamma(1)$ korrekt ist als
auch die Rekursionsformel~\eqref{buch:rekursion:eqn:gammadef} gilt.
Der Wert für $z=1$ ist
\begin{align*}
\Gamma(1)
&=
\int_0^\infty t^{1-1}e^{-t}\,dt
=
\left[ -e^{-t} \right]_0^\infty
=
1.
\end{align*}
Für die Rekursionsformel kann mit Hilfe von partieller Integration
bekommen:
\begin{align*}
\Gamma(z+1)
&=
\int_0^\infty t^{z+1-1}e^{-t}\,dt
=
\biggl[-t^{z}e^{-t}\biggr]_0^\infty
+
\int_0^\infty z t^{z-1}e^{-t}\,dt
\\
&=
z
\int_0^\infty
t^{z-1}e^{-t}\,dt
=
z \Gamma(z).
\end{align*}

Für $0<z<\varepsilon$ für eine $\varepsilon >0$ folgt aus der 
Funktionalgleichung
\[
\Gamma(z) = \frac{\Gamma(1+z)}{z}.
\]
Da $\Gamma(1)=1$ ist und $\Gamma$ eine in einer
Umgebung von $1$ stetige Funktion ist, kann sie in der Form
\(
\Gamma(1+z)=\Gamma(1) + zf(z)
\)
schreiben, wobei  $f(z)$ eine differenzierbare Funktion ist mit
$f'(1)=\Gamma'(1)$.
Daraus ergibt sich für $\Gamma(z)$ der Ausdruck
\[
\Gamma(z) = \frac{\Gamma(1)}{z} + f(z) = \frac{1}{z} + f(z).
\]
Die Gamma-Funktion hat daher and er Stelle $z=0$ einen Pol erster Ordnung.

\subsubsection{Ausdehnung auf $\operatorname{Re}z<0$}
Die Integralformel konvergiert nicht für $\operatorname{Re}z\le 0$.
Durch analytische Fortsetzung, wie sie im
Abschnitt~\ref{buch:funktionentheorie:section:fortsetzung}
beschrieben wird, kann die Funktion auf ganz $\mathbb{C}$ ausgedehnt
werden, mit Ausnahme einzelner Pole.
Die Funktionalgleichung gilt natürlich für alle $z\in\mathbb{C}$,
für die $\Gamma(z)$ definiert ist.
In einer Umgebung von $z=-n$ gilt
\[
\Gamma(z)
=
\frac{\Gamma(z+1)}{z}
=
\frac{\Gamma(z+2)}{z(z+1)}
=
\frac{\Gamma(z+3)}{z(z+1)(z+2)}
=
\dots
=
\frac{\Gamma(z+n)}{z(z+1)(z+2)\cdots(z+n-1)}
\]
Keiner der Faktoren im Nenner verschwindet in der Nähe von $z=-n$, der
Zähler hat aber einen Pol erster Ordnung an dieser Stelle.
Daher hat auch der Quotient einen Pol erster Ordnung.
Abbildung~\ref{buch:rekursion:fig:gamma} zeigt die Pole bei den
nicht negativen ganzen Zahlen.






%
% linear.tex
%
% (c) 2021 Prof Dr Andreas Müller, OST Ostschweizer Fachhochschule
%
\section{Lineare Rekursionsgleichung mit konstanten Koeffizienten
\label{buch:rekursion:section:linear}}
\rhead{Lineare Rekursionsgleichungen}
Die Funktionalgleichung der Gamma-Funktion, die im
Abschnitt~\ref{buch:rekursion:section:gamma} untersucht wurde,
hat die Form einer linearen Rekursionsgleichung
\[
\Gamma(x+1) = x\Gamma(x),\qquad \Gamma(1) = 1.
\]
Gleichungen, die Werte einer Funktion für verschiedene
Argument in Beziehung setzen, heissen {\em Funktionalgleichungen}.
\index{Funktionalgleichung}%
Es war überraschend schwierig, eine Lösung für Funktionalgleichung
der Gamma-Funktion für beliebige komplexe $x$ zu finden.
In diesem Abschnitt soll daher eine Klasse von Rekursionsgleichungen
näher untersucht werden, für die einfache Lösungen möglich sind.

\subsection{Lineare Differenzengleichungen}

\subsection{Lösung mit Polynomfunktionen}







%
% hypergeometrisch.tex
%
% (c) 2021 Prof Dr Andreas Müller, OST Ostschweizer Fachhochschule
%
\section{Hypergeometrische Differentialgleichung
\label{buch:differentialgleichungen:section:hypergeometrisch}}
Die hypergeometrische Funktion $\mathstrut_2F1(a,b;c;x)$ wurde in
Abschnitt~\ref{buch:rekursion:section:hypergeometrische-funktion}
als Potenzreihe mit sehr speziellen Koeffizienten, die sich aus
Pochhammer-Symbolen.
Es stellt sich aber heraus, dass man sie auch als Lösung einer
gewöhnlichen Differentialgleichung bekommen kann, die bereits
Euler studiert hat.

\subsection{Die Eulersche hypergeometrische Differentialgleichung
\label{buch:differentialgleichung:subsection:euler-hypergeometrisch}}
Die hypergeometrische Funktion $\mathstrut_2F_1(a,b;c;x)$ ist eine
Lösung der {\em Eulerschen hypergeometrischen Differentialgleichung}
(zu unterscheiden von der Eulerschen Differentialgleichung, die sich
immer auf eine lineare Differentialgleichung mit konstanten Koeffizienten
reduzieren lässt)
\begin{equation}
x(1-x) \frac{d^2y}{dx^2} + (c-(a+b+1)x)\frac{dy}{dx} - ab y = 0
\label{buch:differentialgleichungen:hypergeo:eulerdgl}
\end{equation}
Wir prüfen dies nach, indem wir die Definition der hypergeometrischen
Funktion 
\begin{align*}
y(x)
&=
\mathstrut_2F_1(a,b;c;x)
=
\sum_{k=0}^\infty
\frac{(a)_k(b)_k}{(c)_k} \frac{x^k}{k!}
\intertext{mit den Ableitungen}
y'(x)
&=
\sum_{k=1}^\infty 
\frac{(a)_k(b)_k}{(c)_k} \frac{x^{k-1}}{(k-1)!}
\\
y''(x)
&=
\sum_{k=2}^\infty 
\frac{(a)_k(b)_k}{(c)_k} \frac{x^{k-2}}{(k-2)!}
\end{align*}
einsetzen.
Die Gleichung, die sich ergibt, ist
\begin{align*}
0
&=
x(1-x)
\sum_{k=2}^\infty
\frac{(a)_k(b)_k}{(c)_k}\frac{x^{k-2}}{(k-2)!}
+
(c-(a+b+1)x)
\sum_{k=1}^\infty
\frac{(a)_k(b)_k}{(c)_k}\frac{x^{k-1}}{(k-1)!}
-ab
\sum_{k=0}^\infty
\frac{(a)_k(b)_k}{(c)_k} \frac{x^k}{k!}
\\
&=
\sum_{k=2}^\infty
\frac{(a)_k(b)_k}{(c)_k}\frac{x^{k-1}}{(k-2)!}
-
\sum_{k=2}^\infty
\frac{(a)_k(b)_k}{(c)_k}\frac{x^k}{(k-2)!}
+
c\sum_{k=1}^\infty
\frac{(a)_k(b)_k}{(c)_k}\frac{x^{k-1}}{(k-1)!}
\\
&\qquad
-(a+b+1)
\sum_{k=1}^\infty
\frac{(a)_k(b)_k}{(c)_k}\frac{x^k}{(k-1)!}
-ab
\sum_{k=0}^\infty
\frac{(a)_k(b)_k}{(c)_k} \frac{x^k}{k!}
\\
&=
\sum_{k=1}^\infty
\frac{(a)_{k+1}(b)_{k+1}}{(c)_{k+1}}\frac{x^k}{(k-1)!}
-
\sum_{k=2}^\infty
\frac{(a)_k(b)_k}{(c)_k}\frac{x^k}{(k-2)!}
+
c\sum_{k=0}^\infty
\frac{(a)_{k+1}(b)_{k+1}}{(c)_{k+1}}\frac{x^k}{k!}
\\
&\qquad
-(a+b+1)
\sum_{k=1}^\infty
\frac{(a)_k(b)_k}{(c)_k}\frac{x^k}{(k-1)!}
-ab
\sum_{k=0}^\infty
\frac{(a)_k(b)_k}{(c)_k} \frac{x^k}{k!}.
\end{align*}
Zum konstanten Koeffizienten für $k=0$ tragen nur die dritte und letzte
Summe bei, dies sind die Terme
\[
c\frac{(a)_1(b)_1}{(c)_1}-ab\frac{(a)_0(b)_0}{(c)_0}
=
c\frac{ab}{c}-ab\frac{1\cdot 1}{1}
=
0.
\]
Für den linearen Term $k=1$ kommen je ein Term aus der ersten aund vierten
Summe hinzu, dies ergibt
\begin{align*}
&\phantom{\mathstrut=\mathstrut}
\frac{(a)_2(b)_2}{(c)_2}
+c\frac{(a)_2(b)_2}{(c)_2}
-(a+b+1)\frac{(a)_1(b)_1}{(c)_1}
-ab\frac{(a)_1(b)_1}{(c)_1}
\\
&=
\frac{a(a+1)b(b+1)}{c(c+1)}
(1+c)
-(ab+a+b+1)
\frac{ab}{c}
\\
&=
\frac{a(a+1)b(b+1)}{c}
-
(a+1)(b+1)\frac{ab}{c}
=0.
\end{align*}
Durch Koeffizientenvergleich erhalten wir für $k\ge 2$ 
\begin{align*}
0
&=
\frac{(a)_{k+1}(b)_{k+1}}{(c)_{k+1}} \frac1{(k-1)!} 
-
\frac{(a)_k(b)_k}{(c)_k} \frac1{(k-2)!} 
+
c\frac{(a)_{k+1}(b)_{k+1}}{(c)_{k+1}} \frac{1}{k!}
\\
&\qquad
-(a+b+1)\frac{(a)_k(b)_k}{(c)_k}\frac{1}{(k-1)!}
-ab \frac{(a)_k(b)_k}{(c)_k}\frac{1}{k!}
\\
&=
\frac{(a)_k(b)_k}{(c)_{k+1}}
\frac{1}{k!}
\biggl(
(a+k)(b+k)k
-(c+k)(k-1)k
+
c(a+k)(b+k)
\\
&\qquad
\qquad
\qquad
-(a+b+1)(c+k)k
-ab(c+k)
\biggr).
\intertext{Der zweite, vierte und fünfte Term können zu}
&=
\frac{(a)_k(b)_k}{(c)_{k+1}}
\frac{1}{k!}
\biggl(
(a+k)(b+k)k
+
c(a+k)(b+k)
-(ab+ak+bk+k^2)(c+k)
\biggr)
\intertext{zusammengefasst werden.
Der Faktor $(ab+ak+bk+k^2)$ kann als Produkt $(a+k)(b+k)$ faktorisiert
werden, der dann als gemeinsamer Faktor aus allen Termen ausgeklammert
werden kann:}
&=
\frac{(a)_k(b)_k}{(c)_{k+1}}
\frac{1}{k!}
\biggl(
(a+k)(b+k)k
+
c(a+k)(b+k)
-(a+k)(b+k)(c+k)
\biggr)
\\
&=
\frac{(a)_{k+1}(b)_{k+1}}{(c)_{k+1}}
\frac{1}{k!}
\biggl(
k
+
c
-(c+k)
\biggr)
=0.
\end{align*}
Damit ist gezeigt, dass $\mathstrut_2F_1(a,b;c;x)$ eine Lösung
der Differentialgleichung ist.

Die hypergeometrische Reihe kann auch direkt mit Hilfe der
Potenzreihenmethode als Lösung der Differentialgleichung gefunden 
werden.

\subsection{Lösung als verallgemeinerte Potenzreihe}
Da die hypergeometrische Reihe eine Differentialgleichung
zweiter Ordnung mit einer Singularität bei $x=0$ ist, 
kann man versuchen eine zweite, linear unabhängige Lösung mit
Hilfe der Methode der verallgemeinerten Potenzreihen zu finden.
Dazu setzt man die Lösung in der Form
\begin{align*}
y_2(x)
&=
\sum_{k=0}^\infty a_kx^{\varrho+k}
&
&\Rightarrow&
y_2'(x)
&=
\sum_{k=0}^\infty (\varrho+k)a_kx^{\varrho+k-1}
\\
&&
&&
y_2''(x)
&=
\sum_{k=0}^\infty (\varrho+k)(\varrho+k-1)a_kx^{\varrho+k-2}
\end{align*}
an, wobei $a_0\ne 0$ sein soll.
Einsetzen in die Differentialgleichung ergibt
\begin{align*}
0&=
x(1-x)y_2''(x) + (c-(a+b+1)x) y_2'(x) -aby_2(x)
\\
&=
x(1-x)
\sum_{k=0}^\infty (\varrho+k)(\varrho+k-1)a_kx^{\varrho+k-2}
+
(c-(a+b+1)x)
\sum_{k=0}^\infty (\varrho+k)a_kx^{\varrho+k-1}
-
abx^{\varrho}\sum_{k=0}^\infty a_kx^{\varrho+k}
\\
&=
-\sum_{k=0}^\infty (\varrho+k)(\varrho+k-1)a_kx^{\varrho+k}
+
\sum_{k=0}^\infty (\varrho+k)(\varrho+k-1)a_kx^{\varrho+k-1}
+
c
\sum_{k=0}^\infty (\varrho+k)a_kx^{\varrho+k-1}
\\
&\qquad
-
(a+b+1)
\sum_{k=0}^\infty (\varrho+k)a_kx^{\varrho+k}
-
ab
\sum_{k=0}^\infty a_kx^{\varrho+k}.
\intertext{Durch Verschiebung des Summationsindex in der zweiten
und dritten Summe wird der Koeffizientenvergleich etwas
einfacher}
&=
-\sum_{k=0}^\infty (\varrho+k)(\varrho+k-1)a_kx^{\varrho+k}
+
\sum_{k=-1}^\infty (\varrho+k+1)(\varrho+k)a_{k+1}x^{\varrho+k}
+
c
\sum_{k=-1}^\infty (\varrho+k+1)a_{k+1}x^{\varrho+k}
\\
&\qquad
-
(a+b+1)
\sum_{k=0}^\infty (\varrho+k)a_kx^{\varrho+k}
-
ab
\sum_{k=0}^\infty a_kx^{\varrho+k}
\\
&=
-\sum_{k=0}^\infty (\varrho+k)(\varrho+k-1)a_kx^{\varrho+k}
+
\sum_{k=-1}^\infty (\varrho+k+1)(\varrho+k+c)a_{k+1}x^{\varrho+k}
\\
&\qquad
-
\sum_{k=0}^\infty ((\varrho+k)(a+b+1)+ab)a_kx^{\varrho+k}
\\
&=
\bigl(
\varrho(\varrho-1)
+c\varrho \bigr)
x^{\varrho-1}
+
\sum_{k=0}^\infty
\bigl(
-(\varrho+k)(\varrho+k-1)a_k
+(\varrho+k+1)(\varrho+k+c)a_{k+1}
\\
&
\qquad
\qquad
\qquad
\qquad
\qquad
\qquad
-((\varrho+k)(a+b+1)+ab)a_k
\bigr)
x^{\varrho+k}.
\end{align*}
Aus dem ersten Term kann man die Indexgleichung
\[
0
=
\varrho(\varrho-1)+c\varrho
=
\varrho(\varrho-1+c)
\]
ablesen, die die Nullstellen $\varrho=0$ und $\varrho=1-c$ hat.
Die Nullstelle $\varrho=0$ ergibt natürlich die bereits gefundene
hypergeometrische Reihe.

Nach Einsetzen der zweiten Lösung der Indexgleichung in der Summe
legt der Koeffizientenvergleich eine Beziehung
\begin{align}
0
&=
\bigl(
-(k-c+1)(k-c)
-(k-c+1)(a+b+1)+ab
\bigr)a_k
+
(k-c+2)(k+1)
a_{k+1} 
\notag
\intertext{zwischen $a_k$ und $a_{k+1}$ fest.
Daraus kann man den Quotienten aufeinanderfolgender
Koeffizienten als}
\frac{a_{k+1}}{a_k}
&=
\frac{
-(k-c+1)(k-c)
-(k-c+1)(a+b+1)+ab
}{
\notag
(k-c+2)(k+1)
}
\\
&=
%(%i4) factor(coeff(y,q,0))
%(%o4)                  - (k - c + a + 1) (k - c + b + 1)
%(%i5) factor(coeff(y,q,1))
%(%o5)                         (k + 1) (k - c + 2)
\frac{
(a-c+1+k)
(b-c+1+k)
}{
(2-c+k)(k+1)
}
\label{buch:differentialgleichungen:hypergeo:verallgkoef}
\end{align}
finden.
Setzt man $a_0=1$, ist die zweite Lösung ist also wieder eine
hypergeometrische Funktion.%, nämlich
%\[
%y_2(x)
%=
%x^{1-c}
%\sum_{k=0}^\infty \frac{(a-c+1)_k(b-c+1)_k}{(2-c)_k}\frac{x^k}{k!}
%=
%x^{1-c}
%\mathstrut_2F_1\biggl(\begin{matrix}a-c+1,b-c+1\\2-c\end{matrix};x\biggr)
%\]
Diese Lösung ist aber nur möglich, wenn in
\eqref{buch:differentialgleichungen:hypergeo:verallgkoef}
der Nenner nicht verschwindet, d.~h.~$2-c+k\ne 0$
oder $c \ne k+2$ für all natürlichen $k$.
$c$ darf also kein natürliche Zahl $\ge 2$ sein.
Wir fassen die Resultate dieses Abschnitts im folgenden Satz zusammen.

\begin{satz}
Die eulersche hypergeometrische Differentialgleichung
\begin{equation}
x(1-x)\frac{d^2y}{dx^2}
+(c+(a+b+1)x)\frac{dy}{dx}
-ab y
=
0
\end{equation}
hat die Lösung
\[
y_1(x)
=
\mathstrut_2F_1\biggl(\begin{matrix}a,b\\c\end{matrix};x\biggr).
\]
Falls $c-2\not\in \mathbb{N}$ ist, ist
\[
y_2(x)
=
x^{1-c} \mathstrut_2F_1\biggl(\begin{matrix}a-c+1,b-c+1\\2-c\end{matrix};x\biggr)
\]
eine zweite, linear unabhängige Lösung.
\end{satz}

%
% Die verallgemeinerte hypergeometrische Differentialgleichung
%
\subsection{Verallgemeinerte hypergeometrische Differentialgleichung}
% https://de.wikipedia.org/wiki/Verallgemeinerte_hypergeometrische_Funktion







\section*{Übungsaufgaben}
\rhead{Übungsaufgaben}
\aufgabetoplevel{chapters/040-rekursion/uebungsaufgaben}
\begin{uebungsaufgaben}
%\uebungsaufgabe{0}
\uebungsaufgabe{1}
\uebungsaufgabe{2}
\end{uebungsaufgaben}



% algebraisch und geometrisch definierte spezielle Funktionen
%%
% chapter.tex -- Beschreibung des Inhaltes
%
% (c) 2021 Prof Dr Andreas Müller, Hochschule Rapperswil
%
% !TeX spellcheck = de_CH
\chapter{Spezielle Funktionen und Rekursion
\label{buch:chapter:rekursion}}
\lhead{Spezielle Funktionen und Rekursion}
\rhead{}

%
% gamma.tex -- Abschnitt über die Gamma-funktion
%
% (c) 2021 Prof Dr Andreas Müller, OST Ostschweizer Fachhochschule
%
\section{Die Gamma-Funktion
\label{buch:rekursion:section:gamma}}
Die Fakultät $x!$ kann rekursiv durch 
\[
	x! = x\cdot (x-1)! \qquad\text{und}\qquad 0!=1
\]
für alle natürlichen Zahlen $x\in\mathbb{N}$ definiert werden.
Äquivalent damit ist eine Funktion 
\begin{equation}
\Gamma(x+1) = x\Gamma(x)
\qquad\text{und}\qquad 
\Gamma(1)=1.
\label{buch:rekursion:eqn:gammadef}
\end{equation}
Kann man eine reelle oder komplexe Funktion finden, die die
Funktionalgleichung~\eqref{buch:rekursion:eqn:gammadef}
erfüllt und damit die Fakultät auf beliebige Argumente ausdehnt?

\subsection{Integralformel für die Gamma-Funktion}
Euler hat die folgende Integraldefinition der Gamma-Funktion gegeben.

\begin{definition}
\label{buch:rekursion:def:gamma}
Die Gamma-Funktion ist die Funktion 
\[
\Gamma
\colon
\{z\in\mathbb{C} \mid \operatorname{Re}z>0\}
\to \mathbb{C}
:
z
\mapsto
\Gamma(z) = \int_0^\infty t^{x-1}e^{-t}\,dt
\]
\end{definition}

Man beachte, dass das Integral für $x=0$ nicht definiert ist, eine
Potenzreihenentwicklung um einen Punkt $x_0$ auf der positiven reellen
Achse kann also höchstens den Konvergenzradius $\varrho=|x_0|$ haben.

\begin{figure}
\centering
\includegraphics{chapters/040-rekursion/images/gammaplot.pdf}
\caption{Graph der Gamma-Funktion $z\mapsto\Gamma(z)$ und der alternativen
Funktion $\Gamma(z)+\sin(\pi z)$, die für ganzzahlige Argumente ebenfalls
die Werte der Fakultät annimmt.
\label{buch:rekursion:fig:gamma}}
\end{figure}

\subsubsection{Alternative Lösungen}
Die Funktion $\Gamma(z)$ ist nicht die einzige Funktion, die natürlichen
Zahlen die Werte $\Gamma(n+1) = n!$ der Fakultät annimmt.
Indem man eine beliebige Funktion $f(z)$ addiert, die auf alle
natürlichen Zahlen verschwindet, also $f(n)=0$ für $n\in\mathbb{N}$,
erhält man eine weitere Funktion, die auf natürlichen Zahlen
die Werte der Fakultät annimmt.
Ein Beispiel einer solchen Funktion ist
\begin{equation}
z\mapsto f(z)=\Gamma(z) + \sin \pi z,
\label{buch:rekursion:eqn:gammaalternative}
\end{equation}
die Funktion $f(z)=\sin\pi z$ verschwindet sogar auf allen ganzen
Zahlen.

In Abbildung~\ref{buch:rekursion:fig:gamma} ist die Gamma-Funktion
in rot geplotet, die Funktion~\eqref{buch:rekursion:eqn:gammaalternative}
in grün.
Die Punkte $(n,(n-1)!)$ sind in blau bezeichnet, sie sind beiden Graphen
gemeinsam.

\subsubsection{Pol erster Ordnung bei $z=0$}
Wir haben zu prüfen, dass sowohl der Wert $\Gamma(1)$ korrekt ist als
auch die Rekursionsformel~\eqref{buch:rekursion:eqn:gammadef} gilt.
Der Wert für $z=1$ ist
\begin{align*}
\Gamma(1)
&=
\int_0^\infty t^{1-1}e^{-t}\,dt
=
\left[ -e^{-t} \right]_0^\infty
=
1.
\end{align*}
Für die Rekursionsformel kann mit Hilfe von partieller Integration
bekommen:
\begin{align*}
\Gamma(z+1)
&=
\int_0^\infty t^{z+1-1}e^{-t}\,dt
=
\biggl[-t^{z}e^{-t}\biggr]_0^\infty
+
\int_0^\infty z t^{z-1}e^{-t}\,dt
\\
&=
z
\int_0^\infty
t^{z-1}e^{-t}\,dt
=
z \Gamma(z).
\end{align*}

Für $0<z<\varepsilon$ für eine $\varepsilon >0$ folgt aus der 
Funktionalgleichung
\[
\Gamma(z) = \frac{\Gamma(1+z)}{z}.
\]
Da $\Gamma(1)=1$ ist und $\Gamma$ eine in einer
Umgebung von $1$ stetige Funktion ist, kann sie in der Form
\(
\Gamma(1+z)=\Gamma(1) + zf(z)
\)
schreiben, wobei  $f(z)$ eine differenzierbare Funktion ist mit
$f'(1)=\Gamma'(1)$.
Daraus ergibt sich für $\Gamma(z)$ der Ausdruck
\[
\Gamma(z) = \frac{\Gamma(1)}{z} + f(z) = \frac{1}{z} + f(z).
\]
Die Gamma-Funktion hat daher and er Stelle $z=0$ einen Pol erster Ordnung.

\subsubsection{Ausdehnung auf $\operatorname{Re}z<0$}
Die Integralformel konvergiert nicht für $\operatorname{Re}z\le 0$.
Durch analytische Fortsetzung, wie sie im
Abschnitt~\ref{buch:funktionentheorie:section:fortsetzung}
beschrieben wird, kann die Funktion auf ganz $\mathbb{C}$ ausgedehnt
werden, mit Ausnahme einzelner Pole.
Die Funktionalgleichung gilt natürlich für alle $z\in\mathbb{C}$,
für die $\Gamma(z)$ definiert ist.
In einer Umgebung von $z=-n$ gilt
\[
\Gamma(z)
=
\frac{\Gamma(z+1)}{z}
=
\frac{\Gamma(z+2)}{z(z+1)}
=
\frac{\Gamma(z+3)}{z(z+1)(z+2)}
=
\dots
=
\frac{\Gamma(z+n)}{z(z+1)(z+2)\cdots(z+n-1)}
\]
Keiner der Faktoren im Nenner verschwindet in der Nähe von $z=-n$, der
Zähler hat aber einen Pol erster Ordnung an dieser Stelle.
Daher hat auch der Quotient einen Pol erster Ordnung.
Abbildung~\ref{buch:rekursion:fig:gamma} zeigt die Pole bei den
nicht negativen ganzen Zahlen.






%
% linear.tex
%
% (c) 2021 Prof Dr Andreas Müller, OST Ostschweizer Fachhochschule
%
\section{Lineare Rekursionsgleichung mit konstanten Koeffizienten
\label{buch:rekursion:section:linear}}
\rhead{Lineare Rekursionsgleichungen}
Die Funktionalgleichung der Gamma-Funktion, die im
Abschnitt~\ref{buch:rekursion:section:gamma} untersucht wurde,
hat die Form einer linearen Rekursionsgleichung
\[
\Gamma(x+1) = x\Gamma(x),\qquad \Gamma(1) = 1.
\]
Gleichungen, die Werte einer Funktion für verschiedene
Argument in Beziehung setzen, heissen {\em Funktionalgleichungen}.
\index{Funktionalgleichung}%
Es war überraschend schwierig, eine Lösung für Funktionalgleichung
der Gamma-Funktion für beliebige komplexe $x$ zu finden.
In diesem Abschnitt soll daher eine Klasse von Rekursionsgleichungen
näher untersucht werden, für die einfache Lösungen möglich sind.

\subsection{Lineare Differenzengleichungen}

\subsection{Lösung mit Polynomfunktionen}







%
% hypergeometrisch.tex
%
% (c) 2021 Prof Dr Andreas Müller, OST Ostschweizer Fachhochschule
%
\section{Hypergeometrische Differentialgleichung
\label{buch:differentialgleichungen:section:hypergeometrisch}}
Die hypergeometrische Funktion $\mathstrut_2F1(a,b;c;x)$ wurde in
Abschnitt~\ref{buch:rekursion:section:hypergeometrische-funktion}
als Potenzreihe mit sehr speziellen Koeffizienten, die sich aus
Pochhammer-Symbolen.
Es stellt sich aber heraus, dass man sie auch als Lösung einer
gewöhnlichen Differentialgleichung bekommen kann, die bereits
Euler studiert hat.

\subsection{Die Eulersche hypergeometrische Differentialgleichung
\label{buch:differentialgleichung:subsection:euler-hypergeometrisch}}
Die hypergeometrische Funktion $\mathstrut_2F_1(a,b;c;x)$ ist eine
Lösung der {\em Eulerschen hypergeometrischen Differentialgleichung}
(zu unterscheiden von der Eulerschen Differentialgleichung, die sich
immer auf eine lineare Differentialgleichung mit konstanten Koeffizienten
reduzieren lässt)
\begin{equation}
x(1-x) \frac{d^2y}{dx^2} + (c-(a+b+1)x)\frac{dy}{dx} - ab y = 0
\label{buch:differentialgleichungen:hypergeo:eulerdgl}
\end{equation}
Wir prüfen dies nach, indem wir die Definition der hypergeometrischen
Funktion 
\begin{align*}
y(x)
&=
\mathstrut_2F_1(a,b;c;x)
=
\sum_{k=0}^\infty
\frac{(a)_k(b)_k}{(c)_k} \frac{x^k}{k!}
\intertext{mit den Ableitungen}
y'(x)
&=
\sum_{k=1}^\infty 
\frac{(a)_k(b)_k}{(c)_k} \frac{x^{k-1}}{(k-1)!}
\\
y''(x)
&=
\sum_{k=2}^\infty 
\frac{(a)_k(b)_k}{(c)_k} \frac{x^{k-2}}{(k-2)!}
\end{align*}
einsetzen.
Die Gleichung, die sich ergibt, ist
\begin{align*}
0
&=
x(1-x)
\sum_{k=2}^\infty
\frac{(a)_k(b)_k}{(c)_k}\frac{x^{k-2}}{(k-2)!}
+
(c-(a+b+1)x)
\sum_{k=1}^\infty
\frac{(a)_k(b)_k}{(c)_k}\frac{x^{k-1}}{(k-1)!}
-ab
\sum_{k=0}^\infty
\frac{(a)_k(b)_k}{(c)_k} \frac{x^k}{k!}
\\
&=
\sum_{k=2}^\infty
\frac{(a)_k(b)_k}{(c)_k}\frac{x^{k-1}}{(k-2)!}
-
\sum_{k=2}^\infty
\frac{(a)_k(b)_k}{(c)_k}\frac{x^k}{(k-2)!}
+
c\sum_{k=1}^\infty
\frac{(a)_k(b)_k}{(c)_k}\frac{x^{k-1}}{(k-1)!}
\\
&\qquad
-(a+b+1)
\sum_{k=1}^\infty
\frac{(a)_k(b)_k}{(c)_k}\frac{x^k}{(k-1)!}
-ab
\sum_{k=0}^\infty
\frac{(a)_k(b)_k}{(c)_k} \frac{x^k}{k!}
\\
&=
\sum_{k=1}^\infty
\frac{(a)_{k+1}(b)_{k+1}}{(c)_{k+1}}\frac{x^k}{(k-1)!}
-
\sum_{k=2}^\infty
\frac{(a)_k(b)_k}{(c)_k}\frac{x^k}{(k-2)!}
+
c\sum_{k=0}^\infty
\frac{(a)_{k+1}(b)_{k+1}}{(c)_{k+1}}\frac{x^k}{k!}
\\
&\qquad
-(a+b+1)
\sum_{k=1}^\infty
\frac{(a)_k(b)_k}{(c)_k}\frac{x^k}{(k-1)!}
-ab
\sum_{k=0}^\infty
\frac{(a)_k(b)_k}{(c)_k} \frac{x^k}{k!}.
\end{align*}
Zum konstanten Koeffizienten für $k=0$ tragen nur die dritte und letzte
Summe bei, dies sind die Terme
\[
c\frac{(a)_1(b)_1}{(c)_1}-ab\frac{(a)_0(b)_0}{(c)_0}
=
c\frac{ab}{c}-ab\frac{1\cdot 1}{1}
=
0.
\]
Für den linearen Term $k=1$ kommen je ein Term aus der ersten aund vierten
Summe hinzu, dies ergibt
\begin{align*}
&\phantom{\mathstrut=\mathstrut}
\frac{(a)_2(b)_2}{(c)_2}
+c\frac{(a)_2(b)_2}{(c)_2}
-(a+b+1)\frac{(a)_1(b)_1}{(c)_1}
-ab\frac{(a)_1(b)_1}{(c)_1}
\\
&=
\frac{a(a+1)b(b+1)}{c(c+1)}
(1+c)
-(ab+a+b+1)
\frac{ab}{c}
\\
&=
\frac{a(a+1)b(b+1)}{c}
-
(a+1)(b+1)\frac{ab}{c}
=0.
\end{align*}
Durch Koeffizientenvergleich erhalten wir für $k\ge 2$ 
\begin{align*}
0
&=
\frac{(a)_{k+1}(b)_{k+1}}{(c)_{k+1}} \frac1{(k-1)!} 
-
\frac{(a)_k(b)_k}{(c)_k} \frac1{(k-2)!} 
+
c\frac{(a)_{k+1}(b)_{k+1}}{(c)_{k+1}} \frac{1}{k!}
\\
&\qquad
-(a+b+1)\frac{(a)_k(b)_k}{(c)_k}\frac{1}{(k-1)!}
-ab \frac{(a)_k(b)_k}{(c)_k}\frac{1}{k!}
\\
&=
\frac{(a)_k(b)_k}{(c)_{k+1}}
\frac{1}{k!}
\biggl(
(a+k)(b+k)k
-(c+k)(k-1)k
+
c(a+k)(b+k)
\\
&\qquad
\qquad
\qquad
-(a+b+1)(c+k)k
-ab(c+k)
\biggr).
\intertext{Der zweite, vierte und fünfte Term können zu}
&=
\frac{(a)_k(b)_k}{(c)_{k+1}}
\frac{1}{k!}
\biggl(
(a+k)(b+k)k
+
c(a+k)(b+k)
-(ab+ak+bk+k^2)(c+k)
\biggr)
\intertext{zusammengefasst werden.
Der Faktor $(ab+ak+bk+k^2)$ kann als Produkt $(a+k)(b+k)$ faktorisiert
werden, der dann als gemeinsamer Faktor aus allen Termen ausgeklammert
werden kann:}
&=
\frac{(a)_k(b)_k}{(c)_{k+1}}
\frac{1}{k!}
\biggl(
(a+k)(b+k)k
+
c(a+k)(b+k)
-(a+k)(b+k)(c+k)
\biggr)
\\
&=
\frac{(a)_{k+1}(b)_{k+1}}{(c)_{k+1}}
\frac{1}{k!}
\biggl(
k
+
c
-(c+k)
\biggr)
=0.
\end{align*}
Damit ist gezeigt, dass $\mathstrut_2F_1(a,b;c;x)$ eine Lösung
der Differentialgleichung ist.

Die hypergeometrische Reihe kann auch direkt mit Hilfe der
Potenzreihenmethode als Lösung der Differentialgleichung gefunden 
werden.

\subsection{Lösung als verallgemeinerte Potenzreihe}
Da die hypergeometrische Reihe eine Differentialgleichung
zweiter Ordnung mit einer Singularität bei $x=0$ ist, 
kann man versuchen eine zweite, linear unabhängige Lösung mit
Hilfe der Methode der verallgemeinerten Potenzreihen zu finden.
Dazu setzt man die Lösung in der Form
\begin{align*}
y_2(x)
&=
\sum_{k=0}^\infty a_kx^{\varrho+k}
&
&\Rightarrow&
y_2'(x)
&=
\sum_{k=0}^\infty (\varrho+k)a_kx^{\varrho+k-1}
\\
&&
&&
y_2''(x)
&=
\sum_{k=0}^\infty (\varrho+k)(\varrho+k-1)a_kx^{\varrho+k-2}
\end{align*}
an, wobei $a_0\ne 0$ sein soll.
Einsetzen in die Differentialgleichung ergibt
\begin{align*}
0&=
x(1-x)y_2''(x) + (c-(a+b+1)x) y_2'(x) -aby_2(x)
\\
&=
x(1-x)
\sum_{k=0}^\infty (\varrho+k)(\varrho+k-1)a_kx^{\varrho+k-2}
+
(c-(a+b+1)x)
\sum_{k=0}^\infty (\varrho+k)a_kx^{\varrho+k-1}
-
abx^{\varrho}\sum_{k=0}^\infty a_kx^{\varrho+k}
\\
&=
-\sum_{k=0}^\infty (\varrho+k)(\varrho+k-1)a_kx^{\varrho+k}
+
\sum_{k=0}^\infty (\varrho+k)(\varrho+k-1)a_kx^{\varrho+k-1}
+
c
\sum_{k=0}^\infty (\varrho+k)a_kx^{\varrho+k-1}
\\
&\qquad
-
(a+b+1)
\sum_{k=0}^\infty (\varrho+k)a_kx^{\varrho+k}
-
ab
\sum_{k=0}^\infty a_kx^{\varrho+k}.
\intertext{Durch Verschiebung des Summationsindex in der zweiten
und dritten Summe wird der Koeffizientenvergleich etwas
einfacher}
&=
-\sum_{k=0}^\infty (\varrho+k)(\varrho+k-1)a_kx^{\varrho+k}
+
\sum_{k=-1}^\infty (\varrho+k+1)(\varrho+k)a_{k+1}x^{\varrho+k}
+
c
\sum_{k=-1}^\infty (\varrho+k+1)a_{k+1}x^{\varrho+k}
\\
&\qquad
-
(a+b+1)
\sum_{k=0}^\infty (\varrho+k)a_kx^{\varrho+k}
-
ab
\sum_{k=0}^\infty a_kx^{\varrho+k}
\\
&=
-\sum_{k=0}^\infty (\varrho+k)(\varrho+k-1)a_kx^{\varrho+k}
+
\sum_{k=-1}^\infty (\varrho+k+1)(\varrho+k+c)a_{k+1}x^{\varrho+k}
\\
&\qquad
-
\sum_{k=0}^\infty ((\varrho+k)(a+b+1)+ab)a_kx^{\varrho+k}
\\
&=
\bigl(
\varrho(\varrho-1)
+c\varrho \bigr)
x^{\varrho-1}
+
\sum_{k=0}^\infty
\bigl(
-(\varrho+k)(\varrho+k-1)a_k
+(\varrho+k+1)(\varrho+k+c)a_{k+1}
\\
&
\qquad
\qquad
\qquad
\qquad
\qquad
\qquad
-((\varrho+k)(a+b+1)+ab)a_k
\bigr)
x^{\varrho+k}.
\end{align*}
Aus dem ersten Term kann man die Indexgleichung
\[
0
=
\varrho(\varrho-1)+c\varrho
=
\varrho(\varrho-1+c)
\]
ablesen, die die Nullstellen $\varrho=0$ und $\varrho=1-c$ hat.
Die Nullstelle $\varrho=0$ ergibt natürlich die bereits gefundene
hypergeometrische Reihe.

Nach Einsetzen der zweiten Lösung der Indexgleichung in der Summe
legt der Koeffizientenvergleich eine Beziehung
\begin{align}
0
&=
\bigl(
-(k-c+1)(k-c)
-(k-c+1)(a+b+1)+ab
\bigr)a_k
+
(k-c+2)(k+1)
a_{k+1} 
\notag
\intertext{zwischen $a_k$ und $a_{k+1}$ fest.
Daraus kann man den Quotienten aufeinanderfolgender
Koeffizienten als}
\frac{a_{k+1}}{a_k}
&=
\frac{
-(k-c+1)(k-c)
-(k-c+1)(a+b+1)+ab
}{
\notag
(k-c+2)(k+1)
}
\\
&=
%(%i4) factor(coeff(y,q,0))
%(%o4)                  - (k - c + a + 1) (k - c + b + 1)
%(%i5) factor(coeff(y,q,1))
%(%o5)                         (k + 1) (k - c + 2)
\frac{
(a-c+1+k)
(b-c+1+k)
}{
(2-c+k)(k+1)
}
\label{buch:differentialgleichungen:hypergeo:verallgkoef}
\end{align}
finden.
Setzt man $a_0=1$, ist die zweite Lösung ist also wieder eine
hypergeometrische Funktion.%, nämlich
%\[
%y_2(x)
%=
%x^{1-c}
%\sum_{k=0}^\infty \frac{(a-c+1)_k(b-c+1)_k}{(2-c)_k}\frac{x^k}{k!}
%=
%x^{1-c}
%\mathstrut_2F_1\biggl(\begin{matrix}a-c+1,b-c+1\\2-c\end{matrix};x\biggr)
%\]
Diese Lösung ist aber nur möglich, wenn in
\eqref{buch:differentialgleichungen:hypergeo:verallgkoef}
der Nenner nicht verschwindet, d.~h.~$2-c+k\ne 0$
oder $c \ne k+2$ für all natürlichen $k$.
$c$ darf also kein natürliche Zahl $\ge 2$ sein.
Wir fassen die Resultate dieses Abschnitts im folgenden Satz zusammen.

\begin{satz}
Die eulersche hypergeometrische Differentialgleichung
\begin{equation}
x(1-x)\frac{d^2y}{dx^2}
+(c+(a+b+1)x)\frac{dy}{dx}
-ab y
=
0
\end{equation}
hat die Lösung
\[
y_1(x)
=
\mathstrut_2F_1\biggl(\begin{matrix}a,b\\c\end{matrix};x\biggr).
\]
Falls $c-2\not\in \mathbb{N}$ ist, ist
\[
y_2(x)
=
x^{1-c} \mathstrut_2F_1\biggl(\begin{matrix}a-c+1,b-c+1\\2-c\end{matrix};x\biggr)
\]
eine zweite, linear unabhängige Lösung.
\end{satz}

%
% Die verallgemeinerte hypergeometrische Differentialgleichung
%
\subsection{Verallgemeinerte hypergeometrische Differentialgleichung}
% https://de.wikipedia.org/wiki/Verallgemeinerte_hypergeometrische_Funktion







\section*{Übungsaufgaben}
\rhead{Übungsaufgaben}
\aufgabetoplevel{chapters/040-rekursion/uebungsaufgaben}
\begin{uebungsaufgaben}
%\uebungsaufgabe{0}
\uebungsaufgabe{1}
\uebungsaufgabe{2}
\end{uebungsaufgaben}


%%
% chapter.tex -- Beschreibung des Inhaltes
%
% (c) 2021 Prof Dr Andreas Müller, Hochschule Rapperswil
%
% !TeX spellcheck = de_CH
\chapter{Spezielle Funktionen und Rekursion
\label{buch:chapter:rekursion}}
\lhead{Spezielle Funktionen und Rekursion}
\rhead{}

%
% gamma.tex -- Abschnitt über die Gamma-funktion
%
% (c) 2021 Prof Dr Andreas Müller, OST Ostschweizer Fachhochschule
%
\section{Die Gamma-Funktion
\label{buch:rekursion:section:gamma}}
Die Fakultät $x!$ kann rekursiv durch 
\[
	x! = x\cdot (x-1)! \qquad\text{und}\qquad 0!=1
\]
für alle natürlichen Zahlen $x\in\mathbb{N}$ definiert werden.
Äquivalent damit ist eine Funktion 
\begin{equation}
\Gamma(x+1) = x\Gamma(x)
\qquad\text{und}\qquad 
\Gamma(1)=1.
\label{buch:rekursion:eqn:gammadef}
\end{equation}
Kann man eine reelle oder komplexe Funktion finden, die die
Funktionalgleichung~\eqref{buch:rekursion:eqn:gammadef}
erfüllt und damit die Fakultät auf beliebige Argumente ausdehnt?

\subsection{Integralformel für die Gamma-Funktion}
Euler hat die folgende Integraldefinition der Gamma-Funktion gegeben.

\begin{definition}
\label{buch:rekursion:def:gamma}
Die Gamma-Funktion ist die Funktion 
\[
\Gamma
\colon
\{z\in\mathbb{C} \mid \operatorname{Re}z>0\}
\to \mathbb{C}
:
z
\mapsto
\Gamma(z) = \int_0^\infty t^{x-1}e^{-t}\,dt
\]
\end{definition}

Man beachte, dass das Integral für $x=0$ nicht definiert ist, eine
Potenzreihenentwicklung um einen Punkt $x_0$ auf der positiven reellen
Achse kann also höchstens den Konvergenzradius $\varrho=|x_0|$ haben.

\begin{figure}
\centering
\includegraphics{chapters/040-rekursion/images/gammaplot.pdf}
\caption{Graph der Gamma-Funktion $z\mapsto\Gamma(z)$ und der alternativen
Funktion $\Gamma(z)+\sin(\pi z)$, die für ganzzahlige Argumente ebenfalls
die Werte der Fakultät annimmt.
\label{buch:rekursion:fig:gamma}}
\end{figure}

\subsubsection{Alternative Lösungen}
Die Funktion $\Gamma(z)$ ist nicht die einzige Funktion, die natürlichen
Zahlen die Werte $\Gamma(n+1) = n!$ der Fakultät annimmt.
Indem man eine beliebige Funktion $f(z)$ addiert, die auf alle
natürlichen Zahlen verschwindet, also $f(n)=0$ für $n\in\mathbb{N}$,
erhält man eine weitere Funktion, die auf natürlichen Zahlen
die Werte der Fakultät annimmt.
Ein Beispiel einer solchen Funktion ist
\begin{equation}
z\mapsto f(z)=\Gamma(z) + \sin \pi z,
\label{buch:rekursion:eqn:gammaalternative}
\end{equation}
die Funktion $f(z)=\sin\pi z$ verschwindet sogar auf allen ganzen
Zahlen.

In Abbildung~\ref{buch:rekursion:fig:gamma} ist die Gamma-Funktion
in rot geplotet, die Funktion~\eqref{buch:rekursion:eqn:gammaalternative}
in grün.
Die Punkte $(n,(n-1)!)$ sind in blau bezeichnet, sie sind beiden Graphen
gemeinsam.

\subsubsection{Pol erster Ordnung bei $z=0$}
Wir haben zu prüfen, dass sowohl der Wert $\Gamma(1)$ korrekt ist als
auch die Rekursionsformel~\eqref{buch:rekursion:eqn:gammadef} gilt.
Der Wert für $z=1$ ist
\begin{align*}
\Gamma(1)
&=
\int_0^\infty t^{1-1}e^{-t}\,dt
=
\left[ -e^{-t} \right]_0^\infty
=
1.
\end{align*}
Für die Rekursionsformel kann mit Hilfe von partieller Integration
bekommen:
\begin{align*}
\Gamma(z+1)
&=
\int_0^\infty t^{z+1-1}e^{-t}\,dt
=
\biggl[-t^{z}e^{-t}\biggr]_0^\infty
+
\int_0^\infty z t^{z-1}e^{-t}\,dt
\\
&=
z
\int_0^\infty
t^{z-1}e^{-t}\,dt
=
z \Gamma(z).
\end{align*}

Für $0<z<\varepsilon$ für eine $\varepsilon >0$ folgt aus der 
Funktionalgleichung
\[
\Gamma(z) = \frac{\Gamma(1+z)}{z}.
\]
Da $\Gamma(1)=1$ ist und $\Gamma$ eine in einer
Umgebung von $1$ stetige Funktion ist, kann sie in der Form
\(
\Gamma(1+z)=\Gamma(1) + zf(z)
\)
schreiben, wobei  $f(z)$ eine differenzierbare Funktion ist mit
$f'(1)=\Gamma'(1)$.
Daraus ergibt sich für $\Gamma(z)$ der Ausdruck
\[
\Gamma(z) = \frac{\Gamma(1)}{z} + f(z) = \frac{1}{z} + f(z).
\]
Die Gamma-Funktion hat daher and er Stelle $z=0$ einen Pol erster Ordnung.

\subsubsection{Ausdehnung auf $\operatorname{Re}z<0$}
Die Integralformel konvergiert nicht für $\operatorname{Re}z\le 0$.
Durch analytische Fortsetzung, wie sie im
Abschnitt~\ref{buch:funktionentheorie:section:fortsetzung}
beschrieben wird, kann die Funktion auf ganz $\mathbb{C}$ ausgedehnt
werden, mit Ausnahme einzelner Pole.
Die Funktionalgleichung gilt natürlich für alle $z\in\mathbb{C}$,
für die $\Gamma(z)$ definiert ist.
In einer Umgebung von $z=-n$ gilt
\[
\Gamma(z)
=
\frac{\Gamma(z+1)}{z}
=
\frac{\Gamma(z+2)}{z(z+1)}
=
\frac{\Gamma(z+3)}{z(z+1)(z+2)}
=
\dots
=
\frac{\Gamma(z+n)}{z(z+1)(z+2)\cdots(z+n-1)}
\]
Keiner der Faktoren im Nenner verschwindet in der Nähe von $z=-n$, der
Zähler hat aber einen Pol erster Ordnung an dieser Stelle.
Daher hat auch der Quotient einen Pol erster Ordnung.
Abbildung~\ref{buch:rekursion:fig:gamma} zeigt die Pole bei den
nicht negativen ganzen Zahlen.






%
% linear.tex
%
% (c) 2021 Prof Dr Andreas Müller, OST Ostschweizer Fachhochschule
%
\section{Lineare Rekursionsgleichung mit konstanten Koeffizienten
\label{buch:rekursion:section:linear}}
\rhead{Lineare Rekursionsgleichungen}
Die Funktionalgleichung der Gamma-Funktion, die im
Abschnitt~\ref{buch:rekursion:section:gamma} untersucht wurde,
hat die Form einer linearen Rekursionsgleichung
\[
\Gamma(x+1) = x\Gamma(x),\qquad \Gamma(1) = 1.
\]
Gleichungen, die Werte einer Funktion für verschiedene
Argument in Beziehung setzen, heissen {\em Funktionalgleichungen}.
\index{Funktionalgleichung}%
Es war überraschend schwierig, eine Lösung für Funktionalgleichung
der Gamma-Funktion für beliebige komplexe $x$ zu finden.
In diesem Abschnitt soll daher eine Klasse von Rekursionsgleichungen
näher untersucht werden, für die einfache Lösungen möglich sind.

\subsection{Lineare Differenzengleichungen}

\subsection{Lösung mit Polynomfunktionen}







%
% hypergeometrisch.tex
%
% (c) 2021 Prof Dr Andreas Müller, OST Ostschweizer Fachhochschule
%
\section{Hypergeometrische Differentialgleichung
\label{buch:differentialgleichungen:section:hypergeometrisch}}
Die hypergeometrische Funktion $\mathstrut_2F1(a,b;c;x)$ wurde in
Abschnitt~\ref{buch:rekursion:section:hypergeometrische-funktion}
als Potenzreihe mit sehr speziellen Koeffizienten, die sich aus
Pochhammer-Symbolen.
Es stellt sich aber heraus, dass man sie auch als Lösung einer
gewöhnlichen Differentialgleichung bekommen kann, die bereits
Euler studiert hat.

\subsection{Die Eulersche hypergeometrische Differentialgleichung
\label{buch:differentialgleichung:subsection:euler-hypergeometrisch}}
Die hypergeometrische Funktion $\mathstrut_2F_1(a,b;c;x)$ ist eine
Lösung der {\em Eulerschen hypergeometrischen Differentialgleichung}
(zu unterscheiden von der Eulerschen Differentialgleichung, die sich
immer auf eine lineare Differentialgleichung mit konstanten Koeffizienten
reduzieren lässt)
\begin{equation}
x(1-x) \frac{d^2y}{dx^2} + (c-(a+b+1)x)\frac{dy}{dx} - ab y = 0
\label{buch:differentialgleichungen:hypergeo:eulerdgl}
\end{equation}
Wir prüfen dies nach, indem wir die Definition der hypergeometrischen
Funktion 
\begin{align*}
y(x)
&=
\mathstrut_2F_1(a,b;c;x)
=
\sum_{k=0}^\infty
\frac{(a)_k(b)_k}{(c)_k} \frac{x^k}{k!}
\intertext{mit den Ableitungen}
y'(x)
&=
\sum_{k=1}^\infty 
\frac{(a)_k(b)_k}{(c)_k} \frac{x^{k-1}}{(k-1)!}
\\
y''(x)
&=
\sum_{k=2}^\infty 
\frac{(a)_k(b)_k}{(c)_k} \frac{x^{k-2}}{(k-2)!}
\end{align*}
einsetzen.
Die Gleichung, die sich ergibt, ist
\begin{align*}
0
&=
x(1-x)
\sum_{k=2}^\infty
\frac{(a)_k(b)_k}{(c)_k}\frac{x^{k-2}}{(k-2)!}
+
(c-(a+b+1)x)
\sum_{k=1}^\infty
\frac{(a)_k(b)_k}{(c)_k}\frac{x^{k-1}}{(k-1)!}
-ab
\sum_{k=0}^\infty
\frac{(a)_k(b)_k}{(c)_k} \frac{x^k}{k!}
\\
&=
\sum_{k=2}^\infty
\frac{(a)_k(b)_k}{(c)_k}\frac{x^{k-1}}{(k-2)!}
-
\sum_{k=2}^\infty
\frac{(a)_k(b)_k}{(c)_k}\frac{x^k}{(k-2)!}
+
c\sum_{k=1}^\infty
\frac{(a)_k(b)_k}{(c)_k}\frac{x^{k-1}}{(k-1)!}
\\
&\qquad
-(a+b+1)
\sum_{k=1}^\infty
\frac{(a)_k(b)_k}{(c)_k}\frac{x^k}{(k-1)!}
-ab
\sum_{k=0}^\infty
\frac{(a)_k(b)_k}{(c)_k} \frac{x^k}{k!}
\\
&=
\sum_{k=1}^\infty
\frac{(a)_{k+1}(b)_{k+1}}{(c)_{k+1}}\frac{x^k}{(k-1)!}
-
\sum_{k=2}^\infty
\frac{(a)_k(b)_k}{(c)_k}\frac{x^k}{(k-2)!}
+
c\sum_{k=0}^\infty
\frac{(a)_{k+1}(b)_{k+1}}{(c)_{k+1}}\frac{x^k}{k!}
\\
&\qquad
-(a+b+1)
\sum_{k=1}^\infty
\frac{(a)_k(b)_k}{(c)_k}\frac{x^k}{(k-1)!}
-ab
\sum_{k=0}^\infty
\frac{(a)_k(b)_k}{(c)_k} \frac{x^k}{k!}.
\end{align*}
Zum konstanten Koeffizienten für $k=0$ tragen nur die dritte und letzte
Summe bei, dies sind die Terme
\[
c\frac{(a)_1(b)_1}{(c)_1}-ab\frac{(a)_0(b)_0}{(c)_0}
=
c\frac{ab}{c}-ab\frac{1\cdot 1}{1}
=
0.
\]
Für den linearen Term $k=1$ kommen je ein Term aus der ersten aund vierten
Summe hinzu, dies ergibt
\begin{align*}
&\phantom{\mathstrut=\mathstrut}
\frac{(a)_2(b)_2}{(c)_2}
+c\frac{(a)_2(b)_2}{(c)_2}
-(a+b+1)\frac{(a)_1(b)_1}{(c)_1}
-ab\frac{(a)_1(b)_1}{(c)_1}
\\
&=
\frac{a(a+1)b(b+1)}{c(c+1)}
(1+c)
-(ab+a+b+1)
\frac{ab}{c}
\\
&=
\frac{a(a+1)b(b+1)}{c}
-
(a+1)(b+1)\frac{ab}{c}
=0.
\end{align*}
Durch Koeffizientenvergleich erhalten wir für $k\ge 2$ 
\begin{align*}
0
&=
\frac{(a)_{k+1}(b)_{k+1}}{(c)_{k+1}} \frac1{(k-1)!} 
-
\frac{(a)_k(b)_k}{(c)_k} \frac1{(k-2)!} 
+
c\frac{(a)_{k+1}(b)_{k+1}}{(c)_{k+1}} \frac{1}{k!}
\\
&\qquad
-(a+b+1)\frac{(a)_k(b)_k}{(c)_k}\frac{1}{(k-1)!}
-ab \frac{(a)_k(b)_k}{(c)_k}\frac{1}{k!}
\\
&=
\frac{(a)_k(b)_k}{(c)_{k+1}}
\frac{1}{k!}
\biggl(
(a+k)(b+k)k
-(c+k)(k-1)k
+
c(a+k)(b+k)
\\
&\qquad
\qquad
\qquad
-(a+b+1)(c+k)k
-ab(c+k)
\biggr).
\intertext{Der zweite, vierte und fünfte Term können zu}
&=
\frac{(a)_k(b)_k}{(c)_{k+1}}
\frac{1}{k!}
\biggl(
(a+k)(b+k)k
+
c(a+k)(b+k)
-(ab+ak+bk+k^2)(c+k)
\biggr)
\intertext{zusammengefasst werden.
Der Faktor $(ab+ak+bk+k^2)$ kann als Produkt $(a+k)(b+k)$ faktorisiert
werden, der dann als gemeinsamer Faktor aus allen Termen ausgeklammert
werden kann:}
&=
\frac{(a)_k(b)_k}{(c)_{k+1}}
\frac{1}{k!}
\biggl(
(a+k)(b+k)k
+
c(a+k)(b+k)
-(a+k)(b+k)(c+k)
\biggr)
\\
&=
\frac{(a)_{k+1}(b)_{k+1}}{(c)_{k+1}}
\frac{1}{k!}
\biggl(
k
+
c
-(c+k)
\biggr)
=0.
\end{align*}
Damit ist gezeigt, dass $\mathstrut_2F_1(a,b;c;x)$ eine Lösung
der Differentialgleichung ist.

Die hypergeometrische Reihe kann auch direkt mit Hilfe der
Potenzreihenmethode als Lösung der Differentialgleichung gefunden 
werden.

\subsection{Lösung als verallgemeinerte Potenzreihe}
Da die hypergeometrische Reihe eine Differentialgleichung
zweiter Ordnung mit einer Singularität bei $x=0$ ist, 
kann man versuchen eine zweite, linear unabhängige Lösung mit
Hilfe der Methode der verallgemeinerten Potenzreihen zu finden.
Dazu setzt man die Lösung in der Form
\begin{align*}
y_2(x)
&=
\sum_{k=0}^\infty a_kx^{\varrho+k}
&
&\Rightarrow&
y_2'(x)
&=
\sum_{k=0}^\infty (\varrho+k)a_kx^{\varrho+k-1}
\\
&&
&&
y_2''(x)
&=
\sum_{k=0}^\infty (\varrho+k)(\varrho+k-1)a_kx^{\varrho+k-2}
\end{align*}
an, wobei $a_0\ne 0$ sein soll.
Einsetzen in die Differentialgleichung ergibt
\begin{align*}
0&=
x(1-x)y_2''(x) + (c-(a+b+1)x) y_2'(x) -aby_2(x)
\\
&=
x(1-x)
\sum_{k=0}^\infty (\varrho+k)(\varrho+k-1)a_kx^{\varrho+k-2}
+
(c-(a+b+1)x)
\sum_{k=0}^\infty (\varrho+k)a_kx^{\varrho+k-1}
-
abx^{\varrho}\sum_{k=0}^\infty a_kx^{\varrho+k}
\\
&=
-\sum_{k=0}^\infty (\varrho+k)(\varrho+k-1)a_kx^{\varrho+k}
+
\sum_{k=0}^\infty (\varrho+k)(\varrho+k-1)a_kx^{\varrho+k-1}
+
c
\sum_{k=0}^\infty (\varrho+k)a_kx^{\varrho+k-1}
\\
&\qquad
-
(a+b+1)
\sum_{k=0}^\infty (\varrho+k)a_kx^{\varrho+k}
-
ab
\sum_{k=0}^\infty a_kx^{\varrho+k}.
\intertext{Durch Verschiebung des Summationsindex in der zweiten
und dritten Summe wird der Koeffizientenvergleich etwas
einfacher}
&=
-\sum_{k=0}^\infty (\varrho+k)(\varrho+k-1)a_kx^{\varrho+k}
+
\sum_{k=-1}^\infty (\varrho+k+1)(\varrho+k)a_{k+1}x^{\varrho+k}
+
c
\sum_{k=-1}^\infty (\varrho+k+1)a_{k+1}x^{\varrho+k}
\\
&\qquad
-
(a+b+1)
\sum_{k=0}^\infty (\varrho+k)a_kx^{\varrho+k}
-
ab
\sum_{k=0}^\infty a_kx^{\varrho+k}
\\
&=
-\sum_{k=0}^\infty (\varrho+k)(\varrho+k-1)a_kx^{\varrho+k}
+
\sum_{k=-1}^\infty (\varrho+k+1)(\varrho+k+c)a_{k+1}x^{\varrho+k}
\\
&\qquad
-
\sum_{k=0}^\infty ((\varrho+k)(a+b+1)+ab)a_kx^{\varrho+k}
\\
&=
\bigl(
\varrho(\varrho-1)
+c\varrho \bigr)
x^{\varrho-1}
+
\sum_{k=0}^\infty
\bigl(
-(\varrho+k)(\varrho+k-1)a_k
+(\varrho+k+1)(\varrho+k+c)a_{k+1}
\\
&
\qquad
\qquad
\qquad
\qquad
\qquad
\qquad
-((\varrho+k)(a+b+1)+ab)a_k
\bigr)
x^{\varrho+k}.
\end{align*}
Aus dem ersten Term kann man die Indexgleichung
\[
0
=
\varrho(\varrho-1)+c\varrho
=
\varrho(\varrho-1+c)
\]
ablesen, die die Nullstellen $\varrho=0$ und $\varrho=1-c$ hat.
Die Nullstelle $\varrho=0$ ergibt natürlich die bereits gefundene
hypergeometrische Reihe.

Nach Einsetzen der zweiten Lösung der Indexgleichung in der Summe
legt der Koeffizientenvergleich eine Beziehung
\begin{align}
0
&=
\bigl(
-(k-c+1)(k-c)
-(k-c+1)(a+b+1)+ab
\bigr)a_k
+
(k-c+2)(k+1)
a_{k+1} 
\notag
\intertext{zwischen $a_k$ und $a_{k+1}$ fest.
Daraus kann man den Quotienten aufeinanderfolgender
Koeffizienten als}
\frac{a_{k+1}}{a_k}
&=
\frac{
-(k-c+1)(k-c)
-(k-c+1)(a+b+1)+ab
}{
\notag
(k-c+2)(k+1)
}
\\
&=
%(%i4) factor(coeff(y,q,0))
%(%o4)                  - (k - c + a + 1) (k - c + b + 1)
%(%i5) factor(coeff(y,q,1))
%(%o5)                         (k + 1) (k - c + 2)
\frac{
(a-c+1+k)
(b-c+1+k)
}{
(2-c+k)(k+1)
}
\label{buch:differentialgleichungen:hypergeo:verallgkoef}
\end{align}
finden.
Setzt man $a_0=1$, ist die zweite Lösung ist also wieder eine
hypergeometrische Funktion.%, nämlich
%\[
%y_2(x)
%=
%x^{1-c}
%\sum_{k=0}^\infty \frac{(a-c+1)_k(b-c+1)_k}{(2-c)_k}\frac{x^k}{k!}
%=
%x^{1-c}
%\mathstrut_2F_1\biggl(\begin{matrix}a-c+1,b-c+1\\2-c\end{matrix};x\biggr)
%\]
Diese Lösung ist aber nur möglich, wenn in
\eqref{buch:differentialgleichungen:hypergeo:verallgkoef}
der Nenner nicht verschwindet, d.~h.~$2-c+k\ne 0$
oder $c \ne k+2$ für all natürlichen $k$.
$c$ darf also kein natürliche Zahl $\ge 2$ sein.
Wir fassen die Resultate dieses Abschnitts im folgenden Satz zusammen.

\begin{satz}
Die eulersche hypergeometrische Differentialgleichung
\begin{equation}
x(1-x)\frac{d^2y}{dx^2}
+(c+(a+b+1)x)\frac{dy}{dx}
-ab y
=
0
\end{equation}
hat die Lösung
\[
y_1(x)
=
\mathstrut_2F_1\biggl(\begin{matrix}a,b\\c\end{matrix};x\biggr).
\]
Falls $c-2\not\in \mathbb{N}$ ist, ist
\[
y_2(x)
=
x^{1-c} \mathstrut_2F_1\biggl(\begin{matrix}a-c+1,b-c+1\\2-c\end{matrix};x\biggr)
\]
eine zweite, linear unabhängige Lösung.
\end{satz}

%
% Die verallgemeinerte hypergeometrische Differentialgleichung
%
\subsection{Verallgemeinerte hypergeometrische Differentialgleichung}
% https://de.wikipedia.org/wiki/Verallgemeinerte_hypergeometrische_Funktion







\section*{Übungsaufgaben}
\rhead{Übungsaufgaben}
\aufgabetoplevel{chapters/040-rekursion/uebungsaufgaben}
\begin{uebungsaufgaben}
%\uebungsaufgabe{0}
\uebungsaufgabe{1}
\uebungsaufgabe{2}
\end{uebungsaufgaben}


%%
% chapter.tex -- Beschreibung des Inhaltes
%
% (c) 2021 Prof Dr Andreas Müller, Hochschule Rapperswil
%
% !TeX spellcheck = de_CH
\chapter{Spezielle Funktionen und Rekursion
\label{buch:chapter:rekursion}}
\lhead{Spezielle Funktionen und Rekursion}
\rhead{}

%
% gamma.tex -- Abschnitt über die Gamma-funktion
%
% (c) 2021 Prof Dr Andreas Müller, OST Ostschweizer Fachhochschule
%
\section{Die Gamma-Funktion
\label{buch:rekursion:section:gamma}}
Die Fakultät $x!$ kann rekursiv durch 
\[
	x! = x\cdot (x-1)! \qquad\text{und}\qquad 0!=1
\]
für alle natürlichen Zahlen $x\in\mathbb{N}$ definiert werden.
Äquivalent damit ist eine Funktion 
\begin{equation}
\Gamma(x+1) = x\Gamma(x)
\qquad\text{und}\qquad 
\Gamma(1)=1.
\label{buch:rekursion:eqn:gammadef}
\end{equation}
Kann man eine reelle oder komplexe Funktion finden, die die
Funktionalgleichung~\eqref{buch:rekursion:eqn:gammadef}
erfüllt und damit die Fakultät auf beliebige Argumente ausdehnt?

\subsection{Integralformel für die Gamma-Funktion}
Euler hat die folgende Integraldefinition der Gamma-Funktion gegeben.

\begin{definition}
\label{buch:rekursion:def:gamma}
Die Gamma-Funktion ist die Funktion 
\[
\Gamma
\colon
\{z\in\mathbb{C} \mid \operatorname{Re}z>0\}
\to \mathbb{C}
:
z
\mapsto
\Gamma(z) = \int_0^\infty t^{x-1}e^{-t}\,dt
\]
\end{definition}

Man beachte, dass das Integral für $x=0$ nicht definiert ist, eine
Potenzreihenentwicklung um einen Punkt $x_0$ auf der positiven reellen
Achse kann also höchstens den Konvergenzradius $\varrho=|x_0|$ haben.

\begin{figure}
\centering
\includegraphics{chapters/040-rekursion/images/gammaplot.pdf}
\caption{Graph der Gamma-Funktion $z\mapsto\Gamma(z)$ und der alternativen
Funktion $\Gamma(z)+\sin(\pi z)$, die für ganzzahlige Argumente ebenfalls
die Werte der Fakultät annimmt.
\label{buch:rekursion:fig:gamma}}
\end{figure}

\subsubsection{Alternative Lösungen}
Die Funktion $\Gamma(z)$ ist nicht die einzige Funktion, die natürlichen
Zahlen die Werte $\Gamma(n+1) = n!$ der Fakultät annimmt.
Indem man eine beliebige Funktion $f(z)$ addiert, die auf alle
natürlichen Zahlen verschwindet, also $f(n)=0$ für $n\in\mathbb{N}$,
erhält man eine weitere Funktion, die auf natürlichen Zahlen
die Werte der Fakultät annimmt.
Ein Beispiel einer solchen Funktion ist
\begin{equation}
z\mapsto f(z)=\Gamma(z) + \sin \pi z,
\label{buch:rekursion:eqn:gammaalternative}
\end{equation}
die Funktion $f(z)=\sin\pi z$ verschwindet sogar auf allen ganzen
Zahlen.

In Abbildung~\ref{buch:rekursion:fig:gamma} ist die Gamma-Funktion
in rot geplotet, die Funktion~\eqref{buch:rekursion:eqn:gammaalternative}
in grün.
Die Punkte $(n,(n-1)!)$ sind in blau bezeichnet, sie sind beiden Graphen
gemeinsam.

\subsubsection{Pol erster Ordnung bei $z=0$}
Wir haben zu prüfen, dass sowohl der Wert $\Gamma(1)$ korrekt ist als
auch die Rekursionsformel~\eqref{buch:rekursion:eqn:gammadef} gilt.
Der Wert für $z=1$ ist
\begin{align*}
\Gamma(1)
&=
\int_0^\infty t^{1-1}e^{-t}\,dt
=
\left[ -e^{-t} \right]_0^\infty
=
1.
\end{align*}
Für die Rekursionsformel kann mit Hilfe von partieller Integration
bekommen:
\begin{align*}
\Gamma(z+1)
&=
\int_0^\infty t^{z+1-1}e^{-t}\,dt
=
\biggl[-t^{z}e^{-t}\biggr]_0^\infty
+
\int_0^\infty z t^{z-1}e^{-t}\,dt
\\
&=
z
\int_0^\infty
t^{z-1}e^{-t}\,dt
=
z \Gamma(z).
\end{align*}

Für $0<z<\varepsilon$ für eine $\varepsilon >0$ folgt aus der 
Funktionalgleichung
\[
\Gamma(z) = \frac{\Gamma(1+z)}{z}.
\]
Da $\Gamma(1)=1$ ist und $\Gamma$ eine in einer
Umgebung von $1$ stetige Funktion ist, kann sie in der Form
\(
\Gamma(1+z)=\Gamma(1) + zf(z)
\)
schreiben, wobei  $f(z)$ eine differenzierbare Funktion ist mit
$f'(1)=\Gamma'(1)$.
Daraus ergibt sich für $\Gamma(z)$ der Ausdruck
\[
\Gamma(z) = \frac{\Gamma(1)}{z} + f(z) = \frac{1}{z} + f(z).
\]
Die Gamma-Funktion hat daher and er Stelle $z=0$ einen Pol erster Ordnung.

\subsubsection{Ausdehnung auf $\operatorname{Re}z<0$}
Die Integralformel konvergiert nicht für $\operatorname{Re}z\le 0$.
Durch analytische Fortsetzung, wie sie im
Abschnitt~\ref{buch:funktionentheorie:section:fortsetzung}
beschrieben wird, kann die Funktion auf ganz $\mathbb{C}$ ausgedehnt
werden, mit Ausnahme einzelner Pole.
Die Funktionalgleichung gilt natürlich für alle $z\in\mathbb{C}$,
für die $\Gamma(z)$ definiert ist.
In einer Umgebung von $z=-n$ gilt
\[
\Gamma(z)
=
\frac{\Gamma(z+1)}{z}
=
\frac{\Gamma(z+2)}{z(z+1)}
=
\frac{\Gamma(z+3)}{z(z+1)(z+2)}
=
\dots
=
\frac{\Gamma(z+n)}{z(z+1)(z+2)\cdots(z+n-1)}
\]
Keiner der Faktoren im Nenner verschwindet in der Nähe von $z=-n$, der
Zähler hat aber einen Pol erster Ordnung an dieser Stelle.
Daher hat auch der Quotient einen Pol erster Ordnung.
Abbildung~\ref{buch:rekursion:fig:gamma} zeigt die Pole bei den
nicht negativen ganzen Zahlen.






%
% linear.tex
%
% (c) 2021 Prof Dr Andreas Müller, OST Ostschweizer Fachhochschule
%
\section{Lineare Rekursionsgleichung mit konstanten Koeffizienten
\label{buch:rekursion:section:linear}}
\rhead{Lineare Rekursionsgleichungen}
Die Funktionalgleichung der Gamma-Funktion, die im
Abschnitt~\ref{buch:rekursion:section:gamma} untersucht wurde,
hat die Form einer linearen Rekursionsgleichung
\[
\Gamma(x+1) = x\Gamma(x),\qquad \Gamma(1) = 1.
\]
Gleichungen, die Werte einer Funktion für verschiedene
Argument in Beziehung setzen, heissen {\em Funktionalgleichungen}.
\index{Funktionalgleichung}%
Es war überraschend schwierig, eine Lösung für Funktionalgleichung
der Gamma-Funktion für beliebige komplexe $x$ zu finden.
In diesem Abschnitt soll daher eine Klasse von Rekursionsgleichungen
näher untersucht werden, für die einfache Lösungen möglich sind.

\subsection{Lineare Differenzengleichungen}

\subsection{Lösung mit Polynomfunktionen}







%
% hypergeometrisch.tex
%
% (c) 2021 Prof Dr Andreas Müller, OST Ostschweizer Fachhochschule
%
\section{Hypergeometrische Differentialgleichung
\label{buch:differentialgleichungen:section:hypergeometrisch}}
Die hypergeometrische Funktion $\mathstrut_2F1(a,b;c;x)$ wurde in
Abschnitt~\ref{buch:rekursion:section:hypergeometrische-funktion}
als Potenzreihe mit sehr speziellen Koeffizienten, die sich aus
Pochhammer-Symbolen.
Es stellt sich aber heraus, dass man sie auch als Lösung einer
gewöhnlichen Differentialgleichung bekommen kann, die bereits
Euler studiert hat.

\subsection{Die Eulersche hypergeometrische Differentialgleichung
\label{buch:differentialgleichung:subsection:euler-hypergeometrisch}}
Die hypergeometrische Funktion $\mathstrut_2F_1(a,b;c;x)$ ist eine
Lösung der {\em Eulerschen hypergeometrischen Differentialgleichung}
(zu unterscheiden von der Eulerschen Differentialgleichung, die sich
immer auf eine lineare Differentialgleichung mit konstanten Koeffizienten
reduzieren lässt)
\begin{equation}
x(1-x) \frac{d^2y}{dx^2} + (c-(a+b+1)x)\frac{dy}{dx} - ab y = 0
\label{buch:differentialgleichungen:hypergeo:eulerdgl}
\end{equation}
Wir prüfen dies nach, indem wir die Definition der hypergeometrischen
Funktion 
\begin{align*}
y(x)
&=
\mathstrut_2F_1(a,b;c;x)
=
\sum_{k=0}^\infty
\frac{(a)_k(b)_k}{(c)_k} \frac{x^k}{k!}
\intertext{mit den Ableitungen}
y'(x)
&=
\sum_{k=1}^\infty 
\frac{(a)_k(b)_k}{(c)_k} \frac{x^{k-1}}{(k-1)!}
\\
y''(x)
&=
\sum_{k=2}^\infty 
\frac{(a)_k(b)_k}{(c)_k} \frac{x^{k-2}}{(k-2)!}
\end{align*}
einsetzen.
Die Gleichung, die sich ergibt, ist
\begin{align*}
0
&=
x(1-x)
\sum_{k=2}^\infty
\frac{(a)_k(b)_k}{(c)_k}\frac{x^{k-2}}{(k-2)!}
+
(c-(a+b+1)x)
\sum_{k=1}^\infty
\frac{(a)_k(b)_k}{(c)_k}\frac{x^{k-1}}{(k-1)!}
-ab
\sum_{k=0}^\infty
\frac{(a)_k(b)_k}{(c)_k} \frac{x^k}{k!}
\\
&=
\sum_{k=2}^\infty
\frac{(a)_k(b)_k}{(c)_k}\frac{x^{k-1}}{(k-2)!}
-
\sum_{k=2}^\infty
\frac{(a)_k(b)_k}{(c)_k}\frac{x^k}{(k-2)!}
+
c\sum_{k=1}^\infty
\frac{(a)_k(b)_k}{(c)_k}\frac{x^{k-1}}{(k-1)!}
\\
&\qquad
-(a+b+1)
\sum_{k=1}^\infty
\frac{(a)_k(b)_k}{(c)_k}\frac{x^k}{(k-1)!}
-ab
\sum_{k=0}^\infty
\frac{(a)_k(b)_k}{(c)_k} \frac{x^k}{k!}
\\
&=
\sum_{k=1}^\infty
\frac{(a)_{k+1}(b)_{k+1}}{(c)_{k+1}}\frac{x^k}{(k-1)!}
-
\sum_{k=2}^\infty
\frac{(a)_k(b)_k}{(c)_k}\frac{x^k}{(k-2)!}
+
c\sum_{k=0}^\infty
\frac{(a)_{k+1}(b)_{k+1}}{(c)_{k+1}}\frac{x^k}{k!}
\\
&\qquad
-(a+b+1)
\sum_{k=1}^\infty
\frac{(a)_k(b)_k}{(c)_k}\frac{x^k}{(k-1)!}
-ab
\sum_{k=0}^\infty
\frac{(a)_k(b)_k}{(c)_k} \frac{x^k}{k!}.
\end{align*}
Zum konstanten Koeffizienten für $k=0$ tragen nur die dritte und letzte
Summe bei, dies sind die Terme
\[
c\frac{(a)_1(b)_1}{(c)_1}-ab\frac{(a)_0(b)_0}{(c)_0}
=
c\frac{ab}{c}-ab\frac{1\cdot 1}{1}
=
0.
\]
Für den linearen Term $k=1$ kommen je ein Term aus der ersten aund vierten
Summe hinzu, dies ergibt
\begin{align*}
&\phantom{\mathstrut=\mathstrut}
\frac{(a)_2(b)_2}{(c)_2}
+c\frac{(a)_2(b)_2}{(c)_2}
-(a+b+1)\frac{(a)_1(b)_1}{(c)_1}
-ab\frac{(a)_1(b)_1}{(c)_1}
\\
&=
\frac{a(a+1)b(b+1)}{c(c+1)}
(1+c)
-(ab+a+b+1)
\frac{ab}{c}
\\
&=
\frac{a(a+1)b(b+1)}{c}
-
(a+1)(b+1)\frac{ab}{c}
=0.
\end{align*}
Durch Koeffizientenvergleich erhalten wir für $k\ge 2$ 
\begin{align*}
0
&=
\frac{(a)_{k+1}(b)_{k+1}}{(c)_{k+1}} \frac1{(k-1)!} 
-
\frac{(a)_k(b)_k}{(c)_k} \frac1{(k-2)!} 
+
c\frac{(a)_{k+1}(b)_{k+1}}{(c)_{k+1}} \frac{1}{k!}
\\
&\qquad
-(a+b+1)\frac{(a)_k(b)_k}{(c)_k}\frac{1}{(k-1)!}
-ab \frac{(a)_k(b)_k}{(c)_k}\frac{1}{k!}
\\
&=
\frac{(a)_k(b)_k}{(c)_{k+1}}
\frac{1}{k!}
\biggl(
(a+k)(b+k)k
-(c+k)(k-1)k
+
c(a+k)(b+k)
\\
&\qquad
\qquad
\qquad
-(a+b+1)(c+k)k
-ab(c+k)
\biggr).
\intertext{Der zweite, vierte und fünfte Term können zu}
&=
\frac{(a)_k(b)_k}{(c)_{k+1}}
\frac{1}{k!}
\biggl(
(a+k)(b+k)k
+
c(a+k)(b+k)
-(ab+ak+bk+k^2)(c+k)
\biggr)
\intertext{zusammengefasst werden.
Der Faktor $(ab+ak+bk+k^2)$ kann als Produkt $(a+k)(b+k)$ faktorisiert
werden, der dann als gemeinsamer Faktor aus allen Termen ausgeklammert
werden kann:}
&=
\frac{(a)_k(b)_k}{(c)_{k+1}}
\frac{1}{k!}
\biggl(
(a+k)(b+k)k
+
c(a+k)(b+k)
-(a+k)(b+k)(c+k)
\biggr)
\\
&=
\frac{(a)_{k+1}(b)_{k+1}}{(c)_{k+1}}
\frac{1}{k!}
\biggl(
k
+
c
-(c+k)
\biggr)
=0.
\end{align*}
Damit ist gezeigt, dass $\mathstrut_2F_1(a,b;c;x)$ eine Lösung
der Differentialgleichung ist.

Die hypergeometrische Reihe kann auch direkt mit Hilfe der
Potenzreihenmethode als Lösung der Differentialgleichung gefunden 
werden.

\subsection{Lösung als verallgemeinerte Potenzreihe}
Da die hypergeometrische Reihe eine Differentialgleichung
zweiter Ordnung mit einer Singularität bei $x=0$ ist, 
kann man versuchen eine zweite, linear unabhängige Lösung mit
Hilfe der Methode der verallgemeinerten Potenzreihen zu finden.
Dazu setzt man die Lösung in der Form
\begin{align*}
y_2(x)
&=
\sum_{k=0}^\infty a_kx^{\varrho+k}
&
&\Rightarrow&
y_2'(x)
&=
\sum_{k=0}^\infty (\varrho+k)a_kx^{\varrho+k-1}
\\
&&
&&
y_2''(x)
&=
\sum_{k=0}^\infty (\varrho+k)(\varrho+k-1)a_kx^{\varrho+k-2}
\end{align*}
an, wobei $a_0\ne 0$ sein soll.
Einsetzen in die Differentialgleichung ergibt
\begin{align*}
0&=
x(1-x)y_2''(x) + (c-(a+b+1)x) y_2'(x) -aby_2(x)
\\
&=
x(1-x)
\sum_{k=0}^\infty (\varrho+k)(\varrho+k-1)a_kx^{\varrho+k-2}
+
(c-(a+b+1)x)
\sum_{k=0}^\infty (\varrho+k)a_kx^{\varrho+k-1}
-
abx^{\varrho}\sum_{k=0}^\infty a_kx^{\varrho+k}
\\
&=
-\sum_{k=0}^\infty (\varrho+k)(\varrho+k-1)a_kx^{\varrho+k}
+
\sum_{k=0}^\infty (\varrho+k)(\varrho+k-1)a_kx^{\varrho+k-1}
+
c
\sum_{k=0}^\infty (\varrho+k)a_kx^{\varrho+k-1}
\\
&\qquad
-
(a+b+1)
\sum_{k=0}^\infty (\varrho+k)a_kx^{\varrho+k}
-
ab
\sum_{k=0}^\infty a_kx^{\varrho+k}.
\intertext{Durch Verschiebung des Summationsindex in der zweiten
und dritten Summe wird der Koeffizientenvergleich etwas
einfacher}
&=
-\sum_{k=0}^\infty (\varrho+k)(\varrho+k-1)a_kx^{\varrho+k}
+
\sum_{k=-1}^\infty (\varrho+k+1)(\varrho+k)a_{k+1}x^{\varrho+k}
+
c
\sum_{k=-1}^\infty (\varrho+k+1)a_{k+1}x^{\varrho+k}
\\
&\qquad
-
(a+b+1)
\sum_{k=0}^\infty (\varrho+k)a_kx^{\varrho+k}
-
ab
\sum_{k=0}^\infty a_kx^{\varrho+k}
\\
&=
-\sum_{k=0}^\infty (\varrho+k)(\varrho+k-1)a_kx^{\varrho+k}
+
\sum_{k=-1}^\infty (\varrho+k+1)(\varrho+k+c)a_{k+1}x^{\varrho+k}
\\
&\qquad
-
\sum_{k=0}^\infty ((\varrho+k)(a+b+1)+ab)a_kx^{\varrho+k}
\\
&=
\bigl(
\varrho(\varrho-1)
+c\varrho \bigr)
x^{\varrho-1}
+
\sum_{k=0}^\infty
\bigl(
-(\varrho+k)(\varrho+k-1)a_k
+(\varrho+k+1)(\varrho+k+c)a_{k+1}
\\
&
\qquad
\qquad
\qquad
\qquad
\qquad
\qquad
-((\varrho+k)(a+b+1)+ab)a_k
\bigr)
x^{\varrho+k}.
\end{align*}
Aus dem ersten Term kann man die Indexgleichung
\[
0
=
\varrho(\varrho-1)+c\varrho
=
\varrho(\varrho-1+c)
\]
ablesen, die die Nullstellen $\varrho=0$ und $\varrho=1-c$ hat.
Die Nullstelle $\varrho=0$ ergibt natürlich die bereits gefundene
hypergeometrische Reihe.

Nach Einsetzen der zweiten Lösung der Indexgleichung in der Summe
legt der Koeffizientenvergleich eine Beziehung
\begin{align}
0
&=
\bigl(
-(k-c+1)(k-c)
-(k-c+1)(a+b+1)+ab
\bigr)a_k
+
(k-c+2)(k+1)
a_{k+1} 
\notag
\intertext{zwischen $a_k$ und $a_{k+1}$ fest.
Daraus kann man den Quotienten aufeinanderfolgender
Koeffizienten als}
\frac{a_{k+1}}{a_k}
&=
\frac{
-(k-c+1)(k-c)
-(k-c+1)(a+b+1)+ab
}{
\notag
(k-c+2)(k+1)
}
\\
&=
%(%i4) factor(coeff(y,q,0))
%(%o4)                  - (k - c + a + 1) (k - c + b + 1)
%(%i5) factor(coeff(y,q,1))
%(%o5)                         (k + 1) (k - c + 2)
\frac{
(a-c+1+k)
(b-c+1+k)
}{
(2-c+k)(k+1)
}
\label{buch:differentialgleichungen:hypergeo:verallgkoef}
\end{align}
finden.
Setzt man $a_0=1$, ist die zweite Lösung ist also wieder eine
hypergeometrische Funktion.%, nämlich
%\[
%y_2(x)
%=
%x^{1-c}
%\sum_{k=0}^\infty \frac{(a-c+1)_k(b-c+1)_k}{(2-c)_k}\frac{x^k}{k!}
%=
%x^{1-c}
%\mathstrut_2F_1\biggl(\begin{matrix}a-c+1,b-c+1\\2-c\end{matrix};x\biggr)
%\]
Diese Lösung ist aber nur möglich, wenn in
\eqref{buch:differentialgleichungen:hypergeo:verallgkoef}
der Nenner nicht verschwindet, d.~h.~$2-c+k\ne 0$
oder $c \ne k+2$ für all natürlichen $k$.
$c$ darf also kein natürliche Zahl $\ge 2$ sein.
Wir fassen die Resultate dieses Abschnitts im folgenden Satz zusammen.

\begin{satz}
Die eulersche hypergeometrische Differentialgleichung
\begin{equation}
x(1-x)\frac{d^2y}{dx^2}
+(c+(a+b+1)x)\frac{dy}{dx}
-ab y
=
0
\end{equation}
hat die Lösung
\[
y_1(x)
=
\mathstrut_2F_1\biggl(\begin{matrix}a,b\\c\end{matrix};x\biggr).
\]
Falls $c-2\not\in \mathbb{N}$ ist, ist
\[
y_2(x)
=
x^{1-c} \mathstrut_2F_1\biggl(\begin{matrix}a-c+1,b-c+1\\2-c\end{matrix};x\biggr)
\]
eine zweite, linear unabhängige Lösung.
\end{satz}

%
% Die verallgemeinerte hypergeometrische Differentialgleichung
%
\subsection{Verallgemeinerte hypergeometrische Differentialgleichung}
% https://de.wikipedia.org/wiki/Verallgemeinerte_hypergeometrische_Funktion







\section*{Übungsaufgaben}
\rhead{Übungsaufgaben}
\aufgabetoplevel{chapters/040-rekursion/uebungsaufgaben}
\begin{uebungsaufgaben}
%\uebungsaufgabe{0}
\uebungsaufgabe{1}
\uebungsaufgabe{2}
\end{uebungsaufgaben}


%%
% chapter.tex -- Beschreibung des Inhaltes
%
% (c) 2021 Prof Dr Andreas Müller, Hochschule Rapperswil
%
% !TeX spellcheck = de_CH
\chapter{Spezielle Funktionen und Rekursion
\label{buch:chapter:rekursion}}
\lhead{Spezielle Funktionen und Rekursion}
\rhead{}

%
% gamma.tex -- Abschnitt über die Gamma-funktion
%
% (c) 2021 Prof Dr Andreas Müller, OST Ostschweizer Fachhochschule
%
\section{Die Gamma-Funktion
\label{buch:rekursion:section:gamma}}
Die Fakultät $x!$ kann rekursiv durch 
\[
	x! = x\cdot (x-1)! \qquad\text{und}\qquad 0!=1
\]
für alle natürlichen Zahlen $x\in\mathbb{N}$ definiert werden.
Äquivalent damit ist eine Funktion 
\begin{equation}
\Gamma(x+1) = x\Gamma(x)
\qquad\text{und}\qquad 
\Gamma(1)=1.
\label{buch:rekursion:eqn:gammadef}
\end{equation}
Kann man eine reelle oder komplexe Funktion finden, die die
Funktionalgleichung~\eqref{buch:rekursion:eqn:gammadef}
erfüllt und damit die Fakultät auf beliebige Argumente ausdehnt?

\subsection{Integralformel für die Gamma-Funktion}
Euler hat die folgende Integraldefinition der Gamma-Funktion gegeben.

\begin{definition}
\label{buch:rekursion:def:gamma}
Die Gamma-Funktion ist die Funktion 
\[
\Gamma
\colon
\{z\in\mathbb{C} \mid \operatorname{Re}z>0\}
\to \mathbb{C}
:
z
\mapsto
\Gamma(z) = \int_0^\infty t^{x-1}e^{-t}\,dt
\]
\end{definition}

Man beachte, dass das Integral für $x=0$ nicht definiert ist, eine
Potenzreihenentwicklung um einen Punkt $x_0$ auf der positiven reellen
Achse kann also höchstens den Konvergenzradius $\varrho=|x_0|$ haben.

\begin{figure}
\centering
\includegraphics{chapters/040-rekursion/images/gammaplot.pdf}
\caption{Graph der Gamma-Funktion $z\mapsto\Gamma(z)$ und der alternativen
Funktion $\Gamma(z)+\sin(\pi z)$, die für ganzzahlige Argumente ebenfalls
die Werte der Fakultät annimmt.
\label{buch:rekursion:fig:gamma}}
\end{figure}

\subsubsection{Alternative Lösungen}
Die Funktion $\Gamma(z)$ ist nicht die einzige Funktion, die natürlichen
Zahlen die Werte $\Gamma(n+1) = n!$ der Fakultät annimmt.
Indem man eine beliebige Funktion $f(z)$ addiert, die auf alle
natürlichen Zahlen verschwindet, also $f(n)=0$ für $n\in\mathbb{N}$,
erhält man eine weitere Funktion, die auf natürlichen Zahlen
die Werte der Fakultät annimmt.
Ein Beispiel einer solchen Funktion ist
\begin{equation}
z\mapsto f(z)=\Gamma(z) + \sin \pi z,
\label{buch:rekursion:eqn:gammaalternative}
\end{equation}
die Funktion $f(z)=\sin\pi z$ verschwindet sogar auf allen ganzen
Zahlen.

In Abbildung~\ref{buch:rekursion:fig:gamma} ist die Gamma-Funktion
in rot geplotet, die Funktion~\eqref{buch:rekursion:eqn:gammaalternative}
in grün.
Die Punkte $(n,(n-1)!)$ sind in blau bezeichnet, sie sind beiden Graphen
gemeinsam.

\subsubsection{Pol erster Ordnung bei $z=0$}
Wir haben zu prüfen, dass sowohl der Wert $\Gamma(1)$ korrekt ist als
auch die Rekursionsformel~\eqref{buch:rekursion:eqn:gammadef} gilt.
Der Wert für $z=1$ ist
\begin{align*}
\Gamma(1)
&=
\int_0^\infty t^{1-1}e^{-t}\,dt
=
\left[ -e^{-t} \right]_0^\infty
=
1.
\end{align*}
Für die Rekursionsformel kann mit Hilfe von partieller Integration
bekommen:
\begin{align*}
\Gamma(z+1)
&=
\int_0^\infty t^{z+1-1}e^{-t}\,dt
=
\biggl[-t^{z}e^{-t}\biggr]_0^\infty
+
\int_0^\infty z t^{z-1}e^{-t}\,dt
\\
&=
z
\int_0^\infty
t^{z-1}e^{-t}\,dt
=
z \Gamma(z).
\end{align*}

Für $0<z<\varepsilon$ für eine $\varepsilon >0$ folgt aus der 
Funktionalgleichung
\[
\Gamma(z) = \frac{\Gamma(1+z)}{z}.
\]
Da $\Gamma(1)=1$ ist und $\Gamma$ eine in einer
Umgebung von $1$ stetige Funktion ist, kann sie in der Form
\(
\Gamma(1+z)=\Gamma(1) + zf(z)
\)
schreiben, wobei  $f(z)$ eine differenzierbare Funktion ist mit
$f'(1)=\Gamma'(1)$.
Daraus ergibt sich für $\Gamma(z)$ der Ausdruck
\[
\Gamma(z) = \frac{\Gamma(1)}{z} + f(z) = \frac{1}{z} + f(z).
\]
Die Gamma-Funktion hat daher and er Stelle $z=0$ einen Pol erster Ordnung.

\subsubsection{Ausdehnung auf $\operatorname{Re}z<0$}
Die Integralformel konvergiert nicht für $\operatorname{Re}z\le 0$.
Durch analytische Fortsetzung, wie sie im
Abschnitt~\ref{buch:funktionentheorie:section:fortsetzung}
beschrieben wird, kann die Funktion auf ganz $\mathbb{C}$ ausgedehnt
werden, mit Ausnahme einzelner Pole.
Die Funktionalgleichung gilt natürlich für alle $z\in\mathbb{C}$,
für die $\Gamma(z)$ definiert ist.
In einer Umgebung von $z=-n$ gilt
\[
\Gamma(z)
=
\frac{\Gamma(z+1)}{z}
=
\frac{\Gamma(z+2)}{z(z+1)}
=
\frac{\Gamma(z+3)}{z(z+1)(z+2)}
=
\dots
=
\frac{\Gamma(z+n)}{z(z+1)(z+2)\cdots(z+n-1)}
\]
Keiner der Faktoren im Nenner verschwindet in der Nähe von $z=-n$, der
Zähler hat aber einen Pol erster Ordnung an dieser Stelle.
Daher hat auch der Quotient einen Pol erster Ordnung.
Abbildung~\ref{buch:rekursion:fig:gamma} zeigt die Pole bei den
nicht negativen ganzen Zahlen.






%
% linear.tex
%
% (c) 2021 Prof Dr Andreas Müller, OST Ostschweizer Fachhochschule
%
\section{Lineare Rekursionsgleichung mit konstanten Koeffizienten
\label{buch:rekursion:section:linear}}
\rhead{Lineare Rekursionsgleichungen}
Die Funktionalgleichung der Gamma-Funktion, die im
Abschnitt~\ref{buch:rekursion:section:gamma} untersucht wurde,
hat die Form einer linearen Rekursionsgleichung
\[
\Gamma(x+1) = x\Gamma(x),\qquad \Gamma(1) = 1.
\]
Gleichungen, die Werte einer Funktion für verschiedene
Argument in Beziehung setzen, heissen {\em Funktionalgleichungen}.
\index{Funktionalgleichung}%
Es war überraschend schwierig, eine Lösung für Funktionalgleichung
der Gamma-Funktion für beliebige komplexe $x$ zu finden.
In diesem Abschnitt soll daher eine Klasse von Rekursionsgleichungen
näher untersucht werden, für die einfache Lösungen möglich sind.

\subsection{Lineare Differenzengleichungen}

\subsection{Lösung mit Polynomfunktionen}







%
% hypergeometrisch.tex
%
% (c) 2021 Prof Dr Andreas Müller, OST Ostschweizer Fachhochschule
%
\section{Hypergeometrische Differentialgleichung
\label{buch:differentialgleichungen:section:hypergeometrisch}}
Die hypergeometrische Funktion $\mathstrut_2F1(a,b;c;x)$ wurde in
Abschnitt~\ref{buch:rekursion:section:hypergeometrische-funktion}
als Potenzreihe mit sehr speziellen Koeffizienten, die sich aus
Pochhammer-Symbolen.
Es stellt sich aber heraus, dass man sie auch als Lösung einer
gewöhnlichen Differentialgleichung bekommen kann, die bereits
Euler studiert hat.

\subsection{Die Eulersche hypergeometrische Differentialgleichung
\label{buch:differentialgleichung:subsection:euler-hypergeometrisch}}
Die hypergeometrische Funktion $\mathstrut_2F_1(a,b;c;x)$ ist eine
Lösung der {\em Eulerschen hypergeometrischen Differentialgleichung}
(zu unterscheiden von der Eulerschen Differentialgleichung, die sich
immer auf eine lineare Differentialgleichung mit konstanten Koeffizienten
reduzieren lässt)
\begin{equation}
x(1-x) \frac{d^2y}{dx^2} + (c-(a+b+1)x)\frac{dy}{dx} - ab y = 0
\label{buch:differentialgleichungen:hypergeo:eulerdgl}
\end{equation}
Wir prüfen dies nach, indem wir die Definition der hypergeometrischen
Funktion 
\begin{align*}
y(x)
&=
\mathstrut_2F_1(a,b;c;x)
=
\sum_{k=0}^\infty
\frac{(a)_k(b)_k}{(c)_k} \frac{x^k}{k!}
\intertext{mit den Ableitungen}
y'(x)
&=
\sum_{k=1}^\infty 
\frac{(a)_k(b)_k}{(c)_k} \frac{x^{k-1}}{(k-1)!}
\\
y''(x)
&=
\sum_{k=2}^\infty 
\frac{(a)_k(b)_k}{(c)_k} \frac{x^{k-2}}{(k-2)!}
\end{align*}
einsetzen.
Die Gleichung, die sich ergibt, ist
\begin{align*}
0
&=
x(1-x)
\sum_{k=2}^\infty
\frac{(a)_k(b)_k}{(c)_k}\frac{x^{k-2}}{(k-2)!}
+
(c-(a+b+1)x)
\sum_{k=1}^\infty
\frac{(a)_k(b)_k}{(c)_k}\frac{x^{k-1}}{(k-1)!}
-ab
\sum_{k=0}^\infty
\frac{(a)_k(b)_k}{(c)_k} \frac{x^k}{k!}
\\
&=
\sum_{k=2}^\infty
\frac{(a)_k(b)_k}{(c)_k}\frac{x^{k-1}}{(k-2)!}
-
\sum_{k=2}^\infty
\frac{(a)_k(b)_k}{(c)_k}\frac{x^k}{(k-2)!}
+
c\sum_{k=1}^\infty
\frac{(a)_k(b)_k}{(c)_k}\frac{x^{k-1}}{(k-1)!}
\\
&\qquad
-(a+b+1)
\sum_{k=1}^\infty
\frac{(a)_k(b)_k}{(c)_k}\frac{x^k}{(k-1)!}
-ab
\sum_{k=0}^\infty
\frac{(a)_k(b)_k}{(c)_k} \frac{x^k}{k!}
\\
&=
\sum_{k=1}^\infty
\frac{(a)_{k+1}(b)_{k+1}}{(c)_{k+1}}\frac{x^k}{(k-1)!}
-
\sum_{k=2}^\infty
\frac{(a)_k(b)_k}{(c)_k}\frac{x^k}{(k-2)!}
+
c\sum_{k=0}^\infty
\frac{(a)_{k+1}(b)_{k+1}}{(c)_{k+1}}\frac{x^k}{k!}
\\
&\qquad
-(a+b+1)
\sum_{k=1}^\infty
\frac{(a)_k(b)_k}{(c)_k}\frac{x^k}{(k-1)!}
-ab
\sum_{k=0}^\infty
\frac{(a)_k(b)_k}{(c)_k} \frac{x^k}{k!}.
\end{align*}
Zum konstanten Koeffizienten für $k=0$ tragen nur die dritte und letzte
Summe bei, dies sind die Terme
\[
c\frac{(a)_1(b)_1}{(c)_1}-ab\frac{(a)_0(b)_0}{(c)_0}
=
c\frac{ab}{c}-ab\frac{1\cdot 1}{1}
=
0.
\]
Für den linearen Term $k=1$ kommen je ein Term aus der ersten aund vierten
Summe hinzu, dies ergibt
\begin{align*}
&\phantom{\mathstrut=\mathstrut}
\frac{(a)_2(b)_2}{(c)_2}
+c\frac{(a)_2(b)_2}{(c)_2}
-(a+b+1)\frac{(a)_1(b)_1}{(c)_1}
-ab\frac{(a)_1(b)_1}{(c)_1}
\\
&=
\frac{a(a+1)b(b+1)}{c(c+1)}
(1+c)
-(ab+a+b+1)
\frac{ab}{c}
\\
&=
\frac{a(a+1)b(b+1)}{c}
-
(a+1)(b+1)\frac{ab}{c}
=0.
\end{align*}
Durch Koeffizientenvergleich erhalten wir für $k\ge 2$ 
\begin{align*}
0
&=
\frac{(a)_{k+1}(b)_{k+1}}{(c)_{k+1}} \frac1{(k-1)!} 
-
\frac{(a)_k(b)_k}{(c)_k} \frac1{(k-2)!} 
+
c\frac{(a)_{k+1}(b)_{k+1}}{(c)_{k+1}} \frac{1}{k!}
\\
&\qquad
-(a+b+1)\frac{(a)_k(b)_k}{(c)_k}\frac{1}{(k-1)!}
-ab \frac{(a)_k(b)_k}{(c)_k}\frac{1}{k!}
\\
&=
\frac{(a)_k(b)_k}{(c)_{k+1}}
\frac{1}{k!}
\biggl(
(a+k)(b+k)k
-(c+k)(k-1)k
+
c(a+k)(b+k)
\\
&\qquad
\qquad
\qquad
-(a+b+1)(c+k)k
-ab(c+k)
\biggr).
\intertext{Der zweite, vierte und fünfte Term können zu}
&=
\frac{(a)_k(b)_k}{(c)_{k+1}}
\frac{1}{k!}
\biggl(
(a+k)(b+k)k
+
c(a+k)(b+k)
-(ab+ak+bk+k^2)(c+k)
\biggr)
\intertext{zusammengefasst werden.
Der Faktor $(ab+ak+bk+k^2)$ kann als Produkt $(a+k)(b+k)$ faktorisiert
werden, der dann als gemeinsamer Faktor aus allen Termen ausgeklammert
werden kann:}
&=
\frac{(a)_k(b)_k}{(c)_{k+1}}
\frac{1}{k!}
\biggl(
(a+k)(b+k)k
+
c(a+k)(b+k)
-(a+k)(b+k)(c+k)
\biggr)
\\
&=
\frac{(a)_{k+1}(b)_{k+1}}{(c)_{k+1}}
\frac{1}{k!}
\biggl(
k
+
c
-(c+k)
\biggr)
=0.
\end{align*}
Damit ist gezeigt, dass $\mathstrut_2F_1(a,b;c;x)$ eine Lösung
der Differentialgleichung ist.

Die hypergeometrische Reihe kann auch direkt mit Hilfe der
Potenzreihenmethode als Lösung der Differentialgleichung gefunden 
werden.

\subsection{Lösung als verallgemeinerte Potenzreihe}
Da die hypergeometrische Reihe eine Differentialgleichung
zweiter Ordnung mit einer Singularität bei $x=0$ ist, 
kann man versuchen eine zweite, linear unabhängige Lösung mit
Hilfe der Methode der verallgemeinerten Potenzreihen zu finden.
Dazu setzt man die Lösung in der Form
\begin{align*}
y_2(x)
&=
\sum_{k=0}^\infty a_kx^{\varrho+k}
&
&\Rightarrow&
y_2'(x)
&=
\sum_{k=0}^\infty (\varrho+k)a_kx^{\varrho+k-1}
\\
&&
&&
y_2''(x)
&=
\sum_{k=0}^\infty (\varrho+k)(\varrho+k-1)a_kx^{\varrho+k-2}
\end{align*}
an, wobei $a_0\ne 0$ sein soll.
Einsetzen in die Differentialgleichung ergibt
\begin{align*}
0&=
x(1-x)y_2''(x) + (c-(a+b+1)x) y_2'(x) -aby_2(x)
\\
&=
x(1-x)
\sum_{k=0}^\infty (\varrho+k)(\varrho+k-1)a_kx^{\varrho+k-2}
+
(c-(a+b+1)x)
\sum_{k=0}^\infty (\varrho+k)a_kx^{\varrho+k-1}
-
abx^{\varrho}\sum_{k=0}^\infty a_kx^{\varrho+k}
\\
&=
-\sum_{k=0}^\infty (\varrho+k)(\varrho+k-1)a_kx^{\varrho+k}
+
\sum_{k=0}^\infty (\varrho+k)(\varrho+k-1)a_kx^{\varrho+k-1}
+
c
\sum_{k=0}^\infty (\varrho+k)a_kx^{\varrho+k-1}
\\
&\qquad
-
(a+b+1)
\sum_{k=0}^\infty (\varrho+k)a_kx^{\varrho+k}
-
ab
\sum_{k=0}^\infty a_kx^{\varrho+k}.
\intertext{Durch Verschiebung des Summationsindex in der zweiten
und dritten Summe wird der Koeffizientenvergleich etwas
einfacher}
&=
-\sum_{k=0}^\infty (\varrho+k)(\varrho+k-1)a_kx^{\varrho+k}
+
\sum_{k=-1}^\infty (\varrho+k+1)(\varrho+k)a_{k+1}x^{\varrho+k}
+
c
\sum_{k=-1}^\infty (\varrho+k+1)a_{k+1}x^{\varrho+k}
\\
&\qquad
-
(a+b+1)
\sum_{k=0}^\infty (\varrho+k)a_kx^{\varrho+k}
-
ab
\sum_{k=0}^\infty a_kx^{\varrho+k}
\\
&=
-\sum_{k=0}^\infty (\varrho+k)(\varrho+k-1)a_kx^{\varrho+k}
+
\sum_{k=-1}^\infty (\varrho+k+1)(\varrho+k+c)a_{k+1}x^{\varrho+k}
\\
&\qquad
-
\sum_{k=0}^\infty ((\varrho+k)(a+b+1)+ab)a_kx^{\varrho+k}
\\
&=
\bigl(
\varrho(\varrho-1)
+c\varrho \bigr)
x^{\varrho-1}
+
\sum_{k=0}^\infty
\bigl(
-(\varrho+k)(\varrho+k-1)a_k
+(\varrho+k+1)(\varrho+k+c)a_{k+1}
\\
&
\qquad
\qquad
\qquad
\qquad
\qquad
\qquad
-((\varrho+k)(a+b+1)+ab)a_k
\bigr)
x^{\varrho+k}.
\end{align*}
Aus dem ersten Term kann man die Indexgleichung
\[
0
=
\varrho(\varrho-1)+c\varrho
=
\varrho(\varrho-1+c)
\]
ablesen, die die Nullstellen $\varrho=0$ und $\varrho=1-c$ hat.
Die Nullstelle $\varrho=0$ ergibt natürlich die bereits gefundene
hypergeometrische Reihe.

Nach Einsetzen der zweiten Lösung der Indexgleichung in der Summe
legt der Koeffizientenvergleich eine Beziehung
\begin{align}
0
&=
\bigl(
-(k-c+1)(k-c)
-(k-c+1)(a+b+1)+ab
\bigr)a_k
+
(k-c+2)(k+1)
a_{k+1} 
\notag
\intertext{zwischen $a_k$ und $a_{k+1}$ fest.
Daraus kann man den Quotienten aufeinanderfolgender
Koeffizienten als}
\frac{a_{k+1}}{a_k}
&=
\frac{
-(k-c+1)(k-c)
-(k-c+1)(a+b+1)+ab
}{
\notag
(k-c+2)(k+1)
}
\\
&=
%(%i4) factor(coeff(y,q,0))
%(%o4)                  - (k - c + a + 1) (k - c + b + 1)
%(%i5) factor(coeff(y,q,1))
%(%o5)                         (k + 1) (k - c + 2)
\frac{
(a-c+1+k)
(b-c+1+k)
}{
(2-c+k)(k+1)
}
\label{buch:differentialgleichungen:hypergeo:verallgkoef}
\end{align}
finden.
Setzt man $a_0=1$, ist die zweite Lösung ist also wieder eine
hypergeometrische Funktion.%, nämlich
%\[
%y_2(x)
%=
%x^{1-c}
%\sum_{k=0}^\infty \frac{(a-c+1)_k(b-c+1)_k}{(2-c)_k}\frac{x^k}{k!}
%=
%x^{1-c}
%\mathstrut_2F_1\biggl(\begin{matrix}a-c+1,b-c+1\\2-c\end{matrix};x\biggr)
%\]
Diese Lösung ist aber nur möglich, wenn in
\eqref{buch:differentialgleichungen:hypergeo:verallgkoef}
der Nenner nicht verschwindet, d.~h.~$2-c+k\ne 0$
oder $c \ne k+2$ für all natürlichen $k$.
$c$ darf also kein natürliche Zahl $\ge 2$ sein.
Wir fassen die Resultate dieses Abschnitts im folgenden Satz zusammen.

\begin{satz}
Die eulersche hypergeometrische Differentialgleichung
\begin{equation}
x(1-x)\frac{d^2y}{dx^2}
+(c+(a+b+1)x)\frac{dy}{dx}
-ab y
=
0
\end{equation}
hat die Lösung
\[
y_1(x)
=
\mathstrut_2F_1\biggl(\begin{matrix}a,b\\c\end{matrix};x\biggr).
\]
Falls $c-2\not\in \mathbb{N}$ ist, ist
\[
y_2(x)
=
x^{1-c} \mathstrut_2F_1\biggl(\begin{matrix}a-c+1,b-c+1\\2-c\end{matrix};x\biggr)
\]
eine zweite, linear unabhängige Lösung.
\end{satz}

%
% Die verallgemeinerte hypergeometrische Differentialgleichung
%
\subsection{Verallgemeinerte hypergeometrische Differentialgleichung}
% https://de.wikipedia.org/wiki/Verallgemeinerte_hypergeometrische_Funktion







\section*{Übungsaufgaben}
\rhead{Übungsaufgaben}
\aufgabetoplevel{chapters/040-rekursion/uebungsaufgaben}
\begin{uebungsaufgaben}
%\uebungsaufgabe{0}
\uebungsaufgabe{1}
\uebungsaufgabe{2}
\end{uebungsaufgaben}



% analytisch definierte spezielle Funktionen
%%
% chapter.tex -- Beschreibung des Inhaltes
%
% (c) 2021 Prof Dr Andreas Müller, Hochschule Rapperswil
%
% !TeX spellcheck = de_CH
\chapter{Spezielle Funktionen und Rekursion
\label{buch:chapter:rekursion}}
\lhead{Spezielle Funktionen und Rekursion}
\rhead{}

%
% gamma.tex -- Abschnitt über die Gamma-funktion
%
% (c) 2021 Prof Dr Andreas Müller, OST Ostschweizer Fachhochschule
%
\section{Die Gamma-Funktion
\label{buch:rekursion:section:gamma}}
Die Fakultät $x!$ kann rekursiv durch 
\[
	x! = x\cdot (x-1)! \qquad\text{und}\qquad 0!=1
\]
für alle natürlichen Zahlen $x\in\mathbb{N}$ definiert werden.
Äquivalent damit ist eine Funktion 
\begin{equation}
\Gamma(x+1) = x\Gamma(x)
\qquad\text{und}\qquad 
\Gamma(1)=1.
\label{buch:rekursion:eqn:gammadef}
\end{equation}
Kann man eine reelle oder komplexe Funktion finden, die die
Funktionalgleichung~\eqref{buch:rekursion:eqn:gammadef}
erfüllt und damit die Fakultät auf beliebige Argumente ausdehnt?

\subsection{Integralformel für die Gamma-Funktion}
Euler hat die folgende Integraldefinition der Gamma-Funktion gegeben.

\begin{definition}
\label{buch:rekursion:def:gamma}
Die Gamma-Funktion ist die Funktion 
\[
\Gamma
\colon
\{z\in\mathbb{C} \mid \operatorname{Re}z>0\}
\to \mathbb{C}
:
z
\mapsto
\Gamma(z) = \int_0^\infty t^{x-1}e^{-t}\,dt
\]
\end{definition}

Man beachte, dass das Integral für $x=0$ nicht definiert ist, eine
Potenzreihenentwicklung um einen Punkt $x_0$ auf der positiven reellen
Achse kann also höchstens den Konvergenzradius $\varrho=|x_0|$ haben.

\begin{figure}
\centering
\includegraphics{chapters/040-rekursion/images/gammaplot.pdf}
\caption{Graph der Gamma-Funktion $z\mapsto\Gamma(z)$ und der alternativen
Funktion $\Gamma(z)+\sin(\pi z)$, die für ganzzahlige Argumente ebenfalls
die Werte der Fakultät annimmt.
\label{buch:rekursion:fig:gamma}}
\end{figure}

\subsubsection{Alternative Lösungen}
Die Funktion $\Gamma(z)$ ist nicht die einzige Funktion, die natürlichen
Zahlen die Werte $\Gamma(n+1) = n!$ der Fakultät annimmt.
Indem man eine beliebige Funktion $f(z)$ addiert, die auf alle
natürlichen Zahlen verschwindet, also $f(n)=0$ für $n\in\mathbb{N}$,
erhält man eine weitere Funktion, die auf natürlichen Zahlen
die Werte der Fakultät annimmt.
Ein Beispiel einer solchen Funktion ist
\begin{equation}
z\mapsto f(z)=\Gamma(z) + \sin \pi z,
\label{buch:rekursion:eqn:gammaalternative}
\end{equation}
die Funktion $f(z)=\sin\pi z$ verschwindet sogar auf allen ganzen
Zahlen.

In Abbildung~\ref{buch:rekursion:fig:gamma} ist die Gamma-Funktion
in rot geplotet, die Funktion~\eqref{buch:rekursion:eqn:gammaalternative}
in grün.
Die Punkte $(n,(n-1)!)$ sind in blau bezeichnet, sie sind beiden Graphen
gemeinsam.

\subsubsection{Pol erster Ordnung bei $z=0$}
Wir haben zu prüfen, dass sowohl der Wert $\Gamma(1)$ korrekt ist als
auch die Rekursionsformel~\eqref{buch:rekursion:eqn:gammadef} gilt.
Der Wert für $z=1$ ist
\begin{align*}
\Gamma(1)
&=
\int_0^\infty t^{1-1}e^{-t}\,dt
=
\left[ -e^{-t} \right]_0^\infty
=
1.
\end{align*}
Für die Rekursionsformel kann mit Hilfe von partieller Integration
bekommen:
\begin{align*}
\Gamma(z+1)
&=
\int_0^\infty t^{z+1-1}e^{-t}\,dt
=
\biggl[-t^{z}e^{-t}\biggr]_0^\infty
+
\int_0^\infty z t^{z-1}e^{-t}\,dt
\\
&=
z
\int_0^\infty
t^{z-1}e^{-t}\,dt
=
z \Gamma(z).
\end{align*}

Für $0<z<\varepsilon$ für eine $\varepsilon >0$ folgt aus der 
Funktionalgleichung
\[
\Gamma(z) = \frac{\Gamma(1+z)}{z}.
\]
Da $\Gamma(1)=1$ ist und $\Gamma$ eine in einer
Umgebung von $1$ stetige Funktion ist, kann sie in der Form
\(
\Gamma(1+z)=\Gamma(1) + zf(z)
\)
schreiben, wobei  $f(z)$ eine differenzierbare Funktion ist mit
$f'(1)=\Gamma'(1)$.
Daraus ergibt sich für $\Gamma(z)$ der Ausdruck
\[
\Gamma(z) = \frac{\Gamma(1)}{z} + f(z) = \frac{1}{z} + f(z).
\]
Die Gamma-Funktion hat daher and er Stelle $z=0$ einen Pol erster Ordnung.

\subsubsection{Ausdehnung auf $\operatorname{Re}z<0$}
Die Integralformel konvergiert nicht für $\operatorname{Re}z\le 0$.
Durch analytische Fortsetzung, wie sie im
Abschnitt~\ref{buch:funktionentheorie:section:fortsetzung}
beschrieben wird, kann die Funktion auf ganz $\mathbb{C}$ ausgedehnt
werden, mit Ausnahme einzelner Pole.
Die Funktionalgleichung gilt natürlich für alle $z\in\mathbb{C}$,
für die $\Gamma(z)$ definiert ist.
In einer Umgebung von $z=-n$ gilt
\[
\Gamma(z)
=
\frac{\Gamma(z+1)}{z}
=
\frac{\Gamma(z+2)}{z(z+1)}
=
\frac{\Gamma(z+3)}{z(z+1)(z+2)}
=
\dots
=
\frac{\Gamma(z+n)}{z(z+1)(z+2)\cdots(z+n-1)}
\]
Keiner der Faktoren im Nenner verschwindet in der Nähe von $z=-n$, der
Zähler hat aber einen Pol erster Ordnung an dieser Stelle.
Daher hat auch der Quotient einen Pol erster Ordnung.
Abbildung~\ref{buch:rekursion:fig:gamma} zeigt die Pole bei den
nicht negativen ganzen Zahlen.






%
% linear.tex
%
% (c) 2021 Prof Dr Andreas Müller, OST Ostschweizer Fachhochschule
%
\section{Lineare Rekursionsgleichung mit konstanten Koeffizienten
\label{buch:rekursion:section:linear}}
\rhead{Lineare Rekursionsgleichungen}
Die Funktionalgleichung der Gamma-Funktion, die im
Abschnitt~\ref{buch:rekursion:section:gamma} untersucht wurde,
hat die Form einer linearen Rekursionsgleichung
\[
\Gamma(x+1) = x\Gamma(x),\qquad \Gamma(1) = 1.
\]
Gleichungen, die Werte einer Funktion für verschiedene
Argument in Beziehung setzen, heissen {\em Funktionalgleichungen}.
\index{Funktionalgleichung}%
Es war überraschend schwierig, eine Lösung für Funktionalgleichung
der Gamma-Funktion für beliebige komplexe $x$ zu finden.
In diesem Abschnitt soll daher eine Klasse von Rekursionsgleichungen
näher untersucht werden, für die einfache Lösungen möglich sind.

\subsection{Lineare Differenzengleichungen}

\subsection{Lösung mit Polynomfunktionen}







%
% hypergeometrisch.tex
%
% (c) 2021 Prof Dr Andreas Müller, OST Ostschweizer Fachhochschule
%
\section{Hypergeometrische Differentialgleichung
\label{buch:differentialgleichungen:section:hypergeometrisch}}
Die hypergeometrische Funktion $\mathstrut_2F1(a,b;c;x)$ wurde in
Abschnitt~\ref{buch:rekursion:section:hypergeometrische-funktion}
als Potenzreihe mit sehr speziellen Koeffizienten, die sich aus
Pochhammer-Symbolen.
Es stellt sich aber heraus, dass man sie auch als Lösung einer
gewöhnlichen Differentialgleichung bekommen kann, die bereits
Euler studiert hat.

\subsection{Die Eulersche hypergeometrische Differentialgleichung
\label{buch:differentialgleichung:subsection:euler-hypergeometrisch}}
Die hypergeometrische Funktion $\mathstrut_2F_1(a,b;c;x)$ ist eine
Lösung der {\em Eulerschen hypergeometrischen Differentialgleichung}
(zu unterscheiden von der Eulerschen Differentialgleichung, die sich
immer auf eine lineare Differentialgleichung mit konstanten Koeffizienten
reduzieren lässt)
\begin{equation}
x(1-x) \frac{d^2y}{dx^2} + (c-(a+b+1)x)\frac{dy}{dx} - ab y = 0
\label{buch:differentialgleichungen:hypergeo:eulerdgl}
\end{equation}
Wir prüfen dies nach, indem wir die Definition der hypergeometrischen
Funktion 
\begin{align*}
y(x)
&=
\mathstrut_2F_1(a,b;c;x)
=
\sum_{k=0}^\infty
\frac{(a)_k(b)_k}{(c)_k} \frac{x^k}{k!}
\intertext{mit den Ableitungen}
y'(x)
&=
\sum_{k=1}^\infty 
\frac{(a)_k(b)_k}{(c)_k} \frac{x^{k-1}}{(k-1)!}
\\
y''(x)
&=
\sum_{k=2}^\infty 
\frac{(a)_k(b)_k}{(c)_k} \frac{x^{k-2}}{(k-2)!}
\end{align*}
einsetzen.
Die Gleichung, die sich ergibt, ist
\begin{align*}
0
&=
x(1-x)
\sum_{k=2}^\infty
\frac{(a)_k(b)_k}{(c)_k}\frac{x^{k-2}}{(k-2)!}
+
(c-(a+b+1)x)
\sum_{k=1}^\infty
\frac{(a)_k(b)_k}{(c)_k}\frac{x^{k-1}}{(k-1)!}
-ab
\sum_{k=0}^\infty
\frac{(a)_k(b)_k}{(c)_k} \frac{x^k}{k!}
\\
&=
\sum_{k=2}^\infty
\frac{(a)_k(b)_k}{(c)_k}\frac{x^{k-1}}{(k-2)!}
-
\sum_{k=2}^\infty
\frac{(a)_k(b)_k}{(c)_k}\frac{x^k}{(k-2)!}
+
c\sum_{k=1}^\infty
\frac{(a)_k(b)_k}{(c)_k}\frac{x^{k-1}}{(k-1)!}
\\
&\qquad
-(a+b+1)
\sum_{k=1}^\infty
\frac{(a)_k(b)_k}{(c)_k}\frac{x^k}{(k-1)!}
-ab
\sum_{k=0}^\infty
\frac{(a)_k(b)_k}{(c)_k} \frac{x^k}{k!}
\\
&=
\sum_{k=1}^\infty
\frac{(a)_{k+1}(b)_{k+1}}{(c)_{k+1}}\frac{x^k}{(k-1)!}
-
\sum_{k=2}^\infty
\frac{(a)_k(b)_k}{(c)_k}\frac{x^k}{(k-2)!}
+
c\sum_{k=0}^\infty
\frac{(a)_{k+1}(b)_{k+1}}{(c)_{k+1}}\frac{x^k}{k!}
\\
&\qquad
-(a+b+1)
\sum_{k=1}^\infty
\frac{(a)_k(b)_k}{(c)_k}\frac{x^k}{(k-1)!}
-ab
\sum_{k=0}^\infty
\frac{(a)_k(b)_k}{(c)_k} \frac{x^k}{k!}.
\end{align*}
Zum konstanten Koeffizienten für $k=0$ tragen nur die dritte und letzte
Summe bei, dies sind die Terme
\[
c\frac{(a)_1(b)_1}{(c)_1}-ab\frac{(a)_0(b)_0}{(c)_0}
=
c\frac{ab}{c}-ab\frac{1\cdot 1}{1}
=
0.
\]
Für den linearen Term $k=1$ kommen je ein Term aus der ersten aund vierten
Summe hinzu, dies ergibt
\begin{align*}
&\phantom{\mathstrut=\mathstrut}
\frac{(a)_2(b)_2}{(c)_2}
+c\frac{(a)_2(b)_2}{(c)_2}
-(a+b+1)\frac{(a)_1(b)_1}{(c)_1}
-ab\frac{(a)_1(b)_1}{(c)_1}
\\
&=
\frac{a(a+1)b(b+1)}{c(c+1)}
(1+c)
-(ab+a+b+1)
\frac{ab}{c}
\\
&=
\frac{a(a+1)b(b+1)}{c}
-
(a+1)(b+1)\frac{ab}{c}
=0.
\end{align*}
Durch Koeffizientenvergleich erhalten wir für $k\ge 2$ 
\begin{align*}
0
&=
\frac{(a)_{k+1}(b)_{k+1}}{(c)_{k+1}} \frac1{(k-1)!} 
-
\frac{(a)_k(b)_k}{(c)_k} \frac1{(k-2)!} 
+
c\frac{(a)_{k+1}(b)_{k+1}}{(c)_{k+1}} \frac{1}{k!}
\\
&\qquad
-(a+b+1)\frac{(a)_k(b)_k}{(c)_k}\frac{1}{(k-1)!}
-ab \frac{(a)_k(b)_k}{(c)_k}\frac{1}{k!}
\\
&=
\frac{(a)_k(b)_k}{(c)_{k+1}}
\frac{1}{k!}
\biggl(
(a+k)(b+k)k
-(c+k)(k-1)k
+
c(a+k)(b+k)
\\
&\qquad
\qquad
\qquad
-(a+b+1)(c+k)k
-ab(c+k)
\biggr).
\intertext{Der zweite, vierte und fünfte Term können zu}
&=
\frac{(a)_k(b)_k}{(c)_{k+1}}
\frac{1}{k!}
\biggl(
(a+k)(b+k)k
+
c(a+k)(b+k)
-(ab+ak+bk+k^2)(c+k)
\biggr)
\intertext{zusammengefasst werden.
Der Faktor $(ab+ak+bk+k^2)$ kann als Produkt $(a+k)(b+k)$ faktorisiert
werden, der dann als gemeinsamer Faktor aus allen Termen ausgeklammert
werden kann:}
&=
\frac{(a)_k(b)_k}{(c)_{k+1}}
\frac{1}{k!}
\biggl(
(a+k)(b+k)k
+
c(a+k)(b+k)
-(a+k)(b+k)(c+k)
\biggr)
\\
&=
\frac{(a)_{k+1}(b)_{k+1}}{(c)_{k+1}}
\frac{1}{k!}
\biggl(
k
+
c
-(c+k)
\biggr)
=0.
\end{align*}
Damit ist gezeigt, dass $\mathstrut_2F_1(a,b;c;x)$ eine Lösung
der Differentialgleichung ist.

Die hypergeometrische Reihe kann auch direkt mit Hilfe der
Potenzreihenmethode als Lösung der Differentialgleichung gefunden 
werden.

\subsection{Lösung als verallgemeinerte Potenzreihe}
Da die hypergeometrische Reihe eine Differentialgleichung
zweiter Ordnung mit einer Singularität bei $x=0$ ist, 
kann man versuchen eine zweite, linear unabhängige Lösung mit
Hilfe der Methode der verallgemeinerten Potenzreihen zu finden.
Dazu setzt man die Lösung in der Form
\begin{align*}
y_2(x)
&=
\sum_{k=0}^\infty a_kx^{\varrho+k}
&
&\Rightarrow&
y_2'(x)
&=
\sum_{k=0}^\infty (\varrho+k)a_kx^{\varrho+k-1}
\\
&&
&&
y_2''(x)
&=
\sum_{k=0}^\infty (\varrho+k)(\varrho+k-1)a_kx^{\varrho+k-2}
\end{align*}
an, wobei $a_0\ne 0$ sein soll.
Einsetzen in die Differentialgleichung ergibt
\begin{align*}
0&=
x(1-x)y_2''(x) + (c-(a+b+1)x) y_2'(x) -aby_2(x)
\\
&=
x(1-x)
\sum_{k=0}^\infty (\varrho+k)(\varrho+k-1)a_kx^{\varrho+k-2}
+
(c-(a+b+1)x)
\sum_{k=0}^\infty (\varrho+k)a_kx^{\varrho+k-1}
-
abx^{\varrho}\sum_{k=0}^\infty a_kx^{\varrho+k}
\\
&=
-\sum_{k=0}^\infty (\varrho+k)(\varrho+k-1)a_kx^{\varrho+k}
+
\sum_{k=0}^\infty (\varrho+k)(\varrho+k-1)a_kx^{\varrho+k-1}
+
c
\sum_{k=0}^\infty (\varrho+k)a_kx^{\varrho+k-1}
\\
&\qquad
-
(a+b+1)
\sum_{k=0}^\infty (\varrho+k)a_kx^{\varrho+k}
-
ab
\sum_{k=0}^\infty a_kx^{\varrho+k}.
\intertext{Durch Verschiebung des Summationsindex in der zweiten
und dritten Summe wird der Koeffizientenvergleich etwas
einfacher}
&=
-\sum_{k=0}^\infty (\varrho+k)(\varrho+k-1)a_kx^{\varrho+k}
+
\sum_{k=-1}^\infty (\varrho+k+1)(\varrho+k)a_{k+1}x^{\varrho+k}
+
c
\sum_{k=-1}^\infty (\varrho+k+1)a_{k+1}x^{\varrho+k}
\\
&\qquad
-
(a+b+1)
\sum_{k=0}^\infty (\varrho+k)a_kx^{\varrho+k}
-
ab
\sum_{k=0}^\infty a_kx^{\varrho+k}
\\
&=
-\sum_{k=0}^\infty (\varrho+k)(\varrho+k-1)a_kx^{\varrho+k}
+
\sum_{k=-1}^\infty (\varrho+k+1)(\varrho+k+c)a_{k+1}x^{\varrho+k}
\\
&\qquad
-
\sum_{k=0}^\infty ((\varrho+k)(a+b+1)+ab)a_kx^{\varrho+k}
\\
&=
\bigl(
\varrho(\varrho-1)
+c\varrho \bigr)
x^{\varrho-1}
+
\sum_{k=0}^\infty
\bigl(
-(\varrho+k)(\varrho+k-1)a_k
+(\varrho+k+1)(\varrho+k+c)a_{k+1}
\\
&
\qquad
\qquad
\qquad
\qquad
\qquad
\qquad
-((\varrho+k)(a+b+1)+ab)a_k
\bigr)
x^{\varrho+k}.
\end{align*}
Aus dem ersten Term kann man die Indexgleichung
\[
0
=
\varrho(\varrho-1)+c\varrho
=
\varrho(\varrho-1+c)
\]
ablesen, die die Nullstellen $\varrho=0$ und $\varrho=1-c$ hat.
Die Nullstelle $\varrho=0$ ergibt natürlich die bereits gefundene
hypergeometrische Reihe.

Nach Einsetzen der zweiten Lösung der Indexgleichung in der Summe
legt der Koeffizientenvergleich eine Beziehung
\begin{align}
0
&=
\bigl(
-(k-c+1)(k-c)
-(k-c+1)(a+b+1)+ab
\bigr)a_k
+
(k-c+2)(k+1)
a_{k+1} 
\notag
\intertext{zwischen $a_k$ und $a_{k+1}$ fest.
Daraus kann man den Quotienten aufeinanderfolgender
Koeffizienten als}
\frac{a_{k+1}}{a_k}
&=
\frac{
-(k-c+1)(k-c)
-(k-c+1)(a+b+1)+ab
}{
\notag
(k-c+2)(k+1)
}
\\
&=
%(%i4) factor(coeff(y,q,0))
%(%o4)                  - (k - c + a + 1) (k - c + b + 1)
%(%i5) factor(coeff(y,q,1))
%(%o5)                         (k + 1) (k - c + 2)
\frac{
(a-c+1+k)
(b-c+1+k)
}{
(2-c+k)(k+1)
}
\label{buch:differentialgleichungen:hypergeo:verallgkoef}
\end{align}
finden.
Setzt man $a_0=1$, ist die zweite Lösung ist also wieder eine
hypergeometrische Funktion.%, nämlich
%\[
%y_2(x)
%=
%x^{1-c}
%\sum_{k=0}^\infty \frac{(a-c+1)_k(b-c+1)_k}{(2-c)_k}\frac{x^k}{k!}
%=
%x^{1-c}
%\mathstrut_2F_1\biggl(\begin{matrix}a-c+1,b-c+1\\2-c\end{matrix};x\biggr)
%\]
Diese Lösung ist aber nur möglich, wenn in
\eqref{buch:differentialgleichungen:hypergeo:verallgkoef}
der Nenner nicht verschwindet, d.~h.~$2-c+k\ne 0$
oder $c \ne k+2$ für all natürlichen $k$.
$c$ darf also kein natürliche Zahl $\ge 2$ sein.
Wir fassen die Resultate dieses Abschnitts im folgenden Satz zusammen.

\begin{satz}
Die eulersche hypergeometrische Differentialgleichung
\begin{equation}
x(1-x)\frac{d^2y}{dx^2}
+(c+(a+b+1)x)\frac{dy}{dx}
-ab y
=
0
\end{equation}
hat die Lösung
\[
y_1(x)
=
\mathstrut_2F_1\biggl(\begin{matrix}a,b\\c\end{matrix};x\biggr).
\]
Falls $c-2\not\in \mathbb{N}$ ist, ist
\[
y_2(x)
=
x^{1-c} \mathstrut_2F_1\biggl(\begin{matrix}a-c+1,b-c+1\\2-c\end{matrix};x\biggr)
\]
eine zweite, linear unabhängige Lösung.
\end{satz}

%
% Die verallgemeinerte hypergeometrische Differentialgleichung
%
\subsection{Verallgemeinerte hypergeometrische Differentialgleichung}
% https://de.wikipedia.org/wiki/Verallgemeinerte_hypergeometrische_Funktion







\section*{Übungsaufgaben}
\rhead{Übungsaufgaben}
\aufgabetoplevel{chapters/040-rekursion/uebungsaufgaben}
\begin{uebungsaufgaben}
%\uebungsaufgabe{0}
\uebungsaufgabe{1}
\uebungsaufgabe{2}
\end{uebungsaufgaben}


%%
% chapter.tex -- Beschreibung des Inhaltes
%
% (c) 2021 Prof Dr Andreas Müller, Hochschule Rapperswil
%
% !TeX spellcheck = de_CH
\chapter{Spezielle Funktionen und Rekursion
\label{buch:chapter:rekursion}}
\lhead{Spezielle Funktionen und Rekursion}
\rhead{}

%
% gamma.tex -- Abschnitt über die Gamma-funktion
%
% (c) 2021 Prof Dr Andreas Müller, OST Ostschweizer Fachhochschule
%
\section{Die Gamma-Funktion
\label{buch:rekursion:section:gamma}}
Die Fakultät $x!$ kann rekursiv durch 
\[
	x! = x\cdot (x-1)! \qquad\text{und}\qquad 0!=1
\]
für alle natürlichen Zahlen $x\in\mathbb{N}$ definiert werden.
Äquivalent damit ist eine Funktion 
\begin{equation}
\Gamma(x+1) = x\Gamma(x)
\qquad\text{und}\qquad 
\Gamma(1)=1.
\label{buch:rekursion:eqn:gammadef}
\end{equation}
Kann man eine reelle oder komplexe Funktion finden, die die
Funktionalgleichung~\eqref{buch:rekursion:eqn:gammadef}
erfüllt und damit die Fakultät auf beliebige Argumente ausdehnt?

\subsection{Integralformel für die Gamma-Funktion}
Euler hat die folgende Integraldefinition der Gamma-Funktion gegeben.

\begin{definition}
\label{buch:rekursion:def:gamma}
Die Gamma-Funktion ist die Funktion 
\[
\Gamma
\colon
\{z\in\mathbb{C} \mid \operatorname{Re}z>0\}
\to \mathbb{C}
:
z
\mapsto
\Gamma(z) = \int_0^\infty t^{x-1}e^{-t}\,dt
\]
\end{definition}

Man beachte, dass das Integral für $x=0$ nicht definiert ist, eine
Potenzreihenentwicklung um einen Punkt $x_0$ auf der positiven reellen
Achse kann also höchstens den Konvergenzradius $\varrho=|x_0|$ haben.

\begin{figure}
\centering
\includegraphics{chapters/040-rekursion/images/gammaplot.pdf}
\caption{Graph der Gamma-Funktion $z\mapsto\Gamma(z)$ und der alternativen
Funktion $\Gamma(z)+\sin(\pi z)$, die für ganzzahlige Argumente ebenfalls
die Werte der Fakultät annimmt.
\label{buch:rekursion:fig:gamma}}
\end{figure}

\subsubsection{Alternative Lösungen}
Die Funktion $\Gamma(z)$ ist nicht die einzige Funktion, die natürlichen
Zahlen die Werte $\Gamma(n+1) = n!$ der Fakultät annimmt.
Indem man eine beliebige Funktion $f(z)$ addiert, die auf alle
natürlichen Zahlen verschwindet, also $f(n)=0$ für $n\in\mathbb{N}$,
erhält man eine weitere Funktion, die auf natürlichen Zahlen
die Werte der Fakultät annimmt.
Ein Beispiel einer solchen Funktion ist
\begin{equation}
z\mapsto f(z)=\Gamma(z) + \sin \pi z,
\label{buch:rekursion:eqn:gammaalternative}
\end{equation}
die Funktion $f(z)=\sin\pi z$ verschwindet sogar auf allen ganzen
Zahlen.

In Abbildung~\ref{buch:rekursion:fig:gamma} ist die Gamma-Funktion
in rot geplotet, die Funktion~\eqref{buch:rekursion:eqn:gammaalternative}
in grün.
Die Punkte $(n,(n-1)!)$ sind in blau bezeichnet, sie sind beiden Graphen
gemeinsam.

\subsubsection{Pol erster Ordnung bei $z=0$}
Wir haben zu prüfen, dass sowohl der Wert $\Gamma(1)$ korrekt ist als
auch die Rekursionsformel~\eqref{buch:rekursion:eqn:gammadef} gilt.
Der Wert für $z=1$ ist
\begin{align*}
\Gamma(1)
&=
\int_0^\infty t^{1-1}e^{-t}\,dt
=
\left[ -e^{-t} \right]_0^\infty
=
1.
\end{align*}
Für die Rekursionsformel kann mit Hilfe von partieller Integration
bekommen:
\begin{align*}
\Gamma(z+1)
&=
\int_0^\infty t^{z+1-1}e^{-t}\,dt
=
\biggl[-t^{z}e^{-t}\biggr]_0^\infty
+
\int_0^\infty z t^{z-1}e^{-t}\,dt
\\
&=
z
\int_0^\infty
t^{z-1}e^{-t}\,dt
=
z \Gamma(z).
\end{align*}

Für $0<z<\varepsilon$ für eine $\varepsilon >0$ folgt aus der 
Funktionalgleichung
\[
\Gamma(z) = \frac{\Gamma(1+z)}{z}.
\]
Da $\Gamma(1)=1$ ist und $\Gamma$ eine in einer
Umgebung von $1$ stetige Funktion ist, kann sie in der Form
\(
\Gamma(1+z)=\Gamma(1) + zf(z)
\)
schreiben, wobei  $f(z)$ eine differenzierbare Funktion ist mit
$f'(1)=\Gamma'(1)$.
Daraus ergibt sich für $\Gamma(z)$ der Ausdruck
\[
\Gamma(z) = \frac{\Gamma(1)}{z} + f(z) = \frac{1}{z} + f(z).
\]
Die Gamma-Funktion hat daher and er Stelle $z=0$ einen Pol erster Ordnung.

\subsubsection{Ausdehnung auf $\operatorname{Re}z<0$}
Die Integralformel konvergiert nicht für $\operatorname{Re}z\le 0$.
Durch analytische Fortsetzung, wie sie im
Abschnitt~\ref{buch:funktionentheorie:section:fortsetzung}
beschrieben wird, kann die Funktion auf ganz $\mathbb{C}$ ausgedehnt
werden, mit Ausnahme einzelner Pole.
Die Funktionalgleichung gilt natürlich für alle $z\in\mathbb{C}$,
für die $\Gamma(z)$ definiert ist.
In einer Umgebung von $z=-n$ gilt
\[
\Gamma(z)
=
\frac{\Gamma(z+1)}{z}
=
\frac{\Gamma(z+2)}{z(z+1)}
=
\frac{\Gamma(z+3)}{z(z+1)(z+2)}
=
\dots
=
\frac{\Gamma(z+n)}{z(z+1)(z+2)\cdots(z+n-1)}
\]
Keiner der Faktoren im Nenner verschwindet in der Nähe von $z=-n$, der
Zähler hat aber einen Pol erster Ordnung an dieser Stelle.
Daher hat auch der Quotient einen Pol erster Ordnung.
Abbildung~\ref{buch:rekursion:fig:gamma} zeigt die Pole bei den
nicht negativen ganzen Zahlen.






%
% linear.tex
%
% (c) 2021 Prof Dr Andreas Müller, OST Ostschweizer Fachhochschule
%
\section{Lineare Rekursionsgleichung mit konstanten Koeffizienten
\label{buch:rekursion:section:linear}}
\rhead{Lineare Rekursionsgleichungen}
Die Funktionalgleichung der Gamma-Funktion, die im
Abschnitt~\ref{buch:rekursion:section:gamma} untersucht wurde,
hat die Form einer linearen Rekursionsgleichung
\[
\Gamma(x+1) = x\Gamma(x),\qquad \Gamma(1) = 1.
\]
Gleichungen, die Werte einer Funktion für verschiedene
Argument in Beziehung setzen, heissen {\em Funktionalgleichungen}.
\index{Funktionalgleichung}%
Es war überraschend schwierig, eine Lösung für Funktionalgleichung
der Gamma-Funktion für beliebige komplexe $x$ zu finden.
In diesem Abschnitt soll daher eine Klasse von Rekursionsgleichungen
näher untersucht werden, für die einfache Lösungen möglich sind.

\subsection{Lineare Differenzengleichungen}

\subsection{Lösung mit Polynomfunktionen}







%
% hypergeometrisch.tex
%
% (c) 2021 Prof Dr Andreas Müller, OST Ostschweizer Fachhochschule
%
\section{Hypergeometrische Differentialgleichung
\label{buch:differentialgleichungen:section:hypergeometrisch}}
Die hypergeometrische Funktion $\mathstrut_2F1(a,b;c;x)$ wurde in
Abschnitt~\ref{buch:rekursion:section:hypergeometrische-funktion}
als Potenzreihe mit sehr speziellen Koeffizienten, die sich aus
Pochhammer-Symbolen.
Es stellt sich aber heraus, dass man sie auch als Lösung einer
gewöhnlichen Differentialgleichung bekommen kann, die bereits
Euler studiert hat.

\subsection{Die Eulersche hypergeometrische Differentialgleichung
\label{buch:differentialgleichung:subsection:euler-hypergeometrisch}}
Die hypergeometrische Funktion $\mathstrut_2F_1(a,b;c;x)$ ist eine
Lösung der {\em Eulerschen hypergeometrischen Differentialgleichung}
(zu unterscheiden von der Eulerschen Differentialgleichung, die sich
immer auf eine lineare Differentialgleichung mit konstanten Koeffizienten
reduzieren lässt)
\begin{equation}
x(1-x) \frac{d^2y}{dx^2} + (c-(a+b+1)x)\frac{dy}{dx} - ab y = 0
\label{buch:differentialgleichungen:hypergeo:eulerdgl}
\end{equation}
Wir prüfen dies nach, indem wir die Definition der hypergeometrischen
Funktion 
\begin{align*}
y(x)
&=
\mathstrut_2F_1(a,b;c;x)
=
\sum_{k=0}^\infty
\frac{(a)_k(b)_k}{(c)_k} \frac{x^k}{k!}
\intertext{mit den Ableitungen}
y'(x)
&=
\sum_{k=1}^\infty 
\frac{(a)_k(b)_k}{(c)_k} \frac{x^{k-1}}{(k-1)!}
\\
y''(x)
&=
\sum_{k=2}^\infty 
\frac{(a)_k(b)_k}{(c)_k} \frac{x^{k-2}}{(k-2)!}
\end{align*}
einsetzen.
Die Gleichung, die sich ergibt, ist
\begin{align*}
0
&=
x(1-x)
\sum_{k=2}^\infty
\frac{(a)_k(b)_k}{(c)_k}\frac{x^{k-2}}{(k-2)!}
+
(c-(a+b+1)x)
\sum_{k=1}^\infty
\frac{(a)_k(b)_k}{(c)_k}\frac{x^{k-1}}{(k-1)!}
-ab
\sum_{k=0}^\infty
\frac{(a)_k(b)_k}{(c)_k} \frac{x^k}{k!}
\\
&=
\sum_{k=2}^\infty
\frac{(a)_k(b)_k}{(c)_k}\frac{x^{k-1}}{(k-2)!}
-
\sum_{k=2}^\infty
\frac{(a)_k(b)_k}{(c)_k}\frac{x^k}{(k-2)!}
+
c\sum_{k=1}^\infty
\frac{(a)_k(b)_k}{(c)_k}\frac{x^{k-1}}{(k-1)!}
\\
&\qquad
-(a+b+1)
\sum_{k=1}^\infty
\frac{(a)_k(b)_k}{(c)_k}\frac{x^k}{(k-1)!}
-ab
\sum_{k=0}^\infty
\frac{(a)_k(b)_k}{(c)_k} \frac{x^k}{k!}
\\
&=
\sum_{k=1}^\infty
\frac{(a)_{k+1}(b)_{k+1}}{(c)_{k+1}}\frac{x^k}{(k-1)!}
-
\sum_{k=2}^\infty
\frac{(a)_k(b)_k}{(c)_k}\frac{x^k}{(k-2)!}
+
c\sum_{k=0}^\infty
\frac{(a)_{k+1}(b)_{k+1}}{(c)_{k+1}}\frac{x^k}{k!}
\\
&\qquad
-(a+b+1)
\sum_{k=1}^\infty
\frac{(a)_k(b)_k}{(c)_k}\frac{x^k}{(k-1)!}
-ab
\sum_{k=0}^\infty
\frac{(a)_k(b)_k}{(c)_k} \frac{x^k}{k!}.
\end{align*}
Zum konstanten Koeffizienten für $k=0$ tragen nur die dritte und letzte
Summe bei, dies sind die Terme
\[
c\frac{(a)_1(b)_1}{(c)_1}-ab\frac{(a)_0(b)_0}{(c)_0}
=
c\frac{ab}{c}-ab\frac{1\cdot 1}{1}
=
0.
\]
Für den linearen Term $k=1$ kommen je ein Term aus der ersten aund vierten
Summe hinzu, dies ergibt
\begin{align*}
&\phantom{\mathstrut=\mathstrut}
\frac{(a)_2(b)_2}{(c)_2}
+c\frac{(a)_2(b)_2}{(c)_2}
-(a+b+1)\frac{(a)_1(b)_1}{(c)_1}
-ab\frac{(a)_1(b)_1}{(c)_1}
\\
&=
\frac{a(a+1)b(b+1)}{c(c+1)}
(1+c)
-(ab+a+b+1)
\frac{ab}{c}
\\
&=
\frac{a(a+1)b(b+1)}{c}
-
(a+1)(b+1)\frac{ab}{c}
=0.
\end{align*}
Durch Koeffizientenvergleich erhalten wir für $k\ge 2$ 
\begin{align*}
0
&=
\frac{(a)_{k+1}(b)_{k+1}}{(c)_{k+1}} \frac1{(k-1)!} 
-
\frac{(a)_k(b)_k}{(c)_k} \frac1{(k-2)!} 
+
c\frac{(a)_{k+1}(b)_{k+1}}{(c)_{k+1}} \frac{1}{k!}
\\
&\qquad
-(a+b+1)\frac{(a)_k(b)_k}{(c)_k}\frac{1}{(k-1)!}
-ab \frac{(a)_k(b)_k}{(c)_k}\frac{1}{k!}
\\
&=
\frac{(a)_k(b)_k}{(c)_{k+1}}
\frac{1}{k!}
\biggl(
(a+k)(b+k)k
-(c+k)(k-1)k
+
c(a+k)(b+k)
\\
&\qquad
\qquad
\qquad
-(a+b+1)(c+k)k
-ab(c+k)
\biggr).
\intertext{Der zweite, vierte und fünfte Term können zu}
&=
\frac{(a)_k(b)_k}{(c)_{k+1}}
\frac{1}{k!}
\biggl(
(a+k)(b+k)k
+
c(a+k)(b+k)
-(ab+ak+bk+k^2)(c+k)
\biggr)
\intertext{zusammengefasst werden.
Der Faktor $(ab+ak+bk+k^2)$ kann als Produkt $(a+k)(b+k)$ faktorisiert
werden, der dann als gemeinsamer Faktor aus allen Termen ausgeklammert
werden kann:}
&=
\frac{(a)_k(b)_k}{(c)_{k+1}}
\frac{1}{k!}
\biggl(
(a+k)(b+k)k
+
c(a+k)(b+k)
-(a+k)(b+k)(c+k)
\biggr)
\\
&=
\frac{(a)_{k+1}(b)_{k+1}}{(c)_{k+1}}
\frac{1}{k!}
\biggl(
k
+
c
-(c+k)
\biggr)
=0.
\end{align*}
Damit ist gezeigt, dass $\mathstrut_2F_1(a,b;c;x)$ eine Lösung
der Differentialgleichung ist.

Die hypergeometrische Reihe kann auch direkt mit Hilfe der
Potenzreihenmethode als Lösung der Differentialgleichung gefunden 
werden.

\subsection{Lösung als verallgemeinerte Potenzreihe}
Da die hypergeometrische Reihe eine Differentialgleichung
zweiter Ordnung mit einer Singularität bei $x=0$ ist, 
kann man versuchen eine zweite, linear unabhängige Lösung mit
Hilfe der Methode der verallgemeinerten Potenzreihen zu finden.
Dazu setzt man die Lösung in der Form
\begin{align*}
y_2(x)
&=
\sum_{k=0}^\infty a_kx^{\varrho+k}
&
&\Rightarrow&
y_2'(x)
&=
\sum_{k=0}^\infty (\varrho+k)a_kx^{\varrho+k-1}
\\
&&
&&
y_2''(x)
&=
\sum_{k=0}^\infty (\varrho+k)(\varrho+k-1)a_kx^{\varrho+k-2}
\end{align*}
an, wobei $a_0\ne 0$ sein soll.
Einsetzen in die Differentialgleichung ergibt
\begin{align*}
0&=
x(1-x)y_2''(x) + (c-(a+b+1)x) y_2'(x) -aby_2(x)
\\
&=
x(1-x)
\sum_{k=0}^\infty (\varrho+k)(\varrho+k-1)a_kx^{\varrho+k-2}
+
(c-(a+b+1)x)
\sum_{k=0}^\infty (\varrho+k)a_kx^{\varrho+k-1}
-
abx^{\varrho}\sum_{k=0}^\infty a_kx^{\varrho+k}
\\
&=
-\sum_{k=0}^\infty (\varrho+k)(\varrho+k-1)a_kx^{\varrho+k}
+
\sum_{k=0}^\infty (\varrho+k)(\varrho+k-1)a_kx^{\varrho+k-1}
+
c
\sum_{k=0}^\infty (\varrho+k)a_kx^{\varrho+k-1}
\\
&\qquad
-
(a+b+1)
\sum_{k=0}^\infty (\varrho+k)a_kx^{\varrho+k}
-
ab
\sum_{k=0}^\infty a_kx^{\varrho+k}.
\intertext{Durch Verschiebung des Summationsindex in der zweiten
und dritten Summe wird der Koeffizientenvergleich etwas
einfacher}
&=
-\sum_{k=0}^\infty (\varrho+k)(\varrho+k-1)a_kx^{\varrho+k}
+
\sum_{k=-1}^\infty (\varrho+k+1)(\varrho+k)a_{k+1}x^{\varrho+k}
+
c
\sum_{k=-1}^\infty (\varrho+k+1)a_{k+1}x^{\varrho+k}
\\
&\qquad
-
(a+b+1)
\sum_{k=0}^\infty (\varrho+k)a_kx^{\varrho+k}
-
ab
\sum_{k=0}^\infty a_kx^{\varrho+k}
\\
&=
-\sum_{k=0}^\infty (\varrho+k)(\varrho+k-1)a_kx^{\varrho+k}
+
\sum_{k=-1}^\infty (\varrho+k+1)(\varrho+k+c)a_{k+1}x^{\varrho+k}
\\
&\qquad
-
\sum_{k=0}^\infty ((\varrho+k)(a+b+1)+ab)a_kx^{\varrho+k}
\\
&=
\bigl(
\varrho(\varrho-1)
+c\varrho \bigr)
x^{\varrho-1}
+
\sum_{k=0}^\infty
\bigl(
-(\varrho+k)(\varrho+k-1)a_k
+(\varrho+k+1)(\varrho+k+c)a_{k+1}
\\
&
\qquad
\qquad
\qquad
\qquad
\qquad
\qquad
-((\varrho+k)(a+b+1)+ab)a_k
\bigr)
x^{\varrho+k}.
\end{align*}
Aus dem ersten Term kann man die Indexgleichung
\[
0
=
\varrho(\varrho-1)+c\varrho
=
\varrho(\varrho-1+c)
\]
ablesen, die die Nullstellen $\varrho=0$ und $\varrho=1-c$ hat.
Die Nullstelle $\varrho=0$ ergibt natürlich die bereits gefundene
hypergeometrische Reihe.

Nach Einsetzen der zweiten Lösung der Indexgleichung in der Summe
legt der Koeffizientenvergleich eine Beziehung
\begin{align}
0
&=
\bigl(
-(k-c+1)(k-c)
-(k-c+1)(a+b+1)+ab
\bigr)a_k
+
(k-c+2)(k+1)
a_{k+1} 
\notag
\intertext{zwischen $a_k$ und $a_{k+1}$ fest.
Daraus kann man den Quotienten aufeinanderfolgender
Koeffizienten als}
\frac{a_{k+1}}{a_k}
&=
\frac{
-(k-c+1)(k-c)
-(k-c+1)(a+b+1)+ab
}{
\notag
(k-c+2)(k+1)
}
\\
&=
%(%i4) factor(coeff(y,q,0))
%(%o4)                  - (k - c + a + 1) (k - c + b + 1)
%(%i5) factor(coeff(y,q,1))
%(%o5)                         (k + 1) (k - c + 2)
\frac{
(a-c+1+k)
(b-c+1+k)
}{
(2-c+k)(k+1)
}
\label{buch:differentialgleichungen:hypergeo:verallgkoef}
\end{align}
finden.
Setzt man $a_0=1$, ist die zweite Lösung ist also wieder eine
hypergeometrische Funktion.%, nämlich
%\[
%y_2(x)
%=
%x^{1-c}
%\sum_{k=0}^\infty \frac{(a-c+1)_k(b-c+1)_k}{(2-c)_k}\frac{x^k}{k!}
%=
%x^{1-c}
%\mathstrut_2F_1\biggl(\begin{matrix}a-c+1,b-c+1\\2-c\end{matrix};x\biggr)
%\]
Diese Lösung ist aber nur möglich, wenn in
\eqref{buch:differentialgleichungen:hypergeo:verallgkoef}
der Nenner nicht verschwindet, d.~h.~$2-c+k\ne 0$
oder $c \ne k+2$ für all natürlichen $k$.
$c$ darf also kein natürliche Zahl $\ge 2$ sein.
Wir fassen die Resultate dieses Abschnitts im folgenden Satz zusammen.

\begin{satz}
Die eulersche hypergeometrische Differentialgleichung
\begin{equation}
x(1-x)\frac{d^2y}{dx^2}
+(c+(a+b+1)x)\frac{dy}{dx}
-ab y
=
0
\end{equation}
hat die Lösung
\[
y_1(x)
=
\mathstrut_2F_1\biggl(\begin{matrix}a,b\\c\end{matrix};x\biggr).
\]
Falls $c-2\not\in \mathbb{N}$ ist, ist
\[
y_2(x)
=
x^{1-c} \mathstrut_2F_1\biggl(\begin{matrix}a-c+1,b-c+1\\2-c\end{matrix};x\biggr)
\]
eine zweite, linear unabhängige Lösung.
\end{satz}

%
% Die verallgemeinerte hypergeometrische Differentialgleichung
%
\subsection{Verallgemeinerte hypergeometrische Differentialgleichung}
% https://de.wikipedia.org/wiki/Verallgemeinerte_hypergeometrische_Funktion







\section*{Übungsaufgaben}
\rhead{Übungsaufgaben}
\aufgabetoplevel{chapters/040-rekursion/uebungsaufgaben}
\begin{uebungsaufgaben}
%\uebungsaufgabe{0}
\uebungsaufgabe{1}
\uebungsaufgabe{2}
\end{uebungsaufgaben}


%%
% chapter.tex -- Beschreibung des Inhaltes
%
% (c) 2021 Prof Dr Andreas Müller, Hochschule Rapperswil
%
% !TeX spellcheck = de_CH
\chapter{Spezielle Funktionen und Rekursion
\label{buch:chapter:rekursion}}
\lhead{Spezielle Funktionen und Rekursion}
\rhead{}

%
% gamma.tex -- Abschnitt über die Gamma-funktion
%
% (c) 2021 Prof Dr Andreas Müller, OST Ostschweizer Fachhochschule
%
\section{Die Gamma-Funktion
\label{buch:rekursion:section:gamma}}
Die Fakultät $x!$ kann rekursiv durch 
\[
	x! = x\cdot (x-1)! \qquad\text{und}\qquad 0!=1
\]
für alle natürlichen Zahlen $x\in\mathbb{N}$ definiert werden.
Äquivalent damit ist eine Funktion 
\begin{equation}
\Gamma(x+1) = x\Gamma(x)
\qquad\text{und}\qquad 
\Gamma(1)=1.
\label{buch:rekursion:eqn:gammadef}
\end{equation}
Kann man eine reelle oder komplexe Funktion finden, die die
Funktionalgleichung~\eqref{buch:rekursion:eqn:gammadef}
erfüllt und damit die Fakultät auf beliebige Argumente ausdehnt?

\subsection{Integralformel für die Gamma-Funktion}
Euler hat die folgende Integraldefinition der Gamma-Funktion gegeben.

\begin{definition}
\label{buch:rekursion:def:gamma}
Die Gamma-Funktion ist die Funktion 
\[
\Gamma
\colon
\{z\in\mathbb{C} \mid \operatorname{Re}z>0\}
\to \mathbb{C}
:
z
\mapsto
\Gamma(z) = \int_0^\infty t^{x-1}e^{-t}\,dt
\]
\end{definition}

Man beachte, dass das Integral für $x=0$ nicht definiert ist, eine
Potenzreihenentwicklung um einen Punkt $x_0$ auf der positiven reellen
Achse kann also höchstens den Konvergenzradius $\varrho=|x_0|$ haben.

\begin{figure}
\centering
\includegraphics{chapters/040-rekursion/images/gammaplot.pdf}
\caption{Graph der Gamma-Funktion $z\mapsto\Gamma(z)$ und der alternativen
Funktion $\Gamma(z)+\sin(\pi z)$, die für ganzzahlige Argumente ebenfalls
die Werte der Fakultät annimmt.
\label{buch:rekursion:fig:gamma}}
\end{figure}

\subsubsection{Alternative Lösungen}
Die Funktion $\Gamma(z)$ ist nicht die einzige Funktion, die natürlichen
Zahlen die Werte $\Gamma(n+1) = n!$ der Fakultät annimmt.
Indem man eine beliebige Funktion $f(z)$ addiert, die auf alle
natürlichen Zahlen verschwindet, also $f(n)=0$ für $n\in\mathbb{N}$,
erhält man eine weitere Funktion, die auf natürlichen Zahlen
die Werte der Fakultät annimmt.
Ein Beispiel einer solchen Funktion ist
\begin{equation}
z\mapsto f(z)=\Gamma(z) + \sin \pi z,
\label{buch:rekursion:eqn:gammaalternative}
\end{equation}
die Funktion $f(z)=\sin\pi z$ verschwindet sogar auf allen ganzen
Zahlen.

In Abbildung~\ref{buch:rekursion:fig:gamma} ist die Gamma-Funktion
in rot geplotet, die Funktion~\eqref{buch:rekursion:eqn:gammaalternative}
in grün.
Die Punkte $(n,(n-1)!)$ sind in blau bezeichnet, sie sind beiden Graphen
gemeinsam.

\subsubsection{Pol erster Ordnung bei $z=0$}
Wir haben zu prüfen, dass sowohl der Wert $\Gamma(1)$ korrekt ist als
auch die Rekursionsformel~\eqref{buch:rekursion:eqn:gammadef} gilt.
Der Wert für $z=1$ ist
\begin{align*}
\Gamma(1)
&=
\int_0^\infty t^{1-1}e^{-t}\,dt
=
\left[ -e^{-t} \right]_0^\infty
=
1.
\end{align*}
Für die Rekursionsformel kann mit Hilfe von partieller Integration
bekommen:
\begin{align*}
\Gamma(z+1)
&=
\int_0^\infty t^{z+1-1}e^{-t}\,dt
=
\biggl[-t^{z}e^{-t}\biggr]_0^\infty
+
\int_0^\infty z t^{z-1}e^{-t}\,dt
\\
&=
z
\int_0^\infty
t^{z-1}e^{-t}\,dt
=
z \Gamma(z).
\end{align*}

Für $0<z<\varepsilon$ für eine $\varepsilon >0$ folgt aus der 
Funktionalgleichung
\[
\Gamma(z) = \frac{\Gamma(1+z)}{z}.
\]
Da $\Gamma(1)=1$ ist und $\Gamma$ eine in einer
Umgebung von $1$ stetige Funktion ist, kann sie in der Form
\(
\Gamma(1+z)=\Gamma(1) + zf(z)
\)
schreiben, wobei  $f(z)$ eine differenzierbare Funktion ist mit
$f'(1)=\Gamma'(1)$.
Daraus ergibt sich für $\Gamma(z)$ der Ausdruck
\[
\Gamma(z) = \frac{\Gamma(1)}{z} + f(z) = \frac{1}{z} + f(z).
\]
Die Gamma-Funktion hat daher and er Stelle $z=0$ einen Pol erster Ordnung.

\subsubsection{Ausdehnung auf $\operatorname{Re}z<0$}
Die Integralformel konvergiert nicht für $\operatorname{Re}z\le 0$.
Durch analytische Fortsetzung, wie sie im
Abschnitt~\ref{buch:funktionentheorie:section:fortsetzung}
beschrieben wird, kann die Funktion auf ganz $\mathbb{C}$ ausgedehnt
werden, mit Ausnahme einzelner Pole.
Die Funktionalgleichung gilt natürlich für alle $z\in\mathbb{C}$,
für die $\Gamma(z)$ definiert ist.
In einer Umgebung von $z=-n$ gilt
\[
\Gamma(z)
=
\frac{\Gamma(z+1)}{z}
=
\frac{\Gamma(z+2)}{z(z+1)}
=
\frac{\Gamma(z+3)}{z(z+1)(z+2)}
=
\dots
=
\frac{\Gamma(z+n)}{z(z+1)(z+2)\cdots(z+n-1)}
\]
Keiner der Faktoren im Nenner verschwindet in der Nähe von $z=-n$, der
Zähler hat aber einen Pol erster Ordnung an dieser Stelle.
Daher hat auch der Quotient einen Pol erster Ordnung.
Abbildung~\ref{buch:rekursion:fig:gamma} zeigt die Pole bei den
nicht negativen ganzen Zahlen.






%
% linear.tex
%
% (c) 2021 Prof Dr Andreas Müller, OST Ostschweizer Fachhochschule
%
\section{Lineare Rekursionsgleichung mit konstanten Koeffizienten
\label{buch:rekursion:section:linear}}
\rhead{Lineare Rekursionsgleichungen}
Die Funktionalgleichung der Gamma-Funktion, die im
Abschnitt~\ref{buch:rekursion:section:gamma} untersucht wurde,
hat die Form einer linearen Rekursionsgleichung
\[
\Gamma(x+1) = x\Gamma(x),\qquad \Gamma(1) = 1.
\]
Gleichungen, die Werte einer Funktion für verschiedene
Argument in Beziehung setzen, heissen {\em Funktionalgleichungen}.
\index{Funktionalgleichung}%
Es war überraschend schwierig, eine Lösung für Funktionalgleichung
der Gamma-Funktion für beliebige komplexe $x$ zu finden.
In diesem Abschnitt soll daher eine Klasse von Rekursionsgleichungen
näher untersucht werden, für die einfache Lösungen möglich sind.

\subsection{Lineare Differenzengleichungen}

\subsection{Lösung mit Polynomfunktionen}







%
% hypergeometrisch.tex
%
% (c) 2021 Prof Dr Andreas Müller, OST Ostschweizer Fachhochschule
%
\section{Hypergeometrische Differentialgleichung
\label{buch:differentialgleichungen:section:hypergeometrisch}}
Die hypergeometrische Funktion $\mathstrut_2F1(a,b;c;x)$ wurde in
Abschnitt~\ref{buch:rekursion:section:hypergeometrische-funktion}
als Potenzreihe mit sehr speziellen Koeffizienten, die sich aus
Pochhammer-Symbolen.
Es stellt sich aber heraus, dass man sie auch als Lösung einer
gewöhnlichen Differentialgleichung bekommen kann, die bereits
Euler studiert hat.

\subsection{Die Eulersche hypergeometrische Differentialgleichung
\label{buch:differentialgleichung:subsection:euler-hypergeometrisch}}
Die hypergeometrische Funktion $\mathstrut_2F_1(a,b;c;x)$ ist eine
Lösung der {\em Eulerschen hypergeometrischen Differentialgleichung}
(zu unterscheiden von der Eulerschen Differentialgleichung, die sich
immer auf eine lineare Differentialgleichung mit konstanten Koeffizienten
reduzieren lässt)
\begin{equation}
x(1-x) \frac{d^2y}{dx^2} + (c-(a+b+1)x)\frac{dy}{dx} - ab y = 0
\label{buch:differentialgleichungen:hypergeo:eulerdgl}
\end{equation}
Wir prüfen dies nach, indem wir die Definition der hypergeometrischen
Funktion 
\begin{align*}
y(x)
&=
\mathstrut_2F_1(a,b;c;x)
=
\sum_{k=0}^\infty
\frac{(a)_k(b)_k}{(c)_k} \frac{x^k}{k!}
\intertext{mit den Ableitungen}
y'(x)
&=
\sum_{k=1}^\infty 
\frac{(a)_k(b)_k}{(c)_k} \frac{x^{k-1}}{(k-1)!}
\\
y''(x)
&=
\sum_{k=2}^\infty 
\frac{(a)_k(b)_k}{(c)_k} \frac{x^{k-2}}{(k-2)!}
\end{align*}
einsetzen.
Die Gleichung, die sich ergibt, ist
\begin{align*}
0
&=
x(1-x)
\sum_{k=2}^\infty
\frac{(a)_k(b)_k}{(c)_k}\frac{x^{k-2}}{(k-2)!}
+
(c-(a+b+1)x)
\sum_{k=1}^\infty
\frac{(a)_k(b)_k}{(c)_k}\frac{x^{k-1}}{(k-1)!}
-ab
\sum_{k=0}^\infty
\frac{(a)_k(b)_k}{(c)_k} \frac{x^k}{k!}
\\
&=
\sum_{k=2}^\infty
\frac{(a)_k(b)_k}{(c)_k}\frac{x^{k-1}}{(k-2)!}
-
\sum_{k=2}^\infty
\frac{(a)_k(b)_k}{(c)_k}\frac{x^k}{(k-2)!}
+
c\sum_{k=1}^\infty
\frac{(a)_k(b)_k}{(c)_k}\frac{x^{k-1}}{(k-1)!}
\\
&\qquad
-(a+b+1)
\sum_{k=1}^\infty
\frac{(a)_k(b)_k}{(c)_k}\frac{x^k}{(k-1)!}
-ab
\sum_{k=0}^\infty
\frac{(a)_k(b)_k}{(c)_k} \frac{x^k}{k!}
\\
&=
\sum_{k=1}^\infty
\frac{(a)_{k+1}(b)_{k+1}}{(c)_{k+1}}\frac{x^k}{(k-1)!}
-
\sum_{k=2}^\infty
\frac{(a)_k(b)_k}{(c)_k}\frac{x^k}{(k-2)!}
+
c\sum_{k=0}^\infty
\frac{(a)_{k+1}(b)_{k+1}}{(c)_{k+1}}\frac{x^k}{k!}
\\
&\qquad
-(a+b+1)
\sum_{k=1}^\infty
\frac{(a)_k(b)_k}{(c)_k}\frac{x^k}{(k-1)!}
-ab
\sum_{k=0}^\infty
\frac{(a)_k(b)_k}{(c)_k} \frac{x^k}{k!}.
\end{align*}
Zum konstanten Koeffizienten für $k=0$ tragen nur die dritte und letzte
Summe bei, dies sind die Terme
\[
c\frac{(a)_1(b)_1}{(c)_1}-ab\frac{(a)_0(b)_0}{(c)_0}
=
c\frac{ab}{c}-ab\frac{1\cdot 1}{1}
=
0.
\]
Für den linearen Term $k=1$ kommen je ein Term aus der ersten aund vierten
Summe hinzu, dies ergibt
\begin{align*}
&\phantom{\mathstrut=\mathstrut}
\frac{(a)_2(b)_2}{(c)_2}
+c\frac{(a)_2(b)_2}{(c)_2}
-(a+b+1)\frac{(a)_1(b)_1}{(c)_1}
-ab\frac{(a)_1(b)_1}{(c)_1}
\\
&=
\frac{a(a+1)b(b+1)}{c(c+1)}
(1+c)
-(ab+a+b+1)
\frac{ab}{c}
\\
&=
\frac{a(a+1)b(b+1)}{c}
-
(a+1)(b+1)\frac{ab}{c}
=0.
\end{align*}
Durch Koeffizientenvergleich erhalten wir für $k\ge 2$ 
\begin{align*}
0
&=
\frac{(a)_{k+1}(b)_{k+1}}{(c)_{k+1}} \frac1{(k-1)!} 
-
\frac{(a)_k(b)_k}{(c)_k} \frac1{(k-2)!} 
+
c\frac{(a)_{k+1}(b)_{k+1}}{(c)_{k+1}} \frac{1}{k!}
\\
&\qquad
-(a+b+1)\frac{(a)_k(b)_k}{(c)_k}\frac{1}{(k-1)!}
-ab \frac{(a)_k(b)_k}{(c)_k}\frac{1}{k!}
\\
&=
\frac{(a)_k(b)_k}{(c)_{k+1}}
\frac{1}{k!}
\biggl(
(a+k)(b+k)k
-(c+k)(k-1)k
+
c(a+k)(b+k)
\\
&\qquad
\qquad
\qquad
-(a+b+1)(c+k)k
-ab(c+k)
\biggr).
\intertext{Der zweite, vierte und fünfte Term können zu}
&=
\frac{(a)_k(b)_k}{(c)_{k+1}}
\frac{1}{k!}
\biggl(
(a+k)(b+k)k
+
c(a+k)(b+k)
-(ab+ak+bk+k^2)(c+k)
\biggr)
\intertext{zusammengefasst werden.
Der Faktor $(ab+ak+bk+k^2)$ kann als Produkt $(a+k)(b+k)$ faktorisiert
werden, der dann als gemeinsamer Faktor aus allen Termen ausgeklammert
werden kann:}
&=
\frac{(a)_k(b)_k}{(c)_{k+1}}
\frac{1}{k!}
\biggl(
(a+k)(b+k)k
+
c(a+k)(b+k)
-(a+k)(b+k)(c+k)
\biggr)
\\
&=
\frac{(a)_{k+1}(b)_{k+1}}{(c)_{k+1}}
\frac{1}{k!}
\biggl(
k
+
c
-(c+k)
\biggr)
=0.
\end{align*}
Damit ist gezeigt, dass $\mathstrut_2F_1(a,b;c;x)$ eine Lösung
der Differentialgleichung ist.

Die hypergeometrische Reihe kann auch direkt mit Hilfe der
Potenzreihenmethode als Lösung der Differentialgleichung gefunden 
werden.

\subsection{Lösung als verallgemeinerte Potenzreihe}
Da die hypergeometrische Reihe eine Differentialgleichung
zweiter Ordnung mit einer Singularität bei $x=0$ ist, 
kann man versuchen eine zweite, linear unabhängige Lösung mit
Hilfe der Methode der verallgemeinerten Potenzreihen zu finden.
Dazu setzt man die Lösung in der Form
\begin{align*}
y_2(x)
&=
\sum_{k=0}^\infty a_kx^{\varrho+k}
&
&\Rightarrow&
y_2'(x)
&=
\sum_{k=0}^\infty (\varrho+k)a_kx^{\varrho+k-1}
\\
&&
&&
y_2''(x)
&=
\sum_{k=0}^\infty (\varrho+k)(\varrho+k-1)a_kx^{\varrho+k-2}
\end{align*}
an, wobei $a_0\ne 0$ sein soll.
Einsetzen in die Differentialgleichung ergibt
\begin{align*}
0&=
x(1-x)y_2''(x) + (c-(a+b+1)x) y_2'(x) -aby_2(x)
\\
&=
x(1-x)
\sum_{k=0}^\infty (\varrho+k)(\varrho+k-1)a_kx^{\varrho+k-2}
+
(c-(a+b+1)x)
\sum_{k=0}^\infty (\varrho+k)a_kx^{\varrho+k-1}
-
abx^{\varrho}\sum_{k=0}^\infty a_kx^{\varrho+k}
\\
&=
-\sum_{k=0}^\infty (\varrho+k)(\varrho+k-1)a_kx^{\varrho+k}
+
\sum_{k=0}^\infty (\varrho+k)(\varrho+k-1)a_kx^{\varrho+k-1}
+
c
\sum_{k=0}^\infty (\varrho+k)a_kx^{\varrho+k-1}
\\
&\qquad
-
(a+b+1)
\sum_{k=0}^\infty (\varrho+k)a_kx^{\varrho+k}
-
ab
\sum_{k=0}^\infty a_kx^{\varrho+k}.
\intertext{Durch Verschiebung des Summationsindex in der zweiten
und dritten Summe wird der Koeffizientenvergleich etwas
einfacher}
&=
-\sum_{k=0}^\infty (\varrho+k)(\varrho+k-1)a_kx^{\varrho+k}
+
\sum_{k=-1}^\infty (\varrho+k+1)(\varrho+k)a_{k+1}x^{\varrho+k}
+
c
\sum_{k=-1}^\infty (\varrho+k+1)a_{k+1}x^{\varrho+k}
\\
&\qquad
-
(a+b+1)
\sum_{k=0}^\infty (\varrho+k)a_kx^{\varrho+k}
-
ab
\sum_{k=0}^\infty a_kx^{\varrho+k}
\\
&=
-\sum_{k=0}^\infty (\varrho+k)(\varrho+k-1)a_kx^{\varrho+k}
+
\sum_{k=-1}^\infty (\varrho+k+1)(\varrho+k+c)a_{k+1}x^{\varrho+k}
\\
&\qquad
-
\sum_{k=0}^\infty ((\varrho+k)(a+b+1)+ab)a_kx^{\varrho+k}
\\
&=
\bigl(
\varrho(\varrho-1)
+c\varrho \bigr)
x^{\varrho-1}
+
\sum_{k=0}^\infty
\bigl(
-(\varrho+k)(\varrho+k-1)a_k
+(\varrho+k+1)(\varrho+k+c)a_{k+1}
\\
&
\qquad
\qquad
\qquad
\qquad
\qquad
\qquad
-((\varrho+k)(a+b+1)+ab)a_k
\bigr)
x^{\varrho+k}.
\end{align*}
Aus dem ersten Term kann man die Indexgleichung
\[
0
=
\varrho(\varrho-1)+c\varrho
=
\varrho(\varrho-1+c)
\]
ablesen, die die Nullstellen $\varrho=0$ und $\varrho=1-c$ hat.
Die Nullstelle $\varrho=0$ ergibt natürlich die bereits gefundene
hypergeometrische Reihe.

Nach Einsetzen der zweiten Lösung der Indexgleichung in der Summe
legt der Koeffizientenvergleich eine Beziehung
\begin{align}
0
&=
\bigl(
-(k-c+1)(k-c)
-(k-c+1)(a+b+1)+ab
\bigr)a_k
+
(k-c+2)(k+1)
a_{k+1} 
\notag
\intertext{zwischen $a_k$ und $a_{k+1}$ fest.
Daraus kann man den Quotienten aufeinanderfolgender
Koeffizienten als}
\frac{a_{k+1}}{a_k}
&=
\frac{
-(k-c+1)(k-c)
-(k-c+1)(a+b+1)+ab
}{
\notag
(k-c+2)(k+1)
}
\\
&=
%(%i4) factor(coeff(y,q,0))
%(%o4)                  - (k - c + a + 1) (k - c + b + 1)
%(%i5) factor(coeff(y,q,1))
%(%o5)                         (k + 1) (k - c + 2)
\frac{
(a-c+1+k)
(b-c+1+k)
}{
(2-c+k)(k+1)
}
\label{buch:differentialgleichungen:hypergeo:verallgkoef}
\end{align}
finden.
Setzt man $a_0=1$, ist die zweite Lösung ist also wieder eine
hypergeometrische Funktion.%, nämlich
%\[
%y_2(x)
%=
%x^{1-c}
%\sum_{k=0}^\infty \frac{(a-c+1)_k(b-c+1)_k}{(2-c)_k}\frac{x^k}{k!}
%=
%x^{1-c}
%\mathstrut_2F_1\biggl(\begin{matrix}a-c+1,b-c+1\\2-c\end{matrix};x\biggr)
%\]
Diese Lösung ist aber nur möglich, wenn in
\eqref{buch:differentialgleichungen:hypergeo:verallgkoef}
der Nenner nicht verschwindet, d.~h.~$2-c+k\ne 0$
oder $c \ne k+2$ für all natürlichen $k$.
$c$ darf also kein natürliche Zahl $\ge 2$ sein.
Wir fassen die Resultate dieses Abschnitts im folgenden Satz zusammen.

\begin{satz}
Die eulersche hypergeometrische Differentialgleichung
\begin{equation}
x(1-x)\frac{d^2y}{dx^2}
+(c+(a+b+1)x)\frac{dy}{dx}
-ab y
=
0
\end{equation}
hat die Lösung
\[
y_1(x)
=
\mathstrut_2F_1\biggl(\begin{matrix}a,b\\c\end{matrix};x\biggr).
\]
Falls $c-2\not\in \mathbb{N}$ ist, ist
\[
y_2(x)
=
x^{1-c} \mathstrut_2F_1\biggl(\begin{matrix}a-c+1,b-c+1\\2-c\end{matrix};x\biggr)
\]
eine zweite, linear unabhängige Lösung.
\end{satz}

%
% Die verallgemeinerte hypergeometrische Differentialgleichung
%
\subsection{Verallgemeinerte hypergeometrische Differentialgleichung}
% https://de.wikipedia.org/wiki/Verallgemeinerte_hypergeometrische_Funktion







\section*{Übungsaufgaben}
\rhead{Übungsaufgaben}
\aufgabetoplevel{chapters/040-rekursion/uebungsaufgaben}
\begin{uebungsaufgaben}
%\uebungsaufgabe{0}
\uebungsaufgabe{1}
\uebungsaufgabe{2}
\end{uebungsaufgaben}


%%
% chapter.tex -- Beschreibung des Inhaltes
%
% (c) 2021 Prof Dr Andreas Müller, Hochschule Rapperswil
%
% !TeX spellcheck = de_CH
\chapter{Spezielle Funktionen und Rekursion
\label{buch:chapter:rekursion}}
\lhead{Spezielle Funktionen und Rekursion}
\rhead{}

%
% gamma.tex -- Abschnitt über die Gamma-funktion
%
% (c) 2021 Prof Dr Andreas Müller, OST Ostschweizer Fachhochschule
%
\section{Die Gamma-Funktion
\label{buch:rekursion:section:gamma}}
Die Fakultät $x!$ kann rekursiv durch 
\[
	x! = x\cdot (x-1)! \qquad\text{und}\qquad 0!=1
\]
für alle natürlichen Zahlen $x\in\mathbb{N}$ definiert werden.
Äquivalent damit ist eine Funktion 
\begin{equation}
\Gamma(x+1) = x\Gamma(x)
\qquad\text{und}\qquad 
\Gamma(1)=1.
\label{buch:rekursion:eqn:gammadef}
\end{equation}
Kann man eine reelle oder komplexe Funktion finden, die die
Funktionalgleichung~\eqref{buch:rekursion:eqn:gammadef}
erfüllt und damit die Fakultät auf beliebige Argumente ausdehnt?

\subsection{Integralformel für die Gamma-Funktion}
Euler hat die folgende Integraldefinition der Gamma-Funktion gegeben.

\begin{definition}
\label{buch:rekursion:def:gamma}
Die Gamma-Funktion ist die Funktion 
\[
\Gamma
\colon
\{z\in\mathbb{C} \mid \operatorname{Re}z>0\}
\to \mathbb{C}
:
z
\mapsto
\Gamma(z) = \int_0^\infty t^{x-1}e^{-t}\,dt
\]
\end{definition}

Man beachte, dass das Integral für $x=0$ nicht definiert ist, eine
Potenzreihenentwicklung um einen Punkt $x_0$ auf der positiven reellen
Achse kann also höchstens den Konvergenzradius $\varrho=|x_0|$ haben.

\begin{figure}
\centering
\includegraphics{chapters/040-rekursion/images/gammaplot.pdf}
\caption{Graph der Gamma-Funktion $z\mapsto\Gamma(z)$ und der alternativen
Funktion $\Gamma(z)+\sin(\pi z)$, die für ganzzahlige Argumente ebenfalls
die Werte der Fakultät annimmt.
\label{buch:rekursion:fig:gamma}}
\end{figure}

\subsubsection{Alternative Lösungen}
Die Funktion $\Gamma(z)$ ist nicht die einzige Funktion, die natürlichen
Zahlen die Werte $\Gamma(n+1) = n!$ der Fakultät annimmt.
Indem man eine beliebige Funktion $f(z)$ addiert, die auf alle
natürlichen Zahlen verschwindet, also $f(n)=0$ für $n\in\mathbb{N}$,
erhält man eine weitere Funktion, die auf natürlichen Zahlen
die Werte der Fakultät annimmt.
Ein Beispiel einer solchen Funktion ist
\begin{equation}
z\mapsto f(z)=\Gamma(z) + \sin \pi z,
\label{buch:rekursion:eqn:gammaalternative}
\end{equation}
die Funktion $f(z)=\sin\pi z$ verschwindet sogar auf allen ganzen
Zahlen.

In Abbildung~\ref{buch:rekursion:fig:gamma} ist die Gamma-Funktion
in rot geplotet, die Funktion~\eqref{buch:rekursion:eqn:gammaalternative}
in grün.
Die Punkte $(n,(n-1)!)$ sind in blau bezeichnet, sie sind beiden Graphen
gemeinsam.

\subsubsection{Pol erster Ordnung bei $z=0$}
Wir haben zu prüfen, dass sowohl der Wert $\Gamma(1)$ korrekt ist als
auch die Rekursionsformel~\eqref{buch:rekursion:eqn:gammadef} gilt.
Der Wert für $z=1$ ist
\begin{align*}
\Gamma(1)
&=
\int_0^\infty t^{1-1}e^{-t}\,dt
=
\left[ -e^{-t} \right]_0^\infty
=
1.
\end{align*}
Für die Rekursionsformel kann mit Hilfe von partieller Integration
bekommen:
\begin{align*}
\Gamma(z+1)
&=
\int_0^\infty t^{z+1-1}e^{-t}\,dt
=
\biggl[-t^{z}e^{-t}\biggr]_0^\infty
+
\int_0^\infty z t^{z-1}e^{-t}\,dt
\\
&=
z
\int_0^\infty
t^{z-1}e^{-t}\,dt
=
z \Gamma(z).
\end{align*}

Für $0<z<\varepsilon$ für eine $\varepsilon >0$ folgt aus der 
Funktionalgleichung
\[
\Gamma(z) = \frac{\Gamma(1+z)}{z}.
\]
Da $\Gamma(1)=1$ ist und $\Gamma$ eine in einer
Umgebung von $1$ stetige Funktion ist, kann sie in der Form
\(
\Gamma(1+z)=\Gamma(1) + zf(z)
\)
schreiben, wobei  $f(z)$ eine differenzierbare Funktion ist mit
$f'(1)=\Gamma'(1)$.
Daraus ergibt sich für $\Gamma(z)$ der Ausdruck
\[
\Gamma(z) = \frac{\Gamma(1)}{z} + f(z) = \frac{1}{z} + f(z).
\]
Die Gamma-Funktion hat daher and er Stelle $z=0$ einen Pol erster Ordnung.

\subsubsection{Ausdehnung auf $\operatorname{Re}z<0$}
Die Integralformel konvergiert nicht für $\operatorname{Re}z\le 0$.
Durch analytische Fortsetzung, wie sie im
Abschnitt~\ref{buch:funktionentheorie:section:fortsetzung}
beschrieben wird, kann die Funktion auf ganz $\mathbb{C}$ ausgedehnt
werden, mit Ausnahme einzelner Pole.
Die Funktionalgleichung gilt natürlich für alle $z\in\mathbb{C}$,
für die $\Gamma(z)$ definiert ist.
In einer Umgebung von $z=-n$ gilt
\[
\Gamma(z)
=
\frac{\Gamma(z+1)}{z}
=
\frac{\Gamma(z+2)}{z(z+1)}
=
\frac{\Gamma(z+3)}{z(z+1)(z+2)}
=
\dots
=
\frac{\Gamma(z+n)}{z(z+1)(z+2)\cdots(z+n-1)}
\]
Keiner der Faktoren im Nenner verschwindet in der Nähe von $z=-n$, der
Zähler hat aber einen Pol erster Ordnung an dieser Stelle.
Daher hat auch der Quotient einen Pol erster Ordnung.
Abbildung~\ref{buch:rekursion:fig:gamma} zeigt die Pole bei den
nicht negativen ganzen Zahlen.






%
% linear.tex
%
% (c) 2021 Prof Dr Andreas Müller, OST Ostschweizer Fachhochschule
%
\section{Lineare Rekursionsgleichung mit konstanten Koeffizienten
\label{buch:rekursion:section:linear}}
\rhead{Lineare Rekursionsgleichungen}
Die Funktionalgleichung der Gamma-Funktion, die im
Abschnitt~\ref{buch:rekursion:section:gamma} untersucht wurde,
hat die Form einer linearen Rekursionsgleichung
\[
\Gamma(x+1) = x\Gamma(x),\qquad \Gamma(1) = 1.
\]
Gleichungen, die Werte einer Funktion für verschiedene
Argument in Beziehung setzen, heissen {\em Funktionalgleichungen}.
\index{Funktionalgleichung}%
Es war überraschend schwierig, eine Lösung für Funktionalgleichung
der Gamma-Funktion für beliebige komplexe $x$ zu finden.
In diesem Abschnitt soll daher eine Klasse von Rekursionsgleichungen
näher untersucht werden, für die einfache Lösungen möglich sind.

\subsection{Lineare Differenzengleichungen}

\subsection{Lösung mit Polynomfunktionen}







%
% hypergeometrisch.tex
%
% (c) 2021 Prof Dr Andreas Müller, OST Ostschweizer Fachhochschule
%
\section{Hypergeometrische Differentialgleichung
\label{buch:differentialgleichungen:section:hypergeometrisch}}
Die hypergeometrische Funktion $\mathstrut_2F1(a,b;c;x)$ wurde in
Abschnitt~\ref{buch:rekursion:section:hypergeometrische-funktion}
als Potenzreihe mit sehr speziellen Koeffizienten, die sich aus
Pochhammer-Symbolen.
Es stellt sich aber heraus, dass man sie auch als Lösung einer
gewöhnlichen Differentialgleichung bekommen kann, die bereits
Euler studiert hat.

\subsection{Die Eulersche hypergeometrische Differentialgleichung
\label{buch:differentialgleichung:subsection:euler-hypergeometrisch}}
Die hypergeometrische Funktion $\mathstrut_2F_1(a,b;c;x)$ ist eine
Lösung der {\em Eulerschen hypergeometrischen Differentialgleichung}
(zu unterscheiden von der Eulerschen Differentialgleichung, die sich
immer auf eine lineare Differentialgleichung mit konstanten Koeffizienten
reduzieren lässt)
\begin{equation}
x(1-x) \frac{d^2y}{dx^2} + (c-(a+b+1)x)\frac{dy}{dx} - ab y = 0
\label{buch:differentialgleichungen:hypergeo:eulerdgl}
\end{equation}
Wir prüfen dies nach, indem wir die Definition der hypergeometrischen
Funktion 
\begin{align*}
y(x)
&=
\mathstrut_2F_1(a,b;c;x)
=
\sum_{k=0}^\infty
\frac{(a)_k(b)_k}{(c)_k} \frac{x^k}{k!}
\intertext{mit den Ableitungen}
y'(x)
&=
\sum_{k=1}^\infty 
\frac{(a)_k(b)_k}{(c)_k} \frac{x^{k-1}}{(k-1)!}
\\
y''(x)
&=
\sum_{k=2}^\infty 
\frac{(a)_k(b)_k}{(c)_k} \frac{x^{k-2}}{(k-2)!}
\end{align*}
einsetzen.
Die Gleichung, die sich ergibt, ist
\begin{align*}
0
&=
x(1-x)
\sum_{k=2}^\infty
\frac{(a)_k(b)_k}{(c)_k}\frac{x^{k-2}}{(k-2)!}
+
(c-(a+b+1)x)
\sum_{k=1}^\infty
\frac{(a)_k(b)_k}{(c)_k}\frac{x^{k-1}}{(k-1)!}
-ab
\sum_{k=0}^\infty
\frac{(a)_k(b)_k}{(c)_k} \frac{x^k}{k!}
\\
&=
\sum_{k=2}^\infty
\frac{(a)_k(b)_k}{(c)_k}\frac{x^{k-1}}{(k-2)!}
-
\sum_{k=2}^\infty
\frac{(a)_k(b)_k}{(c)_k}\frac{x^k}{(k-2)!}
+
c\sum_{k=1}^\infty
\frac{(a)_k(b)_k}{(c)_k}\frac{x^{k-1}}{(k-1)!}
\\
&\qquad
-(a+b+1)
\sum_{k=1}^\infty
\frac{(a)_k(b)_k}{(c)_k}\frac{x^k}{(k-1)!}
-ab
\sum_{k=0}^\infty
\frac{(a)_k(b)_k}{(c)_k} \frac{x^k}{k!}
\\
&=
\sum_{k=1}^\infty
\frac{(a)_{k+1}(b)_{k+1}}{(c)_{k+1}}\frac{x^k}{(k-1)!}
-
\sum_{k=2}^\infty
\frac{(a)_k(b)_k}{(c)_k}\frac{x^k}{(k-2)!}
+
c\sum_{k=0}^\infty
\frac{(a)_{k+1}(b)_{k+1}}{(c)_{k+1}}\frac{x^k}{k!}
\\
&\qquad
-(a+b+1)
\sum_{k=1}^\infty
\frac{(a)_k(b)_k}{(c)_k}\frac{x^k}{(k-1)!}
-ab
\sum_{k=0}^\infty
\frac{(a)_k(b)_k}{(c)_k} \frac{x^k}{k!}.
\end{align*}
Zum konstanten Koeffizienten für $k=0$ tragen nur die dritte und letzte
Summe bei, dies sind die Terme
\[
c\frac{(a)_1(b)_1}{(c)_1}-ab\frac{(a)_0(b)_0}{(c)_0}
=
c\frac{ab}{c}-ab\frac{1\cdot 1}{1}
=
0.
\]
Für den linearen Term $k=1$ kommen je ein Term aus der ersten aund vierten
Summe hinzu, dies ergibt
\begin{align*}
&\phantom{\mathstrut=\mathstrut}
\frac{(a)_2(b)_2}{(c)_2}
+c\frac{(a)_2(b)_2}{(c)_2}
-(a+b+1)\frac{(a)_1(b)_1}{(c)_1}
-ab\frac{(a)_1(b)_1}{(c)_1}
\\
&=
\frac{a(a+1)b(b+1)}{c(c+1)}
(1+c)
-(ab+a+b+1)
\frac{ab}{c}
\\
&=
\frac{a(a+1)b(b+1)}{c}
-
(a+1)(b+1)\frac{ab}{c}
=0.
\end{align*}
Durch Koeffizientenvergleich erhalten wir für $k\ge 2$ 
\begin{align*}
0
&=
\frac{(a)_{k+1}(b)_{k+1}}{(c)_{k+1}} \frac1{(k-1)!} 
-
\frac{(a)_k(b)_k}{(c)_k} \frac1{(k-2)!} 
+
c\frac{(a)_{k+1}(b)_{k+1}}{(c)_{k+1}} \frac{1}{k!}
\\
&\qquad
-(a+b+1)\frac{(a)_k(b)_k}{(c)_k}\frac{1}{(k-1)!}
-ab \frac{(a)_k(b)_k}{(c)_k}\frac{1}{k!}
\\
&=
\frac{(a)_k(b)_k}{(c)_{k+1}}
\frac{1}{k!}
\biggl(
(a+k)(b+k)k
-(c+k)(k-1)k
+
c(a+k)(b+k)
\\
&\qquad
\qquad
\qquad
-(a+b+1)(c+k)k
-ab(c+k)
\biggr).
\intertext{Der zweite, vierte und fünfte Term können zu}
&=
\frac{(a)_k(b)_k}{(c)_{k+1}}
\frac{1}{k!}
\biggl(
(a+k)(b+k)k
+
c(a+k)(b+k)
-(ab+ak+bk+k^2)(c+k)
\biggr)
\intertext{zusammengefasst werden.
Der Faktor $(ab+ak+bk+k^2)$ kann als Produkt $(a+k)(b+k)$ faktorisiert
werden, der dann als gemeinsamer Faktor aus allen Termen ausgeklammert
werden kann:}
&=
\frac{(a)_k(b)_k}{(c)_{k+1}}
\frac{1}{k!}
\biggl(
(a+k)(b+k)k
+
c(a+k)(b+k)
-(a+k)(b+k)(c+k)
\biggr)
\\
&=
\frac{(a)_{k+1}(b)_{k+1}}{(c)_{k+1}}
\frac{1}{k!}
\biggl(
k
+
c
-(c+k)
\biggr)
=0.
\end{align*}
Damit ist gezeigt, dass $\mathstrut_2F_1(a,b;c;x)$ eine Lösung
der Differentialgleichung ist.

Die hypergeometrische Reihe kann auch direkt mit Hilfe der
Potenzreihenmethode als Lösung der Differentialgleichung gefunden 
werden.

\subsection{Lösung als verallgemeinerte Potenzreihe}
Da die hypergeometrische Reihe eine Differentialgleichung
zweiter Ordnung mit einer Singularität bei $x=0$ ist, 
kann man versuchen eine zweite, linear unabhängige Lösung mit
Hilfe der Methode der verallgemeinerten Potenzreihen zu finden.
Dazu setzt man die Lösung in der Form
\begin{align*}
y_2(x)
&=
\sum_{k=0}^\infty a_kx^{\varrho+k}
&
&\Rightarrow&
y_2'(x)
&=
\sum_{k=0}^\infty (\varrho+k)a_kx^{\varrho+k-1}
\\
&&
&&
y_2''(x)
&=
\sum_{k=0}^\infty (\varrho+k)(\varrho+k-1)a_kx^{\varrho+k-2}
\end{align*}
an, wobei $a_0\ne 0$ sein soll.
Einsetzen in die Differentialgleichung ergibt
\begin{align*}
0&=
x(1-x)y_2''(x) + (c-(a+b+1)x) y_2'(x) -aby_2(x)
\\
&=
x(1-x)
\sum_{k=0}^\infty (\varrho+k)(\varrho+k-1)a_kx^{\varrho+k-2}
+
(c-(a+b+1)x)
\sum_{k=0}^\infty (\varrho+k)a_kx^{\varrho+k-1}
-
abx^{\varrho}\sum_{k=0}^\infty a_kx^{\varrho+k}
\\
&=
-\sum_{k=0}^\infty (\varrho+k)(\varrho+k-1)a_kx^{\varrho+k}
+
\sum_{k=0}^\infty (\varrho+k)(\varrho+k-1)a_kx^{\varrho+k-1}
+
c
\sum_{k=0}^\infty (\varrho+k)a_kx^{\varrho+k-1}
\\
&\qquad
-
(a+b+1)
\sum_{k=0}^\infty (\varrho+k)a_kx^{\varrho+k}
-
ab
\sum_{k=0}^\infty a_kx^{\varrho+k}.
\intertext{Durch Verschiebung des Summationsindex in der zweiten
und dritten Summe wird der Koeffizientenvergleich etwas
einfacher}
&=
-\sum_{k=0}^\infty (\varrho+k)(\varrho+k-1)a_kx^{\varrho+k}
+
\sum_{k=-1}^\infty (\varrho+k+1)(\varrho+k)a_{k+1}x^{\varrho+k}
+
c
\sum_{k=-1}^\infty (\varrho+k+1)a_{k+1}x^{\varrho+k}
\\
&\qquad
-
(a+b+1)
\sum_{k=0}^\infty (\varrho+k)a_kx^{\varrho+k}
-
ab
\sum_{k=0}^\infty a_kx^{\varrho+k}
\\
&=
-\sum_{k=0}^\infty (\varrho+k)(\varrho+k-1)a_kx^{\varrho+k}
+
\sum_{k=-1}^\infty (\varrho+k+1)(\varrho+k+c)a_{k+1}x^{\varrho+k}
\\
&\qquad
-
\sum_{k=0}^\infty ((\varrho+k)(a+b+1)+ab)a_kx^{\varrho+k}
\\
&=
\bigl(
\varrho(\varrho-1)
+c\varrho \bigr)
x^{\varrho-1}
+
\sum_{k=0}^\infty
\bigl(
-(\varrho+k)(\varrho+k-1)a_k
+(\varrho+k+1)(\varrho+k+c)a_{k+1}
\\
&
\qquad
\qquad
\qquad
\qquad
\qquad
\qquad
-((\varrho+k)(a+b+1)+ab)a_k
\bigr)
x^{\varrho+k}.
\end{align*}
Aus dem ersten Term kann man die Indexgleichung
\[
0
=
\varrho(\varrho-1)+c\varrho
=
\varrho(\varrho-1+c)
\]
ablesen, die die Nullstellen $\varrho=0$ und $\varrho=1-c$ hat.
Die Nullstelle $\varrho=0$ ergibt natürlich die bereits gefundene
hypergeometrische Reihe.

Nach Einsetzen der zweiten Lösung der Indexgleichung in der Summe
legt der Koeffizientenvergleich eine Beziehung
\begin{align}
0
&=
\bigl(
-(k-c+1)(k-c)
-(k-c+1)(a+b+1)+ab
\bigr)a_k
+
(k-c+2)(k+1)
a_{k+1} 
\notag
\intertext{zwischen $a_k$ und $a_{k+1}$ fest.
Daraus kann man den Quotienten aufeinanderfolgender
Koeffizienten als}
\frac{a_{k+1}}{a_k}
&=
\frac{
-(k-c+1)(k-c)
-(k-c+1)(a+b+1)+ab
}{
\notag
(k-c+2)(k+1)
}
\\
&=
%(%i4) factor(coeff(y,q,0))
%(%o4)                  - (k - c + a + 1) (k - c + b + 1)
%(%i5) factor(coeff(y,q,1))
%(%o5)                         (k + 1) (k - c + 2)
\frac{
(a-c+1+k)
(b-c+1+k)
}{
(2-c+k)(k+1)
}
\label{buch:differentialgleichungen:hypergeo:verallgkoef}
\end{align}
finden.
Setzt man $a_0=1$, ist die zweite Lösung ist also wieder eine
hypergeometrische Funktion.%, nämlich
%\[
%y_2(x)
%=
%x^{1-c}
%\sum_{k=0}^\infty \frac{(a-c+1)_k(b-c+1)_k}{(2-c)_k}\frac{x^k}{k!}
%=
%x^{1-c}
%\mathstrut_2F_1\biggl(\begin{matrix}a-c+1,b-c+1\\2-c\end{matrix};x\biggr)
%\]
Diese Lösung ist aber nur möglich, wenn in
\eqref{buch:differentialgleichungen:hypergeo:verallgkoef}
der Nenner nicht verschwindet, d.~h.~$2-c+k\ne 0$
oder $c \ne k+2$ für all natürlichen $k$.
$c$ darf also kein natürliche Zahl $\ge 2$ sein.
Wir fassen die Resultate dieses Abschnitts im folgenden Satz zusammen.

\begin{satz}
Die eulersche hypergeometrische Differentialgleichung
\begin{equation}
x(1-x)\frac{d^2y}{dx^2}
+(c+(a+b+1)x)\frac{dy}{dx}
-ab y
=
0
\end{equation}
hat die Lösung
\[
y_1(x)
=
\mathstrut_2F_1\biggl(\begin{matrix}a,b\\c\end{matrix};x\biggr).
\]
Falls $c-2\not\in \mathbb{N}$ ist, ist
\[
y_2(x)
=
x^{1-c} \mathstrut_2F_1\biggl(\begin{matrix}a-c+1,b-c+1\\2-c\end{matrix};x\biggr)
\]
eine zweite, linear unabhängige Lösung.
\end{satz}

%
% Die verallgemeinerte hypergeometrische Differentialgleichung
%
\subsection{Verallgemeinerte hypergeometrische Differentialgleichung}
% https://de.wikipedia.org/wiki/Verallgemeinerte_hypergeometrische_Funktion







\section*{Übungsaufgaben}
\rhead{Übungsaufgaben}
\aufgabetoplevel{chapters/040-rekursion/uebungsaufgaben}
\begin{uebungsaufgaben}
%\uebungsaufgabe{0}
\uebungsaufgabe{1}
\uebungsaufgabe{2}
\end{uebungsaufgaben}


% Gamma und Pi
% Eulersche Beta-Funktion

% Spezielle Funktionenfamilien
%%
% chapter.tex -- Beschreibung des Inhaltes
%
% (c) 2021 Prof Dr Andreas Müller, Hochschule Rapperswil
%
% !TeX spellcheck = de_CH
\chapter{Spezielle Funktionen und Rekursion
\label{buch:chapter:rekursion}}
\lhead{Spezielle Funktionen und Rekursion}
\rhead{}

%
% gamma.tex -- Abschnitt über die Gamma-funktion
%
% (c) 2021 Prof Dr Andreas Müller, OST Ostschweizer Fachhochschule
%
\section{Die Gamma-Funktion
\label{buch:rekursion:section:gamma}}
Die Fakultät $x!$ kann rekursiv durch 
\[
	x! = x\cdot (x-1)! \qquad\text{und}\qquad 0!=1
\]
für alle natürlichen Zahlen $x\in\mathbb{N}$ definiert werden.
Äquivalent damit ist eine Funktion 
\begin{equation}
\Gamma(x+1) = x\Gamma(x)
\qquad\text{und}\qquad 
\Gamma(1)=1.
\label{buch:rekursion:eqn:gammadef}
\end{equation}
Kann man eine reelle oder komplexe Funktion finden, die die
Funktionalgleichung~\eqref{buch:rekursion:eqn:gammadef}
erfüllt und damit die Fakultät auf beliebige Argumente ausdehnt?

\subsection{Integralformel für die Gamma-Funktion}
Euler hat die folgende Integraldefinition der Gamma-Funktion gegeben.

\begin{definition}
\label{buch:rekursion:def:gamma}
Die Gamma-Funktion ist die Funktion 
\[
\Gamma
\colon
\{z\in\mathbb{C} \mid \operatorname{Re}z>0\}
\to \mathbb{C}
:
z
\mapsto
\Gamma(z) = \int_0^\infty t^{x-1}e^{-t}\,dt
\]
\end{definition}

Man beachte, dass das Integral für $x=0$ nicht definiert ist, eine
Potenzreihenentwicklung um einen Punkt $x_0$ auf der positiven reellen
Achse kann also höchstens den Konvergenzradius $\varrho=|x_0|$ haben.

\begin{figure}
\centering
\includegraphics{chapters/040-rekursion/images/gammaplot.pdf}
\caption{Graph der Gamma-Funktion $z\mapsto\Gamma(z)$ und der alternativen
Funktion $\Gamma(z)+\sin(\pi z)$, die für ganzzahlige Argumente ebenfalls
die Werte der Fakultät annimmt.
\label{buch:rekursion:fig:gamma}}
\end{figure}

\subsubsection{Alternative Lösungen}
Die Funktion $\Gamma(z)$ ist nicht die einzige Funktion, die natürlichen
Zahlen die Werte $\Gamma(n+1) = n!$ der Fakultät annimmt.
Indem man eine beliebige Funktion $f(z)$ addiert, die auf alle
natürlichen Zahlen verschwindet, also $f(n)=0$ für $n\in\mathbb{N}$,
erhält man eine weitere Funktion, die auf natürlichen Zahlen
die Werte der Fakultät annimmt.
Ein Beispiel einer solchen Funktion ist
\begin{equation}
z\mapsto f(z)=\Gamma(z) + \sin \pi z,
\label{buch:rekursion:eqn:gammaalternative}
\end{equation}
die Funktion $f(z)=\sin\pi z$ verschwindet sogar auf allen ganzen
Zahlen.

In Abbildung~\ref{buch:rekursion:fig:gamma} ist die Gamma-Funktion
in rot geplotet, die Funktion~\eqref{buch:rekursion:eqn:gammaalternative}
in grün.
Die Punkte $(n,(n-1)!)$ sind in blau bezeichnet, sie sind beiden Graphen
gemeinsam.

\subsubsection{Pol erster Ordnung bei $z=0$}
Wir haben zu prüfen, dass sowohl der Wert $\Gamma(1)$ korrekt ist als
auch die Rekursionsformel~\eqref{buch:rekursion:eqn:gammadef} gilt.
Der Wert für $z=1$ ist
\begin{align*}
\Gamma(1)
&=
\int_0^\infty t^{1-1}e^{-t}\,dt
=
\left[ -e^{-t} \right]_0^\infty
=
1.
\end{align*}
Für die Rekursionsformel kann mit Hilfe von partieller Integration
bekommen:
\begin{align*}
\Gamma(z+1)
&=
\int_0^\infty t^{z+1-1}e^{-t}\,dt
=
\biggl[-t^{z}e^{-t}\biggr]_0^\infty
+
\int_0^\infty z t^{z-1}e^{-t}\,dt
\\
&=
z
\int_0^\infty
t^{z-1}e^{-t}\,dt
=
z \Gamma(z).
\end{align*}

Für $0<z<\varepsilon$ für eine $\varepsilon >0$ folgt aus der 
Funktionalgleichung
\[
\Gamma(z) = \frac{\Gamma(1+z)}{z}.
\]
Da $\Gamma(1)=1$ ist und $\Gamma$ eine in einer
Umgebung von $1$ stetige Funktion ist, kann sie in der Form
\(
\Gamma(1+z)=\Gamma(1) + zf(z)
\)
schreiben, wobei  $f(z)$ eine differenzierbare Funktion ist mit
$f'(1)=\Gamma'(1)$.
Daraus ergibt sich für $\Gamma(z)$ der Ausdruck
\[
\Gamma(z) = \frac{\Gamma(1)}{z} + f(z) = \frac{1}{z} + f(z).
\]
Die Gamma-Funktion hat daher and er Stelle $z=0$ einen Pol erster Ordnung.

\subsubsection{Ausdehnung auf $\operatorname{Re}z<0$}
Die Integralformel konvergiert nicht für $\operatorname{Re}z\le 0$.
Durch analytische Fortsetzung, wie sie im
Abschnitt~\ref{buch:funktionentheorie:section:fortsetzung}
beschrieben wird, kann die Funktion auf ganz $\mathbb{C}$ ausgedehnt
werden, mit Ausnahme einzelner Pole.
Die Funktionalgleichung gilt natürlich für alle $z\in\mathbb{C}$,
für die $\Gamma(z)$ definiert ist.
In einer Umgebung von $z=-n$ gilt
\[
\Gamma(z)
=
\frac{\Gamma(z+1)}{z}
=
\frac{\Gamma(z+2)}{z(z+1)}
=
\frac{\Gamma(z+3)}{z(z+1)(z+2)}
=
\dots
=
\frac{\Gamma(z+n)}{z(z+1)(z+2)\cdots(z+n-1)}
\]
Keiner der Faktoren im Nenner verschwindet in der Nähe von $z=-n$, der
Zähler hat aber einen Pol erster Ordnung an dieser Stelle.
Daher hat auch der Quotient einen Pol erster Ordnung.
Abbildung~\ref{buch:rekursion:fig:gamma} zeigt die Pole bei den
nicht negativen ganzen Zahlen.






%
% linear.tex
%
% (c) 2021 Prof Dr Andreas Müller, OST Ostschweizer Fachhochschule
%
\section{Lineare Rekursionsgleichung mit konstanten Koeffizienten
\label{buch:rekursion:section:linear}}
\rhead{Lineare Rekursionsgleichungen}
Die Funktionalgleichung der Gamma-Funktion, die im
Abschnitt~\ref{buch:rekursion:section:gamma} untersucht wurde,
hat die Form einer linearen Rekursionsgleichung
\[
\Gamma(x+1) = x\Gamma(x),\qquad \Gamma(1) = 1.
\]
Gleichungen, die Werte einer Funktion für verschiedene
Argument in Beziehung setzen, heissen {\em Funktionalgleichungen}.
\index{Funktionalgleichung}%
Es war überraschend schwierig, eine Lösung für Funktionalgleichung
der Gamma-Funktion für beliebige komplexe $x$ zu finden.
In diesem Abschnitt soll daher eine Klasse von Rekursionsgleichungen
näher untersucht werden, für die einfache Lösungen möglich sind.

\subsection{Lineare Differenzengleichungen}

\subsection{Lösung mit Polynomfunktionen}







%
% hypergeometrisch.tex
%
% (c) 2021 Prof Dr Andreas Müller, OST Ostschweizer Fachhochschule
%
\section{Hypergeometrische Differentialgleichung
\label{buch:differentialgleichungen:section:hypergeometrisch}}
Die hypergeometrische Funktion $\mathstrut_2F1(a,b;c;x)$ wurde in
Abschnitt~\ref{buch:rekursion:section:hypergeometrische-funktion}
als Potenzreihe mit sehr speziellen Koeffizienten, die sich aus
Pochhammer-Symbolen.
Es stellt sich aber heraus, dass man sie auch als Lösung einer
gewöhnlichen Differentialgleichung bekommen kann, die bereits
Euler studiert hat.

\subsection{Die Eulersche hypergeometrische Differentialgleichung
\label{buch:differentialgleichung:subsection:euler-hypergeometrisch}}
Die hypergeometrische Funktion $\mathstrut_2F_1(a,b;c;x)$ ist eine
Lösung der {\em Eulerschen hypergeometrischen Differentialgleichung}
(zu unterscheiden von der Eulerschen Differentialgleichung, die sich
immer auf eine lineare Differentialgleichung mit konstanten Koeffizienten
reduzieren lässt)
\begin{equation}
x(1-x) \frac{d^2y}{dx^2} + (c-(a+b+1)x)\frac{dy}{dx} - ab y = 0
\label{buch:differentialgleichungen:hypergeo:eulerdgl}
\end{equation}
Wir prüfen dies nach, indem wir die Definition der hypergeometrischen
Funktion 
\begin{align*}
y(x)
&=
\mathstrut_2F_1(a,b;c;x)
=
\sum_{k=0}^\infty
\frac{(a)_k(b)_k}{(c)_k} \frac{x^k}{k!}
\intertext{mit den Ableitungen}
y'(x)
&=
\sum_{k=1}^\infty 
\frac{(a)_k(b)_k}{(c)_k} \frac{x^{k-1}}{(k-1)!}
\\
y''(x)
&=
\sum_{k=2}^\infty 
\frac{(a)_k(b)_k}{(c)_k} \frac{x^{k-2}}{(k-2)!}
\end{align*}
einsetzen.
Die Gleichung, die sich ergibt, ist
\begin{align*}
0
&=
x(1-x)
\sum_{k=2}^\infty
\frac{(a)_k(b)_k}{(c)_k}\frac{x^{k-2}}{(k-2)!}
+
(c-(a+b+1)x)
\sum_{k=1}^\infty
\frac{(a)_k(b)_k}{(c)_k}\frac{x^{k-1}}{(k-1)!}
-ab
\sum_{k=0}^\infty
\frac{(a)_k(b)_k}{(c)_k} \frac{x^k}{k!}
\\
&=
\sum_{k=2}^\infty
\frac{(a)_k(b)_k}{(c)_k}\frac{x^{k-1}}{(k-2)!}
-
\sum_{k=2}^\infty
\frac{(a)_k(b)_k}{(c)_k}\frac{x^k}{(k-2)!}
+
c\sum_{k=1}^\infty
\frac{(a)_k(b)_k}{(c)_k}\frac{x^{k-1}}{(k-1)!}
\\
&\qquad
-(a+b+1)
\sum_{k=1}^\infty
\frac{(a)_k(b)_k}{(c)_k}\frac{x^k}{(k-1)!}
-ab
\sum_{k=0}^\infty
\frac{(a)_k(b)_k}{(c)_k} \frac{x^k}{k!}
\\
&=
\sum_{k=1}^\infty
\frac{(a)_{k+1}(b)_{k+1}}{(c)_{k+1}}\frac{x^k}{(k-1)!}
-
\sum_{k=2}^\infty
\frac{(a)_k(b)_k}{(c)_k}\frac{x^k}{(k-2)!}
+
c\sum_{k=0}^\infty
\frac{(a)_{k+1}(b)_{k+1}}{(c)_{k+1}}\frac{x^k}{k!}
\\
&\qquad
-(a+b+1)
\sum_{k=1}^\infty
\frac{(a)_k(b)_k}{(c)_k}\frac{x^k}{(k-1)!}
-ab
\sum_{k=0}^\infty
\frac{(a)_k(b)_k}{(c)_k} \frac{x^k}{k!}.
\end{align*}
Zum konstanten Koeffizienten für $k=0$ tragen nur die dritte und letzte
Summe bei, dies sind die Terme
\[
c\frac{(a)_1(b)_1}{(c)_1}-ab\frac{(a)_0(b)_0}{(c)_0}
=
c\frac{ab}{c}-ab\frac{1\cdot 1}{1}
=
0.
\]
Für den linearen Term $k=1$ kommen je ein Term aus der ersten aund vierten
Summe hinzu, dies ergibt
\begin{align*}
&\phantom{\mathstrut=\mathstrut}
\frac{(a)_2(b)_2}{(c)_2}
+c\frac{(a)_2(b)_2}{(c)_2}
-(a+b+1)\frac{(a)_1(b)_1}{(c)_1}
-ab\frac{(a)_1(b)_1}{(c)_1}
\\
&=
\frac{a(a+1)b(b+1)}{c(c+1)}
(1+c)
-(ab+a+b+1)
\frac{ab}{c}
\\
&=
\frac{a(a+1)b(b+1)}{c}
-
(a+1)(b+1)\frac{ab}{c}
=0.
\end{align*}
Durch Koeffizientenvergleich erhalten wir für $k\ge 2$ 
\begin{align*}
0
&=
\frac{(a)_{k+1}(b)_{k+1}}{(c)_{k+1}} \frac1{(k-1)!} 
-
\frac{(a)_k(b)_k}{(c)_k} \frac1{(k-2)!} 
+
c\frac{(a)_{k+1}(b)_{k+1}}{(c)_{k+1}} \frac{1}{k!}
\\
&\qquad
-(a+b+1)\frac{(a)_k(b)_k}{(c)_k}\frac{1}{(k-1)!}
-ab \frac{(a)_k(b)_k}{(c)_k}\frac{1}{k!}
\\
&=
\frac{(a)_k(b)_k}{(c)_{k+1}}
\frac{1}{k!}
\biggl(
(a+k)(b+k)k
-(c+k)(k-1)k
+
c(a+k)(b+k)
\\
&\qquad
\qquad
\qquad
-(a+b+1)(c+k)k
-ab(c+k)
\biggr).
\intertext{Der zweite, vierte und fünfte Term können zu}
&=
\frac{(a)_k(b)_k}{(c)_{k+1}}
\frac{1}{k!}
\biggl(
(a+k)(b+k)k
+
c(a+k)(b+k)
-(ab+ak+bk+k^2)(c+k)
\biggr)
\intertext{zusammengefasst werden.
Der Faktor $(ab+ak+bk+k^2)$ kann als Produkt $(a+k)(b+k)$ faktorisiert
werden, der dann als gemeinsamer Faktor aus allen Termen ausgeklammert
werden kann:}
&=
\frac{(a)_k(b)_k}{(c)_{k+1}}
\frac{1}{k!}
\biggl(
(a+k)(b+k)k
+
c(a+k)(b+k)
-(a+k)(b+k)(c+k)
\biggr)
\\
&=
\frac{(a)_{k+1}(b)_{k+1}}{(c)_{k+1}}
\frac{1}{k!}
\biggl(
k
+
c
-(c+k)
\biggr)
=0.
\end{align*}
Damit ist gezeigt, dass $\mathstrut_2F_1(a,b;c;x)$ eine Lösung
der Differentialgleichung ist.

Die hypergeometrische Reihe kann auch direkt mit Hilfe der
Potenzreihenmethode als Lösung der Differentialgleichung gefunden 
werden.

\subsection{Lösung als verallgemeinerte Potenzreihe}
Da die hypergeometrische Reihe eine Differentialgleichung
zweiter Ordnung mit einer Singularität bei $x=0$ ist, 
kann man versuchen eine zweite, linear unabhängige Lösung mit
Hilfe der Methode der verallgemeinerten Potenzreihen zu finden.
Dazu setzt man die Lösung in der Form
\begin{align*}
y_2(x)
&=
\sum_{k=0}^\infty a_kx^{\varrho+k}
&
&\Rightarrow&
y_2'(x)
&=
\sum_{k=0}^\infty (\varrho+k)a_kx^{\varrho+k-1}
\\
&&
&&
y_2''(x)
&=
\sum_{k=0}^\infty (\varrho+k)(\varrho+k-1)a_kx^{\varrho+k-2}
\end{align*}
an, wobei $a_0\ne 0$ sein soll.
Einsetzen in die Differentialgleichung ergibt
\begin{align*}
0&=
x(1-x)y_2''(x) + (c-(a+b+1)x) y_2'(x) -aby_2(x)
\\
&=
x(1-x)
\sum_{k=0}^\infty (\varrho+k)(\varrho+k-1)a_kx^{\varrho+k-2}
+
(c-(a+b+1)x)
\sum_{k=0}^\infty (\varrho+k)a_kx^{\varrho+k-1}
-
abx^{\varrho}\sum_{k=0}^\infty a_kx^{\varrho+k}
\\
&=
-\sum_{k=0}^\infty (\varrho+k)(\varrho+k-1)a_kx^{\varrho+k}
+
\sum_{k=0}^\infty (\varrho+k)(\varrho+k-1)a_kx^{\varrho+k-1}
+
c
\sum_{k=0}^\infty (\varrho+k)a_kx^{\varrho+k-1}
\\
&\qquad
-
(a+b+1)
\sum_{k=0}^\infty (\varrho+k)a_kx^{\varrho+k}
-
ab
\sum_{k=0}^\infty a_kx^{\varrho+k}.
\intertext{Durch Verschiebung des Summationsindex in der zweiten
und dritten Summe wird der Koeffizientenvergleich etwas
einfacher}
&=
-\sum_{k=0}^\infty (\varrho+k)(\varrho+k-1)a_kx^{\varrho+k}
+
\sum_{k=-1}^\infty (\varrho+k+1)(\varrho+k)a_{k+1}x^{\varrho+k}
+
c
\sum_{k=-1}^\infty (\varrho+k+1)a_{k+1}x^{\varrho+k}
\\
&\qquad
-
(a+b+1)
\sum_{k=0}^\infty (\varrho+k)a_kx^{\varrho+k}
-
ab
\sum_{k=0}^\infty a_kx^{\varrho+k}
\\
&=
-\sum_{k=0}^\infty (\varrho+k)(\varrho+k-1)a_kx^{\varrho+k}
+
\sum_{k=-1}^\infty (\varrho+k+1)(\varrho+k+c)a_{k+1}x^{\varrho+k}
\\
&\qquad
-
\sum_{k=0}^\infty ((\varrho+k)(a+b+1)+ab)a_kx^{\varrho+k}
\\
&=
\bigl(
\varrho(\varrho-1)
+c\varrho \bigr)
x^{\varrho-1}
+
\sum_{k=0}^\infty
\bigl(
-(\varrho+k)(\varrho+k-1)a_k
+(\varrho+k+1)(\varrho+k+c)a_{k+1}
\\
&
\qquad
\qquad
\qquad
\qquad
\qquad
\qquad
-((\varrho+k)(a+b+1)+ab)a_k
\bigr)
x^{\varrho+k}.
\end{align*}
Aus dem ersten Term kann man die Indexgleichung
\[
0
=
\varrho(\varrho-1)+c\varrho
=
\varrho(\varrho-1+c)
\]
ablesen, die die Nullstellen $\varrho=0$ und $\varrho=1-c$ hat.
Die Nullstelle $\varrho=0$ ergibt natürlich die bereits gefundene
hypergeometrische Reihe.

Nach Einsetzen der zweiten Lösung der Indexgleichung in der Summe
legt der Koeffizientenvergleich eine Beziehung
\begin{align}
0
&=
\bigl(
-(k-c+1)(k-c)
-(k-c+1)(a+b+1)+ab
\bigr)a_k
+
(k-c+2)(k+1)
a_{k+1} 
\notag
\intertext{zwischen $a_k$ und $a_{k+1}$ fest.
Daraus kann man den Quotienten aufeinanderfolgender
Koeffizienten als}
\frac{a_{k+1}}{a_k}
&=
\frac{
-(k-c+1)(k-c)
-(k-c+1)(a+b+1)+ab
}{
\notag
(k-c+2)(k+1)
}
\\
&=
%(%i4) factor(coeff(y,q,0))
%(%o4)                  - (k - c + a + 1) (k - c + b + 1)
%(%i5) factor(coeff(y,q,1))
%(%o5)                         (k + 1) (k - c + 2)
\frac{
(a-c+1+k)
(b-c+1+k)
}{
(2-c+k)(k+1)
}
\label{buch:differentialgleichungen:hypergeo:verallgkoef}
\end{align}
finden.
Setzt man $a_0=1$, ist die zweite Lösung ist also wieder eine
hypergeometrische Funktion.%, nämlich
%\[
%y_2(x)
%=
%x^{1-c}
%\sum_{k=0}^\infty \frac{(a-c+1)_k(b-c+1)_k}{(2-c)_k}\frac{x^k}{k!}
%=
%x^{1-c}
%\mathstrut_2F_1\biggl(\begin{matrix}a-c+1,b-c+1\\2-c\end{matrix};x\biggr)
%\]
Diese Lösung ist aber nur möglich, wenn in
\eqref{buch:differentialgleichungen:hypergeo:verallgkoef}
der Nenner nicht verschwindet, d.~h.~$2-c+k\ne 0$
oder $c \ne k+2$ für all natürlichen $k$.
$c$ darf also kein natürliche Zahl $\ge 2$ sein.
Wir fassen die Resultate dieses Abschnitts im folgenden Satz zusammen.

\begin{satz}
Die eulersche hypergeometrische Differentialgleichung
\begin{equation}
x(1-x)\frac{d^2y}{dx^2}
+(c+(a+b+1)x)\frac{dy}{dx}
-ab y
=
0
\end{equation}
hat die Lösung
\[
y_1(x)
=
\mathstrut_2F_1\biggl(\begin{matrix}a,b\\c\end{matrix};x\biggr).
\]
Falls $c-2\not\in \mathbb{N}$ ist, ist
\[
y_2(x)
=
x^{1-c} \mathstrut_2F_1\biggl(\begin{matrix}a-c+1,b-c+1\\2-c\end{matrix};x\biggr)
\]
eine zweite, linear unabhängige Lösung.
\end{satz}

%
% Die verallgemeinerte hypergeometrische Differentialgleichung
%
\subsection{Verallgemeinerte hypergeometrische Differentialgleichung}
% https://de.wikipedia.org/wiki/Verallgemeinerte_hypergeometrische_Funktion







\section*{Übungsaufgaben}
\rhead{Übungsaufgaben}
\aufgabetoplevel{chapters/040-rekursion/uebungsaufgaben}
\begin{uebungsaufgaben}
%\uebungsaufgabe{0}
\uebungsaufgabe{1}
\uebungsaufgabe{2}
\end{uebungsaufgaben}


%%
% chapter.tex -- Beschreibung des Inhaltes
%
% (c) 2021 Prof Dr Andreas Müller, Hochschule Rapperswil
%
% !TeX spellcheck = de_CH
\chapter{Spezielle Funktionen und Rekursion
\label{buch:chapter:rekursion}}
\lhead{Spezielle Funktionen und Rekursion}
\rhead{}

%
% gamma.tex -- Abschnitt über die Gamma-funktion
%
% (c) 2021 Prof Dr Andreas Müller, OST Ostschweizer Fachhochschule
%
\section{Die Gamma-Funktion
\label{buch:rekursion:section:gamma}}
Die Fakultät $x!$ kann rekursiv durch 
\[
	x! = x\cdot (x-1)! \qquad\text{und}\qquad 0!=1
\]
für alle natürlichen Zahlen $x\in\mathbb{N}$ definiert werden.
Äquivalent damit ist eine Funktion 
\begin{equation}
\Gamma(x+1) = x\Gamma(x)
\qquad\text{und}\qquad 
\Gamma(1)=1.
\label{buch:rekursion:eqn:gammadef}
\end{equation}
Kann man eine reelle oder komplexe Funktion finden, die die
Funktionalgleichung~\eqref{buch:rekursion:eqn:gammadef}
erfüllt und damit die Fakultät auf beliebige Argumente ausdehnt?

\subsection{Integralformel für die Gamma-Funktion}
Euler hat die folgende Integraldefinition der Gamma-Funktion gegeben.

\begin{definition}
\label{buch:rekursion:def:gamma}
Die Gamma-Funktion ist die Funktion 
\[
\Gamma
\colon
\{z\in\mathbb{C} \mid \operatorname{Re}z>0\}
\to \mathbb{C}
:
z
\mapsto
\Gamma(z) = \int_0^\infty t^{x-1}e^{-t}\,dt
\]
\end{definition}

Man beachte, dass das Integral für $x=0$ nicht definiert ist, eine
Potenzreihenentwicklung um einen Punkt $x_0$ auf der positiven reellen
Achse kann also höchstens den Konvergenzradius $\varrho=|x_0|$ haben.

\begin{figure}
\centering
\includegraphics{chapters/040-rekursion/images/gammaplot.pdf}
\caption{Graph der Gamma-Funktion $z\mapsto\Gamma(z)$ und der alternativen
Funktion $\Gamma(z)+\sin(\pi z)$, die für ganzzahlige Argumente ebenfalls
die Werte der Fakultät annimmt.
\label{buch:rekursion:fig:gamma}}
\end{figure}

\subsubsection{Alternative Lösungen}
Die Funktion $\Gamma(z)$ ist nicht die einzige Funktion, die natürlichen
Zahlen die Werte $\Gamma(n+1) = n!$ der Fakultät annimmt.
Indem man eine beliebige Funktion $f(z)$ addiert, die auf alle
natürlichen Zahlen verschwindet, also $f(n)=0$ für $n\in\mathbb{N}$,
erhält man eine weitere Funktion, die auf natürlichen Zahlen
die Werte der Fakultät annimmt.
Ein Beispiel einer solchen Funktion ist
\begin{equation}
z\mapsto f(z)=\Gamma(z) + \sin \pi z,
\label{buch:rekursion:eqn:gammaalternative}
\end{equation}
die Funktion $f(z)=\sin\pi z$ verschwindet sogar auf allen ganzen
Zahlen.

In Abbildung~\ref{buch:rekursion:fig:gamma} ist die Gamma-Funktion
in rot geplotet, die Funktion~\eqref{buch:rekursion:eqn:gammaalternative}
in grün.
Die Punkte $(n,(n-1)!)$ sind in blau bezeichnet, sie sind beiden Graphen
gemeinsam.

\subsubsection{Pol erster Ordnung bei $z=0$}
Wir haben zu prüfen, dass sowohl der Wert $\Gamma(1)$ korrekt ist als
auch die Rekursionsformel~\eqref{buch:rekursion:eqn:gammadef} gilt.
Der Wert für $z=1$ ist
\begin{align*}
\Gamma(1)
&=
\int_0^\infty t^{1-1}e^{-t}\,dt
=
\left[ -e^{-t} \right]_0^\infty
=
1.
\end{align*}
Für die Rekursionsformel kann mit Hilfe von partieller Integration
bekommen:
\begin{align*}
\Gamma(z+1)
&=
\int_0^\infty t^{z+1-1}e^{-t}\,dt
=
\biggl[-t^{z}e^{-t}\biggr]_0^\infty
+
\int_0^\infty z t^{z-1}e^{-t}\,dt
\\
&=
z
\int_0^\infty
t^{z-1}e^{-t}\,dt
=
z \Gamma(z).
\end{align*}

Für $0<z<\varepsilon$ für eine $\varepsilon >0$ folgt aus der 
Funktionalgleichung
\[
\Gamma(z) = \frac{\Gamma(1+z)}{z}.
\]
Da $\Gamma(1)=1$ ist und $\Gamma$ eine in einer
Umgebung von $1$ stetige Funktion ist, kann sie in der Form
\(
\Gamma(1+z)=\Gamma(1) + zf(z)
\)
schreiben, wobei  $f(z)$ eine differenzierbare Funktion ist mit
$f'(1)=\Gamma'(1)$.
Daraus ergibt sich für $\Gamma(z)$ der Ausdruck
\[
\Gamma(z) = \frac{\Gamma(1)}{z} + f(z) = \frac{1}{z} + f(z).
\]
Die Gamma-Funktion hat daher and er Stelle $z=0$ einen Pol erster Ordnung.

\subsubsection{Ausdehnung auf $\operatorname{Re}z<0$}
Die Integralformel konvergiert nicht für $\operatorname{Re}z\le 0$.
Durch analytische Fortsetzung, wie sie im
Abschnitt~\ref{buch:funktionentheorie:section:fortsetzung}
beschrieben wird, kann die Funktion auf ganz $\mathbb{C}$ ausgedehnt
werden, mit Ausnahme einzelner Pole.
Die Funktionalgleichung gilt natürlich für alle $z\in\mathbb{C}$,
für die $\Gamma(z)$ definiert ist.
In einer Umgebung von $z=-n$ gilt
\[
\Gamma(z)
=
\frac{\Gamma(z+1)}{z}
=
\frac{\Gamma(z+2)}{z(z+1)}
=
\frac{\Gamma(z+3)}{z(z+1)(z+2)}
=
\dots
=
\frac{\Gamma(z+n)}{z(z+1)(z+2)\cdots(z+n-1)}
\]
Keiner der Faktoren im Nenner verschwindet in der Nähe von $z=-n$, der
Zähler hat aber einen Pol erster Ordnung an dieser Stelle.
Daher hat auch der Quotient einen Pol erster Ordnung.
Abbildung~\ref{buch:rekursion:fig:gamma} zeigt die Pole bei den
nicht negativen ganzen Zahlen.






%
% linear.tex
%
% (c) 2021 Prof Dr Andreas Müller, OST Ostschweizer Fachhochschule
%
\section{Lineare Rekursionsgleichung mit konstanten Koeffizienten
\label{buch:rekursion:section:linear}}
\rhead{Lineare Rekursionsgleichungen}
Die Funktionalgleichung der Gamma-Funktion, die im
Abschnitt~\ref{buch:rekursion:section:gamma} untersucht wurde,
hat die Form einer linearen Rekursionsgleichung
\[
\Gamma(x+1) = x\Gamma(x),\qquad \Gamma(1) = 1.
\]
Gleichungen, die Werte einer Funktion für verschiedene
Argument in Beziehung setzen, heissen {\em Funktionalgleichungen}.
\index{Funktionalgleichung}%
Es war überraschend schwierig, eine Lösung für Funktionalgleichung
der Gamma-Funktion für beliebige komplexe $x$ zu finden.
In diesem Abschnitt soll daher eine Klasse von Rekursionsgleichungen
näher untersucht werden, für die einfache Lösungen möglich sind.

\subsection{Lineare Differenzengleichungen}

\subsection{Lösung mit Polynomfunktionen}







%
% hypergeometrisch.tex
%
% (c) 2021 Prof Dr Andreas Müller, OST Ostschweizer Fachhochschule
%
\section{Hypergeometrische Differentialgleichung
\label{buch:differentialgleichungen:section:hypergeometrisch}}
Die hypergeometrische Funktion $\mathstrut_2F1(a,b;c;x)$ wurde in
Abschnitt~\ref{buch:rekursion:section:hypergeometrische-funktion}
als Potenzreihe mit sehr speziellen Koeffizienten, die sich aus
Pochhammer-Symbolen.
Es stellt sich aber heraus, dass man sie auch als Lösung einer
gewöhnlichen Differentialgleichung bekommen kann, die bereits
Euler studiert hat.

\subsection{Die Eulersche hypergeometrische Differentialgleichung
\label{buch:differentialgleichung:subsection:euler-hypergeometrisch}}
Die hypergeometrische Funktion $\mathstrut_2F_1(a,b;c;x)$ ist eine
Lösung der {\em Eulerschen hypergeometrischen Differentialgleichung}
(zu unterscheiden von der Eulerschen Differentialgleichung, die sich
immer auf eine lineare Differentialgleichung mit konstanten Koeffizienten
reduzieren lässt)
\begin{equation}
x(1-x) \frac{d^2y}{dx^2} + (c-(a+b+1)x)\frac{dy}{dx} - ab y = 0
\label{buch:differentialgleichungen:hypergeo:eulerdgl}
\end{equation}
Wir prüfen dies nach, indem wir die Definition der hypergeometrischen
Funktion 
\begin{align*}
y(x)
&=
\mathstrut_2F_1(a,b;c;x)
=
\sum_{k=0}^\infty
\frac{(a)_k(b)_k}{(c)_k} \frac{x^k}{k!}
\intertext{mit den Ableitungen}
y'(x)
&=
\sum_{k=1}^\infty 
\frac{(a)_k(b)_k}{(c)_k} \frac{x^{k-1}}{(k-1)!}
\\
y''(x)
&=
\sum_{k=2}^\infty 
\frac{(a)_k(b)_k}{(c)_k} \frac{x^{k-2}}{(k-2)!}
\end{align*}
einsetzen.
Die Gleichung, die sich ergibt, ist
\begin{align*}
0
&=
x(1-x)
\sum_{k=2}^\infty
\frac{(a)_k(b)_k}{(c)_k}\frac{x^{k-2}}{(k-2)!}
+
(c-(a+b+1)x)
\sum_{k=1}^\infty
\frac{(a)_k(b)_k}{(c)_k}\frac{x^{k-1}}{(k-1)!}
-ab
\sum_{k=0}^\infty
\frac{(a)_k(b)_k}{(c)_k} \frac{x^k}{k!}
\\
&=
\sum_{k=2}^\infty
\frac{(a)_k(b)_k}{(c)_k}\frac{x^{k-1}}{(k-2)!}
-
\sum_{k=2}^\infty
\frac{(a)_k(b)_k}{(c)_k}\frac{x^k}{(k-2)!}
+
c\sum_{k=1}^\infty
\frac{(a)_k(b)_k}{(c)_k}\frac{x^{k-1}}{(k-1)!}
\\
&\qquad
-(a+b+1)
\sum_{k=1}^\infty
\frac{(a)_k(b)_k}{(c)_k}\frac{x^k}{(k-1)!}
-ab
\sum_{k=0}^\infty
\frac{(a)_k(b)_k}{(c)_k} \frac{x^k}{k!}
\\
&=
\sum_{k=1}^\infty
\frac{(a)_{k+1}(b)_{k+1}}{(c)_{k+1}}\frac{x^k}{(k-1)!}
-
\sum_{k=2}^\infty
\frac{(a)_k(b)_k}{(c)_k}\frac{x^k}{(k-2)!}
+
c\sum_{k=0}^\infty
\frac{(a)_{k+1}(b)_{k+1}}{(c)_{k+1}}\frac{x^k}{k!}
\\
&\qquad
-(a+b+1)
\sum_{k=1}^\infty
\frac{(a)_k(b)_k}{(c)_k}\frac{x^k}{(k-1)!}
-ab
\sum_{k=0}^\infty
\frac{(a)_k(b)_k}{(c)_k} \frac{x^k}{k!}.
\end{align*}
Zum konstanten Koeffizienten für $k=0$ tragen nur die dritte und letzte
Summe bei, dies sind die Terme
\[
c\frac{(a)_1(b)_1}{(c)_1}-ab\frac{(a)_0(b)_0}{(c)_0}
=
c\frac{ab}{c}-ab\frac{1\cdot 1}{1}
=
0.
\]
Für den linearen Term $k=1$ kommen je ein Term aus der ersten aund vierten
Summe hinzu, dies ergibt
\begin{align*}
&\phantom{\mathstrut=\mathstrut}
\frac{(a)_2(b)_2}{(c)_2}
+c\frac{(a)_2(b)_2}{(c)_2}
-(a+b+1)\frac{(a)_1(b)_1}{(c)_1}
-ab\frac{(a)_1(b)_1}{(c)_1}
\\
&=
\frac{a(a+1)b(b+1)}{c(c+1)}
(1+c)
-(ab+a+b+1)
\frac{ab}{c}
\\
&=
\frac{a(a+1)b(b+1)}{c}
-
(a+1)(b+1)\frac{ab}{c}
=0.
\end{align*}
Durch Koeffizientenvergleich erhalten wir für $k\ge 2$ 
\begin{align*}
0
&=
\frac{(a)_{k+1}(b)_{k+1}}{(c)_{k+1}} \frac1{(k-1)!} 
-
\frac{(a)_k(b)_k}{(c)_k} \frac1{(k-2)!} 
+
c\frac{(a)_{k+1}(b)_{k+1}}{(c)_{k+1}} \frac{1}{k!}
\\
&\qquad
-(a+b+1)\frac{(a)_k(b)_k}{(c)_k}\frac{1}{(k-1)!}
-ab \frac{(a)_k(b)_k}{(c)_k}\frac{1}{k!}
\\
&=
\frac{(a)_k(b)_k}{(c)_{k+1}}
\frac{1}{k!}
\biggl(
(a+k)(b+k)k
-(c+k)(k-1)k
+
c(a+k)(b+k)
\\
&\qquad
\qquad
\qquad
-(a+b+1)(c+k)k
-ab(c+k)
\biggr).
\intertext{Der zweite, vierte und fünfte Term können zu}
&=
\frac{(a)_k(b)_k}{(c)_{k+1}}
\frac{1}{k!}
\biggl(
(a+k)(b+k)k
+
c(a+k)(b+k)
-(ab+ak+bk+k^2)(c+k)
\biggr)
\intertext{zusammengefasst werden.
Der Faktor $(ab+ak+bk+k^2)$ kann als Produkt $(a+k)(b+k)$ faktorisiert
werden, der dann als gemeinsamer Faktor aus allen Termen ausgeklammert
werden kann:}
&=
\frac{(a)_k(b)_k}{(c)_{k+1}}
\frac{1}{k!}
\biggl(
(a+k)(b+k)k
+
c(a+k)(b+k)
-(a+k)(b+k)(c+k)
\biggr)
\\
&=
\frac{(a)_{k+1}(b)_{k+1}}{(c)_{k+1}}
\frac{1}{k!}
\biggl(
k
+
c
-(c+k)
\biggr)
=0.
\end{align*}
Damit ist gezeigt, dass $\mathstrut_2F_1(a,b;c;x)$ eine Lösung
der Differentialgleichung ist.

Die hypergeometrische Reihe kann auch direkt mit Hilfe der
Potenzreihenmethode als Lösung der Differentialgleichung gefunden 
werden.

\subsection{Lösung als verallgemeinerte Potenzreihe}
Da die hypergeometrische Reihe eine Differentialgleichung
zweiter Ordnung mit einer Singularität bei $x=0$ ist, 
kann man versuchen eine zweite, linear unabhängige Lösung mit
Hilfe der Methode der verallgemeinerten Potenzreihen zu finden.
Dazu setzt man die Lösung in der Form
\begin{align*}
y_2(x)
&=
\sum_{k=0}^\infty a_kx^{\varrho+k}
&
&\Rightarrow&
y_2'(x)
&=
\sum_{k=0}^\infty (\varrho+k)a_kx^{\varrho+k-1}
\\
&&
&&
y_2''(x)
&=
\sum_{k=0}^\infty (\varrho+k)(\varrho+k-1)a_kx^{\varrho+k-2}
\end{align*}
an, wobei $a_0\ne 0$ sein soll.
Einsetzen in die Differentialgleichung ergibt
\begin{align*}
0&=
x(1-x)y_2''(x) + (c-(a+b+1)x) y_2'(x) -aby_2(x)
\\
&=
x(1-x)
\sum_{k=0}^\infty (\varrho+k)(\varrho+k-1)a_kx^{\varrho+k-2}
+
(c-(a+b+1)x)
\sum_{k=0}^\infty (\varrho+k)a_kx^{\varrho+k-1}
-
abx^{\varrho}\sum_{k=0}^\infty a_kx^{\varrho+k}
\\
&=
-\sum_{k=0}^\infty (\varrho+k)(\varrho+k-1)a_kx^{\varrho+k}
+
\sum_{k=0}^\infty (\varrho+k)(\varrho+k-1)a_kx^{\varrho+k-1}
+
c
\sum_{k=0}^\infty (\varrho+k)a_kx^{\varrho+k-1}
\\
&\qquad
-
(a+b+1)
\sum_{k=0}^\infty (\varrho+k)a_kx^{\varrho+k}
-
ab
\sum_{k=0}^\infty a_kx^{\varrho+k}.
\intertext{Durch Verschiebung des Summationsindex in der zweiten
und dritten Summe wird der Koeffizientenvergleich etwas
einfacher}
&=
-\sum_{k=0}^\infty (\varrho+k)(\varrho+k-1)a_kx^{\varrho+k}
+
\sum_{k=-1}^\infty (\varrho+k+1)(\varrho+k)a_{k+1}x^{\varrho+k}
+
c
\sum_{k=-1}^\infty (\varrho+k+1)a_{k+1}x^{\varrho+k}
\\
&\qquad
-
(a+b+1)
\sum_{k=0}^\infty (\varrho+k)a_kx^{\varrho+k}
-
ab
\sum_{k=0}^\infty a_kx^{\varrho+k}
\\
&=
-\sum_{k=0}^\infty (\varrho+k)(\varrho+k-1)a_kx^{\varrho+k}
+
\sum_{k=-1}^\infty (\varrho+k+1)(\varrho+k+c)a_{k+1}x^{\varrho+k}
\\
&\qquad
-
\sum_{k=0}^\infty ((\varrho+k)(a+b+1)+ab)a_kx^{\varrho+k}
\\
&=
\bigl(
\varrho(\varrho-1)
+c\varrho \bigr)
x^{\varrho-1}
+
\sum_{k=0}^\infty
\bigl(
-(\varrho+k)(\varrho+k-1)a_k
+(\varrho+k+1)(\varrho+k+c)a_{k+1}
\\
&
\qquad
\qquad
\qquad
\qquad
\qquad
\qquad
-((\varrho+k)(a+b+1)+ab)a_k
\bigr)
x^{\varrho+k}.
\end{align*}
Aus dem ersten Term kann man die Indexgleichung
\[
0
=
\varrho(\varrho-1)+c\varrho
=
\varrho(\varrho-1+c)
\]
ablesen, die die Nullstellen $\varrho=0$ und $\varrho=1-c$ hat.
Die Nullstelle $\varrho=0$ ergibt natürlich die bereits gefundene
hypergeometrische Reihe.

Nach Einsetzen der zweiten Lösung der Indexgleichung in der Summe
legt der Koeffizientenvergleich eine Beziehung
\begin{align}
0
&=
\bigl(
-(k-c+1)(k-c)
-(k-c+1)(a+b+1)+ab
\bigr)a_k
+
(k-c+2)(k+1)
a_{k+1} 
\notag
\intertext{zwischen $a_k$ und $a_{k+1}$ fest.
Daraus kann man den Quotienten aufeinanderfolgender
Koeffizienten als}
\frac{a_{k+1}}{a_k}
&=
\frac{
-(k-c+1)(k-c)
-(k-c+1)(a+b+1)+ab
}{
\notag
(k-c+2)(k+1)
}
\\
&=
%(%i4) factor(coeff(y,q,0))
%(%o4)                  - (k - c + a + 1) (k - c + b + 1)
%(%i5) factor(coeff(y,q,1))
%(%o5)                         (k + 1) (k - c + 2)
\frac{
(a-c+1+k)
(b-c+1+k)
}{
(2-c+k)(k+1)
}
\label{buch:differentialgleichungen:hypergeo:verallgkoef}
\end{align}
finden.
Setzt man $a_0=1$, ist die zweite Lösung ist also wieder eine
hypergeometrische Funktion.%, nämlich
%\[
%y_2(x)
%=
%x^{1-c}
%\sum_{k=0}^\infty \frac{(a-c+1)_k(b-c+1)_k}{(2-c)_k}\frac{x^k}{k!}
%=
%x^{1-c}
%\mathstrut_2F_1\biggl(\begin{matrix}a-c+1,b-c+1\\2-c\end{matrix};x\biggr)
%\]
Diese Lösung ist aber nur möglich, wenn in
\eqref{buch:differentialgleichungen:hypergeo:verallgkoef}
der Nenner nicht verschwindet, d.~h.~$2-c+k\ne 0$
oder $c \ne k+2$ für all natürlichen $k$.
$c$ darf also kein natürliche Zahl $\ge 2$ sein.
Wir fassen die Resultate dieses Abschnitts im folgenden Satz zusammen.

\begin{satz}
Die eulersche hypergeometrische Differentialgleichung
\begin{equation}
x(1-x)\frac{d^2y}{dx^2}
+(c+(a+b+1)x)\frac{dy}{dx}
-ab y
=
0
\end{equation}
hat die Lösung
\[
y_1(x)
=
\mathstrut_2F_1\biggl(\begin{matrix}a,b\\c\end{matrix};x\biggr).
\]
Falls $c-2\not\in \mathbb{N}$ ist, ist
\[
y_2(x)
=
x^{1-c} \mathstrut_2F_1\biggl(\begin{matrix}a-c+1,b-c+1\\2-c\end{matrix};x\biggr)
\]
eine zweite, linear unabhängige Lösung.
\end{satz}

%
% Die verallgemeinerte hypergeometrische Differentialgleichung
%
\subsection{Verallgemeinerte hypergeometrische Differentialgleichung}
% https://de.wikipedia.org/wiki/Verallgemeinerte_hypergeometrische_Funktion







\section*{Übungsaufgaben}
\rhead{Übungsaufgaben}
\aufgabetoplevel{chapters/040-rekursion/uebungsaufgaben}
\begin{uebungsaufgaben}
%\uebungsaufgabe{0}
\uebungsaufgabe{1}
\uebungsaufgabe{2}
\end{uebungsaufgaben}


%%
% chapter.tex -- Beschreibung des Inhaltes
%
% (c) 2021 Prof Dr Andreas Müller, Hochschule Rapperswil
%
% !TeX spellcheck = de_CH
\chapter{Spezielle Funktionen und Rekursion
\label{buch:chapter:rekursion}}
\lhead{Spezielle Funktionen und Rekursion}
\rhead{}

%
% gamma.tex -- Abschnitt über die Gamma-funktion
%
% (c) 2021 Prof Dr Andreas Müller, OST Ostschweizer Fachhochschule
%
\section{Die Gamma-Funktion
\label{buch:rekursion:section:gamma}}
Die Fakultät $x!$ kann rekursiv durch 
\[
	x! = x\cdot (x-1)! \qquad\text{und}\qquad 0!=1
\]
für alle natürlichen Zahlen $x\in\mathbb{N}$ definiert werden.
Äquivalent damit ist eine Funktion 
\begin{equation}
\Gamma(x+1) = x\Gamma(x)
\qquad\text{und}\qquad 
\Gamma(1)=1.
\label{buch:rekursion:eqn:gammadef}
\end{equation}
Kann man eine reelle oder komplexe Funktion finden, die die
Funktionalgleichung~\eqref{buch:rekursion:eqn:gammadef}
erfüllt und damit die Fakultät auf beliebige Argumente ausdehnt?

\subsection{Integralformel für die Gamma-Funktion}
Euler hat die folgende Integraldefinition der Gamma-Funktion gegeben.

\begin{definition}
\label{buch:rekursion:def:gamma}
Die Gamma-Funktion ist die Funktion 
\[
\Gamma
\colon
\{z\in\mathbb{C} \mid \operatorname{Re}z>0\}
\to \mathbb{C}
:
z
\mapsto
\Gamma(z) = \int_0^\infty t^{x-1}e^{-t}\,dt
\]
\end{definition}

Man beachte, dass das Integral für $x=0$ nicht definiert ist, eine
Potenzreihenentwicklung um einen Punkt $x_0$ auf der positiven reellen
Achse kann also höchstens den Konvergenzradius $\varrho=|x_0|$ haben.

\begin{figure}
\centering
\includegraphics{chapters/040-rekursion/images/gammaplot.pdf}
\caption{Graph der Gamma-Funktion $z\mapsto\Gamma(z)$ und der alternativen
Funktion $\Gamma(z)+\sin(\pi z)$, die für ganzzahlige Argumente ebenfalls
die Werte der Fakultät annimmt.
\label{buch:rekursion:fig:gamma}}
\end{figure}

\subsubsection{Alternative Lösungen}
Die Funktion $\Gamma(z)$ ist nicht die einzige Funktion, die natürlichen
Zahlen die Werte $\Gamma(n+1) = n!$ der Fakultät annimmt.
Indem man eine beliebige Funktion $f(z)$ addiert, die auf alle
natürlichen Zahlen verschwindet, also $f(n)=0$ für $n\in\mathbb{N}$,
erhält man eine weitere Funktion, die auf natürlichen Zahlen
die Werte der Fakultät annimmt.
Ein Beispiel einer solchen Funktion ist
\begin{equation}
z\mapsto f(z)=\Gamma(z) + \sin \pi z,
\label{buch:rekursion:eqn:gammaalternative}
\end{equation}
die Funktion $f(z)=\sin\pi z$ verschwindet sogar auf allen ganzen
Zahlen.

In Abbildung~\ref{buch:rekursion:fig:gamma} ist die Gamma-Funktion
in rot geplotet, die Funktion~\eqref{buch:rekursion:eqn:gammaalternative}
in grün.
Die Punkte $(n,(n-1)!)$ sind in blau bezeichnet, sie sind beiden Graphen
gemeinsam.

\subsubsection{Pol erster Ordnung bei $z=0$}
Wir haben zu prüfen, dass sowohl der Wert $\Gamma(1)$ korrekt ist als
auch die Rekursionsformel~\eqref{buch:rekursion:eqn:gammadef} gilt.
Der Wert für $z=1$ ist
\begin{align*}
\Gamma(1)
&=
\int_0^\infty t^{1-1}e^{-t}\,dt
=
\left[ -e^{-t} \right]_0^\infty
=
1.
\end{align*}
Für die Rekursionsformel kann mit Hilfe von partieller Integration
bekommen:
\begin{align*}
\Gamma(z+1)
&=
\int_0^\infty t^{z+1-1}e^{-t}\,dt
=
\biggl[-t^{z}e^{-t}\biggr]_0^\infty
+
\int_0^\infty z t^{z-1}e^{-t}\,dt
\\
&=
z
\int_0^\infty
t^{z-1}e^{-t}\,dt
=
z \Gamma(z).
\end{align*}

Für $0<z<\varepsilon$ für eine $\varepsilon >0$ folgt aus der 
Funktionalgleichung
\[
\Gamma(z) = \frac{\Gamma(1+z)}{z}.
\]
Da $\Gamma(1)=1$ ist und $\Gamma$ eine in einer
Umgebung von $1$ stetige Funktion ist, kann sie in der Form
\(
\Gamma(1+z)=\Gamma(1) + zf(z)
\)
schreiben, wobei  $f(z)$ eine differenzierbare Funktion ist mit
$f'(1)=\Gamma'(1)$.
Daraus ergibt sich für $\Gamma(z)$ der Ausdruck
\[
\Gamma(z) = \frac{\Gamma(1)}{z} + f(z) = \frac{1}{z} + f(z).
\]
Die Gamma-Funktion hat daher and er Stelle $z=0$ einen Pol erster Ordnung.

\subsubsection{Ausdehnung auf $\operatorname{Re}z<0$}
Die Integralformel konvergiert nicht für $\operatorname{Re}z\le 0$.
Durch analytische Fortsetzung, wie sie im
Abschnitt~\ref{buch:funktionentheorie:section:fortsetzung}
beschrieben wird, kann die Funktion auf ganz $\mathbb{C}$ ausgedehnt
werden, mit Ausnahme einzelner Pole.
Die Funktionalgleichung gilt natürlich für alle $z\in\mathbb{C}$,
für die $\Gamma(z)$ definiert ist.
In einer Umgebung von $z=-n$ gilt
\[
\Gamma(z)
=
\frac{\Gamma(z+1)}{z}
=
\frac{\Gamma(z+2)}{z(z+1)}
=
\frac{\Gamma(z+3)}{z(z+1)(z+2)}
=
\dots
=
\frac{\Gamma(z+n)}{z(z+1)(z+2)\cdots(z+n-1)}
\]
Keiner der Faktoren im Nenner verschwindet in der Nähe von $z=-n$, der
Zähler hat aber einen Pol erster Ordnung an dieser Stelle.
Daher hat auch der Quotient einen Pol erster Ordnung.
Abbildung~\ref{buch:rekursion:fig:gamma} zeigt die Pole bei den
nicht negativen ganzen Zahlen.






%
% linear.tex
%
% (c) 2021 Prof Dr Andreas Müller, OST Ostschweizer Fachhochschule
%
\section{Lineare Rekursionsgleichung mit konstanten Koeffizienten
\label{buch:rekursion:section:linear}}
\rhead{Lineare Rekursionsgleichungen}
Die Funktionalgleichung der Gamma-Funktion, die im
Abschnitt~\ref{buch:rekursion:section:gamma} untersucht wurde,
hat die Form einer linearen Rekursionsgleichung
\[
\Gamma(x+1) = x\Gamma(x),\qquad \Gamma(1) = 1.
\]
Gleichungen, die Werte einer Funktion für verschiedene
Argument in Beziehung setzen, heissen {\em Funktionalgleichungen}.
\index{Funktionalgleichung}%
Es war überraschend schwierig, eine Lösung für Funktionalgleichung
der Gamma-Funktion für beliebige komplexe $x$ zu finden.
In diesem Abschnitt soll daher eine Klasse von Rekursionsgleichungen
näher untersucht werden, für die einfache Lösungen möglich sind.

\subsection{Lineare Differenzengleichungen}

\subsection{Lösung mit Polynomfunktionen}







%
% hypergeometrisch.tex
%
% (c) 2021 Prof Dr Andreas Müller, OST Ostschweizer Fachhochschule
%
\section{Hypergeometrische Differentialgleichung
\label{buch:differentialgleichungen:section:hypergeometrisch}}
Die hypergeometrische Funktion $\mathstrut_2F1(a,b;c;x)$ wurde in
Abschnitt~\ref{buch:rekursion:section:hypergeometrische-funktion}
als Potenzreihe mit sehr speziellen Koeffizienten, die sich aus
Pochhammer-Symbolen.
Es stellt sich aber heraus, dass man sie auch als Lösung einer
gewöhnlichen Differentialgleichung bekommen kann, die bereits
Euler studiert hat.

\subsection{Die Eulersche hypergeometrische Differentialgleichung
\label{buch:differentialgleichung:subsection:euler-hypergeometrisch}}
Die hypergeometrische Funktion $\mathstrut_2F_1(a,b;c;x)$ ist eine
Lösung der {\em Eulerschen hypergeometrischen Differentialgleichung}
(zu unterscheiden von der Eulerschen Differentialgleichung, die sich
immer auf eine lineare Differentialgleichung mit konstanten Koeffizienten
reduzieren lässt)
\begin{equation}
x(1-x) \frac{d^2y}{dx^2} + (c-(a+b+1)x)\frac{dy}{dx} - ab y = 0
\label{buch:differentialgleichungen:hypergeo:eulerdgl}
\end{equation}
Wir prüfen dies nach, indem wir die Definition der hypergeometrischen
Funktion 
\begin{align*}
y(x)
&=
\mathstrut_2F_1(a,b;c;x)
=
\sum_{k=0}^\infty
\frac{(a)_k(b)_k}{(c)_k} \frac{x^k}{k!}
\intertext{mit den Ableitungen}
y'(x)
&=
\sum_{k=1}^\infty 
\frac{(a)_k(b)_k}{(c)_k} \frac{x^{k-1}}{(k-1)!}
\\
y''(x)
&=
\sum_{k=2}^\infty 
\frac{(a)_k(b)_k}{(c)_k} \frac{x^{k-2}}{(k-2)!}
\end{align*}
einsetzen.
Die Gleichung, die sich ergibt, ist
\begin{align*}
0
&=
x(1-x)
\sum_{k=2}^\infty
\frac{(a)_k(b)_k}{(c)_k}\frac{x^{k-2}}{(k-2)!}
+
(c-(a+b+1)x)
\sum_{k=1}^\infty
\frac{(a)_k(b)_k}{(c)_k}\frac{x^{k-1}}{(k-1)!}
-ab
\sum_{k=0}^\infty
\frac{(a)_k(b)_k}{(c)_k} \frac{x^k}{k!}
\\
&=
\sum_{k=2}^\infty
\frac{(a)_k(b)_k}{(c)_k}\frac{x^{k-1}}{(k-2)!}
-
\sum_{k=2}^\infty
\frac{(a)_k(b)_k}{(c)_k}\frac{x^k}{(k-2)!}
+
c\sum_{k=1}^\infty
\frac{(a)_k(b)_k}{(c)_k}\frac{x^{k-1}}{(k-1)!}
\\
&\qquad
-(a+b+1)
\sum_{k=1}^\infty
\frac{(a)_k(b)_k}{(c)_k}\frac{x^k}{(k-1)!}
-ab
\sum_{k=0}^\infty
\frac{(a)_k(b)_k}{(c)_k} \frac{x^k}{k!}
\\
&=
\sum_{k=1}^\infty
\frac{(a)_{k+1}(b)_{k+1}}{(c)_{k+1}}\frac{x^k}{(k-1)!}
-
\sum_{k=2}^\infty
\frac{(a)_k(b)_k}{(c)_k}\frac{x^k}{(k-2)!}
+
c\sum_{k=0}^\infty
\frac{(a)_{k+1}(b)_{k+1}}{(c)_{k+1}}\frac{x^k}{k!}
\\
&\qquad
-(a+b+1)
\sum_{k=1}^\infty
\frac{(a)_k(b)_k}{(c)_k}\frac{x^k}{(k-1)!}
-ab
\sum_{k=0}^\infty
\frac{(a)_k(b)_k}{(c)_k} \frac{x^k}{k!}.
\end{align*}
Zum konstanten Koeffizienten für $k=0$ tragen nur die dritte und letzte
Summe bei, dies sind die Terme
\[
c\frac{(a)_1(b)_1}{(c)_1}-ab\frac{(a)_0(b)_0}{(c)_0}
=
c\frac{ab}{c}-ab\frac{1\cdot 1}{1}
=
0.
\]
Für den linearen Term $k=1$ kommen je ein Term aus der ersten aund vierten
Summe hinzu, dies ergibt
\begin{align*}
&\phantom{\mathstrut=\mathstrut}
\frac{(a)_2(b)_2}{(c)_2}
+c\frac{(a)_2(b)_2}{(c)_2}
-(a+b+1)\frac{(a)_1(b)_1}{(c)_1}
-ab\frac{(a)_1(b)_1}{(c)_1}
\\
&=
\frac{a(a+1)b(b+1)}{c(c+1)}
(1+c)
-(ab+a+b+1)
\frac{ab}{c}
\\
&=
\frac{a(a+1)b(b+1)}{c}
-
(a+1)(b+1)\frac{ab}{c}
=0.
\end{align*}
Durch Koeffizientenvergleich erhalten wir für $k\ge 2$ 
\begin{align*}
0
&=
\frac{(a)_{k+1}(b)_{k+1}}{(c)_{k+1}} \frac1{(k-1)!} 
-
\frac{(a)_k(b)_k}{(c)_k} \frac1{(k-2)!} 
+
c\frac{(a)_{k+1}(b)_{k+1}}{(c)_{k+1}} \frac{1}{k!}
\\
&\qquad
-(a+b+1)\frac{(a)_k(b)_k}{(c)_k}\frac{1}{(k-1)!}
-ab \frac{(a)_k(b)_k}{(c)_k}\frac{1}{k!}
\\
&=
\frac{(a)_k(b)_k}{(c)_{k+1}}
\frac{1}{k!}
\biggl(
(a+k)(b+k)k
-(c+k)(k-1)k
+
c(a+k)(b+k)
\\
&\qquad
\qquad
\qquad
-(a+b+1)(c+k)k
-ab(c+k)
\biggr).
\intertext{Der zweite, vierte und fünfte Term können zu}
&=
\frac{(a)_k(b)_k}{(c)_{k+1}}
\frac{1}{k!}
\biggl(
(a+k)(b+k)k
+
c(a+k)(b+k)
-(ab+ak+bk+k^2)(c+k)
\biggr)
\intertext{zusammengefasst werden.
Der Faktor $(ab+ak+bk+k^2)$ kann als Produkt $(a+k)(b+k)$ faktorisiert
werden, der dann als gemeinsamer Faktor aus allen Termen ausgeklammert
werden kann:}
&=
\frac{(a)_k(b)_k}{(c)_{k+1}}
\frac{1}{k!}
\biggl(
(a+k)(b+k)k
+
c(a+k)(b+k)
-(a+k)(b+k)(c+k)
\biggr)
\\
&=
\frac{(a)_{k+1}(b)_{k+1}}{(c)_{k+1}}
\frac{1}{k!}
\biggl(
k
+
c
-(c+k)
\biggr)
=0.
\end{align*}
Damit ist gezeigt, dass $\mathstrut_2F_1(a,b;c;x)$ eine Lösung
der Differentialgleichung ist.

Die hypergeometrische Reihe kann auch direkt mit Hilfe der
Potenzreihenmethode als Lösung der Differentialgleichung gefunden 
werden.

\subsection{Lösung als verallgemeinerte Potenzreihe}
Da die hypergeometrische Reihe eine Differentialgleichung
zweiter Ordnung mit einer Singularität bei $x=0$ ist, 
kann man versuchen eine zweite, linear unabhängige Lösung mit
Hilfe der Methode der verallgemeinerten Potenzreihen zu finden.
Dazu setzt man die Lösung in der Form
\begin{align*}
y_2(x)
&=
\sum_{k=0}^\infty a_kx^{\varrho+k}
&
&\Rightarrow&
y_2'(x)
&=
\sum_{k=0}^\infty (\varrho+k)a_kx^{\varrho+k-1}
\\
&&
&&
y_2''(x)
&=
\sum_{k=0}^\infty (\varrho+k)(\varrho+k-1)a_kx^{\varrho+k-2}
\end{align*}
an, wobei $a_0\ne 0$ sein soll.
Einsetzen in die Differentialgleichung ergibt
\begin{align*}
0&=
x(1-x)y_2''(x) + (c-(a+b+1)x) y_2'(x) -aby_2(x)
\\
&=
x(1-x)
\sum_{k=0}^\infty (\varrho+k)(\varrho+k-1)a_kx^{\varrho+k-2}
+
(c-(a+b+1)x)
\sum_{k=0}^\infty (\varrho+k)a_kx^{\varrho+k-1}
-
abx^{\varrho}\sum_{k=0}^\infty a_kx^{\varrho+k}
\\
&=
-\sum_{k=0}^\infty (\varrho+k)(\varrho+k-1)a_kx^{\varrho+k}
+
\sum_{k=0}^\infty (\varrho+k)(\varrho+k-1)a_kx^{\varrho+k-1}
+
c
\sum_{k=0}^\infty (\varrho+k)a_kx^{\varrho+k-1}
\\
&\qquad
-
(a+b+1)
\sum_{k=0}^\infty (\varrho+k)a_kx^{\varrho+k}
-
ab
\sum_{k=0}^\infty a_kx^{\varrho+k}.
\intertext{Durch Verschiebung des Summationsindex in der zweiten
und dritten Summe wird der Koeffizientenvergleich etwas
einfacher}
&=
-\sum_{k=0}^\infty (\varrho+k)(\varrho+k-1)a_kx^{\varrho+k}
+
\sum_{k=-1}^\infty (\varrho+k+1)(\varrho+k)a_{k+1}x^{\varrho+k}
+
c
\sum_{k=-1}^\infty (\varrho+k+1)a_{k+1}x^{\varrho+k}
\\
&\qquad
-
(a+b+1)
\sum_{k=0}^\infty (\varrho+k)a_kx^{\varrho+k}
-
ab
\sum_{k=0}^\infty a_kx^{\varrho+k}
\\
&=
-\sum_{k=0}^\infty (\varrho+k)(\varrho+k-1)a_kx^{\varrho+k}
+
\sum_{k=-1}^\infty (\varrho+k+1)(\varrho+k+c)a_{k+1}x^{\varrho+k}
\\
&\qquad
-
\sum_{k=0}^\infty ((\varrho+k)(a+b+1)+ab)a_kx^{\varrho+k}
\\
&=
\bigl(
\varrho(\varrho-1)
+c\varrho \bigr)
x^{\varrho-1}
+
\sum_{k=0}^\infty
\bigl(
-(\varrho+k)(\varrho+k-1)a_k
+(\varrho+k+1)(\varrho+k+c)a_{k+1}
\\
&
\qquad
\qquad
\qquad
\qquad
\qquad
\qquad
-((\varrho+k)(a+b+1)+ab)a_k
\bigr)
x^{\varrho+k}.
\end{align*}
Aus dem ersten Term kann man die Indexgleichung
\[
0
=
\varrho(\varrho-1)+c\varrho
=
\varrho(\varrho-1+c)
\]
ablesen, die die Nullstellen $\varrho=0$ und $\varrho=1-c$ hat.
Die Nullstelle $\varrho=0$ ergibt natürlich die bereits gefundene
hypergeometrische Reihe.

Nach Einsetzen der zweiten Lösung der Indexgleichung in der Summe
legt der Koeffizientenvergleich eine Beziehung
\begin{align}
0
&=
\bigl(
-(k-c+1)(k-c)
-(k-c+1)(a+b+1)+ab
\bigr)a_k
+
(k-c+2)(k+1)
a_{k+1} 
\notag
\intertext{zwischen $a_k$ und $a_{k+1}$ fest.
Daraus kann man den Quotienten aufeinanderfolgender
Koeffizienten als}
\frac{a_{k+1}}{a_k}
&=
\frac{
-(k-c+1)(k-c)
-(k-c+1)(a+b+1)+ab
}{
\notag
(k-c+2)(k+1)
}
\\
&=
%(%i4) factor(coeff(y,q,0))
%(%o4)                  - (k - c + a + 1) (k - c + b + 1)
%(%i5) factor(coeff(y,q,1))
%(%o5)                         (k + 1) (k - c + 2)
\frac{
(a-c+1+k)
(b-c+1+k)
}{
(2-c+k)(k+1)
}
\label{buch:differentialgleichungen:hypergeo:verallgkoef}
\end{align}
finden.
Setzt man $a_0=1$, ist die zweite Lösung ist also wieder eine
hypergeometrische Funktion.%, nämlich
%\[
%y_2(x)
%=
%x^{1-c}
%\sum_{k=0}^\infty \frac{(a-c+1)_k(b-c+1)_k}{(2-c)_k}\frac{x^k}{k!}
%=
%x^{1-c}
%\mathstrut_2F_1\biggl(\begin{matrix}a-c+1,b-c+1\\2-c\end{matrix};x\biggr)
%\]
Diese Lösung ist aber nur möglich, wenn in
\eqref{buch:differentialgleichungen:hypergeo:verallgkoef}
der Nenner nicht verschwindet, d.~h.~$2-c+k\ne 0$
oder $c \ne k+2$ für all natürlichen $k$.
$c$ darf also kein natürliche Zahl $\ge 2$ sein.
Wir fassen die Resultate dieses Abschnitts im folgenden Satz zusammen.

\begin{satz}
Die eulersche hypergeometrische Differentialgleichung
\begin{equation}
x(1-x)\frac{d^2y}{dx^2}
+(c+(a+b+1)x)\frac{dy}{dx}
-ab y
=
0
\end{equation}
hat die Lösung
\[
y_1(x)
=
\mathstrut_2F_1\biggl(\begin{matrix}a,b\\c\end{matrix};x\biggr).
\]
Falls $c-2\not\in \mathbb{N}$ ist, ist
\[
y_2(x)
=
x^{1-c} \mathstrut_2F_1\biggl(\begin{matrix}a-c+1,b-c+1\\2-c\end{matrix};x\biggr)
\]
eine zweite, linear unabhängige Lösung.
\end{satz}

%
% Die verallgemeinerte hypergeometrische Differentialgleichung
%
\subsection{Verallgemeinerte hypergeometrische Differentialgleichung}
% https://de.wikipedia.org/wiki/Verallgemeinerte_hypergeometrische_Funktion







\section*{Übungsaufgaben}
\rhead{Übungsaufgaben}
\aufgabetoplevel{chapters/040-rekursion/uebungsaufgaben}
\begin{uebungsaufgaben}
%\uebungsaufgabe{0}
\uebungsaufgabe{1}
\uebungsaufgabe{2}
\end{uebungsaufgaben}


%%
% chapter.tex -- Beschreibung des Inhaltes
%
% (c) 2021 Prof Dr Andreas Müller, Hochschule Rapperswil
%
% !TeX spellcheck = de_CH
\chapter{Spezielle Funktionen und Rekursion
\label{buch:chapter:rekursion}}
\lhead{Spezielle Funktionen und Rekursion}
\rhead{}

%
% gamma.tex -- Abschnitt über die Gamma-funktion
%
% (c) 2021 Prof Dr Andreas Müller, OST Ostschweizer Fachhochschule
%
\section{Die Gamma-Funktion
\label{buch:rekursion:section:gamma}}
Die Fakultät $x!$ kann rekursiv durch 
\[
	x! = x\cdot (x-1)! \qquad\text{und}\qquad 0!=1
\]
für alle natürlichen Zahlen $x\in\mathbb{N}$ definiert werden.
Äquivalent damit ist eine Funktion 
\begin{equation}
\Gamma(x+1) = x\Gamma(x)
\qquad\text{und}\qquad 
\Gamma(1)=1.
\label{buch:rekursion:eqn:gammadef}
\end{equation}
Kann man eine reelle oder komplexe Funktion finden, die die
Funktionalgleichung~\eqref{buch:rekursion:eqn:gammadef}
erfüllt und damit die Fakultät auf beliebige Argumente ausdehnt?

\subsection{Integralformel für die Gamma-Funktion}
Euler hat die folgende Integraldefinition der Gamma-Funktion gegeben.

\begin{definition}
\label{buch:rekursion:def:gamma}
Die Gamma-Funktion ist die Funktion 
\[
\Gamma
\colon
\{z\in\mathbb{C} \mid \operatorname{Re}z>0\}
\to \mathbb{C}
:
z
\mapsto
\Gamma(z) = \int_0^\infty t^{x-1}e^{-t}\,dt
\]
\end{definition}

Man beachte, dass das Integral für $x=0$ nicht definiert ist, eine
Potenzreihenentwicklung um einen Punkt $x_0$ auf der positiven reellen
Achse kann also höchstens den Konvergenzradius $\varrho=|x_0|$ haben.

\begin{figure}
\centering
\includegraphics{chapters/040-rekursion/images/gammaplot.pdf}
\caption{Graph der Gamma-Funktion $z\mapsto\Gamma(z)$ und der alternativen
Funktion $\Gamma(z)+\sin(\pi z)$, die für ganzzahlige Argumente ebenfalls
die Werte der Fakultät annimmt.
\label{buch:rekursion:fig:gamma}}
\end{figure}

\subsubsection{Alternative Lösungen}
Die Funktion $\Gamma(z)$ ist nicht die einzige Funktion, die natürlichen
Zahlen die Werte $\Gamma(n+1) = n!$ der Fakultät annimmt.
Indem man eine beliebige Funktion $f(z)$ addiert, die auf alle
natürlichen Zahlen verschwindet, also $f(n)=0$ für $n\in\mathbb{N}$,
erhält man eine weitere Funktion, die auf natürlichen Zahlen
die Werte der Fakultät annimmt.
Ein Beispiel einer solchen Funktion ist
\begin{equation}
z\mapsto f(z)=\Gamma(z) + \sin \pi z,
\label{buch:rekursion:eqn:gammaalternative}
\end{equation}
die Funktion $f(z)=\sin\pi z$ verschwindet sogar auf allen ganzen
Zahlen.

In Abbildung~\ref{buch:rekursion:fig:gamma} ist die Gamma-Funktion
in rot geplotet, die Funktion~\eqref{buch:rekursion:eqn:gammaalternative}
in grün.
Die Punkte $(n,(n-1)!)$ sind in blau bezeichnet, sie sind beiden Graphen
gemeinsam.

\subsubsection{Pol erster Ordnung bei $z=0$}
Wir haben zu prüfen, dass sowohl der Wert $\Gamma(1)$ korrekt ist als
auch die Rekursionsformel~\eqref{buch:rekursion:eqn:gammadef} gilt.
Der Wert für $z=1$ ist
\begin{align*}
\Gamma(1)
&=
\int_0^\infty t^{1-1}e^{-t}\,dt
=
\left[ -e^{-t} \right]_0^\infty
=
1.
\end{align*}
Für die Rekursionsformel kann mit Hilfe von partieller Integration
bekommen:
\begin{align*}
\Gamma(z+1)
&=
\int_0^\infty t^{z+1-1}e^{-t}\,dt
=
\biggl[-t^{z}e^{-t}\biggr]_0^\infty
+
\int_0^\infty z t^{z-1}e^{-t}\,dt
\\
&=
z
\int_0^\infty
t^{z-1}e^{-t}\,dt
=
z \Gamma(z).
\end{align*}

Für $0<z<\varepsilon$ für eine $\varepsilon >0$ folgt aus der 
Funktionalgleichung
\[
\Gamma(z) = \frac{\Gamma(1+z)}{z}.
\]
Da $\Gamma(1)=1$ ist und $\Gamma$ eine in einer
Umgebung von $1$ stetige Funktion ist, kann sie in der Form
\(
\Gamma(1+z)=\Gamma(1) + zf(z)
\)
schreiben, wobei  $f(z)$ eine differenzierbare Funktion ist mit
$f'(1)=\Gamma'(1)$.
Daraus ergibt sich für $\Gamma(z)$ der Ausdruck
\[
\Gamma(z) = \frac{\Gamma(1)}{z} + f(z) = \frac{1}{z} + f(z).
\]
Die Gamma-Funktion hat daher and er Stelle $z=0$ einen Pol erster Ordnung.

\subsubsection{Ausdehnung auf $\operatorname{Re}z<0$}
Die Integralformel konvergiert nicht für $\operatorname{Re}z\le 0$.
Durch analytische Fortsetzung, wie sie im
Abschnitt~\ref{buch:funktionentheorie:section:fortsetzung}
beschrieben wird, kann die Funktion auf ganz $\mathbb{C}$ ausgedehnt
werden, mit Ausnahme einzelner Pole.
Die Funktionalgleichung gilt natürlich für alle $z\in\mathbb{C}$,
für die $\Gamma(z)$ definiert ist.
In einer Umgebung von $z=-n$ gilt
\[
\Gamma(z)
=
\frac{\Gamma(z+1)}{z}
=
\frac{\Gamma(z+2)}{z(z+1)}
=
\frac{\Gamma(z+3)}{z(z+1)(z+2)}
=
\dots
=
\frac{\Gamma(z+n)}{z(z+1)(z+2)\cdots(z+n-1)}
\]
Keiner der Faktoren im Nenner verschwindet in der Nähe von $z=-n$, der
Zähler hat aber einen Pol erster Ordnung an dieser Stelle.
Daher hat auch der Quotient einen Pol erster Ordnung.
Abbildung~\ref{buch:rekursion:fig:gamma} zeigt die Pole bei den
nicht negativen ganzen Zahlen.






%
% linear.tex
%
% (c) 2021 Prof Dr Andreas Müller, OST Ostschweizer Fachhochschule
%
\section{Lineare Rekursionsgleichung mit konstanten Koeffizienten
\label{buch:rekursion:section:linear}}
\rhead{Lineare Rekursionsgleichungen}
Die Funktionalgleichung der Gamma-Funktion, die im
Abschnitt~\ref{buch:rekursion:section:gamma} untersucht wurde,
hat die Form einer linearen Rekursionsgleichung
\[
\Gamma(x+1) = x\Gamma(x),\qquad \Gamma(1) = 1.
\]
Gleichungen, die Werte einer Funktion für verschiedene
Argument in Beziehung setzen, heissen {\em Funktionalgleichungen}.
\index{Funktionalgleichung}%
Es war überraschend schwierig, eine Lösung für Funktionalgleichung
der Gamma-Funktion für beliebige komplexe $x$ zu finden.
In diesem Abschnitt soll daher eine Klasse von Rekursionsgleichungen
näher untersucht werden, für die einfache Lösungen möglich sind.

\subsection{Lineare Differenzengleichungen}

\subsection{Lösung mit Polynomfunktionen}







%
% hypergeometrisch.tex
%
% (c) 2021 Prof Dr Andreas Müller, OST Ostschweizer Fachhochschule
%
\section{Hypergeometrische Differentialgleichung
\label{buch:differentialgleichungen:section:hypergeometrisch}}
Die hypergeometrische Funktion $\mathstrut_2F1(a,b;c;x)$ wurde in
Abschnitt~\ref{buch:rekursion:section:hypergeometrische-funktion}
als Potenzreihe mit sehr speziellen Koeffizienten, die sich aus
Pochhammer-Symbolen.
Es stellt sich aber heraus, dass man sie auch als Lösung einer
gewöhnlichen Differentialgleichung bekommen kann, die bereits
Euler studiert hat.

\subsection{Die Eulersche hypergeometrische Differentialgleichung
\label{buch:differentialgleichung:subsection:euler-hypergeometrisch}}
Die hypergeometrische Funktion $\mathstrut_2F_1(a,b;c;x)$ ist eine
Lösung der {\em Eulerschen hypergeometrischen Differentialgleichung}
(zu unterscheiden von der Eulerschen Differentialgleichung, die sich
immer auf eine lineare Differentialgleichung mit konstanten Koeffizienten
reduzieren lässt)
\begin{equation}
x(1-x) \frac{d^2y}{dx^2} + (c-(a+b+1)x)\frac{dy}{dx} - ab y = 0
\label{buch:differentialgleichungen:hypergeo:eulerdgl}
\end{equation}
Wir prüfen dies nach, indem wir die Definition der hypergeometrischen
Funktion 
\begin{align*}
y(x)
&=
\mathstrut_2F_1(a,b;c;x)
=
\sum_{k=0}^\infty
\frac{(a)_k(b)_k}{(c)_k} \frac{x^k}{k!}
\intertext{mit den Ableitungen}
y'(x)
&=
\sum_{k=1}^\infty 
\frac{(a)_k(b)_k}{(c)_k} \frac{x^{k-1}}{(k-1)!}
\\
y''(x)
&=
\sum_{k=2}^\infty 
\frac{(a)_k(b)_k}{(c)_k} \frac{x^{k-2}}{(k-2)!}
\end{align*}
einsetzen.
Die Gleichung, die sich ergibt, ist
\begin{align*}
0
&=
x(1-x)
\sum_{k=2}^\infty
\frac{(a)_k(b)_k}{(c)_k}\frac{x^{k-2}}{(k-2)!}
+
(c-(a+b+1)x)
\sum_{k=1}^\infty
\frac{(a)_k(b)_k}{(c)_k}\frac{x^{k-1}}{(k-1)!}
-ab
\sum_{k=0}^\infty
\frac{(a)_k(b)_k}{(c)_k} \frac{x^k}{k!}
\\
&=
\sum_{k=2}^\infty
\frac{(a)_k(b)_k}{(c)_k}\frac{x^{k-1}}{(k-2)!}
-
\sum_{k=2}^\infty
\frac{(a)_k(b)_k}{(c)_k}\frac{x^k}{(k-2)!}
+
c\sum_{k=1}^\infty
\frac{(a)_k(b)_k}{(c)_k}\frac{x^{k-1}}{(k-1)!}
\\
&\qquad
-(a+b+1)
\sum_{k=1}^\infty
\frac{(a)_k(b)_k}{(c)_k}\frac{x^k}{(k-1)!}
-ab
\sum_{k=0}^\infty
\frac{(a)_k(b)_k}{(c)_k} \frac{x^k}{k!}
\\
&=
\sum_{k=1}^\infty
\frac{(a)_{k+1}(b)_{k+1}}{(c)_{k+1}}\frac{x^k}{(k-1)!}
-
\sum_{k=2}^\infty
\frac{(a)_k(b)_k}{(c)_k}\frac{x^k}{(k-2)!}
+
c\sum_{k=0}^\infty
\frac{(a)_{k+1}(b)_{k+1}}{(c)_{k+1}}\frac{x^k}{k!}
\\
&\qquad
-(a+b+1)
\sum_{k=1}^\infty
\frac{(a)_k(b)_k}{(c)_k}\frac{x^k}{(k-1)!}
-ab
\sum_{k=0}^\infty
\frac{(a)_k(b)_k}{(c)_k} \frac{x^k}{k!}.
\end{align*}
Zum konstanten Koeffizienten für $k=0$ tragen nur die dritte und letzte
Summe bei, dies sind die Terme
\[
c\frac{(a)_1(b)_1}{(c)_1}-ab\frac{(a)_0(b)_0}{(c)_0}
=
c\frac{ab}{c}-ab\frac{1\cdot 1}{1}
=
0.
\]
Für den linearen Term $k=1$ kommen je ein Term aus der ersten aund vierten
Summe hinzu, dies ergibt
\begin{align*}
&\phantom{\mathstrut=\mathstrut}
\frac{(a)_2(b)_2}{(c)_2}
+c\frac{(a)_2(b)_2}{(c)_2}
-(a+b+1)\frac{(a)_1(b)_1}{(c)_1}
-ab\frac{(a)_1(b)_1}{(c)_1}
\\
&=
\frac{a(a+1)b(b+1)}{c(c+1)}
(1+c)
-(ab+a+b+1)
\frac{ab}{c}
\\
&=
\frac{a(a+1)b(b+1)}{c}
-
(a+1)(b+1)\frac{ab}{c}
=0.
\end{align*}
Durch Koeffizientenvergleich erhalten wir für $k\ge 2$ 
\begin{align*}
0
&=
\frac{(a)_{k+1}(b)_{k+1}}{(c)_{k+1}} \frac1{(k-1)!} 
-
\frac{(a)_k(b)_k}{(c)_k} \frac1{(k-2)!} 
+
c\frac{(a)_{k+1}(b)_{k+1}}{(c)_{k+1}} \frac{1}{k!}
\\
&\qquad
-(a+b+1)\frac{(a)_k(b)_k}{(c)_k}\frac{1}{(k-1)!}
-ab \frac{(a)_k(b)_k}{(c)_k}\frac{1}{k!}
\\
&=
\frac{(a)_k(b)_k}{(c)_{k+1}}
\frac{1}{k!}
\biggl(
(a+k)(b+k)k
-(c+k)(k-1)k
+
c(a+k)(b+k)
\\
&\qquad
\qquad
\qquad
-(a+b+1)(c+k)k
-ab(c+k)
\biggr).
\intertext{Der zweite, vierte und fünfte Term können zu}
&=
\frac{(a)_k(b)_k}{(c)_{k+1}}
\frac{1}{k!}
\biggl(
(a+k)(b+k)k
+
c(a+k)(b+k)
-(ab+ak+bk+k^2)(c+k)
\biggr)
\intertext{zusammengefasst werden.
Der Faktor $(ab+ak+bk+k^2)$ kann als Produkt $(a+k)(b+k)$ faktorisiert
werden, der dann als gemeinsamer Faktor aus allen Termen ausgeklammert
werden kann:}
&=
\frac{(a)_k(b)_k}{(c)_{k+1}}
\frac{1}{k!}
\biggl(
(a+k)(b+k)k
+
c(a+k)(b+k)
-(a+k)(b+k)(c+k)
\biggr)
\\
&=
\frac{(a)_{k+1}(b)_{k+1}}{(c)_{k+1}}
\frac{1}{k!}
\biggl(
k
+
c
-(c+k)
\biggr)
=0.
\end{align*}
Damit ist gezeigt, dass $\mathstrut_2F_1(a,b;c;x)$ eine Lösung
der Differentialgleichung ist.

Die hypergeometrische Reihe kann auch direkt mit Hilfe der
Potenzreihenmethode als Lösung der Differentialgleichung gefunden 
werden.

\subsection{Lösung als verallgemeinerte Potenzreihe}
Da die hypergeometrische Reihe eine Differentialgleichung
zweiter Ordnung mit einer Singularität bei $x=0$ ist, 
kann man versuchen eine zweite, linear unabhängige Lösung mit
Hilfe der Methode der verallgemeinerten Potenzreihen zu finden.
Dazu setzt man die Lösung in der Form
\begin{align*}
y_2(x)
&=
\sum_{k=0}^\infty a_kx^{\varrho+k}
&
&\Rightarrow&
y_2'(x)
&=
\sum_{k=0}^\infty (\varrho+k)a_kx^{\varrho+k-1}
\\
&&
&&
y_2''(x)
&=
\sum_{k=0}^\infty (\varrho+k)(\varrho+k-1)a_kx^{\varrho+k-2}
\end{align*}
an, wobei $a_0\ne 0$ sein soll.
Einsetzen in die Differentialgleichung ergibt
\begin{align*}
0&=
x(1-x)y_2''(x) + (c-(a+b+1)x) y_2'(x) -aby_2(x)
\\
&=
x(1-x)
\sum_{k=0}^\infty (\varrho+k)(\varrho+k-1)a_kx^{\varrho+k-2}
+
(c-(a+b+1)x)
\sum_{k=0}^\infty (\varrho+k)a_kx^{\varrho+k-1}
-
abx^{\varrho}\sum_{k=0}^\infty a_kx^{\varrho+k}
\\
&=
-\sum_{k=0}^\infty (\varrho+k)(\varrho+k-1)a_kx^{\varrho+k}
+
\sum_{k=0}^\infty (\varrho+k)(\varrho+k-1)a_kx^{\varrho+k-1}
+
c
\sum_{k=0}^\infty (\varrho+k)a_kx^{\varrho+k-1}
\\
&\qquad
-
(a+b+1)
\sum_{k=0}^\infty (\varrho+k)a_kx^{\varrho+k}
-
ab
\sum_{k=0}^\infty a_kx^{\varrho+k}.
\intertext{Durch Verschiebung des Summationsindex in der zweiten
und dritten Summe wird der Koeffizientenvergleich etwas
einfacher}
&=
-\sum_{k=0}^\infty (\varrho+k)(\varrho+k-1)a_kx^{\varrho+k}
+
\sum_{k=-1}^\infty (\varrho+k+1)(\varrho+k)a_{k+1}x^{\varrho+k}
+
c
\sum_{k=-1}^\infty (\varrho+k+1)a_{k+1}x^{\varrho+k}
\\
&\qquad
-
(a+b+1)
\sum_{k=0}^\infty (\varrho+k)a_kx^{\varrho+k}
-
ab
\sum_{k=0}^\infty a_kx^{\varrho+k}
\\
&=
-\sum_{k=0}^\infty (\varrho+k)(\varrho+k-1)a_kx^{\varrho+k}
+
\sum_{k=-1}^\infty (\varrho+k+1)(\varrho+k+c)a_{k+1}x^{\varrho+k}
\\
&\qquad
-
\sum_{k=0}^\infty ((\varrho+k)(a+b+1)+ab)a_kx^{\varrho+k}
\\
&=
\bigl(
\varrho(\varrho-1)
+c\varrho \bigr)
x^{\varrho-1}
+
\sum_{k=0}^\infty
\bigl(
-(\varrho+k)(\varrho+k-1)a_k
+(\varrho+k+1)(\varrho+k+c)a_{k+1}
\\
&
\qquad
\qquad
\qquad
\qquad
\qquad
\qquad
-((\varrho+k)(a+b+1)+ab)a_k
\bigr)
x^{\varrho+k}.
\end{align*}
Aus dem ersten Term kann man die Indexgleichung
\[
0
=
\varrho(\varrho-1)+c\varrho
=
\varrho(\varrho-1+c)
\]
ablesen, die die Nullstellen $\varrho=0$ und $\varrho=1-c$ hat.
Die Nullstelle $\varrho=0$ ergibt natürlich die bereits gefundene
hypergeometrische Reihe.

Nach Einsetzen der zweiten Lösung der Indexgleichung in der Summe
legt der Koeffizientenvergleich eine Beziehung
\begin{align}
0
&=
\bigl(
-(k-c+1)(k-c)
-(k-c+1)(a+b+1)+ab
\bigr)a_k
+
(k-c+2)(k+1)
a_{k+1} 
\notag
\intertext{zwischen $a_k$ und $a_{k+1}$ fest.
Daraus kann man den Quotienten aufeinanderfolgender
Koeffizienten als}
\frac{a_{k+1}}{a_k}
&=
\frac{
-(k-c+1)(k-c)
-(k-c+1)(a+b+1)+ab
}{
\notag
(k-c+2)(k+1)
}
\\
&=
%(%i4) factor(coeff(y,q,0))
%(%o4)                  - (k - c + a + 1) (k - c + b + 1)
%(%i5) factor(coeff(y,q,1))
%(%o5)                         (k + 1) (k - c + 2)
\frac{
(a-c+1+k)
(b-c+1+k)
}{
(2-c+k)(k+1)
}
\label{buch:differentialgleichungen:hypergeo:verallgkoef}
\end{align}
finden.
Setzt man $a_0=1$, ist die zweite Lösung ist also wieder eine
hypergeometrische Funktion.%, nämlich
%\[
%y_2(x)
%=
%x^{1-c}
%\sum_{k=0}^\infty \frac{(a-c+1)_k(b-c+1)_k}{(2-c)_k}\frac{x^k}{k!}
%=
%x^{1-c}
%\mathstrut_2F_1\biggl(\begin{matrix}a-c+1,b-c+1\\2-c\end{matrix};x\biggr)
%\]
Diese Lösung ist aber nur möglich, wenn in
\eqref{buch:differentialgleichungen:hypergeo:verallgkoef}
der Nenner nicht verschwindet, d.~h.~$2-c+k\ne 0$
oder $c \ne k+2$ für all natürlichen $k$.
$c$ darf also kein natürliche Zahl $\ge 2$ sein.
Wir fassen die Resultate dieses Abschnitts im folgenden Satz zusammen.

\begin{satz}
Die eulersche hypergeometrische Differentialgleichung
\begin{equation}
x(1-x)\frac{d^2y}{dx^2}
+(c+(a+b+1)x)\frac{dy}{dx}
-ab y
=
0
\end{equation}
hat die Lösung
\[
y_1(x)
=
\mathstrut_2F_1\biggl(\begin{matrix}a,b\\c\end{matrix};x\biggr).
\]
Falls $c-2\not\in \mathbb{N}$ ist, ist
\[
y_2(x)
=
x^{1-c} \mathstrut_2F_1\biggl(\begin{matrix}a-c+1,b-c+1\\2-c\end{matrix};x\biggr)
\]
eine zweite, linear unabhängige Lösung.
\end{satz}

%
% Die verallgemeinerte hypergeometrische Differentialgleichung
%
\subsection{Verallgemeinerte hypergeometrische Differentialgleichung}
% https://de.wikipedia.org/wiki/Verallgemeinerte_hypergeometrische_Funktion







\section*{Übungsaufgaben}
\rhead{Übungsaufgaben}
\aufgabetoplevel{chapters/040-rekursion/uebungsaufgaben}
\begin{uebungsaufgaben}
%\uebungsaufgabe{0}
\uebungsaufgabe{1}
\uebungsaufgabe{2}
\end{uebungsaufgaben}


%%
% chapter.tex -- Beschreibung des Inhaltes
%
% (c) 2021 Prof Dr Andreas Müller, Hochschule Rapperswil
%
% !TeX spellcheck = de_CH
\chapter{Spezielle Funktionen und Rekursion
\label{buch:chapter:rekursion}}
\lhead{Spezielle Funktionen und Rekursion}
\rhead{}

%
% gamma.tex -- Abschnitt über die Gamma-funktion
%
% (c) 2021 Prof Dr Andreas Müller, OST Ostschweizer Fachhochschule
%
\section{Die Gamma-Funktion
\label{buch:rekursion:section:gamma}}
Die Fakultät $x!$ kann rekursiv durch 
\[
	x! = x\cdot (x-1)! \qquad\text{und}\qquad 0!=1
\]
für alle natürlichen Zahlen $x\in\mathbb{N}$ definiert werden.
Äquivalent damit ist eine Funktion 
\begin{equation}
\Gamma(x+1) = x\Gamma(x)
\qquad\text{und}\qquad 
\Gamma(1)=1.
\label{buch:rekursion:eqn:gammadef}
\end{equation}
Kann man eine reelle oder komplexe Funktion finden, die die
Funktionalgleichung~\eqref{buch:rekursion:eqn:gammadef}
erfüllt und damit die Fakultät auf beliebige Argumente ausdehnt?

\subsection{Integralformel für die Gamma-Funktion}
Euler hat die folgende Integraldefinition der Gamma-Funktion gegeben.

\begin{definition}
\label{buch:rekursion:def:gamma}
Die Gamma-Funktion ist die Funktion 
\[
\Gamma
\colon
\{z\in\mathbb{C} \mid \operatorname{Re}z>0\}
\to \mathbb{C}
:
z
\mapsto
\Gamma(z) = \int_0^\infty t^{x-1}e^{-t}\,dt
\]
\end{definition}

Man beachte, dass das Integral für $x=0$ nicht definiert ist, eine
Potenzreihenentwicklung um einen Punkt $x_0$ auf der positiven reellen
Achse kann also höchstens den Konvergenzradius $\varrho=|x_0|$ haben.

\begin{figure}
\centering
\includegraphics{chapters/040-rekursion/images/gammaplot.pdf}
\caption{Graph der Gamma-Funktion $z\mapsto\Gamma(z)$ und der alternativen
Funktion $\Gamma(z)+\sin(\pi z)$, die für ganzzahlige Argumente ebenfalls
die Werte der Fakultät annimmt.
\label{buch:rekursion:fig:gamma}}
\end{figure}

\subsubsection{Alternative Lösungen}
Die Funktion $\Gamma(z)$ ist nicht die einzige Funktion, die natürlichen
Zahlen die Werte $\Gamma(n+1) = n!$ der Fakultät annimmt.
Indem man eine beliebige Funktion $f(z)$ addiert, die auf alle
natürlichen Zahlen verschwindet, also $f(n)=0$ für $n\in\mathbb{N}$,
erhält man eine weitere Funktion, die auf natürlichen Zahlen
die Werte der Fakultät annimmt.
Ein Beispiel einer solchen Funktion ist
\begin{equation}
z\mapsto f(z)=\Gamma(z) + \sin \pi z,
\label{buch:rekursion:eqn:gammaalternative}
\end{equation}
die Funktion $f(z)=\sin\pi z$ verschwindet sogar auf allen ganzen
Zahlen.

In Abbildung~\ref{buch:rekursion:fig:gamma} ist die Gamma-Funktion
in rot geplotet, die Funktion~\eqref{buch:rekursion:eqn:gammaalternative}
in grün.
Die Punkte $(n,(n-1)!)$ sind in blau bezeichnet, sie sind beiden Graphen
gemeinsam.

\subsubsection{Pol erster Ordnung bei $z=0$}
Wir haben zu prüfen, dass sowohl der Wert $\Gamma(1)$ korrekt ist als
auch die Rekursionsformel~\eqref{buch:rekursion:eqn:gammadef} gilt.
Der Wert für $z=1$ ist
\begin{align*}
\Gamma(1)
&=
\int_0^\infty t^{1-1}e^{-t}\,dt
=
\left[ -e^{-t} \right]_0^\infty
=
1.
\end{align*}
Für die Rekursionsformel kann mit Hilfe von partieller Integration
bekommen:
\begin{align*}
\Gamma(z+1)
&=
\int_0^\infty t^{z+1-1}e^{-t}\,dt
=
\biggl[-t^{z}e^{-t}\biggr]_0^\infty
+
\int_0^\infty z t^{z-1}e^{-t}\,dt
\\
&=
z
\int_0^\infty
t^{z-1}e^{-t}\,dt
=
z \Gamma(z).
\end{align*}

Für $0<z<\varepsilon$ für eine $\varepsilon >0$ folgt aus der 
Funktionalgleichung
\[
\Gamma(z) = \frac{\Gamma(1+z)}{z}.
\]
Da $\Gamma(1)=1$ ist und $\Gamma$ eine in einer
Umgebung von $1$ stetige Funktion ist, kann sie in der Form
\(
\Gamma(1+z)=\Gamma(1) + zf(z)
\)
schreiben, wobei  $f(z)$ eine differenzierbare Funktion ist mit
$f'(1)=\Gamma'(1)$.
Daraus ergibt sich für $\Gamma(z)$ der Ausdruck
\[
\Gamma(z) = \frac{\Gamma(1)}{z} + f(z) = \frac{1}{z} + f(z).
\]
Die Gamma-Funktion hat daher and er Stelle $z=0$ einen Pol erster Ordnung.

\subsubsection{Ausdehnung auf $\operatorname{Re}z<0$}
Die Integralformel konvergiert nicht für $\operatorname{Re}z\le 0$.
Durch analytische Fortsetzung, wie sie im
Abschnitt~\ref{buch:funktionentheorie:section:fortsetzung}
beschrieben wird, kann die Funktion auf ganz $\mathbb{C}$ ausgedehnt
werden, mit Ausnahme einzelner Pole.
Die Funktionalgleichung gilt natürlich für alle $z\in\mathbb{C}$,
für die $\Gamma(z)$ definiert ist.
In einer Umgebung von $z=-n$ gilt
\[
\Gamma(z)
=
\frac{\Gamma(z+1)}{z}
=
\frac{\Gamma(z+2)}{z(z+1)}
=
\frac{\Gamma(z+3)}{z(z+1)(z+2)}
=
\dots
=
\frac{\Gamma(z+n)}{z(z+1)(z+2)\cdots(z+n-1)}
\]
Keiner der Faktoren im Nenner verschwindet in der Nähe von $z=-n$, der
Zähler hat aber einen Pol erster Ordnung an dieser Stelle.
Daher hat auch der Quotient einen Pol erster Ordnung.
Abbildung~\ref{buch:rekursion:fig:gamma} zeigt die Pole bei den
nicht negativen ganzen Zahlen.






%
% linear.tex
%
% (c) 2021 Prof Dr Andreas Müller, OST Ostschweizer Fachhochschule
%
\section{Lineare Rekursionsgleichung mit konstanten Koeffizienten
\label{buch:rekursion:section:linear}}
\rhead{Lineare Rekursionsgleichungen}
Die Funktionalgleichung der Gamma-Funktion, die im
Abschnitt~\ref{buch:rekursion:section:gamma} untersucht wurde,
hat die Form einer linearen Rekursionsgleichung
\[
\Gamma(x+1) = x\Gamma(x),\qquad \Gamma(1) = 1.
\]
Gleichungen, die Werte einer Funktion für verschiedene
Argument in Beziehung setzen, heissen {\em Funktionalgleichungen}.
\index{Funktionalgleichung}%
Es war überraschend schwierig, eine Lösung für Funktionalgleichung
der Gamma-Funktion für beliebige komplexe $x$ zu finden.
In diesem Abschnitt soll daher eine Klasse von Rekursionsgleichungen
näher untersucht werden, für die einfache Lösungen möglich sind.

\subsection{Lineare Differenzengleichungen}

\subsection{Lösung mit Polynomfunktionen}







%
% hypergeometrisch.tex
%
% (c) 2021 Prof Dr Andreas Müller, OST Ostschweizer Fachhochschule
%
\section{Hypergeometrische Differentialgleichung
\label{buch:differentialgleichungen:section:hypergeometrisch}}
Die hypergeometrische Funktion $\mathstrut_2F1(a,b;c;x)$ wurde in
Abschnitt~\ref{buch:rekursion:section:hypergeometrische-funktion}
als Potenzreihe mit sehr speziellen Koeffizienten, die sich aus
Pochhammer-Symbolen.
Es stellt sich aber heraus, dass man sie auch als Lösung einer
gewöhnlichen Differentialgleichung bekommen kann, die bereits
Euler studiert hat.

\subsection{Die Eulersche hypergeometrische Differentialgleichung
\label{buch:differentialgleichung:subsection:euler-hypergeometrisch}}
Die hypergeometrische Funktion $\mathstrut_2F_1(a,b;c;x)$ ist eine
Lösung der {\em Eulerschen hypergeometrischen Differentialgleichung}
(zu unterscheiden von der Eulerschen Differentialgleichung, die sich
immer auf eine lineare Differentialgleichung mit konstanten Koeffizienten
reduzieren lässt)
\begin{equation}
x(1-x) \frac{d^2y}{dx^2} + (c-(a+b+1)x)\frac{dy}{dx} - ab y = 0
\label{buch:differentialgleichungen:hypergeo:eulerdgl}
\end{equation}
Wir prüfen dies nach, indem wir die Definition der hypergeometrischen
Funktion 
\begin{align*}
y(x)
&=
\mathstrut_2F_1(a,b;c;x)
=
\sum_{k=0}^\infty
\frac{(a)_k(b)_k}{(c)_k} \frac{x^k}{k!}
\intertext{mit den Ableitungen}
y'(x)
&=
\sum_{k=1}^\infty 
\frac{(a)_k(b)_k}{(c)_k} \frac{x^{k-1}}{(k-1)!}
\\
y''(x)
&=
\sum_{k=2}^\infty 
\frac{(a)_k(b)_k}{(c)_k} \frac{x^{k-2}}{(k-2)!}
\end{align*}
einsetzen.
Die Gleichung, die sich ergibt, ist
\begin{align*}
0
&=
x(1-x)
\sum_{k=2}^\infty
\frac{(a)_k(b)_k}{(c)_k}\frac{x^{k-2}}{(k-2)!}
+
(c-(a+b+1)x)
\sum_{k=1}^\infty
\frac{(a)_k(b)_k}{(c)_k}\frac{x^{k-1}}{(k-1)!}
-ab
\sum_{k=0}^\infty
\frac{(a)_k(b)_k}{(c)_k} \frac{x^k}{k!}
\\
&=
\sum_{k=2}^\infty
\frac{(a)_k(b)_k}{(c)_k}\frac{x^{k-1}}{(k-2)!}
-
\sum_{k=2}^\infty
\frac{(a)_k(b)_k}{(c)_k}\frac{x^k}{(k-2)!}
+
c\sum_{k=1}^\infty
\frac{(a)_k(b)_k}{(c)_k}\frac{x^{k-1}}{(k-1)!}
\\
&\qquad
-(a+b+1)
\sum_{k=1}^\infty
\frac{(a)_k(b)_k}{(c)_k}\frac{x^k}{(k-1)!}
-ab
\sum_{k=0}^\infty
\frac{(a)_k(b)_k}{(c)_k} \frac{x^k}{k!}
\\
&=
\sum_{k=1}^\infty
\frac{(a)_{k+1}(b)_{k+1}}{(c)_{k+1}}\frac{x^k}{(k-1)!}
-
\sum_{k=2}^\infty
\frac{(a)_k(b)_k}{(c)_k}\frac{x^k}{(k-2)!}
+
c\sum_{k=0}^\infty
\frac{(a)_{k+1}(b)_{k+1}}{(c)_{k+1}}\frac{x^k}{k!}
\\
&\qquad
-(a+b+1)
\sum_{k=1}^\infty
\frac{(a)_k(b)_k}{(c)_k}\frac{x^k}{(k-1)!}
-ab
\sum_{k=0}^\infty
\frac{(a)_k(b)_k}{(c)_k} \frac{x^k}{k!}.
\end{align*}
Zum konstanten Koeffizienten für $k=0$ tragen nur die dritte und letzte
Summe bei, dies sind die Terme
\[
c\frac{(a)_1(b)_1}{(c)_1}-ab\frac{(a)_0(b)_0}{(c)_0}
=
c\frac{ab}{c}-ab\frac{1\cdot 1}{1}
=
0.
\]
Für den linearen Term $k=1$ kommen je ein Term aus der ersten aund vierten
Summe hinzu, dies ergibt
\begin{align*}
&\phantom{\mathstrut=\mathstrut}
\frac{(a)_2(b)_2}{(c)_2}
+c\frac{(a)_2(b)_2}{(c)_2}
-(a+b+1)\frac{(a)_1(b)_1}{(c)_1}
-ab\frac{(a)_1(b)_1}{(c)_1}
\\
&=
\frac{a(a+1)b(b+1)}{c(c+1)}
(1+c)
-(ab+a+b+1)
\frac{ab}{c}
\\
&=
\frac{a(a+1)b(b+1)}{c}
-
(a+1)(b+1)\frac{ab}{c}
=0.
\end{align*}
Durch Koeffizientenvergleich erhalten wir für $k\ge 2$ 
\begin{align*}
0
&=
\frac{(a)_{k+1}(b)_{k+1}}{(c)_{k+1}} \frac1{(k-1)!} 
-
\frac{(a)_k(b)_k}{(c)_k} \frac1{(k-2)!} 
+
c\frac{(a)_{k+1}(b)_{k+1}}{(c)_{k+1}} \frac{1}{k!}
\\
&\qquad
-(a+b+1)\frac{(a)_k(b)_k}{(c)_k}\frac{1}{(k-1)!}
-ab \frac{(a)_k(b)_k}{(c)_k}\frac{1}{k!}
\\
&=
\frac{(a)_k(b)_k}{(c)_{k+1}}
\frac{1}{k!}
\biggl(
(a+k)(b+k)k
-(c+k)(k-1)k
+
c(a+k)(b+k)
\\
&\qquad
\qquad
\qquad
-(a+b+1)(c+k)k
-ab(c+k)
\biggr).
\intertext{Der zweite, vierte und fünfte Term können zu}
&=
\frac{(a)_k(b)_k}{(c)_{k+1}}
\frac{1}{k!}
\biggl(
(a+k)(b+k)k
+
c(a+k)(b+k)
-(ab+ak+bk+k^2)(c+k)
\biggr)
\intertext{zusammengefasst werden.
Der Faktor $(ab+ak+bk+k^2)$ kann als Produkt $(a+k)(b+k)$ faktorisiert
werden, der dann als gemeinsamer Faktor aus allen Termen ausgeklammert
werden kann:}
&=
\frac{(a)_k(b)_k}{(c)_{k+1}}
\frac{1}{k!}
\biggl(
(a+k)(b+k)k
+
c(a+k)(b+k)
-(a+k)(b+k)(c+k)
\biggr)
\\
&=
\frac{(a)_{k+1}(b)_{k+1}}{(c)_{k+1}}
\frac{1}{k!}
\biggl(
k
+
c
-(c+k)
\biggr)
=0.
\end{align*}
Damit ist gezeigt, dass $\mathstrut_2F_1(a,b;c;x)$ eine Lösung
der Differentialgleichung ist.

Die hypergeometrische Reihe kann auch direkt mit Hilfe der
Potenzreihenmethode als Lösung der Differentialgleichung gefunden 
werden.

\subsection{Lösung als verallgemeinerte Potenzreihe}
Da die hypergeometrische Reihe eine Differentialgleichung
zweiter Ordnung mit einer Singularität bei $x=0$ ist, 
kann man versuchen eine zweite, linear unabhängige Lösung mit
Hilfe der Methode der verallgemeinerten Potenzreihen zu finden.
Dazu setzt man die Lösung in der Form
\begin{align*}
y_2(x)
&=
\sum_{k=0}^\infty a_kx^{\varrho+k}
&
&\Rightarrow&
y_2'(x)
&=
\sum_{k=0}^\infty (\varrho+k)a_kx^{\varrho+k-1}
\\
&&
&&
y_2''(x)
&=
\sum_{k=0}^\infty (\varrho+k)(\varrho+k-1)a_kx^{\varrho+k-2}
\end{align*}
an, wobei $a_0\ne 0$ sein soll.
Einsetzen in die Differentialgleichung ergibt
\begin{align*}
0&=
x(1-x)y_2''(x) + (c-(a+b+1)x) y_2'(x) -aby_2(x)
\\
&=
x(1-x)
\sum_{k=0}^\infty (\varrho+k)(\varrho+k-1)a_kx^{\varrho+k-2}
+
(c-(a+b+1)x)
\sum_{k=0}^\infty (\varrho+k)a_kx^{\varrho+k-1}
-
abx^{\varrho}\sum_{k=0}^\infty a_kx^{\varrho+k}
\\
&=
-\sum_{k=0}^\infty (\varrho+k)(\varrho+k-1)a_kx^{\varrho+k}
+
\sum_{k=0}^\infty (\varrho+k)(\varrho+k-1)a_kx^{\varrho+k-1}
+
c
\sum_{k=0}^\infty (\varrho+k)a_kx^{\varrho+k-1}
\\
&\qquad
-
(a+b+1)
\sum_{k=0}^\infty (\varrho+k)a_kx^{\varrho+k}
-
ab
\sum_{k=0}^\infty a_kx^{\varrho+k}.
\intertext{Durch Verschiebung des Summationsindex in der zweiten
und dritten Summe wird der Koeffizientenvergleich etwas
einfacher}
&=
-\sum_{k=0}^\infty (\varrho+k)(\varrho+k-1)a_kx^{\varrho+k}
+
\sum_{k=-1}^\infty (\varrho+k+1)(\varrho+k)a_{k+1}x^{\varrho+k}
+
c
\sum_{k=-1}^\infty (\varrho+k+1)a_{k+1}x^{\varrho+k}
\\
&\qquad
-
(a+b+1)
\sum_{k=0}^\infty (\varrho+k)a_kx^{\varrho+k}
-
ab
\sum_{k=0}^\infty a_kx^{\varrho+k}
\\
&=
-\sum_{k=0}^\infty (\varrho+k)(\varrho+k-1)a_kx^{\varrho+k}
+
\sum_{k=-1}^\infty (\varrho+k+1)(\varrho+k+c)a_{k+1}x^{\varrho+k}
\\
&\qquad
-
\sum_{k=0}^\infty ((\varrho+k)(a+b+1)+ab)a_kx^{\varrho+k}
\\
&=
\bigl(
\varrho(\varrho-1)
+c\varrho \bigr)
x^{\varrho-1}
+
\sum_{k=0}^\infty
\bigl(
-(\varrho+k)(\varrho+k-1)a_k
+(\varrho+k+1)(\varrho+k+c)a_{k+1}
\\
&
\qquad
\qquad
\qquad
\qquad
\qquad
\qquad
-((\varrho+k)(a+b+1)+ab)a_k
\bigr)
x^{\varrho+k}.
\end{align*}
Aus dem ersten Term kann man die Indexgleichung
\[
0
=
\varrho(\varrho-1)+c\varrho
=
\varrho(\varrho-1+c)
\]
ablesen, die die Nullstellen $\varrho=0$ und $\varrho=1-c$ hat.
Die Nullstelle $\varrho=0$ ergibt natürlich die bereits gefundene
hypergeometrische Reihe.

Nach Einsetzen der zweiten Lösung der Indexgleichung in der Summe
legt der Koeffizientenvergleich eine Beziehung
\begin{align}
0
&=
\bigl(
-(k-c+1)(k-c)
-(k-c+1)(a+b+1)+ab
\bigr)a_k
+
(k-c+2)(k+1)
a_{k+1} 
\notag
\intertext{zwischen $a_k$ und $a_{k+1}$ fest.
Daraus kann man den Quotienten aufeinanderfolgender
Koeffizienten als}
\frac{a_{k+1}}{a_k}
&=
\frac{
-(k-c+1)(k-c)
-(k-c+1)(a+b+1)+ab
}{
\notag
(k-c+2)(k+1)
}
\\
&=
%(%i4) factor(coeff(y,q,0))
%(%o4)                  - (k - c + a + 1) (k - c + b + 1)
%(%i5) factor(coeff(y,q,1))
%(%o5)                         (k + 1) (k - c + 2)
\frac{
(a-c+1+k)
(b-c+1+k)
}{
(2-c+k)(k+1)
}
\label{buch:differentialgleichungen:hypergeo:verallgkoef}
\end{align}
finden.
Setzt man $a_0=1$, ist die zweite Lösung ist also wieder eine
hypergeometrische Funktion.%, nämlich
%\[
%y_2(x)
%=
%x^{1-c}
%\sum_{k=0}^\infty \frac{(a-c+1)_k(b-c+1)_k}{(2-c)_k}\frac{x^k}{k!}
%=
%x^{1-c}
%\mathstrut_2F_1\biggl(\begin{matrix}a-c+1,b-c+1\\2-c\end{matrix};x\biggr)
%\]
Diese Lösung ist aber nur möglich, wenn in
\eqref{buch:differentialgleichungen:hypergeo:verallgkoef}
der Nenner nicht verschwindet, d.~h.~$2-c+k\ne 0$
oder $c \ne k+2$ für all natürlichen $k$.
$c$ darf also kein natürliche Zahl $\ge 2$ sein.
Wir fassen die Resultate dieses Abschnitts im folgenden Satz zusammen.

\begin{satz}
Die eulersche hypergeometrische Differentialgleichung
\begin{equation}
x(1-x)\frac{d^2y}{dx^2}
+(c+(a+b+1)x)\frac{dy}{dx}
-ab y
=
0
\end{equation}
hat die Lösung
\[
y_1(x)
=
\mathstrut_2F_1\biggl(\begin{matrix}a,b\\c\end{matrix};x\biggr).
\]
Falls $c-2\not\in \mathbb{N}$ ist, ist
\[
y_2(x)
=
x^{1-c} \mathstrut_2F_1\biggl(\begin{matrix}a-c+1,b-c+1\\2-c\end{matrix};x\biggr)
\]
eine zweite, linear unabhängige Lösung.
\end{satz}

%
% Die verallgemeinerte hypergeometrische Differentialgleichung
%
\subsection{Verallgemeinerte hypergeometrische Differentialgleichung}
% https://de.wikipedia.org/wiki/Verallgemeinerte_hypergeometrische_Funktion







\section*{Übungsaufgaben}
\rhead{Übungsaufgaben}
\aufgabetoplevel{chapters/040-rekursion/uebungsaufgaben}
\begin{uebungsaufgaben}
%\uebungsaufgabe{0}
\uebungsaufgabe{1}
\uebungsaufgabe{2}
\end{uebungsaufgaben}


%%
% chapter.tex -- Beschreibung des Inhaltes
%
% (c) 2021 Prof Dr Andreas Müller, Hochschule Rapperswil
%
% !TeX spellcheck = de_CH
\chapter{Spezielle Funktionen und Rekursion
\label{buch:chapter:rekursion}}
\lhead{Spezielle Funktionen und Rekursion}
\rhead{}

%
% gamma.tex -- Abschnitt über die Gamma-funktion
%
% (c) 2021 Prof Dr Andreas Müller, OST Ostschweizer Fachhochschule
%
\section{Die Gamma-Funktion
\label{buch:rekursion:section:gamma}}
Die Fakultät $x!$ kann rekursiv durch 
\[
	x! = x\cdot (x-1)! \qquad\text{und}\qquad 0!=1
\]
für alle natürlichen Zahlen $x\in\mathbb{N}$ definiert werden.
Äquivalent damit ist eine Funktion 
\begin{equation}
\Gamma(x+1) = x\Gamma(x)
\qquad\text{und}\qquad 
\Gamma(1)=1.
\label{buch:rekursion:eqn:gammadef}
\end{equation}
Kann man eine reelle oder komplexe Funktion finden, die die
Funktionalgleichung~\eqref{buch:rekursion:eqn:gammadef}
erfüllt und damit die Fakultät auf beliebige Argumente ausdehnt?

\subsection{Integralformel für die Gamma-Funktion}
Euler hat die folgende Integraldefinition der Gamma-Funktion gegeben.

\begin{definition}
\label{buch:rekursion:def:gamma}
Die Gamma-Funktion ist die Funktion 
\[
\Gamma
\colon
\{z\in\mathbb{C} \mid \operatorname{Re}z>0\}
\to \mathbb{C}
:
z
\mapsto
\Gamma(z) = \int_0^\infty t^{x-1}e^{-t}\,dt
\]
\end{definition}

Man beachte, dass das Integral für $x=0$ nicht definiert ist, eine
Potenzreihenentwicklung um einen Punkt $x_0$ auf der positiven reellen
Achse kann also höchstens den Konvergenzradius $\varrho=|x_0|$ haben.

\begin{figure}
\centering
\includegraphics{chapters/040-rekursion/images/gammaplot.pdf}
\caption{Graph der Gamma-Funktion $z\mapsto\Gamma(z)$ und der alternativen
Funktion $\Gamma(z)+\sin(\pi z)$, die für ganzzahlige Argumente ebenfalls
die Werte der Fakultät annimmt.
\label{buch:rekursion:fig:gamma}}
\end{figure}

\subsubsection{Alternative Lösungen}
Die Funktion $\Gamma(z)$ ist nicht die einzige Funktion, die natürlichen
Zahlen die Werte $\Gamma(n+1) = n!$ der Fakultät annimmt.
Indem man eine beliebige Funktion $f(z)$ addiert, die auf alle
natürlichen Zahlen verschwindet, also $f(n)=0$ für $n\in\mathbb{N}$,
erhält man eine weitere Funktion, die auf natürlichen Zahlen
die Werte der Fakultät annimmt.
Ein Beispiel einer solchen Funktion ist
\begin{equation}
z\mapsto f(z)=\Gamma(z) + \sin \pi z,
\label{buch:rekursion:eqn:gammaalternative}
\end{equation}
die Funktion $f(z)=\sin\pi z$ verschwindet sogar auf allen ganzen
Zahlen.

In Abbildung~\ref{buch:rekursion:fig:gamma} ist die Gamma-Funktion
in rot geplotet, die Funktion~\eqref{buch:rekursion:eqn:gammaalternative}
in grün.
Die Punkte $(n,(n-1)!)$ sind in blau bezeichnet, sie sind beiden Graphen
gemeinsam.

\subsubsection{Pol erster Ordnung bei $z=0$}
Wir haben zu prüfen, dass sowohl der Wert $\Gamma(1)$ korrekt ist als
auch die Rekursionsformel~\eqref{buch:rekursion:eqn:gammadef} gilt.
Der Wert für $z=1$ ist
\begin{align*}
\Gamma(1)
&=
\int_0^\infty t^{1-1}e^{-t}\,dt
=
\left[ -e^{-t} \right]_0^\infty
=
1.
\end{align*}
Für die Rekursionsformel kann mit Hilfe von partieller Integration
bekommen:
\begin{align*}
\Gamma(z+1)
&=
\int_0^\infty t^{z+1-1}e^{-t}\,dt
=
\biggl[-t^{z}e^{-t}\biggr]_0^\infty
+
\int_0^\infty z t^{z-1}e^{-t}\,dt
\\
&=
z
\int_0^\infty
t^{z-1}e^{-t}\,dt
=
z \Gamma(z).
\end{align*}

Für $0<z<\varepsilon$ für eine $\varepsilon >0$ folgt aus der 
Funktionalgleichung
\[
\Gamma(z) = \frac{\Gamma(1+z)}{z}.
\]
Da $\Gamma(1)=1$ ist und $\Gamma$ eine in einer
Umgebung von $1$ stetige Funktion ist, kann sie in der Form
\(
\Gamma(1+z)=\Gamma(1) + zf(z)
\)
schreiben, wobei  $f(z)$ eine differenzierbare Funktion ist mit
$f'(1)=\Gamma'(1)$.
Daraus ergibt sich für $\Gamma(z)$ der Ausdruck
\[
\Gamma(z) = \frac{\Gamma(1)}{z} + f(z) = \frac{1}{z} + f(z).
\]
Die Gamma-Funktion hat daher and er Stelle $z=0$ einen Pol erster Ordnung.

\subsubsection{Ausdehnung auf $\operatorname{Re}z<0$}
Die Integralformel konvergiert nicht für $\operatorname{Re}z\le 0$.
Durch analytische Fortsetzung, wie sie im
Abschnitt~\ref{buch:funktionentheorie:section:fortsetzung}
beschrieben wird, kann die Funktion auf ganz $\mathbb{C}$ ausgedehnt
werden, mit Ausnahme einzelner Pole.
Die Funktionalgleichung gilt natürlich für alle $z\in\mathbb{C}$,
für die $\Gamma(z)$ definiert ist.
In einer Umgebung von $z=-n$ gilt
\[
\Gamma(z)
=
\frac{\Gamma(z+1)}{z}
=
\frac{\Gamma(z+2)}{z(z+1)}
=
\frac{\Gamma(z+3)}{z(z+1)(z+2)}
=
\dots
=
\frac{\Gamma(z+n)}{z(z+1)(z+2)\cdots(z+n-1)}
\]
Keiner der Faktoren im Nenner verschwindet in der Nähe von $z=-n$, der
Zähler hat aber einen Pol erster Ordnung an dieser Stelle.
Daher hat auch der Quotient einen Pol erster Ordnung.
Abbildung~\ref{buch:rekursion:fig:gamma} zeigt die Pole bei den
nicht negativen ganzen Zahlen.






%
% linear.tex
%
% (c) 2021 Prof Dr Andreas Müller, OST Ostschweizer Fachhochschule
%
\section{Lineare Rekursionsgleichung mit konstanten Koeffizienten
\label{buch:rekursion:section:linear}}
\rhead{Lineare Rekursionsgleichungen}
Die Funktionalgleichung der Gamma-Funktion, die im
Abschnitt~\ref{buch:rekursion:section:gamma} untersucht wurde,
hat die Form einer linearen Rekursionsgleichung
\[
\Gamma(x+1) = x\Gamma(x),\qquad \Gamma(1) = 1.
\]
Gleichungen, die Werte einer Funktion für verschiedene
Argument in Beziehung setzen, heissen {\em Funktionalgleichungen}.
\index{Funktionalgleichung}%
Es war überraschend schwierig, eine Lösung für Funktionalgleichung
der Gamma-Funktion für beliebige komplexe $x$ zu finden.
In diesem Abschnitt soll daher eine Klasse von Rekursionsgleichungen
näher untersucht werden, für die einfache Lösungen möglich sind.

\subsection{Lineare Differenzengleichungen}

\subsection{Lösung mit Polynomfunktionen}







%
% hypergeometrisch.tex
%
% (c) 2021 Prof Dr Andreas Müller, OST Ostschweizer Fachhochschule
%
\section{Hypergeometrische Differentialgleichung
\label{buch:differentialgleichungen:section:hypergeometrisch}}
Die hypergeometrische Funktion $\mathstrut_2F1(a,b;c;x)$ wurde in
Abschnitt~\ref{buch:rekursion:section:hypergeometrische-funktion}
als Potenzreihe mit sehr speziellen Koeffizienten, die sich aus
Pochhammer-Symbolen.
Es stellt sich aber heraus, dass man sie auch als Lösung einer
gewöhnlichen Differentialgleichung bekommen kann, die bereits
Euler studiert hat.

\subsection{Die Eulersche hypergeometrische Differentialgleichung
\label{buch:differentialgleichung:subsection:euler-hypergeometrisch}}
Die hypergeometrische Funktion $\mathstrut_2F_1(a,b;c;x)$ ist eine
Lösung der {\em Eulerschen hypergeometrischen Differentialgleichung}
(zu unterscheiden von der Eulerschen Differentialgleichung, die sich
immer auf eine lineare Differentialgleichung mit konstanten Koeffizienten
reduzieren lässt)
\begin{equation}
x(1-x) \frac{d^2y}{dx^2} + (c-(a+b+1)x)\frac{dy}{dx} - ab y = 0
\label{buch:differentialgleichungen:hypergeo:eulerdgl}
\end{equation}
Wir prüfen dies nach, indem wir die Definition der hypergeometrischen
Funktion 
\begin{align*}
y(x)
&=
\mathstrut_2F_1(a,b;c;x)
=
\sum_{k=0}^\infty
\frac{(a)_k(b)_k}{(c)_k} \frac{x^k}{k!}
\intertext{mit den Ableitungen}
y'(x)
&=
\sum_{k=1}^\infty 
\frac{(a)_k(b)_k}{(c)_k} \frac{x^{k-1}}{(k-1)!}
\\
y''(x)
&=
\sum_{k=2}^\infty 
\frac{(a)_k(b)_k}{(c)_k} \frac{x^{k-2}}{(k-2)!}
\end{align*}
einsetzen.
Die Gleichung, die sich ergibt, ist
\begin{align*}
0
&=
x(1-x)
\sum_{k=2}^\infty
\frac{(a)_k(b)_k}{(c)_k}\frac{x^{k-2}}{(k-2)!}
+
(c-(a+b+1)x)
\sum_{k=1}^\infty
\frac{(a)_k(b)_k}{(c)_k}\frac{x^{k-1}}{(k-1)!}
-ab
\sum_{k=0}^\infty
\frac{(a)_k(b)_k}{(c)_k} \frac{x^k}{k!}
\\
&=
\sum_{k=2}^\infty
\frac{(a)_k(b)_k}{(c)_k}\frac{x^{k-1}}{(k-2)!}
-
\sum_{k=2}^\infty
\frac{(a)_k(b)_k}{(c)_k}\frac{x^k}{(k-2)!}
+
c\sum_{k=1}^\infty
\frac{(a)_k(b)_k}{(c)_k}\frac{x^{k-1}}{(k-1)!}
\\
&\qquad
-(a+b+1)
\sum_{k=1}^\infty
\frac{(a)_k(b)_k}{(c)_k}\frac{x^k}{(k-1)!}
-ab
\sum_{k=0}^\infty
\frac{(a)_k(b)_k}{(c)_k} \frac{x^k}{k!}
\\
&=
\sum_{k=1}^\infty
\frac{(a)_{k+1}(b)_{k+1}}{(c)_{k+1}}\frac{x^k}{(k-1)!}
-
\sum_{k=2}^\infty
\frac{(a)_k(b)_k}{(c)_k}\frac{x^k}{(k-2)!}
+
c\sum_{k=0}^\infty
\frac{(a)_{k+1}(b)_{k+1}}{(c)_{k+1}}\frac{x^k}{k!}
\\
&\qquad
-(a+b+1)
\sum_{k=1}^\infty
\frac{(a)_k(b)_k}{(c)_k}\frac{x^k}{(k-1)!}
-ab
\sum_{k=0}^\infty
\frac{(a)_k(b)_k}{(c)_k} \frac{x^k}{k!}.
\end{align*}
Zum konstanten Koeffizienten für $k=0$ tragen nur die dritte und letzte
Summe bei, dies sind die Terme
\[
c\frac{(a)_1(b)_1}{(c)_1}-ab\frac{(a)_0(b)_0}{(c)_0}
=
c\frac{ab}{c}-ab\frac{1\cdot 1}{1}
=
0.
\]
Für den linearen Term $k=1$ kommen je ein Term aus der ersten aund vierten
Summe hinzu, dies ergibt
\begin{align*}
&\phantom{\mathstrut=\mathstrut}
\frac{(a)_2(b)_2}{(c)_2}
+c\frac{(a)_2(b)_2}{(c)_2}
-(a+b+1)\frac{(a)_1(b)_1}{(c)_1}
-ab\frac{(a)_1(b)_1}{(c)_1}
\\
&=
\frac{a(a+1)b(b+1)}{c(c+1)}
(1+c)
-(ab+a+b+1)
\frac{ab}{c}
\\
&=
\frac{a(a+1)b(b+1)}{c}
-
(a+1)(b+1)\frac{ab}{c}
=0.
\end{align*}
Durch Koeffizientenvergleich erhalten wir für $k\ge 2$ 
\begin{align*}
0
&=
\frac{(a)_{k+1}(b)_{k+1}}{(c)_{k+1}} \frac1{(k-1)!} 
-
\frac{(a)_k(b)_k}{(c)_k} \frac1{(k-2)!} 
+
c\frac{(a)_{k+1}(b)_{k+1}}{(c)_{k+1}} \frac{1}{k!}
\\
&\qquad
-(a+b+1)\frac{(a)_k(b)_k}{(c)_k}\frac{1}{(k-1)!}
-ab \frac{(a)_k(b)_k}{(c)_k}\frac{1}{k!}
\\
&=
\frac{(a)_k(b)_k}{(c)_{k+1}}
\frac{1}{k!}
\biggl(
(a+k)(b+k)k
-(c+k)(k-1)k
+
c(a+k)(b+k)
\\
&\qquad
\qquad
\qquad
-(a+b+1)(c+k)k
-ab(c+k)
\biggr).
\intertext{Der zweite, vierte und fünfte Term können zu}
&=
\frac{(a)_k(b)_k}{(c)_{k+1}}
\frac{1}{k!}
\biggl(
(a+k)(b+k)k
+
c(a+k)(b+k)
-(ab+ak+bk+k^2)(c+k)
\biggr)
\intertext{zusammengefasst werden.
Der Faktor $(ab+ak+bk+k^2)$ kann als Produkt $(a+k)(b+k)$ faktorisiert
werden, der dann als gemeinsamer Faktor aus allen Termen ausgeklammert
werden kann:}
&=
\frac{(a)_k(b)_k}{(c)_{k+1}}
\frac{1}{k!}
\biggl(
(a+k)(b+k)k
+
c(a+k)(b+k)
-(a+k)(b+k)(c+k)
\biggr)
\\
&=
\frac{(a)_{k+1}(b)_{k+1}}{(c)_{k+1}}
\frac{1}{k!}
\biggl(
k
+
c
-(c+k)
\biggr)
=0.
\end{align*}
Damit ist gezeigt, dass $\mathstrut_2F_1(a,b;c;x)$ eine Lösung
der Differentialgleichung ist.

Die hypergeometrische Reihe kann auch direkt mit Hilfe der
Potenzreihenmethode als Lösung der Differentialgleichung gefunden 
werden.

\subsection{Lösung als verallgemeinerte Potenzreihe}
Da die hypergeometrische Reihe eine Differentialgleichung
zweiter Ordnung mit einer Singularität bei $x=0$ ist, 
kann man versuchen eine zweite, linear unabhängige Lösung mit
Hilfe der Methode der verallgemeinerten Potenzreihen zu finden.
Dazu setzt man die Lösung in der Form
\begin{align*}
y_2(x)
&=
\sum_{k=0}^\infty a_kx^{\varrho+k}
&
&\Rightarrow&
y_2'(x)
&=
\sum_{k=0}^\infty (\varrho+k)a_kx^{\varrho+k-1}
\\
&&
&&
y_2''(x)
&=
\sum_{k=0}^\infty (\varrho+k)(\varrho+k-1)a_kx^{\varrho+k-2}
\end{align*}
an, wobei $a_0\ne 0$ sein soll.
Einsetzen in die Differentialgleichung ergibt
\begin{align*}
0&=
x(1-x)y_2''(x) + (c-(a+b+1)x) y_2'(x) -aby_2(x)
\\
&=
x(1-x)
\sum_{k=0}^\infty (\varrho+k)(\varrho+k-1)a_kx^{\varrho+k-2}
+
(c-(a+b+1)x)
\sum_{k=0}^\infty (\varrho+k)a_kx^{\varrho+k-1}
-
abx^{\varrho}\sum_{k=0}^\infty a_kx^{\varrho+k}
\\
&=
-\sum_{k=0}^\infty (\varrho+k)(\varrho+k-1)a_kx^{\varrho+k}
+
\sum_{k=0}^\infty (\varrho+k)(\varrho+k-1)a_kx^{\varrho+k-1}
+
c
\sum_{k=0}^\infty (\varrho+k)a_kx^{\varrho+k-1}
\\
&\qquad
-
(a+b+1)
\sum_{k=0}^\infty (\varrho+k)a_kx^{\varrho+k}
-
ab
\sum_{k=0}^\infty a_kx^{\varrho+k}.
\intertext{Durch Verschiebung des Summationsindex in der zweiten
und dritten Summe wird der Koeffizientenvergleich etwas
einfacher}
&=
-\sum_{k=0}^\infty (\varrho+k)(\varrho+k-1)a_kx^{\varrho+k}
+
\sum_{k=-1}^\infty (\varrho+k+1)(\varrho+k)a_{k+1}x^{\varrho+k}
+
c
\sum_{k=-1}^\infty (\varrho+k+1)a_{k+1}x^{\varrho+k}
\\
&\qquad
-
(a+b+1)
\sum_{k=0}^\infty (\varrho+k)a_kx^{\varrho+k}
-
ab
\sum_{k=0}^\infty a_kx^{\varrho+k}
\\
&=
-\sum_{k=0}^\infty (\varrho+k)(\varrho+k-1)a_kx^{\varrho+k}
+
\sum_{k=-1}^\infty (\varrho+k+1)(\varrho+k+c)a_{k+1}x^{\varrho+k}
\\
&\qquad
-
\sum_{k=0}^\infty ((\varrho+k)(a+b+1)+ab)a_kx^{\varrho+k}
\\
&=
\bigl(
\varrho(\varrho-1)
+c\varrho \bigr)
x^{\varrho-1}
+
\sum_{k=0}^\infty
\bigl(
-(\varrho+k)(\varrho+k-1)a_k
+(\varrho+k+1)(\varrho+k+c)a_{k+1}
\\
&
\qquad
\qquad
\qquad
\qquad
\qquad
\qquad
-((\varrho+k)(a+b+1)+ab)a_k
\bigr)
x^{\varrho+k}.
\end{align*}
Aus dem ersten Term kann man die Indexgleichung
\[
0
=
\varrho(\varrho-1)+c\varrho
=
\varrho(\varrho-1+c)
\]
ablesen, die die Nullstellen $\varrho=0$ und $\varrho=1-c$ hat.
Die Nullstelle $\varrho=0$ ergibt natürlich die bereits gefundene
hypergeometrische Reihe.

Nach Einsetzen der zweiten Lösung der Indexgleichung in der Summe
legt der Koeffizientenvergleich eine Beziehung
\begin{align}
0
&=
\bigl(
-(k-c+1)(k-c)
-(k-c+1)(a+b+1)+ab
\bigr)a_k
+
(k-c+2)(k+1)
a_{k+1} 
\notag
\intertext{zwischen $a_k$ und $a_{k+1}$ fest.
Daraus kann man den Quotienten aufeinanderfolgender
Koeffizienten als}
\frac{a_{k+1}}{a_k}
&=
\frac{
-(k-c+1)(k-c)
-(k-c+1)(a+b+1)+ab
}{
\notag
(k-c+2)(k+1)
}
\\
&=
%(%i4) factor(coeff(y,q,0))
%(%o4)                  - (k - c + a + 1) (k - c + b + 1)
%(%i5) factor(coeff(y,q,1))
%(%o5)                         (k + 1) (k - c + 2)
\frac{
(a-c+1+k)
(b-c+1+k)
}{
(2-c+k)(k+1)
}
\label{buch:differentialgleichungen:hypergeo:verallgkoef}
\end{align}
finden.
Setzt man $a_0=1$, ist die zweite Lösung ist also wieder eine
hypergeometrische Funktion.%, nämlich
%\[
%y_2(x)
%=
%x^{1-c}
%\sum_{k=0}^\infty \frac{(a-c+1)_k(b-c+1)_k}{(2-c)_k}\frac{x^k}{k!}
%=
%x^{1-c}
%\mathstrut_2F_1\biggl(\begin{matrix}a-c+1,b-c+1\\2-c\end{matrix};x\biggr)
%\]
Diese Lösung ist aber nur möglich, wenn in
\eqref{buch:differentialgleichungen:hypergeo:verallgkoef}
der Nenner nicht verschwindet, d.~h.~$2-c+k\ne 0$
oder $c \ne k+2$ für all natürlichen $k$.
$c$ darf also kein natürliche Zahl $\ge 2$ sein.
Wir fassen die Resultate dieses Abschnitts im folgenden Satz zusammen.

\begin{satz}
Die eulersche hypergeometrische Differentialgleichung
\begin{equation}
x(1-x)\frac{d^2y}{dx^2}
+(c+(a+b+1)x)\frac{dy}{dx}
-ab y
=
0
\end{equation}
hat die Lösung
\[
y_1(x)
=
\mathstrut_2F_1\biggl(\begin{matrix}a,b\\c\end{matrix};x\biggr).
\]
Falls $c-2\not\in \mathbb{N}$ ist, ist
\[
y_2(x)
=
x^{1-c} \mathstrut_2F_1\biggl(\begin{matrix}a-c+1,b-c+1\\2-c\end{matrix};x\biggr)
\]
eine zweite, linear unabhängige Lösung.
\end{satz}

%
% Die verallgemeinerte hypergeometrische Differentialgleichung
%
\subsection{Verallgemeinerte hypergeometrische Differentialgleichung}
% https://de.wikipedia.org/wiki/Verallgemeinerte_hypergeometrische_Funktion







\section*{Übungsaufgaben}
\rhead{Übungsaufgaben}
\aufgabetoplevel{chapters/040-rekursion/uebungsaufgaben}
\begin{uebungsaufgaben}
%\uebungsaufgabe{0}
\uebungsaufgabe{1}
\uebungsaufgabe{2}
\end{uebungsaufgaben}


%%
% chapter.tex -- Beschreibung des Inhaltes
%
% (c) 2021 Prof Dr Andreas Müller, Hochschule Rapperswil
%
% !TeX spellcheck = de_CH
\chapter{Spezielle Funktionen und Rekursion
\label{buch:chapter:rekursion}}
\lhead{Spezielle Funktionen und Rekursion}
\rhead{}

%
% gamma.tex -- Abschnitt über die Gamma-funktion
%
% (c) 2021 Prof Dr Andreas Müller, OST Ostschweizer Fachhochschule
%
\section{Die Gamma-Funktion
\label{buch:rekursion:section:gamma}}
Die Fakultät $x!$ kann rekursiv durch 
\[
	x! = x\cdot (x-1)! \qquad\text{und}\qquad 0!=1
\]
für alle natürlichen Zahlen $x\in\mathbb{N}$ definiert werden.
Äquivalent damit ist eine Funktion 
\begin{equation}
\Gamma(x+1) = x\Gamma(x)
\qquad\text{und}\qquad 
\Gamma(1)=1.
\label{buch:rekursion:eqn:gammadef}
\end{equation}
Kann man eine reelle oder komplexe Funktion finden, die die
Funktionalgleichung~\eqref{buch:rekursion:eqn:gammadef}
erfüllt und damit die Fakultät auf beliebige Argumente ausdehnt?

\subsection{Integralformel für die Gamma-Funktion}
Euler hat die folgende Integraldefinition der Gamma-Funktion gegeben.

\begin{definition}
\label{buch:rekursion:def:gamma}
Die Gamma-Funktion ist die Funktion 
\[
\Gamma
\colon
\{z\in\mathbb{C} \mid \operatorname{Re}z>0\}
\to \mathbb{C}
:
z
\mapsto
\Gamma(z) = \int_0^\infty t^{x-1}e^{-t}\,dt
\]
\end{definition}

Man beachte, dass das Integral für $x=0$ nicht definiert ist, eine
Potenzreihenentwicklung um einen Punkt $x_0$ auf der positiven reellen
Achse kann also höchstens den Konvergenzradius $\varrho=|x_0|$ haben.

\begin{figure}
\centering
\includegraphics{chapters/040-rekursion/images/gammaplot.pdf}
\caption{Graph der Gamma-Funktion $z\mapsto\Gamma(z)$ und der alternativen
Funktion $\Gamma(z)+\sin(\pi z)$, die für ganzzahlige Argumente ebenfalls
die Werte der Fakultät annimmt.
\label{buch:rekursion:fig:gamma}}
\end{figure}

\subsubsection{Alternative Lösungen}
Die Funktion $\Gamma(z)$ ist nicht die einzige Funktion, die natürlichen
Zahlen die Werte $\Gamma(n+1) = n!$ der Fakultät annimmt.
Indem man eine beliebige Funktion $f(z)$ addiert, die auf alle
natürlichen Zahlen verschwindet, also $f(n)=0$ für $n\in\mathbb{N}$,
erhält man eine weitere Funktion, die auf natürlichen Zahlen
die Werte der Fakultät annimmt.
Ein Beispiel einer solchen Funktion ist
\begin{equation}
z\mapsto f(z)=\Gamma(z) + \sin \pi z,
\label{buch:rekursion:eqn:gammaalternative}
\end{equation}
die Funktion $f(z)=\sin\pi z$ verschwindet sogar auf allen ganzen
Zahlen.

In Abbildung~\ref{buch:rekursion:fig:gamma} ist die Gamma-Funktion
in rot geplotet, die Funktion~\eqref{buch:rekursion:eqn:gammaalternative}
in grün.
Die Punkte $(n,(n-1)!)$ sind in blau bezeichnet, sie sind beiden Graphen
gemeinsam.

\subsubsection{Pol erster Ordnung bei $z=0$}
Wir haben zu prüfen, dass sowohl der Wert $\Gamma(1)$ korrekt ist als
auch die Rekursionsformel~\eqref{buch:rekursion:eqn:gammadef} gilt.
Der Wert für $z=1$ ist
\begin{align*}
\Gamma(1)
&=
\int_0^\infty t^{1-1}e^{-t}\,dt
=
\left[ -e^{-t} \right]_0^\infty
=
1.
\end{align*}
Für die Rekursionsformel kann mit Hilfe von partieller Integration
bekommen:
\begin{align*}
\Gamma(z+1)
&=
\int_0^\infty t^{z+1-1}e^{-t}\,dt
=
\biggl[-t^{z}e^{-t}\biggr]_0^\infty
+
\int_0^\infty z t^{z-1}e^{-t}\,dt
\\
&=
z
\int_0^\infty
t^{z-1}e^{-t}\,dt
=
z \Gamma(z).
\end{align*}

Für $0<z<\varepsilon$ für eine $\varepsilon >0$ folgt aus der 
Funktionalgleichung
\[
\Gamma(z) = \frac{\Gamma(1+z)}{z}.
\]
Da $\Gamma(1)=1$ ist und $\Gamma$ eine in einer
Umgebung von $1$ stetige Funktion ist, kann sie in der Form
\(
\Gamma(1+z)=\Gamma(1) + zf(z)
\)
schreiben, wobei  $f(z)$ eine differenzierbare Funktion ist mit
$f'(1)=\Gamma'(1)$.
Daraus ergibt sich für $\Gamma(z)$ der Ausdruck
\[
\Gamma(z) = \frac{\Gamma(1)}{z} + f(z) = \frac{1}{z} + f(z).
\]
Die Gamma-Funktion hat daher and er Stelle $z=0$ einen Pol erster Ordnung.

\subsubsection{Ausdehnung auf $\operatorname{Re}z<0$}
Die Integralformel konvergiert nicht für $\operatorname{Re}z\le 0$.
Durch analytische Fortsetzung, wie sie im
Abschnitt~\ref{buch:funktionentheorie:section:fortsetzung}
beschrieben wird, kann die Funktion auf ganz $\mathbb{C}$ ausgedehnt
werden, mit Ausnahme einzelner Pole.
Die Funktionalgleichung gilt natürlich für alle $z\in\mathbb{C}$,
für die $\Gamma(z)$ definiert ist.
In einer Umgebung von $z=-n$ gilt
\[
\Gamma(z)
=
\frac{\Gamma(z+1)}{z}
=
\frac{\Gamma(z+2)}{z(z+1)}
=
\frac{\Gamma(z+3)}{z(z+1)(z+2)}
=
\dots
=
\frac{\Gamma(z+n)}{z(z+1)(z+2)\cdots(z+n-1)}
\]
Keiner der Faktoren im Nenner verschwindet in der Nähe von $z=-n$, der
Zähler hat aber einen Pol erster Ordnung an dieser Stelle.
Daher hat auch der Quotient einen Pol erster Ordnung.
Abbildung~\ref{buch:rekursion:fig:gamma} zeigt die Pole bei den
nicht negativen ganzen Zahlen.






%
% linear.tex
%
% (c) 2021 Prof Dr Andreas Müller, OST Ostschweizer Fachhochschule
%
\section{Lineare Rekursionsgleichung mit konstanten Koeffizienten
\label{buch:rekursion:section:linear}}
\rhead{Lineare Rekursionsgleichungen}
Die Funktionalgleichung der Gamma-Funktion, die im
Abschnitt~\ref{buch:rekursion:section:gamma} untersucht wurde,
hat die Form einer linearen Rekursionsgleichung
\[
\Gamma(x+1) = x\Gamma(x),\qquad \Gamma(1) = 1.
\]
Gleichungen, die Werte einer Funktion für verschiedene
Argument in Beziehung setzen, heissen {\em Funktionalgleichungen}.
\index{Funktionalgleichung}%
Es war überraschend schwierig, eine Lösung für Funktionalgleichung
der Gamma-Funktion für beliebige komplexe $x$ zu finden.
In diesem Abschnitt soll daher eine Klasse von Rekursionsgleichungen
näher untersucht werden, für die einfache Lösungen möglich sind.

\subsection{Lineare Differenzengleichungen}

\subsection{Lösung mit Polynomfunktionen}







%
% hypergeometrisch.tex
%
% (c) 2021 Prof Dr Andreas Müller, OST Ostschweizer Fachhochschule
%
\section{Hypergeometrische Differentialgleichung
\label{buch:differentialgleichungen:section:hypergeometrisch}}
Die hypergeometrische Funktion $\mathstrut_2F1(a,b;c;x)$ wurde in
Abschnitt~\ref{buch:rekursion:section:hypergeometrische-funktion}
als Potenzreihe mit sehr speziellen Koeffizienten, die sich aus
Pochhammer-Symbolen.
Es stellt sich aber heraus, dass man sie auch als Lösung einer
gewöhnlichen Differentialgleichung bekommen kann, die bereits
Euler studiert hat.

\subsection{Die Eulersche hypergeometrische Differentialgleichung
\label{buch:differentialgleichung:subsection:euler-hypergeometrisch}}
Die hypergeometrische Funktion $\mathstrut_2F_1(a,b;c;x)$ ist eine
Lösung der {\em Eulerschen hypergeometrischen Differentialgleichung}
(zu unterscheiden von der Eulerschen Differentialgleichung, die sich
immer auf eine lineare Differentialgleichung mit konstanten Koeffizienten
reduzieren lässt)
\begin{equation}
x(1-x) \frac{d^2y}{dx^2} + (c-(a+b+1)x)\frac{dy}{dx} - ab y = 0
\label{buch:differentialgleichungen:hypergeo:eulerdgl}
\end{equation}
Wir prüfen dies nach, indem wir die Definition der hypergeometrischen
Funktion 
\begin{align*}
y(x)
&=
\mathstrut_2F_1(a,b;c;x)
=
\sum_{k=0}^\infty
\frac{(a)_k(b)_k}{(c)_k} \frac{x^k}{k!}
\intertext{mit den Ableitungen}
y'(x)
&=
\sum_{k=1}^\infty 
\frac{(a)_k(b)_k}{(c)_k} \frac{x^{k-1}}{(k-1)!}
\\
y''(x)
&=
\sum_{k=2}^\infty 
\frac{(a)_k(b)_k}{(c)_k} \frac{x^{k-2}}{(k-2)!}
\end{align*}
einsetzen.
Die Gleichung, die sich ergibt, ist
\begin{align*}
0
&=
x(1-x)
\sum_{k=2}^\infty
\frac{(a)_k(b)_k}{(c)_k}\frac{x^{k-2}}{(k-2)!}
+
(c-(a+b+1)x)
\sum_{k=1}^\infty
\frac{(a)_k(b)_k}{(c)_k}\frac{x^{k-1}}{(k-1)!}
-ab
\sum_{k=0}^\infty
\frac{(a)_k(b)_k}{(c)_k} \frac{x^k}{k!}
\\
&=
\sum_{k=2}^\infty
\frac{(a)_k(b)_k}{(c)_k}\frac{x^{k-1}}{(k-2)!}
-
\sum_{k=2}^\infty
\frac{(a)_k(b)_k}{(c)_k}\frac{x^k}{(k-2)!}
+
c\sum_{k=1}^\infty
\frac{(a)_k(b)_k}{(c)_k}\frac{x^{k-1}}{(k-1)!}
\\
&\qquad
-(a+b+1)
\sum_{k=1}^\infty
\frac{(a)_k(b)_k}{(c)_k}\frac{x^k}{(k-1)!}
-ab
\sum_{k=0}^\infty
\frac{(a)_k(b)_k}{(c)_k} \frac{x^k}{k!}
\\
&=
\sum_{k=1}^\infty
\frac{(a)_{k+1}(b)_{k+1}}{(c)_{k+1}}\frac{x^k}{(k-1)!}
-
\sum_{k=2}^\infty
\frac{(a)_k(b)_k}{(c)_k}\frac{x^k}{(k-2)!}
+
c\sum_{k=0}^\infty
\frac{(a)_{k+1}(b)_{k+1}}{(c)_{k+1}}\frac{x^k}{k!}
\\
&\qquad
-(a+b+1)
\sum_{k=1}^\infty
\frac{(a)_k(b)_k}{(c)_k}\frac{x^k}{(k-1)!}
-ab
\sum_{k=0}^\infty
\frac{(a)_k(b)_k}{(c)_k} \frac{x^k}{k!}.
\end{align*}
Zum konstanten Koeffizienten für $k=0$ tragen nur die dritte und letzte
Summe bei, dies sind die Terme
\[
c\frac{(a)_1(b)_1}{(c)_1}-ab\frac{(a)_0(b)_0}{(c)_0}
=
c\frac{ab}{c}-ab\frac{1\cdot 1}{1}
=
0.
\]
Für den linearen Term $k=1$ kommen je ein Term aus der ersten aund vierten
Summe hinzu, dies ergibt
\begin{align*}
&\phantom{\mathstrut=\mathstrut}
\frac{(a)_2(b)_2}{(c)_2}
+c\frac{(a)_2(b)_2}{(c)_2}
-(a+b+1)\frac{(a)_1(b)_1}{(c)_1}
-ab\frac{(a)_1(b)_1}{(c)_1}
\\
&=
\frac{a(a+1)b(b+1)}{c(c+1)}
(1+c)
-(ab+a+b+1)
\frac{ab}{c}
\\
&=
\frac{a(a+1)b(b+1)}{c}
-
(a+1)(b+1)\frac{ab}{c}
=0.
\end{align*}
Durch Koeffizientenvergleich erhalten wir für $k\ge 2$ 
\begin{align*}
0
&=
\frac{(a)_{k+1}(b)_{k+1}}{(c)_{k+1}} \frac1{(k-1)!} 
-
\frac{(a)_k(b)_k}{(c)_k} \frac1{(k-2)!} 
+
c\frac{(a)_{k+1}(b)_{k+1}}{(c)_{k+1}} \frac{1}{k!}
\\
&\qquad
-(a+b+1)\frac{(a)_k(b)_k}{(c)_k}\frac{1}{(k-1)!}
-ab \frac{(a)_k(b)_k}{(c)_k}\frac{1}{k!}
\\
&=
\frac{(a)_k(b)_k}{(c)_{k+1}}
\frac{1}{k!}
\biggl(
(a+k)(b+k)k
-(c+k)(k-1)k
+
c(a+k)(b+k)
\\
&\qquad
\qquad
\qquad
-(a+b+1)(c+k)k
-ab(c+k)
\biggr).
\intertext{Der zweite, vierte und fünfte Term können zu}
&=
\frac{(a)_k(b)_k}{(c)_{k+1}}
\frac{1}{k!}
\biggl(
(a+k)(b+k)k
+
c(a+k)(b+k)
-(ab+ak+bk+k^2)(c+k)
\biggr)
\intertext{zusammengefasst werden.
Der Faktor $(ab+ak+bk+k^2)$ kann als Produkt $(a+k)(b+k)$ faktorisiert
werden, der dann als gemeinsamer Faktor aus allen Termen ausgeklammert
werden kann:}
&=
\frac{(a)_k(b)_k}{(c)_{k+1}}
\frac{1}{k!}
\biggl(
(a+k)(b+k)k
+
c(a+k)(b+k)
-(a+k)(b+k)(c+k)
\biggr)
\\
&=
\frac{(a)_{k+1}(b)_{k+1}}{(c)_{k+1}}
\frac{1}{k!}
\biggl(
k
+
c
-(c+k)
\biggr)
=0.
\end{align*}
Damit ist gezeigt, dass $\mathstrut_2F_1(a,b;c;x)$ eine Lösung
der Differentialgleichung ist.

Die hypergeometrische Reihe kann auch direkt mit Hilfe der
Potenzreihenmethode als Lösung der Differentialgleichung gefunden 
werden.

\subsection{Lösung als verallgemeinerte Potenzreihe}
Da die hypergeometrische Reihe eine Differentialgleichung
zweiter Ordnung mit einer Singularität bei $x=0$ ist, 
kann man versuchen eine zweite, linear unabhängige Lösung mit
Hilfe der Methode der verallgemeinerten Potenzreihen zu finden.
Dazu setzt man die Lösung in der Form
\begin{align*}
y_2(x)
&=
\sum_{k=0}^\infty a_kx^{\varrho+k}
&
&\Rightarrow&
y_2'(x)
&=
\sum_{k=0}^\infty (\varrho+k)a_kx^{\varrho+k-1}
\\
&&
&&
y_2''(x)
&=
\sum_{k=0}^\infty (\varrho+k)(\varrho+k-1)a_kx^{\varrho+k-2}
\end{align*}
an, wobei $a_0\ne 0$ sein soll.
Einsetzen in die Differentialgleichung ergibt
\begin{align*}
0&=
x(1-x)y_2''(x) + (c-(a+b+1)x) y_2'(x) -aby_2(x)
\\
&=
x(1-x)
\sum_{k=0}^\infty (\varrho+k)(\varrho+k-1)a_kx^{\varrho+k-2}
+
(c-(a+b+1)x)
\sum_{k=0}^\infty (\varrho+k)a_kx^{\varrho+k-1}
-
abx^{\varrho}\sum_{k=0}^\infty a_kx^{\varrho+k}
\\
&=
-\sum_{k=0}^\infty (\varrho+k)(\varrho+k-1)a_kx^{\varrho+k}
+
\sum_{k=0}^\infty (\varrho+k)(\varrho+k-1)a_kx^{\varrho+k-1}
+
c
\sum_{k=0}^\infty (\varrho+k)a_kx^{\varrho+k-1}
\\
&\qquad
-
(a+b+1)
\sum_{k=0}^\infty (\varrho+k)a_kx^{\varrho+k}
-
ab
\sum_{k=0}^\infty a_kx^{\varrho+k}.
\intertext{Durch Verschiebung des Summationsindex in der zweiten
und dritten Summe wird der Koeffizientenvergleich etwas
einfacher}
&=
-\sum_{k=0}^\infty (\varrho+k)(\varrho+k-1)a_kx^{\varrho+k}
+
\sum_{k=-1}^\infty (\varrho+k+1)(\varrho+k)a_{k+1}x^{\varrho+k}
+
c
\sum_{k=-1}^\infty (\varrho+k+1)a_{k+1}x^{\varrho+k}
\\
&\qquad
-
(a+b+1)
\sum_{k=0}^\infty (\varrho+k)a_kx^{\varrho+k}
-
ab
\sum_{k=0}^\infty a_kx^{\varrho+k}
\\
&=
-\sum_{k=0}^\infty (\varrho+k)(\varrho+k-1)a_kx^{\varrho+k}
+
\sum_{k=-1}^\infty (\varrho+k+1)(\varrho+k+c)a_{k+1}x^{\varrho+k}
\\
&\qquad
-
\sum_{k=0}^\infty ((\varrho+k)(a+b+1)+ab)a_kx^{\varrho+k}
\\
&=
\bigl(
\varrho(\varrho-1)
+c\varrho \bigr)
x^{\varrho-1}
+
\sum_{k=0}^\infty
\bigl(
-(\varrho+k)(\varrho+k-1)a_k
+(\varrho+k+1)(\varrho+k+c)a_{k+1}
\\
&
\qquad
\qquad
\qquad
\qquad
\qquad
\qquad
-((\varrho+k)(a+b+1)+ab)a_k
\bigr)
x^{\varrho+k}.
\end{align*}
Aus dem ersten Term kann man die Indexgleichung
\[
0
=
\varrho(\varrho-1)+c\varrho
=
\varrho(\varrho-1+c)
\]
ablesen, die die Nullstellen $\varrho=0$ und $\varrho=1-c$ hat.
Die Nullstelle $\varrho=0$ ergibt natürlich die bereits gefundene
hypergeometrische Reihe.

Nach Einsetzen der zweiten Lösung der Indexgleichung in der Summe
legt der Koeffizientenvergleich eine Beziehung
\begin{align}
0
&=
\bigl(
-(k-c+1)(k-c)
-(k-c+1)(a+b+1)+ab
\bigr)a_k
+
(k-c+2)(k+1)
a_{k+1} 
\notag
\intertext{zwischen $a_k$ und $a_{k+1}$ fest.
Daraus kann man den Quotienten aufeinanderfolgender
Koeffizienten als}
\frac{a_{k+1}}{a_k}
&=
\frac{
-(k-c+1)(k-c)
-(k-c+1)(a+b+1)+ab
}{
\notag
(k-c+2)(k+1)
}
\\
&=
%(%i4) factor(coeff(y,q,0))
%(%o4)                  - (k - c + a + 1) (k - c + b + 1)
%(%i5) factor(coeff(y,q,1))
%(%o5)                         (k + 1) (k - c + 2)
\frac{
(a-c+1+k)
(b-c+1+k)
}{
(2-c+k)(k+1)
}
\label{buch:differentialgleichungen:hypergeo:verallgkoef}
\end{align}
finden.
Setzt man $a_0=1$, ist die zweite Lösung ist also wieder eine
hypergeometrische Funktion.%, nämlich
%\[
%y_2(x)
%=
%x^{1-c}
%\sum_{k=0}^\infty \frac{(a-c+1)_k(b-c+1)_k}{(2-c)_k}\frac{x^k}{k!}
%=
%x^{1-c}
%\mathstrut_2F_1\biggl(\begin{matrix}a-c+1,b-c+1\\2-c\end{matrix};x\biggr)
%\]
Diese Lösung ist aber nur möglich, wenn in
\eqref{buch:differentialgleichungen:hypergeo:verallgkoef}
der Nenner nicht verschwindet, d.~h.~$2-c+k\ne 0$
oder $c \ne k+2$ für all natürlichen $k$.
$c$ darf also kein natürliche Zahl $\ge 2$ sein.
Wir fassen die Resultate dieses Abschnitts im folgenden Satz zusammen.

\begin{satz}
Die eulersche hypergeometrische Differentialgleichung
\begin{equation}
x(1-x)\frac{d^2y}{dx^2}
+(c+(a+b+1)x)\frac{dy}{dx}
-ab y
=
0
\end{equation}
hat die Lösung
\[
y_1(x)
=
\mathstrut_2F_1\biggl(\begin{matrix}a,b\\c\end{matrix};x\biggr).
\]
Falls $c-2\not\in \mathbb{N}$ ist, ist
\[
y_2(x)
=
x^{1-c} \mathstrut_2F_1\biggl(\begin{matrix}a-c+1,b-c+1\\2-c\end{matrix};x\biggr)
\]
eine zweite, linear unabhängige Lösung.
\end{satz}

%
% Die verallgemeinerte hypergeometrische Differentialgleichung
%
\subsection{Verallgemeinerte hypergeometrische Differentialgleichung}
% https://de.wikipedia.org/wiki/Verallgemeinerte_hypergeometrische_Funktion







\section*{Übungsaufgaben}
\rhead{Übungsaufgaben}
\aufgabetoplevel{chapters/040-rekursion/uebungsaufgaben}
\begin{uebungsaufgaben}
%\uebungsaufgabe{0}
\uebungsaufgabe{1}
\uebungsaufgabe{2}
\end{uebungsaufgaben}



%\begin{appendices}
%\end{appendices}
\vfill
\pagebreak
\ifodd\value{page}\else\null\clearpage\fi
\lhead{Literatur}
\rhead{}
\printbibliography[heading=subbibliography]
\label{buch:literatur}
\end{refsection}


