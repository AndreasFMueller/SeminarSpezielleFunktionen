%
% hypergeometrisch.tex
%
% (c) 2021 Prof Dr Andreas Müller, OST Ostschweizer Fachhochschule
%
\section{Hypergeometrische Differentialgleichung
\label{buch:differentialgleichungen:section:hypergeometrisch}}
\rhead{Hypergeometrische Differentialgleichung}
Die hypergeometrische Funktion $\mathstrut_2F1(a,b;c;x)$ wurde in
Abschnitt~\ref{buch:rekursion:section:hypergeometrische-funktion}
als Potenzreihe mit sehr speziellen Koeffizienten, die sich aus
Pochhammer-Symbolen.
Es stellt sich aber heraus, dass man sie auch als Lösung einer
gewöhnlichen Differentialgleichung bekommen kann, die bereits
Euler studiert hat.

\subsection{Die Eulersche hypergeometrische Differentialgleichung
\label{buch:differentialgleichung:subsection:euler-hypergeometrisch}}
Die hypergeometrische Funktion $\mathstrut_2F_1(a,b;c;x)$ ist eine
Lösung der {\em Eulerschen hypergeometrischen Differentialgleichung}
(zu unterscheiden von der Eulerschen Differentialgleichung, die sich
immer auf eine lineare Differentialgleichung mit konstanten Koeffizienten
reduzieren lässt)
\begin{equation}
x(1-x) \frac{d^2y}{dx^2} + (c-(a+b+1)x)\frac{dy}{dx} - ab y = 0
\label{buch:differentialgleichungen:hypergeo:eulerdgl}
\end{equation}
Wir prüfen dies nach, indem wir die Definition der hypergeometrischen
Funktion 
\begin{align*}
y(x)
&=
\mathstrut_2F_1(a,b;c;x)
=
\sum_{k=0}^\infty
\frac{(a)_k(b)_k}{(c)_k} \frac{x^k}{k!}
\intertext{mit den Ableitungen}
y'(x)
&=
\sum_{k=1}^\infty 
\frac{(a)_k(b)_k}{(c)_k} \frac{x^{k-1}}{(k-1)!}
\\
y''(x)
&=
\sum_{k=2}^\infty 
\frac{(a)_k(b)_k}{(c)_k} \frac{x^{k-2}}{(k-2)!}
\end{align*}
einsetzen.
Die Gleichung, die sich ergibt, ist
\begin{align*}
0
&=
x(1-x)
\sum_{k=2}^\infty
\frac{(a)_k(b)_k}{(c)_k}\frac{x^{k-2}}{(k-2)!}
+
(c-(a+b+1)x)
\sum_{k=1}^\infty
\frac{(a)_k(b)_k}{(c)_k}\frac{x^{k-1}}{(k-1)!}
-ab
\sum_{k=0}^\infty
\frac{(a)_k(b)_k}{(c)_k} \frac{x^k}{k!}
\\
&=
\sum_{k=2}^\infty
\frac{(a)_k(b)_k}{(c)_k}\frac{x^{k-1}}{(k-2)!}
-
\sum_{k=2}^\infty
\frac{(a)_k(b)_k}{(c)_k}\frac{x^k}{(k-2)!}
+
c\sum_{k=1}^\infty
\frac{(a)_k(b)_k}{(c)_k}\frac{x^{k-1}}{(k-1)!}
\\
&\qquad
-(a+b+1)
\sum_{k=1}^\infty
\frac{(a)_k(b)_k}{(c)_k}\frac{x^k}{(k-1)!}
-ab
\sum_{k=0}^\infty
\frac{(a)_k(b)_k}{(c)_k} \frac{x^k}{k!}
\\
&=
\sum_{k=1}^\infty
\frac{(a)_{k+1}(b)_{k+1}}{(c)_{k+1}}\frac{x^k}{(k-1)!}
-
\sum_{k=2}^\infty
\frac{(a)_k(b)_k}{(c)_k}\frac{x^k}{(k-2)!}
+
c\sum_{k=0}^\infty
\frac{(a)_{k+1}(b)_{k+1}}{(c)_{k+1}}\frac{x^k}{k!}
\\
&\qquad
-(a+b+1)
\sum_{k=1}^\infty
\frac{(a)_k(b)_k}{(c)_k}\frac{x^k}{(k-1)!}
-ab
\sum_{k=0}^\infty
\frac{(a)_k(b)_k}{(c)_k} \frac{x^k}{k!}.
\end{align*}
Zum konstanten Koeffizienten für $k=0$ tragen nur die dritte und letzte
Summe bei, dies sind die Terme
\[
c\frac{(a)_1(b)_1}{(c)_1}-ab\frac{(a)_0(b)_0}{(c)_0}
=
c\frac{ab}{c}-ab\frac{1\cdot 1}{1}
=
0.
\]
Für den linearen Term $k=1$ kommen je ein Term aus der ersten aund vierten
Summe hinzu, dies ergibt
\begin{align*}
&\phantom{\mathstrut=\mathstrut}
\frac{(a)_2(b)_2}{(c)_2}
+c\frac{(a)_2(b)_2}{(c)_2}
-(a+b+1)\frac{(a)_1(b)_1}{(c)_1}
-ab\frac{(a)_1(b)_1}{(c)_1}
\\
&=
\frac{a(a+1)b(b+1)}{c(c+1)}
(1+c)
-(ab+a+b+1)
\frac{ab}{c}
\\
&=
\frac{a(a+1)b(b+1)}{c}
-
(a+1)(b+1)\frac{ab}{c}
=0.
\end{align*}
Durch Koeffizientenvergleich erhalten wir für $k\ge 2$ 
\begin{align*}
0
&=
\frac{(a)_{k+1}(b)_{k+1}}{(c)_{k+1}} \frac1{(k-1)!} 
-
\frac{(a)_k(b)_k}{(c)_k} \frac1{(k-2)!} 
+
c\frac{(a)_{k+1}(b)_{k+1}}{(c)_{k+1}} \frac{1}{k!}
\\
&\qquad
-(a+b+1)\frac{(a)_k(b)_k}{(c)_k}\frac{1}{(k-1)!}
-ab \frac{(a)_k(b)_k}{(c)_k}\frac{1}{k!}
\\
&=
\frac{(a)_k(b)_k}{(c)_{k+1}}
\frac{1}{k!}
\biggl(
(a+k)(b+k)k
-(c+k)(k-1)k
+
c(a+k)(b+k)
\\
&\qquad
\qquad
\qquad
-(a+b+1)(c+k)k
-ab(c+k)
\biggr).
\intertext{Der zweite, vierte und fünfte Term können zu}
&=
\frac{(a)_k(b)_k}{(c)_{k+1}}
\frac{1}{k!}
\biggl(
(a+k)(b+k)k
+
c(a+k)(b+k)
-(ab+ak+bk+k^2)(c+k)
\biggr)
\intertext{zusammengefasst werden.
Der Faktor $(ab+ak+bk+k^2)$ kann als Produkt $(a+k)(b+k)$ faktorisiert
werden, der dann als gemeinsamer Faktor aus allen Termen ausgeklammert
werden kann:}
&=
\frac{(a)_k(b)_k}{(c)_{k+1}}
\frac{1}{k!}
\biggl(
(a+k)(b+k)k
+
c(a+k)(b+k)
-(a+k)(b+k)(c+k)
\biggr)
\\
&=
\frac{(a)_{k+1}(b)_{k+1}}{(c)_{k+1}}
\frac{1}{k!}
\biggl(
k
+
c
-(c+k)
\biggr)
=0.
\end{align*}
Damit ist gezeigt, dass $\mathstrut_2F_1(a,b;c;x)$ eine Lösung
der Differentialgleichung ist.

Die hypergeometrische Reihe kann auch direkt mit Hilfe der
Potenzreihenmethode als Lösung der Differentialgleichung gefunden 
werden.

\subsection{Lösung als verallgemeinerte Potenzreihe}
Da die hypergeometrische Reihe eine Differentialgleichung
zweiter Ordnung mit einer Singularität bei $x=0$ ist, 
kann man versuchen eine zweite, linear unabhängige Lösung mit
Hilfe der Methode der verallgemeinerten Potenzreihen zu finden.
Dazu setzt man die Lösung in der Form
\begin{align*}
y_2(x)
&=
\sum_{k=0}^\infty a_kx^{\varrho+k}
&
&\Rightarrow&
y_2'(x)
&=
\sum_{k=0}^\infty (\varrho+k)a_kx^{\varrho+k-1}
\\
&&
&&
y_2''(x)
&=
\sum_{k=0}^\infty (\varrho+k)(\varrho+k-1)a_kx^{\varrho+k-2}
\end{align*}
an, wobei $a_0\ne 0$ sein soll.
Einsetzen in die Differentialgleichung ergibt
\begin{align*}
0&=
x(1-x)y_2''(x) + (c-(a+b+1)x) y_2'(x) -aby_2(x)
\\
&=
x(1-x)
\sum_{k=0}^\infty (\varrho+k)(\varrho+k-1)a_kx^{\varrho+k-2}
+
(c-(a+b+1)x)
\sum_{k=0}^\infty (\varrho+k)a_kx^{\varrho+k-1}
-
abx^{\varrho}\sum_{k=0}^\infty a_kx^{\varrho+k}
\\
&=
-\sum_{k=0}^\infty (\varrho+k)(\varrho+k-1)a_kx^{\varrho+k}
+
\sum_{k=0}^\infty (\varrho+k)(\varrho+k-1)a_kx^{\varrho+k-1}
+
c
\sum_{k=0}^\infty (\varrho+k)a_kx^{\varrho+k-1}
\\
&\qquad
-
(a+b+1)
\sum_{k=0}^\infty (\varrho+k)a_kx^{\varrho+k}
-
ab
\sum_{k=0}^\infty a_kx^{\varrho+k}.
\intertext{Durch Verschiebung des Summationsindex in der zweiten
und dritten Summe wird der Koeffizientenvergleich etwas
einfacher}
&=
-\sum_{k=0}^\infty (\varrho+k)(\varrho+k-1)a_kx^{\varrho+k}
+
\sum_{k=-1}^\infty (\varrho+k+1)(\varrho+k)a_{k+1}x^{\varrho+k}
+
c
\sum_{k=-1}^\infty (\varrho+k+1)a_{k+1}x^{\varrho+k}
\\
&\qquad
-
(a+b+1)
\sum_{k=0}^\infty (\varrho+k)a_kx^{\varrho+k}
-
ab
\sum_{k=0}^\infty a_kx^{\varrho+k}
\\
&=
-\sum_{k=0}^\infty (\varrho+k)(\varrho+k-1)a_kx^{\varrho+k}
+
\sum_{k=-1}^\infty (\varrho+k+1)(\varrho+k+c)a_{k+1}x^{\varrho+k}
\\
&\qquad
-
\sum_{k=0}^\infty ((\varrho+k)(a+b+1)+ab)a_kx^{\varrho+k}
\\
&=
\bigl(
\varrho(\varrho-1)
+c\varrho \bigr)
x^{\varrho-1}
+
\sum_{k=0}^\infty
\bigl(
-(\varrho+k)(\varrho+k-1)a_k
+(\varrho+k+1)(\varrho+k+c)a_{k+1}
\\
&
\qquad
\qquad
\qquad
\qquad
\qquad
\qquad
-((\varrho+k)(a+b+1)+ab)a_k
\bigr)
x^{\varrho+k}.
\end{align*}
Aus dem ersten Term kann man die Indexgleichung
\[
0
=
\varrho(\varrho-1)+c\varrho
=
\varrho(\varrho-1+c)
\]
ablesen, die die Nullstellen $\varrho=0$ und $\varrho=1-c$ hat.
Die Nullstelle $\varrho=0$ ergibt natürlich die bereits gefundene
hypergeometrische Reihe.

Nach Einsetzen der zweiten Lösung der Indexgleichung in der Summe
legt der Koeffizientenvergleich eine Beziehung
\begin{align}
0
&=
\bigl(
-(k-c+1)(k-c)
-(k-c+1)(a+b+1)+ab
\bigr)a_k
+
(k-c+2)(k+1)
a_{k+1} 
\notag
\intertext{zwischen $a_k$ und $a_{k+1}$ fest.
Daraus kann man den Quotienten aufeinanderfolgender
Koeffizienten als}
\frac{a_{k+1}}{a_k}
&=
\frac{
-(k-c+1)(k-c)
-(k-c+1)(a+b+1)+ab
}{
\notag
(k-c+2)(k+1)
}
\\
&=
%(%i4) factor(coeff(y,q,0))
%(%o4)                  - (k - c + a + 1) (k - c + b + 1)
%(%i5) factor(coeff(y,q,1))
%(%o5)                         (k + 1) (k - c + 2)
\frac{
(a-c+1+k)
(b-c+1+k)
}{
(2-c+k)(k+1)
}
\label{buch:differentialgleichungen:hypergeo:verallgkoef}
\end{align}
finden.
Setzt man $a_0=1$, ist die zweite Lösung ist also wieder eine
hypergeometrische Funktion.%, nämlich
%\[
%y_2(x)
%=
%x^{1-c}
%\sum_{k=0}^\infty \frac{(a-c+1)_k(b-c+1)_k}{(2-c)_k}\frac{x^k}{k!}
%=
%x^{1-c}
%\mathstrut_2F_1\biggl(\begin{matrix}a-c+1,b-c+1\\2-c\end{matrix};x\biggr)
%\]
Diese Lösung ist aber nur möglich, wenn in
\eqref{buch:differentialgleichungen:hypergeo:verallgkoef}
der Nenner nicht verschwindet, d.~h.~$2-c+k\ne 0$
oder $c \ne k+2$ für all natürlichen $k$.
$c$ darf also kein natürliche Zahl $\ge 2$ sein.
Wir fassen die Resultate dieses Abschnitts im folgenden Satz zusammen.

\begin{satz}
\index{Satz!Lösung der eulerschen hypergeometrischen Differentialgleichung}%
Die eulersche hypergeometrische Differentialgleichung
\begin{equation}
x(1-x)\frac{d^2y}{dx^2}
+(c-(a+b+1)x)\frac{dy}{dx}
-ab y
=
0
\label{buch:differentialgleichungen:eqn:eulerhyper}
\end{equation}
hat die Lösung
\[
y_1(x)
=
\mathstrut_2F_1\biggl(\begin{matrix}a,b\\c\end{matrix};x\biggr).
\]
Falls $c-2\not\in \mathbb{N}$ ist, ist
\[
y_2(x)
=
x^{1-c} \mathstrut_2F_1\biggl(\begin{matrix}a-c+1,b-c+1\\2-c\end{matrix};x\biggr)
\]
eine zweite, linear unabhängige Lösung.
\end{satz}

%
% Die verallgemeinerte hypergeometrische Differentialgleichung
%
\subsection{Verallgemeinerte hypergeometrische Differentialgleichung}
% https://de.wikipedia.org/wiki/Verallgemeinerte_hypergeometrische_Funktion
Die Ableitungsformel für die hypergeometrischen Funktionen
\[
w(z)
=
\mathstrut_nF_m
\biggl(\begin{matrix}a_1,\dots,a_m\\b_1,\dots,b_n\end{matrix};z\biggr)
\]
drückt die Ableitung $f'(z)$ durch einen Wert einer hypergeometrischen
Funktion mit ganz anderen Parametern aus, nämlich
\[
w'(z)
=
\frac{a_1\cdot\ldots\cdot a_n}{b_1\cdot\ldots\cdot b_m}
\mathstrut_mF_n\biggl(
\begin{matrix}a_1+1,\dots,a_n+1\\b_1+1,\dots,b_m+1\end{matrix};z
\biggr).
\]
Dies erlaubt aber noch nicht, eine Differentialgleichung für $w(z)$
aufzustellen, da auf der rechten Seite alle Parameter inkrementiert
worden sind.
Um eine Differentialgleichung zu erhalten, muss man den gleichen
Effekt auf einem anderen Weg erreichen.

\subsubsection{Operatoren, die genau ein $a_i$ inkrementieren}
Wir suchen einen Operator, der in der hypergeometrischen Funktion
$\mathstrut_nF_m$ nur genau den Parameter $a_i$ inkrementiert.
Der folgende Operator schafft dies:
\begin{align*}
\biggl(z\frac{d}{dz}+a_i\biggr)
\mathstrut_nF_m\biggl(\begin{matrix}a_1,\dots,a_n\\b_1,\dots,b_m\end{matrix};
z\biggr)
&=
\biggl(z\frac{d}{dz}+a_i\biggr)
\sum_{k=0}^\infty
\frac{(a_1)_k\cdot\ldots\cdot(a_n)_k}{(b_1)_k\cdot\ldots\cdot(b_m)_k}
\frac{z^k}{k!}
\\
&=
\sum_{k=0}^\infty
\frac{(a_1)_k\cdot\ldots\cdot(a_n)_k}{(b_1)_k\cdot\ldots\cdot(b_m)_k}
\frac{z^k}{(k-1)!}
+
\sum_{k=0}^\infty
a_i
\frac{(a_1)_k\cdot\ldots\cdot(a_n)_k}{(b_1)_k\cdot\ldots\cdot(b_m)_k}
\frac{z^k}{k!}
\\
&=
\sum_{k=0}^\infty
\frac{
(a_1)_k\cdots\widehat{(a_i)_k}\cdots(a_n)_k
}{
(b_1)_k\cdots(b_m)_k
}
(
k(a_i)_k
+
a_i(a_i)_k
)
\frac{z^k}{k!}
\\
&=
\sum_{k=0}^\infty
\frac{
(a_1)_k\cdots\widehat{(a_i)_k}\cdots(a_n)_k
}{
(b_1)_k\cdots(b_m)_k
}
\underbrace{(a_i)_k(a_i+k)}_{a_i(a_i+1)_k}
\frac{z^k}{k!}
\\
&=
a_i
\sum_{k=0}^\infty
\frac{
(a_1)_k\cdots (a_i+1)_k\cdots(a_n)_k
}{
(b_1)_k\cdots(b_m)_k
}
\underbrace{(a_i)_k(a_i+k)}_{a_i(a_i+1)_k}
\frac{z^k}{k!}
\\
&=
a_i\cdot\mathstrut_nF_m\biggl(
\begin{matrix}
a_1,\dots,a_i+1,\dots,a_n\\
b_1,\dots,b_m
\end{matrix}
;z
\biggr).
\end{align*}
Durch Anwendung aller Operatoren
\[
D_{a_i} = z\frac{d}{dz}+a_i
\]
kann man jetzt die Inkrementierung der $a_i$, die in der Ableitung
von $w(z)$ zu beobachten war, in Einzelschritten erreichen:
\[
D_{a_1}D_{a_2}\cdots D_{a_n} w(z)
=
a_1a_2\cdots a_n \,
\mathstrut_nF_m\biggl(
\begin{matrix}a_1+1,\dots,a_n+1\\b_1,\dots,b_m\end{matrix}; z
\biggr).
\]

\subsubsection{Operatoren, die genau ein $b_j$ dekrementieren}
Die Rechnung für die Operatoren $D_{a_i}$ ist nicht direkt auf die
$b_i$ übertragbar, wir versuchen daher erneut:
\begin{align*}
D_{b_i-1}
\,\mathstrut_nF_m
\biggl(
\begin{matrix}a_1,\dots,a_n\\b_1,\dots,a_m\end{matrix};z
\biggr)
&=
\biggl(z\frac{d}{dz}+b_j-1\biggr)
\sum_{k=0}^\infty
\frac{(a_1)_k\cdots (a_n)_k}{(b_1)_k\cdots (b_m)_k}
\frac{z^k}{k!}
\\
&=
\sum_{k=0}^\infty
\frac{(a_1)_k\cdots (a_n)_k}{(b_1)_k\cdots (b_m)_k}
\frac{z^k}{(k-1)!}
+
(b_j-1)
\sum_{k=0}^\infty
\frac{(a_1)_k\cdots (a_n)_k}{(b_1)_k\cdots (b_m)_k}
\frac{z^k}{k!}
\\
&=
\sum_{k=0}^\infty
\frac{(a_1)_k\cdots (a_n)_k}{(b_1)_k\cdots\widehat{(b_j)_k}\cdots (b_m)_k}
\biggl(\frac{k}{(b_j)_k}+\frac{b_j-1}{(b_j)_k}\biggr)
\frac{z^k}{k!}
\\
&=
\sum_{k=0}^\infty
\frac{(a_1)_k\cdots (a_n)_k}{(b_1)_k\cdots \widehat{(b_j)_k}\cdots (b_m)_k}
\frac{b_j+k-1}{(b_j)_k}
\frac{z^k}{k!}
\\
&=
\sum_{k=0}^\infty
\frac{(a_1)_k\cdots (a_n)_k}{(b_1)_k\cdots \widehat{(b_j)_k}\cdots (b_m)_k}
\frac{b_j-1}{(b_j-1)_k}
\frac{z^k}{k!}
\\
&=
(b_j-1)
\sum_{k=0}^\infty
\frac{(a_1)_k\cdots (a_n)_k}{(b_1)_k\cdots(b_j-1)_k\cdots (b_m)_k}
\frac{z^k}{k!}
\\
&=(b_j-1)
\,
\mathstrut_nF_m\biggl(
\begin{matrix}a_1,\dots,a_n\\
b_1,\dots,b_j-1,\dots,b_m
\end{matrix}
;z
\biggr).
\end{align*}
Durch Anwendung aller Operatoren $D_{b_j-1}$ kann also jeder $b$-Parameter
dekrementiert werden, es gilt also
\[
D_{b_1-1}D_{b_2-1}\cdots D_{b_m-1}w
=
(b_1-1)(b_2-1)\cdots(b_m-1) \,
\mathstrut_nF_m\biggl(
\begin{matrix}
a_1,\dots,a_n\\
b_1-1,\dots,b_m-1
\end{matrix}
;z
\biggr).
\]

\subsubsection{Die Differentialgleichung}
Aus den Operatoren $D_{a_i}$ und $D_{b_j-1}$ kann jetzt eine
Differentialgleichung für die Funktion $w(z)$ konstruieren.
Durch Anwendung von aller Operatoren $D_{b_i-1}$ werden die
$b$-Parameter dekrementiert und die Faktoren $(b_i-1)$ kommen hinzu.
Leitet man dies ab, werden alle Parameter inkrementiert:
\begin{align*}
\frac{d}{dz}
\mathstrut_nF_m\biggl(
\begin{matrix}a_1,\dots,a_n\\
b_1,\dots,b_m\end{matrix};z
\biggr)
&=
\frac{a_1\cdots a_n}{(b_1-1)\cdots(b_m-1)}
(b_1-1)\cdots(b_m-1)
\,\mathstrut_nF_m\biggl(
\begin{matrix}a_1+1,\dots,a_n+1\\
b_1,\dots,b_m\end{matrix};z
\biggr)
\\
&=
a_1\dots a_n
\,\mathstrut_nF_m\biggl(
\begin{matrix}a_1+1,\dots,a_n+1\\
b_1,\dots,b_m\end{matrix};z
\biggr)
\end{align*}
Dies ist aber die gleiche Operation, wie alle Operatoren $D_{a_i}$ 
anzuwenden.
Es folgt daher die Differentialgleichung 
\[
D_{a_1}\cdots D_{a_n} w = \frac{d}{dz} D_{b_1-1}\cdots D_{b_m-1} w
\]
für die Funktion $w(z)$.

\begin{beispiel}
Im Spezialfall $\mathstrut_0F_0$ gibt es keine Operatoren $D_{a_i}$
oder $D_{b_j-1}$ anzuwenden, so dass nur die Differentialgleichung
\[
w=\frac{d}{dz}w
\]
stehen bleibt.
Dies ist natürlich die Differentialgleichung der Exponentialfunktion.
\end{beispiel}

%
% Differentialgleichung für 1F0
%
\subsubsection{Die Differentialgleichungen für $\mathstrut_1F_0$}
In diesen Fälle gibt es nur jeweils einen einzigen Operator
anzuwenden.
Wir betrachten zunächst den Fall $w(z) = \mathstrut_1F_0(a; z)$
und finden direkt die Differentialgleichung
\begin{align*}
\biggl(z\frac{d}{dz}+a\biggr)w
&=
\frac{d}{dz}w
\\
zw'+a w
&=
w'
\\
(1-z)w'
&=
a w.
\end{align*}

\begin{beispiel}
Wir bestimmen die Differentialgleichung für die als hypergeometrische
Reihe darstellbare Funktion
\[
f(x)
=
\sqrt{1+x} = \mathstrut_1F_0(-{\textstyle\frac12};-x).
\]
Zunächst erfüllt die hypergeometrische Funktion
$w(z)=\mathstrut_1F_0(-\frac12;z)$ die Differentialgleichung
\[
(1-z)w'(z) = -\frac12 w(z).
\]
Jetzt setzen wir $z=-x$ in die Funktion ein.
Wegen $f(x)=w(-x)$ folgt $f'(x)=-w'(-x)$
\[
-f'(x)(1+x) = -\frac12 f(x)
\qquad\Rightarrow\qquad
f'(x) = \frac{f(x)}{2(1+x)}.
\]
Tatsächlich ist die Ableitung der Wurzelfunktion $f(x)$
\[
\frac{d}{dx}f(x)
=
\frac{d}{dx}\sqrt{1+x}
=
\frac{1}{2\sqrt{1+x}}
=
\frac{\sqrt{1+x}}{2(1+x)}
=
\frac{f(x)}{2(1+x)},
\]
sie erfüllt also die genannte Differentialgleichung.
\end{beispiel}

%
% Differentialgleichung für 0F1
%
\subsubsection{Die Differentialgleichungen für $\mathstrut_0F_1$}
Für die Funktion $\mathstrut_0F_1$ setzen wir
$w(z)=\mathstrut_0F_1(;b;z)$.
In diesem Fall gibt es keine Operatoren $D_{a_i}$ anzuwenden, die
linke Seite der Differentialgleichung ist also einfach die Funktion $w$.
Für die rechte Seite ist der Operator $D_{b-1}$ anzuwenden, was auf
die Differentialgleichung
\begin{align}
w
&=
\frac{d}{dz}
\biggl(z\frac{d}{dz}+b -1\biggr)w
\notag
\\
w
&=
\frac{d}{dz}(zw'+b w - w)
\notag
\\
w
&=
zw''+w'+b w' -w'
\notag
\\
0
&=
zw''+b w' - w
\label{buch:differentialgleichungen:0f1:dgl}
\end{align}
führt.

\begin{beispiel}
Die Kosinus-Funktion kann durch die hypergeometrische Funktion
$\mathstrut_0F_1$ ausgedrückt werden.
Wir schreiben 
\[
w(z)
=
\mathstrut_0F_1\biggl(
\begin{matrix}\text{---}\\\frac12\end{matrix}
;z\biggr),
\]
$w(z)$ erfüllt die Differentialgleichung
\[
zw''(z) +w'(z) -\frac{3}{2} w(z) = 0.
\]
Die Kosinus-Funktion als Funktion von $w(z)$ ist
\[
f(x)
=
\cos x = \mathstrut_0F_1\biggl(;\frac12;-\frac{x^2}4\biggr)
=
w\biggl(-\frac{x^2}4\biggr),
\]
es muss also $z=-x^2/4$ gesetzt werden.
Wir müssen die Ableitungen von $w$ durch die Ableitungen von $f$
ausdrücken.
Die Ableitungen sind
\begin{align*}
f'(x)
&=
-\frac{x}{2}
w'\biggl(-\frac{x^2}4\biggr)
&&\Rightarrow&
w'\biggl(-\frac{x^2}4\biggr)
&=
-\frac{2}{x}f'(x)
\\
f''(x)
&=
\frac{x^2}{4}w''\biggl(-\frac{x^2}4\biggr)
-\frac12w'\biggl(-\frac{x^2}4\biggr)
&&\Rightarrow&
w''\biggl(-\frac{x^2}4\biggr)
&=
\frac{4}{x^2}f''(x)
+\frac{2}{x^2}w'\biggl(-\frac{x^2}4\biggr)
\\
&&&&
&=
\frac{4}{x^2}f''(x)
-\frac{4}{x^3}f'(x).
\end{align*}
Einsetzen in die Differentialgleichung von $w(z)$ ergibt
\begin{align*}
0=
zw''+\beta w'-w
&=
-\frac{x^2}4
\biggl(
\frac{4}{x^2}f''(x)-\frac{4}{x^3}f'(x)
\biggr)
+\frac12\biggl(
-\frac2xf'(x)
\biggr)
-f(x)
\\
&=
-f''(x)
-f(x),
\end{align*}
was gleichbedeutend ist mit der Differentialgleichung $f''=-f$, die
tatsächlich die Kosinus-Funktion als Lösung hat.
\end{beispiel}

%
% Die Differentialgleichung für 1F1
%
\subsubsection{Die Differentialgleichung für $\mathstrut_1F_1$}
Wir setzen wieder $w(z) = \mathstrut_1F_1(a;b;z)$.
Es sind die Operatoren $D_a$ und $D_{b-1}$ anzuwenden.
Es ergibt sich die Differentialgleichung
\begin{align}
\biggl(z\frac{d}{dz}+a\biggr)w
&=
\frac{d}{dz}\biggl(z\frac{d}{dz} +b-1\biggr)w
\notag
\\
zw'+a w
&=
\frac{d}{dz}
(zw'+b w - w)
\notag
\\
zw'+a w
&=
zw'' +w'+b w' - w'
\notag
\\
0
&=
zw'' + (b - z)w' - a w.
\label{buch:differentialgleichungen:1f1}
\end{align}
Die hypergeometrische Funktion $\mathstrut_1F_1$ ist eine Lösung 
den Anfangsbedingungen $w(0)=1$, $w'(0)=a/b$.
Eine zweite, linear unabhängige Lösung der Differentialgleichung
\eqref{buch:differentialgleichungen:1f1} kann als verallgemeinerte
Potenzreihe $w(z) = z^\varrho v(z)$ gefunden werden.
Die Ableitungen dieses Ansatzes sind
\begin{align*}
w'(z) 
&=
\varrho z^{\varrho-1} v(z) + z^\varrho v'(z)
\\
w''(z)
&=
\varrho(\varrho-1) z^{\varrho-2} v(z)
+
2\varrho z^{\varrho-1} v'(z)
+
z^\varrho v''(z).
\end{align*}
Einsetzen derselben in~\eqref{buch:differentialgleichungen:1f1}
ergibt die Gleichung
\begin{align*}
z\bigl(
\varrho(\varrho-1)z^{\varrho-2}v + 2\varrho z^{\varrho-1}v'+z^\varrho v''
\bigr)
+
(b-z)\bigl(\varrho z^{\varrho-1}v+z^\varrho v'\bigr)
-
a z^\varrho v
&=
0
\\
z^{\varrho+1} v''
+
z^\varrho
(2\varrho + b-z)
v'
+
(\varrho(\varrho-1)z^{\varrho-1}
+(b-z)
\varrho
z^{\varrho-1}
-
az^\varrho
)
v
&=
0
\\
z^{\varrho+1} v''
+
z^\varrho(2\varrho+b-z)v'
+
(\varrho(\varrho-1+b) z^{\varrho-1} v
+
(\varrho-a)z^\varrho v
&=
0
\end{align*}
Die letzte Gleichung wird wieder zu einer Differentialgleichung
der Form~\eqref{buch:differentialgleichungen:1f1}, wenn der erste
der Koeffizienten von $v$ verschwindet, wenn also 
$\varrho-1+b=0$ ist, oder $\varrho=1-b$.
Setzt man diesen Wert ein, entsteht die Differentialgleichung
\[
zv'' + (2(1-b)+b-z) v' - (a+b-1)v = 0
\qquad\Rightarrow\qquad
zv'' + (2-b-z) v' - (a+b-1)v = 0.
\]
Dies ist eine hypergeometrische Differentialgleichung für
$\mathstrut_1F_1$ mit den Parametern $2-b$ und $1-b-a$.
Es folgt, dass 
\[
w_2(z)
=
x^{1-b} \mathstrut_1F_1\biggl(
\begin{matrix}
a+b-1\\
2-b
\end{matrix}
;z
\biggr).
\]
Falls $2-b$ keine negative ganze Zahl ist, ist die hypergeometrische
Funktion wohldefiniert.

Wir fassen diese Resultat zusammen:
\begin{satz}
\index{Satz!1f1@Differentialgleichung von $\mathstrut_1F_1$}%
\label{buch:differentialgleichungen:satz:1f1-dgl-loesungen}
Die Differentialgleichung
\[
zw'' + (b-z)w' - aw = 0
\]
hat die Funktion
\[
w_1(z)
=
\mathstrut_1F_1\biggl(
\begin{matrix}a\\b\end{matrix};z
\biggr)
\]
als Lösung.
Falls $b-2\not\in\mathbb{N}$ ist, ist 
\[
w_2(z)
=
z^{1-b}
\cdot
\mathstrut_1F_1\biggl(
\begin{matrix}a+b-1\\2-b\end{matrix}
;z
\biggr)
\]
eine zweite Lösung.
Für $b=1$ ist $w_2(z)=w_1(z)$.
\end{satz}

%
% Die hypergeometrische Differentialgleichung für 2F1
%
\subsubsection{Die Differentialgleichung für $\mathstrut_2F_1$}
Für die hypergeometrische Funktion $\mathstrut_2F_1(a,b;c;z)$
ist die Difrentialgleichung von der Form
\[
\biggl(z\frac{d}{dz} + a\biggr)
\biggl(z\frac{d}{dz} + b\biggr)w
=
\frac{d}{dz}
\biggl(z\frac{d}{dz}+c -1\biggr)
w.
\]
Durchführen der Ableitungen auf beiden Seiten ergibt für die linke Seite
\begin{align*}
\biggl(z\frac{d}{dz} + a\biggr)
\biggl(z\frac{d}{dz} + b\biggr)w
&=
\biggl(z\frac{d}{dz} + a\biggr)
(zw'+b w)
\\
&=
z^2w'' + zw' + b zw' + a(zw'+b w)
\\
&=
z^2w'' + (1+a+b )zw' + ba w
\intertext{und die rechte Seite}
\frac{d}{dz}\biggl(z\frac{d}{dz}+c-1\biggr)w
&=
\frac{d}{dz}(zw'+c w-w)
\\
&=
zw''+w'+c w' - w'
\\
&= 
zw'' +c w'.
\end{align*}
Durch Gleichsetzen ergibt sich jetzt
\begin{align*}
z^2w'' + (1+a+b )zw' + ab w
&=
zw'' +c w'
\\
0
&=
z(1-z)w''
+
(c-z(1+a+b))w'
-
ab
w
\end{align*}
Dies ist die früher definierte hypergeometrische Differentialgleichung.

%
% Gerade und ungerade Funktionen
%
\subsection{Differentialgleichung für
$x^\varrho\cdot\mathstrut_pF_q(a_i;b_j;sx^\nu)$}
In verschiedenen Beispielen ist gezeigt worden, wie sich
wohlbekannte Funktionen durch hypergeometrische Funktionen
ausdrücken lassen.
Aus der Differentialgleichung der hypergeometrischen Funktionen
muss sich daher auch eine Differentialgleichung für die
gesuchten Funktionen ergeben.
Zum Beispiel lassen sich die Besselfunktionen durch hypergeometrische
Funktionen des Argumentes $-x^2/4$ schreiben, es muss also auch
möglich sein, die besselsche Differentialgleichung wieder aus
der eulerschen hypergeometrischen Differentialgleichung zu gewinnen.

\subsubsection{Gerade und ungerade}
Hypergeometrische Funktionen, deren Reihe mehr als einen Term
enthalten, enthalten immer mindestens eine gerade und eine ungerade
Potenz der unabhängigen Variable.
Sie können also grundsätzlich weder gerade noch ungerade sein.
Andererseits haben die Differentialgleichungen $y''+y=0$ oder $y''-y=0$
besonders praktische Lösungen, die sich zusätzlich durch besondere
Symmetrien auszeichnen.
Die Differentialgleichung $y''+y=0$ hat zum Beispiel die gerade
Lösung $\cos x$ und die ungerade Lösung $\sin x$.
Auch die Differentialgleichung $y''-y=0$ hat eine gerade Lösung,
$\cosh x$, und eine ungerade Lösung, $\sinh x$.

\subsubsection{Symmetrien der eulerschen hypergeometrischen
Differentialgleichung}
Hat die hypergeometrische Differentialgleichung gerade und
ungerade Lösungen?
Wenn es eine gerade Lösung $y(x)$ gibt, dann sollte die Substitution
$x \to -x$ eine neue Differentialgleichung geben, die ebenfalls $y(x)$
als Lösung hat.
\begin{align*}
 x(1-x)\frac{d^2y}{dx^2} + (c+(a+b+1)x)\frac{dy}{dx}-aby&=0
\\
-x(1+x)\frac{d^2y}{dx^2} - (c-(a+b+1)x)\frac{dy}{dx}-aby&=0
\end{align*}
Die Differenz dieser beiden Gleichungen ist
\begin{align*}
2x\frac{d^2y}{dx^2} +2c \frac{dy}{dx}&=0
&&\Rightarrow&
\frac{d}{dx} \log \frac{dy}{dx} &= -\frac{c}{x}
\\
&&&\Rightarrow&
\log \frac{dy}{dx} &= -c\log x
\\
&&&\Rightarrow&
\frac{dy}{dx} &= x^{-c}
\\
&&&\Rightarrow&
y(x) &= \frac{1}{-c+1}x^{-c+1}
\end{align*}
Dies zeigt, dass die hypergeometrische Differentialgleichung im
allgemeinen keine geraden oder ungeraden Lösungen hat.

\subsubsection{Zusammengesetzte Funktionen}
Die gerade oder ungeraden Funktionen, die in früheren Beispielen
als hypergeometrische Funktionen dargestellt wurden, konnten also
nicht Lösungen der hypergeometrische Differentialgleichung sein.
Die Potenzreihenentwicklung einer geraden Funktion enthält nur
gerade Potenzen der unabhängigen Variablen, bei einer ungeraden
Funktion sind es nur die ungeraden Potenzen.
Die einzige Möglichkeit, eine gerade Funktion $g(x)$ oder eine ungerade
Funktion $u(x)$  als eine hypergeometrische Funktion darzustellen,
ist die Verwendung eines quadratischen Arguments, also in der Form
\[
g(x)
=
\mathstrut_pF_q\biggl(\begin{matrix}a_1,\dots,a_p\\b_1,\dots,b_1\end{matrix};x^2\biggr)
\qquad\text{und}\qquad
u(x)
=
x\cdot
\mathstrut_pF_q\biggl(\begin{matrix}a_1,\dots,a_p\\b_1,\dots,b_1\end{matrix};x^2\biggr).
\]

Viele wohlbekannte Funktionen $f(x)$ können aus einer hypergeometrischen
Funktion $w(z)$ als $g(x)=w(\pm x^2)$ oder $u(x)=xw(\pm x^2)$ erhalten
werden.
Für die hypergeometrische Funktion $w(z)$ ist eine definierende
Differentialgleichung bekannt.
Im Folgenden soll daraus eine Differentialgleichung für $f(x)$
abgeleitet werden.

Der Fall $w(-x^2)$ könnte natürlich auch durch Verwendung eines
imaginären Arguments wie in $w((ix)^2)$ auf den Fall $w(x^2)$ 
zurückgeführt werden.
Um aber die Differentialgleichungen reell zu belassen, schreiben
wir $g(x)=w(sx^2)$ für eine gerade Funktion beziehungsweise
$u(x)=xw(sx^2)$ für eine ungerade Funktion.
Der Faktor $s$ kann ausserdem dazu verwendet werden, das Argument
zu skalieren, wie es zum Beispiel in der Darstellung der
Kosinus-Funktion als $\cos x = \mathstrut_0F_1(;\frac12;-\frac{x^2}4)$
nötig ist.

Um die Differentialgleichung für $g$ oder $u$ zu finden, berechnen wir die
Ableitungen von $g$ und $u$, drücken die Ableitungen von $w$ durch die
Ableitungen $g$ und $u$ aus und setzen sie in die Differentialgleichung
für $w$ ein.
Wir wollen dies im Folgenden nur für ein paar Beispiele niedrigerer
Ordnung tun.

%
% Differentialgleichungen für g(x)=w(sx^2)
%
\subsubsection{Differentialgleichungen für $g(x)=w(sx^2)$ für $w=\mathstrut_2F_1$}
Die Ableitungen von $g$ sind
\[
\begin{linsys}{3}
  g(x)&=&w(sx^2)& &           & &                \\
 g'(x)&=&       & &2sxw'(sx^2)& &                \\
g''(x)&=&       & & 2sw'(sx^2)&+&4s^2x^2w''(sx^2)\\
\end{linsys}
\]
Dies sind lineare Gleichungssysteme für die Ableitungen
von $w$, die wir nach $w'$ und $w''$ auflösen können.
Dies wird einfacher, wenn wir das Gleichungssysteme in
Matrixschreibweise darstellen mit der Matrix
\begin{equation*}
\begin{pmatrix}
  g(x)\\
 g'(x)\\
g''(x)
\end{pmatrix}
=
\underbrace{
\begin{pmatrix}
1&   0&       0\\
0& 2sx&       0\\
0&  2s& 4s^2x^2
\end{pmatrix}
}_{\displaystyle = A_g}
\begin{pmatrix}
w(sx^2)\\
w'(sx^2)\\
w''(sx^2)
\end{pmatrix}
\qquad\Rightarrow\qquad
A_g^{-1}
=
\begin{pmatrix}
1&            0       &               0  \\
0&  \frac{1}{2sx}     &               0  \\
0& -\frac{1}{4s^2x^3} & \frac{1}{4s^2x^2}
\end{pmatrix}
\end{equation*}
Damit lassen sich jetzt die Ableitungen von $w$ ausdrücken:
\begin{align*}
  w(sx^2) &= g(x) \\
 w'(sx^2) &= \frac{1}{2sx} g'(x) \\
w''(sx^2) &= \frac{1}{4s^2x^3}g'(x) + \frac{4}{4s^2x^2} g''(x) 
\end{align*}
Durch Einsetzen dieser Ausdrücke in die eulersche hypergeometrische
Differentialgleichung~\eqref{buch:differentialgleichungen:eqn:eulerhyper}
%\[
%z(1-z) \frac{d^2y}{dz^2} + (c+(a+b+1)z)\frac{dy}{dz} - abz = 0
%\]
finden wir jetzt die Differentialgleichungen für $g(x)$ 
wie folgt:
\begin{equation}
x(1-sx^2)g''(x)
+
(2c-1-(2a+2b+1)sx^2) g'(x)
-4absx g(x)
=
0.
\label{buch:differential:hypergeometrisch:geradedgl}
\end{equation}


%
%
%
\subsubsection{Differentialgleichung von $x^\varrho w(sx^\nu)$}
Die Methode der verallgemeinerten Potenzreihen zeigt, dass eine
gewöhnliche Potenzreihe, wie $\mathstrut_pF_q$ eine ist,
manchmal nicht reicht, um eine Lösung einer Differentialgleichung
darzustellen.
Die Besselsche Differentialgleichung ist von dieser Art.
Eine verallgemeinerte Potenzreihe erhält man mit Hilfe eines
zusätzlichen Faktors der Form $x^\varrho$.
Der Fall $\varrho=1$ deckt auch die früher vorgeschlagene
Funktion $u(x)=x\cdot \mathstrut_pF_q(sx^2)$ ab.

In den bisherigen Beispielen haben wir als Argument für eine
hypergeometrische Funktion $\mathstrut_pF_q$ einen Ausdruck
der Form $sx^2$ verwendet, was für die Beispiele gereicht hat,
aber zum Beispiel für die später untersuchten Airy-Funktionen
nicht genügt.
Daher soll jetzt für eine Funktion $f(x)
= x^\varrho\cdot \mathstrut_pF_q(sx^\nu)$ eine Differentialgleichung
aus der Differentialgleichung der hypergeometrischen Funktion
abgeleitet werden.
Dabei soll der im vorangegangenen Abschnitt behandelte Fall
$\varrho=0$ und $\nu=2$ als Leitlinie dienen.

Wie vorhin beginnen wir damit, die Ableitungen von $f(x)$ zu
berechnen:
\[
\begin{linsys}{4}
f(x)   &=& x^\varrho w(sx^\nu)
       & &
       & &
\\
f'(x)  &=& \varrho x^{\varrho-1}   w(sx^\nu)
       &+& \nu s x^{\varrho+\nu-1} w'(sx^\nu)
       & &
\\
f''(x) &=& (\varrho-1)\varrho x^{\varrho-2}      w(sx^\nu)
       &+& \nu(2\varrho+\nu-1)sx^{\varrho+\nu-2} w'(sx^\nu)
       &+& \nu^2 s^2 x^{\varrho+2\nu-2}          w''(sx^\nu)
\end{linsys}
\]
Dies ist ein lineares Gleichungssystem, welches in Matrixform
geschrieben werden kann als
\[
\begin{pmatrix}
f(x)\\
f'(x)\\
f''(x)
\end{pmatrix}
=
%                [          rho          ]
%                [         x             ]
%                [                       ]
%(%o40)  Col 1 = [          rho - 1      ]
%                [     rho x             ]
%                [                       ]
%                [     2         rho - 2 ]
%                [ (rho  - rho) x        ]
%
%         [                   0                   ]
%         [                                       ]
%         [                rho + nu - 1           ]
% Col 2 = [          nu s x                       ]
%         [                                       ]
%         [               2          rho + nu - 2 ]
%         [ (2 nu rho + nu  - nu) s x             ]
%
%         [           0            ]
%         [                        ]
% Col 3 = [           0            ]
%         [                        ]
%         [   2  2  rho + 2 nu - 2 ]
%         [ nu  s  x               ]
\underbrace{
\begin{pmatrix}
x^\varrho
	& 0
		& 0 \\
\varrho x^{\varrho-1}
	& \nu s x^{\varrho+\nu-2}
		& 0 \\
(\varrho-1)\varrho x^{\varrho-2}
	& \nu(2\varrho+\nu-1)sx^{\varrho+\nu-2}
		& \nu^2 s^2 x^{\varrho+2\nu-2}
\end{pmatrix}
}_{\displaystyle = A_f}
=
\begin{pmatrix}
w(sx^\nu)\\
w'(sx^\nu)\\
w''(sx^\nu)
\end{pmatrix}.
\]
Die Inverse der Matrix $A_f$ ist
\[
A_f^{-1}
=
%                [               1                 ]
%                [              ----               ]
%                [               rho               ]
%                [              x                  ]
%                [                                 ]
%                [              (- rho) - nu       ]
%                [         rho x                   ]
%(%o42)  Col 1 = [       - -----------------       ]
%                [               nu s              ]
%                [                                 ]
%                [     2            (- rho) - 2 nu ]
%                [ (rho  + nu rho) x               ]
%                [ ------------------------------- ]
%                [               2  2              ]
%                [             nu  s               ]
%         [                   0                    ]
%         [                                        ]
%         [            (- rho) - nu + 1            ]
%         [           x                            ]
%         [           -----------------            ]
% Col 2 = [                 nu s                   ]
%         [                                        ]
%         [                     (- rho) - 2 nu + 1 ]
%         [   (2 rho + nu - 1) x                   ]
%         [ - ------------------------------------ ]
%         [                    2  2                ]
%         [                  nu  s                 ]
%         [          0          ]
%         [                     ]
%         [          0          ]
%         [                     ]
% Col 3 = [  (- rho) - 2 nu + 2 ]
%         [ x                   ]
%         [ ------------------- ]
%         [         2  2        ]
%         [       nu  s         ]
\renewcommand{\arraystretch}{1.7}
\frac{1}{x^\varrho}
\begin{pmatrix}
\displaystyle 1
	& 0
		& 0
\\
\displaystyle-\frac{\varrho}{\nu s} x^{-\nu}
	&\displaystyle \frac{1}{\nu s} x^{-\nu+1}
		& 0
\\
\displaystyle\frac{\varrho^2+\nu\varrho}{\nu^2s^2}x^{-2\nu}
	&\displaystyle -\frac{2\varrho+\nu-1}{\nu^2s^2} x^{-2\nu+1}
		&\displaystyle \frac{1}{\nu^2s^2} x^{-2\nu+2}
\end{pmatrix}
\] 
Damit kann man jetzt die Funktion $w(sx^\nu)$ und die Ableitungen
$w'(sx^\nu)$ und $w''(sx^\nu)$ durch $f$ und die Ableitungen davon
ausdrücken als
\begin{equation}
\renewcommand{\arraystretch}{2.3}
\begin{linsys}{4}
w(sx^\nu)
	&=& \displaystyle \frac{1}{x^\varrho} f(x)
	& & 
	& &
\\
w'(sx^\nu)
	&=& \displaystyle -\frac{\varrho}{\nu s} x^{-\varrho-\nu} f(x)
	&+& \displaystyle \frac{1}{\nu s}x^{-\varrho-\nu+1}
	& &
\\
w''(sx^\nu)
	&=& \displaystyle \frac{\varrho^2+\nu\varrho}{\nu^2 s^2}x^{-\varrho-2\nu}
	&-& \displaystyle \frac{2\varrho+\nu-1}{\nu^2s^2} x^{-\varrho-2\nu+1}
	&+& \displaystyle \frac{1}{\nu^2 s^2} x^{-\varrho-2\nu+2}.
\end{linsys}
\label{buch:differentialgleichungen:hypergeometrisch:wsubst}
\end{equation}
Einsetzen in die Differentialgleichung der hypergeometrischen Funktion
$\mathstrut_2F_1$ liefert die Differentialgleichung
\begin{equation}
\begin{aligned}
sx^4(x^2-1) f''
%(%i50) ratsimp(subst(0,F,subst(0,Fp,ef))/Fpp)
%(%o50) -x^((-rho)-2*nu)*(s*x^6-x^4)
%(%i51) ratsimp(subst(0,F,subst(0,Fpp,ef))/Fp)
%\\
&-(
	((-2\varrho-\nu+1) s x^5 +x^\nu(b+a+1)\nu s x^3-c\nu x)
	+
	(2\varrho+\nu-1)x^3
)f'
\\
%(%o51) -x^((-rho)-2*nu)*(((-2*rho)-nu+1)*s*x^5+x^nu*((b+a+1)*nu*s*x^3-c*nu*x)
%                                             +(2*rho+nu-1)*x^3)
%(%i52) ratsimp(subst(0,Fp,subst(0,Fpp,ef))/F)
%(%o52) -x^((-rho)-2*nu)*(a*b*nu^2*s*x^(2*nu)+(rho^2+nu*rho)*s*x^4
%                                            +x^nu
%                                             *(((-b)-a-1)*nu*rho*s*x^2
%                                              +c*nu*rho)
%                                            +((-rho^2)-nu*rho)*x^2)
&-(
	ab\nu^2 sx^{2\nu}
	+ \varrho(\varrho+\nu)sx^4
	+ x^\nu((-b-a-1)\nu\varrho s x^2 + c\nu\varrho)
	- \varrho(\varrho+\nu)x^2
)f
=0
\end{aligned}
\label{buch:differentialgleichungen:hypergeometrisch:2f1dgl}
\end{equation}
für die Funktion
\[
f(x)=x^\varrho \cdot\mathstrut_2F_1\biggl(
\begin{matrix}a,b\\c\end{matrix}; sx^\nu
\biggr).
\]
Die Differentialgleichung
\eqref{buch:differentialgleichungen:hypergeometrisch:2f1dgl}
ist etwas unübersichtlich, daher soll sie in einem Beispiel illustriert
werden.

\begin{beispiel}
Früher in Aufgabe \ref{401} auf Seite \pageref{401}
wurde gezeigt, dass 
\[
\arcsin x = x\,\mathstrut_2F_1\biggl(
\begin{matrix}\frac12,\frac12\\\frac32\end{matrix};x^2
\biggr)
=
x^\nu\cdot \mathstrut_2F_1\biggl(
\begin{matrix}a,b\\c\end{matrix};
sx^\nu
\biggr)
\]
ist.
Die Arkus-Sinus-Funktion ist daher Lösung der Differentialgleichung
\eqref{buch:differentialgleichungen:hypergeometrisch:2f1dgl}
mit
\[
\varrho=1,\quad
a=\frac12,\quad
b=\frac12,\quad
c=\frac32,\quad
s=1,\quad
\nu=2,
\]
also
\begin{equation}
-\frac{x^2-1}{x}f''
-f'
=
0
\qquad\Rightarrow\qquad
(1-x^2)f''=xf'.
\label{buch:differentialgleichungen:hypergeometrisch:beispiel:arcsindgl}
\end{equation}
Tatsächlich ist
\[
\frac{d}{dx}\arcsin x
=
\frac{1}{(1-x^2)^{\frac12}}
\qquad\text{und}\qquad
\frac{d^2}{dx^2} \arcsin x
=
\frac{x}{(1-x^2)^{\frac32}},
\]
und nach Einsetzen in die Differentialgleichung
\[
(1-x^2)
\cdot
\frac{x}{(1-x^2)^{\frac32}}
-
x
\cdot
\frac{1}{(1-x^2)^{\frac12}}
=
0.
\]
Die Arkus-Sinus-Funktion ist also tatsächlich eine Lösung der
Differentialgleichung
\eqref{buch:differentialgleichungen:hypergeometrisch:beispiel:arcsindgl}.
\end{beispiel}

%
%
%
\subsubsection{Differentialgleichung für Funktionen, die aus $\mathstrut_0F_1$ zusammengesetzt sind}
Die Substitutionen 
\eqref{buch:differentialgleichungen:hypergeometrisch:wsubst}
angewendet auf die Differentialgleichung
\eqref{buch:differentialgleichungen:0f1:dgl}
der Funktion $\mathstrut_0F_1$
liefert
%
%(%i60) ratsimp(subst(0,F,subst(0,Fp,e0f1))/Fpp)
%(%o60) x^2
%(%i61) ratsimp(subst(0,F,subst(0,Fpp,e0f1))/Fp)
%(%o61) ((-2*rho)+(beta-1)*nu+1)*x
%(%i62) ratsimp(subst(0,Fp,subst(0,Fpp,e0f1))/F)
%(%o62) (-nu^2*s*x^nu)+rho^2+(1-beta)*nu*rho
\begin{equation}
x^2f''
+
(-2\varrho+(\beta-1)\nu+1)xf'
+
(-\nu^2sx^\nu + \varrho^2 -(\beta-1)\nu\varrho)f
=
0.
\label{buch:differentialgleichungen:0F1:dgl}
\end{equation}
Die nächsten zwei Abschnitte sollen zeigen, wie sich daraus für die
Bessel-Funktionen wie auch die Airy-Funktionen, die sich durch
$\mathstrut_0F_1$ ausdrücken, die Besselsche und die Airysche
Differentialgleichung wiedergewonnen werden kann.

\begin{beispiel}
Die hyperbolische Funktion
\[
\sinh x
=
x\cdot \mathstrut_0F_1\biggl(
\begin{matrix}\text{---}\\\frac32\end{matrix};\frac{x^2}{4}
\biggr)
\]
hat die Differentialgleichung
\eqref{buch:differentialgleichungen:0F1:dgl}
mit den Parametern
\[
\varrho=1,\quad
s=\frac14,\quad
\nu=2,\quad
b=\frac32.
\]
Einsetzen der Parameter in
\eqref{buch:differentialgleichungen:0F1:dgl}
liefert
\[
0
=
x^2f''
+
\biggl(-2+\frac12\cdot 2 + 1\biggr) xf'
+
\biggl(-2^2\frac14x^2 + 1^2 - \frac12 \cdot 2 \cdot 1\biggr) f
=
x^2f''
-x^2f
\]
Daraus ergib sich die bekannte Differentialgleichung
$y''-y=0$
der hyperbolischen Funktionen.
\end{beispiel}

%
% Besselsche Differentialgleichung
%
\subsubsection{Besselsche Differentialgleichung}
Die Besselfunktionen lassen sich in der Form
\begin{equation}
J_\alpha(x)
=
\frac{(x/2)^\alpha}{\Gamma(\alpha+1)} \,
\mathstrut_0F_1\biggl(
\begin{matrix}\text{---}\\\alpha+1\end{matrix};-\frac14x^2
\biggr)
=
\frac{1}{2^\alpha\Gamma(\alpha+1)}
x^\varrho\cdot
\mathstrut_0F_1\biggl(
\begin{matrix}\text{---}\\b\end{matrix};sx^\nu
\biggr)
\label{buch:differentialgleichungen:0f1:besselfunktion}
\end{equation}
schreiben.
Somit sollte sich aus der
Differentialgleichung~\eqref{buch:differentialgleichungen:0f1:dgl}
der Funktion $\mathstrut_0F_1$ die Besselsche Differentialgleichung
\eqref{buch:differentialgleichungen:eqn:bessel} rekonstruieren lassen.
Dazu substituieren wir die aus
\eqref{buch:differentialgleichungen:0f1:besselfunktion}
abgelesenen Parameter
\[
\varrho=\alpha,\quad\nu=2,\quad s=-\frac14,\quad b=\alpha+1
\]
in \eqref{buch:differentialgleichungen:0F1:dgl} und erhalten
die Differentialgleichung
\begin{equation}
x^2y''
+
%(-2\alpha+2\alpha+1)xy'
xy'
+
%(-4sx^2 + \alpha^2 -2\alpha^2)y
(x^2 - \alpha^2)y
=
0.
\label{buch:differentialgleichungen:0F1:besseldgl}
\end{equation}
Dies ist tatsächlich die Besselsche Differentialgleichung.

%
% Airy-Differentialgleichung
%
\subsubsection{Die Airy-Differentialgleichung}
Die in Aufgabe \ref{501} untersuchte
Airy-Differentialgleichung $y''-xy=0$ hat die Funktionen
\begin{align*}
y_1(x)
&=
\mathstrut_0F_1\biggl(
\begin{matrix}\text{---}\\\frac23\end{matrix};\frac{x^3}9
\biggr)
=x^\varrho\cdot \mathstrut_0F_1\biggl(
\begin{matrix}\text{---}\\b\end{matrix};sx^\nu
\biggr)
&&\text{mit $\varrho=0$, $\nu=3$, $s=\frac19$, $b=\frac23$, }
\intertext{und}
y_2(x)
&=
x\cdot
\mathstrut_0F_1\biggl(
\begin{matrix}\text{---}\\\frac23\end{matrix};\frac{x^3}9
\biggr)
=x^\varrho\cdot \mathstrut_0F_1\biggl(
\begin{matrix}\text{---}\\b\end{matrix};sx^\nu
\biggr)
&&\text{mit $\varrho=1$, $\nu=3$, $s=\frac19$, $b=\frac43$, }
\end{align*}
als Lösungen.
Die Differentialgleichung von $\mathstrut_0F_1$ sollte sich in diesem
Fall also auf die Airy-Dif\-fe\-ren\-tial\-glei\-chung reduzieren lassen.

Bei der Substition der Parameter in die Differentialgleichung
\eqref{buch:differentialgleichungen:0F1:dgl} beachten wird, dass
die beiden möglichen Werte für $b$ auf $b-1=\pm\frac13$
führen, mit dem positiven Zeichen für den zweiten Fall, in dem $\varrho=1$
ist.
So ergibt sich die Differentialgleichung
\begin{align*}
x^2y''
+
(-2\varrho\pm\frac13\cdot 3+1)xy'
+
(-x^\nu + \varrho^2 \mp\frac13\cdot 3\varrho)y
&=
0
\\
x^2y''
+
(-2\varrho\pm1+1)xy'
+
(-x^3 + \varrho^2 \mp\varrho)y
&=
0
\\
x^2y''
-
x^3y
&=
0
\qquad\Rightarrow\qquad y''-xy=0.
\end{align*}
Dies ist wie erwartet die Airy-Differentialgleichung.

\subsection{Differentialgleichung der Tschebyscheff-Polynome}
Die Tschebyscheff-Polynome erster Art haben die Darstellung
\[
T_n(x) = \cos(n\arccos x).
\]
Die Ableitungen sind
\begin{align*}
T'_n(x) &= \frac{n}{\sqrt{1-x^2}} \sin(n\arccos x)
\\
T''_n(x) &= 
-\frac{n^2}{1-x^2} T_n(x)
+
n\frac{x}{(1-x^2)^{\frac32}} \sin(n\arccos x)
\end{align*}
Multipliziert man $T_n''(x)$  mit $(1-x^2)$ und subtrahiert
man $xT_n'(x)$, fällt der Term $\sin(n\arccos x)$ weg und es bleibt
\begin{equation*}
(1-x^2)T''_n(x) -xT'_n(x) = -n^2 T_n(x),
%\label{buch:differential:tschebyscheff:Tdgl}
\end{equation*}
die Tschebyscheff-Polynome sind also Lösungen der Differentialgleichung
\begin{equation}
(1-x^2)y'' -xy' +n^2 y=0,
\label{buch:differential:tschebyscheff:Tdgl}
\end{equation}
sie heisst die {\em Tschbeyscheff-Differentialgleichung}.

\subsubsection{Tschebyscheff-Differentialgleichung und hypergeometrische
Differentialgleichung}
Die hypergeometrische Differentialgleichung hat eine ähnliche Struktur
wie die Tschebyscheff-Differentialgleichung
\eqref{buch:differential:tschebyscheff:Tdgl}.
Der Koeffizient der zweiten Ableitung hat jedoch die Nullstellen
$\pm 1$ bei der Tschebyscheff-Differentialgleichung, während es bei
der hypergeometrischen Differentialgleichung die Nullstellen
$0$ und $1$ sind.
Wir verwenden daher die Substitution $z = \frac12(1-x)$ und 
$w(z)=y(1-2z)$ und formen damit die hypergeometrische
Differentialgleichung um.
Der Faktor $z(1-z)$ wird damit zu
\[
z(1-z)
=
\frac12(1-x)\biggl(1-\frac12(1-x)\biggr)
=
\frac12(1-x) \frac12(1+x)
=
\frac14 (1-x^2).
\]
Die Ableitungen sind 
\begin{align*}
w'(z) &= -2y'(1-2z) \\
w''(z) &= 4y''(1-2z),
\end{align*}
wir setzen sie in die hypergeometrische Differentialgleichung ein
\begin{align*}
0
&=
z(1-z) w'(z) 
+
(c-(a+b+1)z) w'(z) - ab w(z)
\\
&=
\frac14(1-x^2) 4y''(x)
-
2
\biggl(c-(a+b+1)\frac12(1-x)\biggr)
y'(x)
-aby(x).
\\
&=
(1-x^2)y''
+
(a+b+1-2c-(a+b+1)x) y'
-
aby
\end{align*}
Diese Differentialgleichung kann tatsächlich in die Form der 
Tschebyscheff-Differentialgleichung gebracht werden, wenn man setzt
\begin{equation}
\left.
\begin{aligned}
a&=\phantom{-}n\\
b&=-n\\
c&=\frac{a+b+1}2
\end{aligned}
\right\}
\;
\quad\Rightarrow\quad
(1-x^2)y''+
\biggl(\underbrace{a+b+1-2\frac{a+b+1}2}_{\displaystyle=0}-(\underbrace{n-n+1}_{\displaystyle=1})x\biggr)y'
-n(-n)y=0.
\end{equation}
Die letzte Gleichung ist identisch mit
\eqref{buch:differential:tschebyscheff:Tdgl}.
Die beiden Parameter $a$ und $b$ dürfen natürlich auch vertauscht
werden.

\subsubsection{Tschebyscheff-Polynome als hypergeometrische Funktionen}
Aus der Umformung der eulerschen hypergeometrischen Differentialgleichung
in die Tschebyscheff-Differntialgleichung kann man jetzt ablesen, dass
eine Lösung der Tschebyscheff-Differentialgleichung auch mit der
hypergeometrischen Funktion $\mathstrut_2F_1$ geschrieben werden kann,
nämlich
\[
y(x)
=
\mathstrut_2F_1\biggl(
\begin{matrix}
n,-n\\
\frac12
\end{matrix};\frac{1-x}2
\biggr).
\]
Wegen $b=-n$ ist diese Funktion ein Polynom mit den Werten
\[
\begin{aligned}
y(1) &= 1 \\
y'(1)&= n^2,
\end{aligned}
\]
den gleichen Werten, die auch das Tschbescheff-Polynome $T_n(x)$ annimmt.
Es folgt daher
\begin{equation}
T_n(x)
=
\mathstrut_2F_1\biggl(
\begin{matrix}n,-n\\\frac12\end{matrix};
\frac{1-x}2
\biggr).
\end{equation}
Auch die Tschebyscheff-Polynome lassen sich also mit Hilfe einer
hypergeometrischen Funktion schreiben, wie schon in
\eqref{buch:rekursion:hypergeometrisch:tschebyscheff2f1}
bemerkt wurde.

