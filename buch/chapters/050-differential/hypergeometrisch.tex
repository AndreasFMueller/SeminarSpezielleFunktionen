%
% hypergeometrisch.tex
%
% (c) 2021 Prof Dr Andreas Müller, OST Ostschweizer Fachhochschule
%
\section{Hypergeometrische Differentialgleichung
\label{buch:differentialgleichungen:section:hypergeometrisch}}
Die hypergeometrische Funktion $\mathstrut_2F1(a,b;c;x)$ wurde in
Abschnitt~\ref{buch:rekursion:section:hypergeometrische-funktion}
als Potenzreihe mit sehr speziellen Koeffizienten, die sich aus
Pochhammer-Symbolen.
Es stellt sich aber heraus, dass man sie auch als Lösung einer
gewöhnlichen Differentialgleichung bekommen kann, die bereits
Euler studiert hat.

\subsection{Die Eulersche hypergeometrische Differentialgleichung
\label{buch:differentialgleichung:subsection:euler-hypergeometrisch}}
Die hypergeometrische Funktion $\mathstrut_2F_1(a,b;c;x)$ ist eine
Lösung der {\em Eulerschen hypergeometrischen Differentialgleichung}
(zu unterscheiden von der Eulerschen Differentialgleichung, die sich
immer auf eine lineare Differentialgleichung mit konstanten Koeffizienten
reduzieren lässt)
\begin{equation}
x(1-x) \frac{d^2y}{dx^2} + (c-(a+b+1)x)\frac{dy}{dx} - ab y = 0
\label{buch:differentialgleichungen:hypergeo:eulerdgl}
\end{equation}
Wir prüfen dies nach, indem wir die Definition der hypergeometrischen
Funktion 
\begin{align*}
y(x)
&=
\mathstrut_2F_1(a,b;c;x)
=
\sum_{k=0}^\infty
\frac{(a)_k(b)_k}{(c)_k} \frac{x^k}{k!}
\intertext{mit den Ableitungen}
y'(x)
&=
\sum_{k=1}^\infty 
\frac{(a)_k(b)_k}{(c)_k} \frac{x^{k-1}}{(k-1)!}
\\
y''(x)
&=
\sum_{k=2}^\infty 
\frac{(a)_k(b)_k}{(c)_k} \frac{x^{k-2}}{(k-2)!}
\end{align*}
einsetzen.
Die Gleichung, die sich ergibt, ist
\begin{align*}
0
&=
x(1-x)
\sum_{k=2}^\infty
\frac{(a)_k(b)_k}{(c)_k}\frac{x^{k-2}}{(k-2)!}
+
(c-(a+b+1)x)
\sum_{k=1}^\infty
\frac{(a)_k(b)_k}{(c)_k}\frac{x^{k-1}}{(k-1)!}
-ab
\sum_{k=0}^\infty
\frac{(a)_k(b)_k}{(c)_k} \frac{x^k}{k!}
\\
&=
\sum_{k=2}^\infty
\frac{(a)_k(b)_k}{(c)_k}\frac{x^{k-1}}{(k-2)!}
-
\sum_{k=2}^\infty
\frac{(a)_k(b)_k}{(c)_k}\frac{x^k}{(k-2)!}
+
c\sum_{k=1}^\infty
\frac{(a)_k(b)_k}{(c)_k}\frac{x^{k-1}}{(k-1)!}
\\
&\qquad
-(a+b+1)
\sum_{k=1}^\infty
\frac{(a)_k(b)_k}{(c)_k}\frac{x^k}{(k-1)!}
-ab
\sum_{k=0}^\infty
\frac{(a)_k(b)_k}{(c)_k} \frac{x^k}{k!}
\\
&=
\sum_{k=1}^\infty
\frac{(a)_{k+1}(b)_{k+1}}{(c)_{k+1}}\frac{x^k}{(k-1)!}
-
\sum_{k=2}^\infty
\frac{(a)_k(b)_k}{(c)_k}\frac{x^k}{(k-2)!}
+
c\sum_{k=0}^\infty
\frac{(a)_{k+1}(b)_{k+1}}{(c)_{k+1}}\frac{x^k}{k!}
\\
&\qquad
-(a+b+1)
\sum_{k=1}^\infty
\frac{(a)_k(b)_k}{(c)_k}\frac{x^k}{(k-1)!}
-ab
\sum_{k=0}^\infty
\frac{(a)_k(b)_k}{(c)_k} \frac{x^k}{k!}.
\end{align*}
Zum konstanten Koeffizienten für $k=0$ tragen nur die dritte und letzte
Summe bei, dies sind die Terme
\[
c\frac{(a)_1(b)_1}{(c)_1}-ab\frac{(a)_0(b)_0}{(c)_0}
=
c\frac{ab}{c}-ab\frac{1\cdot 1}{1}
=
0.
\]
Für den linearen Term $k=1$ kommen je ein Term aus der ersten aund vierten
Summe hinzu, dies ergibt
\begin{align*}
&\phantom{\mathstrut=\mathstrut}
\frac{(a)_2(b)_2}{(c)_2}
+c\frac{(a)_2(b)_2}{(c)_2}
-(a+b+1)\frac{(a)_1(b)_1}{(c)_1}
-ab\frac{(a)_1(b)_1}{(c)_1}
\\
&=
\frac{a(a+1)b(b+1)}{c(c+1)}
(1+c)
-(ab+a+b+1)
\frac{ab}{c}
\\
&=
\frac{a(a+1)b(b+1)}{c}
-
(a+1)(b+1)\frac{ab}{c}
=0.
\end{align*}
Durch Koeffizientenvergleich erhalten wir für $k\ge 2$ 
\begin{align*}
0
&=
\frac{(a)_{k+1}(b)_{k+1}}{(c)_{k+1}} \frac1{(k-1)!} 
-
\frac{(a)_k(b)_k}{(c)_k} \frac1{(k-2)!} 
+
c\frac{(a)_{k+1}(b)_{k+1}}{(c)_{k+1}} \frac{1}{k!}
\\
&\qquad
-(a+b+1)\frac{(a)_k(b)_k}{(c)_k}\frac{1}{(k-1)!}
-ab \frac{(a)_k(b)_k}{(c)_k}\frac{1}{k!}
\\
&=
\frac{(a)_k(b)_k}{(c)_k}
\frac{1}{(k-2)!}
\biggl(
\frac{(a+k)(b+k)}{c+k}\frac1{k-1}
-1
+
c\frac{(a+k)(b+k)}{c+k}\frac1{(k-1)k}
\\
&\qquad
-(a+b+1)\frac1{k-1} 
-ab \frac1{(k-1)k}
\biggr)
\\
&=
\frac{(a)_k(b)_k}{(c)_{k+1}}
\frac{1}{k!}
\biggl(
(a+k)(b+k)k - (c+k)(k-1)k + (a+k)(b+k) - (a+b+1)(c+k)k-ab(c+k)
\biggr)
\end{align*}







