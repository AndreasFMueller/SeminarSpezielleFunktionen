%
% hypergeometrisch.tex
%
% (c) 2021 Prof Dr Andreas Müller, OST Ostschweizer Fachhochschule
%
\section{Hypergeometrische Differentialgleichung
\label{buch:differentialgleichungen:section:hypergeometrisch}}
Die hypergeometrische Funktion $\mathstrut_2F1(a,b;c;x)$ wurde in
Abschnitt~\ref{buch:rekursion:section:hypergeometrische-funktion}
als Potenzreihe mit sehr speziellen Koeffizienten, die sich aus
Pochhammer-Symbolen.
Es stellt sich aber heraus, dass man sie auch als Lösung einer
gewöhnlichen Differentialgleichung bekommen kann, die bereits
Euler studiert hat.

\subsection{Die Eulersche hypergeometrische Differentialgleichung
\label{buch:differentialgleichung:subsection:euler-hypergeometrisch}}
Die hypergeometrische Funktion $\mathstrut_2F_1(a,b;c;x)$ ist eine
Lösung der {\em Eulerschen hypergeometrischen Differentialgleichung}
(zu unterscheiden von der Eulerschen Differentialgleichung, die sich
immer auf eine lineare Differentialgleichung mit konstanten Koeffizienten
reduzieren lässt)
\begin{equation}
x(1-x) \frac{d^2y}{dx^2} + (c-(a+b+1)x)\frac{dy}{dx} - ab y = 0
\label{buch:differentialgleichungen:hypergeo:eulerdgl}
\end{equation}
Wir prüfen dies nach, indem wir die Definition der hypergeometrischen
Funktion 
\begin{align*}
y(x)
&=
\mathstrut_2F_1(a,b;c;x)
=
\sum_{k=0}^\infty
\frac{(a)_k(b)_k}{(c)_k} \frac{x^k}{k!}
\intertext{mit den Ableitungen}
y'(x)
&=
\sum_{k=1}^\infty 
\frac{(a)_k(b)_k}{(c)_k} \frac{x^{k-1}}{(k-1)!}
\\
y''(x)
&=
\sum_{k=2}^\infty 
\frac{(a)_k(b)_k}{(c)_k} \frac{x^{k-2}}{(k-2)!}
\end{align*}
einsetzen.
Die Gleichung, die sich ergibt, ist
\begin{align*}
0
&=
x(1-x)
\sum_{k=2}^\infty
\frac{(a)_k(b)_k}{(c)_k}\frac{x^{k-2}}{(k-2)!}
+
(c-(a+b+1)x)
\sum_{k=1}^\infty
\frac{(a)_k(b)_k}{(c)_k}\frac{x^{k-1}}{(k-1)!}
-ab
\sum_{k=0}^\infty
\frac{(a)_k(b)_k}{(c)_k} \frac{x^k}{k!}
\\
&=
\sum_{k=2}^\infty
\frac{(a)_k(b)_k}{(c)_k}\frac{x^{k-1}}{(k-2)!}
-
\sum_{k=2}^\infty
\frac{(a)_k(b)_k}{(c)_k}\frac{x^k}{(k-2)!}
+
c\sum_{k=1}^\infty
\frac{(a)_k(b)_k}{(c)_k}\frac{x^{k-1}}{(k-1)!}
\\
&\qquad
-(a+b+1)
\sum_{k=1}^\infty
\frac{(a)_k(b)_k}{(c)_k}\frac{x^k}{(k-1)!}
-ab
\sum_{k=0}^\infty
\frac{(a)_k(b)_k}{(c)_k} \frac{x^k}{k!}
\\
&=
\sum_{k=1}^\infty
\frac{(a)_{k+1}(b)_{k+1}}{(c)_{k+1}}\frac{x^k}{(k-1)!}
-
\sum_{k=2}^\infty
\frac{(a)_k(b)_k}{(c)_k}\frac{x^k}{(k-2)!}
+
c\sum_{k=0}^\infty
\frac{(a)_{k+1}(b)_{k+1}}{(c)_{k+1}}\frac{x^k}{k!}
\\
&\qquad
-(a+b+1)
\sum_{k=1}^\infty
\frac{(a)_k(b)_k}{(c)_k}\frac{x^k}{(k-1)!}
-ab
\sum_{k=0}^\infty
\frac{(a)_k(b)_k}{(c)_k} \frac{x^k}{k!}.
\end{align*}
Zum konstanten Koeffizienten für $k=0$ tragen nur die dritte und letzte
Summe bei, dies sind die Terme
\[
c\frac{(a)_1(b)_1}{(c)_1}-ab\frac{(a)_0(b)_0}{(c)_0}
=
c\frac{ab}{c}-ab\frac{1\cdot 1}{1}
=
0.
\]
Für den linearen Term $k=1$ kommen je ein Term aus der ersten aund vierten
Summe hinzu, dies ergibt
\begin{align*}
&\phantom{\mathstrut=\mathstrut}
\frac{(a)_2(b)_2}{(c)_2}
+c\frac{(a)_2(b)_2}{(c)_2}
-(a+b+1)\frac{(a)_1(b)_1}{(c)_1}
-ab\frac{(a)_1(b)_1}{(c)_1}
\\
&=
\frac{a(a+1)b(b+1)}{c(c+1)}
(1+c)
-(ab+a+b+1)
\frac{ab}{c}
\\
&=
\frac{a(a+1)b(b+1)}{c}
-
(a+1)(b+1)\frac{ab}{c}
=0.
\end{align*}
Durch Koeffizientenvergleich erhalten wir für $k\ge 2$ 
\begin{align*}
0
&=
\frac{(a)_{k+1}(b)_{k+1}}{(c)_{k+1}} \frac1{(k-1)!} 
-
\frac{(a)_k(b)_k}{(c)_k} \frac1{(k-2)!} 
+
c\frac{(a)_{k+1}(b)_{k+1}}{(c)_{k+1}} \frac{1}{k!}
\\
&\qquad
-(a+b+1)\frac{(a)_k(b)_k}{(c)_k}\frac{1}{(k-1)!}
-ab \frac{(a)_k(b)_k}{(c)_k}\frac{1}{k!}
\\
&=
\frac{(a)_k(b)_k}{(c)_{k+1}}
\frac{1}{k!}
\biggl(
(a+k)(b+k)k
-(c+k)(k-1)k
+
c(a+k)(b+k)
\\
&\qquad
\qquad
\qquad
-(a+b+1)(c+k)k
-ab(c+k)
\biggr).
\intertext{Der zweite, vierte und fünfte Term können zu}
&=
\frac{(a)_k(b)_k}{(c)_{k+1}}
\frac{1}{k!}
\biggl(
(a+k)(b+k)k
+
c(a+k)(b+k)
-(ab+ak+bk+k^2)(c+k)
\biggr)
\intertext{zusammengefasst werden.
Der Faktor $(ab+ak+bk+k^2)$ kann als Produkt $(a+k)(b+k)$ faktorisiert
werden, der dann als gemeinsamer Faktor aus allen Termen ausgeklammert
werden kann:}
&=
\frac{(a)_k(b)_k}{(c)_{k+1}}
\frac{1}{k!}
\biggl(
(a+k)(b+k)k
+
c(a+k)(b+k)
-(a+k)(b+k)(c+k)
\biggr)
\\
&=
\frac{(a)_{k+1}(b)_{k+1}}{(c)_{k+1}}
\frac{1}{k!}
\biggl(
k
+
c
-(c+k)
\biggr)
=0.
\end{align*}
Damit ist gezeigt, dass $\mathstrut_2F_1(a,b;c;x)$ eine Lösung
der Differentialgleichung ist.

Die hypergeometrische Reihe kann auch direkt mit Hilfe der
Potenzreihenmethode als Lösung der Differentialgleichung gefunden 
werden.

\subsection{Lösung als verallgemeinerte Potenzreihe}
Da die hypergeometrische Reihe eine Differentialgleichung
zweiter Ordnung mit einer Singularität bei $x=0$ ist, 
kann man versuchen eine zweite, linear unabhängige Lösung mit
Hilfe der Methode der verallgemeinerten Potenzreihen zu finden.
Dazu setzt man die Lösung in der Form
\begin{align*}
y_2(x)
&=
\sum_{k=0}^\infty a_kx^{\varrho+k}
&
&\Rightarrow&
y_2'(x)
&=
\sum_{k=0}^\infty (\varrho+k)a_kx^{\varrho+k-1}
\\
&&
&&
y_2''(x)
&=
\sum_{k=0}^\infty (\varrho+k)(\varrho+k-1)a_kx^{\varrho+k-2}
\end{align*}
an, wobei $a_0\ne 0$ sein soll.
Einsetzen in die Differentialgleichung ergibt
\begin{align*}
0&=
x(1-x)y_2''(x) + (c-(a+b+1)x) y_2'(x) -aby_2(x)
\\
&=
x(1-x)
\sum_{k=0}^\infty (\varrho+k)(\varrho+k-1)a_kx^{\varrho+k-2}
+
(c-(a+b+1)x)
\sum_{k=0}^\infty (\varrho+k)a_kx^{\varrho+k-1}
-
abx^{\varrho}\sum_{k=0}^\infty a_kx^{\varrho+k}
\\
&=
-\sum_{k=0}^\infty (\varrho+k)(\varrho+k-1)a_kx^{\varrho+k}
+
\sum_{k=0}^\infty (\varrho+k)(\varrho+k-1)a_kx^{\varrho+k-1}
+
c
\sum_{k=0}^\infty (\varrho+k)a_kx^{\varrho+k-1}
\\
&\qquad
-
(a+b+1)
\sum_{k=0}^\infty (\varrho+k)a_kx^{\varrho+k}
-
ab
\sum_{k=0}^\infty a_kx^{\varrho+k}.
\intertext{Durch Verschiebung des Summationsindex in der zweiten
und dritten Summe wird der Koeffizientenvergleich etwas
einfacher}
&=
-\sum_{k=0}^\infty (\varrho+k)(\varrho+k-1)a_kx^{\varrho+k}
+
\sum_{k=-1}^\infty (\varrho+k+1)(\varrho+k)a_{k+1}x^{\varrho+k}
+
c
\sum_{k=-1}^\infty (\varrho+k+1)a_{k+1}x^{\varrho+k}
\\
&\qquad
-
(a+b+1)
\sum_{k=0}^\infty (\varrho+k)a_kx^{\varrho+k}
-
ab
\sum_{k=0}^\infty a_kx^{\varrho+k}
\\
&=
-\sum_{k=0}^\infty (\varrho+k)(\varrho+k-1)a_kx^{\varrho+k}
+
\sum_{k=-1}^\infty (\varrho+k+1)(\varrho+k+c)a_{k+1}x^{\varrho+k}
\\
&\qquad
-
\sum_{k=0}^\infty ((\varrho+k)(a+b+1)+ab)a_kx^{\varrho+k}
\\
&=
\bigl(
\varrho(\varrho-1)
+c\varrho \bigr)
x^{\varrho-1}
+
\sum_{k=0}^\infty
\bigl(
-(\varrho+k)(\varrho+k-1)a_k
+(\varrho+k+1)(\varrho+k+c)a_{k+1}
\\
&
\qquad
\qquad
\qquad
\qquad
\qquad
\qquad
-((\varrho+k)(a+b+1)+ab)a_k
\bigr)
x^{\varrho+k}.
\end{align*}
Aus dem ersten Term kann man die Indexgleichung
\[
0
=
\varrho(\varrho-1)+c\varrho
=
\varrho(\varrho-1+c)
\]
ablesen, die die Nullstellen $\varrho=0$ und $\varrho=1-c$ hat.
Die Nullstelle $\varrho=0$ ergibt natürlich die bereits gefundene
hypergeometrische Reihe.

Nach Einsetzen der zweiten Lösung der Indexgleichung in der Summe
legt der Koeffizientenvergleich eine Beziehung
\begin{align}
0
&=
\bigl(
-(k-c+1)(k-c)
-(k-c+1)(a+b+1)+ab
\bigr)a_k
+
(k-c+2)(k+1)
a_{k+1} 
\notag
\intertext{zwischen $a_k$ und $a_{k+1}$ fest.
Daraus kann man den Quotienten aufeinanderfolgender
Koeffizienten als}
\frac{a_{k+1}}{a_k}
&=
\frac{
-(k-c+1)(k-c)
-(k-c+1)(a+b+1)+ab
}{
\notag
(k-c+2)(k+1)
}
\\
&=
%(%i4) factor(coeff(y,q,0))
%(%o4)                  - (k - c + a + 1) (k - c + b + 1)
%(%i5) factor(coeff(y,q,1))
%(%o5)                         (k + 1) (k - c + 2)
\frac{
(a-c+1+k)
(b-c+1+k)
}{
(2-c+k)(k+1)
}
\label{buch:differentialgleichungen:hypergeo:verallgkoef}
\end{align}
finden.
Setzt man $a_0=1$, ist die zweite Lösung ist also wieder eine
hypergeometrische Funktion.%, nämlich
%\[
%y_2(x)
%=
%x^{1-c}
%\sum_{k=0}^\infty \frac{(a-c+1)_k(b-c+1)_k}{(2-c)_k}\frac{x^k}{k!}
%=
%x^{1-c}
%\mathstrut_2F_1\biggl(\begin{matrix}a-c+1,b-c+1\\2-c\end{matrix};x\biggr)
%\]
Diese Lösung ist aber nur möglich, wenn in
\eqref{buch:differentialgleichungen:hypergeo:verallgkoef}
der Nenner nicht verschwindet, d.~h.~$2-c+k\ne 0$
oder $c \ne k+2$ für all natürlichen $k$.
$c$ darf also kein natürliche Zahl $\ge 2$ sein.
Wir fassen die Resultate dieses Abschnitts im folgenden Satz zusammen.

\begin{satz}
Die eulersche hypergeometrische Differentialgleichung
\begin{equation}
x(1-x)\frac{d^2y}{dx^2}
+(c+(a+b+1)x)\frac{dy}{dx}
-ab y
=
0
\end{equation}
hat die Lösung
\[
y_1(x)
=
\mathstrut_2F_1\biggl(\begin{matrix}a,b\\c\end{matrix};x\biggr).
\]
Falls $c-2\not\in \mathbb{N}$ ist, ist
\[
y_2(x)
=
x^{1-c} \mathstrut_2F_1\biggl(\begin{matrix}a-c+1,b-c+1\\2-c\end{matrix};x\biggr)
\]
eine zweite, linear unabhängige Lösung.
\end{satz}

%
% Die verallgemeinerte hypergeometrische Differentialgleichung
%
\subsection{Verallgemeinerte hypergeometrische Differentialgleichung}
% https://de.wikipedia.org/wiki/Verallgemeinerte_hypergeometrische_Funktion
Die Ableitungsformel für die hypergeometrischen Funktionen
\[
w(z)
=
\mathstrut_nF_m
\biggl(\begin{matrix}a_1,\dots,a_m\\b_1,\dots,b_n\end{matrix};z\biggr)
\]
drückt die Ableitung $f'(z)$ durch einen Wert einer hypergeometrischen
Funktion mit ganz anderen Parametern aus, nämlich
\[
w'(z)
=
\frac{a_1\cdot\ldots\cdot a_n}{b_1\cdot\ldots\cdot b_m}
\mathstrut_mF_n\biggl(
\begin{matrix}a_1+1,\dots,a_n+1\\b_1+1,\dots,b_m+1\end{matrix};z
\biggr).
\]
Dies erlaubt aber noch nicht, eine Differentialgleichung für $w(z)$
aufzustellen, da auf der rechten Seite alle Parameter inkrementiert
worden sind.
Um eine Differentialgleichung zu erhalten, muss man den gleichen
Effekt auf einem anderen Weg erreichen.

\subsubsection{Operatoren, die genau ein $a_i$ inkrementieren}
Wir suchen einen Operator, der in der hypergeometrischen Funktion
$\mathstrut_nF_m$ nur genau den Parameter $a_i$ inkrementiert.
Der folgende Operator schafft dies:
\begin{align*}
\biggl(z\frac{d}{dz}+a_i\biggr)
\mathstrut_nF_m\biggl(\begin{matrix}a_1,\dots,a_n\\b_1,\dots,b_m\end{matrix};
z\biggr)
&=
\biggl(z\frac{d}{dz}+a_i\biggr)
\sum_{k=0}^\infty
\frac{(a_1)_k\cdot\ldots\cdot(a_n)_k}{(b_1)_k\cdot\ldots\cdot(b_m)_k}
\frac{z^k}{k!}
\\
&=
\sum_{k=0}^\infty
\frac{(a_1)_k\cdot\ldots\cdot(a_n)_k}{(b_1)_k\cdot\ldots\cdot(b_m)_k}
\frac{z^k}{(k-1)!}
+
\sum_{k=0}^\infty
a_i
\frac{(a_1)_k\cdot\ldots\cdot(a_n)_k}{(b_1)_k\cdot\ldots\cdot(b_m)_k}
\frac{z^k}{k!}
\\
&=
\sum_{k=0}^\infty
\frac{
(a_1)_k\cdots\widehat{(a_i)_k}\cdots(a_n)_k
}{
(b_1)_k\cdots(b_m)_k
}
(
k(a_i)_k
+
a_i(a_i)_k
)
\frac{z^k}{k!}
\\
&=
\sum_{k=0}^\infty
\frac{
(a_1)_k\cdots\widehat{(a_i)_k}\cdots(a_n)_k
}{
(b_1)_k\cdots(b_m)_k
}
\underbrace{(a_i)_k(a_i+k)}_{a_i(a_i+1)_k}
\frac{z^k}{k!}
\\
&=
a_i
\sum_{k=0}^\infty
\frac{
(a_1)_k\cdots (a_i+1)_k\cdots(a_n)_k
}{
(b_1)_k\cdots(b_m)_k
}
\underbrace{(a_i)_k(a_i+k)}_{a_i(a_i+1)_k}
\frac{z^k}{k!}
\\
&=
a_i\cdot\mathstrut_nF_m\biggl(
\begin{matrix}
a_1,\dots,a_i+1,\dots,a_n\\
b_1,\dots,b_m
\end{matrix}
;z
\biggr).
\end{align*}
Durch Anwendung aller Operatoren
\[
D_{a_i} = z\frac{d}{dz}+a_i
\]
kann man jetzt die Inkrementierung der $a_i$, die in der Ableitung
von $w(z)$ zu beobachten war, in Einzelschritten erreichen:
\[
D_{a_1}D_{a_2}\cdots D_{a_n} w(z)
=
a_1a_2\cdots a_n \,
\mathstrut_nF_m\biggl(
\begin{matrix}a_1+1,\dots,a_n+1\\b_1,\dots,b_m\end{matrix}; z
\biggr).
\]

\subsubsection{Operatoren, die genau ein $b_j$ dekrementieren}
Die Rechnung für die Operatoren $D_{a_i}$ ist nicht direkt auf die
$b_i$ übertragbar, wir versuchen daher erneut:
\begin{align*}
D_{b_i-1}
\,\mathstrut_nF_m
\biggl(
\begin{matrix}a_1,\dots,a_n\\b_1,\dots,a_m\end{matrix};z
\biggr)
&=
\biggl(z\frac{d}{dz}+b_j-1\biggr)
\sum_{k=0}^\infty
\frac{(a_1)_k\cdots (a_n)_k}{(b_1)_k\cdots (b_m)_k}
\frac{z^k}{k!}
\\
&=
\sum_{k=0}^\infty
\frac{(a_1)_k\cdots (a_n)_k}{(b_1)_k\cdots (b_m)_k}
\frac{z^k}{(k-1)!}
+
(b_j-1)
\sum_{k=0}^\infty
\frac{(a_1)_k\cdots (a_n)_k}{(b_1)_k\cdots (b_m)_k}
\frac{z^k}{k!}
\\
&=
\sum_{k=0}^\infty
\frac{(a_1)_k\cdots (a_n)_k}{(b_1)_k\cdots\widehat{(b_j)_k}\cdots (b_m)_k}
\biggl(\frac{k}{(b_j)_k}+\frac{b_j-1}{(b_j)_k}\biggr)
\frac{z^k}{k!}
\\
&=
\sum_{k=0}^\infty
\frac{(a_1)_k\cdots (a_n)_k}{(b_1)_k\cdots \widehat{(b_j)_k}\cdots (b_m)_k}
\frac{b_j+k-1}{(b_j)_k}
\frac{z^k}{k!}
\\
&=
\sum_{k=0}^\infty
\frac{(a_1)_k\cdots (a_n)_k}{(b_1)_k\cdots \widehat{(b_j)_k}\cdots (b_m)_k}
\frac{b_j-1}{(b_j-1)_k}
\frac{z^k}{k!}
\\
&=
(b_j-1)
\sum_{k=0}^\infty
\frac{(a_1)_k\cdots (a_n)_k}{(b_1)_k\cdots(b_j-1)_k\cdots (b_m)_k}
\frac{z^k}{k!}
\\
&=(b_j-1)
\,
\mathstrut_nF_m\biggl(
\begin{matrix}a_1,\dots,a_n\\
b_1,\dots,b_j-1,\dots,b_m
\end{matrix}
;z
\biggr).
\end{align*}
Durch Anwendung aller Operatoren $D_{b_j-1}$ kann also jeder $b$-Parameter
dekrementiert werden, es gilt also
\[
D_{b_1-1}D_{b_2-1}\cdots D_{b_m-1}w
=
(b_1-1)(b_2-1)\cdots(b_m-1) \,
\mathstrut_nF_m\biggl(
\begin{matrix}
a_1,\dots,a_n\\
b_1-1,\dots,b_m-1
\end{matrix}
;z
\biggr).
\]

\subsubsection{Die Differentialgleichung}
Aus den Operatoren $D_{a_i}$ und $D_{b_j-1}$ kann jetzt eine
Differentialgleichung für die Funktion $w(z)$ konstruieren.
Durch Anwendung von aller Operatoren $D_{b_i-1}$ werden die
$b$-Parameter dekrementiert und die Faktoren $(b_i-1)$ kommen hinzu.
Leitet man dies ab, werden alle Parameter inkrementiert:
\begin{align*}
\frac{d}{dz}
\mathstrut_nF_m\biggl(
\begin{matrix}a_1,\dots,a_n\\
b_1,\dots,b_m\end{matrix};z
\biggr)
&=
\frac{a_1\cdots a_n}{(b_1-1)\cdots(b_m-1)}
(b_1-1)\cdots(b_m-1)
\,\mathstrut_nF_m\biggl(
\begin{matrix}a_1+1,\dots,a_n+1\\
b_1,\dots,b_m\end{matrix};z
\biggr)
\\
&=
a_1\dots a_n
\,\mathstrut_nF_m\biggl(
\begin{matrix}a_1+1,\dots,a_n+1\\
b_1,\dots,b_m\end{matrix};z
\biggr)
\end{align*}
Dies ist aber die gleiche Operation, wie alle Operatoren $D_{a_i}$ 
anzuwenden.
Es folgt daher die Differentialgleichung 
\[
D_{a_1}\cdots D_{a_n} w = \frac{d}{dz} D_{b_1-1}\cdots D_{b_m-1} w
\]
für die Funktion $w(z)$.

\begin{beispiel}
Im Spezialfall $\mathstrut_0F_0$ gibt es keine Operatoren $D_{a_i}$
oder $D_{b_j-1}$ anzuwenden, so dass nur die Differentialgleichung
\[
w=\frac{d}{dz}w
\]
stehen bleibt.
Dies ist natürlich die Differentialgleichung der Exponentialfunktion.
\end{beispiel}

%
% Differentialgleichung für 1F0
%
\subsubsection{Die Differentialgleichungen für $\mathstrut_1F_0$}
In diesen Fälle gibt es nur jeweils einen einzigen Operator
anzuwenden.
Wir betrachten zunächst den Fall $w(z) = \mathstrut_1F_0(\alpha; z)$
und finden direkt die Differentialgleichung
\begin{align*}
\biggl(z\frac{d}{dz}+\alpha\biggr)w
&=
\frac{d}{dz}w
\\
zw'+\alpha w
&=
w'
\\
(1-z)w'
&=
\alpha w.
\end{align*}

\begin{beispiel}
Wir bestimmen die Differentialgleichung für die als hypergeometrische
Reihe darstellbare Funktion
\[
f(x)
=
\sqrt{1+x} = \mathstrut_1F_0(-{\textstyle\frac12};-x).
\]
Zunächst erfüllt die hypergeometrische Funktion
$w(z)=\mathstrut_1F_0(-\frac12;z)$ die Differentialgleichung
\[
(1-z)w'(z) = -\frac12 w(z).
\]
Jetzt setzen wir $z=-x$ in die Funktion ein.
Wegen $f(x)=w(-x)$ folgt $f'(x)=-w'(-x)$
\[
-f'(x)(1+x) = -\frac12 f(x)
\qquad\Rightarrow\qquad
f'(x) = \frac{f(x)}{2(1+x)}.
\]
Tatsächlich ist die Ableitung der Wurzelfunktion $f(x)$
\[
\frac{d}{dx}f(x)
=
\frac{d}{dx}\sqrt{1+x}
=
\frac{1}{2\sqrt{1+x}}
=
\frac{\sqrt{1+x}}{2(1+x)}
=
\frac{f(x)}{2(1+x)},
\]
sie erfüllt also die genannte Differentialgleichung.
\end{beispiel}

%
% Differentialgleichung für 0F1
%
\subsubsection{Die Differentialgleichungen für $\mathstrut_0F_1$}
Für die Funktion $\mathstrut_0F_1$ setzen wir
$w(z)=\mathstrut_0F_1(;\beta;z)$.
In diesem Fall gibt es keine Operatoren $D_{a_i}$ anzuwenden, die
linke Seite der Differentialgleichung ist also einfach die Funktion $w$.
Für die rechte Seite ist der Operator $D_{\beta-1}$ anzuwenden, was auf
die Differentialgleichung
\begin{align*}
w
&=
\frac{d}{dz}
\biggl(z\frac{d}{dz}+\beta -1\biggr)w
\\
w
&=
\frac{d}{dz}(zw'+\beta w - w)
\\
w
&=
zw''+w'+\beta w' -w'
\\
0
&=
zw''+\beta w' - w
\end{align*}
führt.

\begin{beispiel}
Die Kosinus-Funktion kann durch die hypergeometrische Funktion
$\mathstrut_0F_1$ ausgedrückt werden.
Wir schreiben 
\[
w(z)
=
\mathstrut_0F_1\biggl(
\begin{matrix}\text{---}\\\frac12\end{matrix}
;z\biggr),
\]
$w(z)$ erfüllt die Differentialgleichung
\[
zw''(z) +w'(z) -\frac{3}{2} w(z) = 0.
\]
Die Kosinus-Funktion als Funktion von $w(z)$ ist
\[
f(x)
=
\cos x = \mathstrut_0F_1\biggl(;\frac12;-\frac{x^2}4\biggr)
=
w\biggl(-\frac{x^2}4\biggr),
\]
es muss also $z=-x^2/4$ gesetzt werden.
Wir müssen die Ableitungen von $w$ durch die Ableitungen von $f$
ausdrücken.
Die Ableitungen sind
\begin{align*}
f'(x)
&=
-\frac{x}{2}
w'\biggl(-\frac{x^2}4\biggr)
&&\Rightarrow&
w'\biggl(-\frac{x^2}4\biggr)
&=
-\frac{2}{x}f'(x)
\\
f''(x)
&=
\frac{x^2}{4}w''\biggl(-\frac{x^2}4\biggr)
-\frac12w'\biggl(-\frac{x^2}4\biggr)
&&\Rightarrow&
w''\biggl(-\frac{x^2}4\biggr)
&=
\frac{4}{x^2}f''(x)
+\frac{2}{x^2}w'\biggl(-\frac{x^2}4\biggr)
\\
&&&&
&=
\frac{4}{x^2}f''(x)
-\frac{4}{x^3}f'(x).
\end{align*}
Einsetzen in die Differentialgleichung von $w(z)$ ergibt
\begin{align*}
0=
zw''+\beta w'-w
&=
-\frac{x^2}4
\biggl(
\frac{4}{x^2}f''(x)-\frac{4}{x^3}f'(x)
\biggr)
+\frac12\biggl(
-\frac2xf'(x)
\biggr)
-f(x)
\\
&=
-f''(x)
-f(x),
\end{align*}
was gleichbedeutend ist mit der Differentialgleichung $f''=-f$, die
tatsächlich die Kosinus-Funktion als Lösung hat.
\end{beispiel}

%
% Die Differentialgleichung für 1F1
%
\subsubsection{Die Differentialgleichung für $\mathstrut_1F_1$}
Wir setzen wieder $w(z) = \mathstrut_1F_1(\alpha;\beta;z)$.
Es sind die Operatoren $D_\alpha$ und $D_{\beta-1}$ anzuwenden.
Es ergibt sich die Differentialgleichung
\begin{align*}
\biggl(z\frac{d}{dz}+\alpha\biggr)w
&=
\frac{d}{dz}\biggl(z\frac{d}{dz} +\beta-1\biggr)w
\\
zw'+\alpha w
&=
\frac{d}{dz}
(zw'+\beta w - w)
\\
zw'+\alpha w
&=
zw'' +w'+\beta w' - w'
\\
0
&=
zw'' + (\beta - z)w' - \alpha w.
\end{align*}

%
% Die hypergeometrische Differentialgleichung für 2F1
%
\subsubsection{Die Differentialgleichung für $\mathstrut_2F_1$}
Für die hypergeometrische Funktion $\mathstrut_2F_1(\alpha,\beta;\gamma;z)$
ist die Difrentialgleichung von der Form
\[
\biggl(z\frac{d}{dz} + \alpha\biggr)
\biggl(z\frac{d}{dz} + \beta\biggr)w
=
\frac{d}{dz}
\biggl(z\frac{d}{dz}+\gamma -1\biggr)
w.
\]
Durchführen der Ableitungen auf beiden Seiten ergibt für die linke Seite
\begin{align*}
\biggl(z\frac{d}{dz} + \alpha\biggr)
\biggl(z\frac{d}{dz} + \beta\biggr)w
&=
\biggl(z\frac{d}{dz} + \alpha\biggr)
(zw'+\beta w)
\\
&=
z^2w'' + zw' + \beta zw' + \alpha(zw'+\beta w)
\\
&=
z^2w'' + (1+\alpha+\beta )zw' + \beta\alpha w
\intertext{und die rechte Seite}
\frac{d}{dz}\biggl(z\frac{d}{dz}+\gamma-1\biggr)w
&=
\frac{d}{dz}(zw'+\gamma w-w)
\\
&=
zw''+w'+\gamma w' - w'
\\
&= 
zw'' +\gamma w'.
\end{align*}
Durch Gleichsetzen ergibt sich jetzt
\begin{align*}
z^2w'' + (1+\alpha+\beta )zw' + \alpha\beta w
&=
zw'' +\gamma w'
\\
0
&=
z(1-z)w''
+
(\gamma-z(1+\alpha+\beta))w'
-
\alpha\beta
w
\end{align*}
Dies ist die früher definierte hypergeometrische Differentialgleichung.

\subsubsection{Differentialgleichungen für $w(x^2)$ und $xw(x^2)}


%
%
%
\subsubsection{Hypergeometrische Funktionen von $x^2$ und $x^3$}
Die hypergeometrischen Funktionen $w(z)$ als Lösungen der hypergeometrischen
Differentialgleichungen sind Potenzreihen, in denen kein Koeffizient
verschwindet, sofern die Lösung nicht ein Polynom ist.
Die trigonometrischen Funktionen sind nicht von dieser Art.
Sie lassen sich als Funktionen von $x^2$ schreiben.
Wir untersuchen in diesem Abschnitt, wie sich eine Differentialgleichung
von $y(x) = x^lw(tx^k)$ aus der Differentialgleichung für $w(z)$ gewinnen
lässt.



