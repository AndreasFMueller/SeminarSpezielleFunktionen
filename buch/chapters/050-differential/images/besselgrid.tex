%
% besselgrid.tex -- Indexmenge für die Herleitung der erzeugenden Funktion
%
% (c) 2021 Prof Dr Andreas Müller, OST Ostschweizer Fachhochschule
%
\documentclass[tikz]{standalone}
\usepackage{amsmath}
\usepackage{times}
\usepackage{txfonts}
\usepackage{pgfplots}
\usepackage{csvsimple}
\usetikzlibrary{arrows,intersections,math}
\begin{document}
\def\skala{1}
\def\r{0.2}
\def\hintergrund{
	\fill[color=gray!40,opacity=0.5]
		(-6.3,5.8)
		-- (-0.1,-0.4) 
		-- (6.5,-0.4) -- (6.5,-0.8) -- (-6.3,-0.8) -- cycle;
}
\def\nachse#1{
	\draw[->,color=#1] (-6.3,0) -- (6.6,0) coordinate[label={$n$}];
}
\def\kachse{
	\draw[->] (0,-0.3) -- (0,6.6) coordinate[label={right:$k$}];
}
\def\machse{
	\draw[->] (-0.3,-0.3) -- (6.3,6.3) coordinate[label={above right:$m$}];
}
\def\achsen{
	\draw[->] (0,-0.3) -- (0,6.6) coordinate[label={right:$k$}];
	\draw[->] (-6.3,0) -- (6.6,0) coordinate[label={$n$}];
	\draw[->] (0.3,-0.3) -- (-6.3,6.3) coordinate[label={above right:$m$}];
}
\def\nsumme{
	\foreach \n in {-6,...,6}{
		\fill[color=red!40,opacity=0.5] ({\n-\r},6.3)
			-- ({\n-\r},0) arc (-180:0:\r)
			-- ({\n+\r},6.3)
			-- cycle;
	}
}
\def\msumme{
	\foreach \k in {0,...,6}{
		\fill[color=blue!20] (6.3,{\k+\r})
			-- ({-\k},{\k+\r}) arc (90:270:\r)
			-- (6.3,{\k-\r})
			-- cycle;
	}
}
\def\punkte{
	\begin{scope}
	\clip (-6.2,-0.2) rectangle (6.2,6.2);

	\foreach \x in {1,...,6}{
		\foreach \y in {0,...,5}{
			\fill[color=gray] ({-\x-\y},\y) circle[radius=0.1];
		}
	}
	\foreach \x in {0,...,12}{
		\foreach \y in {0,...,6}{
			\fill[color=red] ({\x-\y},\y) circle[radius=0.1];
		}
	}
	\end{scope}

}
\begin{tikzpicture}[>=latex,thick,scale=\skala]

\begin{scope}
	\hintergrund
	\nsumme
	\punkte
	\nachse{black}
	\kachse
	\node at (-4.5,1.5) {$\Gamma(n+k+1)=\infty$};
\end{scope}

\begin{scope}[yshift=-7.8cm]
	\hintergrund
	\msumme
	\punkte
	\draw[->] (0.3,-0.3) -- (-6.4,6.4) coordinate[label={above right:$k$}];
	\draw[->] (-3.3,3) -- (6.6,3) coordinate[label={right:$m$}];
	\node at (-4.5,1.5) {$\Gamma(m+1)=\infty$};
\end{scope}

\end{tikzpicture}
\end{document}

