%
% beispiele.tex
%
% (c) 2021 Prof Dr Andreas Müller, OST Ostschweizer Fachhochschule
%
\section{Beispiele
\label{buch:differentialgleichungen:section:beispiele}}
\rhead{Beispiele}
Viele der bisher betrachteten speziellen Funktionen können 
durch gewöhnliche Differentialgleichungen charakterisiert werden,
als deren Lösungen sie auftreten.

\subsection{Potenzen und Wurzeln
\label{buch:differentialgleichungen:subsection:potenzen-und-wurzeln}}
Die Potenzfunktionen und die zugehörigen Wurzeln als die ältesten
speziellen Funktionen bieten bereits eine erste kleine Schwierigkeit.
Die Differentialgleichung, die man aus einem naiven Ansatz ableitet,
ist singulär.

\subsubsection{Differentialgleichung in $(0,\infty)$}
Die Ableitung einer Potenzfunktion $x\mapsto y(x)=x^\alpha$ ist
\[
y'(x) =
\begin{cases}
\alpha x^{\alpha-1} &\qquad \alpha\ne -1\\
\log x&\qquad\text{sonst}
\end{cases}
\]
Im Folgenden wollen wir uns auf den Fall $\alpha\ne -1$ konzentrieren.
Die Ableitungsoperation läuft in diesem Fall darauf hinaus, dass der
Grad um $1$ reduziert wird.
Dies könnte man mit einem Faktor $x$ komponsieren.
Wir fragen daher nach der allgmeinen Lösung der linearen
Differentialgleichung der Form
\begin{equation}
xy' = \alpha y.
\label{buch:differentialgleichungen:eqn:wurzeldgl}
\end{equation}
Diese Gleichung ist separierbar, die Separation von $x$ und $y$ liefert
die Integrale
\[
\int \frac{dy}{y} = \alpha \int \frac{dx}{x} + C.
\]
Die Durchführunge der Integration liefert 
\[
\log |y| = \alpha \log|x| + C.
\]
Wendet man die Exponentialfunktion an, erhält man wieder
\[
y = Dx^\alpha,\quad D=\exp C.
\]

Die Differentialgleichung~\eqref{buch:differentialgleichungen:eqn:wurzeldgl}
hat aber eine schwerwiegenden Mangel.
Ihre explizite Form lautet
\begin{equation}
y' = \frac{\alpha}{x}\cdot y.
\label{buch:differentialgleichungen:eqn:wurzelsing}
\end{equation}
Dies ist zwar durchaus eine lineare Differentialgleichung erster Ordnung,
aber der Koeffiziente $\alpha/x$ wächst für $x\to 0$ über alle Grenzen.
Man kann daher den Wert der Potenzfunktion im Nullpunkt gar nicht aus der
Differentialgleichung erhalten, es ist dazu mindestens noch ein Grenzübergang
$x\to 0+$ nötig.

\subsubsection{Differentialgleichung in der Nähe von $x=1$}
Um dem Problem des singulären Koeffizienten der
Differentialgleichung~\eqref{buch:differentialgleichungen:eqn:wurzelsing}
aus dem Weg zu gehen, verwenden wir die Variable $t$ mit $x=1+t$ und
versuchen eine Differentialgleichung für die Potenzfunktion
$(1+t)^\alpha$ zu finden.
Es gilt natürlich
\begin{equation}
\frac{d}{dt} (1+t)^\alpha
=
\alpha (1+t)^{\alpha-1}
\qquad\Rightarrow\qquad
(1+t) \dot{y} = \alpha y.
\label{buch:differentialgleichungen:eqn:wurzeldgl1}
\end{equation}
Diese Differentialgleichung kann natürlich auch wieder mit Separation
gelöst werden, es ist
\begin{equation}
\int
\frac{dy}{y} 
=
\alpha
\int
\frac{dt}{1+t}
+
C
\qquad\Rightarrow\qquad
\log|y| = \alpha \log|1+t| + C
\label{buch:differentialgleichungen:eqn:wurzeldgl1loesung}
\end{equation}
und daraus die Potenzfunktion
\[
y=D(1+t)^\alpha
\]
wie vorhin.
Der Vorteil der
Form~\eqref{buch:differentialgleichungen:eqn:wurzeldgl1}
wird sich später bei dem Versuch zeigen, die Fuktion $y(t)$
direkt als Potenzreihenlösung der Differentialgleichung zu finden.


\subsection{Exponentialfunktion und ihre Varianten
\label{buch:differentialgleichungen:subsection:exponentialfunktion}}

\subsubsection{Lineare Differentialgleichung erster Ordnung mit konstanten Koeffizienten}

\subsubsection{Lineare Differentialgleichungen zweiter Ordnung mit konstanten Koeffizienten}
