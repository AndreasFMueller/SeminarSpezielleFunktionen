%
% beispiele.tex
%
% (c) 2021 Prof Dr Andreas Müller, OST Ostschweizer Fachhochschule
%
\section{Beispiele
\label{buch:differentialgleichungen:section:beispiele}}
\rhead{Beispiele}
Viele der bisher betrachteten speziellen Funktionen können 
durch gewöhnliche Differentialgleichungen charakterisiert werden,
als deren Lösungen sie auftreten.

\subsection{Potenzen und Wurzeln
\label{buch:differentialgleichungen:subsection:potenzen-und-wurzeln}}
Die Potenzfunktionen und die zugehörigen Wurzeln als die ältesten
speziellen Funktionen bieten bereits eine erste kleine Schwierigkeit.
Die Differentialgleichung, die man aus einem naiven Ansatz ableitet,
ist singulär.

\subsubsection{Differentialgleichung in $(0,\infty)$}
Die Ableitung einer Potenzfunktion $x\mapsto y(x)=x^\alpha$ ist
\[
y'(x) =
\begin{cases}
\alpha x^{\alpha-1} &\qquad \alpha\ne -1\\
\log x&\qquad\text{sonst}
\end{cases}
\]
Im Folgenden wollen wir uns auf den Fall $\alpha\ne -1$ konzentrieren.
Die Ableitungsoperation läuft in diesem Fall darauf hinaus, dass der
Grad um $1$ reduziert wird.
Dies könnte man mit einem Faktor $x$ komponsieren.
Wir fragen daher nach der allgmeinen Lösung der linearen
Differentialgleichung der Form
\begin{equation}
xy' = \alpha y.
\label{buch:differentialgleichungen:eqn:wurzeldgl}
\end{equation}
Diese Gleichung ist separierbar, die Separation von $x$ und $y$ liefert
die Integrale
\[
\int \frac{dy}{y} = \alpha \int \frac{dx}{x} + C.
\]
Die Durchführunge der Integration liefert 
\[
\log |y| = \alpha \log|x| + C.
\]
Wendet man die Exponentialfunktion an, erhält man wieder
\[
y = Dx^\alpha,\quad D=\exp C.
\]

Die Differentialgleichung~\eqref{buch:differentialgleichungen:eqn:wurzeldgl}
hat aber eine schwerwiegenden Mangel.
Ihre explizite Form lautet
\begin{equation}
y' = \frac{\alpha}{x}\cdot y.
\label{buch:differentialgleichungen:eqn:wurzelsing}
\end{equation}
Dies ist zwar durchaus eine lineare Differentialgleichung erster Ordnung,
aber der Koeffiziente $\alpha/x$ wächst für $x\to 0$ über alle Grenzen.
Man kann daher den Wert der Potenzfunktion im Nullpunkt gar nicht aus der
Differentialgleichung erhalten, es ist dazu mindestens noch ein Grenzübergang
$x\to 0+$ nötig.

\subsubsection{Differentialgleichung in der Nähe von $x=1$}
Um dem Problem des singulären Koeffizienten der
Differentialgleichung~\eqref{buch:differentialgleichungen:eqn:wurzelsing}
aus dem Weg zu gehen, verwenden wir die Variable $t$ mit $x=1+t$ und
versuchen eine Differentialgleichung für die Potenzfunktion
$(1+t)^\alpha$ zu finden.
Es gilt natürlich
\begin{equation}
\frac{d}{dt} (1+t)^\alpha
=
\alpha (1+t)^{\alpha-1}
\qquad\Rightarrow\qquad
(1+t) \dot{y} = \alpha y.
\label{buch:differentialgleichungen:eqn:wurzeldgl1}
\end{equation}
Diese Differentialgleichung kann natürlich auch wieder mit Separation
gelöst werden, es ist
\begin{equation}
\int
\frac{dy}{y} 
=
\alpha
\int
\frac{dt}{1+t}
+
C
\qquad\Rightarrow\qquad
\log|y| = \alpha \log|1+t| + C
\label{buch:differentialgleichungen:eqn:wurzeldgl1loesung}
\end{equation}
und daraus die Potenzfunktion
\[
y=D(1+t)^\alpha
\]
wie vorhin.
Der Vorteil der
Form~\eqref{buch:differentialgleichungen:eqn:wurzeldgl1}
wird sich später bei dem Versuch zeigen, die Fuktion $y(t)$
direkt als Potenzreihenlösung der Differentialgleichung zu finden.


\subsection{Exponentialfunktion und ihre Varianten
\label{buch:differentialgleichungen:subsection:exponentialfunktion}}
In Kapitel~\ref{buch:chapter:exponential} wurde die Exponentialfunktion
auf algebraische Weise definiert, die Berechnung wurde ermöglicht
mit Hilfe von Grenzwerten und Potenzreihen.
Dabei blieb die Ableitung der Exponentialfunktion aussen vor.
Die Exponentialfunktion lässt sich aber natürlich auch über
Differentialgleichungen charakterisieren.

\subsubsection{Die Ableitung der Exponentialfunktion}
Aus der Potenzreihendarstellung
\[
\exp(x)
=
\sum_{k=0}^\infty \frac{x^k}{k!}
\]
folgt sofort, dass die Ableitung
\[
\frac{d}{dx}\exp(x)
=
\frac{d}{dx}
\sum_{k=0}^\infty
\frac{x^k}{k!}
=
\sum_{k=1}^\infty \frac{kx^{k-1}}{k!}
=
\sum_{k=1}^\infty{x^{k-1}}{(k-1)!}
=
\sum_{l=0}^\infty \frac{x^l}{l!}
=
\exp(x),
\]
wobei $l=k-1$ gesetzt wurde.
Die Exponentialfunktion ist also ihre eigene Ableitung.

\subsubsection{Lineare Differentialgleichung mit konstanten Koeffizienten}
Mit der Exponentialfunktion lassen sich beliebige homogene lineare
Differentialgleichungen mit konstanten Koeffizienten lösen.
Sei die Differentialgleichung
\[
y^{(n)} + a_{n-1}y^{(n-1)} + \dots + a_2y'' + a_1y' + a_0y = 0
\]
gegeben.
Mit dem Ansatz $y(x)=e^{\lambda x}$ ergibt sich die Gleichung
\[
\lambda^n e^{\lambda x}
+
a_{n-1}\lambda^{n-1} e^{\lambda x}
+
\dots
+
a_2\lambda^2e^{\lambda x}
+
a_1\lambda e^{\lambda x}
+
a_0e^{\lambda x}
=
(\lambda^n + a_{n-1}\lambda^{n-1} + \dots + a_2\lambda^2 + a_1\lambda + a_0)
e^{\lambda x}
=
0.
\]
Da $e^{\lambda x}\ne 0$ ist, kann $y(x)$ nur dann eine Lösung sein, wenn
$\lambda$ eine Nullstelle des {\em charakteristischen Polynoms}
\[
p(\lambda)
=
\lambda^n
+
a_{n-1}\lambda^{n-1}
+
\dots
+
a_2\lambda^2
+
a_1\lambda
+
a_0
\]
ist.

\subsubsection{Ableitungen der trigonometrische Funktionen}
Die Drehmatrix 
\[
D_{\omega t}
=
\begin{pmatrix}
\cos\omega t&-\sin\omega t\\
\sin\omega t& \cos\omega t
\end{pmatrix}
\]
bschreibt eine Drehung der Ebene mit der Winkelgeschwindigkeit 
$\omega$.
Der Punkt $(r,0)$ beschreibt unter dieser Drehung eine Kreisbahn
parametrisiert durch 
\[
t \mapsto \gamma(t)=(r\cos\omega t,r\sin\omega t).
\]
Der Geschwindigkeitsvektor zur Zeit $t$ ist natürlich
\[
\vec{v}(0)
=
\begin{pmatrix}
0\\
r\omega
\end{pmatrix},
\]
zu einer späteren Zeit $t$  ist er
\[
\vec{v}(t)
=
D_{\omega t} \vec{v}(0)
=
\begin{pmatrix}
\cos\omega t&-\sin\omega t\\
\sin\omega t& \cos\omega t
\end{pmatrix}
\begin{pmatrix}
0\\r\omega
\end{pmatrix}
=
r
\begin{pmatrix}
-\omega\sin\omega t\\
 \omega\cos\omega t
\end{pmatrix}
\]
Gleichzeitig ist $\vec{v}(t)$ natürlich auch die Ableitung  von $\gamma(t)$,
also
\[
\dot{\gamma}(t)
=
r
\frac{d}{dt}
\begin{pmatrix}
\cos\omega t\\
\sin\omega t
\end{pmatrix}
=
r
\begin{pmatrix}
-\omega\sin\omega t\\
 \omega\cos\omega t
\end{pmatrix}
\qquad\Rightarrow\qquad
\left\{
\begin{aligned}
\frac{d}{dt} \cos\omega t &= -\omega \sin\omega t\\
\frac{d}{dt} \sin\omega t &= \phantom{-} \omega \cos\omega t
\end{aligned}
\right.
\]
Dies bedeutet, dass die Ableitungen der trigonometrischen Funktionen
\begin{equation}
\begin{aligned}
\frac{d}{dt} \sin t&=\phantom{-}\cos t\\
\frac{d}{dt} \cos t&=-\sin t
\end{aligned}
\label{buch:differentialgleichungen:trigo:ableitungen}
\end{equation}

\subsubsection{Differentialgleichung für trigonometrische Funktionen}
Aus den Ableitungen~\eqref{buch:differentialgleichungen:trigo:ableitungen}
folgt, dass die trigonometrischen Funktionen $\sin t $ und $\cos t$
Lösungen der Differentialgleichung $y''=-y$ sind.
Das zugehörige charakteristische Polynom ist 
\[
\lambda^2+1=0
\qquad\Rightarrow\qquad
\lambda^2=-1
\qquad\Rightarrow\qquad
\lambda=\pm i.
\]
Daraus ergeben sich die Lösungen
\[
y_{\pm}(t) = e^{\pm i t}.
\]
Da eine Differentialgleichung zweiter Ordnung nur zwei linear unabhängige
Lösungen haben kann, müssen sich $\sin t$ und $\cos t$ durch
$e^{\pm it}$ ausdrücken lassen.

Die Kosinus-Funktion zeichnet sich dadurch aus, dass $\cos 0=1$ und
$\cos' 0=0$ ist.
Gesucht ist also eine Lösung der Linearkombination der Lösungen
$y_{\pm}$ der Differentialgleichung mit diesen Anfangswerten.
Zunächst halten wir fest, dass $y_{\pm}(0)=e^{\pm i\cdot 0}=1$.
Für die Ableitungen von $y^{\pm it}$ gilt
\[
\frac{d}{dt}
=
e^{\pm i t}
=
\pm ie^{\pm i t}
\qquad\Rightarrow\qquad
\frac{d}{dt}y_{\pm}(0) = \pm i.
\]
Die Linearkombination $Ay_+(t)+By_-(t)$ hat die Anfangswerte
\begin{align*}
Ay_+(0)+By_-(0)&=A+B\\
Ay'_+(0)+By'_-(0)&=Ai-Bi.
\end{align*}
Damit die Linearkombination $\cos t=Ay_+(t)+By_-(t)$ ist, müssen
$A$ und $B$ Lösungen des Gleichungssystems
\[
\begin{linsys}{2}
 A&+& B&=&1\\
iA&-&iB&=&0
\end{linsys}
\qquad\Rightarrow\qquad
\begin{linsys}{2}
 A&+& B&=&1\\
 A&-& B&=&0
\end{linsys}
\]
Die Summe und Differenz der beiden Gleichungen führt auf
\[
\begin{aligned}
2A&=1\\
2B&=1
\end{aligned}
\qquad\Rightarrow\qquad
\begin{aligned}
A=&\textstyle\frac12\\
B=&\textstyle\frac12
\end{aligned}
\qquad\Rightarrow\qquad
\cos t = \frac{e^{it}+e^{-it}}{2}.
\]

Andererseits hat die Sinus-Funktion die Anfangswerte $\sin 0=0$ und
$\sin' 0=1$, dies führt auf das Gleichungssystem
\[
\begin{linsys}{2}
 A&+& B&=&0\\
iA&-&iB&=&1
\end{linsys}
\qquad\Rightarrow\qquad
\begin{linsys}{2}
 A&+& B&=&0\\
 A&-& B&=&\frac{1}i
\end{linsys}
\]
Diesemal führen
Summe und Differenz der beiden Gleichungen auf
\[
\begin{aligned}
2A&=\frac{1}i\\
2B&=-\frac{1}i
\end{aligned}
\qquad\Rightarrow\qquad
\begin{aligned}
A=&\textstyle\phantom{-}\frac1{2i}\\
B=&\textstyle{-\frac1{2i}}
\end{aligned}
\qquad\Rightarrow\qquad
\sin t = \frac{e^{it}-e^{-it}}{2i}.
\]

\subsubsection{Potenzreihen für $\sin t$ und $\cos t$}
Aus der Potenzreihe der Exponentialfunktion kann man jetzt auch
Potenzreihen für $\sin t$ und $\cos t$ ableiten.
Zunächst ist
\begin{align*}
y_+(t)
&=
1 + it - \frac{t^2}{2!} - \frac{it^3}{3!} + \frac{t^4}{4!} + \frac{it^5}{5!}
- \frac{t^6}{6!} - \frac{it^7}{7!} + \dots
\\
y_+(t)
&=
1 - it - \frac{t^2}{2!} + \frac{it^3}{3!} + \frac{t^4}{4!} - \frac{it^5}{5!}
- \frac{t^6}{6!} + \frac{it^7}{7!} + \dots
\intertext{Die trigonometrischen Funktionen können daraus linear kombiniert
werden, zum Beispiel ist die Kosinus-Funktion}
\cos t
=
\frac{y_+(t)+y_-(t)}{2}
&=
1-\frac{t^2}{2!} + \frac{t^4}{4!} -\frac{t^6}{6!}+\dots
=
\sum_{k=0}^\infty (-1)^k\frac{t^{2k}}{(2k)!}.
\intertext{Auf die gleiche Art findet man für die Sinus-Funktion}
\sin t
=
\frac{y_+(t)-y_-(t)}{2i}
&=
t-\frac{t^3}{3!} + \frac{t^5}{5!} - \frac{t^7}{t!} + \dots
=
\sum_{k=0}^\infty (-1)^k \frac{t^{2k+1}}{(2k+1)!}.
\end{align*}

\subsubsection{Hyperbolische Funktionen}






