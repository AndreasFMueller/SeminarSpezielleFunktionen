Lösen Sie die Differentialgleichung $y''+y=0$ der trigonometrischen
Funktionen mit Hilfe eines Potenzreihenansatzes.
\index{Potenzreihenansatz}%
Finden Sie Lösungen $s(t)$ mit $s(0)=0$ und $s'(0)=1$ und
$c(t)$ mit $c(0)=1$ und $c'(0)=0$.

\begin{loesung}
Der Potenzreihenansatz
\begin{align*}
y(x)
&=
\sum_{k=0}^\infty a_kx^k
\intertext{hat die Ableitungen}
y'(x)&=\sum_{k=1}^\infty ka_kx^{k-1}
&&\text{und}
&
y''(x)&=\sum_{k=2}^\infty k(k-1)a_kx^{k-2}
=
\sum_{k=0}^\infty (k+1)(k+2)a_{k+2}x^k.
\end{align*}
Eingesetzt in die Differentialgleichung ergibt sich
\[
y''(x) + y(x)
=
\sum_{k=0}^\infty a_kx^k
+
\sum_{k=0}^\infty (k+1)(k+2)a_{k+2}x^k
=
\sum_{k=0}^\infty \bigl(a_k + (k+1)(k+2)a_{k+2}\bigr)x^k.
\]
Koeffizientenvergleich ergibt die Rekursionsformel
\[
a_{k+2} = -\frac{1}{(k+1)(k+2)}a_k
\]
für die Koeffizienten $a_k$.
Die Koeffizienten $a_0$ und $a_1$ sind bestimmt durch die Anfangsbedingungen
festgelegt.

Für die Funktion $s(t)$ ist $a_0=s(0)=0$ und $s'(0)=a_1=1$, daraus ergeben sich
die Koeffizienten
\begin{align*}
a_0&=0\\
a_1&=1\\
a_2&=-\frac{1}{1\cdot 2}a_0=0\\
a_3&=-\frac{1}{2\cdot 3}a_1=-\frac{1}{3!}\\
a_4&=-\frac{1}{3\cdot 4}a_2=0\\
a_5&=-\frac{1}{4\cdot 5}a_1 = \frac{1}{3!\cdot4\cdot 5}=\frac{1}{5!}\\
   &\vdots
\end{align*}
also
\[
s(t) = 1 - \frac{t^3}{3!} + \frac{t^5}{5!} - \dots
=
\sin t.
\]

Für die Funktion $c(t)$ ist $a_0=c(0)=1$ und $a_1=c'(0)=0$, daraus ergeben
sich die Koeffizienten
\begin{align*}
a_0&=1\\
a_1&=0\\
a_2&=-\frac{1}{1\cdot 2}a_1 = -\frac{1}{2!}\\
a_3&=-\frac{1}{2\cdot 3}a_2 = 0\\
a_4&=-\frac{1}{3\cdot 4}a_3 = \frac{1}{2!\cdot 3 \cdot }=\frac{1}{4!}\\
a_5&=-\frac{1}{4\cdot 5}a_4 = 0\\
a_6&=-\frac{1}{5\cdot 6}a_5 = -\frac{1}{4!\cdot 5\cdot 6} = -\frac{1}{6!} \\
   &\vdots
\end{align*}
und damit
\[
c(t)
=
1-\frac{t^2}{2!} + \frac{t^4}{4!} - \frac{t^6}{6!} + \dots
=
\cos t.
\qedhere
\]
\end{loesung}



