Finden Sie eine Lösung der Airy Differentialgleichung
\index{Airy-Differentialgleichung}%
\[
y''-xy=0
\]
mit Anfangsbedingungen $y(0)=a$ und $y'(0)=b$.

\begin{loesung}
An der Stelle $x=0$ folgt aus der Differentialgleichung, dass $y''(0)=0$
gelten muss.
In einem Potenzreihenansatz der Form
\begin{align*}
y(x)
&=
\sum_{k=0}^\infty a_kx^k
&&\Rightarrow&
y'(x)
&=
\sum_{k=1}^\infty a_kx^{k-1}
\\
&&&&
y''(x)
&=
\sum_{k=2}^\infty k(k-1)a_kx^{k-2}
\end{align*}
kann man daher $a_2=0$ setzen und damit die Summation in der
Reihenentwicklung für $y''(x)$ erst bei $k=3$ beginnen.

Setzt man den Ansatz in die Differentialgleichung ein, erhält man
\begin{align*}
0
&=
y''(x)-xy(x)
\\
&=
\sum_{k=3}^\infty k(k-1)a_kx^{k-2}
-
\sum_{k=0}^\infty a_kx^{k+1}
\\
&=
\sum_{k=0}^\infty (k+3)(k+2)a_{k+3}x^{k+1}
-
\sum_{k=0}^\infty a_{k}x^{k+1}
\\
&=
\sum_{k=0}^\infty \bigl((k+3)(k+2)a_{k+3}-a_{k}\bigr)x^{k+1}.
\end{align*}
Koeffizientenvergleich liefert jetzt die Rekursionsbeziehungen
\[
a_{k+3} = \frac1{(k+3)(k+2)} {a_k}.
\]
Da $a_2=0$ ist folgt daraus auch, dass $a_5=a_8=a_{11}=\dots=0$ ist.

Aus den Anfangsbedingungen liest man ab dass $a_0=a$ und $a_1=b$, daraus
kann man jetzt die Lösung konstruieren, es ist
\[
y(x)
=
a\biggl(1+\frac{1}{2\cdot 3}x^3 + \frac{1}{2\cdot3\cdot5\cdot 6}x^6 + \dots\biggr)
+
b\biggl(x+\frac{1}{3\cdot 4}x^4 + \frac{1}{3\cdot 4\cdot 6\cdot 7}x^7+\dots\biggr).
\qedhere
\]
\end{loesung}
