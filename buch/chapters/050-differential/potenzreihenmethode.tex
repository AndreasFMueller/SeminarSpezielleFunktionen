%
% potenzreihenmethode.tex
%
% (c) 2021 Prof Dr Andreas Müller, OST Ostschweizer Fachhochschule
%
\section{Potenzreihenmethode
\label{buch:differentialgleichungen:section:potenzreihenmethode}}
\rhead{Potenzreihenmethode}
Die Potenzreihenmethode versucht die Lösung einer gewöhnlichen
Differentialgleichung als Potenzreihe um die Anfangsbedingung zu
entwickeln.
Wir gehen in diesem Abschnitt von einer Differentialgleichung der
Form
\begin{equation}
b_n(x)y^{(n)}(x)
+
b_{n-1}(x)y^{(n-1)}(x)
+
\dots
+
b_1(x)y'(x)
+
b_0(x)y(x)
=
f(x)
\label{buch:differentialgleichungen:eqn:potenzreihendgl}
\end{equation}
mit der Randbedingung $y(0)=y_0$ aus.
Schon im einfachsten Fall einer homogenen Differentialgleichung erster
Ordnung ergibt sich die Beziehung
\[
b_1(x) y'(x) = b_0(x)y(x),
\]
wobei wir uns $y(x)$ und damit auch $y'(x)$ als Potenzreihe vorstellen.
Insbesondere ist 
\[
\frac{b_1(x)}{b_0(x)} = \frac{y(x)}{y'(x)}
\]
ein Quotient von Potenzreihen, den man natürlich wieder als 
Potenzreihe schreiben kann.
Da es nur auf den Quotienten ankommt, kann man sich auf den Fall
beschränken, dass die Koeffizienten Potenzreihen sind.
Tatsächlich gilt der folgende sehr viel allgemeinere Satz von
Cauchy und Kowalevskaja:

\begin{satz}[Cauchy-Kowalevskaja]
Eine partielle Differentialgleichung der Ordnung $k$ für eine
Funktion $u(x_1,\dots,x_n,t)=u(x,t)$ 
in expliziter Form
\[
\frac{\partial^k}{\partial t^k}
=
G\biggl(x,t,
\frac{\partial^j\partial^\alpha}{\partial t^j\,\partial x^k}
\biggr)
\quad\text{mit $j<k$ und $|\alpha|+j\le k$}
\]
mit einer Funktion $G$, die analytisch ist in allen Variablen
und der Randbedingung
\[
\frac{\partial^j}{\partial t^j}u(x,0)
=
\varphi_j(x)\quad\text{für $k=0,\dots,k-1$}
\]
mit analytischen Funktion $\varphi_j$ hat eine in einer Umgebung von 
$t=0$ eindeutige analytische Lösung.
\end{satz}

Im folgenden werden wir daher weitere einschränkende Annahmen über
die Koeffizienten $b_k(x)$ machen.

%
% Potenzreihenansatz und Koeffizientenvergleich
%
\subsection{Potenzreihenansatz und Koeffizientenvergleich}
In Abschnitt~\ref{buch:differentialgleichungen:section:beispiele}
wurde von einer grossen Zahl interessanter Funktionen gezeigt, dass
sie einerseits eine Lösungen einer Differentialgleichung sind, 
andererseits aber auch eine Potenzreihendarstellung sind.
Der Satz von Cauchy-Kowalevskaja hat gezeigt, dass dies das zu
erwartende Resultat ist.

Da wir bei einer linearen Differentialgleichung mit analytischen
Koeffizienten eine analytische Lösungsfunktion erwarten dürfen,
können wir auch versuchen, die Lösung der Differentialgleichung
von Anfang an als Potenzreihe
\[
y(x)
=
\sum_{k=0}^{\infty} a_kx^k
\]
anzusetzen.
Die Ableitungen von $y(x)$ sind gleichermassen als Potenzreihen
\begin{align*}
y'(x)
&=
\sum_{k=1}^\infty ka_kx^{k-1}
\\
y''(x)
&=
\sum_{k=2}^\infty k(k-1)a_kx^{k-2}
\\
&\vdots\\
y^{(n)}(x)
&=
\sum_{k=n}^\infty
k(k-1)\cdots(k-n+1) a_kx^{k-n}
=
\sum_{k=n}^\infty
(k-n+1)_n a_k x^{k-n}
=
\sum_{l=0}^\infty
(l+1)_na_{l+n}x^l
\end{align*}
darstellbar.

Der Ansatz für $y(x)$ und seine Ableitungen kann jetzt in die
Differentialgleichung eingsetzt werden.
Durch Ausmultiplizieren wird die Differentialgleichung zu
einer Identität von Potenzreihen.
Zwei Potenzreihen können nur dann übereinstimmen, wenn alle
Koeffizienten übereinstimmen.
So entsteht eine Menge von linearen Gleichungen für die
Koeffizienten $a_k$.
Die Koeffizienten $a_0$ bis $a_{n-1}$ werden gegeben durch die
Anfangswerte der Funktion und der ersten $n-1$ Ableitungen, die
ebenfalls nötig sind, um die Lösungsfunktion eindeutig festzulegen.
Durch Lösen des linearen Gleichungssystems können jetzt die Koeffizienten
und damit die Lösung bestimmt werden.

Setzt man zum Beispiel voraus, dass $b_n(0)\ne 0$ ist, dann ist der
konstante Term
\begin{equation}
b_n(0) n! a_n + b_{n-1}(0) (n-1)! a_{n-1}
+ \dots +
b_2(0) 2! a_2 + b_1(0) a_1 + b_0(0) a_0 = 0.
\label{buch:differntialgleichungen:eqn:konstterm}
\end{equation}
Diese Gleichung ermöglicht, nach $a_n$ aufzulösen:
\[
a_n
=
-
\frac{1}{b_n(0)\,n!}\bigl(
b_{n-1}(0)\,(n-1)!\,a_{n-1} + \dots + 
b_2(0)\,2!\,a_2 + b_1(0)\, a_1 + b_0(0)\, a_0
\bigr).
\]
Falls jedoch der Koeffizient $b_n(x)$ eine Nullstelle bei $x=0$
hat, ist es mit Gleichung~\eqref{buch:differntialgleichungen:eqn:konstterm}
allein nicht möglich, $a_n$ zu bestimmen.

Ein besonders einfacher Fall ist jener, in dem alle Koeffizienten der
Differentialgleichung konstant sind. 
In diesem Fall führen die Koeffizienten von $x^k$ auf die Gleichung
\begin{equation}
b_n n! a_{n+k} + b_{n-1} (n-1)! a_{n-1+k}
+ \dots +
b_2 2! a_{2+k} + b_1 a_{1+k} + b_0 a_{k} = 0.
\label{buch:differntialgleichungen:eqn:kterm}
\end{equation}
für alle $k$.
Die Gleichungen sind also immer lösbar und ergeben
\[
a_{n+k}
=
-
\frac{1}{b_n\,n!}\bigl(
b_{n-1}\,(n-1)!\,a_{n-1+k} + \dots + 
b_2\,2!\,a_{2+k} + b_1\, a_{1+k} + b_0\, a_k
\bigr).
\]



%
% Die Newtonsche Reihe
%
\subsection{Die Newtonsche Reihe
\label{buch:differentialgleichungen:subsection:newtonschereihe}}
Wir lösen die
Differentialgleichung~\eqref{buch:differentialgleichungen:eqn:wurzeldgl1}
mit der Anfangsbedingung $y(t)=1$ mit der Potenzreihenmethode.
Wir setzen daher für die Lösung die Potenzreihe an
\[
y(t)
=
a_0 + a_1t + a_2t^2 + a_3t^3 + \dots + a_kt^k + \dots
\]
Die Ableitung ist
\[
\dot{y}(t)
=
a_1 + 2a_2t + 3a_3t^2 + \dots  + ka_kt^{k-1} + \dots
\]
Einsetzen in die 
Differentialgleichung~\eqref{buch:differentialgleichungen:eqn:wurzeldgl1}
liefert
\begin{align*}
(1+t)
(
a_1 + 2a_2t + 3a_3t^2 + \dots  + ka_kt^{k-1} + \dots
)
&=
\alpha
(
a_0 + a_1t + a_2t^2 + a_3t^3 + \dots + a_kt^k + \dots
)
\\
a_1
+(a_1+2a_2)t
+(2a_2+3a_3)t^2
+(3a_3+4a_4)t^3
+\dots
&=
\alpha a_0 + \alpha a_1t + \alpha a_2t^2 + \alpha a_3t^3 + \dots
\end{align*}
Der Koeffizientenvergleich ergbiet die Gleichungen
\begin{align}
a_1&=\alpha a_0
\notag
\\
a_1+2a_2 &= \alpha a_1 &&\Rightarrow& 2a_2 &= (\alpha-1) a_1
\notag
\\
2a_2+3a_3 &= \alpha a_2&&\Rightarrow& 3a_3 &= (\alpha-2) a_2
\notag
\\
3a_3+4a_4 &= \alpha a_3&&\Rightarrow& 4a_4 &= (\alpha-3) a_3
\notag
\\
4a_4+5a_5 &= \alpha a_4&&\Rightarrow& 5a_5 &= (\alpha-4) a_4
\notag
\\
&\vdots
\notag
\\
&&&& \llap{$(k+1)a_{k+1}$} &= (\alpha-k) a_k
&&\Rightarrow&
a_{k+1} = \frac{\alpha-k}{k+1}a_k.
\label{buch:differentialgleichungen:eqn:newtonreiherekursion}
\end{align}
Die
Rekursionsformel~\eqref{buch:differentialgleichungen:eqn:newtonreiherekursion}
gilt auch im Fall $k=0$.
Aus der Anfangsbedingung folgt $a_0=1$.
Durch wiederholte Anwendung der 
Rekursionsformel~\eqref{buch:differentialgleichungen:eqn:newtonreiherekursion}
erhalten wir jetzt die Koeffizienten
\begin{align*}
a_0&=1
\\
a_1&=\alpha
\\
a_2&=\frac{\alpha(\alpha-1)}{1\cdot 2}
\\
a_3&=\frac{\alpha(\alpha-1)(\alpha-2)}{1\cdot 2\cdot 3}
\\
a_4&=\frac{\alpha(\alpha-1)(\alpha-2)(\alpha-3)}{1\cdot 2\cdot 3\cdot 4}
\\
&\;\vdots
\\
a_k&=\frac{\alpha(\alpha-1)(\alpha-2)\dots(\alpha-k+1)}{k!}.
\end{align*}
Für ganzzahliges $\alpha$ ist $a_k$ der Binomialkoeffizient
\[
a_k=\binom{\alpha}{k}
\]
und $a_k=0$ für $k>\alpha$.
Für nicht ganzzahliges $\alpha$ sind alle Koeffizienten $a_k\ne 0$.

Die Lösung der 
Differentialgleichung~\eqref{buch:differentialgleichungen:eqn:wurzeldgl1}
ist daher die Reihe
\begin{equation}
(1+t)^\alpha
=
\sum_{k=0}^\infty
\frac{\alpha(\alpha-1)\dots(\alpha-k+1)}{k!}\, t^k.
\label{buch:differentialgleichungen:eqn:newtonreihe}
\end{equation}
Für ganzzahliges $\alpha$ wird daraus die binomische Formel
\[
(1+t)^\alpha
=
\sum_{k=0}^\infty
\frac{\alpha(\alpha-1)\dots(\alpha-k+1)}{k!}\, t^k
=
\sum_{k=0}^\alpha \binom{\alpha}{k} t^k.
\]

%
% Lösung als hypergeometrische Riehe
%
\subsubsection{Lösung als hypergeometrische Funktion}
Die Newtonreihe verwendet ein absteigendes Produkt im Zähler.
Man kann sie aber in eine Form bringen, die besser zu den aufsteigenden
Produkten bringen, die wir im Zusammenhang mit der Gamma-Funktion
angetroffen und als Pochhammer-Symbole formalisiert haben.

Eine hypergeometrische Funktion zeichnet sich dadurch aus, dass
die Quotienten aufeinanderfolgender Koeffizienten der Reihe rationale
Funktionen von $k$ sind.
Der Quotient ist
nach~\eqref{buch:differentialgleichungen:eqn:newtonreiherekursion}
\[
\frac{a_{k+1}}{a_k}
=
\frac{\alpha-k}{k+1}.
\]
Der Nenner wird nie $0$, aber das Zählerpolynom hat genau die Nullstelle
$-\alpha$.
Die Newtonsche Reihe muss sich daher als Wert der hypergeometrischen
Funktion $\mathstrut_1F_0$ schreiben lassen.

Das Produkt im Zähler von $a_k$ hat $k$ Faktoren, indem wir jeden Faktor
mit $-1$ multiplizieren, erhalten wir
\begin{align*}
\alpha(\alpha-1)(\alpha-2)\dots(\alpha-k+1)
&=
(-\alpha)(-\alpha+1)(-\alpha+2)\dots(-\alpha+k-1) (-1)^k
\\
&=
(-\alpha)_k (-1)^k.
\end{align*}
Indem wir den Faktor $-1$ in der Variablen absorbieren, erhalten
wir die Darstellung
\[
(1+t)^\alpha
=
\sum_{k=0}^\infty
(-\alpha)_k\frac{(-t)^k}{k!}.
\]
Damit haben wir den folgenden Satz gezeigt.

\begin{satz}
\label{buch:differentialgleichungen:satz:newtonschereihe}
Die Newtonsche Reihe für $(1-t)^\alpha$ ist der Wert
\[
(1-t)^\alpha
=
\sum_{k=0}^\infty (-\alpha)_k \frac{t^k}{k!}
=
\mathstrut_1F_0(-\alpha;t)
\]
der hypergeometrischen Funktion $\mathstrut_1F_0$.
\end{satz}

%
% Verallgemeinerte Potenzreihen
%
\subsection{Lösung mit verallgemeinerten Potenzreihen
\label{buch:differentialgleichungen:subsection:verallgemeinrt}}
In vielen Anwendungsfällen hat die Differentialgleichung die Form
\begin{equation}
x^2y'' + p(x)xy' + q(x)y = 0,
\label{buch:differentialgleichungen:eqn:dglverallg}
\end{equation}
gesucht ist eine Lösung $y(x)$ auf dem Intervall $[0,\infty)$.
Für die folgende Diskussion nehmen wir an, dass sich die Funktionen
$p(x)$ und $q(x)$ in konvergente Potenzreihen
\begin{align*}
p(x)&=\sum_{k=0}^\infty p_kx^k = p_0+p_1x+p_2x^2+p_3x^3+\dots
\\
q(x)&=\sum_{k=0}^\infty q_kx^k = q_0+q_1x+q_2x^2+q_3x^3+\dots
\end{align*}
entwickeln lassen.

%
% Potenzreihenmethode funktioniert nicht
%
\subsubsection{Die Potenzreihenmethode funktioniert nicht}
Für die Differentialgleichung
\eqref{buch:differentialgleichungen:eqn:dglverallg}
funktioniert die Potenzreihenmethod oft nicht.
Sind die Funktionen $p(x)$ und $q(x)$ zum Beispiel Konstante 
$p(x)=p_0$ und $q(x)=q_0$, dann führt der Potenzreihenansatz
\[
y(x) = \sum_{k=0}^\infty a_kx^k
\]
auf die Gleichung
\begin{align*}
x^2\sum_{k=0}^\infty a_kk(k-1)x^{k-2}
+
p_0x\sum_{k=0}^\infty a_kkx^{k-1}
+
q_0\sum_{k=0}^\infty a_kx^k
&=
0
\\
\Rightarrow\qquad
\sum_{k=0}^\infty\bigl(
k(k-1)
+
p_0k
+
q_0
\bigr)a_kx^k
&=
0.
\end{align*}
Durch Koeffizientenvergleich folgt dann, dass für jedes $k$ mindestens
eine der Gleichungen
\[
k(k-1) +p_0k +q_0 = k^2 + (p_0-1)k +q_0 = 0
\qquad\text{und}\qquad
a_k=0
\]
erfüllt sein muss.
Die erste Gleichung ist eine quadratische Gleichung in $k$, es gibt also
höchstens zwei Koeffizienten, für die die erste Gleichung erfüllt sein
kann, für die also auch die Koeffizienten $a_k\ne 0$ sein können.
Sind die Lösungen nicht ganzzahlig, dann müssen alle Koeffizienten 
$a_k=0$ sein, die einzige Potenzreihe ist die triviale Funktion $y(x)=0$.

%
% Verallgemeinerte Potenzreihe
%
\subsubsection{Verallgemeinerte Potenzreihe}
Für Differentialgleichungen der Art
\eqref{buch:differentialgleichungen:eqn:dglverallg}
ist also ein anderer Ansatz nötig.
Die Schwierigkeit bestand darin, dass die Gleichungen für die einzelnen
Koeffizienten $a_k$ voneinander unabhängig waren.
Mit einem zusätzlichen Potenzfaktor $x^\varrho$ mit nicht
notwendigerweise ganzzahligen Wert kann die nötige Flexibilität
erreicht werden.
Wir verwenden daher den Ansatz
\[
y(x)
=
x^\varrho \sum_{k=0}^\infty a_kx^k
=
\sum_{k=0}^\infty a_k x^{\varrho+k}
\]
und versuchen nicht nur die Koeffizienten $a_k$ sondern auch den
Exponenten $\varrho$ zu bestimmen.
Durch Modifikation von $\varrho$ können wir immer erreichen, dass
$a_0\ne 0$ ist.

Die Ableitungen von $y(x)$ mit der zugehörigen Potenz von x  sind
\begin{align*}
xy'(x)
&=
\sum_{k=0}^\infty
(\varrho+k)a_kx^{\varrho+k}
=
\sum_{k=1}^\infty
(\varrho+k-1)a_{k-1}x^{\varrho+k}
\\
x^2y''(x)
&=
\sum_{k=0}^\infty
(\varrho+k)(\varrho+k-1)a_kx^{\varrho+k}.
\end{align*}
Diese Ableitungen setzen wir jetzt in die Differentialgleichung ein, 
die dadurch zu
\begin{equation}
\sum_{k=0}^\infty  (\varrho+k)(\varrho+k-1) a_k x^{\varrho+k}
+
\sum_{k=0}^\infty \sum_{l=0}^\infty (\varrho+k) p_l a_kx^{\varrho+k+l}
+
\sum_{k=0}^\infty \sum_{l=0}^\infty q_l a_k x^{\varrho+k+l}
=
0
\label{buch:differentialgleichungen:eqn:veralgpotenzsumme}
\end{equation}
wird.

Ausgeschrieben geben die einzelnen Terme
\begin{align*}
0
&=
\varrho(\varrho-1)a_0x^\varrho
+
(\varrho+1)\varrho a_1x^{\varrho+1}
+
(\varrho+2)(\varrho+1)a_2x^{\varrho+2}
+
(\varrho+3)(\varrho+2)a_3x^{\varrho+3}
+
\dots
\\
&+
\varrho p_0 a_0 x^{\varrho}
+
\bigl((\varrho +1)a_1p_0 + \varrho a_0 p_1\bigr) x^{\varrho+1}
+
\bigl((\varrho +2)a_2p_0 + (\varrho+1)a_1p_1 + \varrho a_0 p_2\bigr) x^{\varrho+2}
+
\dots
\\
&+
q_0a_0x^{\varrho}
+
(q_0a_1+q_1a_0) x^{\varrho+1}
+
(q_0a_2+q_1a_1+q_2a_0) x^{\varrho+2}
+
(q_0a_3+q_1a_2+q_2a_1+q_3a_0) x^{\varrho+3}
+
\dots
\end{align*}
Fasst man die Terme mit gleichem Exponenten zusammen, findet man
\begin{align*}
0
&=
\bigl(
\varrho(\varrho-1) + \varrho p_0 + q_0
\bigr)a_0 x^{\varrho}
\\
&+
\bigl(
((\varrho+1)\varrho 
+
(\varrho+1) p_0
+
q_0) a_1
+
( \varrho p_1 + q_1)a_0
\bigr)x^{\varrho+1}
\\
&+
\bigl(
(
(\varrho+2)(\varrho+1)
+
(\varrho+2)p_0
+
q_0)a_2
+
(\varrho+1)p_1 a_1
+
\varrho p_2 a_0
+q_1a_1+q_2a_0
\bigr)x^{\varrho+2}
\\
&+\dots
\end{align*}
Der Koeffizientenvergleich ergibt dann
\[
\renewcommand{\arraycolsep}{0pt}
\begin{array}{rcrlcrlcrl}
0&\mathstrut=\mathstrut&(\varrho(\varrho-1)     + \varrho p_0 + q_0)&a_0
	& &                 &
		&&
\\
0&\mathstrut=\mathstrut&((\varrho+1)\varrho     + \varrho p_0 + q_0)&a_1
	&\mathstrut+\mathstrut&(\varrho p_1+q_1)&a_0
		& &
\\
0&\mathstrut=\mathstrut&((\varrho+2)(\varrho+1) + \varrho p_0 + q_0)&a_2
	&\mathstrut+\mathstrut&((\varrho+1)p_1+q_1)&a_1
		&\mathstrut+\mathstrut&(\varrho p_2+q_0)&a_0
\end{array}
\]

Diese Rechnung kann man auch allgemein durchführen.
Für den Koeffizientenvergleich müssen die Terme in
\eqref{buch:differentialgleichungen:eqn:veralgpotenzsumme}
mit gleicher
Potenz $x^{\varrho+n}$ zusammengefasst werden.
Dazu schreiben wir zunächst die Summen alle so, dass die Potenz von $x$
in der Form $x^{\varrho+n}$ auftritt.
So entsteht die Gleichung
\begin{align*}
\sum_{n=0}^\infty
(\varrho+n)(\varrho+n-1) a_n x^{\varrho+n}
+
\sum_{n=0}^\infty
\biggl(
\sum_{l=0}^n
(\varrho+n-l) p_{n-l} a_{l}
\biggr)
x^{\varrho+n}
+
\sum_{n=0}^\infty
\biggl(\sum_{l=0}^n q_{n-l} a_{l}\biggr)
x^{\varrho+n}
&=
0
\end{align*}
Jetzt kann der Koeffizientenvergleich durchgeführt werden.
Der Koeffizient von $x^{\varrho+n}$ ist
\[
(\varrho+n)(\varrho+n-1) a_n x^{\varrho+n}
+
\sum_{l=0}^n
(\varrho+n-l) p_{n-l} a_{l}
+
\sum_{l=0}^n q_{n-l} a_{l}.
\]
Alle diese Koeffizienten müssen verschwinden.
Indem wir die Terme in den beiden Summen über $l$ zusammenfassen,
erhalten wir die Gleichungen
\begin{equation}
\bigl(
(\varrho+n)(\varrho + n-1)
+
\varrho p_0
+
q_0
\bigr)a_n
+
\sum_{l=0}^{n-1}
\bigl(
(\varrho+n-l) p_{n-l}
+
q_{n-l}
\bigr) a_{l}
= 0,
\label{buch:differentialgleichungen:eqn:verallgkoefgl}
\end{equation}
die für jedes $n$ erfüllt sein müssen.

%
% Indexgleichung
%
\subsubsection{Indexgleichung}
Die Gleichungen~\eqref{buch:differentialgleichungen:eqn:verallgkoefgl}
müssen erfüllt sein, wenn eine Lösung in Form einer verallgemeinerten
Potenzreihe existieren soll.
Der Koeffizient $a_n$ mit dem grössten $n$ in jeder Gleichung hat
den gemeinsamen Faktor $F(\varrho+n)$ für das Polynom
\[
F(X) = X(X+1) +Xp_0 + q_0.
\]
Da wir in der Definition einer verallgemeinerten Potenzreihe vorausgesetzt
haben, dass $a_0\ne 0$ sein muss, ist der Ansatz überhaupt nur dann
erfolgreich, wenn \begin{equation}
F(\varrho) = \varrho(\varrho-1) + \varrho p_0 + q_0 = 0
\label{buch:differentialgleichungen:eqn:indexgleichung}
\end{equation}
gilt.
Die Gleichung~\eqref{buch:differentialgleichungen:eqn:indexgleichung}
heisst die {\em Indexgleichung}.
Der Exponent $\varrho$ muss also eine Nullstelle der Indexgleichung sein.

Die Indexgleichung ist eine quadratische Gleichung und hat daher
im allgemeinen zwei Lösungen.
Wir bezeichnen die beiden Nullstellen mit $\varrho_1$ und $\varrho_2$.
Wenn $p_0$ und $q_0$ reell sind, sind die Nullstellen entweder reell
oder konjugiert komplex.

%
% Rekursive Bestimmung der $a_n$
%
\subsubsection{Rekursive Bestimmung der $a_n$}
Der Koeffizient $a_{n}$ kann nur dann aus den vorangegangene
Koeffizienten $a_{n-1},a_{n-2},\dots$ bestimmt werden, wenn
$F(\varrho+n)\ne 0$ ist.
In diesem Fall gilt
\begin{equation}
a_n
=
\frac{1}{F(\varrho+n)}
\sum_{l=0}^{n-1}\bigl( (\varrho+n-l)p_{n-l} + q_{n-l}\bigr)a_l.
\label{buch:differentialgleichungen:eqn:anrekursion}
\end{equation}
Dies funktioniert aber nur, wenn $F(\varrho+n)\ne 0$ für alle
natürlichen $n > 0$ gilt.
Dies ist gleichbedeutend damit, dass die Differenz $\varrho_1-\varrho_2$
keine ganze Zahl ist.

\begin{itemize}
\item
Fall 1: $\varrho_1-\varrho_2$ ist keine ganze Zahl.
In diesem Fall lassen sich zwei Lösungen
\begin{align*}
y_1(x) &= x^{\varrho_1}\sum_{k=0}^\infty a_k x^k
\\
y_2(x) &= x^{\varrho_2}\sum_{k=0}^\infty b_k x^k
\end{align*}
bestimmen, wobei die Koeffizienten $a_k$ und $b_k$ für $k>0$ durch
die Rekursionformel~\eqref{buch:differentialgleichungen:eqn:anrekursion}
aus $a_0$ und $b_0$ bestimmt werden müssen.

\item
Fall 2: $\varrho$ ist eine doppelte Nullstelle ($\varrho_1-\varrho_2=0$).
In diesem Fall kann nur eine Lösung als verallgemeinerte Potenzreihe
gefunden werden.
Um eine zweite Lösung zu finden, muss die Technik der analytischen
Fortsetzung verwendet werden, die in
Kapitel~\ref{buch:chapter:funktionentheorie}
dargestellt werden.

\item
Fall 3: $\varrho_1-\varrho-2$ ist eine positive ganze Zahl.
In diesem Fall ist im Allgemeinen nur eine Lösung in Form einer
verallgemeinerten Potenzreihe möglich.
Auch hier müssen Techniken der Funktionentheorie aus
Kapitel~\ref{buch:chapter:funktionentheorie}
verwendet werden, um eine zweite Lösung zu finden.

\end{itemize}

Wenn $\varrho_1-\varrho_2$ eine negative ganze Zahl ist, kann man die
beiden Nullstellen vertauschen.
Es folgt dann, dass es eine  





