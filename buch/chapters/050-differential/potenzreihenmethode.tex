%
% potenzreihenmethode.tex
%
% (c) 2021 Prof Dr Andreas Müller, OST Ostschweizer Fachhochschule
%
\section{Potenzreihenmethode
\label{buch:differentialgleichungen:section:potenzreihenmethode}}
Die Potenzreihenmethode versucht die Lösung einer gewöhnlichen
Differentialgleichung als Potenzreihe um die Anfangsbedingung zu
entwickeln.
Wir gehen in diesem Abschnitt von einer Differentialgleichung der
Form
\begin{equation}
a_n(x)y^{(n)}(x)
+
a_{n-1}(x)y^{(n-1)}(x)
+
\dots
+
a_1(x)y'(x)
+
a_0(x)y(x)
=
f(x)
\label{buch:differentialgleichungen:eqn:potenzreihendgl}
\end{equation}
mit der Randbedingung $y(0)=y_0$ aus.
Schon im einfachsten Fall einer homogenen Differentialgleichung erster
Ordnung ergibt sich die Beziehung
\[
a_1(x) y'(x) = a_0(x)y(x),
\]
wobei wir uns $y(x)$ und damit auch $y'(x)$ als Potenzreihe vorstellen.
Insbesondere ist 
\[
\frac{a_1(x)}{a_0(x)} = \frac{y(x)}{y'(x)}
\]
ein Quotient von Potenzreihen, den man natürlich wieder als 
Potenzreihe schreiben kann.
Da es nur auf den Quotienten ankommt, kann man sich auf den Fall
beschränken, dass die Koeffizienten Potenzreihen sind.
Tatsächlich gilt der folgende sehr viel allgemeinere Satz von
Cauchy und Kowalevskaja:

\begin{satz}[Cauchy-Kowalevskaja]
Eine partielle Differentialgleichung der Ordnung $k$ für eine
Funktion $u(x_1,\dots,x_n,t)=u(x,t)$ 
in expliziter Form
\[
\frac{\partial^k}{\partial t^k}
=
G\biggl(x,t,
\frac{\partial^j\partial^\alpha}{\partial t^j\,\partial x^k}
\biggr)
\quad\text{mit $j<k$ und $|\alpha|+j\le k$}
\]
mit einer Funktion $G$, die analytisch ist in allen Variablen
und der Randbedingung
\[
\frac{\partial j}{\partial t^j}u(x,0) = \varphi_j(x)\quad\text{für $k=0,\dots,k-1$}
\]
mit analytischen Funktion $\varphi_j$ hat eine in einer Umgebung von 
$t=0$ eindeutige analytische Lösung.
\end{satz}

Im folgenden werden wir daher weitere einschränkende Annahmen über
die Koeffizienten $a_k(x)$ machen.

\subsection{Potenzreihenansatz und Koeffizientenvergleich}



\subsection{Die Newtonsche Reihe}
Wir lösen die
Differentialgleichung~\eqref{buch:differentialgleichungen:eqn:wurzeldgl1}
mit der Anfangsbedingung $y(t)=1$ mit der Potenzreihenmethode.
Wir setzen daher für die Lösung die Potenzreihe an
\[
y(t)
=
a_0 + a_1t + a_2t^2 + a_3t^3 + \dots + a_kt^k + \dots
\]
Die Ableitung ist
\[
\dot{y}(t)
=
a_1 + 2a_2t + 3a_3t^2 + \dots  + ka_kt^{k-1} + \dots
\]
Einsetzen in die 
Differentialgleichung~\eqref{buch:differentialgleichungen:eqn:wurzeldgl1}
liefert
\begin{align*}
(1+t)
(
a_1 + 2a_2t + 3a_3t^2 + \dots  + ka_kt^{k-1} + \dots
)
&=
\alpha
(
a_0 + a_1t + a_2t^2 + a_3t^3 + \dots + a_kt^k + \dots
)
\\
a_1
+(a_1+2a_2)t
+(2a_2+3a_3)t^2
+(3a_3+4a_4)t^3
+\dots
&=
\alpha a_0 + \alpha a_1t + \alpha a_2t^2 + \alpha a_3t^3 + \dots
\end{align*}
Der Koeffizientenvergleich ergbiet die Gleichungen
\begin{align}
a_1&=\alpha a_0
\notag
\\
a_1+2a_2 &= \alpha a_1 &&\Rightarrow& 2a_2 &= (\alpha-1) a_1
\notag
\\
2a_2+3a_3 &= \alpha a_2&&\Rightarrow& 3a_3 &= (\alpha-2) a_2
\notag
\\
3a_3+4a_4 &= \alpha a_3&&\Rightarrow& 4a_4 &= (\alpha-3) a_3
\notag
\\
4a_4+5a_5 &= \alpha a_4&&\Rightarrow& 5a_5 &= (\alpha-4) a_4
\notag
\\
&\vdots
\notag
\\
&&&& \llap{$(k+1)a_{k+1}$} &= (\alpha-k) a_k
&&\Rightarrow&
a_{k+1} = \frac{\alpha-k}{k+1}a_k.
\label{buch:differentialgleichungen:eqn:newtonreiherekursion}
\end{align}
Die
Rekursionsformel~\eqref{buch:differentialgleichungen:eqn:newtonreiherekursion}
gilt auch im Fall $k=0$.
Aus der Anfangsbedingung folgt $a_0=1$.
Durch wiederholte Anwendung der 
Rekursionsformel~\eqref{buch:differentialgleichungen:eqn:newtonreiherekursion}
erhalten wir jetzt die Koeffizienten
\begin{align*}
a_0&=1
\\
a_1&=\alpha
\\
a_2&=\frac{\alpha(\alpha-1)}{1\cdot 2}
\\
a_3&=\frac{\alpha(\alpha-1)(\alpha-2)}{1\cdot 2\cdot 3}
\\
a_4&=\frac{\alpha(\alpha-1)(\alpha-2)(\alpha-3)}{1\cdot 2\cdot 3\cdot 4}
\\
&\;\vdots
\\
a_k&=\frac{\alpha(\alpha-1)(\alpha-2)\dots(\alpha-k+1)}{k!}.
\end{align*}
Für ganzzahliges $\alpha$ ist $a_k$ der Binomialkoeffizient
\[
a_k=\binom{\alpha}{k}
\]
und $a_k=0$ für $k>\alpha$.
Für nicht ganzzahliges $\alpha$ sind alle Koeffizienten $a_k\ne 0$.

Die Lösung der 
Differentialgleichung~\eqref{buch:differentialgleichungen:eqn:wurzeldgl1}
ist daher die Reihe
\begin{equation}
(1+t)^\alpha
=
\sum_{k=0}^\infty
\frac{\alpha(\alpha-1)\dots(\alpha-k+1)}{k!}\, t^k.
\label{buch:differentialgleichungen:eqn:newtonreihe}
\end{equation}
Für ganzzahliges $\alpha$ wird daraus die binomische Formel
\[
(1+t)^\alpha
=
\sum_{k=0}^\infty
\frac{\alpha(\alpha-1)\dots(\alpha-k+1)}{k!}\, t^k
=
\sum_{k=0}^\alpha \binom{\alpha}{k} t^k.
\]

%
% Lösung als hypergeometrische Riehe
%
\subsubsection{Lösung als hypergeometrische Funktion}
Die Newtonreihe verwendet ein absteigendes Produkt im Zähler.
Man kann sie aber in eine Form bringen, die besser zu den aufsteigenden
Produkten bringen, die wir im Zusammenhang mit der Gamma-Funktion
angetroffen und als Pochhammer-Symbole formalisiert haben.

Eine hypergeometrische Funktion zeichnet sich dadurch aus, dass
die Quotienten aufeinanderfolgender Koeffizienten der Reihe rationale
Funktionen von $k$ sind.
Der Quotient ist
nach~\eqref{buch:differentialgleichungen:eqn:newtonreiherekursion}
\[
\frac{a_{k+1}}{a_k}
=
\frac{\alpha-k}{k+1}.
\]
Der Nenner wird nie $0$, aber das Zählerpolynom hat genau die Nullstelle
$-\alpha$.
Die Newtonsche Reihe muss sich daher als Wert der hypergeometrischen
Funktion $\mathstrut_1F_0$ schreiben lassen.

Das Produkt im Zähler von $a_k$ hat $k$ Faktoren, indem wir jeden Faktor
mit $-1$ multiplizieren, erhalten wir
\begin{align*}
\alpha(\alpha-1)(\alpha-2)\dots(\alpha-k+1)
&=
(-\alpha)(-\alpha+1)(-\alpha+2)\dots(-\alpha+k-1) (-1)^k
\\
&=
(-\alpha)_k (-1)^k.
\end{align*}
Indem wir den Faktor $-1$ in der Variablen absorbieren, erhalten
wir die Darstellung
\[
(1+t)^\alpha
=
\sum_{k=0}^\infty
(-\alpha)_k\frac{(-t)^k}{k!}.
\]
Damit haben wir den folgenden Satz gezeigt.

\begin{satz}
Die Newtonsche Reihe für $(1-t)^\alpha$ ist der Wert
\[
(1-t)^\alpha
=
\sum_{k=0}^\infty (-\alpha)_k \frac{t^k}{k!}
=
\mathstrut_1F_0(-\alpha;t)
\]
der hypergeometrischen Funktion $\mathstrut_1F_0$.
\end{satz}


