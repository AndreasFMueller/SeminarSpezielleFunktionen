%
% bessel.tex
%
% (c) 2021 Prof Dr Andreas Müller, OST Ostschweizer Fachhochschule
%
\section{Bessel-Funktionen
\label{buch:differntialgleichungen:section:bessel}}
Die Besselsche Differentialgleichung
erlaubt Wellen mit zylindrischer
Symmetrie und die Strömung in einem zylindrischen Rohr zu beschreiben.
Die Auflösung eines optischen Systems wird durch die Beugung limitiert,
die Helligkeitskverteilung des Bildes einer Punktquelle ist
zylindersymmetrisch und kann mit Hilfe von Lösungen der Besselschen
Differentialgleichung beschrieben werden.
Die Besselsche Differentialgleichung hat im Allgemeinen keine Lösung,
die sich durch bekannte Funktionen ausdrücken lassen, es ist also
nötig, eine neue Familie von speziellen Funktionen zu definieren,
die Bessel-Funktionen.

\subsection{Die Besselsche Differentialgleichung}
% XXX Wo taucht diese Gleichung auf
Die Besselsche Differentialgleichung ist die Differentialgleichung
\[
x^2\frac{d^2y}{dx^2} + x\frac{dy}{dx} + (x^2-\alpha^2)y = 0
\]
zweiter Ordnung
für eine auf dem Interval $[0,\infty)$ definierte Funktion $y(x)$.
Der Parameter $\alpha$ ist eine beliebige komplexe Zahl $\alpha\in \mathbb{C}$,
die Lösungsfunktionen hängen daher von $\alpha$ ab.

Die Besselsche Differentialgleichung ist eine Differentialgleichung
der Art~\eqref{buch:differentialgleichungen:eqn:dglverallg} mit
\[
p(x) = 1
\qquad\text{und}\qquad
q(x) = x^2-\alpha^2.
\]
Nach den Ausführungen von
Abschnitt~\ref{buch:differentialgleichungen:subsection:verallgemeinrt},
muss die Lösung in der Form einer verallgemeinerten Potenzreihe 
gesucht werden.
Dazu muss zunächst die Indexgleichung
\[
0
=
X(X-1) + Xp_0 + q_0
=
X(X-1) + X - \alpha^2
=
X^2-\alpha^2
=
(X-\alpha)(X+\alpha)
\]
gelöst werden.
Die Nullstellen sind offenbar $\varrho_1=\alpha$ und $\varrho_2=-\alpha$.

Die beiden Vorzeichen der Nullstellen der Indexgleichung führen
auf die gleiche Differentialgleichung.
Der Lösungsraum der Differentialgleichung ist natürlich trotzdem
zweidimensional, so dass es immer noch möglich ist, den
beiden Nullstellen der Indexgleichung verschiedene Lösungen
zuzuordnen.
Die Diskussion in
Abschnitt~\ref{buch:differentialgleichungen:subsection:verallgemeinrt}
hat Kriterien ergeben, unter denen zwei linear unabhängige Lösungen
mit Hilfe einer verallgemeinerten Potenzreihe gefunden werden können.
Falls nur eine solche Lösung gefunden werden kann, wird sie der grösseren
der beiden Zahlen $\alpha$ und $-\alpha$ zugeordnet
(oder $0$, falls $\alpha=-\alpha=0$).
Eine weitere Lösung kann mit Hilfe analytischer Fortsetzung gefunden werden,
wie später gezeigt wird.

Für nicht reelles $\alpha$ kann $\varrho_1-\varrho_2=2\alpha$ keine 
Ganzzahl sein, es ist also garantiert, dass zwei linear unabhängig
Lösungen der Form
\begin{equation}
y_1(x) = x^\alpha\sum_{k=0}^\infty a_kx^k
\qquad\text{und}\qquad
y_2(x) = x^{-\alpha}\sum_{k=0}^\infty b_kx^k.
\label{buch:differentialgleichungen:eqn:besselloesungen}
\end{equation}
existieren.

Für reelles $\alpha\in\mathbb{R}$ gibt es zwei Lösungen der
Form~\eqref{buch:differentialgleichungen:eqn:besselloesungen}
genau dann, wenn $\varrho_1-\varrho_2$ keine Ganzzahl ist.
Nur eine Lösung kann man finden, wenn 
\[
\alpha-(-\alpha)=2\alpha \in \mathbb{Z}
\qquad\Rightarrow\qquad
\alpha = \frac{k}{2},\quad k\in\mathbb{Z}
\]
ist.



\subsection{Bessel-Funktionen erster Art}
Für $\alpha \ge 0$ gibt es immer mindestens eine Lösung der Besselgleichung
als verallgemeinerte Potenzreihe mit $\varrho=\alpha$.
Die Funktion $q(x)=x^2-\alpha^2$ ist ein Polynom, die einzigen
von $0$ verschiedenen Koeffizienten sind $q_0=-\alpha^2$
und $q_2=1$.
Für den ersten Koeffizienten $a_0$ gibt es keine Einschränkungen,
wir wählen $a_0=1$.

Die Rekursionsformel für $n=1$ ist
\[
F(\varrho+1) a_1 = (\varrho p_1+q_1)a_0,
\]
aber die Koeffizienten $p_1$ und $q_1$ verschwinden beide und damit
die ganze rechte Seite.
Da $F(\varrho+1)\ne 0$ ist, folgt dass $a_1=0$ sein muss.

% Fall n=1 gesondert behandeln

\subsubsection{Der allgemeine Fall}
Für die höheren Potenzen $n>1$ wird die Rekursionsformel für die
Koeffizienten $a_n$ der verallgemeinerten Potenzreihe
\[
a_{n} =
-\frac{ q_2 a_{n-2} }{F(\varrho+n)}
=
-\frac{a_{n-2}}{(\varrho+n)^2-\alpha^2}
=
-\frac{a_{n-2}}{\varrho^2 + 2\varrho n+n^2-\alpha^2}
=
-\frac{a_{n-2}}{n(n+2\varrho)}.
\]
Im letzten Schritt haben wir verwendet, dass $\varrho=\pm\alpha$
und damit $\varrho^2=\alpha^2$ ist.
Daraus folgt wegen $a_1=0$, dass auch $a_{2k+1}=0$ für alle $k$.
Damit können wir jetzt die Reihe hinschreiben:
\begin{align*}
y(x)
&=
x^{\varrho}\biggl(
1
-
\frac{1}{2(2+2\varrho)} x^2
+
\frac{1}{2(2+2\varrho)4(4+2\varrho)} x^4
-
\frac{1}{2(2+2\varrho)4(4+2\varrho)6(6+2\varrho)} x^6
+
\dots
\biggr)
\\
&=
x^{\varrho}
\biggl(
1
+
\frac{(-x^2/4)}{1\cdot (1+\varrho)}
+
\frac{(-x^2/4)^2}{1\cdot 2\cdot (1+\varrho)\cdot(2-\varrho)}
+
\frac{(-x^2/4)^3}{1\cdot 2\cdot 3\cdot (1+\varrho)\cdot(2+\varrho)\cdot(3+\varrho)}
+
\dots
\biggr)
\\
&=
x^\varrho\biggl(
1
+
\frac{1}{(\varrho+1)}\frac{(-x^2/4)}{1!}
+
\frac{1}{(\varrho+1)(\varrho+2)}\frac{(-x^2/4)^2}{2!}
+
\frac{1}{(\varrho+1)(\varrho+2)(\varrho+3)}\frac{(-x^2/4)^3}{3!}
+
\dots
\biggr)
\\
&=
x^\varrho \sum_{k=0}^\infty
\frac{1}{(\varrho+1)_k} \frac{(-x^2/4)}{k!}
=
\mathstrut_0F_1\biggl(;\varrho+1;-\frac{x^2}{4}\biggr)
\end{align*}
Falls also $\alpha$ kein ganzzahliges Vielfaches von $\frac12$ ist, finden
wir zwei Lösungsfunktionen
\begin{align}
J_\alpha(x)
&=
x^{\alpha\phantom{-}}
\sum_{k=0}^\infty
\frac{1}{(\alpha+1)_k}
\frac{(-x^2/4)^k}{k!}
=
\mathstrut_0F_1\biggl(;\alpha+1;-\frac{x^2}{4}\biggr),
\label{buch:differentialgleichunge:bessel:erste}
\\
J_{-\alpha}(x)
&=
x^{-\alpha} \sum_{k=0}^\infty
\frac{1}{(-\alpha+1)_k} \frac{(-x^2/4)^k}{k!}
=
\mathstrut_0F_1\biggl(;-\alpha+1;-\frac{x^2}{4}\biggr).
\label{buch:differentialgleichunge:bessel:zweite}
\end{align}
Die Funktionen $J_{\pm\alpha}(x)$ heissen {\em Bessel-Funktionen
der Ordnung $\alpha$}.

\subsubsection{Der Fall $\alpha=0$}
Im Fall $\alpha=0$ hat das Indexpolynom eine doppelte Nullstelle, wir
können daher nur eine Lösung erwarten.
Im Fall $\alpha=0$ wird das Produkt im Nenner zu $n!$, so dass die
Lösungsfunktion
\[
J_0(x)
=
\sum_{k=0}^\infty
\frac{(-1)^k}{(k!)^2}
\biggl(\frac{x}{2}\biggr)^{2k}
\]
geschrieben werden kann.

% XXX Zweite Lösung explizit durchrechnen

\subsubsection{Der Fall $\alpha=p$, $p\in\mathbb{N}, p > 0$}
In diesem Fall kann nur die erste
Lösung~\eqref{buch:differentialgleichunge:bessel:erste}
verwendet werden.
Die Pochhammer-Symbole im Nenner können ebenfalls als
Quotient
\[
\frac{1}{(p+1)_k}
=
\frac{1}{(p+1)\cdot(p+k)}
=
\frac{p!}{(p+k)!}
\]
von Fakultäten geschrieben werden.
Damit erhält die Lösungsfunktion die Form
\[
J_p(x)
=
\sum_{k=0}^\infty
\frac{(-1)^k}{k!(p+k)!}\biggl(\frac{x}{2}\biggr)^{p+2k}.
\]

\subsubsection{Der Fall $\alpha=n+\frac12$, $n\in\mathbb{N}$}
Obwohl $2\alpha$ eine Ganzzahl ist, sind die beiden Lösungen
\label{buch:differentialgleichunge:bessel:erste}
und
\label{buch:differentialgleichunge:bessel:zweite}
linear unabhängig.

Man kann zeigen, dass sich die Lösungsfunktionen in diesem Fall
durch bereits bekannte elementare Funktionen ausdrücken lassen.
Wir rechnen dies für $n=0$ nach.
Zunächst drücken wir die Pochhammer-Symbole im Nenner anders aus.
Es ist
\begin{align*}
\biggl(\frac12 + 1\biggr)_k
&=
\biggl(\frac12 + 1\biggr)
\biggl(\frac12 + 2\biggr)
\cdots
\biggl(\frac12 + k\biggr)
=
\frac{1}{2^k}\bigl(3\cdot 5\cdot\ldots\cdot (2k+1)\bigr)
=
\frac{(2k+1)!}{2^{2k+1}\cdot k!}
\\
\biggl(-\frac12 + 1\biggr)_k
&=
\biggl(-\frac12 + 1\biggr)
\biggl(-\frac12 + 2\biggr)
\cdots
\biggl(-\frac12 + k\biggr)
\\
&=
\biggl(\frac12 + 0\biggr)
\biggl(\frac12 + 1\biggr)
\cdots
\biggl(\frac12 + k-1\biggr)
=
\frac{1}{2^k}\bigl(1\cdot 3 \cdot\ldots\cdot (2(k-1)+1)\bigr)
=
\frac{(2k-1)!}{2^{2k}\cdot (k-1)!}
\end{align*}
Damit können jetzt die Reihenentwicklungen der Lösung wie folgt
umgeformt werden
\begin{align*}
y_1(x)
&=
\sqrt{x}
\sum_{k=0}^\infty
\frac{1}{(\alpha+1)_k}
\frac{(-x^2/4)^k}{k!}
=
\sqrt{x}
\sum_{k=0}^\infty
\frac{2^{2k+1}k!}{(2k+1)!}
\frac{(-x^2/4)^k}{k!}
=
\sqrt{x}
\sum_{k=0}^\infty
(-1)^k
\frac{2\cdot x^{2k}}{(2k+1)!}
\\
&=
\frac{1}{2\sqrt{x}}
\sum_{k=0}^\infty
(-1)^k
\frac{x^{2k+1}}{(2k+1)!}
=
\frac{1}{2\sqrt{x}} \sin x
\\
y_2(x)
&=
\frac{1}{\sqrt{x}}
\sum_{k=0}^\infty
\frac{2^{2k}\cdot (k-1)!}{(2k-1)!}
\frac{(-x^2/4)^k}{k!}
=
\frac{1}{\sqrt{x}}
\sum_{k=0}^\infty
(-1)^k
\frac{x^{2k}}{(2k-1)!\cdot k}
\\
&=
\frac{2}{\sqrt{x}}
\sum_{k=0}^\infty
(-1)^k
\frac{x^{2k}}{(2k-1)!\cdot 2k}
=
\frac{2}{\sqrt{x}} \cos x.
\end{align*}

% XXX Nachrechnen, dass diese Funktionen
% XXX Lösungen der Differentialgleichung sind

\subsection{Analytische Fortsetzung und Bessel-Funktionen zweiter Art}





