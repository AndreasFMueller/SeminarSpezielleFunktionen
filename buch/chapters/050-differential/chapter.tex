%
% chapter.tex -- Kapitel zur Funktionen, die als Lösungen von Differential-
%                gleichungen definiert sind
%
% (c) 2021 Prof Dr Andreas Müller, Hochschule Rapperswil
%
% !TeX spellcheck = de_CH
\chapter{Differentialgleichungen
\label{buch:chapter:differential}}
\kopflinks{Differentialgleichungen}
Allgemeine Sätze über die Existenz und Eindeutigkeit der Lösungen
gewöhnlicher Differentialgleichungen garantieren für fast jede
einigermassen vernünftige Gleichung mindestens für kurze Zeit
eine eindeutige Lösung für fast jede Anfangsbedingung.
Die Konstruktion solcher Lösungen stellt sich jedoch als deutlich
schwieriger heraus.

Für einzelne Kategorien von Differentialgleichungen sind 
gut funktionierende Lösungsverfahren gefunden worden, zum Beispiel
für lineare Differentialgleichungen mit konstanten Koeffizienten.
Damit können auch Gleichungen gelöst werden, die sich zum Beispiel
durch eine Variablentransformation auf eine lineare Differentialgleichung
mit konstanten Koeffizienten reduzieren lassen, wie die Eulersche
Differentialgleichung.

Die Methode der Separation der Variablen führt die Lösung
einer Differentialgleichung erster Ordnung auf die Bestimmung 
zweier Stammfunktionen und das Finden einer Umkehrfunktion zurück.
Dieses Verfahren ist jedoch nicht auf Vektordifferentialgleichungen
oder auf Differentialgleichungen höherer Ordnung verallgemeinerungsfähig.

Daneben gibt es eine Reihe von ``Spezialfällen'' wie die
Clairaut-Differentialgleichung oder die damit verwandte
Lagrangesche Differentialgleichung, deren Lösungen eine sehr
spezielle Form haben.

Sehr viele Differentialgleichungen in den Anwendungen können aber
mit keinem der genannten Verfahren gelöst werden.
Hier bleibt nichts anderes übrig, als neue spezielle Funktionen
zu definieren, die Lösungen dieser Differentialgleichungen sind.
Dabei ist man bestrebt, möglichst universell einsetzbare Funktionen
zu definieren, die ein breites Anwendungsfeld haben.

In den folgenden Abschnitten wird zunächst gezeigt, dass viele
der bereits bekannten speziellen Funktionen ebenfalls als Lösungen
gewöhnlicher Differentialgleichungen erhalten werden können.
Die numerische Lösung gewöhnlicher Differentialgleichungen ist
oft keine effizientes Vorgehen zur Bestimmung von einzelnen Werten,
daher wird in
Abschnitt~\ref{buch:differentialgleichungen:section:potenzreihenmethode}
eine universelle Methode vorgestellt, mit der eine Potenzreihenentwicklung
gefunden werden kann.
Eine Potenzreihendarstellung ermöglicht nicht nur die Berechnung
einzelner Werte, sondern auch beliebiger Ableitungen und die
analytische Untersuchung der Funktion mit den Methoden der
komplexen Analysis.
Als Beispiel für dieses Verfahren werden in
Abschnitt~\ref{buch:differntialgleichungen:section:bessel}
die Bessel-Funktionen erster Art vorgestellt.

%
% 1-beispiele.tex
%
% (c) 2021 Prof Dr Andreas Müller, OST Ostschweizer Fachhochschule
%
\section{Beispiele
\label{buch:differentialgleichungen:section:beispiele}}
\kopfrechts{Beispiele}
Viele der bisher betrachteten speziellen Funktionen können 
durch gewöhnliche Differentialgleichungen charakterisiert werden,
\index{Differentialgleichung}%
als deren Lösungen sie auftreten.

%
% Potenzen und Wurzeln
%
\subsection{Potenzen und Wurzeln
\label{buch:differentialgleichungen:subsection:potenzen-und-wurzeln}}
Die Potenzfunktionen und die zugehörigen Wurzeln als die ältesten
speziellen Funktionen bieten bereits eine erste kleine Schwierigkeit.
\index{Potenzfunktion}%
\index{Wurzelfunktion}%
Die Differentialgleichung, die man aus einem naiven Ansatz ableitet,
ist singulär.

%
% Differentialgleichung in (0,\infty)
%
\subsubsection{Differentialgleichung in $(0,\infty)$}
Die Ableitung einer Potenzfunktion $x\mapsto y(x)=x^\alpha$ ist
\[
y'(x) =
\begin{cases}
\alpha x^{\alpha-1} &\qquad \alpha\ne -1\\
\log x&\qquad\text{sonst}
\end{cases}
\]
Im Folgenden wollen wir uns auf den Fall $\alpha\ne -1$ konzentrieren.
Die Ableitungsoperation läuft in diesem Fall darauf hinaus, dass der
Grad um $1$ reduziert wird.
Dies könnte man mit einem Faktor $x$ komponsieren.
Wir fragen daher nach der allgmeinen Lösung der linearen
Differentialgleichung der Form
\begin{equation}
xy' = \alpha y.
\label{buch:differentialgleichungen:eqn:wurzeldgl}
\end{equation}
Diese Gleichung ist separierbar, die Separation von $x$ und $y$ liefert
die Integrale
\[
\int \frac{dy}{y} = \alpha \int \frac{dx}{x} + C.
\]
Die Durchführung der Integration liefert 
\[
\log |y| = \alpha \log|x| + C.
\]
Wendet man die Exponentialfunktion an, erhält man wieder
\[
y = Dx^\alpha,\quad D=\exp C.
\]

Die Differentialgleichung~\eqref{buch:differentialgleichungen:eqn:wurzeldgl}
hat aber eine schwerwiegenden Mangel.
\index{Differentialgleichung}%
Ihre explizite Form lautet
\begin{equation}
y' = \frac{\alpha}{x}\cdot y.
\label{buch:differentialgleichungen:eqn:wurzelsing}
\end{equation}
Dies ist zwar durchaus eine lineare Differentialgleichung erster Ordnung,
aber der Koeffiziente $\alpha/x$ wächst für $x\to 0$ über alle Grenzen.
Man kann daher den Wert der Potenzfunktion im Nullpunkt gar nicht aus der
Differentialgleichung erhalten, es ist dazu mindestens noch ein Grenzübergang
$x\to 0+$ nötig.

%
% Differentialgleichung in der Nähe von x=1
%
\subsubsection{Differentialgleichung in der Nähe von $x=1$}
Um dem Problem des singulären Koeffizienten der
Differentialgleichung~\eqref{buch:differentialgleichungen:eqn:wurzelsing}
aus dem Weg zu gehen, verwenden wir die Variable $t$ mit $x=1+t$ und
versuchen eine Differentialgleichung für die Potenzfunktion
$(1+t)^\alpha$ zu finden.
Es gilt natürlich
\begin{equation}
\frac{d}{dt} (1+t)^\alpha
=
\alpha (1+t)^{\alpha-1}
\qquad\Rightarrow\qquad
(1+t) \dot{y} = \alpha y.
\label{buch:differentialgleichungen:eqn:wurzeldgl1}
\end{equation}
Diese Differentialgleichung kann natürlich auch wieder mit Separation
gelöst werden, es ist
\begin{equation}
\int
\frac{dy}{y} 
=
\alpha
\int
\frac{dt}{1+t}
+
C
\qquad\Rightarrow\qquad
\log|y| = \alpha \log|1+t| + C
\label{buch:differentialgleichungen:eqn:wurzeldgl1loesung}
\end{equation}
und daraus die Potenzfunktion
\[
y=D(1+t)^\alpha
\]
wie vorhin.
Der Vorteil der
Form~\eqref{buch:differentialgleichungen:eqn:wurzeldgl1}
wird sich später bei dem Versuch zeigen, die Funktion $y(t)$
direkt als Potenzreihenlösung der Differentialgleichung zu finden.

%
% Exponentialfunktion und ihre Varianten
%
\subsection{Exponentialfunktion und ihre Varianten
\label{buch:differentialgleichungen:subsection:exponentialfunktion}}
\index{Exponentialfunktion}%
In Kapitel~\ref{buch:chapter:exponential} wurde die Exponentialfunktion
auf algebraische Weise definiert, die Berechnung wurde ermöglicht
mit Hilfe von Grenzwerten und Potenzreihen.
Dabei blieb die Ableitung der Exponentialfunktion aussen vor.
Die Exponentialfunktion lässt sich aber natürlich auch über
Differentialgleichungen charakterisieren.

%
% Die Ableitung der Exponentialfunktion
%
\subsubsection{Die Ableitung der Exponentialfunktion}
\index{Ableitung!der Exponentialfunktion}%
Aus der Potenzreihendarstellung
\[
\exp(x)
=
\sum_{k=0}^\infty \frac{x^k}{k!}
\]
folgt sofort, dass die Ableitung
\[
\frac{d}{dx}\exp(x)
=
\frac{d}{dx}
\sum_{k=0}^\infty
\frac{x^k}{k!}
=
\sum_{k=1}^\infty \frac{kx^{k-1}}{k!}
=
\sum_{k=1}^\infty{x^{k-1}}{(k-1)!}
=
\sum_{l=0}^\infty \frac{x^l}{l!}
=
\exp(x)
\]
ist,
wobei $l=k-1$ gesetzt wurde.
Die Exponentialfunktion ist also ihre eigene Ableitung.

%
% Lineare Differentialgleichung mit konstanten Koeffizienten
%
\subsubsection{Lineare Differentialgleichung mit konstanten Koeffizienten}
Mit der Exponentialfunktion lassen sich beliebige homogene lineare
Differentialgleichungen mit konstanten Koeffizienten lösen.
\index{Differentialgleichung!linear mit konstanten Koeffizienten}%
Sei die Differentialgleichung
\[
y^{(n)} + a_{n-1}y^{(n-1)} + \dots + a_2y'' + a_1y' + a_0y = 0
\]
gegeben.
Mit dem Ansatz $y(x)=e^{\lambda x}$ ergibt sich die Gleichung
\[
\lambda^n e^{\lambda x}
+
a_{n-1}\lambda^{n-1} e^{\lambda x}
+
\dots
+
a_2\lambda^2e^{\lambda x}
+
a_1\lambda e^{\lambda x}
+
a_0e^{\lambda x}
=
(\lambda^n + a_{n-1}\lambda^{n-1} + \dots + a_2\lambda^2 + a_1\lambda + a_0)
e^{\lambda x}
=
0.
\]
Da $e^{\lambda x}\ne 0$ ist, kann $y(x)$ nur dann eine Lösung sein, wenn
$\lambda$ eine Nullstelle des {\em charakteristischen Polynoms}
\index{charakteristisches Polynome}%
\[
p(\lambda)
=
\lambda^n
+
a_{n-1}\lambda^{n-1}
+
\dots
+
a_2\lambda^2
+
a_1\lambda
+
a_0
\]
ist.

%
% Ableitunhen der trigonometrischen Funktionen
%
\subsubsection{Ableitungen der trigonometrischen Funktionen}
\index{Ableitung!trigonometrische Funktion}%
Die Drehmatrix 
\[
D_{\omega t}
=
\begin{pmatrix}
\cos\omega t&         - \sin\omega t\\
\sin\omega t&\phantom{-}\cos\omega t
\end{pmatrix}
\]
beschreibt eine Drehung der Ebene mit der Winkelgeschwindigkeit 
\index{Winkelgeschwindigkeit}
$\omega$.
Der Punkt $(r,0)$ beschreibt unter dieser Drehung eine Kreisbahn
\index{Kreisbahn}%
parametrisiert durch 
\[
t \mapsto \gamma(t)=(r\cos\omega t,r\sin\omega t).
\]
Der Geschwindigkeitsvektor zur Zeit $t$ ist natürlich
\[
\vec{v}(0)
=
\begin{pmatrix}
0\\
r\omega
\end{pmatrix},
\]
zu einer späteren Zeit $t$  ist er
\[
\vec{v}(t)
=
D_{\omega t} \vec{v}(0)
=
\begin{pmatrix}
\cos\omega t&         - \sin\omega t\\
\sin\omega t&\phantom{-}\cos\omega t
\end{pmatrix}
\begin{pmatrix}
0\\r\omega
\end{pmatrix}
=
r
\begin{pmatrix}
         - \omega\sin\omega t\\
\phantom{-}\omega\cos\omega t
\end{pmatrix}.
\]
Gleichzeitig ist $\vec{v}(t)$ natürlich auch die Ableitung  von $\gamma(t)$,
also
\[
\dot{\gamma}(t)
=
r
\frac{d}{dt}
\begin{pmatrix}
\cos\omega t\\
\sin\omega t
\end{pmatrix}
=
r
\begin{pmatrix}
         - \omega\sin\omega t\\
\phantom{-}\omega\cos\omega t
\end{pmatrix}
\qquad\Rightarrow\qquad
\left\{
\begin{aligned}
\frac{d}{dt} \cos\omega t &= -\omega \sin\omega t\\
\frac{d}{dt} \sin\omega t &= \phantom{-} \omega \cos\omega t.
\end{aligned}
\right.
\]
Dies bedeutet, dass die Ableitungen der trigonometrischen Funktionen
\begin{equation}
\begin{aligned}
\frac{d}{dt} \sin t&=\phantom{-}\cos t\\
\frac{d}{dt} \cos t&=-\sin t
\end{aligned}
\label{buch:differentialgleichungen:trigo:ableitungen}
\end{equation}
sind.

%
% Differentialgleichung für trigonometrische Funktionen
%
\subsubsection{Differentialgleichung für trigonometrische Funktionen}
\index{Differentialgleichung!trigonometrische Funktion}%
Aus den Ableitungen~\eqref{buch:differentialgleichungen:trigo:ableitungen}
folgt, dass die trigonometrischen Funktionen $\sin t $ und $\cos t$
Lösungen der Differentialgleichung $y''=-y$ sind.
Das zugehörige charakteristische Polynom ist 
\index{charakteristisches Polynom}%
\[
\lambda^2+1=0
\qquad\Rightarrow\qquad
\lambda^2=-1
\qquad\Rightarrow\qquad
\lambda=\pm i.
\]
Daraus ergeben sich die Lösungen
\[
y_{\pm}(t) = e^{\pm i t}.
\]
Da eine Differentialgleichung zweiter Ordnung nur zwei linear unabhängige
Lösungen haben kann, müssen sich $\sin t$ und $\cos t$ durch
$e^{\pm it}$ ausdrücken lassen.

Die Kosinus-Funktion zeichnet sich dadurch aus, dass $\cos 0=1$ und
$\cos' 0=0$ ist.
\index{Kosinus-Funktion}
Gesucht ist also eine Linearkombination der Lösungen
$y_{\pm}$ der Differentialgleichung mit diesen Anfangswerten.
Zunächst halten wir fest, dass $y_{\pm}(0)=e^{\pm i\cdot 0}=1$.
Für die Ableitungen von $y^{\pm it}$ gilt
\[
\frac{d}{dt}
=
e^{\pm i t}
=
\pm ie^{\pm i t}
\qquad\Rightarrow\qquad
\frac{d}{dt}y_{\pm}(0) = \pm i.
\]
Die Linearkombination $Ay_+(t)+By_-(t)$ hat die Anfangswerte
\begin{align*}
Ay_+(0)+By_-(0)&=A+B\\
Ay'_+(0)+By'_-(0)&=Ai-Bi.
\end{align*}
Damit die Linearkombination $\cos t=Ay_+(t)+By_-(t)$ ist, müssen
$A$ und $B$ Lösungen des Gleichungssystems
\[
\renewcommand\arraycolsep{2pt}
\begin{array}{rcrcr}
 A&+& B&=&1\\
iA&-&iB&=&0
\end{array}
\qquad\Rightarrow\qquad
\begin{array}{rcrcr}
 A&+& B&=&1\\
 A&-& B&=&0
\end{array}
\]
Die Summe und Differenz der beiden Gleichungen führt auf
\[
\left.
\begin{aligned}
2A&=1\\
2B&=1
\end{aligned}
\;
\right\}
\qquad\Rightarrow\qquad
\left.
\begin{aligned}
A&=\textstyle\frac12\\
B&=\textstyle\frac12
\end{aligned}
\;
\right\}
\qquad\Rightarrow\qquad
\cos t = \frac{e^{it}+e^{-it}}{2}.
\]

Andererseits hat die Sinus-Funktion die Anfangswerte $\sin 0=0$ und
\index{Sinus-Funktion}%
$\sin' 0=1$, dies führt auf das Gleichungssystem
\[
\renewcommand\arraycolsep{2pt}
\left.
\begin{array}{rcrcr}
 A&+& B&=&0\\
iA&-&iB&=&1
\end{array}
\;\right\}
\qquad\Rightarrow\qquad
\begin{array}{rcrcr}
 A&+& B&=&0\phantom{.}\\
 A&-& B&=&\frac{1}i.
\end{array}
\]
Diesmal führen
Summe und Differenz der beiden Gleichungen auf
\[
\left.
\begin{aligned}
2A&=\phantom{-}\frac{1}i\\
2B&=-\frac{1}i
\end{aligned}
\;\right\}
\qquad\Rightarrow\qquad
\left.
\begin{aligned}
A&=\phantom{-}\frac1{2i}\\
B&=-\frac1{2i}
\end{aligned}
\;\right\}
\qquad\Rightarrow\qquad
\sin t = \frac{e^{it}-e^{-it}}{2i}.
\]

%
% Potenzreihen für sin(t) und cos(t)
%
\subsubsection{Potenzreihen für $\sin t$ und $\cos t$}
Aus der Potenzreihe der Exponentialfunktion kann man jetzt auch
Potenzreihen für $\sin t$ und $\cos t$ ableiten.
\index{Potenzreihe!Exponentialfunktion}
Zunächst ist
\begin{align*}
y_+(t)
&=
1 + it - \frac{t^2}{2!} - \frac{it^3}{3!} + \frac{t^4}{4!} + \frac{it^5}{5!}
- \frac{t^6}{6!} - \frac{it^7}{7!} + \dots
\\
y_+(t)
&=
1 - it - \frac{t^2}{2!} + \frac{it^3}{3!} + \frac{t^4}{4!} - \frac{it^5}{5!}
- \frac{t^6}{6!} + \frac{it^7}{7!} + \dots.
\intertext{Die trigonometrischen Funktionen können daraus linear kombiniert
werden, zum Beispiel ist die Kosinus-Funktion}
\cos t
=
\frac{y_+(t)+y_-(t)}{2}
&=
1-\frac{t^2}{2!} + \frac{t^4}{4!} -\frac{t^6}{6!}+\dots
=
\sum_{k=0}^\infty (-1)^k\frac{t^{2k}}{(2k)!}.
\intertext{Auf die gleiche Art findet man für die Sinus-Funktion}
\sin t
=
\frac{y_+(t)-y_-(t)}{2i}
&=
t-\frac{t^3}{3!} + \frac{t^5}{5!} - \frac{t^7}{t!} + \dots
=
\sum_{k=0}^\infty (-1)^k \frac{t^{2k+1}}{(2k+1)!}.
\end{align*}

%
% Hyperbolische Funktionen
%
\subsubsection{Hyperbolische Funktionen}
Die Ableitungen der hyperbolischen Funktionen sind
\index{hyperbolische Funktion}%
\begin{equation}
\left.
\begin{aligned}
\frac{d}{dt} \sinh t & = \cosh t \\
\frac{d}{dt} \cosh t & = \sinh t\\
\end{aligned}
\;
\right\}
\qquad\Rightarrow\qquad
\left\{\quad
\begin{aligned}
\frac{d^2}{dt^2}\sinh t&=\sinh t\phantom{.}\\
\frac{d^2}{dt^2}\cosh t&=\cosh t.\\
\end{aligned}\right.
\label{buch:differentialgleichungen:trigo:hyperabl}
\end{equation}
Man beachte die Ähnlichkeit zu den entsprechenden Formeln
\eqref{buch:differentialgleichungen:trigo:ableitungen}
für die trigonometrischen Funktionen.
Die hyperbolischen Funktionen sind also linear unabhängige Lösungen
der Differentialgleichung
\index{Differentialgleichung!hyperbolische Funktion}%
\begin{equation}
y'' -y = 0.
\label{buch:differentialgleichungen:trigo:hyperdgl}
\end{equation}
zweiter Ordnung, die wieder linear und mit konstanten
Koeffizienten sind.
Das charakteristische Polynom von
\index{charakteristisches Polynom}%
\eqref{buch:differentialgleichungen:trigo:hyperdgl}
ist
\[
\lambda^2-1 = (\lambda+1)(\lambda-1) = 0
\]
mit den Nullstellen $\pm 1$.
Die Lösungen von
\eqref{buch:differentialgleichungen:trigo:hyperdgl}
müssen also Linearkombinationen von $y_{\pm}(x)=e^{\pm x}$ sein.
Wir schreiben $y(x)=Ay_+(x)+By_-(x)$.

Die Gleichung für die Anfangsbedingungen 
\index{Anfangsbedingung}%
\[
\begin{pmatrix}
 y(0)\\
y'(0)
\end{pmatrix}
=
\begin{pmatrix}
 y_+(0) &  y_-(0) \\
y'_+(0) & y'_-(0) 
\end{pmatrix}
\begin{pmatrix}
A\\B
\end{pmatrix}
=
\begin{pmatrix}
  1     &    1    \\
  1     &   -1
\end{pmatrix}
\begin{pmatrix}
A\\B
\end{pmatrix}
\]
kann mit Hilfe der inversen Matrix aufgelöst werden:
\[
\begin{pmatrix}
A\\B
\end{pmatrix}
=
\begin{pmatrix}
  1     &    1    \\
  1     &   -1
\end{pmatrix}^{-1}
\begin{pmatrix}
 y(0)\\
y'(0)
\end{pmatrix}
=
\frac12
\begin{pmatrix*}[r]
1&1\\
1&-1
\end{pmatrix*}
\begin{pmatrix}
 y(0)\\
y'(0)
\end{pmatrix}.
\]
Für die Standardbasisvektoren als Anfangswerte findet man jetzt wie bei
den trigonometrischen Funktionen 
\begin{align*}
\left.
\begin{aligned}
 y(0)&=1\\
y'(0)&=0
\end{aligned}
\quad\right\}
&&&\Rightarrow&
\begin{pmatrix}A\\B\end{pmatrix}
&=
\frac12
\begin{pmatrix*}[r]
1&1\\
1&-1
\end{pmatrix*}
\begin{pmatrix}1\\0\end{pmatrix}
=
\begin{pmatrix}\frac12\\\frac12\end{pmatrix}
&&\Rightarrow&
y(x)&=\frac{e^x+e^{-x}}2=\cosh x
\\
\left.
\begin{aligned}
 y(0)&=0\\
y'(0)&=1
\end{aligned}
\quad\right\}
&&&\Rightarrow&
\begin{pmatrix}A\\B\end{pmatrix}
&=
\frac12
\begin{pmatrix*}[r]
1&1\\
1&-1
\end{pmatrix*}
\begin{pmatrix}0\\1\end{pmatrix}
=
\begin{pmatrix*}[r]\frac12\\-\frac12\end{pmatrix*}
&&\Rightarrow&
y(x)&=\frac{e^x-e^{-x}}2=\sinh x.
\end{align*}

Die Ableitung der Matrix $H_{\tau}$ von 
Satz~\ref{buch:geometrie:hyperbolisch:Hparametrisierung} ist
\begin{align*}
\frac{d}{d\tau} H_{\tau}
&=
\frac{d}{d\tau}
\begin{pmatrix}
\cosh\tau & \sinh\tau\\
\sinh\tau & \cosh\tau
\end{pmatrix}
=
\begin{pmatrix}
\sinh\tau & \cosh\tau\\
\cosh\tau & \sinh\tau
\end{pmatrix}
\\
\frac{d}{d\tau} H_{\tau}
\bigg|_{\tau=0}
&=
\begin{pmatrix}
0&1\\
1&0
\end{pmatrix}
=
K,
\end{align*}
wobei $K$ die Matrix von \eqref{buch:geometrie:hyperbolisch:matrixK} ist.





\input{chapters/050-differential/2-potenzreihenmethode.tex}
%
% bessel.tex
%
% (c) 2021 Prof Dr Andreas Müller, OST Ostschweizer Fachhochschule
%
\section{Bessel-Funktionen
\label{buch:differntialgleichungen:section:bessel}}
\rhead{Bessel-Funktionen}
Die Besselsche Differentialgleichung
erlaubt Wellen mit zylindrischer
Symmetrie und die Strömung in einem zylindrischen Rohr zu beschreiben.
Die Auflösung eines optischen Systems wird durch die Beugung limitiert,
die Helligkeitskverteilung des Bildes einer Punktquelle ist
zylindersymmetrisch und kann mit Hilfe von Lösungen der Besselschen
Differentialgleichung beschrieben werden.
Das Kapitel~\ref{chapter:kreismembran} zeigt, wie die Bessel-Funktionen
bei der Lösung der Wellengleichung für eine kreisförmige Membran
auftreten.
Die Besselsche Differentialgleichung hat im Allgemeinen keine Lösung,
die sich durch bekannte Funktionen ausdrücken lassen, es ist also
nötig, eine neue Familie von speziellen Funktionen zu definieren,
die Bessel-Funktionen.

%
% Besselsche Differentialgleichung
%
\subsection{Die Besselsche Differentialgleichung}
% XXX Wo taucht diese Gleichung auf
Die Besselsche Differentialgleichung ist die Differentialgleichung
\begin{equation}
x^2\frac{d^2y}{dx^2} + x\frac{dy}{dx} + (x^2-\alpha^2)y = 0
\label{buch:differentialgleichungen:eqn:bessel}
\end{equation}
\index{Differentialgleichung!Besselsche}%
\index{Besselsche Differentialgleichung}%
zweiter Ordnung
für eine auf dem Interval $[0,\infty)$ definierte Funktion $y(x)$.
Der Parameter $\alpha$ ist eine beliebige reelle oder sogar komplexe
Zahl $\alpha\in \mathbb{C}$,
die Lösungsfunktionen hängen daher von $\alpha$ ab.

%
% Eigenwertproblem
%
\subsubsection{Eigenwertproblem}
Die Besselsche Differentialgleichung
\eqref{buch:differentialgleichungen:eqn:bessel}
kann man auch als Eigenwertproblem für den Bessel-Operator
\index{Bessel-Operator}%
\index{Operator!Bessel-}%
\begin{equation}
B = x^2\frac{d^2}{dx^2} + x\frac{d}{dx} + x^2
\label{buch:differentialgleichungen:bessel-operator}
\end{equation}
schreiben.
Eine Lösung $y(x)$ der Gleichung
\eqref{buch:differentialgleichungen:eqn:bessel}
erfüllt
\[
By
=
x^2y''+xy'+x^2y
=\alpha^2 y,
\]
ist also eine Eigenfunktion des Bessel-Operators zum Eigenwert
$\alpha^2$.

%
% Indexgleichung
%
\subsubsection{Indexgleichung}
Die Besselsche Differentialgleichung ist eine Differentialgleichung
der Art~\eqref{buch:differentialgleichungen:eqn:dglverallg} mit
\[
p(x) = 1
\qquad\text{und}\qquad
q(x) = x^2-\alpha^2.
\]
Nach den Ausführungen von
Abschnitt~\ref{buch:differentialgleichungen:subsection:verallgemeinrt},
muss die Lösung in der Form einer verallgemeinerten Potenzreihe 
gesucht werden.
Dazu muss zunächst die Indexgleichung
\[
0
=
X(X-1) + Xp_0 + q_0
=
X(X-1) + X - \alpha^2
=
X^2-\alpha^2
=
(X-\alpha)(X+\alpha)
\]
gelöst werden.
Die Nullstellen sind offenbar $\varrho_1=\alpha$ und $\varrho_2=-\alpha$.

Die beiden Vorzeichen der Nullstellen der Indexgleichung führen
auf die gleiche Differentialgleichung.
Der Lösungsraum der Differentialgleichung ist natürlich trotzdem
zweidimensional, so dass es immer noch möglich ist, den
beiden Nullstellen der Indexgleichung verschiedene Lösungen
zuzuordnen.
Die Diskussion in
Abschnitt~\ref{buch:differentialgleichungen:subsection:verallgemeinrt}
hat Kriterien ergeben, unter denen zwei linear unabhängige Lösungen
mit Hilfe einer verallgemeinerten Potenzreihe gefunden werden können.
Falls nur eine solche Lösung gefunden werden kann, wird sie der grösseren
der beiden Zahlen $\alpha$ und $-\alpha$ zugeordnet
(oder $0$, falls $\alpha=-\alpha=0$).
Eine weitere Lösung kann mit Hilfe analytischer Fortsetzung gefunden werden,
wie später in Kapitel~\ref{buch:chapter:funktionentheorie} gezeigt wird.

Für nicht reelles $\alpha$ kann $\varrho_1-\varrho_2=2\alpha$ keine 
Ganzzahl sein, es ist also garantiert, dass zwei linear unabhängig
Lösungen der Form
\begin{equation}
y_1(x) = x^\alpha\sum_{k=0}^\infty a_kx^k
\qquad\text{und}\qquad
y_2(x) = x^{-\alpha}\sum_{k=0}^\infty b_kx^k.
\label{buch:differentialgleichungen:eqn:besselloesungen}
\end{equation}
existieren.

Für reelles $\alpha\in\mathbb{R}$ gibt es zwei Lösungen der
Form~\eqref{buch:differentialgleichungen:eqn:besselloesungen}
genau dann, wenn $\varrho_1-\varrho_2$ keine Ganzzahl ist.
Nur eine Lösung kann man finden, wenn 
\[
\alpha-(-\alpha)=2\alpha \in \mathbb{Z}
\qquad\Rightarrow\qquad
\alpha = \frac{k}{2},\quad k\in\mathbb{Z}
\]
ist.

%
% Bessel-Funktionen erster Art
%
\subsection{Bessel-Funktionen erster Art
\label{buch:differentialgleichungen:subsection:bessel1steart}}
Für $\alpha \ge 0$ gibt es immer mindestens eine Lösung der Bessel-Gleichung
als verallgemeinerte Potenzreihe mit $\varrho=\alpha$.
Die Funktion $q(x)=x^2-\alpha^2$ ist ein Polynom, die einzigen
von $0$ verschiedenen Koeffizienten sind $q_0=-\alpha^2$
und $q_2=1$.
Für den ersten Koeffizienten $a_0$ gibt es keine Einschränkungen,
wir wählen $a_0=1$.

Die Rekursionsformel für $n=1$ ist
\[
F(\varrho+1) a_1 = (\varrho p_1+q_1)a_0,
\]
aber die Koeffizienten $p_1$ und $q_1$ verschwinden beide und damit
die ganze rechte Seite.
Da $F(\varrho+1)\ne 0$ ist, folgt dass $a_1=0$ sein muss.

% Fall n=1 gesondert behandeln

%
% Der allgemeine Fall
%
\subsubsection{Der allgemeine Fall}
Für die höheren Potenzen $n>1$ wird die Rekursionsformel für die
Koeffizienten $a_n$ der verallgemeinerten Potenzreihe
\[
a_{n} =
-\frac{ q_2 a_{n-2} }{F(\varrho+n)}
=
-\frac{a_{n-2}}{(\varrho+n)^2-\alpha^2}
=
-\frac{a_{n-2}}{\varrho^2 + 2\varrho n+n^2-\alpha^2}
=
-\frac{a_{n-2}}{n(n+2\varrho)}.
\]
Im letzten Schritt haben wir verwendet, dass $\varrho=\pm\alpha$
und damit $\varrho^2=\alpha^2$ ist.
Daraus folgt wegen $a_1=0$, dass auch $a_{2k+1}=0$ für alle $k$.
Damit können wir jetzt die Reihe hinschreiben:
\begin{align*}
y(x)
&=
x^{\varrho}\biggl(
1
-
\frac{1}{2(2+2\varrho)} x^2
+
\frac{1}{2(2+2\varrho)4(4+2\varrho)} x^4
-
\frac{1}{2(2+2\varrho)4(4+2\varrho)6(6+2\varrho)} x^6
+
\dots
\biggr)
\\
&=
x^{\varrho}
\biggl(
1
+
\frac{(-x^2/4)}{1\cdot (1+\varrho)}
+
\frac{(-x^2/4)^2}{1\cdot 2\cdot (1+\varrho)\cdot(2-\varrho)}
+
\frac{(-x^2/4)^3}{1\cdot 2\cdot 3\cdot (1+\varrho)\cdot(2+\varrho)\cdot(3+\varrho)}
+
\dots
\biggr)
\\
&=
x^\varrho\biggl(
1
+
\frac{1}{(\varrho+1)}\frac{(-x^2/4)}{1!}
+
\frac{1}{(\varrho+1)(\varrho+2)}\frac{(-x^2/4)^2}{2!}
+
\frac{1}{(\varrho+1)(\varrho+2)(\varrho+3)}\frac{(-x^2/4)^3}{3!}
+
\dots
\biggr)
\\
&=
x^\varrho \sum_{k=0}^\infty
\frac{1}{(\varrho+1)_k} \frac{(-x^2/4)}{k!}
=
x^\varrho
\cdot
\mathstrut_0F_1\biggl(;\varrho+1;-\frac{x^2}{4}\biggr)
\end{align*}
Wir finden also zwei Lösungsfunktionen
\begin{align}
y_1(x)
%J_\alpha(x)
&=
x^{\alpha\phantom{-}}
\sum_{k=0}^\infty
\frac{1}{(\alpha+1)_k}
\frac{(-x^2/4)^k}{k!}
=
x^\alpha
\cdot
\mathstrut_0F_1\biggl(;\alpha+1;-\frac{x^2}{4}\biggr),
\label{buch:differentialgleichunge:bessel:eqn:erste}
\\
y_2(x)
%J_{-\alpha}(x)
&=
x^{-\alpha} \sum_{k=0}^\infty
\frac{1}{(-\alpha+1)_k} \frac{(-x^2/4)^k}{k!}
=
x^{-\alpha}
\cdot
\mathstrut_0F_1\biggl(;-\alpha+1;-\frac{x^2}{4}\biggr).
\label{buch:differentialgleichunge:bessel:eqn:zweite}
\end{align}
Man beachte, dass die zweite Lösung für ganzzahliges $\alpha>0$ nicht
definiert ist.
Man kann auch direkt nachrechnen, dass diese Funktionen Lösungen
der Besselschen Differentialgleichung sind.

%
% Bessel-Funktionen
%
\subsubsection{Bessel-Funktionen}
Da die Besselsche Differentialgleichung linear ist, ist auch
jede Linearkombination der Funktionen
\eqref{buch:differentialgleichunge:bessel:eqn:erste}
und
\eqref{buch:differentialgleichunge:bessel:eqn:zweite}
eine Lösung.
Satz~\ref{buch:rekursion:gamma:satz:gamma-pochhammer}
ermöglicht, das Pochhammer-Symbol durch Werte der Gamma-Funktion
wie in
\[
(\alpha+1)_n = \frac{\Gamma(\alpha+k+1)}{\Gamma(\alpha+1)}
\]
auszudrücken.
Damit wird
\begin{align}
y_1(x)
&=
x^\alpha
\sum_{k=0}^\infty
\frac{\Gamma(\alpha+1)}{\Gamma(\alpha+k+1)}
\frac{(-x^2/4)^k}{k!}
=
\Gamma(\alpha+1) 2^{\alpha}
\biggl(\frac{x}{2}\biggr)^\alpha
\sum_{k=0}^\infty
\frac{(-1)^k}{k!\,\Gamma(\alpha+k+1)} \biggl(\frac{x}{2}\biggr)^{2k}
\label{buch:differentialgleichungen:bessel:normierungsgleichung}
\end{align}
Nur gerade der Faktor $2^\alpha\Gamma(\alpha+1)$ ist von $k$ und $x$ 
unabhängig, daher ist die folgende Definition sinnvoll:

\begin{definition}
\label{buch:differentialgleichungen:bessel:definition}
Die Funktion
\[
J_{\alpha}(x)
=
\biggl(\frac{x}{2}\biggr)^\alpha
\sum_{k=0}^\infty
\frac{(-1)^k}{k!\,\Gamma(\alpha+k+1)}
\biggl(\frac{x}{2}\biggr)^{2k}
\]
heisst {\em Bessel-Funktion erster Art der Ordnung $\alpha$}.
\index{Bessel-Funktion!erster Art}%
\end{definition}

Die Bessel-Funktion $J_\alpha(x)$ der Ordnung $\alpha$ unterscheidet sich
nur durch einen Normierungsfaktor von der Lösung $y_1(x)$.
Dasselbe gilt für $J_{-\alpha}(x)$ und $y_2(x)$:
\begin{align*}
J_{\alpha}(x)
&=
\frac{1}{2^\alpha\Gamma(\alpha+1)}
\cdot
y_1(x)
\\
J_{-\alpha}(x)
&=
\frac{1}{2^{-\alpha}\Gamma(-\alpha+1)}
\cdot
y_2(x).
\end{align*}

%
% Ganzzahlige Ordnung
%
\subsubsection{Bessel-Funktionen ganzzahliger Ordnung}
Man beachte, dass diese Definition für beliebige ganzzahlige 
$\alpha$ funktioniert.
Ist $\alpha=-n<0$, $n\in\mathbb{N}$, dann hat der Nenner Pole 
an den Stellen $k=0,1,\dots,n-1$.
Die Summe beginnt also erst bei $k=n$ oder
\begin{align*}
J_{-n}(x)
&=
\sum_{k=n}^\infty \frac{(-1)^k}{m!\,k!}\biggl(\frac{x}{2}\biggr)^{2k-n}
=
\sum_{l=0}^\infty
\frac{(-1)^{l+n}}{m!\,(l+n)!}\biggl(\frac{x}{2}\biggr)^{2(l+n)-n}
=
(-1)^n
\sum_{l=0}^\infty
\frac{(-1)^l}{m!\,\Gamma(l+n+1)}\biggl(\frac{x}{2}\biggr)^{2l+n}
\\
&=
(-1)^n
J_{n}(x).
\end{align*}
Insbesondere unterscheiden sich $J_n(x)$ und $J_{-n}(x)$ nur durch
ein Vorzeichen.

Als lineare Differentialgleichung zweiter Ordnung erwarten wir noch
eine zweite, linear unabhängige Lösung.
Diese kann jedoch nicht allein mit der Potenzreihenmethode
bestimmt werden,
dazu sind die Methoden der Funktionentheorie nötig.
Im Abschnitt~\ref{buch:funktionentheorie:subsection:dglsing}
wird gezeigt, wie dies möglich ist und auf
Seite~\pageref{buch:funktionentheorie:subsubsection:bessel2art}
werden die damit zu findenden Bessel-Funktionen 0-ter Ordnung und
zweiter Art vorgestellt.

%
% Erzeugende Funktione
%
\subsubsection{Erzeugende Funktion}
\begin{figure}
\centering
\includegraphics{chapters/050-differential/images/besselgrid.pdf}
\caption{Indexmenge für Herleitung der erzeugenden Funktion der
Bessel-Funktionen.
Die rote Summe in \eqref{buch:differentialgleichungen:bessel:eqn:rotesumme}
entspricht den vertikalen roten Streifen oben,
die blaue Summe in
\eqref{buch:differentialgleichungen:bessel:eqn:blauesumme}
den horizontalen Streifen in der Abbildung unten.
Alle Terme enthalten $\Gamma(n+k+1)$ im Nenner,
im grau hinterlegten Gebiet verschwinden sie.
\label{buch:differentialgleichungen:bessel:fig:indexmenge}}
\end{figure}
Die erzeugende Funktion der Bessel-Funktionen ist die Summe
\begin{align}
\sum_{n\in\mathbb{Z}} J_n(x)z^n
&=
\sum_{n\in\mathbb{Z}}
{\color{darkred}
\sum_{k=0}^\infty
\frac{(-1)^k}{k!\,\Gamma(k+n+1)}
\biggl(\frac{x}{2}\biggr)^{2k+n}
}
z^n.
\label{buch:differentialgleichungen:bessel:eqn:rotesumme}
\intertext{Die rote Summe entspricht den vertikalen roten Streifen in
Abbildung~\ref{buch:differentialgleichungen:bessel:fig:indexmenge} oben.
Die grau hinterlegten Punkte in der Abbildung gehören zu verschwindenden
Termen.
Wir schreiben $m=k+n$ und drücken alle Terme durch $k$ und $m$ aus:}
&=
\sum_{n\in \mathbb{Z}}
\sum_{k=0}^\infty
\frac{(-1)^k}{k!\,\Gamma(n+k+1)}
\biggl(\frac{x}{2}\biggr)^k
\biggl(\frac{x}{2}\biggr)^{n+k}
z^{n+k}
z^{-k}
\notag
\\
&=
\sum_{m\in \mathbb{Z}}
\sum_{k=0}^\infty \frac{(-1)^k}{k!}
\biggl(\frac{x}{2}\biggr)^k
z^{-k}
\frac{1}{\Gamma(m+1)}
\biggl(\frac{x}{2}\biggr)^{m}
z^{n+k}.
\notag
\intertext{Auch in dieser Summe fallen wieder die Terme mit $m<0$
wegen $\Gamma(m+1)=\infty$ weg.
Die Grenzen der Summation über $k$ hängen nicht von $m$ ab, daher
können wir die Summationsreihenfolge vertauschen.
Die Summation über $m$ entspricht den horizontalen blauen Streifen
in 
Abbildung~\ref{buch:differentialgleichungen:bessel:fig:indexmenge}
unten.
Es ergibt sich die Summe}
&=
\sum_{k=0}^\infty
\sum_{m=0}^\infty
\frac{(-1)^k}{k!}
\biggl(\frac{x}{2}\biggr)^k
z^{-k}
\frac{1}{\Gamma(m+1)}
\biggl(\frac{x}{2}\biggr)^{m}
z^{m}
\notag
\\
&=
\sum_{k=0}^\infty \frac{(-1)^k}{k!}
\biggl(\frac{x}{2}\biggr)^k
z^{-k}
\cdot
{\color{blue}
\sum_{m=0}^\infty
\frac{1}{\Gamma(m+1)}
\biggl(\frac{x}{2}\biggr)^{m}
z^{m}
}.
\label{buch:differentialgleichungen:bessel:eqn:blauesumme}
\intertext{Beide Reihen sind Exponentialreihen, was man besser sehen kann,
wenn man die Gamma-Funktion in der zweiten Summe wieder als die
Fakultät $\Gamma(m+1)=m!$ schreibt.
Die beiden Exponentialreihen sind
}
&=
\sum_{k=0}^\infty \frac{\bigl(-\frac{x}2\cdot\frac1z\bigr)}{k!}
\cdot
\sum_{m=0}^\infty
\frac{\bigl(z\frac{x}2\bigr)^m}{m!}
=
\exp\biggl(\frac{x}2\cdot\biggl(-\frac1z\biggr)\biggr)
\cdot
\exp\biggl(\frac{x}2\cdot z\biggr)
=
\exp\biggl(\frac{x}2\cdot\biggl(z-\frac1z\biggr)\biggr).
\notag
\end{align}
Wir fassen das Resultat im folgenden Satz zusammen.

\begin{satz}[Erzeugende Funktion der Bessel-Funktionen]
Die erzeugende Funktion der Besselfunktionen ist
\[
\sum_{k=-\infty}^\infty
J_k(x)
=
\exp\biggl(
\frac{x}2\cdot\biggl(1-\frac1z\biggr)
\biggr)
\]
\end{satz}
\index{erzeugende Funktion}%

%
% Additionstheorem
%
\subsubsection{Additionstheorem}
Die erzeugende Funktion kann dazu verwendet werden, das Additionstheorem
für die Bessel-Funktionen zu beweisen.

\begin{satz}
\index{Satz!Additionstheorem für Bessel-Funktionen}%
Für $l\in\mathbb{Z}$ und $x,y\in\mathbb{R}$ gilt
\[
J_l(x+y) = \sum_{m=-\infty}^\infty J_m(x)J_{l-m}(y).
\]
\end{satz}

\begin{proof}[Beweis]
Die Koeffizienten der erzeugenden Funktion der Bessel-Funktionen für
das Argument $x+y$ ist
\begin{align*}
\exp\biggl(\frac{x+y}2\biggl(z+\frac1z\biggr)\biggr)
&=
\sum_{n=-\infty}^\infty J_n(x+y)z^n.
\intertext{%
Wir verwenden die Exponentialgesetze auf der linken Seite und 
erhalten}
&=
\exp\biggl(\frac{x}2\biggl(z+\frac1z\biggr)\biggr)
\cdot
\exp\biggl(\frac{y}2\biggl(z+\frac1z\biggr)\biggr).
\intertext{Beide Faktoren sind erzeugende Funktionen von Bessel-Funktionen,
wir können sie also als}
&=
\sum_{m=-\infty}^\infty J_m(x)z^m
\cdot
\sum_{k=-\infty}^\infty J_k(y)z^k
\intertext{schreiben.
Durch Ausmultiplizieren und Zusammenfassen von Termen mit gleichem
Exponenten finden wir
}
&=
\sum_{m,k} J_m(x)J_k(y) z^{k+m}
=
\sum_{l=-\infty}^\infty
\biggl(
\sum_{m=-\infty}^\infty J_m(x)J_{l-m}(y)
\biggr)
z^l.
\intertext{Daraus folgt schliesslich mit Koeffizientenvergleich das
Additionstheorem}
J_l(x+y) &= \sum_{m=-\infty}^\infty J_m(x)J_{l-m}(y)
\end{align*}
für alle $l$.
\end{proof}

%
% Der Fall \alpha=0
% 
\subsubsection{Der Fall $\alpha=0$}
Im Fall $\alpha=0$ hat das Indexpolynom eine doppelte Nullstelle, wir
können daher nur eine Lösung erwarten.
Im Fall $\alpha=0$ wird das Produkt im Nenner zu $n!$, so dass die
Lösungsfunktion
\[
J_0(x)
=
\sum_{k=0}^\infty
\frac{(-1)^k}{(k!)^2}
\biggl(\frac{x}{2}\biggr)^{2k}
\]
geschrieben werden kann.


%
% Der Fall \alpha=p, p\in \mathbb{N}
%
\subsubsection{Der Fall $\alpha=p$, $p\in\mathbb{N}, p > 0$}
In diesem Fall kann nur die erste
Lösung~\eqref{buch:differentialgleichunge:bessel:erste}
verwendet werden.
Damit erhält die Lösungsfunktion die Form
\[
J_p(x)
=
\sum_{k=0}^\infty
\frac{(-1)^k}{k!(p+k)!}\biggl(\frac{x}{2}\biggr)^{p+2k}.
\]

%
% Der Fall $\alpha=n+\frac12$
%
\subsubsection{Der Fall $\alpha=n+\frac12$, $n\in\mathbb{N}$}
Obwohl $2\alpha$ eine Ganzzahl ist, sind die beiden Lösungen
\label{buch:differentialgleichunge:bessel:erste}
und
\label{buch:differentialgleichunge:bessel:zweite}
linear unabhängig.

Man kann zeigen, dass sich die Lösungsfunktionen in diesem Fall
durch bereits bekannte elementare Funktionen ausdrücken lassen.
Wir rechnen dies für $n=0$ nach.
Zunächst drücken wir die Pochhammer-Symbole im Nenner anders aus.
Es ist
\begin{align*}
\biggl(\frac12 + 1\biggr)_k
&=
\biggl(\frac12 + 1\biggr)
\biggl(\frac12 + 2\biggr)
\cdots
\biggl(\frac12 + k\biggr)
=
\frac{1}{2^k}\bigl(3\cdot 5\cdot\ldots\cdot (2k+1)\bigr)
=
\frac{(2k+1)!}{2^{2k}\cdot k!}
\\
\biggl(-\frac12 + 1\biggr)_k
&=
\biggl(-\frac12 + 1\biggr)
\biggl(-\frac12 + 2\biggr)
\cdots
\biggl(-\frac12 + k\biggr)
\\
&=
\biggl(\frac12 + 0\biggr)
\biggl(\frac12 + 1\biggr)
\cdots
\biggl(\frac12 + k-1\biggr)
=
\frac{1}{2^k}\bigl(1\cdot 3 \cdot\ldots\cdot (2(k-1)+1)\bigr)
=
\frac{(2k-1)!}{2^{2k-1}\cdot (k-1)!}
\end{align*}
Damit können jetzt die Reihenentwicklungen der Lösung wie folgt
umgeformt werden
\begin{align*}
y_1(x)
&=
x^\alpha
\sum_{k=0}^\infty
\frac{1}{(\alpha+1)_k}
\frac{(-x^2/4)^k}{k!}
=
\sqrt{x}
\sum_{k=0}^\infty
\frac{2^{2k}k!}{(2k+1)!}
\frac{(-x^2/4)^k}{k!}
=
\sqrt{x}
\sum_{k=0}^\infty
(-1)^k
\frac{x^{2k}}{(2k+1)!}
\\
&=
\frac{1}{\sqrt{x}}
\sum_{k=0}^\infty
(-1)^k
\frac{x^{2k+1}}{(2k+1)!}
=
\frac{1}{\sqrt{x}} \sin x
\\
y_2(x)
&=
x^{-\alpha}
\sum_{k=0}^\infty
\frac{1}{(-\alpha+1)_k}
\frac{(-x^2/4)^k}{k!}
=
x^{-\frac12}
\sum_{k=0}^\infty
\frac{2^{2k-1}\cdot (k-1)!}{(2k-1)!}
\frac{(-x^2/4)^k}{k!}
\\
&=
\frac{1}{\sqrt{x}}
\sum_{k=0}^\infty
(-1)^k
\frac{x^{2k}}{(2k-1)!\cdot 2k}
=
\frac{1}{\sqrt{x}} \cos x.
\end{align*}

Die Bessel-Funktionen verwenden aber eine andere Normierung. 
Die Gleichung~\eqref{buch:differentialgleichungen:bessel:normierungsgleichung}
zeigt, dass die Bessel-Funktionen durch Division
der Funktion $y_1(x)$ und $y_2(x)$ durch $2^\alpha \Gamma(\alpha+1)$ 
erhalten werden können.
Dies ergibt
\begin{equation*}
\renewcommand{\arraycolsep}{1pt}
\begin{array}{rclclclcl}
J_{\frac12}(x)
&=&
\displaystyle\frac{1}{2^{\frac12}\Gamma(\frac12+1)}
y_1(x)
&=&
\displaystyle\frac{1}{2^{\frac12}\frac12\Gamma(\frac12)}
y_1(x)
&=&
\displaystyle\frac{\sqrt{2}}{\Gamma(\frac12)}
y_1(x)
&=&
\displaystyle\frac{1}{\Gamma(\frac12)}
\sqrt{ \frac{2}{x}}
\sin x,
\\
J_{-\frac12}(x)
&=&
\displaystyle\frac{1}{2^{-\frac12}\Gamma(-\frac12+1)}
y_2(x)
&=&
\displaystyle\frac{2^{\frac12}}{\Gamma(\frac12)}
y_2(x)
&=&
\displaystyle\frac{\sqrt{2}}{\Gamma(\frac12)}
y_2(x)
&=&
\displaystyle\frac{1}{\Gamma(\frac12)}
\sqrt{\frac{2}{x}}
\cos x.
\end{array}
\end{equation*}
Wegen $\Gamma(\frac12)=\sqrt{\pi}$ sind die
halbzahligen Bessel-Funktionen daher
\begin{align*}
J_{\frac12}(x)
&=
\sqrt{\frac{2}{\pi x}} \sin x
=
\sqrt{\frac{2}{\pi}} x^{-\frac12}\sin x
&
&\text{und}&
J_{-\frac12}(x)
&=
\sqrt{\frac{2}{\pi x}} \cos x
=
\sqrt{\frac{2}{\pi}} x^{-\frac12}\cos x.
\end{align*}

%
% Direkte Verifikation der Lösungen
%
\subsubsection{Direkte Verifikation der Lösungen für $\alpha=\pm\frac12$}
Tatsächlich führt die Anwendung des Bessel-Operators auf die beiden
Funktionen auf
\begin{align*}
\sqrt{\frac{\pi}2}
BJ_{\frac12}(x)
&=
\sqrt{\frac{\pi}2}
\biggl(
x^2J_{\frac12}''(x) + xJ_{\frac12}'(x) + x^2J_{\frac12}(x)
\biggr)
\\
&=
x^2(x^{-\frac12}\sin x)''
+
x(x^{-\frac12}\sin x)'
+
x^2(x^{-\frac12}\sin x)
\\
&=
x^2(
x^{-{\textstyle\frac12}}\cos x
-{\textstyle\frac12}x^{-\frac32}\sin x
)'
+
x(
x^{-\frac12}\cos x
-{\textstyle\frac12}x^{-\frac32}\sin x
)
+
x^{\frac32}\sin x
\\
&=
x^2(
-x^{-\frac12}\sin x
-{\textstyle\frac12}x^{-\frac32}\cos x
-{\textstyle\frac12}x^{-\frac32}\cos x
+{\textstyle\frac{3}{4}}x^{-\frac52}\sin x
)
+
x^{\frac12}\cos x
+
x^{-\frac12}(x-{\textstyle\frac12})\sin x
\\
&=
(
-x^{\frac32}
+{\textstyle\frac34}x^{-\frac12}
+x^{\frac32}
-{\textstyle\frac12}x^{-\frac12}
)
\sin x
=
\frac14x^{-\frac12}\sin x
=
\frac14
\sqrt{\frac{\pi}2}
J_{\frac12}(x)
\\
BJ_{\frac12}(x)
&=
\biggl(\frac12\biggr)^2 J_{\frac12}(x).
\end{align*}
Dies zeigt, dass $J_{\frac12}(x)$ tatsächlich eine Eigenfunktion
des Bessel-Operators zum Eigenwert $\alpha^2 = \frac14$ ist.
Analog kann man die Lösung $y_2(x)$ für $-\frac12$ verifizieren.


%
% hypergeometrisch.tex
%
% (c) 2021 Prof Dr Andreas Müller, OST Ostschweizer Fachhochschule
%
\section{Hypergeometrische Funktionen
\label{buch:rekursion:section:hypergeometrische-funktion}}
\rhead{Hypergeometrische Funktionen}
Kann man eine Formel für die Lösung $S_n$ der lineare Differenzengleichung
\[
n^3S_{n}
=
16(n-{\textstyle\frac12})(2n^2-2n+1)S_{n-1}
-256(n-1)^3S_3
\]
mit Anfangswerten $S_0=1$ und $S_1=8$ angeben?
Dies scheint auf den ersten Blick unmöglich kompliziert, man kann aber
zeigen, dass
\begin{equation}
S_n
=
\sum_{k=0}^n 
\binom{2n-2k}{n-k}^2 \binom{2k}{k}^2
\label{buch:rekursion:hypergeometrisch:eqn:Sn}
\end{equation}
gilt (\cite[p.~xi]{buch:ab}).
Die Lösung ist also eine Summe von Summanden, die sehr viel einfacher
aussehen und vor allem die besondere Eigenschaft haben, dass die
Quotienten aufeinanderfolgender Terme rationale Funktionen von $k$
sind.

\begin{definition}
Ein Folge heisst {\em hypergeometrisch}, wenn der Quotient aufeinanderfolgender
\index{hypergeometrische Folge}%
\index{Folge, hypergeometrisch}%
Terme eine rationale Funktion des Folgenindex ist.
\end{definition}

Die Terme der Reihenentwicklungen aller bisher behandelten speziellen
Funktionen waren hypergeometrisch.
Im aktuellen Abschnitt soll daher die Klasse der sogenannten
hypergeometrischen Funktionen untersucht werden, die durch diese
Eigenschaft charakterisiert sind.

In Abschnitt~\ref{buch:rekursion:hypergeometrisch:binomialkoeffizienten}
wird klar, dass Folgen, deren Terme aus Fakultäten und Binomialkoeffizienten
immer hypergeometrisch sind.
Die Untersuchung der geometrischen Reihe in
Abschnitt~\ref{buch:rekursion:hypergeometrisch:geometrisch}
motiviert die Namensgebung.
Abschnitt~\ref{buch:rekursion:hypergeometrisch:reihen}
definiert den Begriff der hypergeometrischen Reihe und zeigt, 
wie sie in eine Standardform gebracht werden können.
In Abschnitt~\ref{buch:rekursion:hypergeometrisch:beispiele}
schliesslich wird an Hand von Beispielen gezeigt, wie bekannte
Funktionen als hypergeometrische Funktionen interpretiert werden können.

%
% Quotienten von Binomialkoeffizienten
%
\subsection{Quotienten von Binomialkoeffizienten
\label{buch:rekursion:hypergeometrisch:binomialkoeffizienten}}
Aufeinanderfolgende Terme der Summe
\eqref{buch:rekursion:hypergeometrisch:eqn:Sn}
sollen als Quotienten eine rationale Funktion haben.
Dies ist eine allgemeine Eigenschaft von Folgen, die durch Fakultäten
oder Binomialkoeffizienten definiert sind, wie die beiden folgenden
Sätze zeigen.

\begin{satz}
\index{Satz!Quotienten von Fakultäten}%
\label{buch:rekursion:hypergeometrisch:satz:fakquo}
Der Quotient aufeinanderfolgender Folgenglieder
der Folge $c_k=(a+bk)!$ ist der ein Polynom vom Grad $b$.
\end{satz}
\begin{proof}[Beweis]
\begin{align*}
\frac{c_{k+1}}{c_k}
&=
\frac{(a+b(k+1))!}{(a+bk)!}
=
\frac{(a+bk+b)!}{(a+b)!}
\\
&=
(a+bk+1)(a+bk+2)\cdots(a+bk+b)
=
(a+bk+1)_b
\end{align*}
Das Pochhammer-Symbol hat $b$ Faktoren, es ist ein Polynom vom Grad $b$.
\end{proof}

\begin{satz}
\index{Satz!Quotienten von Binomialkoeffizienten}%
\label{buch:rekursion:hypergeometrisch:satz:binomquo}
Die Quotienten aufeinanderfolgender Werte der Binomialkoeffizienten
\[
f_k
=
\binom{a+bk}{c+dk}
\]
ist eine rationale Funktion von $k$ mit Zähler- und Nennergrad $b$.
\end{satz}

\begin{proof}[Beweis]
Indem man die Binomialkoeffizienten mit Fakultäten als
\[
\binom{a+bk}{c+dk}
=
\frac{(a+bk)!}{(c+dk)!(a-c+(b-d)k)!}
\]
ausschreibt, findet man mit
Satz~\ref{buch:rekursion:hypergeometrisch:satz:fakquo}
für die Quotienten
\begin{align}
\frac{f_{k+1}}{f_k}
&=
\frac{(a+bk+1)_b}{(c+dk+1)_d\cdot(a-c+(b-d)k+1)_{b-d}}
\label{buch:rekursion:eqn:binomquotient}
\end{align}
Die Pochhammer-Symbole sind Polynome vom Grad $b$, $d$ bzw.~$b-d$.
Insbesondere ist auch das Nenner-Polynom vom Grad $d+(b-d)=b$.
\end{proof}

Aus den Sätzen~\ref{buch:rekursion:hypergeometrisch:satz:fakquo}
und
\ref{buch:rekursion:hypergeometrisch:satz:binomquo}
folgt jetzt sofort, dass auch der Quotient aufeinanderfolgender
Summanden der Summe~\eqref{buch:rekursion:hypergeometrisch:eqn:Sn}
eine rationale Funktion von $k$ ist.

%
% Die geometrische Reihe
%
\subsection{Die geometrische Reihe
\label{buch:rekursion:hypergeometrisch:geometrisch}}
Die Reihe
\[
f(q)
=
\sum_{k=0}^\infty aq^k
\]
heisst die {\em geometrische Reihe} ist besonders einfache
Reihe mit einer hypergeometrischen Folge von Termen.
\index{geometrische Reihe}%
\index{Reihe!geometrische}%
Die Partialsummen 
\[
S_n
=
\sum_{k=0}^n aq^k
\]
können aus der Differenz
\begin{equation}
(1-q)S_n
=
S_n - qS_n
=
\sum_{k=0}^n aq^k
-
\sum_{k=1}^{n+1} aq^k
=
a -aq^{n+1}
\label{buch:rekursion:hypergeometrisch:eqn:qsumme}
\end{equation}
berechnet werden, die man nach
\begin{equation}
S_n 
=
a\frac{1-q^{n+1}}{1-q}
\label{buch:rekursion:hypergeometrisch:eqn:geomsumme}
\end{equation}
auflösen kann.
Für $q<1$ geht $q^n\to 0$ und damit konvergiert
$S_n$  gegen
\[
\sum_{k=0}^\infty aq^k
=
a\frac{1}{1-q}.
\]

Die geometrische Reihe ist charakterisiert dadurch, dass aufeinanderfolgende
Terme den gleichen Quotienten
\[
\frac{aq^{k+1}}{aq^k}
=
q
\]
haben.
Die Berechnung der Summe in 
\eqref{buch:rekursion:hypergeometrisch:eqn:qsumme}
beruht darauf, dass die Multiplikation mit $q$ einen ``anderen''
Teil der Summe ergibt, der sich in der Differenze weghebt.

%
% Hypergeometrische Reihen
%
\subsection{Hypergeometrische Reihen
\label{buch:rekursion:hypergeometrisch:reihen}}
Es ist plausibel, dass eine etwas lockerere Bedingung an die
Quotienten aufeinanderfolgender Terme einer Reihe immer noch
ermöglichen wird, interessante Aussagen über die durch die
Reihe beschriebenen Funktionen zu machen.

\begin{definition}
\label{buch:rekursion:hypergeometrisch:def:allg}
Eine durch die Reihe
\[
f(x) = \sum_{k=0}^\infty a_k x^k
\]
definierte Funktion $f(x)$ heisst {\em hypergeometrisch},
wenn der Quotient aufeinanderfolgender
\index{hypergeometrisch}
\index{Reihe!hypergeometrisch}
Koeffizienten eine rationale Funktion von $k$ ist,
wenn also
\[
\frac{a_{k+1}}{a_k}
=
\frac{p(k)}{q(k)}
\]
mit Polynomen $p(k)$ und $q(k)$ ist.
\end{definition}

%
% Beispiele von hypergeometrischen Funktionen
%
\subsubsection{Beispiele von hypergeometrischen Funktionen}
Die geometrische Reihe ist natürlich eine hypergeometrische Reihe,
wobei $p(k)/q(k)=1$ ist.
Etwas interessanter ist die Exponentialfunktion, die durch die Taylor-Reihe
\[
e^x = \sum_{k=0}^\infty \frac{x^k}{k!}
\]
dargestellt werden kann.
Der Quotient aufeinanderfolgender Koeffizienten ist
\[
\frac{a_{k+1}}{a_k}
=
\frac{1/(k+1)!}{1/k!}
=
\frac{k!}{(k+1)!}
=
\frac{1}{k+1},
\]
eine rationale Funktion mit Zählergrad $0$ und Nennergrad $1$.

Die Kosinus-Funktion wird durch die Taylor-Reihe
\[
\cos x = \sum_{k=0}^\infty \frac{(-1)^k}{(2k)!} x^{2k}
\]
dargestellt.
Als Potenzreihe in $x$ kann die Kosinus-Reihe nicht hypergeometrisch sein,
die ungeraden Koeffizienten verschwinden und damit undefinierte
Quotienten haben.
Als Reihe in $z=x^2$ ist aber
\[
\sum_{k=0}^\infty \frac{(-1)^k}{(2k)!} z^k
\qquad\Rightarrow\qquad
a_k = \frac{(-1)^k}{(2k)!}
\]
hypergeometrisch, weil der Quotient aufeinanderfolgender Koeffizienten
\[
\frac{a_{k+1}}{a_k}
=
\frac{(-1)^{k+1}}{(2k+2)!}\cdot \frac{(2k)!}{(-1)^k}
=
-\frac{1}{(2k+2)(2k+1)},
\]
eine rationale Funktion mit Zählergrad $0$ und Nennergrad $2$.
Es gibt also eine hypergeometrische Reihe $f(z)$ derart, dass
$\cos x = f(x^2)$ ist.

%
% Die hypergeometrischen Funktione pFq
%
\subsubsection{Die hypergeometrischen Funktionen $\mathstrut_pF_q$}
Die Definition~\ref{buch:rekursion:hypergeometrisch:def:allg}
einer hypergeometrischen Funktion wie auch die Verschiedenartigkeit
der Beispiele kännen den Eindruck vermitteln, dass die diese Klasse
von Funktionen unübersichtlich gross sein könnte.
Dem ist jedoch nicht so.
In diesem Abschnitt soll gezeigt werden, dass alle hypergeometrischen
Funktionen durch die in
Definition~\ref{buch:rekursion:hypergeometrisch:def} definierten
Funktionen $\mathstrut_pF_q$ ausgedrückt werden.
Die hypergeometrischen Funktionen können also vollständig parametrisiert
werden.

Zu diesem Zweick sie
\[
f(x)
=
\sum_{k=0}^\infty a_kx^k
\]
eine hypergeometrische Funktion und
seien $p(k)$ und $q(k)$ zwei Polynome derart, dass
\[
\frac{a_{k+1}}{a_k} = \frac{p(k)}{q(k)}.
\]
Daraus lässt sich der Koeffizient $a_{k+1}$ als
\begin{equation}
a_{k+1}
=
\frac{p(k)}{q(k)}
\cdot
a_k
=
\frac{p(k)}{q(k)}
\cdot
\frac{p(k-1)}{q(k-1)}
\cdot
a_{k-1}
=\dots=
\frac{p(k)}{q(k)}
\frac{p(k-1)}{q(k-1)}
\cdots
\frac{p(1)}{q(1)}
\frac{p(0)}{q(0)}
a_0
\label{buch:rekursion:hypergeometrisch:ak+1}
\end{equation}
berechnen.
Alle Koeffizienten haben also den Faktor $a_0=f(0)$ gemeinsam.

Die Produkte von Quotienten $p(k)/q(k)$ sollen jetzt weiter
vereinfacht werden.
Sei $n$ der Grad von $p(k)$ und $m$ der Grad von $q(k)$.
Dazu nehmen wir an, dass $a_i$, $i=1,\dots,n$ die Nullstellen von $p(k)$ sind
und $b_j$, $j=1,\dots,m$ die Nullstellen von $q(k)$, dass man also
die Polynome als
\begin{align*}
p(k) &= s(k-a_1)(k-a_2)\cdots(k-a_n)
\\
q(k) &= (k-b_1)(k-b_2)\cdots(k-b_m)
\end{align*}
schreiben kann.
Der Faktor $s$ ist nötig, weil die Polynome $p(k)$ und $q(k)$ nicht
notwendigerweise normiert sind.

Um das Produkt der Quotienten zu vereinfachen, nehmen wir für den Moment
an, dass Zähler und Nenner vom Grad $n=m=1$ ist.
Dann ist nach 
\eqref{buch:rekursion:hypergeometrisch:ak+1}
\[
a_{k}
=
s^{k}
\frac{
(k-1-a_1) \cdots (2-a_1)(1-a_1)(0-a_1)
}{
(k-1-b_1) \cdots (2-b_1)(1-b_1)(0-b_1)
}
=
\frac{(-a_1)_k}{(-b_1)_k} s^k.
\]
Die Koeffizienten können daher als Quotienten von Pochhammer-Symbolen
geschrieben werden.
Für Polynome $p(k)$ und $q(k)$ höheren Grades sind die Koeffizienten
von der Form
\[
a_k
=
\frac{(-a_1)_k(-a_2)_k\cdots (-a_n)_k}{(-b_1)_k(-b_2)_k\cdots(-b_m)_k}
s^ka_0.
\]
Jede hypergeometrische Funktion kann daher in der Form
\[
f(x)
=
a_0
\sum_{k=0}^\infty
\frac{(-a_1)_k(-a_2)_k\cdots (-a_n)_k}{(-b_1)_k(-b_2)_k\cdots(-b_m)_k}
s^k
x^k
\]
geschrieben werden.

\begin{definition}
\label{buch:rekursion:hypergeometrisch:def}
Die hypergeometrische Funktion
$\mathstrut_pF_q$ ist definiert durch die Reihe
\[
\mathstrut_pF_q
\biggl(
\begin{matrix}
a_1,\dots,a_p\\
b_1,\dots,b_q
\end{matrix}
;
x
\biggr)
=
\mathstrut_pF_q(a_1,\dots,a_p;b_1,\dots,b_q;x)
=
\sum_{k=0}^\infty
\frac{(a_1)_k\cdots(a_p)_k}{(b_1)_k\cdots(b_q)_k}\frac{x^k}{k!}.
\]
\end{definition}

Da $(1)_k=k!$ hätte die Definition den Nenner $k!$ in der Reihe
auch durch eines der Pochhammer-Symbole ausdrücken können.
Wird dieser Nenner nicht gebraucht, kann man ihn durch einen 
zusätzlichen Faktor $(1)_k$ im Zähler des Bruchs von Pochhammer-Symbolen
kompensieren, wodurch sich der Grad $p$ des Zählers natürlich um $1$
erhöht.

Die oben analysierte Summe für $f(x)$ kann mit der
Definition~\ref{buch:rekursion:hypergeometrisch:def} als
\[
f(x)
=
a_0
\cdot
\mathstrut_{n+1}F_m \biggl(
\begin{matrix}
-a_1,-a_2,\dots,-a_n,1\\
-b_1,-b_2,\dots,-a_m
\end{matrix}; sx
\biggr)
\]
beschrieben werden.

%
% Elementare Rechenregeln
%
\subsubsection{Elementare Rechenregeln}
Die Funktionen $\mathstrut_pF_q$ sind nicht alle unabhängig.
In Abschnitt~\ref{buch:rekursion:hypergeometrisch:stammableitung}
wird gezeigt werden, dass Ableitung und Stammfunktion einer hypergeometrischen
Funktion durch Manipulation der Parameter $a_k$ und $b_k$ bestimmt werden
können.
Viel einfacher sind jedoch die folgenden, aus
Definition~\ref{buch:rekursion:hypergeometrisch:def}
offensichtlichen Regeln:

\begin{satz}[Permutationsregel]
\index{Satz!Permutationsregel für hypergeometrische Funktionen}%
\label{buch:rekursion:hypergeometrisch:satz:permuationsregel}
Sei $\pi$ eine beliebige Permutation der Zahlen $1,\dots,p$ und $\sigma$ eine
beliebige Permutation der Zahlen $1,\dots,q$, dann ist 
\begin{equation}
\mathstrut_pF_q\biggl(
\begin{matrix}
a_1,\dots,a_p\\b_1,\dots,a_q
\end{matrix}
;x
\biggr)
=
\mathstrut_pF_q\biggl(
\begin{matrix}
a_{\pi(1)},\dots,a_{\pi(p)}\\b_{\sigma(1)},\dots,b_{\sigma(q)}
\end{matrix}
;x
\biggr).
\label{buch:rekursion:hypergeometrisch:eqn:permuationsregel}
\end{equation}
\end{satz}

\begin{satz}[Kürzungsformel]
\index{Satz!Kürzungsformel für hypergeometrische Funktionen}%
\label{buch:rekursion:hypergeometrisch:satz:kuerzungsregel}
Stimmt einer der Koeffizienten $a_k$ mit einem der Koeffizienten $b_i$
überein, dann können sie weggelassen werden:
\begin{equation}
\mathstrut_{p+1}F_{q+1}\biggl(
\begin{matrix}
c,a_1,\dots,a_p\\
c,b_1,\dots,b_q
\end{matrix};
x
\biggr)
=
\mathstrut_{p}F_{q}\biggl(
\begin{matrix}
a_1,\dots,a_p\\
b_1,\dots,b_q
\end{matrix};
x
\biggr).
\label{buch:rekursion:hypergeometrisch:eqn:kuerzungsregel}
\end{equation}
\end{satz}

%
% Beispiele von hypergeometrischen Funktionen
%
\subsection{Beispiele von hypergeometrischen Funktionen
\label{buch:rekursion:hypergeometrisch:beispiele}}
Viele der bekannten Reihenentwicklungen häufig verwendeter Funktionen
lassen sich durch die hypergeometrischen Funktionen von
Definition~\ref{buch:rekursion:hypergeometrisch:def} ausdrücken.
In diesem Abschnitt werden einige Beispiel dazu gegeben.

%
% Die geometrische Reihe
%
\subsubsection{Die geometrische Reihe}
In der geometrischen Reihe fehlt der Nenner $k!$, es braucht
daher einen Term $(1)_k$ im Zähler, um den Nenner zu kompensieren.
Somit ist die geometrische Reihe
\[
\frac{a}{1-x}
=
\sum_{k=0}^\infty
ax^k
=
a\sum_{k=0}^\infty
\frac{(1)_k}{1}
\frac{x^k}{k!}
=
a\cdot\mathstrut_1F_0(1,x).
\]

%
% Die Exponentialfunktion
%
\subsubsection{Exponentialfunktion}
Die Exponentialfunktion ist die Reihe
\[
e^x = \sum_{k=0}^\infty \frac{x^k}{k!}.
\]
In diesem Fall werden keine Quotienten von Pochhammer-Symbolen
benötigt, es ist daher
\[
e^x = \mathstrut_0F_0(x).
\]

%
% Wurzelfunktionen
%
\subsubsection{Wurzelfunktionen}
Die Wurzelfunktion $x\mapsto \sqrt{x}$ hat keine Taylor-Entwicklung
in $x=0$, aber die Funktion $x\mapsto\sqrt{1+x}$ hat die Taylor-Reihe
\[
\sqrt{1+x}
=
1
+
\frac12 x
-
\frac{1\cdot 1}{2\cdot 4}x^2
+
\frac{1\cdot 1\cdot 3}{2\cdot 4\cdot 6}x^3
-
\frac{1\cdot 1\cdot 3\cdot 5}{2\cdot 4\cdot 6\cdot 8}x^4
+
\dots
\]
Um die Verbindung zu einer hypergeometrischen Funktion herzustellen,
müssen wir den Term $x^k/k!$ abspalten.
Dann wird
\begin{align*}
\sqrt{1+x}
&=
1
+
\frac12 \frac{x}{1!}
-
\frac{1\cdot 1}{2^2}\frac{x^2}{2!}
+
\frac{1\cdot 1\cdot 3}{2^3}\frac{x^3}{3!}
-
\frac{1\cdot 1\cdot 3\cdot 5}{2^4}\frac{x^4}{4!}
+
\dots
\\
&=
1
+
\frac12 \cdot\frac{x}{1!}
-
\frac{1}{2}\cdot \frac{1}{2}\cdot\frac{x^2}{2!}
+
\frac{1}{2}\cdot \frac{1}2\cdot \frac{3}{2}\cdot\frac{x^3}{3!}
-
\frac{1}{2}\cdot \frac{1}{2}\cdot \frac{3}{2}\cdot \frac{5}{2}\cdot\frac{x^4}{4!}
+
\dots
\end{align*}
Es ist noch etwas undurchsichtig, warum die ersten beiden Terme
das gleiche Vorzeichen haben und warum der Faktor $\frac12$ in jedem
Term zweimal vorkommt.
Diese Unklarheit kann jedoch beseitigt werden, wenn man den ersten
Faktor als $-\frac12$ schreibt:
\begin{align*}
\sqrt{1+x}
&=
1
-
\biggl(-\frac12\biggr)\cdot\frac{x}{1!}
+
\biggl(-\frac{1}{2}\biggr)\cdot \frac{1}{2}\cdot\frac{x^2}{2!}
-
\biggl(-\frac{1}{2}\biggr)\cdot \frac{1}2\cdot \frac{3}{2}\cdot\frac{x^3}{3!}
+
\biggl(-\frac{1}{2}\biggr)\cdot \frac{1}{2}\cdot \frac{3}{2}\cdot \frac{5}{2}\cdot\frac{x^4}{4!}
+
\dots
\\
&=
1 + 
\biggl(-\frac12\biggr)\cdot\frac{-x}{1!}
+
\biggl(-\frac{1}{2}\biggr)\cdot \frac{1}{2}\cdot\frac{(-x)^2}{2!}
+
\biggl(-\frac{1}{2}\biggr)\cdot \frac{1}2\cdot \frac{3}{2}\cdot\frac{(-x)^3}{3!}
+
\biggl(-\frac{1}{2}\biggr)\cdot \frac{1}{2}\cdot \frac{3}{2}\cdot \frac{5}{2}\cdot\frac{(-x)^4}{4!}
+
\dots
\end{align*}
Die Koeffizienten sind aufsteigende Produkte mit $k$ Faktoren, die alle bei
$-\frac12$ beginnen, sie können daher als Pochhammer-Symbole $(-\frac12)_k$
geschrieben werden.
Die Wurzelfunktion ist daher die hypergeometrische Funktion
\[
\sqrt{1\pm x}
=
\sum_{k=0}^\infty
\biggl(-\frac12\biggr)_k \frac{(\pm x)^k}{k!}
=
\mathstrut_1F_0(-{\textstyle\frac12};\mp x).
\]
Mit der Newtonschen Binomialreihe, die in 
Abschnitt~\ref{buch:differentialgleichungen:subsection:newtonschereihe}
hergleitet wird,
kann man ganz analog jede beliebige Wurzelfunktion
\begin{align*}
(1+x)^\alpha
&=
1+\alpha x + \frac{\alpha(\alpha-1)}{2!}x^2 + \frac{\alpha(\alpha-1)(\alpha-2)}{3!}x^3+\dots
%\\
%&
=
\sum_{k=0}^\infty \frac{(-\alpha)_k}{k!}x^k
=
\mathstrut_1F_0\biggl(\begin{matrix}-\alpha\\\text{---}\end{matrix};-x\biggr)
\end{align*}
durch $\mathstrut_1F_0$ ausdrücken.
Dieses Resultat ist der Inhalt von
Satz~\ref{buch:differentialgleichungen:satz:newtonschereihe}


%
% Logarithmusfunktion
%
\subsubsection{Logarithmusfunktion}
Für $x\in (-1,1)$ konvergiert die Taylor-Reihe
\[
\log(1+x)
=
x-\frac{x^2}{2}+\frac{x^3}{3}-\frac{x^4}{4}+\dots
\]
der Logarithmusfunktion im Punkt $x=0$.
Die Reihe beginnt nicht mit einem konstanten Term, daher klammern wir
zunächst einen Faktor $x$ aus:
\[
\log(1+x)
=
x\cdot
\biggl(
1-\frac{x}{2}+\frac{x^2}{3}-\frac{x^3}{4}+\dots
\biggr)
\]
Um dies in die Form einer hypergeometrischen Funktion zu bringen,
muss zunächst wieder der Nenner $k!$ hergestellt werden.
\begin{align*}
\log(1+x)
&=
x\cdot\biggl(
1
- \frac{1!}{2} \frac{x}{1!}
+ \frac{2!}{3} \frac{x^2}{2!} 
- \frac{3!}{4} \frac{x^3}{3!}+\dots
\biggr).
\intertext{Den Nenner $k+1$ kann man als Quotienten $k!/(k+1)!$ erhalten,
also}
\log(1+x)
&=
x\cdot\biggl(
1
- \frac{(1!)^2}{2!} \frac{x}{1!}
+ \frac{(2!)^2}{3!} \frac{x^2}{2!} 
- \frac{(3!)^2}{4!} \frac{x^3}{3!}+\dots
\biggr).
\end{align*}
Die Fakultät
\[
(k+1)!
=
1\cdot 2 \cdot 3 \cdot\ldots\cdot k\cdot (k+1)
=
2 \cdot (2 + 1) \cdot (2+2) \cdot\ldots\cdot (2+k-2) \cdot (2+k-1)
=
(2)_{k}
\]
ist auch ein Pochhammer-Symbol, so dass die Logarithmusfunktion
zur hypergeometrischen Funktion
\[
\log(1+x)
=
x\cdot\biggl(
1
+ \frac{(1)_1(1)_1}{(2)_1} \frac{(-x)}{1!}
+ \frac{(1)_2(1)_2}{(2)_2} \frac{(-x)^2}{2!} 
+ \frac{(1)_3(1)_3}{(2)_2} \frac{(-x)^3}{3!}+\dots
\biggr)
=
x\cdot
\mathstrut_2F_1\biggl(\begin{matrix}1,1\\2\end{matrix};-x\biggr).
\]

%
% Trigonometrische Funktionen
%
\subsubsection{Trigonometrische Funktionen}
\index{trigonometrische Funktionen!als hypergeometrische Funktionen}%
Die Kosinus-Funktion wurde bereits als hypergeometrische Funktion erkannt,
im Folgenden soll dies auch noch für die Sinus-Funktion
durchgeführt werden.
Die Taylor-Reihe der Sinus-Funktion im Punkt $0$ ist
\begin{align*}
\sin x
&=
x-\frac{x^3}{3!}+\frac{x^5}{5!}-\frac{x^7}{7!}+\dots
\end{align*}
In dieser Reihe fehlen die geraden Potenzen, wir Klammern daher einen
Faktor $x$ aus und schreiben den Rest als eine Funktion von $-x^2$
\begin{align*}
\sin x
&=
x
\biggl(
1+\frac{-x^2}{3!}+\frac{(-x^2)^2}{5!}-\frac{(-x^2)^3}{7!}+\dots
\biggr)
=
x f(-x^2).
\end{align*}
Die Funktion $f(z)$ soll jetzt als hypergeometrische Funktion geschrieben
werden.
Dazu muss zunächst wieder der Nenner $k!$ wiederhergestellt werden:
\begin{equation*}
f(z)
=
1
+
\frac{1!}{3!}\cdot \frac{z}{1!}
+
\frac{2!}{5!}\cdot \frac{z^2}{2!}
+
\frac{3!}{7!}\cdot \frac{z^3}{3!}
+
\dots
\end{equation*}
Die Koeffizienten $k!/(2k+1)!$ müssen jetzt durch Pochhammer-Symbole
mit jeweils $k$ Faktoren ausgedrückt werden.
Dazu muss die Fakultät $(2k+1)!$ in zwei Produkte
\[
(2k+1)!
=
2\cdot 3 \cdot 4\cdot 5\cdot \ldots \cdot 2k \cdot (2k+1)
=
\underbrace{(2\cdot 4 \cdot 6\cdot\ldots\cdot 2k)}_{\textstyle\text{gerade Faktoren}}
\cdot
\underbrace{(3\cdot 5\cdot 7\cdot \ldots \cdot (2k+1))}_{\textstyle\text{ungerade Faktoren}}
\]
aufgespaltet werden.
Diese Produkte haben zwar jeweils $k$ Faktoren, aber sie sind keine
Pochhammer-Symbole, weil die Differenz aufeinanderfolgender Faktoren 
jeweils $2$ ist.
Wir dividieren sowohl die geraden Faktoren wie auch die 
ungeraden Faktoren durch $2$, damit sich das Produkt nicht ändert,
müssen wird mit $2^{2k}$ kompensieren:
\begin{align*}
(2k+1)!
&=
2^k(1\cdot2\cdot3\cdot\ldots\cdot k)
\cdot
2^k
\biggl(
\frac{3}{2}\cdot
\frac{5}{2}\cdot
\frac{7}{2}\cdot
\ldots\cdot
\frac{2k+1}{2}
\biggr)
\\
&=
4^k
\cdot
(1)_k\cdot \biggl(\frac{3}{2}\biggr)_k
\end{align*}
Setzt man dies in die Reihe ein, wird
\begin{equation}
f(z)
=
\sum_{k=0}^\infty
\frac{(1)_k}{(1)_k\cdot (\frac{3}{2})_k\cdot 4^k}
z^k
=
\mathstrut_1F_2\biggl(
\begin{matrix}1\\1,\frac{3}{2}\end{matrix};\frac{z}4
\biggr)
=
\mathstrut_0F_1\biggl(
\begin{matrix}\text{---}\\\frac{3}{2}\end{matrix};\frac{z}4
\biggr).
\label{buch:rekursion:hyperbolisch:eqn:hilfsfunktionf}
\end{equation}
Im letzten Schritt wurde die Kürzungsregel
\eqref{buch:rekursion:hypergeometrisch:eqn:kuerzungsregel}
von
Satz~\ref{buch:rekursion:hypergeometrisch:satz:kuerzungsregel}
angewendet.
Damit lässt sich die Sinus-Funktion als
\begin{equation}
\sin x
=
x\cdot \mathstrut_1F_2\biggl(
\begin{matrix}1\\1,\frac32\end{matrix};-\frac{x^2}4
\biggr)
=
x\cdot\mathstrut_0F_1\biggl(
\begin{matrix}\text{---}\\\frac32\end{matrix};-\frac{x^2}4
\biggr)
\label{buch:rekursion:hypergeometrisch:eqn:sinhyper}
\end{equation}
durch eine hypergeometrische Funktion ausdrücken.

%
% Hyperbolische Funktionen
%
\subsubsection{Hyperbolische Funktionen}
\index{hyperbolische Funktionen!als hypergeometrische Funktionen}%
Die für die Sinus-Funktion angewendete Methode lässt sich auch
auf die Funktion 
\begin{align*}
\sinh x
&=
\sum_{k=0}^\infty \frac{x^{2k+1}}{(2k+1)!}
%\\
%&
=
x
\,
\biggl(
1+\frac{x^2}{3!} + \frac{x^4}{5!}+\frac{x^6}{7!}+\dots
\biggr)
\intertext{Die Reihe in der Klammer lässt sich mit der Funktion
$f$ von \eqref{buch:rekursion:hyperbolisch:eqn:hilfsfunktionf}
schreiben als}
&=
x\,f(-x^2)
%=
%x\cdot\mathstrut_1F_2\biggl(
%\begin{matrix}1\\1,\frac{3}{2}\end{matrix}
%;\frac{x^2}{4}
%\biggr)
=
x\cdot\mathstrut_0F_1\biggl(
\begin{matrix}\text{---}\\\frac{3}{2}\end{matrix}
;\frac{x^2}4
\biggr).
\end{align*}
Bis auf das Vorzeichen des Arguments der hypergeometrischen Funktion
ist diese Darstellung identisch mit der von $\sin x$.
Dies illustriert die Rolle der hypergeometrischen Funktionen als
``grosse Vereinheitlichung'' der bekannten speziellen Funktionen.

%
% Tschebyscheff-Polynome
%
\subsubsection{Tschebyscheff-Polynome}
\index{Tschebyscheff-Polynome}%
Man kann zeigen, dass auch die Tschebyscheff-Polynome sich durch die
hypergeometrischen Funktionen
\begin{equation}
T_n(x)
=
\mathstrut_2F_1\biggl(
\begin{matrix}-n,n\\\frac12\end{matrix}
;
\frac12(1-x)
\biggr)
\label{buch:rekursion:hypergeometrisch:tschebyscheff2f1}
\end{equation}
ausdrücken lassen.
Beweisen kann man diese Beziehung zum Beispiel mit Hilfe der
Differentialgleichungen, denen die Funktionen genügen.
Diese Methode wird in
Abschnitt~\ref{buch:differentialgleichungen:section:hypergeometrisch}
von Kapitel~\ref{buch:chapter:differential} vorgestellt.

Die Tschebyscheff-Polynome sind nicht die einzigen Familien von Polynomen,
\index{Tschebyscheff-Polynome!als hypergeometrische Funktion}
die sich durch $\mathstrut_pF_q$ ausdrücken lassen.
Für die zahlreichen Familien von orthogonalen Polynomen, die in
Kapitel~\ref{buch:chapter:orthogonalitaet} untersucht werden,
trifft dies auch zu.
Ein Funktion
\[
\mathstrut_pF_q
\biggl(
\begin{matrix}
a_1,\dots,a_p\\
b_1,\dots,b_q
\end{matrix}
;z
\biggr)
\]
ist genau dann ein Polynom, wenn mindestens einer der Parameter
$a_k$ eine negative ganze Zahl ist.
Der Grad des Polynoms ist der kleinste Betrag der negativ ganzzahligen
Werte unter den Parametern $a_k$.

%
% Die Funktionen 0F1
%
\subsubsection{Die Funktionen $\mathstrut_0F_1$}
\begin{figure}
\centering
\includegraphics{chapters/040-rekursion/images/0f1.pdf}
\caption{Graphen der Funktionen $\mathstrut_0F_1(;\alpha;x)$ für
verschiedene Werte von $\alpha$.
\label{buch:rekursion:hypergeometrisch:0f1}}
\end{figure}
Die Funktionen $\mathstrut_0F_1$ sind in den Beispielen mit der
beschränkten trigonometrischen Funktion $\sin x$ und mit der
exponentiell unbeschränkten Funktion $\sinh x$ mit dem gleichen
Wert des Parameters und nur einem Wechsel des Vorzeichens des
Arguments verbunden worden.
Die Graphen der Funktionen $\mathstrut_0F_1$, die in 
Abbildung~\ref{buch:rekursion:hypergeometrisch:0f1} dargestellt sind,
machen dieses Verhalten plausibel.
Es wird sich später zeigen, dass $\mathstrut_0F_1$ auch mit den Bessel-
und den Airy-Funktionen verwandt sind.


%
% Ableitung und Stammfunktion
%
\subsection{Ableitung und Stammfunktion hypergeometrischer Funktionen
\label{buch:rekursion:hypergeometrisch:stammableitung}}
Sowohl Ableitung wie auch Stammfunktion einer hypergeometrischen
Funktion lässt sich immer durch hypergeometrische Reihen ausdrücken.
%
% Ableitung
%
\subsubsection{Ableitung}
Wir gehen aus von der Funktion
\begin{equation}
f(x)
=
\mathstrut_nF_m\biggl(
\begin{matrix}a_1,\dots,a_n\\b_1,\dots,b_m\end{matrix};
x\biggr)
=
\sum_{k=0}^\infty
\frac{
(a_1)_k\cdot\ldots\cdot(a_n)_k
}{
(b_1)_k\cdot\ldots\cdot(b_m)_k
}
\frac{x^k}{k!}.
\label{buch:rekursion:hypergeometrisch:eqn:f}
\end{equation}
Die Ableitung von $f(x)$ ist
\[
f'(x)
=
\sum_{k=0}^\infty
\frac{
(a_1)_k\cdot\ldots\cdot(a_n)_k
}{
(b_1)_k\cdot\ldots\cdot(b_m)_k
}
\frac{x^{k-1}}{(k-1)!}
=
\sum_{k=1}^\infty
\frac{
(a_1)_{k+1}\cdot\ldots\cdot(a_n)_{k+1}
}{
(b_1)_{k+1}\cdot\ldots\cdot(b_m)_{k+1}
}
\frac{x^k}{k!}.
\]
Der Koeffizient besteht zwar aus lauter Pochhammer-Symbolen, aber sie
haben jeweils zu einen Faktor zuviel.
Indem man den jeweils ersten Faktor ausklammert, kann man die
Terme wieder in die Form einer hypergeometrischen Reihe bringen.
\begin{align*}
f'(x)
&=
\sum_{k=1}^\infty
\frac{
a_1(a_1)_{k}\cdot\ldots\cdot a_n(a_n)_{k}
}{
b_1(b_1)_{k}\cdot\ldots\cdot b_m(b_m)_{k}
}
\frac{x^k}{k!}
\\
&=
\sum_{k=1}^\infty
\frac{
a_1\cdot\ldots\cdot a_n
}{
b_1\cdot\ldots\cdot b_m
}
\frac{
(a_1+1)_{k}\cdot\ldots\cdot(a_n+1)_{k}
}{
(b_1+1)_{k}\cdot\ldots\cdot(b_m+1)_{k}
}
\frac{x^k}{k!}
\\
&=
\frac{
a_1\cdot\ldots\cdot a_n
}{
b_1\cdot\ldots\cdot b_m
}
\,
\mathstrut_nF_m\biggl(
\begin{matrix}a_1+1,\dots,a_n+1\\b_1+1,\dots,b_m+1\end{matrix};
x\biggr).
\end{align*}

\begin{beispiel}
Die Kosinus-Funktion
\[
\cos x
=
1 - \frac{x^2}{2!} + \frac{x^4}{4!} - \frac{x^6}{6!} + \dots
=
\sum_{k=0}^\infty
\frac{(-1)^k}{(2k)!}x^{2k}
\]
kann wie folgt als hypergeometrische Funktion geschrieben werden.
Der Nenner hat $2k$ Faktoren, er muss also aus zwei Pochhammer-Symbolen
zusammengesetzt werden.
Dazu muss er erst um den Faktor $2^{2k}$ gekürzt werden, was
\[
\frac{(2k)!}{2^{2k}}
=
\frac12\cdot\frac32\cdot\frac52\cdot\ldots\cdot\frac{2k-1}2
\cdot
\frac22\cdot\frac42\cdot\frac62\cdot\ldots\cdot\frac{2k}2
=
({\textstyle\frac12})_k\cdot k!.
\]
Damit kann jetzt die Kosinus-Funktion als
\begin{align*}
\cos x
&=
\sum_{k=0}^\infty
\frac{2^k}{(2k)!}\biggl(\frac{-x^2}{4}\biggr)^k
=
\sum_{k=0}^\infty
\frac{1}{(\frac12)_k}
\frac{1}{k!}\biggl(\frac{-x^2}{4}\biggr)^k
=
\mathstrut_0F_1\biggl(\begin{matrix}\text{---}\\\frac12\end{matrix};-\frac{x^2}4\biggr)
\end{align*}
geschrieben werden kann.

Die Ableitung der Kosinus-Funktion ist daher
\begin{align*}
\frac{d}{dx} \cos x
&=
\frac{d}{dx}
\mathstrut_0F_1\biggl(
\begin{matrix}\text{---}\\\frac12\end{matrix};-\frac{x^2}4
\biggr)
=
\frac{1}{\frac12}
\,
\mathstrut_0F_1\biggl(
\begin{matrix}\text{---}\\\frac32\end{matrix};-\frac{x^2}4
\biggr)
\cdot\biggl(-\frac{x}2\biggr)
=
-x
\cdot
\mathstrut_0F_1\biggl(
\begin{matrix}\text{---}\\\frac32\end{matrix};-\frac{x^2}4
\biggr)
\intertext{Dies stimmt mit der in
\eqref{buch:rekursion:hypergeometrisch:eqn:sinhyper}
gefundenen Darstellung der Sinusfunktion mit Hilfe der hypergeometrischen
Funktion $\mathstrut_0F_1$ überein, es ist also wie erwartet}
&=-\sin x.
\qedhere
\end{align*}
\end{beispiel}

%
% Stammfunktion
%
\subsubsection{Stammfunktion}
Eine Stammfunktion kann man auf die gleiche Art und Weise wie
die Ableitung finden.
Termweises Integrieren der Funktion
\eqref{buch:rekursion:hypergeometrisch:eqn:f}
ergibt
\begin{align}
\int f(x)\,dx
&=
\sum_{k=0}^\infty
\frac{
(a_1)_k\cdot\ldots\cdot(a_n)_k
}{
(b_1)_k\cdot\ldots\cdot(b_m)_k
}
\frac{x^{k+1}}{(k+1)!}.
\notag
\intertext{Wieder muss man die Pochhammer-Symbole durch solche mit
einem zusätzlichen Faktor schreiben.
Dies ist möglich, wenn keiner der Parameter $a_i=1$ und $b_j=1$
ist.
Die Stammfunktion wird daher
}
&=
\sum_{k=1}^\infty
\frac{
(a_1-1)(a_1)_k
\cdot\ldots\cdot
(a_n-1)(a_n)_k
}{
(b_1-1)(b_1)_k
\cdot\ldots\cdot
(b_m-1)(b_m)_k
}
\frac{x^k}{k!}
\notag
\\
&=
\sum_{k=1}^\infty
\frac{
(a_1-1)_{k+1}
\cdot\ldots\cdot
(a_n-1)_{k+1}
}{
(b_1-1)_{k+1}
\cdot\ldots\cdot
(b_m-1)_{k+1}
}
\frac{x^k}{k!}
\label{buch:rekursion:hypergeometrisch:eqn:stammfunktion:summe}
\\
&=
\mathstrut_nF_m\biggl(
\begin{matrix}
a_1-1,\dots,a_n-1\\
b_1-1,\dots,b_m-1
\end{matrix}
;x
\biggr)
-
\frac{(a_1-1)\dots(a_n-1)}{(b_1-1)\dots(b_m-1)}.
\notag
\end{align}
Der Term auf der rechten Seite kompensiert den konstanten
Term, der in der hypergeometrischen Funktion $\mathstrut_nF_m$
vorkommt, aber nicht in der
Summe~\eqref{buch:rekursion:hypergeometrisch:eqn:stammfunktion:summe}.

%
% Berechnung hypergeometrischer Funktionen
%
\subsection{Numerische Berechnung}
\begin{figure}
\centering
\includegraphics{chapters/040-rekursion/images/powergamma.pdf}
\caption{Anwachsen von Zähler $x^k$ und Nenner $k!$ in der
Exponentialreihe für $x=10$. 
Ab $k\approx 30$ werden die Terme der Reihe zunehmend kleiner,
was schliesslich zu Konvergenz der Reihe führt.
Für $k\approx 50$ sind die einzelnen Terme kleiner als die
Maschinengenauigkeit darzustellen gestattet.
Für negative $x$ wird starke Auslöschung die Genauigkeit des Resultates
unbrauchbar machen.
\label{buch:rekursion:hypergeometrisch:fig:powergamma}}
\end{figure}
Die naheliegendste Methode zur Berechnung einer hypergeometrischen
Funktion  $\mathstrut_pF_q$ ist die Auswertung der Potenzreihe.
Schon die einfachste hypergeometrische Funktion $\mathstrut_0F_0(x)=e^x$
zeigt aber, dass dabei numerische Schwierigkeiten auftreten können.
Für negative Argumente wird die Summe der Potenzreihe sehr klein.
Dies ist möglich, weil der Nenner $k!$ in der Potenzreihe
\[
\mathstrut_0F_0(x)
=
e^x
= 
1+x+\frac{x^2}{2!} + \frac{x^3}{3!} + \dots + \frac{x^k}{k!}+\dots
\]
sehr viel schneller anwächst als der Zähler $x^k$
(Abbildung~\ref{buch:rekursion:hypergeometrisch:fig:powergamma}).
Dazu muss aber $k$ eine gewisse Grösse haben, für $x=10$ muss
z.~B.~$k\gtrapprox30$ sein.
Für negative $x$ ist $e^x$ sehr klein, aber einzelne Terme in der
Summe sind viele Grössenordnungen grösser, was zu Auslöschung führt.

Im Falle der Exponentialfunktion gibt es eine einfache Lösung für
dieses Problem.
Um $y=e^x$ für negatives $x$ zu berechnen, verwendet man mit Vorteil
$y=1/e^{-x}$. Da $-x$ positiv ist, entsteht keine Auslöschung und der
Kehrwert führt ebenfalls keine weiteren Fehler ein.
So ist die Berechnung von $e^x$ auch für negative $x$ mit hoher
Genauigkeit möglich.

Ein ähnliches Phänomen muss ganz offensichtlich auch bei den
trigonometrischen Funktion $\sin x$ und $\cos x$ für beliebige
reelle Argumente auftreten.
Da diese Funktionen beschränkt sind, muss für grosse Absolutwerte
des Argumentes Auslöschung auftreten.
In diesem Fall können die Periodizität und goniometrische
Identitäten verwendet werden, um die Berechnungsaufgabe in eine zu
transformieren, die mit hoher Genauigkeit ausgeführt werden kann.

Die allgemeine Theorie der hypergeometrischen Funktionen stellt
eine Reihe ähnlicher Transformationen bereit, die bei der Berechnung
ebenfalls hilfreich sein können.
Die Darstellung dieser Transformationen würde den Rahmen dieses
Buches sprengen.
Eine speziell interessante Technik, der Gausssche Kettenbruch,
erlaubt Darstellungen einer Funktion als Kettenbruch zu finden, die
oft sehr gute numerische Eigenschaften haben.
Kapitel~\ref{chapter:0f1} untersucht einige dieser Möglichkeiten.


%\subsection{TODO}
%\begin{itemize}
%\item Hypergeometrische Transformationen
%\item Gausscher Kettenbruch \url{https://en.wikipedia.org/wiki/Gauss\%27s_continued_fraction}
%\end{itemize}


\section*{Übungsaufgaben}
\rhead{Übungsaufgaben}
\aufgabetoplevel{chapters/050-differential/uebungsaufgaben}
\begin{uebungsaufgaben}
%\uebungsaufgabe{0}
\uebungsaufgabe{504}
\uebungsaufgabe{501}
\uebungsaufgabe{502}
\uebungsaufgabe{503}
\end{uebungsaufgaben}

