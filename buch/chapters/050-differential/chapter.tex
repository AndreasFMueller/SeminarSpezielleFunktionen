%
% chapter.tex -- Kapitel zur Funktionen, die als Lösungen von Differential-
%                gleichungen definiert sind
%
% (c) 2021 Prof Dr Andreas Müller, Hochschule Rapperswil
%
% !TeX spellcheck = de_CH
\chapter{Differentialgleichungen
\label{buch:chapter:differential}}
\lhead{Differentialgleichungen}
\rhead{}
Allgemeine Sätze über die Existenz und Eindeutigkeit der Lösungen
gewöhnlicher Differentialgleichungen garantieren für fast jeder 
einigermassen vernünftige Gleichung mindestens für kurze Zeit
eine eindeutige Lösung für fast jede Anfangsbedingung.
Die Konstruktion solcher Lösungen stellt sich jedoch als deutlich
schwieriger heraus.

Für einzelne Kategorien von Differentialgleichungen sind 
gut funktionierende Lösungsverfahren gefunden worden, zum Beispiel
für lineare Differentialgleichungen mit konstanten Koeffizienten.
Damit konnten auch Gleichungen gelöst werden, die sich zum Beispiel
durch eine Variablentransformation auf eine lineare Differentialgleichung
mit konstanten Koeffizienten reduzieren lassen, wie die Eulersche
Differentialgleichung.

Die Methode der Separation der Variablen liefert führt die Lösung
einer Differentialgleichung erster Ordnung auf die Bestimmung 
zweier Stammfunktionen und deren Invertierung zurück.
Dieses Verfahren ist jedoch nicht auf Vektordifferentialgleichungen
oder auf Differentialgleichungen höherer Ordnung verallgemeinerungsfähig.

Daneben gibt es eine Reihe von ``Spezialfällen'' wie die
Clairaut-Differentialgleichung oder die damit verwandte
Lagrangesche Differentialgleichung, deren Lösung eine sehr
spezielle Form haben.

Sehr viele Differentialgleichungen in den Anwendungen können aber
mit keinem der genannten Verfahren gelöst werden.
Hier bleibt nichts anderes übrig, als neue spezielle Funktionen
zu definieren, die Lösungen dieser Differentialgleichungen sind.
Dabei ist man bestrebt, möglichst universell einsetzbare Funktionen
zu definieren, die ein breites Anwendungsfeld haben.

In den folgenden Abschnitten wird zunächst gezeigt, dass viele
der bereits bekannten speziellen Funktionen ebenfalls als Lösungen
gewöhnlicher Differentialgleichungen erhalten werden können.
Die numerische Lösung gewöhnlicher Differentialgleichungen ist
oft keine effizientes Vorgehen zur Bestimmung von einzelnen Werten,
daher wird in
Abschnitt~\ref{buch:differentialgleichungen:section:potenzreihenmethode}
eine universelle Methode vorgestellt, mit der eine Potenzreihenentwicklung
gefunden werden kann.
Eine Potenzreihendarstellung ermöglicht nicht nur die Berechnung
einzelner Werte, sondern auch beliebiger Ableitungen und die
analytische Untersuchung der Funktion mit den Methoden der
komplexen Analysis.
Als Beispiel für dieses Verfahren werden in
Abschnitt~\ref{buch:differntialgleichungen:section:bessel}
die Bessel-Funktionen erster Art vorgestellt.

%
% teil2.tex -- Beispiel-File für teil2 
%
% (c) 2020 Prof Dr Andreas Müller, Hochschule Rapperswil
%
\section{Beispiele 
\label{sturmliouville:section:teil2}}
\rhead{Beispiele}


%
% potenzreihenmethode.tex
%
% (c) 2021 Prof Dr Andreas Müller, OST Ostschweizer Fachhochschule
%
\section{Potenzreihenmethode
\label{buch:differentialgleichungen:section:potenzreihenmethode}}
Die Potenzreihenmethode versucht die Lösung einer gewöhnlichen
Differentialgleichung als Potenzreihe um die Anfangsbedingung zu
entwickeln.
Wir gehen in diesem Abschnitt von einer Differentialgleichung der
Form
\begin{equation}
a_n(x)y^{(n)}(x)
+
a_{n-1}(x)y^{(n-1)}(x)
+
\dots
+
a_1(x)y'(x)
+
a_0(x)y(x)
=
f(x)
\label{buch:differentialgleichungen:eqn:potenzreihendgl}
\end{equation}
mit der Randbedingung $y(0)=y_0$ aus.
Schon im einfachsten Fall einer homogenen Differentialgleichung erster
Ordnung ergibt sich die Beziehung
\[
a_1(x) y'(x) = a_0(x)y(x),
\]
wobei wir uns $y(x)$ und damit auch $y'(x)$ als Potenzreihe vorstellen.
Insbesondere ist 
\[
\frac{a_1(x)}{a_0(x)} = \frac{y(x)}{y'(x)}
\]
ein Quotient von Potenzreihen, den man natürlich wieder als 
Potenzreihe schreiben kann.
Da es nur auf den Quotienten ankommt, kann man sich auf den Fall
beschränken, dass die Koeffizienten Potenzreihen sind.
Tatsächlich gilt der folgende sehr viel allgemeinere Satz von
Cauchy und Kowalevskaja:

\begin{satz}[Cauchy-Kowalevskaja]
Eine partielle Differentialgleichung der Ordnung $k$ für eine
Funktion $u(x_1,\dots,x_n,t)=u(x,t)$ 
in expliziter Form
\[
\frac{\partial^k}{\partial t^k}
=
G\biggl(x,t,
\frac{\partial^j\partial^\alpha}{\partial t^j\,\partial x^k}
\biggr)
\quad\text{mit $j<k$ und $|\alpha|+j\le k$}
\]
mit einer Funktion $G$, die analytisch ist in allen Variablen
und der Randbedingung
\[
\frac{\partial j}{\partial t^j}u(x,0) = \varphi_j(x)\quad\text{für $k=0,\dots,k-1$}
\]
mit analytischen Funktion $\varphi_j$ hat eine in einer Umgebung von 
$t=0$ eindeutige analytische Lösung.
\end{satz}

Im folgenden werden wir daher weitere einschränkende Annahmen über
die Koeffizienten $a_k(x)$ machen.

\subsection{Potenzreihenansatz und Koeffizientenvergleich}



\subsection{Die Newtonsche Reihe}
Wir lösen die
Differentialgleichung~\eqref{buch:differentialgleichungen:eqn:wurzeldgl1}
mit der Anfangsbedingung $y(t)=1$ mit der Potenzreihenmethode.
Wir setzen daher für die Lösung die Potenzreihe an
\[
y(t)
=
a_0 + a_1t + a_2t^2 + a_3t^3 + \dots + a_kt^k + \dots
\]
Die Ableitung ist
\[
\dot{y}(t)
=
a_1 + 2a_2t + 3a_3t^2 + \dots  + ka_kt^{k-1} + \dots
\]
Einsetzen in die 
Differentialgleichung~\eqref{buch:differentialgleichungen:eqn:wurzeldgl1}
liefert
\begin{align*}
(1+t)
(
a_1 + 2a_2t + 3a_3t^2 + \dots  + ka_kt^{k-1} + \dots
)
&=
\alpha
(
a_0 + a_1t + a_2t^2 + a_3t^3 + \dots + a_kt^k + \dots
)
\\
a_1
+(a_1+2a_2)t
+(2a_2+3a_3)t^2
+(3a_3+4a_4)t^3
+\dots
&=
\alpha a_0 + \alpha a_1t + \alpha a_2t^2 + \alpha a_3t^3 + \dots
\end{align*}
Der Koeffizientenvergleich ergbiet die Gleichungen
\begin{align}
a_1&=\alpha a_0
\notag
\\
a_1+2a_2 &= \alpha a_1 &&\Rightarrow& 2a_2 &= (\alpha-1) a_1
\notag
\\
2a_2+3a_3 &= \alpha a_2&&\Rightarrow& 3a_3 &= (\alpha-2) a_2
\notag
\\
3a_3+4a_4 &= \alpha a_3&&\Rightarrow& 4a_4 &= (\alpha-3) a_3
\notag
\\
4a_4+5a_5 &= \alpha a_4&&\Rightarrow& 5a_5 &= (\alpha-4) a_4
\notag
\\
&\vdots
\notag
\\
&&&& \llap{$(k+1)a_{k+1}$} &= (\alpha-k) a_k
&&\Rightarrow&
a_{k+1} = \frac{\alpha-k}{k+1}a_k.
\label{buch:differentialgleichungen:eqn:newtonreiherekursion}
\end{align}
Die
Rekursionsformel~\eqref{buch:differentialgleichungen:eqn:newtonreiherekursion}
gilt auch im Fall $k=0$.
Aus der Anfangsbedingung folgt $a_0=1$.
Durch wiederholte Anwendung der 
Rekursionsformel~\eqref{buch:differentialgleichungen:eqn:newtonreiherekursion}
erhalten wir jetzt die Koeffizienten
\begin{align*}
a_0&=1
\\
a_1&=\alpha
\\
a_2&=\frac{\alpha(\alpha-1)}{1\cdot 2}
\\
a_3&=\frac{\alpha(\alpha-1)(\alpha-2)}{1\cdot 2\cdot 3}
\\
a_4&=\frac{\alpha(\alpha-1)(\alpha-2)(\alpha-3)}{1\cdot 2\cdot 3\cdot 4}
\\
&\;\vdots
\\
a_k&=\frac{\alpha(\alpha-1)(\alpha-2)\dots(\alpha-k+1)}{k!}.
\end{align*}
Für ganzzahliges $\alpha$ ist $a_k$ der Binomialkoeffizient
\[
a_k=\binom{\alpha}{k}
\]
und $a_k=0$ für $k>\alpha$.
Für nicht ganzzahliges $\alpha$ sind alle Koeffizienten $a_k\ne 0$.

Die Lösung der 
Differentialgleichung~\eqref{buch:differentialgleichungen:eqn:wurzeldgl1}
ist daher die Reihe
\begin{equation}
(1+t)^\alpha
=
\sum_{k=0}^\infty
\frac{\alpha(\alpha-1)\dots(\alpha-k+1)}{k!}\, t^k.
\label{buch:differentialgleichungen:eqn:newtonreihe}
\end{equation}
Für ganzzahliges $\alpha$ wird daraus die binomische Formel
\[
(1+t)^\alpha
=
\sum_{k=0}^\infty
\frac{\alpha(\alpha-1)\dots(\alpha-k+1)}{k!}\, t^k
=
\sum_{k=0}^\alpha \binom{\alpha}{k} t^k.
\]

%
% Lösung als hypergeometrische Riehe
%
\subsubsection{Lösung als hypergeometrische Funktion}
Die Newtonreihe verwendet ein absteigendes Produkt im Zähler.
Man kann sie aber in eine Form bringen, die besser zu den aufsteigenden
Produkten bringen, die wir im Zusammenhang mit der Gamma-Funktion
angetroffen und als Pochhammer-Symbole formalisiert haben.

Eine hypergeometrische Funktion zeichnet sich dadurch aus, dass
die Quotienten aufeinanderfolgender Koeffizienten der Reihe rationale
Funktionen von $k$ sind.
Der Quotient ist
nach~\eqref{buch:differentialgleichungen:eqn:newtonreiherekursion}
\[
\frac{a_{k+1}}{a_k}
=
\frac{\alpha-k}{k+1}.
\]
Der Nenner wird nie $0$, aber das Zählerpolynom hat genau die Nullstelle
$-\alpha$.
Die Newtonsche Reihe muss sich daher als Wert der hypergeometrischen
Funktion $\mathstrut_1F_0$ schreiben lassen.

Das Produkt im Zähler von $a_k$ hat $k$ Faktoren, indem wir jeden Faktor
mit $-1$ multiplizieren, erhalten wir
\begin{align*}
\alpha(\alpha-1)(\alpha-2)\dots(\alpha-k+1)
&=
(-\alpha)(-\alpha+1)(-\alpha+2)\dots(-\alpha+k-1) (-1)^k
\\
&=
(-\alpha)_k (-1)^k.
\end{align*}
Indem wir den Faktor $-1$ in der Variablen absorbieren, erhalten
wir die Darstellung
\[
(1+t)^\alpha
=
\sum_{k=0}^\infty
(-\alpha)_k\frac{(-t)^k}{k!}.
\]
Damit haben wir den folgenden Satz gezeigt.

\begin{satz}
Die Newtonsche Reihe für $(1-t)^\alpha$ ist der Wert
\[
(1-t)^\alpha
=
\sum_{k=0}^\infty (-\alpha)_k \frac{t^k}{k!}
=
\mathstrut_1F_0(-\alpha;t)
\]
der hypergeometrischen Funktion $\mathstrut_1F_0$.
\end{satz}



%
% Besselfunktionen also orthogonale Funktionenfamilie
%
\section{Bessel-Funktionen als orthogonale Funktionenfamilie}
\rhead{Bessel-Funktionen}
Auch die Besselfunktionen sind eine orthogonale Funktionenfamilie.
Sie sind Funktionen differenzierbaren Funktionen $f(r)$ für $r>0$
mit $f'(r)=0$ und für $r\to\infty$ nimmt $f(r)$ so schnell ab, dass
auch $rf(r)$ noch gegen $0$ strebt.
Das Skalarprodukt ist
\[
\langle f,g\rangle
=
\int_0^\infty r f(r) g(r)\,dr,
\]
als Operator verwenden wir
\[
A = \frac{d^2}{dr^2} + \frac{1}{r}\frac{d}{dr} + s(r),
\]
wobei $s(r)$ eine beliebige integrierbare Funktion sein kann.
Zunächst überprüfen wir, ob dieser Operator wirklich selbstadjungiert ist.
Dazu rechnen wir
\begin{align}
\langle Af,g\rangle
&=
\int_0^\infty
r\,\biggl(f''(r)+\frac1rf'(r)+s(r)f(r)\biggr) g(r)
\,dr
\notag
\\
&=
\int_0^\infty rf''(r)g(r)\,dr
+
\int_0^\infty f'(r)g(r)\,dr
+
\int_0^\infty s(r)f(r)g(r)\,dr.
\notag
\intertext{Der letzte Term ist symmetrisch in $f$ und $g$, daher
ändern wir daran weiter nichts.
Auf das erste Integral kann man partielle Integration anwenden und erhält}
&=
\biggl[rf'(r)g(r)\biggr]_0^\infty
-
\int_0^\infty f'(r)g(r) + rf'(r)g'(r)\,dr
+
\int_0^\infty f'(r)g(r)\,dr
+
\int_0^\infty s(r)f(r)g(r)\,dr.
\notag
\intertext{Der erste Term verschwindet wegen der Bedingungen an die
Funktionen $f$ und $g$.
Der erste Term im zweiten Integral hebt sich gegen das
zweite Integral weg.
Der letzte Term ist das Skalarprodukt von $f'$ und $g'$.
Somit ergibt sich
}
&=
-\langle f',g'\rangle
+
\int_0^\infty s(r) f(r)g(r)\,dr.
\label{buch:integrale:orthogonal:besselsa}
\end{align}
Vertauscht man die Rollen von $f$ und $g$, erhält man das Gleiche, da im
letzten Ausdruck~\eqref{buch:integrale:orthogonal:besselsa} die Funktionen
$f$ und $g$ symmetrische auftreten.
Damit ist gezeigt, dass der Operator $A$ selbstadjungiert ist.
Es folgt nun, dass Eigenvektoren des Operators $A$ automatisch
orthogonal sind.

Eigenfunktionen von $A$ sind aber Lösungen der Differentialgleichung
\[
\begin{aligned}
&&
Af&=\lambda f
\\
&\Rightarrow\qquad&
f''(r) +\frac1rf'(r) + s(r)f(r) &= \lambda f(r)
\\
&\Rightarrow\qquad&
r^2f''(r) +rf'(r)+ (-\lambda r^2+s(r)r^2)f(r) &= 0
\end{aligned}
\]
sind.

Durch die Wahl $s(r)=1$ wird der Operator $A$ zum Bessel-Operator
$B$ definiert in
\eqref{buch:differentialgleichungen:bessel-operator}.
Die Lösungen der Besselschen Differentialgleichung zu verschiedenen Werten
des Parameters müssen also orthogonal sein, insbesondere sind die
Besselfunktion $J_\nu(r)$ und $J_\mu(r)$ orthogonal wenn $\mu\ne\nu$ ist.


%
% hypergeometrisch.tex
%
% (c) 2021 Prof Dr Andreas Müller, OST Ostschweizer Fachhochschule
%
\section{Hypergeometrische Differentialgleichung
\label{buch:differentialgleichungen:section:hypergeometrisch}}
Die hypergeometrische Funktion $\mathstrut_2F1(a,b;c;x)$ wurde in
Abschnitt~\ref{buch:rekursion:section:hypergeometrische-funktion}
als Potenzreihe mit sehr speziellen Koeffizienten, die sich aus
Pochhammer-Symbolen.
Es stellt sich aber heraus, dass man sie auch als Lösung einer
gewöhnlichen Differentialgleichung bekommen kann, die bereits
Euler studiert hat.

\subsection{Die Eulersche hypergeometrische Differentialgleichung
\label{buch:differentialgleichung:subsection:euler-hypergeometrisch}}
Die hypergeometrische Funktion $\mathstrut_2F_1(a,b;c;x)$ ist eine
Lösung der {\em Eulerschen hypergeometrischen Differentialgleichung}
(zu unterscheiden von der Eulerschen Differentialgleichung, die sich
immer auf eine lineare Differentialgleichung mit konstanten Koeffizienten
reduzieren lässt)
\begin{equation}
x(1-x) \frac{d^2y}{dx^2} + (c-(a+b+1)x)\frac{dy}{dx} - ab y = 0
\label{buch:differentialgleichungen:hypergeo:eulerdgl}
\end{equation}
Wir prüfen dies nach, indem wir die Definition der hypergeometrischen
Funktion 
\begin{align*}
y(x)
&=
\mathstrut_2F_1(a,b;c;x)
=
\sum_{k=0}^\infty
\frac{(a)_k(b)_k}{(c)_k} \frac{x^k}{k!}
\intertext{mit den Ableitungen}
y'(x)
&=
\sum_{k=1}^\infty 
\frac{(a)_k(b)_k}{(c)_k} \frac{x^{k-1}}{(k-1)!}
\\
y''(x)
&=
\sum_{k=2}^\infty 
\frac{(a)_k(b)_k}{(c)_k} \frac{x^{k-2}}{(k-2)!}
\end{align*}
einsetzen.
Die Gleichung, die sich ergibt, ist
\begin{align*}
0
&=
x(1-x)
\sum_{k=2}^\infty
\frac{(a)_k(b)_k}{(c)_k}\frac{x^{k-2}}{(k-2)!}
+
(c-(a+b+1)x)
\sum_{k=1}^\infty
\frac{(a)_k(b)_k}{(c)_k}\frac{x^{k-1}}{(k-1)!}
-ab
\sum_{k=0}^\infty
\frac{(a)_k(b)_k}{(c)_k} \frac{x^k}{k!}
\\
&=
\sum_{k=2}^\infty
\frac{(a)_k(b)_k}{(c)_k}\frac{x^{k-1}}{(k-2)!}
-
\sum_{k=2}^\infty
\frac{(a)_k(b)_k}{(c)_k}\frac{x^k}{(k-2)!}
+
c\sum_{k=1}^\infty
\frac{(a)_k(b)_k}{(c)_k}\frac{x^{k-1}}{(k-1)!}
\\
&\qquad
-(a+b+1)
\sum_{k=1}^\infty
\frac{(a)_k(b)_k}{(c)_k}\frac{x^k}{(k-1)!}
-ab
\sum_{k=0}^\infty
\frac{(a)_k(b)_k}{(c)_k} \frac{x^k}{k!}
\\
&=
\sum_{k=1}^\infty
\frac{(a)_{k+1}(b)_{k+1}}{(c)_{k+1}}\frac{x^k}{(k-1)!}
-
\sum_{k=2}^\infty
\frac{(a)_k(b)_k}{(c)_k}\frac{x^k}{(k-2)!}
+
c\sum_{k=0}^\infty
\frac{(a)_{k+1}(b)_{k+1}}{(c)_{k+1}}\frac{x^k}{k!}
\\
&\qquad
-(a+b+1)
\sum_{k=1}^\infty
\frac{(a)_k(b)_k}{(c)_k}\frac{x^k}{(k-1)!}
-ab
\sum_{k=0}^\infty
\frac{(a)_k(b)_k}{(c)_k} \frac{x^k}{k!}.
\end{align*}
Zum konstanten Koeffizienten für $k=0$ tragen nur die dritte und letzte
Summe bei, dies sind die Terme
\[
c\frac{(a)_1(b)_1}{(c)_1}-ab\frac{(a)_0(b)_0}{(c)_0}
=
c\frac{ab}{c}-ab\frac{1\cdot 1}{1}
=
0.
\]
Für den linearen Term $k=1$ kommen je ein Term aus der ersten aund vierten
Summe hinzu, dies ergibt
\begin{align*}
&\phantom{\mathstrut=\mathstrut}
\frac{(a)_2(b)_2}{(c)_2}
+c\frac{(a)_2(b)_2}{(c)_2}
-(a+b+1)\frac{(a)_1(b)_1}{(c)_1}
-ab\frac{(a)_1(b)_1}{(c)_1}
\\
&=
\frac{a(a+1)b(b+1)}{c(c+1)}
(1+c)
-(ab+a+b+1)
\frac{ab}{c}
\\
&=
\frac{a(a+1)b(b+1)}{c}
-
(a+1)(b+1)\frac{ab}{c}
=0.
\end{align*}
Durch Koeffizientenvergleich erhalten wir für $k\ge 2$ 
\begin{align*}
0
&=
\frac{(a)_{k+1}(b)_{k+1}}{(c)_{k+1}} \frac1{(k-1)!} 
-
\frac{(a)_k(b)_k}{(c)_k} \frac1{(k-2)!} 
+
c\frac{(a)_{k+1}(b)_{k+1}}{(c)_{k+1}} \frac{1}{k!}
\\
&\qquad
-(a+b+1)\frac{(a)_k(b)_k}{(c)_k}\frac{1}{(k-1)!}
-ab \frac{(a)_k(b)_k}{(c)_k}\frac{1}{k!}
\\
&=
\frac{(a)_k(b)_k}{(c)_{k+1}}
\frac{1}{k!}
\biggl(
(a+k)(b+k)k
-(c+k)(k-1)k
+
c(a+k)(b+k)
\\
&\qquad
\qquad
\qquad
-(a+b+1)(c+k)k
-ab(c+k)
\biggr).
\intertext{Der zweite, vierte und fünfte Term können zu}
&=
\frac{(a)_k(b)_k}{(c)_{k+1}}
\frac{1}{k!}
\biggl(
(a+k)(b+k)k
+
c(a+k)(b+k)
-(ab+ak+bk+k^2)(c+k)
\biggr)
\intertext{zusammengefasst werden.
Der Faktor $(ab+ak+bk+k^2)$ kann als Produkt $(a+k)(b+k)$ faktorisiert
werden, der dann als gemeinsamer Faktor aus allen Termen ausgeklammert
werden kann:}
&=
\frac{(a)_k(b)_k}{(c)_{k+1}}
\frac{1}{k!}
\biggl(
(a+k)(b+k)k
+
c(a+k)(b+k)
-(a+k)(b+k)(c+k)
\biggr)
\\
&=
\frac{(a)_{k+1}(b)_{k+1}}{(c)_{k+1}}
\frac{1}{k!}
\biggl(
k
+
c
-(c+k)
\biggr)
=0.
\end{align*}
Damit ist gezeigt, dass $\mathstrut_2F_1(a,b;c;x)$ eine Lösung
der Differentialgleichung ist.

Die hypergeometrische Reihe kann auch direkt mit Hilfe der
Potenzreihenmethode als Lösung der Differentialgleichung gefunden 
werden.

\subsection{Lösung als verallgemeinerte Potenzreihe}
Da die hypergeometrische Reihe eine Differentialgleichung
zweiter Ordnung mit einer Singularität bei $x=0$ ist, 
kann man versuchen eine zweite, linear unabhängige Lösung mit
Hilfe der Methode der verallgemeinerten Potenzreihen zu finden.
Dazu setzt man die Lösung in der Form
\begin{align*}
y_2(x)
&=
\sum_{k=0}^\infty a_kx^{\varrho+k}
&
&\Rightarrow&
y_2'(x)
&=
\sum_{k=0}^\infty (\varrho+k)a_kx^{\varrho+k-1}
\\
&&
&&
y_2''(x)
&=
\sum_{k=0}^\infty (\varrho+k)(\varrho+k-1)a_kx^{\varrho+k-2}
\end{align*}
an, wobei $a_0\ne 0$ sein soll.
Einsetzen in die Differentialgleichung ergibt
\begin{align*}
0&=
x(1-x)y_2''(x) + (c-(a+b+1)x) y_2'(x) -aby_2(x)
\\
&=
x(1-x)
\sum_{k=0}^\infty (\varrho+k)(\varrho+k-1)a_kx^{\varrho+k-2}
+
(c-(a+b+1)x)
\sum_{k=0}^\infty (\varrho+k)a_kx^{\varrho+k-1}
-
abx^{\varrho}\sum_{k=0}^\infty a_kx^{\varrho+k}
\\
&=
-\sum_{k=0}^\infty (\varrho+k)(\varrho+k-1)a_kx^{\varrho+k}
+
\sum_{k=0}^\infty (\varrho+k)(\varrho+k-1)a_kx^{\varrho+k-1}
+
c
\sum_{k=0}^\infty (\varrho+k)a_kx^{\varrho+k-1}
\\
&\qquad
-
(a+b+1)
\sum_{k=0}^\infty (\varrho+k)a_kx^{\varrho+k}
-
ab
\sum_{k=0}^\infty a_kx^{\varrho+k}.
\intertext{Durch Verschiebung des Summationsindex in der zweiten
und dritten Summe wird der Koeffizientenvergleich etwas
einfacher}
&=
-\sum_{k=0}^\infty (\varrho+k)(\varrho+k-1)a_kx^{\varrho+k}
+
\sum_{k=-1}^\infty (\varrho+k+1)(\varrho+k)a_{k+1}x^{\varrho+k}
+
c
\sum_{k=-1}^\infty (\varrho+k+1)a_{k+1}x^{\varrho+k}
\\
&\qquad
-
(a+b+1)
\sum_{k=0}^\infty (\varrho+k)a_kx^{\varrho+k}
-
ab
\sum_{k=0}^\infty a_kx^{\varrho+k}
\\
&=
-\sum_{k=0}^\infty (\varrho+k)(\varrho+k-1)a_kx^{\varrho+k}
+
\sum_{k=-1}^\infty (\varrho+k+1)(\varrho+k+c)a_{k+1}x^{\varrho+k}
\\
&\qquad
-
\sum_{k=0}^\infty ((\varrho+k)(a+b+1)+ab)a_kx^{\varrho+k}
\\
&=
\bigl(
\varrho(\varrho-1)
+c\varrho \bigr)
x^{\varrho-1}
+
\sum_{k=0}^\infty
\bigl(
-(\varrho+k)(\varrho+k-1)a_k
+(\varrho+k+1)(\varrho+k+c)a_{k+1}
\\
&
\qquad
\qquad
\qquad
\qquad
\qquad
\qquad
-((\varrho+k)(a+b+1)+ab)a_k
\bigr)
x^{\varrho+k}.
\end{align*}
Aus dem ersten Term kann man die Indexgleichung
\[
0
=
\varrho(\varrho-1)+c\varrho
=
\varrho(\varrho-1+c)
\]
ablesen, die die Nullstellen $\varrho=0$ und $\varrho=1-c$ hat.
Die Nullstelle $\varrho=0$ ergibt natürlich die bereits gefundene
hypergeometrische Reihe.

Nach Einsetzen der zweiten Lösung der Indexgleichung in der Summe
legt der Koeffizientenvergleich eine Beziehung
\begin{align}
0
&=
\bigl(
-(k-c+1)(k-c)
-(k-c+1)(a+b+1)+ab
\bigr)a_k
+
(k-c+2)(k+1)
a_{k+1} 
\notag
\intertext{zwischen $a_k$ und $a_{k+1}$ fest.
Daraus kann man den Quotienten aufeinanderfolgender
Koeffizienten als}
\frac{a_{k+1}}{a_k}
&=
\frac{
-(k-c+1)(k-c)
-(k-c+1)(a+b+1)+ab
}{
\notag
(k-c+2)(k+1)
}
\\
&=
%(%i4) factor(coeff(y,q,0))
%(%o4)                  - (k - c + a + 1) (k - c + b + 1)
%(%i5) factor(coeff(y,q,1))
%(%o5)                         (k + 1) (k - c + 2)
\frac{
(a-c+1+k)
(b-c+1+k)
}{
(2-c+k)(k+1)
}
\label{buch:differentialgleichungen:hypergeo:verallgkoef}
\end{align}
finden.
Setzt man $a_0=1$, ist die zweite Lösung ist also wieder eine
hypergeometrische Funktion.%, nämlich
%\[
%y_2(x)
%=
%x^{1-c}
%\sum_{k=0}^\infty \frac{(a-c+1)_k(b-c+1)_k}{(2-c)_k}\frac{x^k}{k!}
%=
%x^{1-c}
%\mathstrut_2F_1\biggl(\begin{matrix}a-c+1,b-c+1\\2-c\end{matrix};x\biggr)
%\]
Diese Lösung ist aber nur möglich, wenn in
\eqref{buch:differentialgleichungen:hypergeo:verallgkoef}
der Nenner nicht verschwindet, d.~h.~$2-c+k\ne 0$
oder $c \ne k+2$ für all natürlichen $k$.
$c$ darf also kein natürliche Zahl $\ge 2$ sein.
Wir fassen die Resultate dieses Abschnitts im folgenden Satz zusammen.

\begin{satz}
Die eulersche hypergeometrische Differentialgleichung
\begin{equation}
x(1-x)\frac{d^2y}{dx^2}
+(c+(a+b+1)x)\frac{dy}{dx}
-ab y
=
0
\end{equation}
hat die Lösung
\[
y_1(x)
=
\mathstrut_2F_1\biggl(\begin{matrix}a,b\\c\end{matrix};x\biggr).
\]
Falls $c-2\not\in \mathbb{N}$ ist, ist
\[
y_2(x)
=
x^{1-c} \mathstrut_2F_1\biggl(\begin{matrix}a-c+1,b-c+1\\2-c\end{matrix};x\biggr)
\]
eine zweite, linear unabhängige Lösung.
\end{satz}

%
% Die verallgemeinerte hypergeometrische Differentialgleichung
%
\subsection{Verallgemeinerte hypergeometrische Differentialgleichung}
% https://de.wikipedia.org/wiki/Verallgemeinerte_hypergeometrische_Funktion







\section*{Übungsaufgaben}
\rhead{Übungsaufgaben}
\aufgabetoplevel{chapters/050-differential/uebungsaufgaben}
\begin{uebungsaufgaben}
%\uebungsaufgabe{0}
\uebungsaufgabe{504}
\uebungsaufgabe{501}
\uebungsaufgabe{502}
\uebungsaufgabe{503}
\end{uebungsaufgaben}

