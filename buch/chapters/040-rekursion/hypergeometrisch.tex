%
% hypergeometrisch.tex
%
% (c) 2021 Prof Dr Andreas Müller, OST Ostschweizer Fachhochschule
%
\section{Hypergeometrische Funktionen
\label{buch:rekursion:section:hypergeometrische-funktion}}
\rhead{Hypergeometrische Funktionen}
Kann man eine Formel für die Lösung $S_n$ der lineare Differenzengleichung
\[
n^3S_{n}
=
16(n-{\textstyle\frac12})(2n^2-2n+1)S_{n-1}
-256(n-1)^3S_3
\]
mit Anfangswerten $S_0=1$ und $S_1=8$ angeben?
Dies scheint auf den ersten Blick unmöglich kompliziert, man kann aber
zeigen, dass
\[
S_n
=
\sum_{k=0}^n 
\binom{2n-2k}{n-k}^2 \binom{2k}{k}^2
\]
gilt (\cite[p.~xi]{buch:ab}).
Die Lösung ist also eine Summe von Summanden, die sehr viel einfacher
aussehen und vor allem die besondere Eigenschaft haben, dass die
Quotienten aufeinanderfolgender Terme rationale Funktionen von von $k$
sind.
% XXX Quotient berechnen

Eine besonders simple solche Funktion ist die geometrische Reihe, die
im Abschnitt~\ref{buch:rekursion:hypergeometrisch:geometrisch}
in Erinnerung gerufen wird.
Abschnitt~\ref{buch:rekursion:hypergeometrisch:reihen}
definiert den Begriff der hypergeometrischen Reihe und zeigt, 
wie sie in eine Standardform gebracht werden können.
In Abschnitt~\ref{buch:rekursion:hypergeometrisch:beispiele}
schliesslich wird an Hand von Beispielen gezeigt, wie bekannte
Funktionen als hypergeometrische Funktionen interpretiert werden können.

\subsection{Die geometrische Reihe
\label{buch:rekursion:hypergeometrisch:geometrisch}}
Die besonders einfache Potenzreihe
\[
f(q)
=
\sum_{k=0}^\infty aq^k
\]
heisst die {\em geometrische Reihe}.
Die Partialsummen 
\[
S_n
=
\sum_{k=0}^n aq^k
\]
kann mit der Differenz
\begin{equation}
(1-q)S_n
=
S_n - qS_n
=
\sum_{k=0}^n aq^k
-
\sum_{k=1}^{n+1} aq^k
=
a -aq^{n+1}
\label{buch:rekursion:hypergeometrisch:eqn:qsumme}
\end{equation}
berechnet werden, die man nach
\begin{equation}
S_n 
=
a\frac{1-q^{n+1}}{1-q}
\label{buch:rekursion:hypergeometrisch:eqn:geomsumme}
\end{equation}
auflösen kann.

Fü $q<1$ geht $q^n\to 0$ und damit konvergiert
$S_n$  gegen
\[
\sum_{k=0}^\infty aq^k
=
a\frac{1}{1-q}.
\]

Die geometrische Reihe ist charakterisiert dadurch, dass aufeinanderfolgende
Terme den gleichen Quotienten
\[
\frac{aq^{k+1}}{aq^k}
=
q
\]
haben.
Die Berechnung der Summe in 
\eqref{buch:rekursion:hypergeometrisch:eqn:qsumme}
beruht darauf, dass die Multiplikation mit $q$ einen ``anderen''
Teil der Summe ergibt, der sich in der Differenze weghebt.

\subsection{Hypergeometrische Reihen
\label{buch:rekursion:hypergeometrisch:reihen}}
Es ist plausibel, dass eine etwas lockerere Bedingung an die
Quotienten aufeinanderfolgender Terme einer Reihe immer noch
ermöglichen wird, interessante Aussagen über die durch die
Reihe beschriebenen Funktionen zu machen.

\begin{definition}
Eine Reihe
\[
f(x) = \sum_{k=0}^\infty a_k x^k
\]
heisst {\em hypergeometrisch}, wenn der Quotient aufeinanderfolgender
Koeffizienten eine rationale Funktion von $k$ ist,
wenn also
\[
\frac{a_{k+1}}{a_k}
=
\frac{p(k)}{q(k)}
\]
mit Polynomen $p(k)$ und $q(k)$ ist.
\end{definition}

Die geometrische Reihe ist natürlich eine hypergeometrische Reihe,
wobei $p(k)/q(k)=1$ ist.
Etwas interessanter ist die Exponentialfunktion, durch die Taylor-Reihe
\[
e^x = \sum_{k=0}^\infty \frac{x^k}{k!}
\]
dargestellt werden kann.
Der Quotient aufeinanderfolgender Koeffizienten ist
\[
\frac{a_{k+1}}{a_k}
=
\frac{1/(k+1)!}{1/k!}
=
\frac{k!}{(k+1)!}
=
\frac{1}{k+1},
\]
eine rationale Funktion mit Zählergrad $0$ und Nennergrad $1$.

Die Kosinus-Funktion wird durch die Taylor-Reihe
\[
\cos x = \sum_{k=0}^\infty \frac{(-1)^k}{(2k)!} x^{2k}
\]
dargestellt.
Als Potenzreihe in $x$ kann die Kosinus-Reihe nicht hypergeometrisch sein,
die ungeraden Koeffizienten verschwinden und damit undefinierte
Quotienten haben.
Als Reihe in $z=x^2$ ist aber
\[
\sum_{k=0}^\infty \frac{(-1)^k}{(2k)!} z^k
\qquad\Rightarrow\qquad
a_k = \frac{(-1)^k}{(2k)!}
\]
hypergeometrisch, weil der Quotient aufeinanderfolgender Koeffizienten
\[
\frac{a_{k+1}}{a_k}
=
\frac{(-1)^{k+1}}{(2k+2)!}\cdot \frac{(2k)!}{(-1)^k}
=
-\frac{1}{(2k+2)(2k+1)},
\]
eine rationale Funktion mit Zählergrad $0$ und Nennergrad $2$.
Es gibt also eine hypergeometrische Reihe $f(z)$ derart, dass
$\cos x = f(x^2)$ ist.

Seien $p(k)$ und $q(k)$ zwei Polynome derart, dass
\[
\frac{a_{k+1}}{a_k} = \frac{p(k)}{q(k)}.
\]
Daraus lässt sich der Koeffizient $a_{k+1}$ als
\begin{equation}
a_{k+1}
=
\frac{p(k)}{q(k)}
\cdot
a_k
=
\frac{p(k)}{q(k)}
\cdot
\frac{p(k-1)}{q(k-1)}
\cdot
a_{k-1}
=\dots=
\frac{p(k)}{q(k)}
\frac{p(k-1)}{q(k-1)}
\cdots
\frac{p(1)}{q(1)}
\frac{p(0)}{q(0)}
a_0
\label{buch:rekursion:hypergeometrisch:ak+1}
\end{equation}
berechnen.
Alle Koeffizienten haben also den Faktor $a_0=f(0)$ gemeinsam.

Die Produkte von Quotienten $p(k)/q(k)$ sollen jetzt weiter
vereinfacht werden.
Sei $n$ der Grad von $p(k)$ und $m$ der Grad von $q(k)$.
Dazu nehmen wir an, dass $a_i$, $i=1,\dots,n$ die Nullstellen von $p(k)$ sind
und $b_j$, $j=1,\dots,m$ die Nullstellen von $q(k)$, dass man also
die Polynome als
\begin{align*}
p(k) &= x(k-a_1)(k-a_2)\cdots(k-a_n)
\\
q(k) &= (k-b_1)(k-b_2)\cdots(k-b_m)
\end{align*}
schreiben kann.
Der Faktor $x$ ist nötig, weil die Polynome $p(k)$ und $q(k)$ nicht
notwendigerweise normiert sind.

Um das Produkt der Quotienten zu vereinfachen, nehmen wir für den Moment
an, dass Zähler und Nenner vom Grad $n=m=1$ ist.
Dann ist nach 
\eqref{buch:rekursion:hypergeometrisch:ak+1}
\[
a_{k}
=
x^{k}
\frac{
(k-1-a_1) \cdots (2-a_1)(1-a_1)(0-a_1)
}{
(k-1-b_1) \cdots (2-b_1)(1-b_1)(0-b_1)
}
=
\frac{(-a_1)_k}{(-b_1)_k} x^k.
\]
Die Koeffizienten können daher als Quotienten von Pochhammer-Symbolen
geschrieben werden.
Für Polynome $p(k)$ und $q(k)$ höheren Grades sind die Koeffizienten
von der Form
\[
a_k
=
\frac{(-a_1)_k(-a_2)_k\cdots (-a_n)_k}{(-b_1)_k(-b_2)_k\cdots(-b_m)_k}
x^ka_0.
\]
Jede hypergeometrische Reihe kann daher in der Form
\[
a_0
\sum_{k=0}^\infty
\frac{(-a_1)_k(-a_2)_k\cdots (-a_n)_k}{(-b_1)_k(-b_2)_k\cdots(-b_m)_k}
x^k
\]
geschrieben werden.

\begin{definition}
\label{buch:rekursion:hypergeometrisch:def}
Die hypergeometrische Funktion
$\mathstrut_pF_q$ ist definiert durch die Reihe
\[
\mathstrut_pF_q
\biggl(
\begin{matrix}
a_1,\dots,a_p\\
b_1,\dots,b_q
\end{matrix}
;
x
\biggr)
=
\mathstrut_pF_q(a_1,\dots,a_p;b_1,\dots,b_q;x)
=
\sum_{k=0}^\infty
\frac{(a_1)_k\cdots(a_p)_k}{(b_1)_k\cdots(b_q)_k}\frac{x^k}{k!}.
\]
\end{definition}

Da $(1)_k=k!$ hätte die Definition den Nenner $k!$ in der Reihe
auch durch eines der Pochhammer-Symbole ausdrücken können.
Wird dieser Nenner nicht gebraucht, kann man ihn durch einen 
zusätzlichen Faktor $(1)_k$ im Zähler des Bruchs von Pochhammer-Symbolen
kompensieren, wodurch sich der Grad $p$ des Zählers natürlich um $1$
erhöht.

Die oben analysierte Summe $S$ kann mit der Definition als
\[
S
=
a_0
\,
\mathstrut_{n+1}F_m \biggl(
\begin{matrix}
-a_1,-a_2,\dots,-a_n,1\\
-b_1,-b_2,\dots,-a_m
\end{matrix}; x
\biggr)
\]
beschrieben werden.

\subsection{Beispiele von hypergeometrischen Funktionen
\label{buch:rekursion:hypergeometrisch:beispiele}}
Viele der bekannten Reihenentwicklungen häufig verwendeter Funktionen
lassen sich durch die hypergeometrischen Funktionen von
Definition~\ref{buch:rekursion:hypergeometrisch:def} ausdrücken.
In diesem Abschnitt werden einige Beispiel dazu gegeben.

\subsubsection{Die geometrische Reihe}
In der geometrischen Reihe fehlt der Nenner $k!$, es braucht
daher einen Term $(1)_k$ im Zähler, um den Nenner zu kompensieren.
Somit ist die geometrische Reihe
\[
\frac{a}{1-x}
=
\sum_{k=0}^\infty
ax^k
=
a\sum_{k=0}^\infty
\frac{(1)_k}{1}
\frac{x^k}{k!}
=
a\,\mathstrut_1F_0(1,x).
\]

\subsubsection{Exponentialfunktion}
Die Exponentialfunktion ist die Reihe
\[
e^x = \sum_{k=0}^\infty \frac{x^k}{k!}.
\]
In diesem Fall werden keine Quotienten von Pochhammer-Symbolen
benötigt, es ist daher
\[
e^x = \mathstrut_0F_0(x).
\]

\subsubsection{Wurzelfunktion}
Die Wurzelfunktion $x\mapsto \sqrt{x}$ hat keine Taylor-Entwicklung
in $x=0$, aber die Funktion $x\mapsto\sqrt{1+x}$ hat die Taylor-Reihe
\[
\sqrt{1+x}
=
1
+
\frac12 x
-
\frac{1\cdot 1}{2\cdot 4}x^2
+
\frac{1\cdot 1\cdot 3}{2\cdot 4\cdot 6}x^3
-
\frac{1\cdot 1\cdot 3\cdot 5}{2\cdot 4\cdot 6\cdot 8}x^4
+
\dots
\]
Um die Verbindung zu einer hypergeometrischen Funktion herzustellen,
müssen wir den Term $x^k/k!$ abspalten.
Dann wird
\begin{align*}
\sqrt{1+x}
&=
1
+
\frac12 \frac{x}{1!}
-
\frac{1\cdot 1}{2^2}\frac{x^2}{2!}
+
\frac{1\cdot 1\cdot 3}{2^3}\frac{x^3}{3!}
-
\frac{1\cdot 1\cdot 3\cdot 5}{2^4}\frac{x^4}{4!}
+
\dots
\\
&=
1
+
\frac12 \cdot\frac{x}{1!}
-
\frac{1}{2}\cdot \frac{1}{2}\cdot\frac{x^2}{2!}
+
\frac{1}{2}\cdot \frac{1}2\cdot \frac{3}{2}\cdot\frac{x^3}{3!}
-
\frac{1}{2}\cdot \frac{1}{2}\cdot \frac{3}{2}\cdot \frac{5}{2}\cdot\frac{x^4}{4!}
+
\dots
\end{align*}
Es ist noch etwas undurchsichtig, warum die ersten beiden Terme
das gleiche Vorzeichen haben und warum der Faktor $\frac12$ in jedem
Term zweimal vorkommt.
Diese Unklarheit kann jedoch beseitigt werden, wenn man den ersten
Faktor als $-\frac12$ schreibt:
\begin{align*}
\sqrt{1+x}
&=
1
-
\biggl(-\frac12\biggr)\cdot\frac{x}{1!}
+
\biggl(-\frac{1}{2}\biggr)\cdot \frac{1}{2}\cdot\frac{x^2}{2!}
-
\biggl(-\frac{1}{2}\biggr)\cdot \frac{1}2\cdot \frac{3}{2}\cdot\frac{x^3}{3!}
+
\biggl(-\frac{1}{2}\biggr)\cdot \frac{1}{2}\cdot \frac{3}{2}\cdot \frac{5}{2}\cdot\frac{x^4}{4!}
+
\dots
\\
&=
1 + 
\biggl(-\frac12\biggr)\cdot\frac{-x}{1!}
+
\biggl(-\frac{1}{2}\biggr)\cdot \frac{1}{2}\cdot\frac{(-x)^2}{2!}
+
\biggl(-\frac{1}{2}\biggr)\cdot \frac{1}2\cdot \frac{3}{2}\cdot\frac{(-x)^3}{3!}
+
\biggl(-\frac{1}{2}\biggr)\cdot \frac{1}{2}\cdot \frac{3}{2}\cdot \frac{5}{2}\cdot\frac{(-x)^4}{4!}
+
\dots
\end{align*}
Die Koeffizienten sind aufsteigende Produkte mit $k$ Faktoren, die alle bei
$-\frac12$ beginnen, sie können daher als Pochhammer-Symbole $(-\frac12)_k$
geschrieben werden.
Die Wurzelfunktion ist daher die hypergeometrische Funktion
\[
\sqrt{1\pm x}
=
\sum_{k=0}^\infty
\biggl(-\frac12\biggr)_k \frac{(-x)^k}{k!}
=
\mathstrut_1F_0(-{\textstyle\frac12};\mp x).
\]

\subsubsection{Logarithmusfunktion}
Für $x\in (-1,1)$ konvergiert die Taylor-Reihe
\[
\log(1+x)
=
x-\frac{x^2}{2}+\frac{x^3}{3}-\frac{x^4}{4}+\dots
\]
der Logarithmusfunktion im Punkt $x=0$.
Die Reihe beginnt nicht mit einem konstanten Term, daher klammern wir
zunächst einen Faktor $x$ aus:
\[
\log(1+x)
=
x\cdot
\biggl(
1-\frac{x}{2}+\frac{x^2}{3}-\frac{x^3}{4}+\dots
\biggr)
\]
Um dies in die Form einer hypergeometrischen Funktion zu bringen,
muss zunächst wieder der Nenner $k!$ hergestellt werden.
\begin{align*}
\log(1+x)
&=
x\cdot\biggl(
1
- \frac{1!}{2} \frac{x}{1!}
+ \frac{2!}{3} \frac{x^2}{2!} 
- \frac{3!}{4} \frac{x^3}{3!}+\dots
\biggr).
\intertext{Den Nenner $k+1$ kann man als Quotienten $k!/(k+1)!$ erhalten,
also}
\log(1+x)
&=
x\cdot\biggl(
1
- \frac{(1!)^2}{2!} \frac{x}{1!}
+ \frac{(2!)^2}{3!} \frac{x^2}{2!} 
- \frac{(3!)^2}{4!} \frac{x^3}{3!}+\dots
\biggr).
\end{align*}
Die Fakultät
\[
(k+1)!
=
1\cdot 2 \cdot 3 \cdot\ldots\cdot k\cdot (k+1)
=
2 \cdot (2 + 1) \cdot (2+2) \cdot\ldots\cdot (2+k-2) \cdot (2+k-1)
=
(2)_{k}
\]
ist auch ein Pochhammer-Symbol, so dass die Logarithmusfunktion
zur hypergeometrischen Funktion
\[
\log(1+x)
=
x\cdot\biggl(
1
+ \frac{(1)_1(1)_1}{(2)_1} \frac{(-x)}{1!}
+ \frac{(1)_2(1)_2}{(2)_2} \frac{(-x)^2}{2!} 
+ \frac{(1)_3(1)_3}{(2)_2} \frac{(-x)^3}{3!}+\dots
\biggr)
=
x\cdot
\mathstrut_2F_1\biggl(\begin{matrix}1,1\\2\end{matrix};-x\biggr).
\]


\subsubsection{Trigonometrische Funktionen}
Die Kosinus-Funktion wurde bereits als hypergeometrische Funktion erkannt,
im Folgenden soll dies auch noch für die Sinus-Funktion
durchgeführt werden.
Die Taylor-Reihe der Sinus-Funktion im Punkt $0$ ist
\begin{align*}
\sin x
&=
x-\frac{x^3}{3!}+\frac{x^5}{5!}-\frac{x^7}{7!}+\dots
\end{align*}
In dieser Reihe fehlen die geraden Potenzen, wir Klammern daher einen
Faktor $x$ aus und schreiben den Rest als eine Funktion von $-x^2$
\begin{align*}
\sin x
&=
x
\biggl(
1+\frac{-x^2}{3!}+\frac{(-x^2)^2}{5!}-\frac{(-x^2)^3}{7!}+\dots
\biggr)
=
x f(-x^2).
\end{align*}
Die Funktion $f(z)$ soll jetzt als hypergeometrische Funktion geschrieben
werden.
Dazu muss zunächst wieder der Nenner $k!$ wiederhergestellt werden:
\[
f(z)
=
1
+
\frac{1!}{3!}\cdot \frac{z}{1!}
+
\frac{2!}{5!}\cdot \frac{z^2}{2!}
+
\frac{3!}{7!}\cdot \frac{z^3}{3!}
+
\dots
\]
Die Koeffizienten $k!/(2k+1)!$ müssen jetzt durch Pochhammer-Symbole
mit jeweils $k$ Faktoren ausgedrückt werden.
Dazu muss die Fakultät $(2k+1)!$ in zwei Produkte
\[
(2k+1)
=
2\cdot 3 \cdot 4\cdot 5\cdot \ldots \cdot 2k \cdot (2k+1)
=
(2\cdot 4 \cdot 6\cdot\ldots\cdot 2k)
\cdot
(3\cdot 5\cdot 7\cdot \ldots \cdot (2k+1))
\]
aufgespaltet werden.
Diese Produkte haben zwar $k$-Faktoren, aber sie sind keine
Pochhammer-Symbole, weil die Differenz aufeinanderfolgender Faktoren 
jeweils $2$ ist.
Wir dividieren die geraden Faktoren durch $2$ und dividieren die 
ungeraden durch $2$, dadurch ändert sich das Produkt nicht und wird
\[
(2k+1)!
=
(1\cdot2\cdot3\cdot\ldots\cdot k)
\cdot
\biggl(
\frac{3}{2}\cdot
\frac{5}{2}\cdot
\frac{7}{2}\cdot
\ldots\cdot
\frac{2k+1}{2}
\biggr)
=
(1)_k\cdot \biggl(\frac{3}{2}\biggr)_k
\]
Setzt man dies in die Reihe ein, wird
\[
f(z)
=
\sum_{k=0}^\infty
\frac{(1)_k}{(1)_k\cdot (\frac{3}{2})_k}
z^k
=
\mathstrut_1F_2(1;1,\frac{3}{2};z).
\]
Damit lässt sich die Sinus-Funktion als
\begin{equation}
\sin x
=
x\,\mathstrut_1F_2\biggl(\begin{matrix}1\\1,\frac32\end{matrix};-x^2\biggr)
=
x\,\mathstrut_1F_2\biggl(\begin{matrix}\text{---}\\\frac32\end{matrix};-x^2\biggr)
\label{buch:rekursion:hypergeometrisch:eqn:sinhyper}
\end{equation}
durch eine hypergeometrische Funktion ausdrücken.

\subsubsection{Hyperbolische Funktionen}
Die für die Sinus-Funktion angewendete Methode lässt sich auch
auf die Funktion 
\begin{align*}
\sinh x
&=
\sum_{k=0}^\infty \frac{x^{2k+1}}{(2k+1)!}
\\
&=
x
\,
\biggl(
1+\frac{x^2}{3!} + \frac{x^4}{5!}+\frac{x^6}{7!}+\dots
\biggr)
\\
&=
xf(-x^2)
=
x\,\mathstrut_1F_2\biggl(
\begin{matrix}1\\1,\frac{3}{2}\end{matrix}
;x^2
\biggr)
=
x\,\mathstrut_0F_1\biggl(
\begin{matrix}\text{---}\\,\frac{3}{2}\end{matrix}
;x^2
\biggr).
\end{align*}
Bis auf das Vorzeichen des Arguments der hypergeometrischen Funktion
ist diese Darstellung identisch mit der von $\sin x$.
Dies illustriert die Rolle der hypergeometrischen Funktionen als
``grosse Vereinheitlichung'' der bekannten speziellen Funktionen.

%
% Ableitung und Stammfunktion
%
\subsection{Ableitung und Stammfunktion hypergeometrischer Funktionen}
Sowohl Ableitung wie auch Stammfunktion einer hypergeometrischen
Funktion lässt sich immer durch hypergeometrische Reihen ausdrücken.

\subsubsection{Ableitung}
Wir gehen aus von der Funktion
\begin{equation}
f(x)
=
\mathstrut_nF_m\biggl(
\begin{matrix}a_1,\dots,a_n\\b_1,\dots,b_m\end{matrix};
x\biggr)
=
\sum_{k=0}^\infty
\frac{
(a_1)_k\cdot\ldots\cdot(a_n)_k
}{
(b_1)_k\cdot\ldots\cdot(b_m)_k
}
\frac{x^k}{k!}.
\label{buch:rekursion:hypergeometrisch:eqn:f}
\end{equation}
Die Ableitung von $f(x)$ ist
\[
f'(x)
=
\sum_{k=0}^\infty
\frac{
(a_1)_k\cdot\ldots\cdot(a_n)_k
}{
(b_1)_k\cdot\ldots\cdot(b_m)_k
}
\frac{x^{k-1}}{(k-1)!}
=
\sum_{k=1}^\infty
\frac{
(a_1)_{k+1}\cdot\ldots\cdot(a_n)_{k+1}
}{
(b_1)_{k+1}\cdot\ldots\cdot(b_m)_{k+1}
}
\frac{x^k}{k!}.
\]
Der Koeffizient besteht zwar aus lauter Pochhammer-Symbolen, aber sie
haben jeweils zu einen Faktor zuviel.
Indem man den jeweils ersten Faktor ausklammert, kann man die
Terme wieder in die Form einer hypergeometrischen Reihe bringen.
\begin{align*}
f'(x)
&=
\sum_{k=1}^\infty
\frac{
a_1(a_1)_{k}\cdot\ldots\cdot a_n(a_n)_{k}
}{
b_1(b_1)_{k}\cdot\ldots\cdot b_m(b_m)_{k}
}
\frac{x^k}{k!}
\\
&=
\sum_{k=1}^\infty
\frac{
a_1\cdot\ldots\cdot a_n
}{
b_1\cdot\ldots\cdot b_m
}
\frac{
(a_1+1)_{k}\cdot\ldots\cdot(a_n+1)_{k}
}{
(b_1+1)_{k}\cdot\ldots\cdot(b_m+1)_{k}
}
\frac{x^k}{k!}
\\
&=
\frac{
a_1\cdot\ldots\cdot a_n
}{
b_1\cdot\ldots\cdot b_m
}
\,
\mathstrut_nF_m\biggl(
\begin{matrix}a_1+1,\dots,a_n+1\\b_1+1,\dots,b_m+1\end{matrix};
x\biggr).
\end{align*}

\begin{beispiel}
Die Kosinus-Funktion
\[
\cos x
=
1 - \frac{x^2}{2!} + \frac{x^4}{4!} - \frac{x^6}{6!} + \dots
=
\sum_{k=0}^\infty
\frac{(-1)^k}{(2k)!}x^{2k}
\]
kann wie folgt als hypergeometrische Funktion geschrieben werden.
Der Nenner hat $2k$ Faktoren, er muss also aus zwei Pochhammer-Symbolen
zusammengesetzt werden.
Dazu muss er erst um den Faktor $2^{2k}$ gekürzt werden, was
\[
\frac{(2k)!}{2^{2k}}
=
\frac12\cdot\frac32\cdot\frac52\cdot\ldots\cdot\frac{2k-1}2
\cdot
\frac22\cdot\frac42\cdot\frac62\cdot\ldots\cdot\frac{2k}2
=
({\textstyle\frac12})_k\cdot k!.
\]
Damit kann jetzt die Kosinus-Funktion als
\begin{align*}
\cos x
&=
\sum_{k=0}^\infty
\frac{2^k}{(2k)!}\biggl(\frac{-x^2}{4}\biggr)^k
=
\sum_{k=0}^\infty
\frac{1}{(\frac12)_k}
\frac{1}{k!}\biggl(\frac{-x^2}{4}\biggr)^k
=
\mathstrut_0F_1\biggl(;\frac12;-\frac{x^2}4\biggr)
\end{align*}
geschrieben werden kann.

Die Ableitung der Kosinus-Funktion ist daher
\begin{align*}
\frac{d}{dx} \cos x
&=
\frac{d}{dx}
\mathstrut_0F_1\biggl(;\frac12;-\frac{x^2}4\biggr)
=
\frac{1}{\frac12}
\,
\mathstrut_0F_1\biggl(;\frac32;-\frac{x^2}4\biggr)
\cdot\biggl(-\frac{x}2\biggr)
=
-x
\,
\mathstrut_0F_1\biggl(;\frac32;-\frac{x^2}4\biggr)
\intertext{Dies stimmt mit der in
\eqref{buch:rekursion:hypergeometrisch:eqn:sinhyper}
gefundenen Darstellung der Sinusfunktion mit Hilfe der hypergeometrischen
Funktion $\mathstrut_0F_1$ überein, es ist also wie erwartet}
&=-\sin x.
\qedhere
\end{align*}
\end{beispiel}

\subsubsection{Stammfunktion}
Eine Stammfunktion kann man auf die gleiche Art und Weise wie
die Ableitung finden.
Termweises Integrieren der Funktion
\eqref{buch:rekursion:hypergeometrisch:eqn:f}
ergibt
\begin{align}
\int f(x)\,dx
&=
\sum_{k=0}^\infty
\frac{
(a_1)_k\cdot\ldots\cdot(a_n)_k
}{
(b_1)_k\cdot\ldots\cdot(b_m)_k
}
\frac{x^{k+1}}{(k+1)!}.
\notag
\intertext{Wieder muss man die Pochhammer-Symbole durch solche mit
einem zusätzlichen Faktor schreiben.
Dies ist möglich, wenn keiner der Parameter $a_i=1$ und $b_j=1$
ist.
Die Stammfunktion wird daher
}
&=
\sum_{k=1}^\infty
\frac{
(a_1-1)(a_1)_k
\cdot\ldots\cdot
(a_n-1)(a_n)_k
}{
(b_1-1)(b_1)_k
\cdot\ldots\cdot
(b_m-1)(b_m)_k
}
\frac{x^k}{k!}
\notag
\\
&=
\sum_{k=1}^\infty
\frac{
(a_1-1)_{k+1}
\cdot\ldots\cdot
(a_n-1)_{k+1}
}{
(b_1-1)_{k+1}
\cdot\ldots\cdot
(b_m-1)_{k+1}
}
\frac{x^k}{k!}
\label{buch:rekursion:hypergeometrisch:eqn:stammfunktion:summe}
\\
&=
\mathstrut_nF_m\biggl(
\begin{matrix}
a_1-1,\dots,a_n-1\\
b_1-1,\dots,b_m-1
\end{matrix}
;x
\biggr)
-
\frac{(a_1-1)\dots(a_n-1)}{(b_1-1)\dots(b_m-1)}.
\notag
\end{align}
Der Term auf der rechten Seite kompensiert den konstanten
Term, der in der hypergeometrischen Funktion $\mathstrut_nF_m$
vorkommt, aber nicht in der
Summe~\eqref{buch:rekursion:hypergeometrisch:eqn:stammfunktion:summe}.


