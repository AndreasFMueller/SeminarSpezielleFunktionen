%
% Beta-Integrale
%
% (c) 2021 Prof Dr Andreas Müller, OST Ostschweizer Fachhochschule
%
\subsection{Die Beta-Funktion
\label{buch:rekursion:gamma:subsection:beta}}
Die Eulersche Integralformel für die Gamma-Funktion in
Definition~\ref{buch:rekursion:def:gamma} wurde bisher nicht
gerechtfertigt.
In diesem Abschnitt wird das Beta-Integral eingeführt, eine Funktion
von zwei Variablen, welches eine Integral-Definition mit einer
reichaltigen Menge von Rekursionsbeziehungen hat, die sich direkt auf
die Gamma-Funktion zurückführen lassen.
Daraus wird sich dann ein Beweis für die Integralformel für die
Gamma-Funktion ergeben.

\begin{definition}
\label{buch:rekursion:gamma:def:beta-funktion}
Das Beta-Integral ist das Integral
\[
B(x,y)
=
\int_0^1 t^{x-1} (1-t)^{y-1}\,dt
\]
für $\operatorname{Re}x>0$, $\operatorname{Re}y>0$.
\end{definition}

Aus der Definition kann man sofort ablesen, dass $B(x,y)=B(y,x)$.
Für $y=1$ folgt ausserdem
\begin{equation}
B(x,1)
=
\int_0^1 t^{x-1}\,dt
=
\biggl[ \frac{t^x}{x}\biggr]_0^1
=
\frac{1}{x}.
\label{buch:rekursion:gamma:betax1}
\end{equation}
Speziell gilt $B(1,1)=1$.

\subsubsection{Rekursionsformeln für das Beta-Integral}
Aus der Definition folgt direkt
\begin{align*}
B(x,y+1)
&=
\int_0^1 t^{x-1} (1-t)^{y+1-1}\,dt
=
\int_0^1 (1-t) t^{x-1} (1-t)^{y-1}\,dt
\\
&=
\int_0^1 t^{x-1} (1-t)^{y-1}\,dt
-
\int_0^1 t^{x} (1-t)^{y-1}\,dt
\\
&=
B(x,y) - B(x+1,y)
\end{align*}
oder
\begin{equation}
B(x,y) = B(x+1,y) + B(x,y+1).
\label{buch:rekursion:gamma:betarek1}
\end{equation}
%
%XXX Vergleich mit der Rekursionsformel für Binomialkoeffizienten
%
Durch partielle Integration kann man eine weitere Rekursionsformel finden.
Dazu berechnet man
\begin{align}
B(x,y+1)
&=
\int_0^1 t^{x-1}(1-t)^{y}\,dt
\notag
\\
&=
\biggl[\frac{t^x}x(1-t)^y\biggr]_0^1
+
\frac{y}x \int_0^1 t^x(1-t)^{y-1}\,dt
\notag
\\
&=
 \frac{y}x B(x+1,y).
\label{buch:rekursion:gamma:betarek2}
\end{align}
Durch Gleichsetzen
\eqref{buch:rekursion:gamma:betarek1}
und
\eqref{buch:rekursion:gamma:betarek2}
entsteht die Rekursionsformel
\[
B(x,y)-B(x,y+1)
=
B(x+1,y)
=
\frac{x}{y}B(x,y+1)
\]
oder
\begin{equation}
B(x,y)
=
\frac{x+y}{y}B(x,y+1).
\label{buch:rekursion:gamma:betarek3}
\end{equation}

\subsubsection{Beta-Funktion und Gamma-Funktion}
Die Rekursionsbeziehung~\eqref{buch:rekursion:gamma:betarek3}
kann jetzt dazu verwendet werden, eine Darstellung der Beta-Funktion
durch die Gamma-Funktion zu finden.
Durch $n$-fache Anwendung von \eqref{buch:rekursion:gamma:betarek3}
ergibt sich zunächst
\begin{align*}
B(x,y)
&=
\frac{x+y}{y}
B(x,y+1)
=
\frac{x+y}{y}
\frac{x+y+1}{y+1}
B(x,y+2)
\\
&=
\frac{x+y}{y}
\frac{x+y+1}{y+1}
\cdot
\ldots
\cdot
\frac{x+y+n-1}{y+n-1}
B(x,y+n)
=
\frac{(x+y)_n}{(y)_n}
B(x,y+n)
\intertext{Die Beta-Funktion auf der rechten Seite kann als Integral
geschrieben werden:}
&=
\frac{(x+y)_n}{(y)_n}
\int_0^1 t^{x-1}(1-t)^{y+n-1}\,dt.
\end{align*}
Wir halten dieses Zwischenresultat für spätere Verwendung fest.

\begin{lemma}
\label{buch:rekursion:gamma:betareklemma}
Für $n\in\mathbb{N}$ gilt
\[
B(x,y+n) = \frac{(y)_n}{(x+y)_n} B(x,y).
\]
\end{lemma}

Wir streben an, mit dem Grenzübergang $n\to\infty$ aus den
Pochhammer-Symbolen Gamma-Funktionen zu machen, dazu müssen gemäss
Definition~\ref{buch:rekursion:gamma:def:definition} weitere Faktoren
$1/(n!\,n^{x-1})$ vorhanden sein.
Wir erweitern geeignet und nehmen die übrig bleibenden Faktoren in
das Integral.
So ergibt sich
\begin{align}
B(x,y)
&=
\frac{(x+y)_n}{n!\, n^{x+y-1}}
\frac{n!\,n^{y-1}}{(y)_n}
\int_0^1 n^{x} t^{x-1}(1-t)^{y+n-1}\,dt.
\notag
\intertext{Mit der Substition $s/n=t$ wird das Integral zu einem Integral
über das Interval $[0,n]$}
&=
\frac{(x+y)_n}{n!\, n^{x+y-1}}
\frac{n!\,n^{y-1}}{(y)_n}
\int_0^n
n^{x}
\biggl(\frac{s}{n}\biggr)^{x-1}
\biggl(1-\frac{s}{n}\biggr)^{y+n-1}
\,\frac{ds}{n}.
\notag
\\
&=
\frac{(x+y)_n}{n!\, n^{x+y-1}}
\frac{n!\,n^{y-1}}{(y)_n}
\int_0^n
n^{x-1}
\biggl(\frac{s}{n}\biggr)^{x-1}
\biggl(1-\frac{s}{n}\biggr)^{y+n-1}
\,ds.
\intertext{Beim Grenzübergang $n\to\infty$ wird daraus}
&=
\underbrace{\frac{(x+y)_n}{n!\, n^{x+y-1}}}_{\displaystyle \to 1/\Gamma(x+y)}
\underbrace{\frac{n!\,n^{y-1}}{(y)_n}}_{\displaystyle\to \Gamma(y)}
\int_0^n
s^{x-1}
\underbrace{\biggl(1-\frac{s}{n}\biggr)^{n}}_{\displaystyle\to e^{-s}}
\underbrace{\biggl(1-\frac{s}{n}\biggr)^{y-1}}_{\displaystyle\to 1}
\,ds.
\notag
\\
&\to \frac{\Gamma(y)}{\Gamma(x+y)} \int_0^\infty s^{x-1}e^{-s}\,ds.
\label{buch:rekursion:gamma:betagamma}
\end{align}
Das Integral im letzten Ausdruck ist die Integraldarstellung für 
die Gamma-Funktion von Definition~\ref{buch:rekursion:def:gamma},
die bis anhin noch nicht gerechtfertigt wurde.

In~\eqref{buch:rekursion:gamma:betax1} ist gezeigt worden, dass
$B(x,1)=1/x$.
Andererseits zeigt \eqref{buch:rekursion:gamma:betagamma} für $y=1$,
dass
\begin{align}
\frac1x
=
B(x,1)
&= 
\frac{\Gamma(1)}{\Gamma(x+1)}\int_0^\infty s^{x-1}e^{-s}\,ds.
\notag
\intertext{%
Wegen $\Gamma(1)=1$ und $\Gamma(x+1)=x\Gamma(x)$ finden wir nach
Multiplikation mit $x\Gamma(x)$:}
\Gamma(x)
&=
\int_0^\infty s^{x-1}e^{-s}\,ds,
\label{buch:rekursion:gamma:integralbeweis}
\end{align}
was die Integraldarstellung
von Definition~\ref{buch:rekursion:def:gamma},
der Gamma-Funktion beweist.
Durch Einsetzen der Integralformel im Ausdruck
\eqref{buch:rekursion:gamma:betagamma} folgt der folgende
Satz.

\begin{satz}
Die Beta-Funktion kann aus der Gamma-Funktion nach
\begin{equation}
B(x,y) = \frac{\Gamma(x)\Gamma(y)}{\Gamma(x+y)}
\label{buch:rekursion:gamma:betagamma}
\end{equation}
berechnet werden.
\end{satz}

\subsubsection{Nochmals der Wert von $\Gamma(\frac12)$?}
Der Wert von $\Gamma(\frac12)=\sqrt{\pi}$ wurde bereits in
\eqref{buch:rekursion:gamma:wert12}
direkt mit Hilfe der Integraldefinition berechnet.
Hier wird eine alternative Berechnungsmöglichkeit mit Hilfe der
Beta-Funktion vorgestellt.

Als Anwendung der Formel~\eqref{buch:rekursion:gamma:betagamma}
untersuchen wir den Fall $y=1-x$.
In diesem Fall wird der Nenner zu $\Gamma(x+1-x)=\Gamma(1)=1$ und damit
\begin{equation}
\Gamma(x)\Gamma(1-x)
=
B(x,1-x) 
=
\int_0^1 t^{x-1}(1-t)^{-x}\,dt.
\label{buch:rekursion:gamma:spiegelung-betaintegral}
\end{equation}
Sofern man in der Lage ist, das Integral auf der rechten Seite von
\eqref{buch:rekursion:gamma:spiegelung-betaintegral} auszuwerten,
kann man eine einfache Beziehung zwischen zwei Werten der Gamma-Funktion
an Stellen, die durch eine Spiegelung an der Geraden
$\operatorname{Re}x=\frac12$ auseinander hervorgehen.
Für $x=\frac12$ wird der Ausdruck besonders einfach:
\[
\Gamma({\textstyle\frac12})^2
=
\int_0^1 t^{-\frac12}(1-t)^{-\frac12}\,dt
=
\int_0^1 \frac{1}{\sqrt{t(1-t)\mathstrut}}\,dt.
\]
Mit der Substition $t=\sin^2 s$ wird daraus
\[
\int_0^{\frac{\pi}2}
\frac{1}{
\sqrt{\sin^2s(1-\sin^2s)}
}
2\sin s\cos s
\,ds
=
2
\int_0^{\frac{\pi}2}
\,ds
=
\pi,
\]
wobei wir $dt = 2\sin s\cos s\,ds$ verwendet haben.
Somit folgt
\begin{equation}
\Gamma({\textstyle\frac12})^2 = \pi
\qquad\Rightarrow\qquad
\Gamma({\textstyle\frac12}) = \sqrt{\pi}.
\label{buch:rekursion:gamma:gamma12}
\end{equation}
Matt Parker hat auf seinem Youtube-Kanal {\em Stand-up Maths} dieses Resultat
sogar zum Titel eines Videos\footnote{\url{https://youtu.be/dGnIJFzkLI4}}
gemacht:
{\em What is the factorial of $-\nicefrac{1}{2}$?}
Die Antwort ist natürlich nur möglich, indem man
$(-\frac12)!$ als Wert
\[
(-{\textstyle\frac12})!
=
\Gamma(-{\textstyle\frac12}+1)
=
\Gamma({\textstyle\frac12})
=
\sqrt{\pi}
\]
der Gamma-Funktion interpretiert.

\subsubsection{Alternative Parametrisierungen}
Die Substitution $t=\sin^2 s$ hat im vorangegangenen Abschnitt
ermöglicht, $\Gamma(\frac12)$ zu ermitteln.
Die Substition erlaubt aber auch, das Beta-Integral in eine alternative
Form zu bringen.
Aus der Definition~\ref{buch:rekursion:gamma:def:beta-funktion}
wird damit
\begin{align*}
B(x,y)
&=
\int_0^1 t^{x-1} (1-t)^{y-1}\,dt
\\
&=
2
\int_0^{\frac{\pi}2} \sin^{2(x-1)} s\cdot (1-\sin^2 s)^{y-1}
\cdot \sin s\cos s\,ds
\\
&=
2
\int_0^{\frac{\pi}2} \sin^{2x-1}s \cos^{2y-1} s\,ds.
\intertext{Unter Verwendung der Formel~\eqref{buch:rekursion:gamma:betagamma},
die die Beta-Funktion durch Gamma-Funktionen auszudrücken erlaubt, findet
man die Formel}
\int_0^{\frac{\pi}2} \sin^{2x-1}s \cos^{2y-1} s\,ds
&=
\frac{\Gamma(x)\Gamma(y)}{2\Gamma(x+y)}
\end{align*}
für ein bestimmtes Integral von Potenzen von Sinus- und Kosinus-Funktionen.

Die alternative Substitution $t = s/(s+1)$ verwandelt das Beta-Integral
$B(x,y)$ in ein Integral über die positive Halbachse ab:
\begin{align}
B(x,y)
&=
\int_0^1 t^{x-1}(1-t)^{y-1}\,dt
\notag
\\
&=
\int_0^\infty
\frac{s^{x-1}}{(s+1)^{x-1}}
\frac{1}{(s+1)^{y-1}}
\frac{ds}{(s+1)^2}
\notag
\\
&=
\int_0^\infty
\frac{s^{x-1}}{(s+1)^{x+y}}\,ds,
\label{buch:rekursion:gamma:beta:sinf}
\end{align}
wobei wir
\[
\frac{dt}{ds}
=
\frac{d}{ds}
\frac{s}{s+1}
=
\frac{(s+1)-s}{(s+1)^2}
=
\frac{1}{(s+1)^2}
\]
verwendet haben.
Diese Darstellung des Beta-Integrals wird später
% XXX Ort ergänzen
dazu verwendet, die Spiegelungsformel für die Gamma-Funktion
herzuleiten.

Eine weitere mögliche Parametrisierung verwendet $t = (1+s)/2$
mit $dt=\frac12 ds$.
Damit wird das Beta-Integral
\begin{equation}
B(x,y)
=
\int_0^1 t^{x-1}(1-t)^{y-1}\,dt
=
\frac12
\int_{-1}^1
\biggl(\frac{1+s}2\biggr)^{x-1}
\biggl(\frac{1-s}2\biggr)^{y-1}
\,ds
=
2^{1-x-y}
\int_{-1}^1
(1+s)^{x-1}(1-s)^{y-1}
\,ds.
\label{buch:rekursion:gamma:beta:symm}
\end{equation}

\subsubsection{Die Verdoppelungsformel von Legendre}
Die trigonometrische Substitution kann dazu verwendet werden, die
Legendresche Verdoppelungsformel für die Gamma-Funktion herzuleiten.

\begin{satz}[Legendre]
\[
\Gamma(x)\Gamma(x+{\textstyle\frac12})
=
2^{1-2x}\sqrt{\pi}
\Gamma(2x)
\]
\end{satz}

\begin{proof}[Beweis]
Der Wert $\Gamma(2x)$ entsteht, wenn man $B(x,x)$ mit Hilfe der
Gamma-Funktion als
\[
B(x,x)
=
\frac{\Gamma(x)^2}{\Gamma(2x)}
\]
schreibt.
Das Ziel ist, $B(x,x)$ auf einem alternativen Weg zu berechnen.

Mit Hilfe von \eqref{buch:rekursion:gamma:beta:symm}
kann man das Beta-Integral zu
\begin{align*}
B(x,x)
&=
2^{1-2x}
\int_{-1}^1
(1+s)^{x-1}(1-s)^{x-1}
\,ds
=
2^{1-2x}
\int_{-1}^1(1-s^2)^{x-1}\,ds
\end{align*}
vereinfachen.
Der Integrand ist gerade, es folgt
\[
B(x,x)
=
2^{1-2x}
\cdot 2
\int_0^1(1-s^2)^{x-1}\,ds.
\]
Das Integral kann mit der Substitution $s^2=t$ wieder in die Form
eines Beta-Integrals gebracht werden:
\begin{align*}
2\int_0^1(1-s^2)^{x-1}\,ds
&=
\int_0^1 (1-t)^{x-1} \,\frac{dt}{\sqrt{t}}
=
\int_0^1 t^{\frac12-1}(1-t)^{x-1}\,dt
=
B({\textstyle\frac12},x).
\end{align*}
In der Substitution haben wir $2s\,ds = dt$ oder $2\,ds = dt/\sqrt{t}$
verwendet.
Das letzte Beta-Integral kann man nun wieder mit Gamma-Funktionen
schreiben, nämlich als
\[
B({\textstyle\frac12},x)
=
\frac{\Gamma({\textstyle\frac12})\Gamma(x)}{\Gamma(x+{\textstyle\frac12})}.
\]
Setzt man alles zusammen, erhält man jetzt
\begin{align*}
\frac{\Gamma(x)^2}{\Gamma(2x)}
&=
\frac1{2^{2x-1}}
\frac{\Gamma({\textstyle\frac12})\Gamma(x)}{\Gamma(x+{\textstyle\frac12})}
\\
\Rightarrow\qquad
\Gamma(x)\Gamma(x+{\textstyle\frac12})
&=
2^{1-2x}
\Gamma({\textstyle\frac12})\Gamma(2x)
=
2^{1-2x}\sqrt{\pi}\Gamma(2x),
\end{align*}
wobei wir den bekannten Wert $\Gamma(\frac12)=\sqrt{\pi}$ verwendet haben.
\end{proof}

Setzt man $x=\frac12$ in die Verdoppelungsformel ein, erhält man
\[
\Gamma({\textstyle\frac12})\Gamma(1) = 2^{1-2\frac12}\sqrt{\pi}\Gamma(1)
\qquad\Rightarrow\qquad
\Gamma({\textstyle\frac12}) = \sqrt{\pi},
\]
in Übereinstimmung mit dem aus \eqref{buch:rekursion:gamma:gamma12}
bereits bekannten Wert.

\subsubsection{Beta-Funktion und Binomialkoeffizienten}
Die Binomialkoeffizienten können mit Hilfe der Fakultät als
\begin{align*}
\binom{n}{k}
&=
\frac{n!}{(n-k)!\,k!}
\intertext{geschrieben werden.
Drückt man die Fakultäten durch die Gamma-Funktion aus, erhält man}
&=
\frac{\Gamma(n+1)}{\Gamma(n-k+1)\Gamma(k+1)}.
\intertext{Schreibt man $x=k-1$ und $y=n-k+1$, wird daraus 
wegen $x+y=k+1+n-k+1=n+2=(n+1)+1$}
&=
\frac{\Gamma(x+y-1)}{\Gamma(x)\Gamma(y)}.
\intertext{Die Rekursionsformel für die Gamma-Funktion erlaubt,
den Zähler umzuwandeln in $\Gamma(x+y-1)=\Gamma(x+y)/(x+y-1)$, so dass
der Binomialkoeffizient schliesslich}
&=
\frac{\Gamma(x+y)}{(x+y-1)\Gamma(x)\Gamma(y)}
=
\frac{1}{(n-1)B(n-k+1,k+1)}
\label{buch:rekursion:gamma:binombeta}
\end{align*}
geschrieben werden kann.
Die Rekursionsbeziehung
\[
\binom{n+1}{k} = \binom{n}{k-1} + \binom{n}{k}
\]
der Binomialkoeffizienten erzeugt das vertraute Pascal-Dreieck,
die Formel \eqref{buch:rekursion:gamma:binombeta} für die
Binomialkoeffizienten macht daraus
\[
\frac{n-1}{B(n-k,k-1)}
=
\frac{n-2}{B(n-k,k-2)}
+
\frac{n-2}{B(n-k-1,k-1)},
\]
die für ganzzahlige Argumente gilt.
Wir wollen nachrechnen, dass dies für beliebige Argumente gilt.
\begin{align*}
\frac{(n-1)\Gamma(n-1)}{\Gamma(n-k)\Gamma(k-1)}
&=
\frac{(n-2)\Gamma(n-2)}{\Gamma(n-k)\Gamma(k-2)}
+
\frac{(n-2)\Gamma(n-2)}{\Gamma(n-k-1)\Gamma(k-1)}
\\
\frac{\Gamma(n)}{\Gamma(n-k)\Gamma(k-1)}
&=
\frac{\Gamma(n-1)}{\Gamma(n-k)\Gamma(k-2)}
+
\frac{\Gamma(n-1)}{\Gamma(n-k-1)\Gamma(k-1)}
\intertext{Durch Zusammenfassen der Faktoren im Zähler mit Hilfe
der Rekursionsformel für die Gamma-Funktion und Multiplizieren
mit dem gemeinsamen Nenner
$\Gamma(n-k)\Gamma(k-1)=(n-k-1)\Gamma(n-k-1)(k-2)\Gamma(k-2)$ wird daraus}
\Gamma(n)
&=
(k-2)
\Gamma(n-1)
+
(n-k-1)
\Gamma(n-1)
\intertext{Indem wir die Rekursionsformel für die Gamma-Funktion auf
die rechte Seite anwenden können wir erreichen, dass in allen Termen
ein Faktor
$\Gamma(n-1)$ auftritt:}
(n-1)\Gamma(n-1)
&=
(k-2)\Gamma(n-1)
+
(n+k-1)\Gamma(n-1)
\\
n-1
&=
k-2
+
n-k-1
\end{align*}

