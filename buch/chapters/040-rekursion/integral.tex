%
% integral.tex
%
% (c) 2022 Prof Dr Andreas Müller, OST Ostschweizer Hochschule
%
\subsection{Integraldarstellung und der Satz von Bohr-Mollerup
\label{buch:subsection:integral-eindeutig}}
Die Integralformel
\[
f(x)
=
\int_0^\infty t^{x-1}e^{-t}\,dt
\]
für die Gamma-Funktion erfüllt die Funktionalgleichung der Gamma-Funktion.
Aus dem Satz von Bohr-Mollerup~\ref{buch:satz:bohr-mollerup} folgt,
dass $f(x)=\Gamma(x)$, wenn gezeigt werden kann, dass $\log f(x)$
konvex ist.
Dies soll im Folgenden gezeigt werden.

\subsubsection{Logarithmische Ableitung}
Die Ableitungen der Funktion $\log f(x)$ sind die erste und
zweite logarithmische
Ableitung
\begin{align}
\frac{d}{dx}\log f(x)
&=
\frac{f'(x)}{f(x)}
\notag
\\
\frac{d^2}{dx^2} \log f(x)
&=
\frac{f''(x)f(x)-f'(x)^2}{f(x)^2}.
\label{buch:rekursion:eqn:zweiteablteitung}
\end{align}
Durch Ableiten unter dem Integralzeichen können die Ableitungen
von $f$ als
\begin{align*}
f'(x)
&=
\int_0^\infty \log(t)\, t^{x-1} e^{-t}\,dt
\\
f''(x)
&=
\int_0^\infty \log(t)^2\, t^{x-1} e^{-t}\,dt
\end{align*}
bestimmt werden.
Um nachzuweisen, dass $\log f(x)$ konvex ist, muss nur gezeigt werden,
dass die zweite logarithmische Ableitung von $f(x)$ positiv ist, was
gemäss~\eqref{buch:rekursion:eqn:zweiteablteitung} mit
\begin{equation}
f''(x)f(x)-f'(x)^2
=
\int_0^\infty \log(t)^2\, t^{x-1}e^{-t}\,dt
\int_0^\infty t^{x-1}e^{-t}\,dt
-
\biggl(
\int_0^\infty \log(t)\, t^{x-1}e^{-t}\,dt
\biggr)^2
\ge 0
\label{buch:rekursion:gamma-integral:ungleichung}
\end{equation}
gleichbedeutend ist.

\subsubsection{Skalarprodukt}
Die Integral in~\eqref{buch:rekursion:gamma-integral:ungleichung}
können als Werte eines Skalarproduktes von Funktionen auf $\mathbb{R}^+$
gelesen werden.
Dazu definieren wir
\begin{align}
\langle u,v\rangle
&=
\int_0^\infty u(t)v(t)\,t^{x-1}e^{-t}\,dt
\label{buch:rekursion:gamma-integral:eqn:skalarprodukt}
\\
\|u\|^2
&=
\int_0^\infty u(t)^2 \,t^{x-1}e^{-t}\,dt,
\notag
\end{align}
für alle Funktionen $u$ und $v$, für die die Integrale definiert sind.

\subsubsection{Cauchy-Schwarz-Ungleichung}
Die Cauchy-Schwarz-Ungleichung für das
Skalarprodukt~\eqref{buch:rekursion:gamma-integral:eqn:skalarprodukt}
für die Funktion $u(t)=1$ und $v(t)=\log(t)$
lautet
\[
|\langle u,v\rangle|^2
=
\biggl|
\int_0^1 \log(t)\,t^{x-1}e^{-t}\,dt
\biggr|^2
\le
\|u\|^2\cdot \|v\|^2
=
\int_0^\infty 1\cdot t^{x-1}e^{-t}\,dt
\int_0^\infty \log(t)^2\cdot t^{x-1}e^{-t}\,dt.
\]
Daraus folgt aber durch Umstellen unmittelbar die
Ungleichung~\eqref{buch:rekursion:gamma-integral:ungleichung}.
Damit ist gezeigt, dass $\log f(t)$ konvex ist und nach
dem Satz~\ref{buch:satz:bohr-mollerup} folgt nun, dass $f(x)=\Gamma(x)$.

