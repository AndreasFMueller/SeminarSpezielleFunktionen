%
% fibonacci.tex -- template for standalon tikz images
%
% (c) 2021 Prof Dr Andreas Müller, OST Ostschweizer Fachhochschule
%
\documentclass[tikz]{standalone}
\usepackage{amsmath}
\usepackage{times}
\usepackage{txfonts}
\usepackage{pgfplots}
\usepackage{csvsimple}
\usetikzlibrary{arrows,intersections,math}
\begin{document}
\pgfplotsset{compat=1.18}
\input{fibonaccigrid.tex}
\def\skala{1}
\definecolor{darkgreen}{rgb}{0,0.6,0}
\begin{tikzpicture}[>=latex,thick,scale=\skala]

\def\achsen{
	\draw[->] (-2.1,0) -- (10.8,0)
		coordinate[label={$\operatorname{Re}F(z)$}];
	\draw[->] (0,-2.6) -- (0,2.7)
		coordinate[label={right:$\operatorname{Im}F(z)$}];
	\node at (10.8,2.7) {$\mathbb{C}$};
}

\def\zahl#1#2{
	\fill[color=white,opacity=0.8]
		({#1-0.65},-0.6) rectangle ({#1+0.65},-0.2);
	\node[color=darkgreen] at (#1,-0.43) {$#2\mathstrut$};
}

\draw[color=gray!20,line width=2.0pt] (-2,3.2) -- (11.3,3.2);

\def\topskala{1.55}

\draw[->,color=darkgreen!30,line width=5pt]
		({-0.2*\topskala},4.3) to[out=-120,in=120] (-0.1,0.2);
\draw[->,color=darkgreen!30,line width=5pt]
		({0.9*\topskala},4.3) to[out=-110,in=70] (0.9,-5.1);
\draw[->,color=darkgreen!30,line width=5pt]
		({2.0*\topskala},4.3) to[out=-90,in=76] (1.0,-10.6);

\draw[->,color=darkgreen!30,line width=5pt]
		({3.0*\topskala},4.3) to[out=-90,in=90] (2.0,0.2);
\draw[->,color=darkgreen!30,line width=5pt]
		({3.9*\topskala},4.3) to[out=-110,in=80] (3.0,-5.3);
\draw[->,color=darkgreen!30,line width=5pt]
		({4.8*\topskala},4.3) to[out=-120,in=90] (5.0,-10.8);

\draw[->,color=darkgreen!30,line width=5pt]
		({6.0*\topskala},4.3) to[out=-90,in=65] (8.1,0.2);


\begin{scope}[yshift=4.8cm,scale=\topskala]

	\draw[->] (-0.7,0) -- (6.8,0)
		coordinate[label={$\operatorname{Re}z$}];
	\draw[->] (0,-0.7) -- (0,0.8)
		coordinate[label={right:$\operatorname{Im}z$}];
	\foreach \x in {-0.5,-0.4,...,6.501}{
		\draw[color=gray,line width=0.1pt] (\x,-0.5) -- (\x,0.5);
	}
	\foreach \y in {-0.5,-0.4,...,0.501}{
		\draw[color=gray,line width=0.1pt] (-0.5,\y) -- (6.5,\y);
	}

	\foreach \n in {0,3,6}{
		\foreach \x in {-0.5,-0.4,...,0.501}{
			\draw[color=red,line width=0.7pt]
				({\n+\x},-0.5) -- ({\n+\x},0.5);
		}
		\foreach \y in {-0.5,-0.4,...,0.501}{
			\draw[color=blue,line width=0.7pt]
				({\n-0.5},\y) -- ({\n+0.5},\y);
		}
	}
	\draw[color=darkgreen,line width=1.4pt] (0,0) -- (6.5,0);
	\foreach \n in {0,...,6}{
		\fill[color=white,opacity=0.8]
			({\n-0.1},-0.28) rectangle ({\n+0.1},-0.05);
		\node[color=darkgreen] at (\n,0) [below] {$\n$};
		\fill[color=darkgreen] (\n,0) circle[radius=0.05];
		\fill[color=white] (\n,0) circle[radius=0.02];
	}

	\node at (6.8,0.8) {$\mathbb{C}$};

\end{scope}

\begin{scope}[scale=1]
	\begin{scope}
		\clip (-2,-2.6) rectangle (10.5,2.6);
		\fibgrid
	\end{scope}
	\achsen
	\begin{scope}
		\clip (-2,-2.6) rectangle (10.5,2.6);
		\fibzero
		\fibthree
		\fibsix
	\end{scope}
	\fibcurve
	\node[color=magenta] at (-1.3,-1.8) {$z=0$};
	\node[color=magenta] at (1.9,0.8) {$z=3$};
	\node[color=magenta] at (8,2.3) {$z=6$};
	\zahl{0}{F(0)=0}
	\zahl{2}{F(3)=2}
	\zahl{8}{F(6)=8}
\end{scope}

\begin{scope}[yshift=-5.5cm,scale=1]
	\begin{scope}
		\clip (-2,-2.6) rectangle (10.5,2.6);
		\fibgrid
	\end{scope}
	\achsen
	\begin{scope}
		\clip (-2,-2.6) rectangle (10.5,2.6);
		\fibone
		\fibfour
	\end{scope}
	\fibcurve
	\node[color=magenta] at (1,1.2) {$z=1$};
	\node[color=magenta] at (3,1.1) {$z=4$};
	\zahl{1}{F(1)=1}
	\zahl{3}{F(4)=3}
\end{scope}

\begin{scope}[yshift=-11cm,scale=1]
	\begin{scope}
		\clip (-2,-2.6) rectangle (10.5,2.6);
		\fibgrid
	\end{scope}
	\achsen
	\begin{scope}
		\clip (-2,-2.6) rectangle (10.5,2.6);
		\fibtwo
		\fibfive
	\end{scope}
	\fibcurve
	\node[color=magenta] at (0.7,1.1) {$z=2$};
	\node[color=magenta] at (5,1.5) {$z=5$};
	\zahl{2}{F(2)=1}
	\zahl{5}{F(5)=5}
\end{scope}

\end{tikzpicture}
\end{document}

