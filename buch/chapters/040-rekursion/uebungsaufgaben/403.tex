Finden Sie eine Formel für $\Gamma(\frac12+n)$ für $n\in\mathbb{N}$.

\begin{loesung}
Die Funktionalgleichung für die Gamma-Funktion bedeutet
\begin{align*}
\Gamma({\textstyle\frac12}+n)
&=
({\textstyle\frac12}+n-1)
\Gamma({\textstyle\frac12}+n-1)
\\
&=
({\textstyle\frac12}+n-1)
({\textstyle\frac12}+n-2)
\Gamma({\textstyle\frac12}+n-2)
\\
&=
({\textstyle\frac12}+n-1)
({\textstyle\frac12}+n-2)
\dots
({\textstyle\frac12})
\cdot
\Gamma({\textstyle\frac12})
\\
&=
\Gamma({\textstyle\frac12})
\cdot
({\textstyle\frac12})
\dots
({\textstyle\frac12}+n-1)
=
\Gamma({\textstyle\frac12})\cdot ({\textstyle\frac12})_n
=
\sqrt{\pi\mathstrut}\cdot ({\textstyle\frac12})_n.
\end{align*}
Mit dem Resultat von Aufgaben~\ref{404} kann jetzt das Pochhammer-Symbol
durch bekanntere Funktionen dargestellt und somit der
gesuchte $\Gamma$-Funktionswert als
\[
\Gamma({\textstyle\frac12}+n)
=
\frac{(2n)!\cdot \sqrt{\pi\mathstrut}}{n!\cdot 2^{2n}}
\]
geschrieben werden.
\end{loesung}
