Berechnen Sie
\begin{teilaufgaben}
\item $\Gamma(\frac{5}2)$
\item $\displaystyle \frac{\Gamma(\frac{16}3)}{\Gamma(\frac{10}3)}$
\end{teilaufgaben}

\begin{loesung}
\begin{teilaufgaben}
\item
Mit Hilfe der Funktionalgleichung der Gamma-Funktion findet man
\[
\Gamma({\textstyle\frac52})
=
\frac32
\cdot
\Gamma({\textstyle\frac32})
=
\frac32
\cdot
\frac12
\cdot
\Gamma({\textstyle\frac12})
=
\frac{3}{4}\sqrt{\pi}.
\]
\item
Ebenfalls unter Verwendung der Funktionalgleichung der Gamma-Funktion 
findet man
\[
\Gamma({\textstyle\frac{16}3})
=
\frac{13}3
\cdot
\Gamma({\textstyle\frac{13}3})
=
\frac{13}3
\cdot
\frac{10}3
\cdot
\Gamma({\textstyle\frac{10}3})
\quad\Rightarrow\quad
\frac{\Gamma(\frac{16}3)}{\Gamma(\frac{10}3)}
=
\frac{13}3\cdot\frac{10}3
=
\frac{130}{9}
\approx
14.4444.
\qedhere
\]
\end{teilaufgaben}
\end{loesung}
