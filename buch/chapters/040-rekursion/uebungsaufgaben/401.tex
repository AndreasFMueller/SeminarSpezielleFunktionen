Schreiben Sie die Funktion
\[
\arcsin x
=
x
+
\frac{1}{2} \frac{x^3}{5}
+
\frac{1\cdot 3}{2\cdot 4}\frac{x^5}{5}
+
\frac{1\cdot 3\cdot 5}{2\cdot 4\cdot 6}\frac{x^7}{7}
+
\dots
+
\frac{1\cdot 3\cdot 5\cdot (2k-1)}{2\cdot4\cdot 6\cdot (2k)}
\frac{x^{2k+1}}{2k+1}
+
\dots
\]
mit Hilfe der hypergeometrischen Funktion $\mathstrut_2F_1$.

\begin{loesung}
Zunächst betrachten wir die Produkte
\[
p_k
=
\frac{1\cdot 3\cdot \ldots \cdot (2k-1)}{2\cdot 4\cdot\ldots\cdot (2k)}.
\]
Durch Kürzen mit $2^k$ erhalten wir Produkte im Zähler und im Nenner, deren
Faktoren in Einerschritten ansteigen:
\[
p_k
=
\frac{
\frac12\cdot
\bigl(
\frac12+1\bigr)\cdot\ldots\cdot\bigl(\frac12+k-1\bigr)
}{
1\cdot 2\cdot \ldots \cdot k
}
=
\frac{(\frac12)_k}{(1)_k}
=
\frac{(\frac12)_k}{k!}
\]
Damit haben wir den ersten Faktor mit Pochhammer-Symbolen geschrieben.
Den Nenner können wir für den obligatorischen Nenner $k!$ verwenden,
der in einer hypergeometrischen Reihe vorkommt.

Den verbleibenden Teil muss jetzt in der Form $qz^k$ geschrieben werden,
wobei $q$ ein Quotient von Pochhammer-Symbolen sein muss.
Da die Potenzen von $x$ in Zweierschritten ansteigen, müssen wir als
Argument $z=x^2$ verwenden und einen gemeinsamen Faktor $x$ aus der
Funktion ausklammern.

Im Faktor $1/(2k+1)$ nimmt der Nenner in Zweierschritten zu, wir schreiben
ihn daher zunächst als
\[
\frac{1}{2k+1}
=
\frac{1}{2}\cdot \frac{1}{\frac12+k}
=
\frac{1}{2}\cdot\frac{1}{\frac32+k-1}.
\]
Den zweiten Bruch können wir jetzt als Quotienten von Pochhammer-Symbolen
schreiben, nämlich
\begin{align*}
\frac{1}{\frac32+k-1}
&=
\frac{
\frac32
\cdot
\bigl(\frac32+1)
\cdot
\bigl(\frac32+2)
\cdots
\bigl(\frac32+k-2)
\phantom{
\mathstrut
\cdot
\bigl(\frac32+k-1)
}
}{
\frac32
\cdot
\bigl(\frac32+1)
\cdot
\bigl(\frac32+2)
\cdots
\bigl(\frac32+k-2)
\cdot
\bigl(\frac32+k-1)
}
\\
&=
2
\frac{
\frac12
\cdot
\frac32
\cdot
\bigl(\frac32+1)
\cdot
\bigl(\frac32+2)
\cdots
\bigl(\frac32+k-2)
\phantom{
\mathstrut
\cdot
\bigl(\frac32+k-1)
}
}{
\phantom{
\frac12
\cdot
\mathstrut
}
\frac32
\cdot
\bigl(\frac32+1)
\cdot
\bigl(\frac32+2)
\cdots
\bigl(\frac32+k-2)
\cdot
\bigl(\frac32+k-1)
}
\\
&=
2\frac{(\frac12)_k}{(\frac32)_k}.
\end{align*}
Damit wird die Reihe
\[
\arcsin x
=
x
\sum_{k=0}^\infty
\frac{(\frac12)_k}{(1)_k}
\cdot
\frac{(\frac12)_k}{(\frac32)_k}
\cdot
(x^2)^k
=
x
\sum_{k=0}^\infty
\frac{(\frac12)_k(\frac12)_k}{(\frac32)_k}
\cdot
\frac{(x^2)^k}{k!}
=
x\cdot
\mathstrut_2F_1\biggl(
\begin{matrix}
\frac12,\frac12\\ \frac32
\end{matrix}
;x^2
\biggr).
\qedhere
\]
\end{loesung}
