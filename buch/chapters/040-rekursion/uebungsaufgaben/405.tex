Schreiben Sie die Potenzreihe
\begin{align*}
\arctan x
&=
x
-
\frac{x^3}{3}
+
\frac{x^5}{5}
-
\frac{x^7}{7}
+
\dots
\intertext{als}
\arctan x
&=
x\, \biggl(
\frac{1}{2\cdot 0+1}(-x^2)^0
+
\frac{1}{2\cdot 1 + 1}(-x^2)^1
+
\frac{1}{2\cdot 2 + 1}(-x^2)^2
+
\frac{1}{2\cdot 3+1}(-x^2)^3
\biggr)
=
x f(-x^2),
\intertext{mit der Funktion}
f(z)
&=
1
+\frac{1}{3}z
+\frac{1}{5}z^2
+\frac{1}{7}z^3
+\dots
=
\sum_{k=0}^\infty
\frac{1}{2k+1}z^k.
\end{align*}
Schreiben Sie $f(z)$ mit Hilfe der hypergeometrischen Reihe
$\mathstrut_2F_1$.

\begin{hinweis}
Verwenden Sie dazu
$({\textstyle\frac12})_k$ und
$({\textstyle\frac32})_k$.
\end{hinweis}

\begin{loesung}
Gemäss dem Hinweis betrachtet man
\begin{align*}
({\textstyle\frac12})_k
&=
\frac12\cdot\frac32\cdot\frac52\cdot\ldots\cdot\frac{2k-1}{2}
\\
({\textstyle\frac32})_k
&=
\phantom{\frac12\cdot\mathstrut}
\frac32\cdot\frac52\cdot\ldots
\cdot\frac{2k-1}{2}
\cdot\frac{2k+1}{2}.
\end{align*}
Da beide Pochhammer-Symbole jeweils $k$ Faktoren $2$ im Nenner haben,
kürzen sich diese im Quotienten alle weg.
Der Quotient ist daher
\[
\frac{(\frac12)_k}{(\frac32)_k}
=
\frac{1}{2k+1},
\]
also genau der Nenner, den wir für die Potenzreihe von $f(z)$ brauchen.
Somit ist
\[
f(z)
=
\sum_{k=0}^\infty
\frac{(\frac12)_k}{(\frac32)_k}z^k.
\]
Man könnte versucht sein zu schliessen, dass
$f(z)=\mathstrut_1F_1(\frac12;\frac32;z)$ sei, dies ist
aber nicht korrekt, da in der hypergeometrischen Reihe immer 
auch ein Nenner $k!$ vorkommt.
Wir brauchen daher einen zusätzlichen Faktor $(a_2)_k$, der
sich gegen $k!$ wegkürzen lässt, oder
\[
f(z)
=
\sum_{k=0}^\infty
\frac{(\frac12)_k}{(\frac32)_k}z^k
=
\sum_{k=0}^\infty
\frac{(\frac12)_k(a_2)_k}{(\frac32)_k}\frac{z^k}{k!}.
\]
Dies geht natürlich nur, wenn $(a_2)_k=k!$, also $a_2=1$.
Somit ist die gesuchte Funktion
\[
f(z)
=
\sum_{k=0}^\infty
\frac{(\frac12)_k(1)_k}{(\frac32)_k}
\frac{z^k}{k!}
=
\mathstrut_2F_1\biggl(
\begin{matrix}\frac12,1\\\frac32\end{matrix};z
\biggr).
\]
Damit kann man jetzt den Arkustangens schreiben als
\[
\arctan x
=
x\cdot\mathstrut_2F_1\biggl(
\begin{matrix}\frac12,1\\\frac32\end{matrix};-x^2
\biggr).
\qedhere
\]
\end{loesung}


