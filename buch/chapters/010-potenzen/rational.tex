%
% rational.tex
%
% (c) 2022 Prof Dr Andreas Müller, OST Ostschweizer Fachhochschule
%
\section{Rationale Funktionen
\label{buch:polynome:section:rationale-funktionen}}
\rhead{Rationale Funktionen}
Polynome sind sehr einfach auszuwerten und können auf einem
Interval jede stetige Funktion beliebig gut approximieren.
Auf einem unbeschränkten Definitionsbereich wachsen Polynome aber
immer unbeschränkt an.
Der führende Term $a_nx^n$ dominiert das Verhalten eines Polynoms
für $x\to\infty$ wegen
\[
\lim_{x\to\infty} a_nx^n
=
\operatorname{sgn} a_n \cdot\infty
\qquad\text{und}\qquad
\lim_{x\to-\infty} a_nx^n
=
(-1)^n \operatorname{sgn} a_n\cdot \infty.
\]
Insbesondere kann man nicht erwarten, dass sich eine beschränkte
Funktion wie $\sin x$ durch Polynome auf dem ganzen Definitionsbereich
gut approximieren lässt.
Der Unterschied $p(x)-\sin x$ wird für jedes beliebige Polynome $p(x)$
für $x\to\pm\infty$ unbeschränkt anwachsen.

Eine weitere Einschränkung ist, dass die Menge der Polynome bezüglich
der arithmetischen Operationen nicht abgeschlossen ist.
Man kann zwar Polynome addieren und multiplizieren, aber der Quotient
ist nicht notwendigerweise ein Polynom.
Abhilfe schafft nur, wenn man Quotienten von Polynomen zulässt.

\begin{definition}
Eine Funktion $f(x)$ heisst {\em rationale Funktion}, wenn sie Quotient
\index{rationale Funktion}%
zweier Polynome ist, wenn es also Polynome $p(x), q(x)\in K[x]$ gibt mit
\[
f(x) = \frac{p(x)}{q(x)}.
\]
Die Menge der rationalen Funktione mit Koeffizienten in $K$ wird mit
$K(x)$ bezeichnet.
\end{definition}

Polynome sind rationale Funktionen, deren Nennergrad $1$ ist.
Rationale Funktionen können ebenfalls zur Approximation von Funktionen
verwendet werden.
Da sie beschränkt sein können, haben sie das Potential, 
beschränkte Funktionen besser zu approximieren, als dies mit 
Polynomen allein möglich wäre.
Die Theorie der Padé-Approximation, wie sie zum Beispiel im Buch
\index{Pade-Approximation@Padé-Approximation}%
\cite{buch:pade} dargestellt ist, ist zum Beispiel auch in der
Regelungstechnik von Interesse, da sich rationale Funktionen mit
linearen Komponenten schaltungstechnisch realisieren lassen.
Weitere Anwendungen werden in Kapitel~\ref{chapter:transfer}
gezeigt.


