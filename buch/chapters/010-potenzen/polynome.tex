%
% polynome.tex
%
% (c) 2021 Prof Dr Andreas Müller, OST Ostschweizer Fachhochschule
%
\section{Polynome
\label{buch:potenzen:section:polynome}}
\rhead{Polynome}
Die wohl einfachsten Funktionen, die sich mit den arithmetischen
Operationen konstruieren lassen, sind die Polynome.

\begin{definition}
\index{Polynom}%
Ein {\em Polynome} vom Grad $n$ ist die Funktion
\[
p(x) = a_nx^2n + a_{n-1}x^{n-1} + \dots + a_2x^2 + a_1x + a_0,
\]
wobei $a_n\ne 0$ sein muss.
Das Polynom heisst {\em normiert}, wenn $a_n=1$ ist.
\index{normiert}%
Die Menge aller Polynome mit Koeffizienten in der Menge $K$ wird mit
$K[x]$ bezeichnet.
\end{definition}

Die Menge $K[x]$ ist heisst auch der {\em Polynomring}, weil $K[x]$
mit der Addition, Subtraktion und Multiplikation von Polynomen ein
Ring mit $1$ ist.
Im Folgenden werden wir uns auf die Fälle $K=\mathbb{R}$ und $K=\mathbb{C}$
beschränken.

In Abschnitt~\ref{buch:orthogonalitaet:section:orthogonale-funktionen} werden
Familien von Polynomen konstruiert werden, die sich durch eine
Orthogonalitätseigenschaft auszeichnen.
Diese Polynome lassen sich typischerweise auch als Lösungen von
Differentialgleichungen finden.
Ausserdem werden hypergeometrische Funktionen
\[
\mathstrut_pF_q\biggl(\begin{matrix}a_1,\dots,a_p\\b_1,\dots,b_q\end{matrix};z\biggr),
\] die in
Abschnitt~\ref{buch:rekursion:section:hypergeometrische-funktion}
definiert werden, zu Polynomen, wenn mindestens einer der
Parameter $a_k$ negativ ganzzahlig ist.
Polynome sind also bereits eine Vielfältige Quelle von speziellen
Funktionen.

Viele spezielle Funktionen werden aber komplizierter sein und
sich nicht als einfache Polynome ausdrücken lassen.
Genau diese Unmöglichkeit rechtfertigt ja, neue Funktionen
zu definieren.
Es bleibt aber immer noch die Notwendigkeit, effiziente 
Berechnungsverfahren für die speziellen Funktionen zu konstruieren.
Dank des folgenden Satzes kann dies immer mit Polynomen geschehen.

\begin{satz}[Weierstrass]
\label{buch:potenzen:satz:weierstrass}
Eine auf einem kompakten Intervall $[a,b]$ stetige Funktion $f(x)$
lässt sich durch eine Folge $p_n(x)$ von Polynomen gleichmässig
approximieren.
\end{satz}

Der Satz sagt in dieser Form nichts darüber aus, wie die
Approximationspolynome konstruiert werden sollen.
Von Bernstein gibt es konstruktive Beweise dieses Satzes,
welche auch explizit eine Folge von Approximationspolynomen
konstruieren.
In der späteren Entwicklung werden wir für die meisten
speziellen Funktionen Potenzreihen entwickeln, deren Partialsummen
ebenfalls als Approximationen dienen können.
Weitere Möglichkeiten liefern Interpolationsmethoden der
numerischen Mathematik.

\subsection{Faktorisierung und Nullstellen}
% wird später gebraucht um bei der Definition der hypergeometrischen Reihe
% die Zaehler- und Nenner-Polynome als Pochhammer-Symbole zu entwickeln

\subsection{Koeffizienten-Vergleich}
% Wird gebraucht für die Potenzreihen-Methode
% Muss später ausgedehnt werden auf Potenzreihen

\subsection{Polynom-Berechnung}
Die naive Berechnung der Werte eines Polynoms beginnt mit der Berechnung
der Potenzen.
Die Anzahl nötiger Multiplikationen kann minimiert werden, indem man
das Polynom als
\[
a_nx^n
+
a_{n+1}x^{n+1}
+
\dots
+
a_1x
+
a_0
=
((\dots((a_nx+a_{n-1})x+a_{n-2})x+\dots )x+a_1)x+a_0
\]
schreibt.
Beginnend bei der innersten Klammer sind genau $n$ Multiplikationen
und $n+1$ Additionen nötig, im Gegensatz zu $2n$ Multiplikationen
und $n$ Additionen bei der naiven Vorgehensweise.



