%
% potenzreihen.tex
%
% (c) 2021 Prof Dr Andreas Müller, OST Ostschweizer Fachhochschule
%
\section{Potenzreihen
\label{buch:potenzen:section:potenzreihen}}
Nach dem Satz von Weierstrass können
Polynome beliebige stetige Funktionen approximieren.
Die Ableitungen werden dabei meistens nicht gut wiedergegeben.
Die Partialsummen einer Potenzreihe sind ebenfalls Polynome,
die aber nicht nur die Funktion sondern auch alle ihre Ableitungen
gut approximieren.

%
% Definition
%
\subsection{Definition
\label{buch:potenzen:potenzreihen:section:definition}}
Eine Folge von Polynomen, deren Terme niedrigen Grades sich nicht
mehr ändern, bei der also immer nur neue Terme höheren Grades
hinzukommen, heisst eine Potenzreihe.

\begin{definition}
\label{buch:polynome:def:potenzreihe}
\index{Potenzreihe}%
Eine {\em Potenzreihe} an der Stelle $z_0$ ist eine unendliche Reihe
der Form
\[
f(z)
=
\sum_{k=0}^\infty a_k (z-z_0)^k
\]
mit Koeffizienten $a_k\in \mathbb{R}$ oder $a_k\in\mathbb{C}$.
\end{definition}

Die Berechnung einer Potenzreihe ist möglich, wenn die Terme höheren
Grades an Bedeutung verlieren.

\begin{definition}
\index{Partialsumme}%
\index{konvergent, Potenzreihe}%
Eine Potenzreihe heisst {\em konvergent}, die Folge der {\em Partialsummen}
\[
s_n = \sum_{k=0}^n a_k(z-z_0)^k
\]
konvergiert.
Sie heisst absolut konvergent, wenn die Reihe
\[
\sum_{k=0}^\infty |a_k (z-z_0)^k|
\]
konvergiert.
\end{definition}

Die Koeffizienten $a_k$ dürfen also nicht schnell anwachsen
werden, denn normalerweise wird bei Polynomen das Verhalten von den
Termen höheren Grades dominiert.
Die Tschebyscheff-Polynome waren ja so konstruiert worden, dass
es nicht zu unzweckmässig starken Oszillationen im Intervall $(-1,1)$
kommt.

%
% Geometrische Reihe
%
\subsection{Die geometrische Reihe
\label{buch:potenzen:potenzreihen:section:geometrische}}
Die wohl einfachste Potenzreihe ist die Reihe mit Koeffizienten
$a_k=a$ für alle $k$, also
\[
f(z)
=
\sum_{k=0}^\infty az^k
=
a+az+az^2+az^3+az^4+\dots
\]
Sie ist charakterisiert durch die Eigenschaft, dass aufeinanderfolgende
Reiheglieder den konstanten Quotienten $z$ haben.
Diese Idee wird
in Abschnitt~\ref{buch:rekursion:section:hypergeometrische-funktion}
auf Quotienten verallgemeinert, die rationale Funktionen sind.
Sie heissen hypergeometrische Funktionen.
\index{hypergeometrische Funktion}%

Sie ist konvergent für $|z|<1$ und divergent für $|z|\ge 1$.
Sie heisst die {\em geometrische Reihe}.
Sie wird gerne als ``Vergleichsreihe'' eingesetzt um die
Konvergenz oder Divergenz anderer Reihen nachzuweisen.

Die geometrische Reihe lässt sich direkt summieren.
Dazu betrachtet man die Differenz der Partialsumme $s_n$ und $zs_n$:
\[
\renewcommand{\arraycolsep}{2pt}
\begin{array}{rcrcrcrcrcrcrl}
 s_n    &=& a&+&az&+&az^2&+&\dots&+&az^n& &        &\multirow{2}{*}{\hspace{3pt}$\biggl\}\mathstrut-\mathstrut$}\\
z\phantom{)}s_n    &=&  & &az&+&az^2&+&\dots&+&az^n&+&az^{n+1\phantom{.}}&\\
\hline
(1-z)s_n&=& a& &  & &    & &     & &    &-&az^{n+1}.&
\end{array}
\]
Durch Auflösen nach $s_n$ erhält man die Summenformel
\[
s_n = a\frac{1-z^{n+1}}{1-z}.
\]
Für $|z|<1$ geht $z^n\to 0$ für $n\to\infty$, die Partialsummen
konvergieren und wir erhalten das Resultat des folgenden Satzes.

\begin{satz}
\index{Satz!geometrische Reihe}%
\label{buch:polynome:satz:geometrischereihe}
Die geometrische Reihe $a+az+az^2+\dots$ konvergiert für $|z|<1$ und hat
die Summe
\[
\sum_{k=0}^\infty az^k = \frac{a}{1-z}.
\]
Für $|z|\ge 1$ divergiert die geometrische Reihe.
\end{satz}

%
% Konvergenzkriterien
%
\subsection{Konvergenzkriterien
\label{buch:potenzen:potenzreihen:section:konvergenzkriterien}}
Die Konvergenz von Reihen ist oft durch Vergleich mit anderen, bereits
als konvergent erkannten Reihen nachweisbar.
Dies ist der Inhalt des folgenden, wohlbekannten Majorantenkriteriums.

\begin{satz}[Majorantenkriterium]
\index{Satz!Majorantenkriterium}%
\label{buch:polynome:satz:majorantenkriterium}
\index{Majorantenkriterium}
Seien $a_k$ und $b_k$ die Glieder zweier unendlicher Reihen.
Es sei zudem $b_k\ge 0$ für alle $k$ und die Reihe
$\sum_{k=0}^\infty b_k$ sei konvergent.
Wenn $|a_k|\ge b_k$ ist für fast alle $k$, dann ist die Reihe
\(
\sum_{k=}^\infty a_k
\)
absolut konvergent.
\end{satz}

\subsubsection{Quotienten- und Wurzelkriterium}
Der Satz~\ref{buch:polynome:satz:geometrischereihe} ermöglicht,
Potenzreihen mit der geometrischen Reihe zu vergleichen und
liefert damit einfach anzuwende Kriterien für die Konvergenz.

\begin{satz}[Quotientenkriterium]
\index{Satz!Quotientenkriterium}%
\label{buch:polynome:satz:quotientenkriterium}
\index{Quotientenkriterium}%
Eine Reihe 
\(
\sum_{k=0}^\infty  a_k
\)
ist absolut konvergent, wenn es eine Zahl $q<1$ gibt derart, dass
\begin{equation}
\biggl|\frac{a_{k+1}}{a_k}\biggr|\le q.
\label{buch:polynome:eqn:quotienten-kriterium}
\end{equation}
Die Reihe ist divergent, wenn für fast alle $k$
\[
\biggl|\frac{a_{k+1}}{a_k}\biggr| \ge 1
\]
gilt.
\end{satz}

\begin{proof}[Beweis]
Wenn \eqref{buch:polynome:eqn:quotienten-kriterium} erfüllt ist, dann
gilt
\[
|a_k| \le |a_0| q^k
\]
und damit ist die Reihe majorisiert durch die geometrische Reihe 
\[
\sum_{k=0}^\infty
|a_0|q^k,
\]
die unter der gegebenen Voraussetzung konvergiert.
\end{proof}

\begin{satz}[Wurzelkriterium]
\index{Satz!Wurzelkriterium}%
\label{buch:polynome:satz:wurzelkriterium}
\index{Wurzelkriterium}
Falls
\begin{equation}
\limsup_{n\to\infty} \root{n}\of{|a_n|} = C < 1
\label{buch:polynome:eqn:wurzel-kriterium}
\end{equation}
ist die Reihe
\(
\sum_{k=0}^\infty a_k
\)
absolut konvergent.
\end{satz}

\begin{proof}[Beweis]
Falls $\root{k}\of{|a_k|}\le q<1$, dann gilt
$|a_k|<q^k$ für alle $k$.
Somit wird die Reihe majorisiert durch die geometrische Reihe
mit Quotient $q$ und ist damit konvergent.

Das Kriterium \eqref{buch:polynome:eqn:wurzel-kriterium} bedeutet,
dass es zu einem gegebenen $\varepsilon > 0$ ein $N$ gibt derart,
dass $\root{n}\of{|a_k|} < C+\varepsilon$ für $n>N$.  
Wählt man $\varepsilon = (1-C)/2$ wird $q=C+\varepsilon=(1+C)/2<1$,
das Reststück der Reihe ab Index $N$ ist daher wieder majorisiert
durch eine konvergente geometrische Reihe.
\end{proof}

%
% Konvergenzradius
%
\subsubsection{Konvergenzradius}
Das Quotienten- und das Wurzel-Kriterium ist auf beliebige Reihen
anwendbar, es berücksichtigt nicht, dass in einer Potenzreihe
die Faktoren $(z-z_0)^k$ für kleine $|z-z_0|$ das Kleiner werden
der Reihenglieder und damit die Konvergenz begünstigen.
Diese Eigenschaft wird vom Konvergenzradius eingefangen, der wie
folgt definiert ist.

\begin{definition}
\label{buch:polynome:definition:konvergenzradius}
\index{Konvergenzradius}%
Der {\em Konvergenzradius} einer Potenzreihe $\sum_{k=0}^\infty a_k(z-z_0)^k$
um den Punkt $z_0$ ist 
\[
\varrho = \sup \biggl\{ |z-z_0|\;\bigg|\;
\text{$\displaystyle\sum_{k=0}^\infty a_k(z-z_0)^k$ konvergiert}
\biggr\}.
\]
\end{definition}

\begin{satz}
\index{Satz!Konvergenzradius}%
\label{buch:polynome:satz:konvergenzradius}
Der Konvergenzradius $\varrho$ einer Potenzreihe
$\sum_{k=0}^\infty a_k(z-z_0)^k$ ist
\begin{equation}
\frac{1}{\varrho}
=
\limsup_{k\to\infty} \root{n}\of{|a_k|}.
\label{buch:polynome:eqn:konvergenzradius}
\end{equation}
\end{satz}

\begin{proof}[Beweis]
Wir wenden das Wurzelkriterium auf ein $z$ mit $|z-z_0|<\varrho$ an.
Es gilt
\[
\root{k}\of{|a_k(z-z_0)^k|}
=
|z-z_0|\root{k}\of{|a_k|}
\qquad
\Rightarrow
\qquad
\limsup_{k\to\infty}\root{k}\of{|a_k(z-z_0)^k|}
=
|z-z_0| \underbrace{\limsup_{k\to\infty}\root{k}\of{|a_k|}}_{\displaystyle=\frac{1}{\varrho}}
<
1.
\]
Nach dem Wurzelktrierium folgt daher, dass die Reihe
$\sum_{k=0}^\infty a_k(z-z_0)^k$ absolut konvergent ist.
\end{proof}

\begin{beispiel}
Der Konvergenzradius der geometrischen Reihe $1+z+z^2+\dots$ ist
\[
\frac{1}{\varrho}
=
\limsup_{n\to\infty} \root{n}\of{|a_n|}
=
\limsup_{n\to\infty} \root{n}\of{1}
=
1.
\]
Dies deckt sich mit der bereits bekannten Tatsache, dass die 
geometrische Reihe für $|z|<1$ konvergiert.
Man beachte auch, dass der Konvergenzradius genau die Entfernung
vom Entwicklungspunkt $z_0=0$ und dem Pol der Summe $1/(1-z)$ bei
$z=1$ ist.
Auf diese allgemeingültige Eigenschaft wird in Abschnitt
\ref{buch:funktionentheorie:subsection:konvergenzradius}
eingegangen.
\end{beispiel}

Auch das Quotientenkriterium kann zur Berechnung des Konvergenzradius
herangezogen werden.
Falls $a_k\ne 0$ ab einem gewissen Index $k$ ist und der Grenzwert
von $a_{k}/a_{k+1}$ für $k\to\infty$ existiert, dann ist der
Grenzwert der Konvergenzradius.

%
% Tayler-Reihe
%
\subsection{Die Taylor-Reihe
\label{buch:polynome:subsection:taylor-reihe}}
Nicht nur der Funktionswert eines Polynoms, sondern auch alle
seine Ableitungen sind sehr einfach zu berechnen.
Dies macht Potenzreihen besonders nützlich im Zusammenhang
mit Lösungen von Differentialgleichungen, wie in Abschnitt
\ref{buch:differentialgleichungen:section:potenzreihenmethode}
untersucht werden wird.
In diesem Abschnitt wird die Taylor-Reihe motiviert, die sich
aus den Ableitungen einer differenzierbaren Funktion konstruieren
lässt.

\subsubsection{Ableitung einer Potenzreihe}
Eine Potenzreihe
$f(z)=\sum_{k=0}^n a_kz^k$
kann gliedweise abgeleitet werden.
Die $k$-te Ableitung ist
\[
f^{(k)}(z)
=
\sum_{n=0}^\infty n(n-1)\cdots(n-k+1) a_nz^{n-k},
\]
aufeinanderfolgende Terme dieser Reihe haben den Quotienten
\[
\frac{
(n+1)n(n-1)\cdots(n-k+2)\phantom{(n-k+1)}
}{
\phantom{(n+1)}n(n-1)\cdots(n-k+2)(n-k+1)
}
\cdot
\biggl|
\frac{a_{n+1}z^{n+1}}{a_nz^n}
\biggr|
=
\frac{n+1}{n-k+1}
\cdot
\biggl|
\frac{a_{n+1}}{a_n}
\biggr|
\cdot|z|.
\]
Da der Quotient $(n+1)/(n-k+1)\to 1$ für $n\to\infty$, ist das
Quotientenkriterium für die Ableitung erfüllt, wenn $|z|$ klein genug ist
und das Kriterium für die Potenzreihe $f(z)$ erfüllt ist.

\subsubsection{Konvergenzradius der abgeleiteten Reihe}
Der Konvergenzradius $\varrho^{(k)}$ der $k$-fach abgeleiteten Reihe ist
\begin{align*}
\root{n}\of{\mathstrut(n+k)(n+k-1)\cdots(n+1) |a_{n+k}|}
&=
\root{n}\of{\mathstrut(n+k)(n+k-1)\cdots(n+1)}
\cdot
\root{n}\of{|a_{n+k}|}
\end{align*}
mit Limes superior
\begin{align*}
\frac{1}{\varrho^{(k)}}
&=
\limsup_{n\to\infty}
\root{n}\of{\mathstrut(n+k)(n+k-1)\cdots(n+1) |a_{n+k}|}
\\
&=
\underbrace{
\lim_{n\to\infty}
\root{n}\of{\mathstrut(n+k)(n+k-1)\cdots(n+1)}
}_{\displaystyle\to 1}
\cdot
\underbrace{
\limsup_{n\to\infty}
\root{n}\of{\mathstrut|a_{n+k}|}
}_{\displaystyle\to\frac{1}{\varrho}}
=
\frac{1}{\varrho},
\end{align*}
die abgeleitet Reihe hat also den gleichen Konvergenzradius wie die
Reihe für $f(z)$.

\subsubsection{Berührung $k$-ten Grades}
Man sagt, die Graphen zweier Funktionen $f(z)$ und $g(z)$ berühren
sich im Punkt $z=z_0$ vom Grade $k$, wenn Funktionswerte und
Ableitungen bis zum Grad $k$ beider Funktionen in $z_0$ übereinstimmen.
Die Ableitungen der Potenzfunktion $(z-z_0)^n$ sind nacheinander
\begin{align*}
\frac{d}{dz}(z-z_0)^n&= n(z-z_0)^{n-1},
\\
\frac{d^2}{dz^2}(z-z_0)^n&=n(n-1)(z-z_0)^{n-2},
\\
\frac{d^3}{dz^3}(z-z_O)^n&=n(n-1)(n-2)(z-z_0)^{n-3},
\\
&\vdots
\\
\frac{d^k}{dz^k}(z-z_0)^n&=n(n-1)(n-2)\dots(n-k+1)(z-z_0)^{n-k},
\\
&\vdots
\\
\frac{d^n}{dz^n}(z-z_0)^n&=n!,
\\
\frac{d^l}{dz^l}(z-z_0)^n&=0\qquad\forall l>n.
\end{align*}
An der Stelle $z=0$ ist nur genau die $n$-te Ableitung von $0$ verschieden
und hat den Wert $n!$.
Zwei Funktionen $f(z)$ und $g(z)$, die als Potenzreihen im Punkt $z_0$
geschrieben werden, berühren sich also genau dann vom Grad $k$, wenn
die Funktionswerte und Ableitungen übereinstimmen, d.~h.
\begin{equation}
\left.
\begin{aligned}
f(z)&=\sum_{l=0}^\infty a_l(z-z_0)^l \\
g(z)&=\sum_{l=0}^\infty b_l(z-z_0)^l 
\end{aligned}
\right\}
\quad\Rightarrow\quad
f^{(l)}(z_0) = g^{(l)}(z_0)
\quad\Rightarrow\quad
l!a_l = l!b_l
\quad\Rightarrow\quad
a_l=b_l
\end{equation}
für $l\le k$.
Das Taylor-Polynom ist ein Polynom, welches die gegeben Funktion
von hohem Grad berührt.

\begin{definition}
\label{buch:polynome:definition:taylor-reihe}
\index{Taylor-Polynom}%
Sie $f(z)$ eine beliebig oft stetig differenzierbare Funktion.
Das {\em Taylor-Polynome} vom Grad $n$ von $f(z)$ an der Stelle
$z_0$ ist die Summe
\begin{equation}
\mathscr{T}_{z_0}^nf (z)
=
\sum_{k=0}^n
\frac{f^{(k)}(z_0)}{k!} (z-z_0)^k
\label{buch:polynome:eqn:taylor-polynom}
\end{equation}
\index{Taylor-Reihe}%
Die {\em Taylor-Reihe} der Funktion $f(z)$ ist die Reihe
\begin{equation}
\mathscr{T}_{z_0}f (z)
=
\sum_{k=0}^\infty
\frac{f^{(k)}(z_0)}{k!} (z-z_0)^k.
\label{buch:polynome:eqn:taylor-reihe}
\end{equation}
\end{definition}

%
% Analytische Funktionen
%
\subsubsection{Analytische Funktionen}
Das Taylor-Polynom $\mathscr{T}_{z_0}^nf(z)$ hat an der Stelle $z_0$
die gleichen Funktionswerte und Ableitungen wie die Funktion $f(z)$,
dies bedeutet aber nicht, dass die Taylorreihe gegen die Funktion 
konvergiert.
Das Beispiel auf
Seite~\pageref{buch:funktionentheorie:beispiel:nichtanalytisch}
zeigt, dass dies nicht immer zutrifft.
Von besonderem Interesse sind die Funktionen, die sich durch eine
konvergente Taylor-Reihe ausdrücken lassen.

\begin{definition}
\label{buch:polynome:def:analytisch}
\index{analytisch}%
Eine Funktion heisst analytisch, wenn sie sich durch eine
konvergente Potenzreihe darstellen lässt.
\end{definition}

Die Klasse der analytischen Funktionen umfasst also nicht alle 
differenzierbaren Funktionen.
Da aber Potenzreihen Gleidweise differenziert und integriert werden
dürfen, können die meisten Konstruktionen der Analysis bis hin zur
Lösung partieller Differentialgleichungen innerhalb der analytischen
Funktionen durchgeführt werden.
Es ist daher nicht überraschend, dass alle in diesem Buch studierten
speziellen Funktionen analytisch sind.


