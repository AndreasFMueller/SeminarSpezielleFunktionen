%
% 1-polynome.tex
%
% (c) 2021 Prof Dr Andreas Müller, OST Ostschweizer Fachhochschule
%
\section{Polynome
\label{buch:potenzen:section:polynome}}
\rhead{Polynome}
Die wohl einfachsten Funktionen, die sich mit den arithmetischen
Operationen konstruieren lassen, sind die Polynome.

\begin{definition}
\index{Polynom}%
Ein {\em Polynome} vom Grad $n$ ist die Funktion
\[
p(x) = a_nx^n + a_{n-1}x^{n-1} + \dots + a_2x^2 + a_1x + a_0,
\]
wobei $a_n\ne 0$ sein muss.
Das Polynom heisst {\em normiert}, wenn $a_n=1$ ist.
\index{normiert}%
\index{Grad eines Polynoms}%
\index{Polynom!Grad}%
Die Menge aller Polynome mit Koeffizienten in der Menge $K$ wird mit
$K[x]$ bezeichnet.
\end{definition}

Die Menge $K[x]$ ist heisst auch der {\em Polynomring}, weil $K[x]$
\index{Polynomring}%
mit der Addition, Subtraktion und Multiplikation von Polynomen eine
algebraische Struktur bildet, die man einen Ring mit $1$ nennt.
\index{Ring}%
Im Folgenden werden wir uns auf die Fälle $K=\mathbb{Q}$, $K=\mathbb{R}$
und $K=\mathbb{C}$ beschränken.

Für den Grad eines Polynoms gelten die bekannten Rechenregeln
\begin{align*}
\deg (a(x) + b(x)) &\le \operatorname{max}(\deg a(x), \deg b(x))
\\
\deg (a(x)\cdot b(x)) &=\deg a(x) + \deg b(x)
\end{align*}
für beliebige Polynome $a(x),b(x)\in K[x]$.

In Abschnitt~\ref{buch:orthogonalitaet:section:orthogonale-funktionen} werden
Familien von Polynomen konstruiert werden, die sich durch eine
Orthogonalitätseigenschaft auszeichnen.
Diese Polynome lassen sich typischerweise auch als Lösungen von
Differentialgleichungen finden.
Ausserdem werden hypergeometrische Funktionen
\[
\mathstrut_pF_q\biggl(
\begin{matrix}a_1,\dots,a_p\\b_1,\dots,b_q\end{matrix};z
\biggr),
\] die in
Abschnitt~\ref{buch:rekursion:section:hypergeometrische-funktion}
definiert werden, zu Polynomen, wenn mindestens einer der
Parameter $a_k$ negativ ganzzahlig ist.
Polynome sind also bereits eine vielfältige Quelle von speziellen
Funktionen.

Viele spezielle Funktionen werden aber komplizierter sein und
sich nicht als einfache Polynome ausdrücken lassen.
Genau diese Unmöglichkeit rechtfertigt ja, neue Funktionen
zu definieren.
Es bleibt aber immer noch die Notwendigkeit, effiziente 
Berechnungsverfahren für die speziellen Funktionen zu konstruieren.
Dank des folgenden Satzes kann dies immer mit Polynomen geschehen.

\begin{satz}[Weierstrass]
\index{Satz!Weierstrass}%
\index{Weierstrasse, Karl}%
\label{buch:potenzen:satz:weierstrass}
\index{Weierstrass, Satz von}%
Eine auf einem kompakten Intervall $[a,b]$ stetige Funktion $f(x)$
lässt sich durch eine Folge $p_n(x)$ von Polynomen gleichmässig
approximieren.
\end{satz}

Der Satz sagt in dieser Form nichts darüber aus, wie die
Approximationspolynome konstruiert werden sollen.
\index{Approximationspolynom}%
Von Bernstein gibt es konstruktive Beweise dieses Satzes,
\index{Bernstein-Polynom}%
welche auch explizit eine Folge von Approximationspolynomen
konstruieren.
In der späteren Entwicklung werden wir für die meisten
speziellen Funktionen Potenzreihen entwickeln, deren Partialsummen
ebenfalls als Approximationen dienen können.
Weitere Möglichkeiten liefern Interpolationsmethoden der
numerischen Mathematik.

Diese Betrachtungsweise von Polynomen als Funktionen trägt
aber den zusätzlichen algebraischen Eigenschaften des Polynomringes
nicht ausreichend Rechnung.
Zum Beispiel bedeutet Gleichheit von zwei reellen Funktion $f(x)$ und
$g(x)$, dass man $f(x)=g(x)$ für alle $x\in\mathbb{R}$ nachprüfen
muss.
Für Polynome reicht es jedoch, die Funktionswerte in nur wenigen
Punkten zu vergleichen.
Dies äussert sich zum Beispiel auch im Prinzip des
Koeffizientenvergleichs von
Satz~\ref{buch:polynome:satz:koeffizientenvergleich}.
Im Gegensatz zu beliebigen Funktionen kann man daher Aussagen
über Polynomen immer mit endlich Algorithmen entscheiden.
Die nächsten Abschnitte sollen diese algebraischen Eigenschaften
zusammenfassen.

%
% Polynomdivision, Teilbarkeit und ggT
%
\subsection{Polynomdivision, Teilbarkeit und grösster gemeinsamer Teiler}
Der schriftliche Divisionsalgorithmus für Zahlen funktioniert 
auch für die Division von Polynomen.
\index{Polynome!Divisionsalgorithmus}%
Zu zwei beliebigen Polynomen $p(x)$ und $q(x)$ lassen sich also
immer zwei Polynome $a(x)$ und $r(x)$ finden derart, dass
$p(x) = a(x) q(x) + r(x)$.
Das Polynom $a(x)$ heisst der {\em Quotient}, $r(x)$ der {\em Rest}
der Division.
Das Polynom $p(x)$ heisst {\em teilbar} durch $q(x)$, geschrieben
\index{teilbar}%
\index{Polynome!teilbar}%
$q(x)\mid p(x)$, wenn $r(x)=0$ ist.

%
% Grösster gemeinsamer Teiler
%
\subsubsection{Grösster gemeinsamer Teiler}
Mit dem Begriff der Teilbarkeit geht auch die Idee des grössten
gemeinsamen Teilers einher.
Ein gemeinsamer Teiler zweier Polynome $a(x)$ und $b(x)$ 
\index{gemeinsamer Teiler}%
ist ein Polynom $g(x)$, welches beide Polynome teilt, also
$g(x)\mid a(x)$ und $g(x)\mid b(x)$.
\index{grösster gemeinsamer Teiler}%
\index{Polynome!grösster gemeinsamer Teiler}%
Ein Polynom $g(x)$ heisst {\em grösster gemeinsamer Teiler} von $a(x)$
und $b(x)$, wenn jeder andere gemeinsame Teiler $f(x)$ von $a(x)$
und $b(x)$ auch ein Teiler von $g(x)$ ist.
Man beachte, dass die skalaren Vielfachen eines grössten gemeinsamen
Teilers ebenfalls grösste gemeinsame Teiler sind, der grösste gemeinsame
Teiler ist also nicht eindeutig bestimmt.

%
% Der euklidische Algorithmus
%
\subsubsection{Der euklidische Algorithmus}
\index{Algorithmus!euklidisch}%
\index{euklidischer Algorithmus}%
Zur Berechnung eines grössten gemeinsamen Teilers steht wie bei den
ganzen Zahlen der euklidische Algorithmus zur Verfügung.
Dazu bildet man die Folgen von Polynomen
\[
\begin{aligned}
a_0(x)&=a(x) & b_0(x) &= b(x)
&
&\Rightarrow&
a_0(x)&=b_0(x) q_0(x) + r_0(x) &&
\\
a_1(x)&=b_0(x) & b_1(x) &= r_0(x)
&
&\Rightarrow&
a_1(x)&=b_1(x) q_1(x) + r_1(x) &&
\\
a_2(x)&=b_1(x) & b_2(x) &= r_1(x)
&
&\Rightarrow&
a_2(x)&=b_2(x) q_2(x) + r_2(x) &&
\\
&&&&&\hspace*{2mm}\vdots&&
\\
a_{m-1}(x)&=b_{m-2}(x) & b_{m-1}(x) &= r_{m-2}(x) 
&
&\Rightarrow&
a_{m-1}(x)&=b_{m-1}(x)q_{m-1}(x) + r_{m-1}(x) &\text{mit }r_{m-1}(x)&\ne 0
\\
a_m(x)&=b_{m-1}(x) & b_m(x)&=r_{m-1}(x)
&
&\Rightarrow&
a_m(x)&=b_m(x)q_m(x).&&
\end{aligned}
\]
Der Index $m$ ist der Index, bei dem zum ersten Mal $r_m(x)=0$ ist.
Dann ist $g(x)=r_{m-1}(x)$ ein grösster gemeinsamer Teiler.

%
% Der erweiterte euklidische Algorithmus
%
\subsubsection{Der erweiterte euklidische Algorithmus}
\index{Polynome!erweiterter euklidischer Algorithmus}%
\index{erweiterter euklidischer Algorithmus}%
\index{euklidischer Algorithmus!erweitert}%
Die Konstruktion der Folgen $a_n(x)$ und $b_n(x)$ kann in Matrixform
kompakter geschrieben werden als
\[
\begin{pmatrix}
a_k(x)\\
b_k(x)
\end{pmatrix}
=
\begin{pmatrix}
b_{k-1}(x)\\
r_{k-1}(x)
\end{pmatrix}
=
\begin{pmatrix}
0 & 1\\
1 & -q_{k-1}(x)
\end{pmatrix}
\begin{pmatrix}
a_{k-1}(x)\\
b_{k-1}(x)
\end{pmatrix}.
\]
Kürzen wir die $2\times 2$-Matrix als
\[
Q_k(x) = \begin{pmatrix} 0&1\\1&-q_k(x)\end{pmatrix}
\]
ab, dann ergibt das Produkt der Matrizen $Q_0(x)$ bis $Q_{m}(x)$
\[
\begin{pmatrix}
g(x)\\
0
\end{pmatrix}
=
\begin{pmatrix}
r_{m-1}(x)\\
r_{m}(x)
\end{pmatrix}
=
Q_{m}(x)
Q_{m-1}(x)
\cdots
Q_1(x)
Q_0(x)
\begin{pmatrix}
a(x)\\
b(x)
\end{pmatrix}.
\]
Zur Berechnung des Produktes der Matrizen $Q_k(x)$ kann man rekursiv
vorgehen mit der Rekursionsformel
\[
S_{k}(x) = Q_{k}(x) S_{k-1}(x)
\qquad\text{mit}\qquad
S_{-1}(x)
=
\begin{pmatrix} 1 & 0 \\ 0 & 1 \end{pmatrix}.
\]
Ausgeschrieben bedeutet dies Matrixrekursionsformel
\[
S_{k-1}(x)
=
\begin{pmatrix} 
c_{k-1} & d_{k-1} \\
c_k     & d_k
\end{pmatrix}
\qquad\Rightarrow\qquad
Q_{k}(x) S_{k-1}(x)
=
\begin{pmatrix}
0&1\\1&-q_k(x)
\end{pmatrix}
\begin{pmatrix} 
c_{k-1} & d_{k-1} \\
c_k     & d_k
\end{pmatrix}
=
\begin{pmatrix}
c_k&d_k\\
c_{k+1}&d_{k+1}
\end{pmatrix}.
\]
Daraus lässt sich für die Matrixelemente die Rekursionsformel
\[
\begin{aligned}
c_{k+1} &= c_{k-1} - q_k(x) c_k(x) \\
d_{k+1} &= d_{k-1} - q_k(x) d_k(x)
\end{aligned}
\quad
\bigg\}
\qquad
\text{mit Startwerten}
\qquad
\bigg\{
\begin{aligned}
\quad
c_{-1} &= 1, & c_0 &= 0 \\
d_{-1} &= 0, & d_0 &= 1.
\end{aligned}
\]
Wendet man die Matrix $S_m(x)$ auf den Vektor mit den Komponenten
$a(x)$ und $b(x)$, erhält man die Beziehungen
\[
g(x) = c_{k-1}(x) a(x) + d_{k-1}(x) b(x)
\qquad\text{und}\qquad
0 = c_k(x) a(x) + d_k(x) b(x).
\]
Dieser Algorithmus heisst der erweiterte euklidische Algorithmus.
Wir fassen die Resultate zusammen im folgenden Satz.

\begin{satz}
Zu zwei Polynomen $a(x),b(x) \in K[x]$ gibt es Polynome
$g(x),c(x),d(x)\in K[x]$
derart, dass $g(x)$ ein grösster gemeinsamer Teiler von $a(x)$ und $b(x)$
ist, und ausserdem
\[
g(x) = c(x)a(x)+d(x)b(x)
\]
gilt.
\end{satz}

%
% Faktorisierung und Nullstellen
%
\subsection{Faktorisierung und Nullstellen
\label{buch:polynome:subsection:faktorisierung-und-nullstellen}}
% wird später gebraucht um bei der Definition der hypergeometrischen Reihe
% die Zaehler- und Nenner-Polynome als Pochhammer-Symbole zu entwickeln
Ist $\alpha$ eine Nullstelle des Polynoms $a(x)$, also $a(\alpha)=0$.
Der Divisionsalgorithmus mit für die Polynome $a(x)$ und $b(x)=x-\alpha$
liefert zwei Polynome $q(x)$ für den Quotienten und $r(x)$ für den Rest
mit den Eigenschaften
\[
a(x)
=
q(x) b(x)
+r(x)
=
q(x)(x-\alpha)+r(x)
\qquad\text{mit}\qquad
\deg r < \deg b(x)=1.
\]
Der Rest $r(x)$ ist somit eine Konstante. 
Setzt man $x=\alpha$ ein, folgt
\[
0
=
a(\alpha)
=
q(\alpha)(\alpha-\alpha)+r(\alpha)
=
r(\alpha),
\]
der Rest $r(x)$ muss also verschwinden.
Für eine Nullstelle $\alpha$ von $a(x)$ ist $a(x)$ durch $(x-\alpha)$
teilbar.
Daraus folgt auch, dass ein Polynom $a(x)$ vom Grad $n=\deg a(x)$ höchstens
$n$ verschiedene Nullstellen haben kann.

Sind $\alpha_1,\dots,\alpha_k$ alle Nullstellen von $a(x)$, dann lässt
sich $a(x)$ zerlegen in Faktoren
\[
a(x)
=
(x-\alpha_1)^{m_1}
(x-\alpha_2)^{m_2}
\cdots
(x-\alpha_k)^{m_k}
b(x).
\]
Das Polynom $b(x)\in K[x]$ hat keine Nullstellen in $K$.

Wenn zwei Polynome $a(x)$ und $b(x)$ eine gemeinsame Nullstelle $\alpha$
haben, dann ist $(x-\alpha)$ ein Teiler beider Polynome und somit auch
ein Teiler eines grössten gemeinsamer Teiler.
Insbesondere sind die Nullstellen des grössten gemeinsamen Teilers
gemeinsame Nullstellen von $a(x)$ und $b(x)$.

%
% Koeffizienten-Vergleich
%
\subsection{Koeffizienten-Vergleich}
% Wird gebraucht für die Potenzreihen-Methode
% Muss später ausgedehnt werden auf Potenzreihen
Wenn zwei Polynome $a(x)$ und $b(x)$ vom Grad $\le n$ die gleichen
Koeffizienten haben, dann sind sie selbstverständlich gleich.
Weniger klar ist, ob zwei Polynome, die die gleichen Werte für beliebige
$x$ haben, auch die gleichen Koeffizienten haben.
Wir nehmen also an, dass $a(x)=b(x)$ gilt für jedes $x\in K$ und
wollen daraus ableiten, dass die Koeffizienten übereinstimmen müssen.
Seien $x_1,\dots,x_n$ verschiedene Elemente in $K$, dann
hat das Polynom $p(x)=a(x)-b(x)$, welches Grad $\le n$ hat,
die $n$ Nullstellen $x_k$ für $k=1,\dots,n$.
$p(x)$ ist also durch alle Polynome $x-x_k$ teilbar.
Weil $\deg p\le n$ ist, muss 
\[
0
=
a(x)-b(x)
=
p(x)
=
p_n
(x-x_1)(x-x_2)\cdots (x-x_n)
\]
sein.
Ist $y\in K$ verschieden von den Nullstellen $x_i$, dann ist 
in $y-x_i\ne 0$ für alle $i$.
Für das Produkt gilt dann
\[
0
=
p(y) 
=
p_n
(\underbrace{x-x_1}_{\displaystyle \ne 0})
\cdots
(\underbrace{x-x_n}_{\displaystyle \ne 0}),
\]
so dass $p_n=0$ sein muss, was schliesslich dazu führt, dass alle
Koeffizienten von $a(x)-b(x)$ verschwinden.
Daraus folgt das Prinzip des Koeffizientenvergleichs:
\index{Koeffizientenvergleich}%
\index{Polynome!Koeffizientenvergleich}%

\begin{satz}[Koeffizientenvergleich]
\index{Satz!Koeffizientenvergleich}%
\label{buch:polynome:satz:koeffizientenvergleich}
Zwei Polynome $a(x)$ und $b(x)$ stimmen genau dann überein, wenn
sie die gleichen Koeffizienten haben.
\end{satz}

Man beachte, dass dieses Prinzip nur funktioniert, wenn es genügend
viele verschiedene Elemente in $K$ gibt.
Für die endlichen Körper $\mathbb{F}_p$ gilt dies nicht, denn es gilt
\[
a(x)
=
x^p-x\equiv 0\mod p
\]
für jede Zahl $x\in\mathbb{F}_p$, das Polynom $a(x)$ mit Grad $p$
hat also genau $p$ Nullstellen, es gibt aber keine weitere Nullstelle,
mit der man wie oben schliessen könnte, dass $a(x)$ das Nullpolynom ist.

%
% Berechnung von Polynom-Werten
%
\subsection{Berechnung von Polynom-Werten}
Die naive Berechnung der Werte eines Polynoms $p(x)$ vom Grad $n$
beginnt mit der Berechnung der Potenzen von $x$.
Da alle Potenzen benötigt werden, wird man dazu $n-1$ Multiplikationen
benötigen.
Die Potenzen müssen anschliessend mit den Koeffizienten multipliziert
werden, dazu sind weitere $n$ Multiplikationen nötig.
Der Wert des Polynoms kann also erhalten werden mit $2n-1$ Multiplikationen
und $n$ Additionen.

Die Anzahl nötiger Multiplikationen kann mit dem folgenden Vorgehen
reduziert werden, welches auch als das {\em Horner-Schema} bekannt ist.
\index{Horner-Schema}%
\index{Polynome!Horner-Schema}%
Statt erst am Schluss alle Terme zu addieren, addiert man so früh
wie möglich.
Zum Beispiel multipliziert man $(a_nx+a_{n-1})$ mit $x$, was auf
die Multiplikationen beider Terme mit $x$ hinausläuft.
Mit dieser Idee kann man das Polynom als
\[
a_nx^n
+
a_{n+1}x^{n+1}
+
\dots
+
a_1x
+
a_0
=
((\dots((a_nx+a_{n-1})x+a_{n-2})x+\dots )x+a_1)x+a_0
\]
schreiben.
Beginnend bei der innersten Klammer sind genau $n$ Multiplikationen
und $n$ Additionen nötig, man spart mit diesem Vorgehen also
$n-1$ Multiplikationen. 



