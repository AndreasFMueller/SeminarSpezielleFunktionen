%
% holomorph.tex
%
% (c) 2021 Prof Dr Andreas Müller, OST Ostschweizer Fachhochschule
%
\section{Holomorphe Funktionen
\label{buch:funktionentheorie:section:holomorph}}
\rhead{Holomorphe Funktionen}

Wir betrachten in diesem Kapitel komplexwertige Funktionen,
\index{komplexwertige Funktion}%
die ein einem Teilgebiet der komplexen Ebene definiert sind.
Ein {\em Gebiet} ist eine offene Teilmenge $\Omega\subset \mathbb C$.
\index{Gebiet}%
{\em Offen} heisst, dass mit jedem Punkt $z_0\in\Omega$ eine Umgebung
\index{offen}%
\index{Umgebung}%
\[
U=\{z\in\mathbb Z\,|\,|z-z_0|<\varepsilon\}
\]
ebenfalls in $\Omega$ enthalten ist, also $U\subset \Omega$ für genügen
kleines $\varepsilon$.
Sei also $f(z)$ eine in $\Omega\subset\mathbb C$ definierte
Funktion $f\colon\Omega\to\mathbb C$.

Eine komplexwertige Funktion $f(z)$ kann betrachtet werden als zwei
reellwertige Funktionen von zwei Variablen $x$ und $y$:
\[
f(z)=\operatorname{Re}f(x+iy) + i \operatorname{Im}f(x+iy).
\]
Schreibt man
$\operatorname{Re}f(x+iy)=u(x,y)$
und
$\operatorname{In}f(x+iy)=v(x,y)$,
dann ist die komplexe Funktion vollständig durch reelle Funktionen
beschrieben.
Und natürlich wissen wir auch, was unter den Ableitungen der Funktionen
$u(x,y)$ und $v(x,y)$ zu verstehen ist.
Der Funktion $f(z)$ entspricht eine Abbildung $\mathbb R^2\to\mathbb R^2$
\index{Abbildung}%
\[
(x,y)\mapsto\begin{pmatrix}u(x,y)\\v(x,y)\end{pmatrix}.
\]
Die Ableitung einer solchen Funktion im Punkt $(x_0,y_0)$
ist eine lineare Abbildung von Vektoren, die in linearer Näherung
\index{lineare Naherung@lineare Näherung}
\index{Naherung@Näherung, lineare}
den Funktionswert bei $f(z_0 + \Delta z)$
\[
\begin{pmatrix}
u(x+\Delta x, y +\Delta y)\\
v(x+\Delta x, y +\Delta y)
\end{pmatrix}
=
\begin{pmatrix}
\frac{\partial u}{\partial x}&\frac{\partial u}{\partial y}\\
\frac{\partial v}{\partial x}&\frac{\partial v}{\partial y}
\end{pmatrix}
\begin{pmatrix} \Delta x\\\Delta y \end{pmatrix}
+o(\Delta x, \Delta y).
\]
In dieser Sicht einer komplexen Funktion gibt es keine einzelne Zahl, die
die Funktion einer Ableitung übernehmen könnte, die Ableitung
ist eine $2\times 2$-Matrix.

%
% Definition der komplexen Ableitungen
%
\subsection{Komplexe Ableitung}
Die Ableitung einer Funktion einer reellen Variablen wird mit Hilfe des
Grenzwertes
\[
f'(x_0)=\lim_{x\to x_0}\frac{f(x)-f(x_0)}{x-x_0}
\]
definiert, oder als diejenige Zahl $f'(x_0)\in\mathbb R$ mit der Eigenschaft,
dass
\begin{equation}
f(x)=f(x_0)+f'(x_0)(x-x_0) + o(x-x_0)
\label{komplex:abldef}
\end{equation}
gilt.
Der Term $x-x_0$ und die Gleichung \eqref{komplex:abldef} sind aber auch
für komplexe Argument sinnvoll, wir definieren daher

\begin{definition}
\label{buch:funktionentheorie:definition:differenzierbar}
Die komplexe Funktion $f(z)$ heisst im Punkt $z_0$ komplex differenzierbar
und hat die komplexe Ableitung $f'(z_0)\in\mathbb C$, wenn
\index{komplex differenzierbar}%
\index{komplexe Ableitung}%
\index{Ableitung!komplexe}%
\begin{equation}
f(z)=f(z_0) + f'(z_0)(z-z_0) +o(z-z_0)
\label{komplex:defkomplabl}
\end{equation}
gilt.
\end{definition}

\begin{beispiel}
Die Funktion $z\mapsto f(z)=z^n$ ist überall komplex differenzierbar
und hat die Ableitung $nz^{n-1}$.
Um dies nachzuprüfen, müssen wir die Bedingung~\eqref{komplex:defkomplabl}
verifizieren.
Aus einer wohlbekannten Faktorisierung von $z^n - z_0^n$ können wir den
Differenzenquotienten finden:
\begin{align*}
\frac{f(z)-f(z_0)}{z-z_0}
&=
\frac{z^n-z_0^n}{z-z_0}
=
\frac{(z-z_0)(z^{n-1}+z^{n-2}z_0+z^{n-3}z_0^2+\dots+z_0^{n-1})}{z-z_0}
\\
&=
\underbrace{z^{n-1}+z^{n-2}z_0+z^{n-3}z_0^2+\dots+z_0^{n-1}
}_{\displaystyle \text{$n$ Summanden}}.
\end{align*}
Lassen wir jetzt $z$ gegen $z_0$ gehen, wird die rechte Seite
zu $nz_0^{n-1}$.
\end{beispiel}

\begin{beispiel}
Die Funktion $z\mapsto f(z)=\bar z=x-iy$ ist nicht differenzierbar.
Wenn $f(z)=\bar z$ differenzierbar wäre, dann müsste es eine Zahl
$a\in\mathbb C$ geben, so dass
\[
\bar z-\bar z_0=a(z-z_0)+o(z-z_0)
\]
gilt.
wählen wir $z=z_0+x$ bzw.~$z=z_0+iy$, dann erhalten wir
\[
\begin{aligned}
z-z_0&=x:&
\bar z-\bar z_0&=x
&&\Rightarrow&
\bar z-\bar z_0&=1\cdot x
&&\Rightarrow&
a&=1
\\
z-z_0&=iy:&
\bar z-\bar z_0&=-iy
&&\Rightarrow&
\bar z-\bar z_0&=-1\cdot iy
&&\Rightarrow&
a&=-1
\end{aligned}
\]
Es ist also nicht möglich, eine einzige Zahl $a$ zu finden, die als
die Ableitung der Funktion $z\mapsto \bar z$ betrachtet werden könnte.
\end{beispiel}

Das letzte Beispiel zeigt, dass
selbst Funktionen, deren Real- und Imaginärteil beliebig oft stetig
differenzierbare Funktionen sind, nicht komplex differenzierbar
sein müssen.
Komplexe Differenzierbarkeit ist eine wesentlich stärkere Bedingung
an eine Funktion, komplex differenzierbare Funktionen bilden eine
echte Teilmenge aller Funktionen, deren Real- und Imaginärteil
differenzierbar ist.

%
% Cauchy-Riemann-Differentialgleichungen
%
\subsection{Die Cauchy-Riemann-Differentialgleichungen}
Komplexe Funktionen können nur differenzierbar sein, wenn sich die vier
partiellen Ableitungen zu einer einzigen komplexen Zahl zusammenfassen
lassen.
Um diese Beziehung zu finden, gehen wir von einer komplexen Funktion
\[
f(x+iy) = u(x,y) + iv(x,y)
\]
aus, und berechnen die Ableitung auf zwei verschiedene Arten, indem
wir sowohl nach $x$ als auch nach $iy$ ableiten:
\begin{align*}
f'(z)&
=
\lim_{x\to 0}\frac{f(z+x)-f(z)}{x}
=
\frac{\partial u}{\partial x}+i\frac{\partial v}{\partial x}
\\
f'(z)&
=
\lim_{y\to 0}\frac{f(z+iy)-f(z)}{iy}
=
\frac1{i}
\frac{\partial u}{\partial y}+\frac{\partial v}{\partial y}
=
\frac{\partial v}{\partial y}
-i
\frac{\partial u}{\partial y}.
\end{align*}
Dies ist nur möglich, wenn Real- und Imaginärteile übereinstimmen.
Es folgt also

\begin{satz}
\index{Satz!Cauchy-Riemann Differentialgleichungen}%
\label{komplex:satz:cauchy-riemann}
Real- und Imaginärteil $u(x,y)$ und $v(x,y)$ einer
komplex differenzierbaren Funktion $f(z)$ mit $f(x+iy)=u(x,y)+iv(x,y)$
erfüllen die Cauchy-Riemannschen Differentialgleichungen
\index{Cauchy-Riemann-Differentialgleichungen}
\begin{equation}
\begin{aligned}
\frac{\partial u}{\partial x}
&=
\frac{\partial v}{\partial y},
&
\frac{\partial u}{\partial y}
&=
-
\frac{\partial v}{\partial x}.
\end{aligned}
\label{komplex:dgl:cauchy-riemann}
\end{equation}
\end{satz}

Leitet man die Cauchy-Riemann-Differentialgleichungen nochmals nach
$x$ und $y$ ab, erhält man
\begin{equation*}
\begin{aligned}
\frac{\partial^2 u}{\partial x^2}
&=
\frac{\partial^2 v}{\partial x\,\partial y},
&
\frac{\partial^2 u}{\partial x\,\partial y}
&=
-\frac{\partial^2 v}{\partial x^2},
&
\frac{\partial^2 u}{\partial y\,\partial x}
&=
\frac{\partial^2 v}{\partial y^2},
&
\frac{\partial^2 u}{\partial y^2}
&=
-\frac{\partial^2 v}{\partial y\,\partial x}.
\end{aligned}
\end{equation*}
Die erste und die letzte sowie die mittleren zwei können zu jeweils
einer Differentialgleichung für die Funktionen $u$ und $v$ zusammengefasst
werden, nämlich
\begin{equation*}
\frac{\partial^2 u}{\partial x^2}
+
\frac{\partial^2 u}{\partial y^2}
=
0
\qquad\text{und}\qquad
\frac{\partial^2 v}{\partial x^2}
+
\frac{\partial^2 v}{\partial y^2}
=
0.
\end{equation*}

\begin{definition}
Der Operator
\[
\Delta =
\frac{\partial^2}{\partial x^2}
+
\frac{\partial^2}{\partial y^2}
\]
heisst der {\em Laplace-Operator} in zwei Dimensionen.
\index{Laplace-Operator}%
\index{Operator!Laplace-}%
\end{definition}

\begin{definition}
\label{buch:funktionentheorie:definition:harmonisch}
Eine Funktion $h(x,y)$ von zwei Variablen heisst {\em harmonisch}, wenn sie
die Gleichung
\[
\Delta h=0
\]
erfüllt.
\index{harmonische Funktion}%
\index{harmonisch}%
\end{definition}

\begin{satz}
Real- und Imaginärteil einer komplexen Funktion sind harmonische Funktionen.
\end{satz}

Die Cauchy-Riemann-Differentialgleichungen schränken also einerseits stark
ein, welche Funktionen überhaupt als Real- und Imaginärteil einer
komplex differenzierbaren Funktion in Frage kommen.
Andererseits koppeln sie auch Real- und Imaginärteil stark zusammen.

\begin{beispiel}
Von einer komplex differenzierbaren Funktion $f(z)$ sei nur der Realteil
$u(x,y)=x^3 -3xy^2$ bekannt.
Man finde alle möglichen Funktionen $f(z)$.

Zunächst kontrollieren wir, ob dies überhaupt ein Realteil sein kann,
indem wir nachrechnen, ob $u(x,y)$ harmonisch ist.
\begin{equation*}
\begin{aligned}
\frac{\partial u}{\partial x}
&=
3x^2-3y^2
&&\Rightarrow&
\frac{\partial^2 u}{\partial x^2}
&=
6x
\\
\frac{\partial u}{\partial y}
&=
-6xy
&&\Rightarrow&
\frac{\partial^2 u}{\partial y^2}
&=
-6x
\\
&&&&\Delta u&=\frac{\partial^2u}{\partial x^2}+\frac{\partial^2u}{\partial y^2}=6x-6x=0,
\end{aligned}
\end{equation*}
$u$ ist also harmonisch.

Um die Funktion $f$ zu finden, brauchen wir jetzt noch den Imaginärteil.
Wir finden ihn mit Hilfe der Cauchy-Riemann-Differentialgleichungen.
Es gilt
\begin{equation}
\begin{aligned}
\frac{\partial v}{\partial x}
&=
-\frac{\partial u}{\partial y}=6xy,
&
\frac{\partial v}{\partial y}
&=
\frac{\partial u}{\partial x}=3x^2-3y^2
\end{aligned}
\label{komplex:crbeispiel}
\end{equation}
Aus der ersten Gleichung erhält man durch Integrieren nach $x$
\[
v(x,y)=-3x^2y + C(y),
\]
die Integrations-``Konstante'' ist eine Funktion, die aber nur von $y$
abhängen darf.
Die zweite Cauchy-Riemann-Gleichung verwendet die Ableitung von $v$ nach $y$,
sie ist
\[
\frac{\partial v}{\partial y}=3x^2+C'(y).
\]
Aus der zweiten Gleichung von \eqref{komplex:crbeispiel} liest man
ab, dass
\[
C'(y)=-3y^2
\qquad\Rightarrow\qquad
C(y)=-y^3+k
\]
sein muss.
Damit ist $v$ bis auf eine Konstante bestimmt.
Die zugehörige Funktion $f(z)$ ist daher
\[
f(z)=f(x+iy)=x^3-3xy^2+i(3x^2y-y^3)+ik
=x^3 + 3x^2iy + 3x(iy)^2+(iy)^3+ik=z^3+ik.
\]
Wir haben die Funktion $f(z)$ bis auf eine Konstanten $ik$
aus ihrem Realteil rekonstruiert.
\end{beispiel}
Die Cauchy-Riemann-Differentialgleichungen besagen auch, dass man nur
die Ableitungen nach $x$ zu berechnen braucht, um die Ableitung $f'(x)$
zu bestimmen.
Die Rechenregeln für die Ableitung lassen sich daher direkt auf
komplexe Funktionen übertragen:
\begin{align*}
\frac{d}{dz}z^n
&=
nz^{n-1}
\\
\frac{d}{dz}e^z
&=
e^z
\\
\frac{d}{dz}f(g(z))
&=
f'(g(z)) g'(z)
\\
\frac{d}{dz}\bigl(f(z)g(z)\bigr)
&=
f'(z)g(z)+f(z)g'(z)
\end{align*}
Die Ableitungsformeln ändern also nicht, die formalen Ableitungsregeln
für holomorphe Funktionen sind die gleichen wie für reelle Funktionen.



