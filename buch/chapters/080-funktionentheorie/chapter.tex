%
% chapter.tex -- Kapitel zur Funktionentheorie
%
% (c) 2021 Prof Dr Andreas Müller, Hochschule Rapperswil
%
% !TeX spellcheck = de_CH
\chapter{Funktionentheorie
\label{buch:chapter:funktionentheorie}}
\lhead{Funktionentheorie}
\rhead{}
Jede stetige reelle Funktion $f\colon I\to\mathbb{R}$ auf einem
Intervall kann beliebig genau durch Polynome, also durch
differenzierbare approximiert werden.
Für komplex differenzierbare Funktionen sieht die Situation
völlig anders aus.
Bereits die Funktion $z\mapsto \overline{z}$ kann in einer offenen
Teilmenge von $\mathbb{C}$ nicht durch Polynome in der Variablen $z$
approximiert werden.
Es stellt sich heraus, dass komplex differenzierbare Funktionen
immer eine konvergente Taylor-Reihe besitzen.
In Abschnitt~\ref{buch:funktionentheorie:section:analytisch} wird
ein Beispiel einer beliebig oft stetig differenzierbaren rellen
Funktion angegeben, die nur in $0$ verschwindet, deren Taylor-Reihe
in $0$ die Nullfunktion ist.

Wenn man also weiss, dass die Lösung eines Problems nicht nur eine
relle Funktion ist, sondern eine komplex differenzierbare Funktion,
dann unterliegt diese sehr viel strengeren Einschränkungen.
Mit der zugehörigen Potenzreihe können Funktionswerte leicht berechnet
werden, mit dem Cauchy-Integral können Singularitäten studiert werden
und mit der analytischen Fortsetzung kann man Lösungen über Singularitäten
auf der rellen Achse hinaus fortsetzen.

%
% holomorph.tex
%
% (c) 2021 Prof Dr Andreas Müller, OST Ostschweizer Fachhochschule
%
\section{Holomorphe Funktionen
\label{buch:funktionentheorie:section:holomorph}}
\rhead{Holomorphe Funktionen}

Wir betrachten in diesem Kapitel komplexwertige Funktionen,
\index{komplexwertige Funktion}%
die ein einem Teilgebiet der komplexen Ebene definiert sind.
Ein {\em Gebiet} ist eine offene Teilmenge $\Omega\subset \mathbb C$.
\index{Gebiet}%
{\em Offen} heisst, dass mit jedem Punkt $z_0\in\Omega$ eine Umgebung
\index{offen}%
\index{Umgebung}%
\[
U=\{z\in\mathbb Z\,|\,|z-z_0|<\varepsilon\}
\]
ebenfalls in $\Omega$ enthalten ist, also $U\subset \Omega$ für genügen
kleines $\varepsilon$.
Sei also $f(z)$ eine in $\Omega\subset\mathbb C$ definierte
Funktion $f\colon\Omega\to\mathbb C$.

Eine komplexwertige Funktion $f(z)$ kann betrachtet werden als zwei
reellwertige Funktionen von zwei Variablen $x$ und $y$:
\[
f(z)=\operatorname{Re}f(x+iy) + i \operatorname{Im}f(x+iy).
\]
Schreibt man
$\operatorname{Re}f(x+iy)=u(x,y)$
und
$\operatorname{In}f(x+iy)=v(x,y)$,
dann ist die komplexe Funktion vollständig durch reelle Funktionen
beschrieben.
Und natürlich wissen wir auch, was unter den Ableitungen der Funktionen
$u(x,y)$ und $v(x,y)$ zu verstehen ist.
Der Funktion $f(z)$ entspricht eine Abbildung $\mathbb R^2\to\mathbb R^2$
\index{Abbildung}%
\[
(x,y)\mapsto\begin{pmatrix}u(x,y)\\v(x,y)\end{pmatrix}.
\]
Die Ableitung einer solchen Funktion im Punkt $(x_0,y_0)$
ist eine lineare Abbildung von Vektoren, die in linearer Näherung
\index{lineare Naherung@lineare Näherung}
\index{Naherung@Näherung, lineare}
den Funktionswert bei $f(z_0 + \Delta z)$
\[
\begin{pmatrix}
u(x+\Delta x, y +\Delta y)\\
v(x+\Delta x, y +\Delta y)
\end{pmatrix}
=
\begin{pmatrix}
\frac{\partial u}{\partial x}&\frac{\partial u}{\partial y}\\
\frac{\partial v}{\partial x}&\frac{\partial v}{\partial y}
\end{pmatrix}
\begin{pmatrix} \Delta x\\\Delta y \end{pmatrix}
+o(\Delta x, \Delta y).
\]
In dieser Sicht einer komplexen Funktion gibt es keine einzelne Zahl, die
die Funktion einer Ableitung übernehmen könnte, die Ableitung
ist eine $2\times 2$-Matrix.

%
% Definition der komplexen Ableitungen
%
\subsection{Komplexe Ableitung}
Die Ableitung einer Funktion einer reellen Variablen wird mit Hilfe des
Grenzwertes
\[
f'(x_0)=\lim_{x\to x_0}\frac{f(x)-f(x_0)}{x-x_0}
\]
definiert, oder als diejenige Zahl $f'(x_0)\in\mathbb R$ mit der Eigenschaft,
dass
\begin{equation}
f(x)=f(x_0)+f'(x_0)(x-x_0) + o(x-x_0)
\label{komplex:abldef}
\end{equation}
gilt.
Der Term $x-x_0$ und die Gleichung \eqref{komplex:abldef} sind aber auch
für komplexe Argument sinnvoll, wir definieren daher

\begin{definition}
\label{buch:funktionentheorie:definition:differenzierbar}
Die komplexe Funktion $f(z)$ heisst im Punkt $z_0$ komplex differenzierbar
und hat die komplexe Ableitung $f'(z_0)\in\mathbb C$, wenn
\index{komplex differenzierbar}%
\index{komplexe Ableitung}%
\index{Ableitung!komplexe}%
\begin{equation}
f(z)=f(z_0) + f'(z_0)(z-z_0) +o(z-z_0)
\label{komplex:defkomplabl}
\end{equation}
gilt.
\end{definition}

\begin{beispiel}
Die Funktion $z\mapsto f(z)=z^n$ ist überall komplex differenzierbar
und hat die Ableitung $nz^{n-1}$.
Um dies nachzuprüfen, müssen wir die Bedingung~\eqref{komplex:defkomplabl}
verifizieren.
Aus einer wohlbekannten Faktorisierung von $z^n - z_0^n$ können wir den
Differenzenquotienten finden:
\begin{align*}
\frac{f(z)-f(z_0)}{z-z_0}
&=
\frac{z^n-z_0^n}{z-z_0}
=
\frac{(z-z_0)(z^{n-1}+z^{n-2}z_0+z^{n-3}z_0^2+\dots+z^{n-1})}{z-z_0}
\\
&=
\underbrace{z^{n-1}+z^{n-2}z_0+z^{n-3}z_0^2+\dots+z^{n-1}
}_{\displaystyle \text{$n$ Summanden}}.
\end{align*}
Lassen wir jetzt $z$ gegen $z_0$ gehen, wird die rechte Seite
zu $nz_0^{n-1}$.
\end{beispiel}

\begin{beispiel}
Die Funktion $z\mapsto f(z)=\bar z=x-iy$ ist nicht differenzierbar.
Wenn $f(z)=\bar z$ differenzierbar wäre, dann müsste es eine Zahl
$a\in\mathbb C$ geben, so dass
\[
\bar z-\bar z_0=a(z-z_0)+o(z-z_0)
\]
gilt.
wählen wir $z=z_0+x$ bzw.~$z=z_0+iy$, dann erhalten wir
\[
\begin{aligned}
z-z_0&=x:&
\bar z-\bar z_0&=x
&&\Rightarrow&
\bar z-\bar z_0&=1\cdot x
&&\Rightarrow&
a&=1
\\
z-z_0&=iy:&
\bar z-\bar z_0&=-iy
&&\Rightarrow&
\bar z-\bar z_0&=-1\cdot iy
&&\Rightarrow&
a&=-1
\end{aligned}
\]
Es ist also nicht möglich, eine einzige Zahl $a$ zu finden, die als
die Ableitung der Funktion $z\mapsto \bar z$ betrachtet werden könnte.
\end{beispiel}

Das letzte Beispiel zeigt, dass
selbst Funktionen, deren Real- und Imaginärteil beliebig oft stetig
differenzierbare Funktionen sind, nicht komplex differenzierbar
sein müssen.
Komplexe Differenzierbarkeit ist eine wesentlich stärkere Bedingung
an eine Funktion, komplex differenzierbare Funktionen bilden eine
echte Teilmenge aller Funktionen, deren Real- und Imaginärteil
differenzierbar ist.

%
% Cauchy-Riemann-Differentialgleichungen
%
\subsection{Die Cauchy-Riemann-Differentialgleichungen}
Komplexe Funktionen können nur differenzierbar sein, wenn sich die vier
partiellen Ableitungen zu einer einzigen komplexen Zahl zusammenfassen
lassen.
Um diese Beziehung zu finden, gehen wir von einer komplexen Funktion
\[
f(x+iy) = u(x,y) + iv(x,y)
\]
aus, und berechnen die Ableitung auf zwei verschiedene Arten, indem
wir sowohl nach $x$ als auch nach $iy$ ableiten:
\begin{align*}
f'(z)&
=
\lim_{x\to 0}\frac{f(z+x)-f(z)}{x}
=
\frac{\partial u}{\partial x}+i\frac{\partial v}{\partial x}
\\
f'(z)&
=
\lim_{y\to 0}\frac{f(z+iy)-f(z)}{iy}
=
\frac1{i}
\frac{\partial u}{\partial y}+\frac{\partial v}{\partial y}
=
\frac{\partial v}{\partial y}
-i
\frac{\partial u}{\partial y}.
\end{align*}
Dies ist nur möglich, wenn Real- und Imaginärteile übereinstimmen.
Es folgt also

\begin{satz}
\label{komplex:satz:cauchy-riemann}
Real- und Imaginärteil $u(x,y)$ und $v(x,y)$ einer
komplex differenzierbaren Funktion $f(z)$ mit $f(x+iy)=u(x,y)+iv(x,y)$
erfüllen die Cauchy-Riemannschen Differentialgleichungen
\index{Cauchy-Riemann-Differentialgleichungen}
\begin{equation}
\begin{aligned}
\frac{\partial u}{\partial x}
&=
\frac{\partial v}{\partial y},
&
\frac{\partial u}{\partial y}
&=
-
\frac{\partial v}{\partial x}.
\end{aligned}
\label{komplex:dgl:cauchy-riemann}
\end{equation}
\end{satz}

Leitet man die Cauchy-Riemann-Differentialgleichungen nochmals nach
$x$ und $y$ ab, erhält man
\begin{equation*}
\begin{aligned}
\frac{\partial^2 u}{\partial x^2}
&=
\frac{\partial^2 v}{\partial x\,\partial y},
&
\frac{\partial^2 u}{\partial x\,\partial y}
&=
-\frac{\partial^2 v}{\partial x^2},
&
\frac{\partial^2 u}{\partial y\,\partial x}
&=
\frac{\partial^2 v}{\partial y^2},
&
\frac{\partial^2 u}{\partial y^2}
&=
-\frac{\partial^2 v}{\partial y\,\partial x}.
\end{aligned}
\end{equation*}
Die erste und die letzte sowie die mittleren zwei können zu jeweils
einer Differentialgleichung für die Funktionen $u$ und $v$ zusammengefasst
werden, nämlich
\begin{equation*}
\frac{\partial^2 u}{\partial x^2}
+
\frac{\partial^2 u}{\partial y^2}
=
0
\qquad\text{und}\qquad
\frac{\partial^2 v}{\partial x^2}
+
\frac{\partial^2 v}{\partial y^2}
=
0.
\end{equation*}

\begin{definition}
Der Operator
\[
\Delta =
\frac{\partial^2}{\partial x^2}
+
\frac{\partial^2}{\partial y^2}
\]
heisst der {\em Laplace-Operator} in zwei Dimensionen.
\index{Laplace-Operator}%
\end{definition}

\begin{definition}
\label{buch:funktionentheorie:definition:harmonisch}
Eine Funktion $h(x,y)$ von zwei Variablen heisst {\em harmonisch}, wenn sie
die Gleichung
\[
\Delta h=0
\]
erfüllt.
\index{harmonische Funktion}%
\index{harmonisch}%
\end{definition}

\begin{satz}
Real- und Imaginärteil einer komplexen Funktion sind harmonische Funktionen.
\end{satz}

Die Cauchy-Riemann-Differentialgleichungen schränken also einerseits stark
ein, welche Funktionen überhaupt als Real- und Imaginärteil einer
komplex differenzierbaren Funktion in Frage kommen.
Andererseits koppeln sie auch Real- und Imaginärteil stark zusammen.

\begin{beispiel}
Von einer komplex differenzierbaren Funktion $f(z)$ sei nur der Realteil
$u(x,y)=x^3 -3xy^2$ bekannt.
Man finde alle möglichen Funktionen $f(z)$.

Zunächst kontrollieren wir, ob dies überhaupt ein Realteil sein kann,
indem wir nachrechnen, ob $u(x,y)$ harmonisch ist.
\begin{equation*}
\begin{aligned}
\frac{\partial u}{\partial x}
&=
3x^2-3y^2
&&\Rightarrow&
\frac{\partial^2 u}{\partial x^2}
&=
6x
\\
\frac{\partial u}{\partial y}
&=
-6xy
&&\Rightarrow&
\frac{\partial^2 u}{\partial y^2}
&=
-6x
\\
&&&&\Delta u&=\frac{\partial^2u}{\partial x^2}+\frac{\partial^2u}{\partial y^2}=6x-6x=0,
\end{aligned}
\end{equation*}
$u$ ist also harmonisch.

Um die Funktion $f$ zu finden, brauchen wir jetzt noch den Imaginärteil.
Wir finden ihn mit Hilfe der Cauchy-Riemann-Differentialgleichungen.
Es gilt
\begin{equation}
\begin{aligned}
\frac{\partial v}{\partial x}
&=
-\frac{\partial u}{\partial y}=6xy,
&
\frac{\partial v}{\partial y}
&=
\frac{\partial u}{\partial x}=3x^2-3y^2
\end{aligned}
\label{komplex:crbeispiel}
\end{equation}
Aus der ersten Gleichung erhält man durch Integrieren nach $x$
\[
v(x,y)=-3x^2y + C(y),
\]
die Integrations-``Konstante'' ist eine Funktion, die aber nur von $y$
abhängen darf.
Die zweite Cauchy-Riemann-Gleichung verwendet die Ableitung von $v$ nach $y$,
sie ist
\[
\frac{\partial v}{\partial y}=3x^2+C'(y).
\]
Aus der zweiten Gleichung von \eqref{komplex:crbeispiel} liest man
ab, dass
\[
C'(y)=-3y^2
\qquad\Rightarrow\qquad
C(y)=-y^3+k
\]
sein muss.
Damit ist $v$ bis auf eine Konstante bestimmt.
Die zugehörige Funktion $f(z)$ ist daher
\[
f(z)=f(x+iy)=x^3-3xy^2+i(3x^2y-y^3)+ik
=x^3 + 3x^2iy + 3x(iy)^2+(iy)^3+ik=z^3+ik.
\]
Wir haben die Funktion $f(z)$ bis auf eine Konstanten $ik$
aus ihrem Realteil rekonstruiert.
\end{beispiel}
Die Cauchy-Riemann-Differentialgleichungen besagen auch, dass man nur
die Ableitungen nach $x$ zu berechnen braucht, um die Ableitung $f'(x)$
zu bestimmen.
Die Rechenregeln für die Ableitung lassen sich daher direkt auf
komplexe Funktionen übertragen:
\begin{align*}
\frac{d}{dz}z^n
&=
nz^{n-1}
\\
\frac{d}{dz}e^z
&=
e^z
\\
\frac{d}{dz}f(g(z))
&=
f'(g(z)) g'(z)
\\
\frac{d}{dz}\bigl(f(z)g(z)\bigr)
&=
f'(z)g(z)+f(z)g'(z)
\end{align*}
Die Ableitungsformeln ändern also nicht, die formalen Ableitungsregeln
für holomorphe Funktionen sind die gleichen wie für reelle Funktionen.




%
% analytisch.tex
%
% (c) 2021 Prof Dr Andreas Müller, OST Ostschweizer Fachhochschule
%
\section{Analytische Funktionen
\label{buch:funktionentheorie:section:analytisch}}
\rhead{Analytische Funktionen}
Holomorphe Funktionen zeichnen sich dadurch aus, dass sie auch immer
eine konvergente Reihenentwicklung haben, sie sind also analytisch.

\subsection{Definition}
\index{Taylor-Reihe}%
\index{Exponentialfunktion}%
Die Taylor-Reihenentwicklung der Exponentialfunktion ermöglicht deren
effiziente Berechnung.
Es ist aber nicht selbstverständlich, dass die Taylor-Reihe überhaupt
gegen die Funktion konvergiert, aus deren Ableitungen sie gebildet
worden ist, wie das folgende Beispiel illustriert.

\begin{figure}
\centering
\includegraphics{chapters/080-funktionentheorie/images/nonanalytic.pdf}
\caption{Beispiel einer beliebig oft stetig differenzierbaren Funktion,
deren Ableitungen in $x=0$ alle verschwinden.
Die zugehörige Taylor-Reihe ist die Nullfunktion, sie hat nichts mit der
Funktion zu tun.
\label{buch:funktionentheorie:fig:nonanalytic}}
\end{figure}

\begin{beispiel}
\label{buch:funktionentheorie:beispiel:nichtanalytisch}
Wir betrachten die Funktion
\[
f\colon \mathbb{R}\to\mathbb{R}
:
x \mapsto
\begin{cases}
e^{-1/x^2}&\qquad x\ne 0\\
0&\qquad x=0.
\end{cases}
\]
Der Graph $y=f(x)$ ist in Abbildung~\ref{buch:funktionentheorie:fig:nonanalytic}
dargestellt.

Die ersten zwei Ableitungen der Funktion $f$ sind
\begin{align*}
f'(x) &= \frac{2e^{-1/x^2}}{x^3} = \frac{2}{x^3}\cdot f(x)
\\
f''(x) &= \frac{(4-6x^2) e^{-1/x^2}}{x^6} = \frac{4-6x^2}{x^6}\cdot f(x)
\\
&\dots
\end{align*}
Man kann vermuten, dass alle
Ableitungen Funktionen der Form
\begin{equation}
F(x) = \frac{p(x)}{x^n} \cdot f(x),
\label{buch:funktionentheorie:eqn:nonanalytic:form}
\end{equation}
sind,
wobei $p(x)$ ein Polynom ist.
Leitet man eine solche Funktion nach $x$ ab, erhält man
\begin{align*}
\frac{d}{dx} F(x)
&=
\frac{\frac{d}{dx}(p(x)f(x)) x^n - nx^{n-1}p(x)f(x)}{x^{2n}}
\\
&=
\frac{p'(x)f(x) + p(x)f'(x) - nx^{n-1}p(x)f(x)}{x^{2n}} 
\\
&=
\frac{p'(x) + p(x)(2/x^3) - nx^{n-1}p(x)}{x^{2n}} \cdot f(x)
\\
&=
\frac{x^3p'(x)+2p(x)-nx^{n-1}p(x)}{x^{2n+3}}\cdot f(x).
\end{align*}
Dies ist wieder eine Funktion der
Form~\eqref{buch:funktionentheorie:eqn:nonanalytic:form}.

Der Faktor $f(x)=e^{-1/x^2}$ von $F(x)$ geht für $x\to 0$ exponentiell
schnell gegen $0$, schneller als der Nenner $x^n$ gegen $0$ gehen
kann. 
Der Grenzwert $x\to 0$ einer Funktion der 
Form~\eqref{buch:funktionentheorie:eqn:nonanalytic:form}
ist daher immer
\[
\lim_{x\to 0}  F(x) =0.
\]
Damit ist gezeigt, dass alle Ableitungen $f^{(n)}(0)=0$ sind.
Die Taylorreihe von $f(x)$ ist daher die Nullfunktion.
\end{beispiel}

Die Klasse der Funktionen, die sich durch ihre Taylor-Reihe darstellen
lassen, zeichnet sich also durch besondere Eigenschaften aus, die in
der folgenden Definition zusammengefasst werden.

\index{analytisch in einem Punkt}%
\index{analytisch}%
\begin{definition}
Eine auf einem offenen Intervall $I\subset \mathbb {R}$ definierte Funktion
$f\colon U\to\mathbb{R}$ heisst {\em analytisch im Punkt  $x_0\in I$}, wenn
es eine in einer Umgebung von $x_0$ konvergente Potenzreihe
\[
\sum_{k=0}^\infty a_k(x-x_0)^k = f(x)
\]
gibt.
Sie heisst {\em analytisch}, wenn sie analytisch ist in jedem Punkt von $I$.
\end{definition}

Es ist wohlbekannt aus der elementaren Theorie der Potenzreihen, dass
eine analytische Funktion beliebig oft differenzierbar ist und dass
die Potenzreihe im Punkt $x_0$ die Taylor-Reihe sein muss.
Ausserdem sidn Summen, Differenzen und Produkte von analytischen Funktionen
wieder analytisch.

Für eine komplexe Funktion lässt sich der Begriff der
analytischen Funktion genau gleich definieren.

\begin{definition}
Eine in einer offenen Teilmenge $U\subset \mathbb{C}$ definierte Funktion
$f\colon U\to\mathbb{C}$ heisst {\em analytisch im Punkt $z_0\in U$}, wenn
es eine in einer Umgebung von $z_0$ konvergente Potenzreihe
\[
\sum_{k=0}^\infty a_k(z-z_0)^k = f(z)
\]
gibt.
Sie heisst {\em analytisch}, wenn sie analytisch ist in jedem Punkt von $U$.
\end{definition}

Die Verwendung einer offenen Teilmenge $U\subset\mathbb{C}$ ist wesentlich,
denn die Funktion $f\colon z\mapsto \overline{z}$ kann in jedem Punkt
$x_0\in\mathbb{R}$
der reellen Achse $\mathbb{R}\subset\mathbb{C}$ durch die Potenzreihe 
$f(x) = x_0 + (x-x_0)$ dargestellt werden.
Es gibt aber keine Potenzreihe, die $f(z)$ in einer offenen Teilmenge
von $\mathbb{C}$ gegen $f(z)=\overline{z}$ konvergiert.

%
% Der Konvergenzradius einer Potenzreihe
%
\subsection{Konvergenzradius
\label{buch:funktionentheorie:subsection:konvergenzradius}}
In der Theorie der Potenzreihen, die man in einem grundlegenden
Analysiskurs lernt, wird auch genauer untersucht, wie gross
eine Umgebung des Punktes $z_0$ ist, in der die Potenzreihe
im Punkt $z_0$ einer analytischen Funktion konvergiert.

\begin{satz}
\label{buch:funktionentheorie:satz:konvergenzradius}
Die Potenzreihe
\[
f(z) = \sum_{k=0}^\infty a_0(z-z_0)^k
\]
ist konvergent auf einem Kreis mit Radius $\varrho$ und
\[
\frac{1}{\varrho}
=
\limsup_{n\to\infty} \sqrt[k]{|a_k|}.
\]
Falls $a_k\ne 0$ für alle $k$ und der folgende Grenzwert existiert,
dann gilt auch
\[
\varrho = \lim_{n\to\infty} \biggl| \frac{a_n}{a_{n+1}}\biggr|.
\]
\end{satz}

\begin{definition}
\label{buch:funktionentheorie:definition:konvergenzradius}
\index{Konvergenzradius}%
Der in Satz~\ref{buch:funktionentheorie:satz:konvergenzradius}
Radius $\varrho$ des Konvergenzkreises heisst {\em Konvergenzradius}.
\end{definition}

Man kann auch zeigen, dass der Konvergenzkreis immer so gross ist,
dass auf seinem Rand ein Wert $z$ liegt, für den die Potenzreihe nicht
konvergiert.


\input{chapters/080-funktionentheorie/cauchy.tex}
\input{chapters/080-funktionentheorie/fortsetzung.tex}
%
% anwendungen.tex
%
% (c) 2021 Prof Dr Andreas Müller, OST Ostschweizer Fachhochschule
%
\section{Anwendungen
\label{buch:funktionentheorie:section:anwendungen}}
\rhead{Anwendungen}

\input{chapters/080-funktionentheorie/gammareflektion.tex}
%
% carlson.tex
%
% (c) 2022 Prof Dr Andreas Müller, OST Ostschweizer Fachhochschule
%
\subsection{Der Satz von Carlson
\label{buch:funktionentheorie:subsection:satz-von-carlson}}
In Abschnitt~\ref{buch:rekursion:section:gamma} wurde gezeigt,
wie die Gamma-Funktion $\Gamma(x)$ konstruiert werden kann, die
in ganzzahligen Argumenten mit der Fakultät zusammenfällt.
Es wurde auch gezeigt, dass $\Gamma(x)+\sin(\pi x)$ eine
weitere Funktion mit dieser Eigenschaft ist.
Die Integraldefinition der
Gamma-Funktion~\ref{buch:rekursion:def:gamma} zeigt, dass
die Gamma-Funktion holomorph ist.
Der folgende Satz von Carlson zeigt jetzt, dass sich
zwei solche Lösungen um eine unbeschränkte Funktion
unterscheiden müssen.

\begin{satz}[Carlson]
\label{buch:funktionentheorie:satz:carlson}
Ist $f(z)$ eine holomorphe Funktion, die für $\operatorname{R}z\ge 0$
beschränkt ist und an den Stellen $z=1,2,3,\dots$ verschwindet.
Dann ist $f(z)=0$.
\end{satz}

\begin{figure}
\centering
\includegraphics{chapters/080-funktionentheorie/images/carlsonpath.pdf}
\caption{Pfad zum Beweis des Satzes \ref{buch:funktionentheorie:satz:carlson}
von Carlson.
\label{buch:funktionentheorie:fig:carlsonpath}}
\end{figure}

\begin{proof}[Beweis]
Da $f(1)=f(2)=f(3)=\dots=0$ ist auch die Funktion
\[
g_n(z) = \frac{f(z)}{(z-1)(z-2)\cdot\ldots\cdot(z-n)}
\]
eine holomorphe Funktion.
Für $|z|>n$ ist jeder Faktor im Nenner betragsmässig $>1$,
also ist $g_n(z)$ in der rechten Halbebene nicht nur beschränkt,
es gilt sogar
\[
|g_n(z)| =\frac{|f(z)|}{|z-1|\cdot|z-2|\cdot\ldots\cdot|z-n|}
\le \frac{M}{(|z|-n)^n}
=
O\biggl(\frac{1}{|z|^n}\biggr)
\qquad\text{für $|z|\to\infty$}.
\]
Mit dem Cauchy-Integralsatz kann man jetzt $g_n(a)$ für einen
Punkt $a$ in der rechten Halbebene berechnen, er ist
\begin{equation}
g_n(a)
=
\frac{1}{2\pi i}
\oint_{\gamma} \frac{g_n(z)}{z-a}\,dz
=
\frac{f(a)}{(a-1)(a-2)\cdot\ldots\cdot(a-n)},
\label{buch:funktionentheorie:proof:eqn:gna}
\end{equation}
wobei $\gamma$ ein Pfad ist, der $a$ umschliessen muss.

Als Pfad wählen wir einen Halbkreis $C_1$ vom Radius $R$ um den Nullpunkt
und das Segment von $-iR$ bis $iR$, dargestellt in
Abbildung~\ref{buch:funktionentheorie:fig:carlsonpath}.
% XXX Bild des Pfades
Das Integral über den Halbkreis kann durch
\begin{align*}
\biggl|
\frac{1}{2\pi i}
\int_{C_1} \frac{f(z)}{(z-a)(z-1)(z-2)\cdot\ldots\cdot(z-n)}\,dz
\biggr|
&\le
\frac1{2\pi} \max_{|z|=R\wedge\operatorname{Re}z\ge 0}
\frac{M}{|z-a|\cdot|z-1|\cdot|z-2|\cdot\ldots\cdot|z-n|}\pi R
\\
&\le
\frac{M\pi R}{(R-n)^n}
\end{align*}
abgeschätzt werden.
Die rechte Seite geht für $n>1$ gegen $0$ wenn $R\to\infty$ geht.
Das Integral über den Kreisbogen $C_1$ trägt also nichts bei zum
Integral~\eqref{buch:funktionentheorie:proof:eqn:gna}

Es bleibt das Integral über die imaginäre Achse, es ist
\begin{align}
g_n(a)
&=
\frac{1}{2\pi i}
\int_{-\infty}^\infty
\frac{f(it)}{(it-a)(it-1)(it-2)\cdot\ldots\cdot(it-n)}
\,i\,dt
\notag
\\
|g_n(a)|
&=
\frac{1}{2\pi}
\int_{-\infty}^\infty
\frac{|f(it)|}{
\sqrt{(a^2+t^2)(1^2+t^2)(2^2+t^2)\cdot\ldots\cdot(n^2+t^2)}
}
\,dt.
\notag
\intertext{Im Nenner kann man in den Faktoren $(k^2+t^2)$ mit $k>1$
das $k^2$ weglassen, was den Nenner kleiner und damit den ganzen Ausdruck
grösser macht.
Es bleibt dann nur noch der erste Term, in dem wir $a>1$ durch $1$ ersetzen
können.
Insgesamt bekommen wir so die Abschätzung}
&\le
\frac{1}{2\pi} \int_{-\infty}^\infty
\frac{M}{\sqrt{(1+t^2)(1+t^2)}\cdot 2\cdot 3\cdot\ldots\cdot n}
\,dt
=
\frac{M}{2\pi n!}
\int_{-\infty}^\infty\frac{dt}{1+t^2}
=
\frac{M}{2 n!}.
\label{buch:funktionentheorie:carlson:eqn:integral}
\end{align}
Um eine Abschätzung für $f(a)$ zu erhalten, muss man jetzt noch den Nenner
von \eqref{buch:funktionentheorie:proof:eqn:gna} abschätzen.
Da $a$ nicht ganzzahlig ist, ist die nächstkleiner Ganzzahl $[a]\ne a$.
Das Produkt im Nenner von \eqref{buch:funktionentheorie:proof:eqn:gna}
kann daher aufgespaltet werden in die Faktoren $(a-k)$ mit $k<a$ und
die Faktoren  mit $k>a$.
Den Betrag der Faktoren mit $k<a$ kann man vergrössern, indem man $a$
durch $[a]+1$ ersetzt, man erhält
\begin{align*}
|
(a-1)(a-2)\cdots(a-[a])
|
&\le
([a]+1-1)([a]+1-2)\cdots([a]+1-[a])=[a]!.
\intertext{Die nachfolgenden Faktoren kann man vergrössern, indem man $a$ durch $[a]$ ersetzt, was}
|(a-([a]+1))(a-([a]+2))\cdots(a-n)|
&\le
|([a]-([a]+1))([a]-([a]+2))\cdots([a]-n)|
\\
&=
1\cdot 2 \cdot\ldots\cdot |n-[a]|
=
(n-[a])!.
\end{align*}
ergibt.
Aus \eqref{buch:funktionentheorie:proof:eqn:gna} und der Abschätzung
\eqref{buch:funktionentheorie:carlson:eqn:integral}
für $|g_n(a)$
erhält man jetzt
\[
|f(a)|
=
|(a-1)(a-2)\cdots(a-n)|\cdot|g_n(a)|
\le 
\frac{[a]!\,(n-[a])!}{n!}
\frac{M}{2}
=
\frac{M}{2} \binom{n}{[a]}^{-1}.
\]
Für $n>[a]$ ist der Binomialkoeffizient auch $>n$ und somit
\[
|f(a)|\le \frac{M}{2n}
\to 0
\qquad\text{für $n\to\infty$}.
\]
Damit ist gezeigt, dass $f(a)=0$ ist für alle reellen $a>1$.
A fortiori verschwinden auch alle Ableitungen von $f$ und damit
damit auch die zugehörige Potenzreihe, also $f(z)=0$.
\end{proof}


%
% singularitaeten.tex
%
% (c) 2022 Prof Dr Andreas Müller, OST Ostschweizer Fachhochschule
%
\newcommand*\sk{\vcenter{\hbox{\includegraphics[scale=0.8]{chapters/080-funktionentheorie/images/operator-1.pdf}}}}

%
% Löesung linearer Differentialgleichunge mit Singularitäten
%
\subsection{Lösungen von linearen Differentialgleichungen mit Singularitäten
\label{buch:funktionentheorie:subsection:dglsing}}
Die Potenzreihenmethode hat ermöglicht, mindestens eine Lösung gewisser
linearer Differentialgleichungen zu finden.
Bei Differentialgleichungen wie der Besselschen Differentialgleichung,
deren Koeffizienten Singularitäten aufweisen, konnte aber nur eine
Lösung gefunden werden, während die Theorie verlangt, dass eine
Differentialgleichung zweiter Ordnung zwei linear unabhängige Lösungen
haben muss.

Ziel dieses Abschnitts ist zu zeigen, warum dies nicht möglich war und
wie diese Schwierigkeit mit Hilfe der analytischen Fortsetzung überwunden
werden kann.

%
% Differentialgleichungen mit Singularitäten
%
\subsubsection{Differentialgleichungen mit Singularitäten}
Mit der Besselschen
Differentialgleichung~\eqref{buch:differentialgleichungen:eqn:bessel}
ist es nicht möglich, die zweite Ableitung $y''(0)$ an der Stelle $x=0$
zu bestimmen.
Die Differentialgleichung kann an der Stelle $x=0$ nicht nach $y''$
aufgelöst werden.
Wenn man die Differentialgleichung in ein Differntialgleichungssystem
\[
\frac{d}{dx}
\begin{pmatrix}
y_1\\y_2
\end{pmatrix}
=
\begin{pmatrix}
0&1\\
1-\frac{\alpha^2}{x^2}
&
-\frac{1}{x}
\end{pmatrix}
\begin{pmatrix}
y_1\\y_2
\end{pmatrix}
\]
erster Ordnung umwandelt, zeigt sich an der Stelle $x=0$ eine
Singularität in der Matrix, die Ableitung kann also für $x=0$
nicht bestimmt werden.
In einer Umgebung von $x=0$ erfüllt die Differentialgleichung
die Voraussetzungen bekannter Existenz- und Eindeutigkeitssätze
für gewöhnliche Differentialgleichungen nicht.

Ein ähnliches Problem tritt bei jeder hypergeometrischen
Differentialgleichung auf.
Diese werden gemäss Abschnitt
\ref{buch:differentialgleichungen:section:hypergeometrisch}
aus den Differentialoperatoren
\[
D_a=z\frac{d}{dz} + a
\]
zusammengesetzt.
Die Ableitung höchster Ordnung eines Produktes solcher Operationen ist
\[
D_{a_1}
\cdots
D_{a_p}
=
z^p\frac{d^p}{dz^p} + \text{Ableitungen niedrigerer Ordnung}.
\]
Dies zeigt, dass für $p>0$ oder $q>0$ ein Faktor $x$ bei der
Ableitung höchster Ordnung unvermeidlich ist, die Differentialgleichung
kann also wieder nicht nach dieser Ableitung aufgelöst werden und
erfüllt die Voraussetzungen der Existenz- und Eindeutigkeitssätze
in einer Umgebung von $x=0$ wieder nicht.

Die Besselsche Differentialgleichung
hat auch nicht die Form $y''+p(x)xy'+q(x)=0$, die der Theorie der 
Indexgleichung zugrunde lag.
Daher kann es auch keine Garantie geben, dass die Methode der
verallgemeinerten Potenzreihen zwei linear unabhängige Lösungen
liefern kann.
Tatsächlich wurde für ganzzahlige $n$ wegen $J_n(x) = (-1)^n J_{-n}(x)$
nur eine Lösung statt der erwarteten zwei linear unabhängigen
Lösungen gefunden.

Sind die Koeffizienten einer linearen Differentialgleichungen wie
in den genannten Beispielen singulär bei $x=0$, kann man auch nicht
erwarten, dass die Lösungen singulär sind.
Dies war schliesslich die Motivation, einen Lösungsansatz mit einer
verallgemeinerten Potenzreihe zu versuchen.
Mit den Funktion $x^\varrho$ lässt sich bereits eine recht grosse
Klasse von Singularitäten beschreiben, aber es ist nicht klar,
welche weiteren Arten von Singularitäten berücksichtigt werden sollten.
Dies soll im Folgenden geklärt werden.

%
% Der Lösungsraum einer Differentialgleichung zweiter Ordnung
%
\subsubsection{Der Lösungsraum einer Differentialgleichung zweiter Ordnung}
Eine Differentialgleichung $n$-ter Ordnung hat lokal einen $n$-dimensionalen
Vektorraum als Lösungsraum.

\begin{definition}
Sei 
\begin{equation}
\sum_{k=0}^n a_k(x) y^{(n)}(x) = 0
\label{buch:funktionentheorie:singularitaeten:eqn:defdgl}
\end{equation}
eine Differentialgleichung $n$-ter Ordnung mit analytischen Koeffizienten
und $x_0\in \mathbb{C}$.
Dann ist
\[
\mathbb{L}_{x_0}
=
\left\{
y(x)
\;\left|\;
\begin{minipage}{6cm}
$y$ ist Lösung der Differentialgleichung
\eqref{buch:funktionentheorie:singularitaeten:eqn:defdgl}
in einer Umgebung von $x_0$
\end{minipage}
\right.
\right\}
\]
der Lösungsraum der Differentialgleichung
\eqref{buch:funktionentheorie:singularitaeten:eqn:defdgl}.
Wenn der Punkt $x_0$ aus dem Kontext klar ist, kann er auch weggelassen
werden: $\mathbb{L}_{x_0}=\mathbb{L}$.
\end{definition}

%
% Analytische Fortsetzung auf dem Weg um 0
%
\subsubsection{Analytische Fortsetzung auf einem Weg um $0$}
Die betrachteten Differentialgleichungen haben holomorphe
Koeffizienten, Lösungen der Differentialgleichung lassen sich
daher immer in die komplexe Ebene fortsetzen, solange man die
Singularitäten der Koeffizienten vermeidet.
Hat eine Funktion $y(z)$ eine Laurent-Reihe
\[
y(z) = \sum_{k=-\infty}^\infty a_kz^k,
\]
dann ist sie automatisch in einer Umgebung von $0$ definiert
ausser in $0$.
Die analytische Fortsetzung entlang eines Pfades, der $0$
umschliesst, ist die Funktion $y(z)$ selbst.

Für die Wurzelfunktion $y(z)=z^{\frac1n}$ ist dies nicht möglich.
Die analytische Fortsetzung von $\sqrt[n]{x}$ auf der positiven reellen
Achse entlang einer Kurve, die $0$ umschliesst,
produziert die Funktion
\[
\sqrt[n]{z}
=
\sqrt[n]{re^{i\varphi}}
=
\sqrt[n]{r}e^{i\frac{\varphi}n},
\]
die für $\varphi=2\pi$ zu $e^{i\frac{2\pi}n}\sqrt{x}$ wird.
Verallgemeinerte Potenzreihen als Lösungen zeigen daher, dass
die analytische Fortsetzung der Lösung entlang eines Pfades um
eine Singularität nicht mit der Lösung übereinstimmen muss.
Das Studium dieser analytischen Fortsetzung dürfte daher zusätzliche
Informationen über die Lösung hervorbringen.

\begin{definition}
Der {\em Fortsetzungsoperator} $\sk$ ist der lineare Operator, der eine
in einem Punkt $x\in\mathbb{R}^+$ analytische Funktion $f(x)$ entlang eines
geschlossenen Weges fortsetzt, der $0$ im Gegenuhrzeigersinn umläuft.
Die Einschränkung der analytischen Fortsetzung auf $\mathbb{R}^+$ wird
mit $\sk f(x)$ bezeichnet.
\end{definition}

Die obengenannten Beispiele lassen sich mit dem Operator $\sk$ als
\[
\begin{aligned}
\sk z^n
&=
z^n
&\qquad& n \in \mathbb{Z}
\\
\sk
\sum_{k=-\infty}^\infty a_kz^k
&=
\sum_{k=-\infty}^\infty a_kz^k
\\
\sk z^\varrho
&=
e^{2\pi i\varrho} z^\varrho
\end{aligned}
\]
schreiben.

%
% Rechenregeln für die analytische Fortsetzung
%
\subsubsection{Rechenregeln für die analytische Fortsetzung}
Der Operator $\sk$ ist ein Algebrahomomorphismus, d.~h.~für zwei analytische
Funktionen $f$ und $g$ gilt
\[
\begin{aligned}
\sk(\lambda f + \mu g)
&=
\lambda \sk f  + \mu \sk g
\\
\sk(fg)
&=
(\sk f)(\sk g)
\end{aligned}
\]
für beliebige $\lambda,\mu\in\mathbb{C}$.
Ist $f$ eine in ganz $\mathbb{C}$ holomorphe Funktion, dann lässt sie
sich mit Hilfe einer Potenzreihe berechnen.
Der Wert $f(g(z))$ entsteht durch Einsetzen von $g(z)$ in die Potenzreihe.
Analytische Fortsetzung mit $\sk$ reproduziert jeden einzelnen Term
der Potenzreihe, es folgt
$\sk f(g(z)) = f(\sk g(z))$.
Ebenso folgt auch, dass der Operator $\sk$ mit der Ableitung
vertauscht, dass also
\[
\frac{d^n}{dz^n}(\sk f)
=
\sk(f^{(n)}).
\]

%
% Analytische Fortsetzung von Lösungen einer Differentialgleichung
%
\subsubsection{Analytische Fortsetzung von Lösungen einer Differentialgleichung}
Wir untersuchen jetzt die Wirkung des Operators $\sk$ auf
den Lösungsraum $\mathbb{L}$ einer Differentialgleichung mit
analytischen Koeffizienten, die in einer Umgebung von $0$
definiert sind.
Auf den Koeffizienten wirkt $\sk$ als die Identität. 
Ist $y(x)$ eine Lösung der Differentialgleichung, dann gilt
\[
0
=
\sk\biggl(
\sum_{k=0}^n a_k(x) y^{(n)}(x)
\biggr)
=
\sum_{k=0}^n (\sk a_k)(x) \cdot (\sk y)^{(n)}(x)
=
\sum_{k=0}^n a_k(x) \cdot (\sk y)^{(n)}(x),
\]
somit ist $\sk y$ ebenfalls eine Lösung.
Wir schliessen daraus, dass $\sk$ eine lineare Abbildung 
$\mathbb{L}\to\mathbb{L}$ ist.

Der Lösungsraum einer Differentialgleichung $n$-ter Ordnung
ist $n$-dimensional.
Nach Wahl einer Basis des Lösungsraums kann der Operator $\sk$
mit Hilfe einer Matrix $A\in M_{n\times n}(\mathbb{C})$ beschrieben werden.
Sei $\mathscr{W}=\{w_1,\dots,w_n\}$ eine Basis des Lösungsraums, dann
kann $\sk w_j$ wieder eine Lösung der Differentialgleichung
und kann daher geschrieben werden als Linearkombination
\begin{equation}
\sk w_j
=
\sum_{k=1}^n
a_{jk} w_k
\end{equation}
der Funktionen in $\mathscr{W}$.

Die Matrix $A$ mit den Einträgen $a_{jk}$ kann durch Wahl einer
geeigneten Basis in besonders einfache Form gebracht.
Wir führen diese Diskussion im folgenden nur für eine Differentialgleichung
zweiter Ordnung $n=2$.

%
% Fall A diagonalisierbar
%
\subsubsection{Fall $A$ diagonalisierbar: verallgemeinerte Potenzreihen}
In diesem Fall kann man die Lösungsfunktionen $w_1$ und $w_2$ so
wählen, dass die Matrix
\[
A=\begin{pmatrix}\lambda_1&0\\0&\lambda_2\end{pmatrix}
\]
diagonal wird mit Eigenwerten $\lambda_j$, $j=1,2$.
Dies bedeutet, dass $\sk w_j = \lambda_j w_j$.
Wir schreiben
\[
\varrho_j = \frac{1}{2\pi i} \log\lambda_j.
\]
Der Logarithmus ist nicht eindeutig, er ist nur bis auf ein Vielfaches
von $2\pi i$ bestimmt.
Folglich aus auch $\varrho_j$ nicht eindeutig bestimmt, eine
andere Wahl des Logarithmus ändert $\varrho_j$ aber um eine ganze Zahl.

Die Funktion $z^{\varrho_j}$ wird unter der Wirkung von $\sk$ zu
\[
\sk z^{\varrho_j}
=
e^{2\pi i\varrho_j} z^{\varrho_j}
=
e^{\log \lambda_j} z^{\varrho_j}
=
\lambda_j z^{\varrho_j}.
\]
Auf den Funktionen $z^{\varrho_j}$ und $w_j$ wirkt der Operator $\sk$
also die gleich durch Multiplikation mit $\lambda_j$.
Deren Quotient
\[
f(z) = \frac{w_j(z)}{z^{\varrho_j}}
\qquad\text{erfüllt}\qquad
\sk f
=
\frac{\sk w_j}{\sk z^{\varrho_j}}
=
\frac{\lambda_j w_j}{\lambda_j z^{\varrho_j}}
=
\frac{w_j}{z^{\varrho_j}}
=
f.
\]
Die Funktion $f$ kann daher als Laurent-Reihe
\[
f(z) 
=
\sum_{k=-\infty}^\infty a_kz^k
\]
geschrieben werden.
Die Lösung $w_2(z)$ muss daher die Form
\begin{equation}
w_j(z)
=
z^{\varrho_j} f(z)
=
z^{\varrho_j} \sum_{k=-\infty}^\infty a_kz^k
\end{equation}
haben, also die einer verallgemeinerten Potenzreihe.
Auch hier zeigt sich, dass die Wahl des Logarithmus in der Definition
von $\varrho_j$ unbedeutend ist, sie äussert sich nur in einer
Verschiebung der Koeffizienten $a_k$.

Falls der Operator $\sk$ also diagonalisierbar ist, dann gibt es
zwei linear unabhängige Lösungen der Differentialgleichung in der
Form einer verallgemeinerten Potenzreihe.

%
% Fall $A$ nicht diagonalisierbar
%
\subsubsection{Fall $A$ nicht diagonalisierbar: logarithmische Lösungen}
Falls die Matrix $A$ nicht diagonalisierbar ist, hat sie nur einen
Eigenwert $\lambda$ und kann durch geeignete Wahl einer Basis in
Jordansche Normalform
\[
A
=
\begin{pmatrix}
\lambda &    1    \\
   0    & \lambda
\end{pmatrix}
\]
gebracht werden.
Dies bedeutet, dass
\begin{align*}
\sk w_1 &= \lambda w_1 + w_2
\\
\sk w_2 &= \lambda w_2.
\end{align*}
Die Funktion $w_2$ hat unter $\sk$ die gleichen Eigenschaften
wie im diagonalisierbaren Fall, man kann also wieder schliessen,
dass $w_2$ durch eine verallgemeinerte Potenzreihe mit
\[
\varrho=\frac{1}{2\pi i} \log \lambda
\]
dargestellt werden kann.

Für den Quotienten $w_1/w_2$ findet man jetzt das Bild
\begin{equation}
\sk \frac{w_1}{w_2}
=
\frac{\sk w_1}{\sk w_2}
=
\frac{\lambda w_1+w_2}{\lambda w_2}
=
\frac{w_1}{w_2} + \frac{1}{\lambda}
\label{buch:funktionentheorie:singularitaeten:sklog}
\end{equation}
Das Verhalten von $w_1$ unter $\sk$ in
\eqref{buch:funktionentheorie:singularitaeten:sklog}
ist dasselbe wie bei $\log(z)/\lambda$, denn
\[
\sk \frac{\log(z)}{\lambda}
=
\frac{\log(z)}{\lambda} + 1.
\]
Die Differenz $w_1-\log(z)/\lambda$ wird bei der analytischen
Fortsetzung zu
\[
\sk\biggl(
\frac{w_1}{w_2}-\frac{\log(z)}{\lambda}
\biggr)
=
\sk \frac{w_1}{w_2} - \sk\frac{\log(z)}{\lambda}
=
\frac{w_1}{w_2} + \frac{1}{\lambda}
-
\frac{\log(z)}{\lambda}
-\frac{1}{\lambda}
=
\frac{w_1}{w_2}-\frac{\log(z)}{\lambda}.
\]
Die Differenz ist daher wieder als Laurent-Reihe
\[
\frac{w_1}{w_2}-\frac{\log(z)}{\lambda}
=
\sum_{k=-\infty}^\infty b_kz^k
\]
darstellbar, was nach $w_1$ aufgelöst 
\[
w_1(z)
=
\frac{1}{\lambda} \log(z) w_2(z)
+
w_2(z) \sum_{k=-\infty}^\infty b_kz^k
\]
ergibt.
Da $w_2$ eine verallgemeinerte Potenzreihe ist, kann man dies auch
als
\begin{equation}
w_1(z)
=
c \log(z) w_2(z)
+
z^{\varrho}
\sum_{k=-\infty}^{\infty} c_kz^k
\label{buch:funktionentheorie:singularitäten:eqn:w1}
\end{equation}
schreiben, wobei Konstanten $c$ und $c_k$ noch bestimmt werden müssen.
Setzt man
\eqref{buch:funktionentheorie:singularitäten:eqn:w1}
in die ursprüngliche Differentialgleichung ein, verschwindet der
$\log(z)$-Term und für die verbleibenden Koeffizienten kann die
bekannte Methode des Koeffizientenvergleichs verwendet werden.

%
% Bessel-Funktionen zweiter Art
%
\subsubsection{Bessel-Funktionen zweiter Art
\label{buch:funktionentheorie:subsubsection:bessel2art}}
Im Abschnitt~\ref{buch:differentialgleichungen:subsection:bessel1steart}
waren wir nicht in der Lage, für ganzahlige $\alpha$ zwei linear unabhängige
Lösungen der Besselschen Differentialgleichung zu finden.
Die vorangegangenen Ausführungen erklären dies: der Ansatz als
verallgemeinerte Potenzreihe konnte die Singularität nicht wiedergeben.
Inzwischen wissen wir, dass wir nach einer Lösung mit einer logarithmischen
Singularität suchen müssen.

Um dies nachzuprüfen, setzen wir den Ansatz
\[
y(x) = \log(x) J_n(x) + z(x)
\]
in die Besselsche Differentialgleichung ein.
Dazu benötigen wir erst die Ableitungen von $y(x)$:
\begin{align*}
y'(x)
&=
\frac{1}{x} J_n(x) + \log(x)J_n'(x) + z'(x)
\\
xy'(x)
&=
J_n(x) + x\log(x)J_n'(x) + xz'(x)
\\
y''(x)
&=
-\frac{1}{x^2} J_n(x)
+\frac2x J_n'(x)
+\log(x) J_n''(x)
+z''(x)
\\
x^2y''(x)
&=
-J_n(x) + 2xJ'_n(x)+x^2\log(x)J_n''(x) + x^2z''(x).
\end{align*}
Die Wirkung des Bessel-Operators auf $y(x)$ ist
\begin{align*}
By
&=
x^2y''+xy'+x^2y
\\
&=
\log(x) \bigl(
\underbrace{
x^2J_n''(x)
+xJ_n'(x)
+x^2J_n(x)
}_{\displaystyle = n^2J_n(x)}
\bigr)
-J_n(x)+2xJ_n'(x)
+J_n(x)
+
xz'(x)
+
x^2z''(x)
\\
&=
n^2 \log(x)J_n(x)
+
2xJ_n(x)
+
x^2z(x)
+
xz'(x)
+
x^2z''(x)
\end{align*}
Damit $y(x)$ eine Eigenfunktion zum Eigenwert $n^2$ wird, muss 
dies mit $n^2y(x)$ übereinstimmen, also
\begin{align*}
n^2 \log(x)J_n(x)
+
2xJ_n(x)
+
x^2z(x)
+
xz'(x)
+
x^2z''(x)
&=
n^2\log(x)J_n(x) + n^2z(x).
\intertext{Die logarithmischen Terme heben sich weg und es bleibt}
x^2z''(x)
+
xz'(x)
+
(x^2-n^2)z(x)
&=
-2xJ_n(x).
\end{align*}
Eine Lösung für $z(x)$ kann mit Hilfe eines Potenzreihenansatzes
gefunden werden.
Sie ist aber nur bis auf einen Faktor festgelegt.
Tatsächlich kann man aber auch eine direkte Definition geben.

\begin{definition}
Die Bessel-Funktionen zweiter Art der Ordnung $\alpha$ sind die Funktionen
\begin{equation}
Y_\alpha(x)
=
\frac{J_\alpha(x) \cos \alpha\pi  - J_{-\alpha}(x)}{\sin \alpha\pi }.
\label{buch:funktionentheorie:bessel:2teart}
\end{equation}
Für ganzzahliges $\alpha$ verschwindet der Nenner in 
\eqref{buch:funktionentheorie:bessel:2teart},
daher ist
\[
Y_n(x)
=
\lim_{\alpha\to n} Y_{\alpha}(x)
=
\frac{1}{\pi}\biggl(
\frac{d}{d\alpha}J_{\alpha}(x)\bigg|_{\alpha=n}
+
(-1)^n
\frac{d}{d\alpha}J_{\alpha}(x)\bigg|_{\alpha=-n}
\biggr).
\]
\end{definition}

Die Funktionen $Y_\alpha(x)$ sind Linearkombinationen der Lösungen
$J_\alpha(x)$ und $J_{-\alpha}(x)$ und damit automatisch auch Lösungen
der Besselschen Differentialgleichung.
Dies gilt auch für den Grenzwert im Falle ganzahliger Ordnung $\alpha$.
Da $J_{\alpha}(x)$ durch eine Reihenentwicklung definiert ist, kann man
diese Termweise nach $\alpha$ ableiten und damit auch eine 
Reihendarstellung von $Y_n(x)$ finden.
Nach einiger Rechnung findet man:
\begin{align*}
Y_n(x)
&=
\frac{2}{\pi}J_n(x)\log\frac{x}2
-
\frac1{\pi}
\sum_{k=0}^{n-1} \frac{(n-k-1)!}{k!}\biggl(\frac{x}2\biggr)^{2k-n}
\\
&\qquad\qquad
-
\frac1{\pi}
\sum_{k=0}^\infty \frac{(-1)^k}{k!\,(n+k)!}
\biggl(
\frac{\Gamma'(n+k+1)}{\Gamma(n+k+1)}
+
\frac{\Gamma'(k+1)}{\Gamma(k+1)}
\biggr)
\biggl(
\frac{x}2
\biggr)^{2k+n}
\end{align*}
(siehe auch \cite[p.~200]{buch:specialfunctions}).




\section{TODO}
\begin{itemize}
\item Aurgument-Prinzip
\end{itemize}

\section*{Übungsaufgaben}
\rhead{Übungsaufgaben}
\aufgabetoplevel{chapters/080-funktionentheorie/uebungsaufgaben}
\begin{uebungsaufgaben}
%\uebungsaufgabe{0}
\uebungsaufgabe{1}
\uebungsaufgabe{2}
\end{uebungsaufgaben}

