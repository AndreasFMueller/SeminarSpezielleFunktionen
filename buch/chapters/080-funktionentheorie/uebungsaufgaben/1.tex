Verwenden Sie die Eulersche Spiegelungsformel um 
\[
S_n
=
\sum_{k=1}^n
\Gamma\biggl(\frac{1+2k}2\biggr)\Gamma\biggl(\frac{1-2k}2\biggr)
\]
zu berechnen.

\begin{loesung}
Zunächst beachten wir, dass
\[
1 - \frac{1+2k}2
=
\frac{1-2k}2.
\]
Dies bedeutet, dass
\[
\Gamma\biggl(\frac{1+2k}2\biggr)
\Gamma\biggl(\frac{1-2k}2\biggr)
=
\Gamma\biggl(\frac{1+2k}2\biggr)
\Gamma\biggl(1-\frac{1+2k}2\biggr)
=
\frac{\pi}{
\sin\pi\frac{1+2k}2
}
=
\frac{\pi}{\sin(2k+1)\frac{\pi}2}
\]
nach der Eulerschen Spiegelungsformel.
Das Argument der Sinus-Funktion ist ein ungerades Vielfaches 
von $\frac{\pi}2$, die Sinus-Funktion hat dort die Werte $\pm 1$,
genauer
\[
\sin(2k+1)\frac{\pi}2
=
(-1)^k.
\]
Damit wird die gesuchte Summe:
\[
S_n
=
\sum_{k=1}^n
\frac{\pi}{(-1)^k}
=
-\pi+\pi-\pi+\dots+(-1)^n\pi
=
\begin{cases}
0&\qquad\text{$n$ gerade}\\
-\pi&\qquad\text{$n$ ungerade}.
\end{cases}
\qedhere
\]
\end{loesung}
