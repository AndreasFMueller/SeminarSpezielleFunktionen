Verwenden Sie die Legendresche Verdoppelungsformel und
die Eulersche Spiegelungsformel für die Gamma-Funktion,
um $\Gamma(\frac14)\Gamma(\frac34)$ zu berechnen und
verifizieren Sie, dass beide Wege das gleiche Resultat geben.

\begin{loesung}
Aus der Spiegelungsformel für $x=\frac14$ folgt
\[
\Gamma({\textstyle\frac14})\Gamma({\textstyle\frac34})
=
\frac{\pi}{\sin\frac{\pi}4}
=
\frac{\pi}{1/\!\sqrt{\mathstrut 2}}
=
\pi\sqrt{\mathstrut 2}.
\]
Andererseits ist $\frac34=\frac14+\frac12$, so dass aus der Legendreschen
Verdoppelungsformel folgt
\[
\Gamma({\textstyle\frac14})\Gamma({\textstyle\frac34})
=
2^{1-2\cdot \frac14}\sqrt{\mathstrut\pi}\Gamma(2\cdot {\textstyle\frac14})
=
\sqrt{\mathstrut 2}
\sqrt{\mathstrut \pi}\Gamma({\textstyle\frac12})
=
\sqrt{\mathstrut 2}
\pi.
\]
Offensichtlich stimmen die beiden Resultate überein.
\end{loesung}
