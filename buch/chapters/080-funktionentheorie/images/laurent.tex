%
% laurent.tex -- Laurent-Reihen und Cauchy-Integral
%
% (c) 2021 Prof Dr Andreas Müller, OST Ostschweizer Fachhochschule
%
\documentclass[tikz]{standalone}
\usepackage{amsmath}
\usepackage{times}
\usepackage{txfonts}
\usepackage{pgfplots}
\usepackage{csvsimple}
\usetikzlibrary{arrows,intersections,math,calc,decorations.markings}
\begin{document}
\def\skala{1}
\begin{tikzpicture}[>=latex,thick,scale=\skala]

\coordinate (Z0) at (1.5,1);

\def\d{0.03}

\draw[->] (-2.1,0) -- (4.8,0) coordinate[label={$\operatorname{Re}z$}];
\draw[->] (0,-2.1) -- (0,4.3) coordinate[label={$\operatorname{Im}z$}];

\fill[color=red!20,opacity=0.7] (Z0) circle[radius=2.5];

\fill[color=white] (Z0) circle[radius=0.2];


\draw[color=red] (Z0) circle[radius=2.5];
\draw[color=red] (Z0) circle[radius=0.2];
\fill[color=white] ($(Z0)+(0,-\d)$) rectangle ($(Z0)+(3,\d)$);
\begin{scope}[decoration={
    markings,
    mark=at position 0.5 with {\arrow{>}}}
    ]
\draw[color=red,postaction={decorate}]
	($(Z0)+({asin(-\d/2.5)}:2.5)$)
	--
	($(Z0)+({asin(-\d/0.2)}:0.2)$);
\draw[color=red,postaction={decorate}]
	($(Z0)+({asin(\d/0.2)}:0.2)$)
	--
	($(Z0)+({asin(\d/2.5)}:2.5)$);
\end{scope}

\draw (Z0) circle[radius=0.05];
\node at ($(Z0)+(-0.1,-0.1)$) [below left] {$z_0$};

\node[color=red] at (2.75,1) [above] {$l_1$};
\node[color=red] at (2.75,1) [below] {$l_2$};

\node[color=red] at ($(Z0)+(45:2.5)$) [above right] {$\gamma_1$};
\node[color=red] at ($(Z0)+(0,0.2)$) [above] {$\gamma_2$};

\end{tikzpicture}
\end{document}

