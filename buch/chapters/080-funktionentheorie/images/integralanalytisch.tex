%
% integralanalytisch.tex -- Illustration zum Beweis, das das Cauchy-Integral
%                           auf eine analytische Funktion führt
%
% (c) 2021 Prof Dr Andreas Müller, OST Ostschweizer Fachhochschule
%
\documentclass[tikz]{standalone}
\usepackage{amsmath}
\usepackage{times}
\usepackage{txfonts}
\usepackage{pgfplots}
\usepackage{csvsimple}
\usetikzlibrary{arrows,intersections,math}
\begin{document}
\def\skala{1}
\begin{tikzpicture}[>=latex,thick,scale=\skala]

\fill[color=blue!20] (0,0) circle[radius=1.5];
\draw[color=blue,line width=0.7pt] (0,0) circle[radius=1.5];

\draw[->] (0,0) -- (-150:1.5);
\node at (-150:1.0) [below] {$\varrho$};

\begin{scope}
\clip (-4,-3) rectangle (4,3);
\draw[color=red, line width=1.4pt]
	 (-3,0.5) 
	.. controls (-4,-0.5) and (-3,-2) ..
	(-2,-2)
	.. controls (-1,-2) and (-1,-1.5) ..
	(0,-1.5)
	.. controls (1.0,-1.5) and (1.0,-3) ..
	(2,-3)
	.. controls (5,-3) and (3,5) ..
	(-1,2);
\end{scope}

\node[color=red] at (3.2,-1.5) {$\gamma$};

\coordinate (Z) at (1,0.5);

\fill[color=white] (Z) circle[radius=0.05];
\draw (Z) circle[radius=0.05];
\node at (Z) [above] {$z$};

\draw[->] (0,-3.1) -- (0,3.3) coordinate[label={left:$\operatorname{Im}z$}];
\draw[->] (-4.1,0) -- (4.3,0) coordinate[label={$\operatorname{Re}z$}];

\end{tikzpicture}
\end{document}

