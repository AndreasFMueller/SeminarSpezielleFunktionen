%
% xexpx.tex -- template for standalon tikz images
%
% (c) 2021 Prof Dr Andreas Müller, OST Ostschweizer Fachhochschule
%
\documentclass[tikz]{standalone}
\usepackage{amsmath}
\usepackage{times}
\usepackage{txfonts}
\usepackage{pgfplots}
\usepackage{csvsimple}
\usetikzlibrary{arrows,intersections,math}
\begin{document}
\def\skala{1.5}
\begin{tikzpicture}[>=latex,thick,scale=\skala]

\draw[->] (-6.1,0) -- (2.0,0) coordinate[label={$x$}];
\draw[->] (0,-1.1) -- (0,5.2) coordinate[label={right:$y$}];

\foreach \x in {-6,-5,-4,-3,-2,-1}{
	\draw (\x,{-0.1/\skala}) -- (\x,{0.1/\skala});
	\node at (\x,0) [above] {$\x$};
}
\draw (1,{-0.1/\skala}) -- (1,{0.1/\skala});
\node at (1,0) [below] {$1$};
\foreach \y in {1,...,5}{
	\draw ({-0.1/\skala},\y) -- ({0.1/\skala},\y);
	\node at (0,\y) [left] {$\y$};
}

\begin{scope}
\clip (-6,-0.5) rectangle (2,4.8);

\draw[color=red,line width=1.4pt]
	plot[domain=-1:2,samples=100] ({\x},{\x*exp(\x)});
\draw[color=blue,line width=1.4pt]
	plot[domain=-6:-1,samples=100] ({\x},{\x*exp(\x)});

\end{scope}

\fill[color=violet] (-1,{-exp(-1)}) circle[radius=0.04];
\node[color=violet] at (-1,{-exp(-1)}) [below]
	{$\displaystyle f(-1) = -\frac{1}{e}$};

\end{tikzpicture}
\end{document}

