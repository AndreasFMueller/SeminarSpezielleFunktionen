%
% 1-zins.tex
%
% (c) 2021 Prof Dr Andreas Müller, OST Ostscheizer Fachhochschule
%
\section{Exponentialfunktion 
\label{buch:exponential:section:grenzwert}}
\kopfrechts{Exponentialfunktion als Grenzwert}
Mit Hilfe von Potenzen und Wurzeln lassen sich die Potenzen $a^x$
für beliebige rationale Zahlen $x=p/q\in\mathbb{Q}$ als
\[
a^x = a^{\frac{p}{q}} = \root{q}\of{a^p}
\]
definieren.
Da $x\mapsto a^x$ stetig ist, ergibt sich daraus auch eine
stetige Funktion
$a^{\bullet}\colon \mathbb{R}\to\mathbb{R}:x\mapsto a^x$.
Dies ist aber als Basis für eine neue spezielle Funktion nicht
wirklich geeignet, da ausser $x$ auch die Basis variert werden kann.
Die arithmetischen Eigenschaften der Potenzfunktion erlauben aber,
jede der Funktionen $a^x$ auf jede andere $b^x$ zurückzuführen.
Ist $b=a^t$, dann dann ist $b^x = a^{tx}$.
Es stellt sich damit die Frage, ob es eine bevorzugte Basis gibt.

%
% Zins und eulerscher Grenzwert
%
\subsection{Zins und Eulerscher Grenzwert}
Wir ein Kapital $K_0$ mit dem Jahreszinssatz $x$ verzinst,
wächst es jedes Jahr um den Faktor $1+x$ an.
Teilt man die Zinsperiode in kleiner Intervall, zum Beispiel Monate
oder Tage, und passt auch den Zins entsprechend an, dann wächste
das Kapitel in einem Jahr auf
\[
K = \biggl(1+\frac{x}{12}\biggr)^{12}
\qquad\text{und}\qquad
K = \biggl(1+\frac{x}{365}\biggr)^{365}
\]
an.
Für eine Unterteilung in $n$ Zinsperioden ist der Faktor also
\[
\biggl(1+\frac{x}{n}\biggr)^n.
\]
Diese Beobachtung hat Jacob Bernoulli 1683 dazu geführt, den Grenzwert
\[
\lim_{n\to\infty} \biggl(1+\frac1n\biggr)^n
\]
zu studieren, der später mit $e$ bezeichnet wurde.
Später hat Euler gezeigt, dass 
\begin{equation}
\lim_{n\to\infty}\biggl(1+\frac{x}{n}\biggr)^n
=
e^x
\label{buch:exponential:zins:eulerex}
\end{equation}
gilt.

Tatsächlich gilt für ganzzahlige $x$, dass auch die Teilfolge
mit $n=xm$ konvergiert, dass also
\begin{align*}
\lim_{n\to\infty}
\biggl(1+\frac{x}{n}\biggr)^n
&=
\lim_{m\to\infty}
\biggl(1+\frac{x}{xm}\biggr)^{xm}
=
\lim_{m\to\infty}\biggl(1+\frac{1}{m}\biggr)^{xm}
\intertext{sein muss.
Da die Funktion $a\mapsto a^x$ stetig ist, folgt weiter}
&=\biggl(\lim_{m\to\infty}\biggl(1+\frac1m\biggr)^m\biggr)^x.
\end{align*}
Ähnlich kann man für einen Bruch $x=p/q$ vorgehen.
Dazu berechnet man die $q$-te Potenz, wobei man wieder verwenden kann,
dass, die Funktion $a\mapsto a^q$ stetig ist.
So bekommt man
\begin{align*}
\biggl(
\lim_{n\to\infty}
\biggl(1+\frac{x}{n}\biggr)^n
\biggr)^q
&=
\lim_{n\to\infty}
\biggl(+\frac{p}{qn}\biggr)^{nq}
=
\lim_{m\to\infty}
\biggl(
1+\frac{p}{m}
\biggr)^m
=
e^p.
\end{align*}
Zieht man jetzt die $q$-te Wurzel, bekommt man
\[
\lim_{n\to\infty}\biggl(1+\frac{x}{n}\biggr)^n = e^{\frac{p}{q}}.
\]
Da auch die Potenzfunktion $x\mapsto a^x$ stetig ist, folgt schliesslich,
dass für beliebige reelle $x\in\mathbb{R}$ die
Formel~\eqref{buch:exponential:zins:eulerex} gilt.

%
% Approximation durch Jost Bürgi
%
\subsubsection{Approximation durch Jost Bürgi}
Jost Bürgi, Uhrmacher und Mathematiker aus Lichtensteig, Kanton St.~Gallen,
war einer der Erfinder der Logarithmen, für die er allerdings
noch keinen Namen hatte.
Er berechnete eine Tabelle aller Werte von
\[
10^8\cdot(1+10^{-4})^n,\qquad 0\le n\le 23027.
\]
Schreibt man
\[
(1+10^{-4})^n
=
\biggl(1+\frac{1}{10000}\biggr)^{1000\cdot n\cdot10^{-4}},
\]
dann erkennt man, dass Bürgi die Potenzen der Approximation
\[
\biggl(1+\frac{1}{1000}\biggr)^{1000}
=
2.7181459
\]
von $e \approx 2.7182818$ berechnet hat.
Die Wahl dieser Basis hat keine Auswirkungen auf die Genauigkeit
der Anwendung seiner Tabellen, da jede andere Basis genauso
geeignet ist.

%
% Störungen des Eulerschen Grenzwertes
%
\subsubsection{Störungen des Eulerschen Grenzwertes}
Der Grenzwert~\eqref{buch:exponential:zins:eulerex}
bleibt unverändert, wenn man den Term $x$ um einen zusätzlichen
Summanden $x_n$ modifiziert, der schnell genug gegen $0$ geht.

\begin{lemma}
\label{buch:exponential:zins:perturbedeulerlimit}
Sei $x_n$ eine Folge $x_n\in\mathbb{R}$, die gegen $0$ konvergiert.
Dann gilt
\[
\lim_{n\to\infty}\biggl(1+\frac{x+x_n}{n}\biggr)^n
=
\lim_{n\to\infty}\biggl(1+\frac{x}{n}\biggr)^n
=
e^x.
\]
\end{lemma}

\begin{proof}[Beweis]
Für $\varepsilon>0$ gibt es ein $N$ derart, dass
\( |x_n| < \varepsilon \)
für alle $n>N$.
Da 
\[
\biggl(
1+\frac{x-\varepsilon}{n}
\biggr)^n
<
\biggl(
1+\frac{x+x_n}{n}
\biggr)^n
<
\biggl(
1+\frac{x+\varepsilon}{n}
\biggr)^n
\]
ist,
folgt
\[
e^{x-\varepsilon}
\le
\lim_{n\to\infty}
\biggl(
1+\frac{x+x_n}{n}
\biggr)^n
\le
e^{x+\varepsilon}.
\]
Da dies für alle $\varepsilon$ gilt und die Funktion $x\mapsto e^x$
stetig ist, folgt
\[
\lim_{n\to\infty} \biggl(1+\frac{x+x_n}{n}\biggr)^n
=
e^x,
\]
die Behauptung des Lemmas.
\end{proof}

%
% Funktionalgleichung
%
\subsubsection{Funktionalgleichung}
Die Definition der Exponentialfunktion als Potenz $e^x$
hat automatisch zur Folge, dass für beliebige reelle Zahlen
die Funktionalgleichung
\[
e^x\cdot e^y
=
e^{x+y}
\]
gilt.
Eine {\em Funktionalgleichung} setzt Werte einer Funktion für verschiedene
Argumente zueinander in Bezug.
\index{Funktionalgleichung}%
Die Funktionalgleichung der Exponentialfunktion kann auch direkt aus dem
Grenzwert~\eqref{buch:exponential:zins:eulerex}
abgeleitet werden.
Dazu rechnet man
\begin{align*}
\lim_{n\to\infty}\biggl(1+\frac{x}{n}\biggr)^n
\cdot
\lim_{m\to\infty}\biggl(1+\frac{x}{m}\biggr)^m
&=
\lim_{n\to\infty}
\biggl(
\biggl(1+\frac{x}{n}\biggr)
\biggl(1+\frac{y}{n}\biggr)
\biggr)^n
\\
&=
\lim_{n\to\infty}
\biggl( 1+\frac{x+y}{n}+\frac{xy}{n^2} \biggr)^n
\\
&=
\lim_{n\to\infty}
\biggl( 1+\frac{x+y+xy/n}{n}\biggr)^n.
\intertext{Der Term $x_n=xy/n$ konvergiert gegen $0$, daher ist nach dem
Lemma~\ref{buch:exponential:zins:perturbedeulerlimit}
}
&=
e^{x+y}.
\end{align*}
Damit ist die Funktionalgleichung bewiesen.

%
% Potenzreihe
%
\subsection{Potenzreihe}
Die übliche Definition der Exponentialfunktion verwendet eine Potenzreihe.
Zur Unterscheidung schreiben wir in diesem Abschnitt $\exp(x)$ für
die Exponentialreihe, bis wir gezeigt haben, dass sie mit der Potenz
$e^x$ übereinstimmt.

\begin{definition}
\label{buch:exponential:zins:exppotenzreihe}
Die Potenzreihe
\[
\exp(x)
=
\sum_{k=0}^\infty \frac{x^k}{k!}
\]
definiert eine Funktion $\exp\colon \mathbb{C}\to\mathbb{C}$.
\end{definition}

%
% Funktionalgleichung
%
\subsubsection{Funktionalgleichung}
Auch für die Potenzreihendefinition lässt sich die Funktionalgleichung
direkt verifizieren.
Das Produkt von $\exp(x)$ und $\exp(y)$ ist
\begin{align*}
\exp(x)\cdot\exp(y)
&=
\sum_{k=0}^\infty \frac{x^k}{k!} 
\cdot
\sum_{l=0}^\infty \frac{y^l}{l!} .
\intertext{Fasst man die Terme vom Grad $n$ zusammen, erhält man}
&=
\sum_{n=0}^\infty
\sum_{k=0}^n
\frac{1}{k!(n-k)!}
x^ky^{n-k}.
\intertext{Durch Erweitern mit $n!$ wird daraus}
&=
\sum_{n=0}^\infty
\frac{1}{n!}
\sum_{k=0}^n
\frac{n!}{k!(n-k)!}
x^ky^{n-k}.
\intertext{Der Quotient von Fakultäten ist der Binomialkoeffizient, so
dass die Summe mit dem Binomialsatz vereinfacht werden kann:}
&=
\sum_{n=0}^\infty
\frac{1}{n!}
\sum_{k=0}^n
\binom{n}{k}
x^ky^{n-k}
=
\sum_{n=0}^\infty
\frac{1}{n!}
(x+y)^n
=
\exp(x+y),
\end{align*}
damit ist die Funktionalgleichung nachgewiesen und es wird klar, dass
$\exp(x)$ eine Funktion der Form $a^x$ ist.

%
% exp(x) und e^x
%
\subsubsection{$\exp(x)$ und $e^x$}
Die Tatsache, dass $\exp(x)$ die Funktionalgleichung erfüllt, reicht
nicht aus um zu zeigen, dass $\exp(x)$ und $e^x$ dasselbe sind,
da jede beliebige Funktion $a^x$ diese Eigenschaft hat.
Wir können nur schliessen, dass $\exp(x)=\exp(1)^x$.
Wenn wir zeigen wollen, dass $\exp(x)$ und $e^x$ dasselbe sind, dann
müssen wir zeigen, dass $e=\exp(1)$ gilt.
Dazu formen wir den Eulerschen Grenzwert wie folgt um:
\begin{align*}
e=\biggl(1+\frac1n\biggr)^n
&=
\sum_{k=0}^n \binom{n}{k} \frac{1}{n^{n-k}}
=
\sum_{k=0}^n \frac{1}{k!} \frac{n(n-1)\cdots(n-k+1)}{n^{n-k}}
\\
&=
\sum_{k=0}^n \frac{1}{k!}
\underbrace{\frac{n}{n}}_{\displaystyle \downarrow\atop\displaystyle 1}
\cdot
\underbrace{\frac{n-1}{n}}_{\displaystyle\downarrow\atop\displaystyle 1}
\cdots
\underbrace{\frac{n-k+1}{n}}_{\displaystyle\downarrow\atop\displaystyle 1}
\to
\sum_{k=0}^\infty \frac{1}{k!}
=
\exp(1).
\end{align*}
Damit ist gezeigt, dass $e=\exp(1)$ und damit auch $e^x=\exp(x)$ ist.


