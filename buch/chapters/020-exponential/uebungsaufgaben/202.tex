Man finde $x$ derart, dass $(\tan x)^{\tan x}=2$

\begin{loesung}
Zunächst setzen wir $y=\tan x$, dann wird die Gleichung zu $y^y = 2$.
Der Logarithmus davon ist $y\log y = \log 2$.
Mit der Bezeichnung $t=\log y$ wird daraus die Gleichung
\[
te^t = \log 2,
\]
die mit der Lambert-$W$-Funktion gelöst werden kann, die Lösung ist
$t=W(\log 2)$.
Darus kann man jetzt wieder $y=e^t=e^{W(\log 2)}$ bekommen.
So finden wir die Lösung
$x = \arctan e^{W(\log 2)}\approx 1.00064239632968$.
Durch Addition von ganzzahligen Vielfachen von $\pi$ erhält man
weitere Lösungen.
\end{loesung}
