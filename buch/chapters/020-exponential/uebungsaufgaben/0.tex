Man finde $x\in\mathbb{R}$ derart, dass $3^x=2x+2$.

\begin{loesung}
Die Definition der $W$-Funktion verwendet die Exponentialfunktion,
wir schreiben daher zunächst $3^x = e^{x\log 3}$ und erhalten so
die Gleichung
\begin{align*}
e^{x\log 3} &= 2x+2
\\
\frac{1}{3}e^{(x+1)\log 3}
&=2(x+1)
\\
\frac{\log 3}{2\cdot 3}e^{(x+1)\log 3}
&=\log 3(x+1)
=
X
\\
-\frac{\log 3}{6}
&=
-Xe^{-X}.
\end{align*}
Auf der rechten Seite steht ein Ausdruck der Form $ze^z$, der mit der
$W$-Funktion invertiert werden kann, es ist also
\begin{align*}
W\biggl(
-\frac{\log 3}{6}
\biggr)
&=
-X
\qquad\Rightarrow\qquad
X=
-W\biggl(
-\frac{\log 3}{6}
\biggr)
=
(x+1)
\log 3
\end{align*}
Durch Auflösen nach $x$ findet man
\[
x
=
-1
-
\frac{1}{\log 3}
W\biggl(
-\frac{\log 3}{6}
\biggr).
\]
Die numerische Auswertung mit $W_0$ und $W_{-1}$ liefert zwei mögliche
Lösungen, nämlich
\[
x
=
\begin{cases}
\displaystyle -1-\frac{1}{\log 3} W_0\biggl(-\frac{\log 3}{6}\biggr)&=-0.79011\\
\displaystyle -1-\frac{1}{\log 3} W_{-1}\biggl(-\frac{\log 3}{6}\biggr)&=\phantom{-}1.44456.
\end{cases}
\]
Beide Lösungen kann man leicht durch Einsetzen überprüfen.
\end{loesung}
