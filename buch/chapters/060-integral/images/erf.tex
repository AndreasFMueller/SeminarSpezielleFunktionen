%
% erf.tex -- Fehlerfunktion
%
% (c) 2021 Prof Dr Andreas Müller, OST Ostschweizer Fachhochschule
%
\documentclass[tikz]{standalone}
\usepackage{amsmath}
\usepackage{times}
\usepackage{txfonts}
\usepackage{pgfplots}
\usepackage{csvsimple}
\usetikzlibrary{arrows,intersections,math}
\begin{document}
\def\skala{2}
\input{erfpunkte.tex}
\begin{tikzpicture}[>=latex,thick,scale=\skala]

\draw[->] (-3.2,0) -- (3.3,0) coordinate[label={$x$}];
\draw[->] (0,-1.1) -- (0,1.2) coordinate[label={right:$\operatorname{erf}(x)$}];
\foreach \x in {-3,-2,-1,1,2,3}{
	\draw[line width=1pt] (\x,-0.03) -- (\x,0.03);
	\node at (\x,-0.02) [below] {$\x$};
}
\draw[color=gray,line width=1pt] (-3.1,-1) -- (3.1,-1);
\node at (0,-1) [above left] {$-1$};
\draw[color=gray,line width=1pt] (-3.1,1) -- (3.1,1);
\node at (0,1) [below left] {$1$};

\draw[color=red,line width=1.4pt]
\erfpfad
;

\end{tikzpicture}
\end{document}

