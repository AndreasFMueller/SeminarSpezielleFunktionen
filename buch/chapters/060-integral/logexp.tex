%
% logexp.tex
%
% (c) 2022 Prof Dr Andreas Müller, OST Ostschweizer Fachhochschlue
%
\subsection{Log-Exp-Notation für trigonometrische und hyperbolische Funktionen
\label{buch:integral:subsection:logexp}}
Die Integration rationaler Funktionen hat bereits gezeigt, dass
eine Stammfunktion nicht immer im Körper der rationalen Funktionen
existiert.
Es kann notwendig sein, dem Körper logarithmische Erweiterungen der Form
$\log(x-\alpha)$ hinzuzufügen.

Es können jedoch noch ganz andere neue Funktionen auftreten, wie die
folgende Zusammenstellung einiger Stammfunktionen zeigt:
\begin{equation}
\begin{aligned}
\int\frac{dx}{1+x^2}
&=
\arctan x,
\\
\int \cos x\,dx
&=
\sin x,
\\
\int\frac{dx}{\sqrt{1-x^2}}
&=
\arcsin x,
\\
\int
\operatorname{arcosh} x\,dx
&=
x \operatorname{arcosh} x - \sqrt{x^2-1}.
\end{aligned}
\label{buch:integration:risch:allgform}
\end{equation}
In der Stammfunktion treten Funktionen auf, die auf den ersten
Blick nichts mit den Funktionen im Integranden zu tun haben.

\subsubsection{Trigonometrische und hyperbolische Funktionen}
Die trigonometrischen und hyperbolichen Funktionen
in~\eqref{buch:integration:risch:allgform}
lassen sich alle durch Exponentialfunktionen ausdrücken.
So gilt
\begin{equation}
\begin{aligned}
\sin x &= \frac{1}{2i}\bigl( e^{ix} - e^{-ix}\bigr),
&
&\qquad&
\cos x &= \frac{1}{2}\bigl( e^{ix} + e^{-ix}\bigr),
\\
\sinh x &= \frac12\bigl( e^x - e^{-x} \bigr),
&
&\qquad&
\cosh x &= \frac12\bigl( e^x + e^{-x} \bigr).
\end{aligned}
\label{buch:integral:risch:trighyp}
\end{equation}
Nach Multiplikation mit $e^{ix}$ bzw.~$e^{x}$ entsteht eine
quadratische Gleichung in $e^{ix}$ bzw.~$e^{x}$.
Die Lösungsformel für quadratische Gleichungen erlaubt daher, $e^{ix}$
bzw.~$e^{x}$ zu finden und damit auch die Umkehrfunktionen.
Die Rechnung ergibt
\begin{equation}
\begin{aligned}
\arcsin y
&=
\frac{1}{i}\log\bigl(
iy\pm\sqrt{1-y^2}
\bigr),
&
&\qquad&
\arccos y
&=
\log\bigl(
y\pm \sqrt{y^2-1}
\bigr),
\\
\operatorname{arsinh}y
&=
\log\bigl(
y \pm \sqrt{1+y^2}
\bigr),
&
&\qquad&
\operatorname{arcosh} y
&=
\log\bigl(
y\pm \sqrt{y^2-1}
\bigr).
\end{aligned}
\label{buch:integral:risch:trighypinv}
\end{equation}
Alle Funktionen, die man aus dem elementaren Analysisunterricht
kennt, können also mit Hilfe von Exponentialfunktionen und Logarithmen
geschrieben werden.
Man nennt dies die $\log$-$\exp$-Notation der trigonometrischen
und hyperbolischen Funktionen.
\index{logexpnotation@$\log$-$\exp$-Notation}%

\subsubsection{$\log$-$\exp$-Notation}
Wendet man die Substitutionen
\eqref{buch:integral:risch:trighyp}
und
\eqref{buch:integral:risch:trighypinv}
auf die Integrale
\eqref{buch:integration:risch:allgform}
an, entstehen die Beziehungen
\begin{equation}
\begin{aligned}
\int\frac{1}{1+x^2}
&=
\frac12i\bigl(
\log(1-ix) - \log(1+ix)
\bigr),
\\
\int\bigl(
{\textstyle\frac12}
e^{ix}
+
{\textstyle\frac12}
e^{-ix}
\bigr)
&=
-{\textstyle\frac12}ie^{ix}
+{\textstyle\frac12}ie^{-ix},
\\
\int
\frac{1}{\sqrt{1-x^2}}
&=
-i\log\bigl(ix+\sqrt{1-x^2}),
\\
\int \log\bigl(x+\sqrt{x^2-1}\bigr)
&=
x\log\bigl(x+\sqrt{x^2-1}\bigr) - \sqrt{x^2-1}.
\end{aligned}
\label{buch:integration:risch:eqn:integralbeispiel2}
\end{equation}
Die in den Stammfuntionen auftretenden Funktionen treten entweder
schon im Integranden auf oder sind Logarithmen von solchen
Funktionen.
Zum Beispiel hat der Nenner im ersten Integral die Faktorisierung
$1+x^2=(1+ix)(1-ix)$, in der Stammfunktion findet man die Logarithmen
der Faktoren.


