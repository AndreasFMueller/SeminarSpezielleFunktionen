%
% orthogonal.tex
%
% (c) 2021 Prof Dr Andreas Müller, OST Ostschweizer Fachhochschule
%
\section{Orthogonale Polynome
\label{buch:integral:section:orthogonale-polynome}}
Die Fourier-Theorie basiert auf der Idee, Funktionen durch 
Funktionenreihen mit Summanden zu bilden, die im Sinne eines
Skalarproduktes orthogonal sind, welches mit Hilfe eines Integrals
definiert sind.
Solche Funktionenfamilien treten jedoch auch als Lösungen von
Differentialgleichungen.
Besonders interessant wird die Situation, wenn die Funktionen 
Polynome sind.

\subsection{Skalarprodukt}
\subsection{Definition}
\subsection{Orthogonale Polynome und Differentialgleichungen}
\subsubsection{Legendre-Differentialgleichung}
\subsubsection{Legendre-Polyome}
\subsubsection{Legendre-Funktionen zweiter Art}
Siehe Wikipedia-Artikel \url{https://de.wikipedia.org/wiki/Legendre-Polynom}
\subsection{Anwendung: Gauss-Quadratur}

