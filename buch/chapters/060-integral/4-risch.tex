%
% 4-risch.tex
%
% (c) 2021 Prof Dr Andreas Müller, OST Ostschweizer Fachhochschule
%
\section{Der Risch-Algorithmus
\label{buch:integral:section:risch}}
\kopfrechts{Risch-Algorithmus}
Die Lösung des Integrationsproblem für $\mathbb{Q}(x)$ und für
$\mathbb{Q}(x,y)$ mit $y=\!\sqrt{ax^2+bx+c}$ hat gezeigt, dass
ein Differentialkörper genau die richtige Bühne für dieses Unterfangen
sein dürfte.
Die Stammfunktionen konnten in einem Erweiterungskörper gefunden
werden, dem ein paar Logarithmen hinzugefügt worden sind.
Tatsächlich lässt sich in diesem Rahmen sogar ein Algorithmus
formulieren, der in einem noch zu definierenden Sinn ``elementare''
Funktionen als Stammfunktionen finden kann oder beweisen kann, dass
eine solche nicht existiert.
Dieser Abschnitt soll einen Überblick darüber geben.

%
% 41-logexp.tex
%
% (c) 2022 Prof Dr Andreas Müller, OST Ostschweizer Fachhochschlue
%
\subsection{Log-Exp-Notation für trigonometrische und hyperbolische Funktionen
\label{buch:integral:subsection:logexp}}
Die Integration rationaler Funktionen hat bereits gezeigt, dass
eine Stammfunktion nicht immer im Körper der rationalen Funktionen
existiert.
Es kann notwendig sein, dem Körper logarithmische Erweiterungen der Form
$\log(x-\alpha)$ hinzuzufügen.

Es können jedoch noch ganz andere neue Funktionen auftreten, wie die
folgende Zusammenstellung einiger Stammfunktionen zeigt:
\begin{equation}
\begin{aligned}
\int\frac{dx}{1+x^2}
&=
\arctan x,
\\
\int \cos x\,dx
&=
\sin x,
\\
\int\frac{dx}{\sqrt{1-x^2}}
&=
\arcsin x,
\\
\int
\operatorname{arcosh} x\,dx
&=
x \operatorname{arcosh} x - \sqrt{x^2-1}.
\end{aligned}
\label{buch:integration:risch:allgform}
\end{equation}
In der Stammfunktion treten Funktionen auf, die auf den ersten
Blick nichts mit den Funktionen im Integranden zu tun haben.

\subsubsection{Trigonometrische und hyperbolische Funktionen}
Die trigonometrischen und hyperbolichen Funktionen
in~\eqref{buch:integration:risch:allgform}
lassen sich alle durch Exponentialfunktionen ausdrücken.
So gilt
\begin{equation}
\begin{aligned}
\sin x &= \frac{1}{2i}\bigl( e^{ix} - e^{-ix}\bigr),
&
&\qquad&
\cos x &= \frac{1}{2}\bigl( e^{ix} + e^{-ix}\bigr),
\\
\sinh x &= \frac12\bigl( e^x - e^{-x} \bigr),
&
&\qquad&
\cosh x &= \frac12\bigl( e^x + e^{-x} \bigr).
\end{aligned}
\label{buch:integral:risch:trighyp}
\end{equation}
Nach Multiplikation mit $e^{ix}$ bzw.~$e^{x}$ entsteht eine
quadratische Gleichung in $e^{ix}$ bzw.~$e^{x}$.
Die Lösungsformel für quadratische Gleichungen erlaubt daher, $e^{ix}$
bzw.~$e^{x}$ zu finden und damit auch die Umkehrfunktionen.
Die Rechnung ergibt
\begin{equation}
\begin{aligned}
\arcsin y
&=
\frac{1}{i}\log\bigl(
iy\pm\sqrt{1-y^2}
\bigr),
&
&\qquad&
\arccos y
&=
\log\bigl(
y\pm \sqrt{y^2-1}
\bigr),
\\
\operatorname{arsinh}y
&=
\log\bigl(
y \pm \sqrt{1+y^2}
\bigr),
&
&\qquad&
\operatorname{arcosh} y
&=
\log\bigl(
y\pm \sqrt{y^2-1}
\bigr).
\end{aligned}
\label{buch:integral:risch:trighypinv}
\end{equation}
Alle Funktionen, die man aus dem elementaren Analysisunterricht
kennt, können also mit Hilfe von Exponentialfunktionen und Logarithmen
geschrieben werden.
Man nennt dies die $\log$-$\exp$-Notation der trigonometrischen
und hyperbolischen Funktionen.
\index{logexpnotation@$\log$-$\exp$-Notation}%

\subsubsection{$\log$-$\exp$-Notation}
Wendet man die Substitutionen
\eqref{buch:integral:risch:trighyp}
und
\eqref{buch:integral:risch:trighypinv}
auf die Integrale
\eqref{buch:integration:risch:allgform}
an, entstehen die Beziehungen
\begin{equation}
\begin{aligned}
\int\frac{1}{1+x^2}
&=
\frac12i\bigl(
\log(1-ix) - \log(1+ix)
\bigr),
\\
\int\bigl(
{\textstyle\frac12}
e^{ix}
+
{\textstyle\frac12}
e^{-ix}
\bigr)
&=
-{\textstyle\frac12}ie^{ix}
+{\textstyle\frac12}ie^{-ix},
\\
\int
\frac{1}{\sqrt{1-x^2}}
&=
-i\log\bigl(ix+\sqrt{1-x^2}),
\\
\int \log\bigl(x+\sqrt{x^2-1}\bigr)
&=
x\log\bigl(x+\sqrt{x^2-1}\bigr) - \sqrt{x^2-1}.
\end{aligned}
\label{buch:integration:risch:eqn:integralbeispiel2}
\end{equation}
Die in den Stammfuntionen auftretenden Funktionen treten entweder
schon im Integranden auf oder sind Logarithmen von solchen
Funktionen.
Zum Beispiel hat der Nenner im ersten Integral die Faktorisierung
$1+x^2=(1+ix)(1-ix)$, in der Stammfunktion findet man die Logarithmen
der Faktoren.



\input{chapters/060-integral/42-elementar.tex}
%
% liouville.tex
%
% (c) 2022 Prof Dr Andreas Müller, OST Ostschweizer Fachhochschule
%
\subsection{Das Prinzip von Liouville
\label{buch:integral:risch:subsection:liouville}}
Der erste Schritt in Richtung auf einen Algorithmus zur Lösung des
Integrationsproblems ist ein Kriterium dafür, ob eine Stammfunktion
einer Funktion $f\in\mathscr{K}$ existiert.
Der folgende Satz liefert ein solches Kriterium.

\begin{satz}[Prinzip von Liouville]
\label{buch:integral:risch:satz:liouville}
\index{Prinzip von Liouville}%
Sei $\mathscr{K}$ ein Differentialkörper mit dem Konstantenkörper $C$.
Weiter sei $\mathscr{G}$ eine Differentialkörpererweiterung von $\mathscr{K}$
mit dem selben Konstantenkörper.
Wenn es für $f\in\mathscr{K}$ ein $F\in\mathscr{G}$ mit $F'=f$ gibt,
dann existieren Elemente $v_0,\dots,v_m\in\mathscr{K}$ und
Konstanten $c_0,\dots,c_m\in C$ derart, dass
\[
f = v_0' + \sum_{i=1}^m c_i\frac{v_i'}{v_i}
\qquad\text{bzw.}\qquad
F
=
\int f
=
v_0 + \sum_{i=1}^m c_i \log v_i.
\]
\end{satz}

Der Satz besagt, dass Stammfunktionen immer von der Form sind, wie sie in
Abschnitt~\ref{buch:integral:subsection:logexp} beschrieben worden sind.
Die Form des Integranden und der Stammfunktion ist genau die, 
welche wir im Beispiel der rationalen Funktionen in
\eqref{buch:integral:irat:eqn:liouvillefunktion}
und
\eqref{buch:integral:irat:eqn:liouvillestammfunktion}
bereits kennengelernt haben.
Auch das Beispiel $R(x,y)$ von Abschnitt~\ref{buch:integral:subsection:rxy}
hat in der Stammfunktion \eqref{buch:inetgral:sqrat:eqn:liouville}
auf diese Form geführt.
Noch wichtiger ist aber, dass es eine Aussage über die Form des
Integranden macht.
Um zu entscheiden, ob ein Integrand eine elementare Stammfunktion
hat, muss man also  ``nur'' noch das algebraische Problem lösen,
ob der Integrand $f$ in die Form von
Satz~\ref{buch:integral:risch:satz:liouville}
gebracht werden kann.

Diese Aufgabe ist alles andere als einfach, aber die früher diskutierten
Spezialfälle rationaler Funktionen und $R(x,y)$
(siehe Abschnitt~\ref{buch:integral:subsection:rxy}) können ein paar
Hinweise über das Vorgehen geben, mit dem man eine zusätzliche
Funktion $\vartheta$ behandeln kann, mit der der Differentialkörper
erweitert werden soll.
Ein rationale Funktion in $\vartheta$ wird zuänchst mit Hilfe der
Polynomdivision in einen polynomiellen und einen rationalen
Teil zerlegt.
Die Summanden des polynomiellen Teils können jetzt direkt mit
dem Prinzip von Liouville untersucht werden.
Der rationale Teil wird mit der Partialbruchzerlegung in
einfache rationale Funktionen zerlegt.
Im Nenner der Partialbrüche kommen Potenzen von Polynomen in $\vartheta$
vor. 
Im Falle von $R(x,y)$ war es möglich, diese auf einige wenige spezielle
Fälle von Nennern zu reduzieren.
Etwas Ähnliches ist auch im allgemeinen Fall unter Zuhilfenahme
des euklidischen Algorithmus für den Nenner und seine Ableitung möglich.
Damit kann erreicht werden, dass der Nenner quadratfrei ist.
Ein Satz von Rothstein-Trager ermöglicht dann, die Stammfunktion zu finden.





