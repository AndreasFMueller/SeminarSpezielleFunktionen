%
% risch.tex
%
% (c) 2021 Prof Dr Andreas Müller, OST Ostschweizer Fachhochschule
%
\section{Der Risch-Algorithmus
\label{buch:integral:section:risch}}
\rhead{Risch-Algorithmus}
Die Lösung des Integrationsproblem für $\mathbb{Q}(x)$ und für
$\mathbb{Q}(x,y)$ mit $y=\!\sqrt{ax^2+bx+c}$ hat gezeigt, dass
ein Differentialkörper genau die richtige Bühne für dieses Unterfangen
sein dürfte.
Die Stammfunktionen konnten in einem Erweiterungskörper gefunden
werden, der ein paar Logarithmen hinzugefügt worden sind.
Tatsächlich lässt sich in diesem Rahmen sogar ein Algorithmus
formulieren, der in einem noch zu definierenden Sinn ``elementare''
Funktionen als Stammfunktionen finden kann oder beweisen kann, dass
eine solche nicht existiert.
Dieser Abschnitt soll einen Überblick darüber geben.

%
% 41-logexp.tex
%
% (c) 2022 Prof Dr Andreas Müller, OST Ostschweizer Fachhochschlue
%
\subsection{Log-Exp-Notation für trigonometrische und hyperbolische Funktionen
\label{buch:integral:subsection:logexp}}
Die Integration rationaler Funktionen hat bereits gezeigt, dass
eine Stammfunktion nicht immer im Körper der rationalen Funktionen
existiert.
Es kann notwendig sein, dem Körper logarithmische Erweiterungen der Form
$\log(x-\alpha)$ hinzuzufügen.

Es können jedoch noch ganz andere neue Funktionen auftreten, wie die
folgende Zusammenstellung einiger Stammfunktionen zeigt:
\begin{equation}
\begin{aligned}
\int\frac{dx}{1+x^2}
&=
\arctan x,
\\
\int \cos x\,dx
&=
\sin x,
\\
\int\frac{dx}{\sqrt{1-x^2}}
&=
\arcsin x,
\\
\int
\operatorname{arcosh} x\,dx
&=
x \operatorname{arcosh} x - \sqrt{x^2-1}.
\end{aligned}
\label{buch:integration:risch:allgform}
\end{equation}
In der Stammfunktion treten Funktionen auf, die auf den ersten
Blick nichts mit den Funktionen im Integranden zu tun haben.

\subsubsection{Trigonometrische und hyperbolische Funktionen}
Die trigonometrischen und hyperbolichen Funktionen
in~\eqref{buch:integration:risch:allgform}
lassen sich alle durch Exponentialfunktionen ausdrücken.
So gilt
\begin{equation}
\begin{aligned}
\sin x &= \frac{1}{2i}\bigl( e^{ix} - e^{-ix}\bigr),
&
&\qquad&
\cos x &= \frac{1}{2}\bigl( e^{ix} + e^{-ix}\bigr),
\\
\sinh x &= \frac12\bigl( e^x - e^{-x} \bigr),
&
&\qquad&
\cosh x &= \frac12\bigl( e^x + e^{-x} \bigr).
\end{aligned}
\label{buch:integral:risch:trighyp}
\end{equation}
Nach Multiplikation mit $e^{ix}$ bzw.~$e^{x}$ entsteht jeweils eine
quadratische Gleichung in $e^{ix}$ bzw.~$e^{x}$.
Die Lösungsformel für quadratische Gleichungen erlaubt daher, $e^{ix}$
bzw.~$e^{x}$ zu finden und damit auch die Umkehrfunktionen.
Die Rechnung ergibt
\begin{equation}
\begin{aligned}
\arcsin y
&=
\frac{1}{i}\log\bigl(
iy\pm\sqrt{1-y^2}
\bigr),
&
&\qquad&
\arccos y
&=
\log\bigl(
y\pm \sqrt{y^2-1}
\bigr),
\\
\operatorname{arsinh}y
&=
\log\bigl(
y \pm \sqrt{1+y^2}
\bigr),
&
&\qquad&
\operatorname{arcosh} y
&=
\log\bigl(
y\pm \sqrt{y^2-1}
\bigr).
\end{aligned}
\label{buch:integral:risch:trighypinv}
\end{equation}
Alle Funktionen, die man aus dem elementaren Analysisunterricht
kennt, können also mit Hilfe von Exponentialfunktionen und Logarithmen
geschrieben werden.
Man nennt dies die $\log$-$\exp$-Notation der trigonometrischen
und hyperbolischen Funktionen.
\index{logexpnotation@$\log$-$\exp$-Notation}%

\subsubsection{$\log$-$\exp$-Notation}
Wendet man die Substitutionen
\eqref{buch:integral:risch:trighyp}
und
\eqref{buch:integral:risch:trighypinv}
auf die Integrale
\eqref{buch:integration:risch:allgform}
an, entstehen die Beziehungen
\begin{equation}
\begin{aligned}
\int\frac{1}{1+x^2}
&=
\frac12i\bigl(
\log(1-ix) - \log(1+ix)
\bigr),
\\
\int\bigl(
{\textstyle\frac12}
e^{ix}
+
{\textstyle\frac12}
e^{-ix}
\bigr)
&=
-{\textstyle\frac12}ie^{ix}
+{\textstyle\frac12}ie^{-ix},
\\
\int
\frac{1}{\sqrt{1-x^2}}
&=
-i\log\bigl(ix+\sqrt{1-x^2}),
\\
\int \log\bigl(x+\sqrt{x^2-1}\bigr)
&=
x\log\bigl(x+\sqrt{x^2-1}\bigr) - \sqrt{x^2-1}.
\end{aligned}
\label{buch:integration:risch:eqn:integralbeispiel2}
\end{equation}
Die in den Stammfuntionen auftretenden Funktionen treten entweder
schon im Integranden auf oder sind Logarithmen von solchen
Funktionen.
Zum Beispiel hat der Nenner im ersten Integral die Faktorisierung
$1+x^2=(1+ix)(1-ix)$, in der Stammfunktion findet man die Logarithmen
der Faktoren.



%
% 42-elementar.tex
%
% (c) 2022 Prof Dr Andreas Müller, OST Ostschweizer Fachhochschlue
%
\subsection{Elementare Funktionen
\label{buch:integral:subsection:elementar}}
Etwas allgemeiner kann man sagen, dass in den
Beispielen~\eqref{buch:integration:risch:eqn:integralbeispiel2}
algebraische Erweiterungen von $\mathbb{Q}(x)$ und Erweiterungen
um Logarithmen oder Exponentialfunktionen vorgekommen sind.
Die Stammfunktionen verwenden dieselben Funktionen oder höchstens
Erweiterungen um Logarithmen von Funktionen, die man schon im
Integranden gesehen hat.

%
% Exponentielle und logarithmische Funktione
%
\subsubsection{Exponentielle und logarithmische Funktionen}
In Abschnitt~\ref{buch:integral:subsection:diffke} haben wir
bereits die Exponentialfunktion $e^x$ und die Logarithmusfunktion 
$\log x$ charakterisiert als eine Körpererweiterung durch 
Elemente, die der Differentialgleichung
\[
\exp' = \exp
\qquad\text{und}\qquad
\log' = \frac{1}{x}
\]
genügen.
Für die Stammfunktionen, die in 
Abschnitt~\ref{buch:integral:subsection:logexp}
gefunden wurden, sind aber Logarithmusfunktionen nicht von
$x$ sondern von beliebigen über $\mathbb{Q}$ algebraischen Elementen
nötig.
Um zu verstehen, wie wir diese Funktion als Körpererweiterung erhalten
könnten, betrachten wir die Ableitung einer Exponentialfunktion
$\vartheta(x) = \exp(f(x))$ und eines
Logarithmus 
$\psi(x) = \log(f(x))$, wie man sie mit der Kettenregel
berechnet hätte:
\begin{align*}
\vartheta'(x)
&=\exp(f(x)) \cdot f'(x)
&
\psi'(x)
&=
\frac{f'(x)}{f(x)}
\quad\Leftrightarrow\quad
f(x)\psi'(x)
=
f'(x).
\end{align*}
Dies motiviert die folgende Definition

\begin{definition}
\label{buch:integral:def:explog}
Sei $\mathscr{F}$ ein Differentialkörper und $f\in\mathscr{F}$.
Ein Exponentialfunktion von $f$ ist ein $\vartheta\in \mathscr{F}$mit
$\vartheta' = \vartheta f'$.
Ein Logarithmus von $f$ ist ein $\vartheta\in\mathscr{F}$ mit
$f\vartheta'=f'$.
\end{definition}

Für $f=x$ mit $f'=1$ reduziert sich die 
Definition~\ref{buch:integral:def:explog}
auf die Definition der Exponentialfunktion $\exp(x)$ und
Logarithmusfunktion $\log(x)$ auf Seite~\pageref{buch:integral:expundlog}.


%
%
%
\subsubsection{Transzendente Körpererweiterungen}
Die Wurzelfunktionen haben wir früher als algebraische Erweiterungen
eines Differentialkörpers erkannt.
Die logarithmischen und exponentiellen Elemente gemäss
Definition~\ref{buch:integral:def:explog} sind nicht algebraisch.

\begin{definition}
\label{buch:integral:def:transzendent}
Sei $\mathscr{F}\subset\mathscr{G}$ eine Körpererweiterung und
$\vartheta\in\mathscr{G}$.
$\vartheta$ heisst {\em transzendent}, wenn $\vartheta$ nicht
algebraisch ist.
\end{definition}

\begin{beispiel}
Die Funktion $f = e^x + e^{2x} + e^{x/2}$ ist sicher transzendent,
in diesem Beispiel zeigen wir, dass es mindestens drei verschiedene
Möglichkeiten gibt, eine Körpererweiterung von $\mathbb{Q}(x)$ zu
konstruieren, die $f$ enthält.

Erste Möglichkeit: $f=\vartheta_1 + \vartheta_2 + \vartheta_3$ mit
$\vartheta_1=e^x$,
$\vartheta_2=e^{2x}$
und
$\vartheta_3=e^{x/2}$.
Jedes der Elemente $\vartheta_i$ ist exponentiell über $\mathbb{Q}(x)$ und 
$f$ ist in
\[
\mathbb{Q}(x)
\subset
\mathbb{Q}(x,\vartheta_1)
\subset
\mathbb{Q}(x,\vartheta_1,\vartheta_2)
\subset
\mathbb{Q}(x,\vartheta_1,\vartheta_2,\vartheta_3)
\ni
f.
\]
Jede dieser Körpererweiterungen ist transzendent.

Zweite Möglichkeit: $\vartheta_1=e^x$ ist exponentiell über 
$\mathbb{Q}(x)$ und $\mathbb{Q}(x,\vartheta_1)$ enthält wegen
\[
(\vartheta_1^2)'
=
2\vartheta_1\vartheta_1'
=
2\vartheta_1^2,
\]
somit ist $\vartheta_1^2=\vartheta_2$ eine Exponentialfunktion von $2x$
über $\mathbb{Q}(x)$.
Das Element $\vartheta_3=e^{x/2}$ ist zwar auch exponentiell über
$\mathbb{Q}(x)$, es ist aber auch eine Nullstelle des Polynoms
$m(z)=z^2-[\vartheta_1]$.
Die Erweiterung
$\mathbb{Q}(x,\vartheta_1)\subset\mathbb{Q}(x,\vartheta_1,\vartheta_3)$
ist eine algebraische Erweiterung, die
$f=\vartheta_1 + \vartheta_1^2+\vartheta_3$ enthält.

Dritte Möglichkeit: $\vartheta_3=e^{x/2}$ ist exponentiell über
$\mathbb{Q}(x)$.
Die transzendente Körpererweiterung
\[
\mathbb{Q}(x) \subset \mathbb{Q}(x,\vartheta_3)
\]
enthält das Element
$f=\vartheta_3^4+\vartheta_3^2 + \vartheta_3 $.
\end{beispiel}

Das Beispiel zeigt, dass man nicht sagen kann, dass eine Funktion
ausschliesslich in einer algebraischen oder transzendenten Körpererweiterung
zu finden ist. 
Vielmehr gibt es für die gleiche Funktion möglicherweise verschiedene
Körpererweiterungen, die alle die Funktion enthalten können.

%
% Elementare Funktionen
%
\subsubsection{Elementare Funktionen}
Die Stammfunktionen~\eqref{buch:integration:risch:eqn:integralbeispiel2}
können aufgebaut werden, indem man dem Körper $\mathbb{Q}(x)$ schrittweise
sowohl algebraische wie auch transzendente Elemente hinzufügt,
wie in der folgenden Definition, die dies für abstrakte
Differentialkörpererweiterungen formuliert.

\begin{definition}
Eine Körpererweiterung $\mathscr{F}\subset\mathscr{G}$ heisst 
{\em transzendente elementare Erweiterung}, wenn 
\index{transzendente elementare Erweiterung}%
$\mathscr{G} = \mathscr{F}(\vartheta_1,\dots,\vartheta_n)$ und
jedes der Element $\vartheta_i$ transzendent und logarithmisch oder
exponentiell ist über
$\mathscr{F}_{i-1}=\mathscr{F}(\vartheta_1,\dots,\vartheta_{i-1})$.
Die Körpererweiterung $\mathscr{F}\subset\mathscr{G}$ heisst
{\em elementare Erweiterung}, wenn 
\index{elementare Erweiterung}%
$\mathscr{G} = \mathscr{F}(\vartheta_1,\dots,\vartheta_n)$ und
jedes Element $\vartheta_i$ ist entweder logarithmisch, exponentiell
oder algebraisch über $\mathscr{F}_{i-1}$.
\end{definition}

Die Funktionen, die als akzeptable Stammfunktionen für das Integrationsproblem
in Betracht kommen, sind also jene, die in einer geeigneten elementaren
Erweiterung des von $\mathbb{Q}(x)$ liegen.
Ausserdem können auch noch weitere Konstanten nötig sein, sowohl
algebraische Zahlen wie auch Konstanten wie $\pi$ oder $e$.

\begin{definition}
Sei $\mathscr{K}(x)$ der Differentialklörper der rationalen Funktionen
über dem Konstantenkörper $\mathscr{K}\supset\mathbb{Q}$, der in $\mathbb{C}$
enthalten ist.
Ist $\mathscr{F}\supset \mathscr{K}(x)$ eine transzendente elementare 
Erweiterung von $\mathscr{K}(x)$, dann heisst $\mathscr{F}$
ein Körper von {\em transzendenten elementaren Funktionen}.
Ist $\mathscr{F}$ eine elementare Erweiterung von $\mathscr{K}(x)$, dann
heisst $\mathscr{F}$ ein Körper von {\em elementaren Funktionen}.
\end{definition}

\subsubsection{Das Integrationsproblem}
Die elementaren Funktionen enthalten alle Funktionen, die sich mit
arithmetischen Operationen, Wurzeln, Exponentialfunktionen, Logarithmen und
damit auch mit trigonometrischen und hyperbolischen Funktionen und ihren
Umkehrfunktionen aus den rationalen Zahlen, der unabhängigen Variablen $x$ 
und möglicherweise einigen zusätzlichen Konstanten aufbauen lassen.
Sei also $f$ eine Funktion in einem Körper von elementaren
Funktionen
\[
\mathscr(F)
=
\mathbb{Q}(\alpha_1,\dots,\alpha_l)(x,\vartheta_1,\dots,\vartheta_n).
\]
Eine elementare Stammfunktion ist eine Funktion $F=\int f$ in einer
elementaren Körpererweiterung
\[
\mathscr{G}
=
\mathbb{Q}(\alpha_1,\dots,\alpha_l,\dots,\alpha_{l+k})
(x,\vartheta_1,\dots,\vartheta_n,\dots,\vartheta_{n+m})
\]
mit $F'=f$.
Das Ziel ist, $F$ mit Hilfe eines Algorithmus zu bestimmen.




%
% liouville.tex
%
% (c) 2022 Prof Dr Andreas Müller, OST Ostschweizer Fachhochschule
%
\subsection{Das Prinzip von Liouville
\label{buch:integral:risch:subsection:liouville}}
Der erste Schritt in Richtung auf einen Algorithmus zur Lösung des
Integrationsproblems ist ein Kriterium dafür, ob eine Stammfunktion
einer Funktion $f\in\mathscr{K}$ existiert.
Der folgende Satz liefert ein solches Kriterium.

\begin{satz}[Prinzip von Liouville]
\label{buch:integral:risch:satz:liouville}
\index{Prinzip von Liouville}%
Sei $\mathscr{K}$ ein Differentialkörper mit dem Konstantenkörper $C$.
Weiter sei $\mathscr{G}$ eine Differentialkörpererweiterung von $\mathscr{K}$
mit dem selben Konstantenkörper.
Wenn es für $f\in\mathscr{K}$ ein $F\in\mathscr{G}$ mit $F'=f$ gibt,
dann existieren Elemente $v_0,\dots,v_m\in\mathscr{K}$ und
Konstanten $c_0,\dots,c_m\in C$ derart, dass
\[
f = v_0' + \sum_{i=1}^m c_i\frac{v_i'}{v_i}
\qquad\text{bzw.}\qquad
F
=
\int f
=
v_0 + \sum_{i=1}^m c_i \log v_i.
\]
\end{satz}

Der Satz besagt, dass Stammfunktionen immer von der Form sind, wie sie in
Abschnitt~\ref{buch:integral:subsection:logexp} beschrieben worden sind.
Die Form des Integranden und der Stammfunktion ist genau die, 
welche wir im Beispiel der rationalen Funktionen in
\eqref{buch:integral:irat:eqn:liouvillefunktion}
und
\eqref{buch:integral:irat:eqn:liouvillestammfunktion}
bereits kennengelernt haben.
Auch das Beispiel $R(x,y)$ von Abschnitt~\ref{buch:integral:subsection:rxy}
hat in der Stammfunktion \eqref{buch:inetgral:sqrat:eqn:liouville}
auf diese Form geführt.
Noch wichtiger ist aber, dass es eine Aussage über die Form des
Integranden macht.
Um zu entscheiden, ob ein Integrand eine elementare Stammfunktion
hat, muss man also  ``nur'' noch das algebraische Problem lösen,
ob der Integrand $f$ in die Form von
Satz~\ref{buch:integral:risch:satz:liouville}
gebracht werden kann.

Diese Aufgabe ist alles andere als einfach, aber die früher diskutierten
Spezialfälle rationaler Funktionen und $R(x,y)$
(siehe Abschnitt~\ref{buch:integral:subsection:rxy}) können ein paar
Hinweise über das Vorgehen geben, mit dem man eine zusätzliche
Funktion $\vartheta$ behandeln kann, mit der der Differentialkörper
erweitert werden soll.
Ein rationale Funktion in $\vartheta$ wird zuänchst mit Hilfe der
Polynomdivision in einen polynomiellen und einen rationalen
Teil zerlegt.
Die Summanden des polynomiellen Teils können jetzt direkt mit
dem Prinzip von Liouville untersucht werden.
Der rationale Teil wird mit der Partialbruchzerlegung in
einfache rationale Funktionen zerlegt.
Im Nenner der Partialbrüche kommen Potenzen von Polynomen in $\vartheta$
vor. 
Im Falle von $R(x,y)$ war es möglich, diese auf einige wenige spezielle
Fälle von Nennern zu reduzieren.
Etwas Ähnliches ist auch im allgemeinen Fall unter Zuhilfenahme
des euklidischen Algorithmus für den Nenner und seine Ableitung möglich.
Damit kann erreicht werden, dass der Nenner quadratfrei ist.
Ein Satz von Rothstein-Trager ermöglicht dann, die Stammfunktion zu finden.





