%
% iproblem.tex
%
% (c) 2022 Prof Dr Andreas Müller, OST Ostschweizer Fachhochschlue
%
\subsection{Das Integrationsproblem
\label{buch:integral:subsection:integrationsproblem}}
\index{Integrationsproblem}%
Die Ableitung ist ein einem Differentialkörper mit Hilfe der Ableitungsregeln
immer ausführbar, ganz ähnlich wie die Berechnung von Potenzen in einem Körper
immer ausführbar ist.
Die Umkehrung, also eine sogenannte Stammfunktion zu finden, ist dagegen
deutlich schwieriger.

\begin{definition}
\index{Stammfunktion}
Eine {\em Stammfunktion} einer Funktion $f\in\mathscr{K}$ im Funktionenkörper
$\mathscr{K}$ ist eine Funktion $F\in\mathscr{K}$ derart, dass $F'=f$.
Wir schreiben auch $F=\int f$.
\end{definition}

Zwei Stammfunktionen $F_1$ und $F_2$ einer Funktion $f\in\mathscr{K}$
haben die Eigenschaft
\[
\left.\begin{aligned}
F_1' &= f \\
F_2' &= f 
\end{aligned}\quad\right\}
\qquad
\Rightarrow
\qquad
(F_1-F_2)' = 0
\qquad\Rightarrow\qquad
F_1-F_2\in\mathscr{C},
\]
die beiden Stammfunktionen unterscheiden sich daher nur durch eine
Konstante.

\subsubsection{Stammfunktion von Polynomen}
Für Polynome ist das Problem leicht lösbar.
Aus der Ableitungsregel
\[
\frac{d}{dx} x^n = nx^{n-1}
\]
folgt, dass
\[
\int x^n = \frac{1}{n+1} x^{n+1}
\]
eine Stammfunktion von $x^n$ ist.
Da $\int$ linear ist, ergibt sich damit auch eine Stammfunktion für
ein beliebiges Polynom
\[
g(x)
=
g_0 + g_1x + g_2x^2 + \dots g_nx^n
=
\sum_{k=0}^n g_kx^k
\in\mathbb{Q}(x)
\]
angeben:
\begin{equation}
\int g(x)
=
g_0x + \frac12g_1x^2 + \frac13g_2x^3 + \dots \frac{1}{n+1}g_nx^{n+1}
=
\sum_{k=0}^n 
\frac{g_k}{k+1}x^{k+1}.
\label{buch:integral:iproblem:eqn:polyintegral}
\end{equation}

\subsubsection{Körpererweiterungen}
Obwohl die Ableitung in einem Differentialkörper immer ausgeführt werden 
kann, gibt es keine Garantie, dass es eine Stammfunktion im gleichen 
Körper gibt.
Im kleinsten denkbaren Funktionenkörper $\mathbb{Q}(x)$
haben die negativen Potenzen linearer Funktionen die Stammfunktionen
\[
\int
\frac{1}{(x-\alpha)^k}
=
\frac{1}{(-k+1)(x-\alpha)^{k-1}}
\]
für $k\ne 1$, sind also wieder in $\mathbb{Q}(x)$.
Für $k=1$ ist aber
\[
\int \frac{1}{x-\alpha}
=
\log(x-\alpha),
\]
es braucht also eine Körpererweiterung um $\log(x-\alpha)$, damit
$(x-\alpha)^{-1}$ eine Stammfunktion in $\mathbb{Q}(x,\log(x-\alpha))$
hat.

