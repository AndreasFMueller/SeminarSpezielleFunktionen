%
% diffke.tex
%
% (c) 2022 Prof Dr Andreas Müller, OST Ostschweizer Fachhochschlue
%
\subsection{Differentialkörper und ihre Erweiterungen
\label{buch:integral:subsection:diffke}}
Die Ableitung wird in den Grundvorlesungen der Analysis jeweils
als ein Grenzprozess eingeführt.
Die praktische Berechnung von Ableitungen verwendet aber praktisch
nie diese Definition, sondern fast ausschliesslich die rein algebraischen
Ableitungsregeln.
So wie die Wurzelfunktionen im letzten Abschnitt als algebraische
Körpererweiterungen erkannt wurden, muss jetzt auch für die Ableitung
eine rein algebraische Definition gefunden werden.
Die entstehende Struktur ist der Differentialkörper, der in diesem
Abschnitt definiert werden soll.

%
% Derivation
%
\subsubsection{Derivation}
Für die praktische Berechnung der Ableitung einer Funktion verwendet
man in erster Linie die bekannten Rechenregeln.
Dazu gehören für zwei Funktionen $f$ und $g$
\begin{itemize}
\item Linearität: $(\alpha f+\beta g)' = \alpha f' + \beta g'$ für
Konstanten $\alpha$, $\beta$.
\item Produktregel: $(fg)'=f'g+fg'$.
\index{Produktregel}%
\item Quotientenregel: $(f/g)' = (f'g-fg')/g^2$.
\index{Quotientenregel}%
\end{itemize}
Die ebenfalls häufig verwendete Kettenregel $(f\circ g)' = (f'\circ g) g'$
\index{Kettenregel}%
für zusammengesetzte Funktionen wird später kaum benötigt, da wir
Verkettungen durch Körpererweiterungen ersetzen wollen.
Die Ableitung hat somit die rein algebraischen Eigenschaften
einer Derivation gemäss folgender Definition.

\begin{definition}
Sei $\mathscr{F}$ ein Körper.
Eine {\em Derivation} ist eine lineare Abbildung
\index{Derivation}%
$D\colon \mathscr{F}\to\mathscr{F}$
mit der Eigenschaft
\[
D(fg) = (Df)g+f(Dg).
\]
Ein {\em Differentialkörper} ist ein Körper mit einer Derivation.
\index{Differentialkoerper@Differentialkörper}%
\end{definition}

Die Ableitung in einem Funktionenkörper ist eine Derivation,
die sich zusätzlich dadurch auszeichnet, dass $Dx=x'=1$.
Sie wird weiterhin mit dem Strich bezeichnet.

%
% Ableitungsregeln
%
\subsubsection{Ableitungsregeln}
Die Definition einer Derivation macht keine Aussagen über Quotienten,
diese kann man aber aus den Eigenschaften einer Derivation sofort
ableiten.
Wir schreiben $q=f/g$ für $f,g\in\mathscr{F}$, dann ist $f=qg$.
Nach der Kettenregel gilt
\(
f'=q'g+qg'
\).
Substituiert man darin $q=f/g$ und löst nach $q'$ auf, erhält man
\[
f'=q'g+\frac{fg'}{g}
\qquad\Rightarrow\qquad
q'=\frac1{g}\biggl(f'-\frac{fg'}{g}\biggr)
=
\frac{f'g-fg'}{g^2}.
\]


%
% Konstantenkörper
%
\subsubsection{Konstantenkörper}
Die Ableitung einer Konstanten verschwindet.
Beim Hinzufügen von Funktionen zu einem Funktionenkörper können weitere
Konstanten hinzukommen, ohne dass dies auf den ersten Blick sichtbar wird.
Zum Beispiel enthält $\mathbb{Q}(x,\!\sqrt{x+\pi})$ wegen
$(\!\sqrt{x+\pi})^2-x=\pi$ auch die Konstante $\pi$.
Eine Derivation ermöglicht dank des nachfolgenden Satzes auch,
solche Konstanten zu erkennen.

\begin{satz}
Sei $\mathscr{F}$ ein Körper und $D$ eine Derivation in $\mathscr{F}$.
Dann ist die Menge $C=\{a\in\mathscr{F}\;|\;Da=0\}$ ein Körper.
\end{satz}

\begin{proof}[Beweis]
Es muss gezeigt werden, dass Summe und Produkt von Element von $C$ 
wieder in $C$ liegen.
Wenn $Da=Db=0$, dann ist $D(a+b)=Da+Db=0$, also ist $a+b\in C$.
Für das Produkt gilt $D(ab)=(Da)b+a(Db)=0b+a0=0$, also ist auch
$ab\in C$.
\end{proof}

Die Menge $C$ heisst der {\em Konstantenkörper} von $\mathscr{F}$.
\index{Konstantenkörper}%

%
% Logarithmus und Exponantialfunktion
%
\subsubsection{Logarithmus und Exponentialfunktion}
Die Exponentialfunktion und der Logarithmus sind nicht algebraisch
über $\mathbb{Q}(x)$, sie lassen sich nicht durch eine algebraische
Gleichung charakterisieren.
Sie zeichnen sich aber durch besondere Ableitungseigenschaften aus.
Die Theorie der gewöhnlichen Differentialgleichungen garantiert,
dass eine Funktion durch eine Differentialgleichung und Anfangsbedingungen
festgelegt ist.
Für die Exponentialfunktion und der Logarithmus haben die 
Ableitungseigenschaften
\[
\exp'(x) = \exp(x)
\qquad\text{und}\qquad
x \log'(x) = 1.
\]
\index{Exponentialfunktion}%
\index{Logarithmus}%
In der algebraischen Beschreibung eines Funktionenkörpers gibt es
das Konzept des Wertes einer Funktion an einer bestimmten Stelle nicht.
Somit können keine Anfangsbedingungen vorgegeben werden.
Da die Gleichungen linear sind, sind Vielfache einer Lösung wieder
Lösungen.
Insbesondere ist mit $\exp(x)$ auch $a\exp(x)$ eine Lösung und mit
$\log(x)$ auch $a\log(x)$ für alle Konstanten $a$.

Die Eigenschaft, dass die Exponentialfunktion die Umkehrfunktion
des Logarithmus ist, lässt sich mit den Mitteln eines Differentialkörpers
nicht ausdrücken.

