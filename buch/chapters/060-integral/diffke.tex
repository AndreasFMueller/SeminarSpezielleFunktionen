%
% diffke.tex
%
% (c) 2022 Prof Dr Andreas Müller, OST Ostschweizer Fachhochschlue
%
\subsection{Differentialkörper und ihre Erweiterungen
\label{buch:integral:subsection:diffke}}
Die Ableitung wird in den Grundvorlesungen der Analysis jeweils
als ein Grenzprozess eingeführt.
Die praktische Berechnung von Ableitungen verwendet aber praktisch
nie diese Definition, sondern fast ausschliesslich die rein algebraischen
Ableitungsregeln.
So wie die Wurzelfunktionen im letzten Abschnitt als algebraische
Körpererweiterungen erkannt wurden, muss jetzt auch für die Ableitung
eine rein algebraische Definition gefunden werden.
Die entstehende Struktur ist der Differentialkörper, der in diesem
Abschnitt definiert werden soll.

%
% Derivation
%
\subsubsection{Derivation}

\begin{definition}
Sei $\mathscr{F}$ ein Funktionenkörper.
Eine {\em Derivation} ist eine lineare Abbildung
$D\colon \mathscr{F}\to\mathscr{F}$
mit der Eigenschaft
\[
D(fg) = (Df)g+f(Dg).
\]
\end{definition}

%
% Ableitungsregeln
%
\subsubsection{Ableitungsregeln}
% Ableitungsregeln

%
% Konstantenkörper
%
\subsubsection{Konstantenkörper}
% Konstantenkörper

%
% Logarithmus und Exponantialfunktion
%
\subsubsection{Logarithmus und Exponentialfunktion}
% Logarithmus und Exponentialfunktion

