%
% 43-liouville.tex
%
% (c) 2022 Prof Dr Andreas Müller, OST Ostschweizer Fachhochschule
%
\subsection{Das Prinzip von Liouville
\label{buch:integral:risch:subsection:liouville}}
Der erste Schritt in Richtung auf einen Algorithmus zur Lösung des
Integrationsproblems ist ein Kriterium dafür, ob eine Stammfunktion
einer Funktion $f\in\mathscr{K}$ existiert.
Der folgende Satz liefert ein solches Kriterium.

\begin{satz}[Prinzip von Liouville]
\label{buch:integral:risch:satz:liouville}
\index{Prinzip von Liouville}%
Sei $\mathscr{K}$ ein Differentialkörper mit dem Konstantenkörper $C$.
Weiter sei $\mathscr{G}$ eine Differentialkörpererweiterung von $\mathscr{K}$
mit dem selben Konstantenkörper.
Wenn es für $f\in\mathscr{K}$ ein $F\in\mathscr{G}$ mit $F'=f$ gibt,
dann existieren Elemente $v_0,\dots,v_m\in\mathscr{K}$ und
Konstanten $c_0,\dots,c_m\in C$ derart, dass
\[
f = v_0' + \sum_{i=1}^m c_i\frac{v_i'}{v_i}
\qquad\text{bzw.}\qquad
F
=
\int f
=
v_0 + \sum_{i=1}^m c_i \log v_i.
\]
\end{satz}

Der Satz besagt, dass Stammfunktionen immer von der Form sind, wie sie in
Abschnitt~\ref{buch:integral:subsection:logexp} beschrieben worden sind.
Die Form des Integranden und der Stammfunktion ist genau die, 
welche wir im Beispiel der rationalen Funktionen in
\eqref{buch:integral:irat:eqn:liouvillefunktion}
und
\eqref{buch:integral:irat:eqn:liouvillestammfunktion}
bereits kennengelernt haben.
Auch das Beispiel $R(x,y)$ von Abschnitt~\ref{buch:integral:subsection:rxy}
hat in der Stammfunktion \eqref{buch:inetgral:sqrat:eqn:liouville}
auf diese Form geführt.
Noch wichtiger ist aber, dass es eine Aussage über die Form des
Integranden macht.
Um zu entscheiden, ob ein Integrand eine elementare Stammfunktion
hat, muss man also  ``nur'' noch das algebraische Problem lösen,
ob der Integrand $f$ in die Form von
Satz~\ref{buch:integral:risch:satz:liouville}
gebracht werden kann.

Diese Aufgabe ist alles andere als einfach, aber die früher diskutierten
Spezialfälle rationaler Funktionen und $R(x,y)$
(siehe Abschnitt~\ref{buch:integral:subsection:rxy}) können ein paar
Hinweise über das Vorgehen geben, mit dem man eine zusätzliche
Funktion $\vartheta$ behandeln kann, um die der Differentialkörper
erweitert werden soll.
Ein rationale Funktion in $\vartheta$ wird zuänchst mit Hilfe der
Polynomdivision in einen polynomiellen und einen rationalen
Teil zerlegt.
Die Summanden des polynomiellen Teils können jetzt direkt mit
dem Prinzip von Liouville untersucht werden.
Der rationale Teil wird mit der Partialbruchzerlegung in
einfache rationale Funktionen zerlegt.
Im Nenner der Partialbrüche kommen Potenzen von Polynomen in $\vartheta$
vor. 
Im Falle von $R(x,y)$ war es möglich, diese auf einige wenige spezielle
Fälle von Nennern zu reduzieren.
Etwas Ähnliches ist auch im allgemeinen Fall unter Zuhilfenahme
des euklidischen Algorithmus für den Nenner und seine Ableitung möglich.
Damit kann erreicht werden, dass der Nenner quadratfrei ist.
Ein Satz von Rothstein-Trager ermöglicht dann, die Stammfunktion zu finden.

