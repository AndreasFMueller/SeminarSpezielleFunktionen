%
% differentialkoerper.tex
%
% (c) 2021 Prof Dr Andreas Müller, OST Ostschweizer Fachhochschule
%
\section{Differentialkörper und das Integrationsproblem
\label{buch:integrale:section:dkoerper}}
\rhead{Differentialkörper}
Die Einführung einer neuen Funktion $\operatorname{erf}(x)$ wurde
durch die Behauptung gerechtfertigt, dass es für den Integranden
$e^{-x^2}$ keine Stammfunktion in geschlossener Form gäbe.
Die Fehlerfunktion ist bei weitem nicht die einzige mit dieser
Eigenschaft.
Doch woher weiss man, dass es keine solche Funktion gibt, und
was heisst überhaupt ``Stammfunktion in geschlossener Form''?
In diesem Abschnitt wird daher ein algebraischer Rahmen entwickelt,
in dem diese Frage sinnvoll gestellt werden kann.

%
% rational.tex
%
% (c) 2022 Prof Dr Andreas Müller, OST Ostschweizer Fachhochschlue
%
\subsection{Rationale Funktionen und Funktionenkörper
\label{buch:integral:subsection:rational}}


%
% erweiterungen.tex
%
% (c) 2022 Prof Dr Andreas Müller, OST Ostschweizer Fachhochschlue
%
\subsection{Körpererweiterungen
\label{buch:integral:subsection:koerpererweiterungen}}
Das Beispiel des Körpers $\mathbb{Q}(\!\sqrt{2})$ auf Seite
\pageref{buch:integral:beispiel:Qsqrt2} illustriert eine Möglichkeit,
einen kleinen Körper zu vergrössern.
Das Prinzip ist verallgemeinerungsfähig und soll in diesem Abschnitt
erarbeitet werden.

%
% algebraische Zahl-Erweiterungen
\subsubsection{Algebraische Erweiterungen}
Der Körper $\mathbb{Q}(\!\sqrt{2})$ entsteht aus dem Körper $\mathbb{Q}$
dadurch, dass man die Zahl $\sqrt{2}$ hinzufügt und alle erlaubten
arithmetischen Operationen zulässt.
Die Darstellung von Elementen von $\mathbb{Q}(\!\sqrt{2})$ als
$a+b\sqrt{2}$ ist möglich, weil die Zahl $\alpha=\sqrt{2}$ die 
algebraische Relation
\[
\alpha^2-2 = \sqrt{2}^2 -2 = 0
\]
erfüllt.
Voraussetzung für diese Aussage ist, dass es die Zahl $\sqrt{2}$ in einem
geeigneten grösseren Körper gibt. 
Die reellen oder komplexen Zahlen bilden einen solchen Körper.
Wir verallgemeinern diese Situation wie folgt.

\begin{definition}
Ist $K$ ein Körper, dann heisst ein Körper $L$ mit $K\subset L$ ein
{\em Erweiterungskörper} von $K$.
\index{Erweiterungskoerper@Erweiterungskörper}
\end{definition}

\begin{definition}
\label{buch:integral:definition:algebraisch}
Sei $K\subset L$ eine Körpererweiterung.
Das Element $\alpha\in L$ heisst {\em algebraisch} über $K$, wenn es
ein Polynom $p(x)\in K[x]$ gibt derart, dass $\alpha$ eine Nullstelle
von $p(x)$ ist, also gibt mit $p(\alpha)=0$.
Das normierte Polynom $m(x)$ geringsten Grades, welches $m(\alpha)=0$
erfüllt, heisst das {\em Minimalpolynom} von $\alpha$.
\index{Minimalpolynom}%
\end{definition}

Man sagt auch $\alpha$ ist algebraisch vom Grad $n$, wenn das Minimalpolynom
den Grad $n$ hat.
Wenn $\alpha\ne 0$ algebraisch ist, dann ist auch $1/\alpha$ algebraisch,
wie das folgende Argument zeigt.
Für das Minimalpolynom $m(x)$ von $\alpha$, ist $m(\alpha)=0$.
Teilt man diese Gleichung durch $\alpha^n$ teilt, erhält man 
\[
m_0\frac{1}{\alpha^n}
+
m_1\frac{1}{\alpha^{n-1}}
+
\ldots
+
m_{n-1}\frac{1}{\alpha}
+
1
=
0,
\]
das Polynom
\[
\hat{m}(x)
=
m_0x^n + m_1x^{n-1} + \ldots m_{n-1} x + 1
\in
K[x]
\]
hat also $\alpha^{-1}$ als Nullstelle.
Das Polynom $\hat{m}(x)$ beweist daher, dass $\alpha^{-1}$ algebraisch ist.

Die Zahl $\sqrt{2}\in\mathbb{R}$ ist also algebraisch über $\mathbb{Q}$
und jede andere Quadratwurzel von Elementen von $\mathbb{Q}$ ist
ebenfalls algebraisch über $\mathbb{Q}$.
Auch der Körper $\mathbb{Q}(\alpha)$ kann für jede andere Quadratwurzel
auf die genau gleiche Art wie für $\sqrt{2}$ konstruiert werden.

\begin{definition}
\label{buch:integral:definition:algebraischeerweiterung}
Sei $K\subset L$ eine Körpererweiterung und $\alpha\in L$ ein algebraisches
Element mit Minimalpolynom $m(x)\in K[x]$.
Dann heisst die Menge
\begin{equation}
K(\alpha)
=
\{
a_0 + a_1\alpha + \ldots +a_n\alpha^n
\;|\;
a_i\in K
\}
\label{buch:integral:eqn:algelement}
\end{equation}
mit $n=\deg m(x) - 1$ der durch Adjunktion von $\alpha$ erhaltene
Erweiterungsköper.
\end{definition}

Wieder muss nur überprüft werden, dass jedes Produkt oder jeder
Quotient von Ausdrücken der Form~\eqref{buch:integral:eqn:algelement}
wieder in diese Form gebracht werden kann.
Dazu sei
\[
m(x)
=
m_0+m_1x + m_2x^2
+\ldots +m_{n-1}x^{n-1} + x^n
\]
das Minimalpolynom von $\alpha$.
Die Gleichung $m(\alpha)=0$ kann nach $\alpha^n$ aufgelöst werden und
liefert
\[
\alpha^n = -m_0 - m_1\alpha - m_2\alpha^2 -\ldots -m_{n-1}\alpha^{n-1}.
\]
Damit kann jede Potenz von $\alpha$ mit einem Exponenten grösser als $n$
in eine Linearkombination von Potenzen mit kleineren Exponenten
reduziert werden.
Ein Polynom in $\alpha$ kann also immer auf die
Form~\eqref{buch:integral:eqn:algelement}
gebracht werden.

Es muss aber noch gezeigt werden, dass auch der Kehrwert eines Elements
der Form~\eqref{buch:integral:eqn:algelement} in dieser Form geschrieben
werden kann.
Sei also $a(\alpha)$ so ein Element, dann sind die beiden Polynome
$a(x)$ und $m(x)$ teilerfremd, der grösste gemeinsame Teiler ist $1$.
Mit dem erweiterten euklidischen Algorithmus kann man zwei Polynome
$s(x)$ und $t(x)$ finden derart, dass $s(x)a(x)+t(x)m(x)=1$.
Setzt man $\alpha$ für $x$ ein, verschwindet das Minimalpolynom und
es bleibt
\[
s(\alpha)a(\alpha) = 1
\qquad\Rightarrow\qquad
s(\alpha) = \frac{1}{a(\alpha)}.
\]
Damit ist $s(\alpha)$ eine Darstellung von $1/a(\alpha)$ in der 
Form~\eqref{buch:integral:eqn:algelement}.

% Transzendente Körpererweiterungen
\subsubsection{Transzendente Erweiterungen}
Nicht alle Zahlen in $\mathbb{R}$ sind algebraisch.
Lindemann bewies 1882 einen allgemeinen Satz, aus dem folgt,
dass $\pi$ und $e$ nicht algebraisch sind, es gibt also
kein Polynom mit rationalen Koeffizienten, welches $\pi$
oder $e$ als Nullstelle hat.

\begin{definition}
Eine Zahl $\alpha\in L$ in einer Körpererweiterung $K\subset L$ 
heisst {\em transzendent}, wenn $\alpha$ nicht algebraisch ist,
wenn es also kein Polynom in $K[x]$ gibt, welches $\alpha$ als
Nullstelle hat.
\end{definition}

Die Zahlen $\pi$ und $e$ sind also transzendent.
Eine andere Art, diese Eigenschaft zu beschreiben ist zu sagen,
dass die Potenzen
\[
1=\pi^0, \pi, \pi^2,\pi^3,\dots
\]
linear unabhängig sind.
Gäbe es nämlich eine lineare Abhängigkeit, dann gäbe es Koeffizienten
$l_i$ derart, dass
\[
l_0 + l_1\pi^1 + l_2\pi^2 + \ldots + l_{n-1}\pi^{n-1} + l_{n}\pi^n = l(\pi)=0,
\]
und damit wäre dann ein Polynom gefunden, welches $\pi$ als Nullstelle hat.

Selbstverstländlich kann man zu einem transzendenten Element $\alpha$
immer noch einen Körper konstruieren, der alle Zahlen enthält, welche man
mit den arithmetischen Operationen aus $\alpha$ bilden kann.
Man kann ihn schreiben als
\[
K(\alpha)
=
\biggl\{
\frac{p(\alpha)}{q(\alpha)}
\;\bigg|\;
p(x),q(x)\in K[x] \wedge p(x)\ne 0
\biggr\},
\]
aber die Vereinfachungen zur
Form~\eqref{buch:integral:eqn:algelement}, die bei einem algebraischen
Element $\alpha$ möglich waren, können jetzt nicht mehr durchgeführt
werden.
$K\subset K(\alpha)$ ist zwar immer noch eine Körpererweiterung, aber
$K(\alpha)$ ist nicht mehr ein endlichdimensionaler Vektorraum.
Die Körpererweiterung $K\subset K(\alpha)$ heisst {\em transzendent}.

% rationale Funktionen als Körpererweiterungen
\subsubsection{Rationale Funktionen als Körpererweiterung}
Die unabhängige Variable wird bei Rechnen so behandelt, dass die
Potenzen alle linear unabhängig sind.
Dies ist die Grundlage für den Koeffizientenvergleich.
Der Körper der rationalen Funktion $K(x)$
ist also eine transzendente Körpererweiterung von $K$.

% Erweiterungen mit algebraischen Funktionen 
\subsubsection{Algebraische Funktionen}
Für das Integrationsproblem möchten wir nicht nur rationale Funktionen
verwenden können, sondern auch Wurzelfunktionen.
Wir möchten also zum Beispiel auch mit der Funktion $\sqrt{ax^2+bx+c}$
und allem, was man mit arithmetischen Operationen daraus machen kann,
arbeiten können.
Eine Körpererweiterung, die $\sqrt{ax^2+bx+c}$ enthält, enthält auch
alles, was man daraus bilden kann.
Doch wie bekommen wir die Funktion $\sqrt{ax^2+bx+c}$ in den Körper?

Die Art und Weise, wie man Wurzeln in der Schule kennenlernt ist als
eine neue Operation, die zu einer Zahl die Quadratwurzel liefert.
Diese Idee, den Körper mit einer weiteren Funktion anzureichern,
führt über nicht auf eine nützliche neue algebraische Struktur.
Wir dürfen daher $\sqrt{ax^2+bx+c}$ nicht als die Zusammensetzung
einer einzelnen neuen Funktion $\sqrt{\phantom{A}}$ mit
einem Polynom betrachten.

Die Wurzel $\sqrt{ax^2+bx+c}$ ist aber auch die Nullstelle des Polynoms
\[
p(z)
=
z^2 - [ax^2+bx+c]
\in
K(x)[z]
\]
mit Koeffizienten in $K(x)$.
Die eckigen Klammern sollen helfen, die Koeffizienten in $K(x)$
zu erkennen.
Die Funktion $\sqrt{ax^2+bx+c}$ ist also algebraisch über $K(x)$.
Einen Funktionenkörper, der die Funktion enthält, kann man also erhalten,
indem man den Körper $K(x)$ um das über $K(x)$ algebraische Element
$y=\sqrt{ax^2+bx+c}$ zu $K(x,y)=K(x,\sqrt{ax^2+bx+c}$ erweitert.
Wurzelfunktion werden daher nicht als Zusammensetzungen, sondern als
algebraische Erweiterungen eines Funktionenkörpers betrachtet.



%
% diffke.tex
%
% (c) 2022 Prof Dr Andreas Müller, OST Ostschweizer Fachhochschlue
%
\subsection{Differentialkörper und ihre Erweiterungen
\label{buch:integral:subsection:diffke}}
Die Ableitung wird in den Grundvorlesungen der Analysis jeweils
als ein Grenzprozess eingeführt.
Die praktische Berechnung von Ableitungen verwendet aber praktisch
nie diese Definition, sondern fast ausschliesslich die rein algebraischen
Ableitungsregeln.
So wie die Wurzelfunktionen im letzten Abschnitt als algebraische
Körpererweiterungen erkannt wurden, muss jetzt auch für die Ableitung
eine rein algebraische Definition gefunden werden.
Die entstehende Struktur ist der Differentialkörper, der in diesem
Abschnitt definiert werden soll.

%
% Derivation
%
\subsubsection{Derivation}

\begin{definition}
Sei $\mathscr{F}$ ein Funktionenkörper.
Eine {\em Derivation} ist eine lineare Abbildung
$D\colon \mathscr{F}\to\mathscr{F}$
mit der Eigenschaft
\[
D(fg) = (Df)g+f(Dg).
\]
\end{definition}

%
% Ableitungsregeln
%
\subsubsection{Ableitungsregeln}
% Ableitungsregeln

%
% Konstantenkörper
%
\subsubsection{Konstantenkörper}
% Konstantenkörper

%
% Logarithmus und Exponantialfunktion
%
\subsubsection{Logarithmus und Exponentialfunktion}
% Logarithmus und Exponentialfunktion


\input{chapters/060-integral/iproblem.tex}
\input{chapters/060-integral/irat.tex}
%
% sqrat.tex
%
% (c) 2022 Prof Dr Andreas Müller, OST Ostschweizer Fachhochschlue
%
\subsection{Integranden der Form $R(x,\sqrt{ax^2+bx+c})$
\label{buch:integral:subsection:rxy}}


