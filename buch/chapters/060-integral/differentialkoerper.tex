%
% differentialalgebren.tex
%
% (c) 2021 Prof Dr Andreas Müller, OST Ostschweizer Fachhochschule
%
\section{Differentialkörper und der Satz von Liouville
\label{buch:integrale:section:dkoerper}}
\rhead{Differentialkörper und der Satz von Liouville}
Das Problem der Darstellbarkeit eines Integrals in geschlossener
Form verlangt zunächst einmal nach einer Definition dessen, was man
als ``geschlossene Form'' akzeptieren will.
Die sogenannten {\em elementaren Funktionen} von
Abschnitt~\ref{buch:integrale:section:elementar}
bilden dafür den theoretischen Rahmen.
Das Problem ist dann die Frage zu beantworten, ob ein Integral eine
Stammfunktion hat, die eine elementare Funktion ist.
Der Satz von Liouville von Abschnitt~\ref{buch:integrale:section:liouville}
löst das Problem.

\subsection{Eine Analogie
\label{buch:integrale:section:analogie}}
% XXX Analogie: Formel für Polynom-Nullstellen 
% XXX           Stammfunktion als elementare Funktion
Das Analysis-Problem, eine Stammfunktion zu finden, ist analog zum
wohlbekannten algebraischen Problem, Nullstellen von Polynomen zu finden.
Wir entwickeln diese Analogie in etwas mehr Detail, um zu sehen, ob man
aus dem algebraischen Problem etwas über das Problem der Analysis
lernen kann.

Für ein Polynom $p(X) = a_nX^n+a_{n-1}X^{n-1}+\dots+a_1X+a_0\in\mathbb{C}[X]$
mit Koeffizienten $a_k\in\mathbb{C}$ ist es sehr einfach, für jede beliebige
komplexe Zahl $z\in\mathbb{C}$ den Wert $p(z)$ des Polynoms auszurechnen.
Ein paar wenige Rechenregeln genügen dazu, man kann leicht einem Kind 
beibringen, mit einem Taschenrechner so einen Wert auszurechnen.

Ähnlich sieht es mit der Ableitungsoperation aus. 
Einige wenige Ableitungsregeln, die man in der Analysis~I lernt,
erlauben, auf mehr oder weniger mechanische Art und Weise, jede
beliebige Funktion abzuleiten.
Man kann auch leicht einen Computer dazu programmieren, solche Ableitungen
symbolisch zu berechnen.

Aus dem Fundamentalsatz der Algebra, der von Gauss vollständig bewiesen
wurde, ist bekannt, dass jedes Polynom mit Koeffizienten in $\mathbb{C}$
genau so viele Lösungen in $\mathbb{C}$, wie der Grad des Polynoms angibt.
Dies ist aber ein Existenzsatz, er sagt nichts darüber aus, wie man diese
Lösungen finden kann.
In Spezialfällen, wie zum Beispiel für quadratische Polynome, gibt
es spezialsierte Lösungsverfahren, mit denen man Lösungen angeben kann.
Natürlich existieren numerische Methoden wie zum Beispiel das
Newton-Verfahren, mit dem man Nullstellen von Polynomen beliebig genau
bestimmen kann.

Der Fundamentalsatz der Integralrechnung besagt, dass jede stetige 
Funktion eine Stammfunktion hat, die bis auf eine Konstante eindeutig
bestimmt ist.
Auch dieser Existenzsatz gibt keinerlei Hinweise darauf, wie man die
Stammfunktion finden kann.
In der Analysis-Vorlesung lernt man viele Tricks, die in einer
beindruckenden Zahl von Spezialfällen ermöglichen, ein passende
Funktion anzugeben.
Man lernt auch numerische Verfahren kennen, mit denen sich Werte der
Stammfunktion, also bestimmte Integrale, mit beliebiger Genauigkeit
finden kann.

Die numerische Lösung des Nullstellenproblems ist insofern unbefriedigend,
als sie nur schwer eine Diskussion der Abhängigkeit der Nullstellen von
den Koeffizienten des Polynoms ermöglichen.
Eine Formel wie die Lösungsformel für die quadratische Gleichung 
stellt genau für solche Fälle ein ideales Werkzeug bereit.
Was man sich also wünscht ist nicht nur einfach eine Lösung, sondern eine
einfache Formel zur Bestimmung aller Lösungen.
Im Zusammenhang mit algebraischen Gleichungen erwartet man eine Formel,
in der nur arithmetische Operationen und Wurzeln vorkommen.
Für quadratische Gleichungen ist so eine Formel seit dem Altertum bekannt,
Formeln für die kubische Gleichung und die Gleichung vierten Grades wurden
im 16.~Jahrhundert von Cardano bzw.~Ferrari gefunden.
Erst viel später haben Abel und Ruffini gezeigt, dass so eine allgemeine
Formel für Polynome höheren Grades als 4 nicht existiert.
Die Galois-Theorie, die auf den Ideen von Évariste Galois beruht, 
stellt eine vollständige Theorie unter anderem für die Lösbarkeit
von Gleichungen durch Wurzelausdrücke dar.

Numerische Integralwerte haben ebenfalls den Nachteil, dass damit
Diskussionen wie die Abhängigkeit von Parametern eines Integranden
nur schwer möglich sind.
Was man sich daher wünscht ist eine Formel für die Stammfunktion,
die Werte als Zusammensetzung gut bekannter Funktionen wie der Exponential-
und Logarithmus-Funktionen oder der trigonometrischen Funktionen
sowie Wurzeln, Potenzen und den arithmetischen Operationen.
Man sagt, man möchte die Stammfunktion in ``geschlossener Form'' 
dargestellt haben.
Tatsächlich ist dieses Problem auch zu Beginn des 19.~Jahrhunderts
von Joseph Liouville genauer untersucht worden.
Er hat zunächst eine Klasse von ``elementaren Funktionen'' definiert,
die als Darstellungen einer Stammfunktion in Frage kommen.
Der Satz von Liouville besagt dann, dass nur Funktionen mit einer
ganz speziellen Form eine elementare Stammfunktion haben.
Damit wird es möglich, zu entscheiden, ob ein Integrand wie $e^{-x^2}$ 
eine elementare Stammfunktion hat.
Seit dieser Zeit weiss man zum Beispiel, dass die Fehlerfunktion nicht
mit den bekannten Funktionen dargestellt werden kann.

Mit dem Aufkommen der Computer und vor allem der Computer-Algebra-System (CAS)
wurde die Frage nach der Bestimmung einer Stammfunktion erneut aktuell.
Die ebenfalls weiter entwickelte abstrakte Algebra hat ermöglicht, die
Ideen von Liouville in eine erweiterte, sogenannte differentielle 
Galois-Theorie zu verpacken, die eine vollständige Lösung des Problems
darstellt.
Robert Henry Risch hat in den Sechzigerjahren auf dieser Basis
einen Algorithmus entwickelt, mit dem es möglich wird, zu entscheiden,
ob eine Funktion eine elementare Stammfunktion hat und diese
gegebenenfalls auch zu finden.
Moderne CAS implementieren diesen Algorithmus
in Teilen, besonders weit zu gehen scheint das quelloffene System
Axiom.

Der Risch-Algorithmus hat allerdings eine Achillesferse: er benötigt
eine Method zu entscheiden, ob zwei Ausdrücke übereinstimmen.
Dies ist jedoch ein im Allgemeinen nicht entscheidbares Problem.
Moderne CAS treiben einigen Aufwand, um die
Gleichheit von Ausdrücken zu entscheiden, sie können das Problem
aber grundsätzlich nicht vollständig lösen.
Damit kann der Risch-Algorithmus in praktischen Anwendungen das
Stammfunktionsproblem ebenfalls nur mit Einschränkungen lösen,
die durch die Fähigkeiten des Ausdrucksvergleichs in einem CAS
gesetzt werden.

Im Folgenden sollen elementare Funktionen definiert werden, es sollen
die Grundideen der differentiellen Galois-Theorie zusammengetragen werden
und der Satz von Liouvill vorgestellt werden.
An Hand der Fehler-Funktion soll dann gezeigt werden, wie man jetzt
einsehen kann, dass die Fehlerfunktion nicht elementar darstellbar ist.
Im nächsten Abschnitt dann soll der Risch-Algorithmus skizziert werden.

\subsection{Elementare Funktionen
\label{buch:integrale:section:elementar}}


\subsubsection{Rationale Funktionen}

\subsubsection{Wurzeln}

\subsubsection{Die trigonometrischen Funktionen}

\subsection{Differentielle Algebra
\label{buch:integrale:section:dalgebra}}

\subsubsection{Ableitungsoperation}

\subsubsection{Logarithmen und Exponentiale}

\subsubsection{Elementare Körpererweiterungen}

\subsection{Der Satz von Liouville
\label{buch:integrale:section:liouville}}

\subsection{Die Fehlerfunktion ist keine elementare Funktion
\label{buch:integrale:section:fehlernichtelementar}}
% \url{https://youtu.be/bIdPQTVF5n4}
Mit Hilfe des Satzes von Liouville kann man jetzt beweisen, dass 
die Fehlerfunktion keine elementare Funktion ist.
Dazu braucht man die folgende spezielle Form des Satzes.

\begin{satz}
\label{buch:integrale:satz:elementarestammfunktion}
Wenn $f(x)$ und $g(x)$ rationale Funktionen von $x$ sind, dann
ist die Stammfunktion von $f(x)e^{g(x)}$ genau dann eine 
elementare Funktion, wenn es eine rationale Funktion gibt, die
Lsung der Differentialgleichung
\[
r'(x) + g'(x)r(x)=f(x)
\]
ist.
\end{satz}

\begin{satz}
Die Funktion $x\mapsto e^{-x^2}$ hat keine elementare Stammfunktion.
\label{buch:iintegrale:satz:expx2}
\end{satz}

\begin{proof}[Beweis]
Unter Anwendung des Satzes~\ref{buch:integrale:satz:elementarestammfunktion}
auf $f(x)=1$ und $g(x)=-x^2$ folgt, $e^{-x^2}$ genau dann eine rationale 
Stammfunktion hat, wenn es eine rationale Funktion $r(x)$ gibt, die
Lösung der Differentialgleichung
\begin{equation}
r'(x) -2xr(x)=1
\label{buch:integrale:expx2dgl}
\end{equation}
ist.

Zunächst halten wir fest, dass $r(x)$ kein Polynom sein kann.
Wäre nämlich 
\[
r(x)
=
a_0 + a_1x + \dots + a_nx^n
=
\sum_{k=0}^n a_kx^k
\quad\Rightarrow\quad
r'(x)
=
a_1 + 2a_2x + \dots + na_nx^{n-1}
=
\sum_{k=1}^n
ka_kx^{k-1}
\]
ein Polynom, dann ergäbe sich beim Einsetzen in die Differentialgleichung
\begin{align*}
1
&=
r'(x)-2xr(x)
\\
&=
a_1 + 2a_2x + 3a_3x^2 + \dots + (n-1)a_{n-1}x^{n-2} + na_nx^{n-1}
\\
&\qquad
-
2a_0x -2a_1x^2 -2a_2x^3 - \dots - 2a_{n-1}x^n - 2a_nx^{n+1}
\\
&
\hspace{0.7pt}
\renewcommand{\arraycolsep}{1.8pt}
\begin{array}{crcrcrcrcrcrcrcr}
=&a_1&+&2a_2x&+&3a_3x^2&+&\dots&+&(n-1)a_{n-1}x^{n-2}&+&na_{n  }x^{n-1}& &           & & \\
 &   &-&2a_0x&-&2a_1x^2&-&\dots&-&    2a_{n-3}x^{n-2}&-&2a_{n-2}x^{n-1}&-&2a_{n-1}x^n&-&2a_nx^{n+1}
\end{array}
\\
&=
a_1
+
(2a_2-2a_0)x
+
(3a_3-2a_1)x^2
%+
%(4a_4-2a_2)x^3
+
\dots
+
(na_n-2a_{n-2})x^{n-1}
-
2a_{n-1}x^n
-
2a_nx^{n+1}.
\end{align*}
Koeffizientenvergleich zeigt, dass $a_1=1$ sein muss.
Aus den letzten zwei Termen liest man ebenfalls mittels Koeffizientenvergleich
ab, dass $a_n=0$ und $a_{n-1}=0$ sein müssen.
Aus den Koeffizienten $(ka_k-2a_{k-2})=0$ folgt, dass
$a_{k-2}=\frac{k}{2}a_k$ für alle $k>1$ sein muss, diese Koeffizienten
verschwinden also auch, inklusive $a_1=0$.
Dies ist allerdings im Widerspruch zu $a_1=1$.
Es folgt, dass $r(x)$ kein Polynom sein kann.

Der Nenner der rationalen Funktion $r(x)$ hat also mindestens eine Nullstelle
$\alpha$, man kann daher $r(x)$ auch schreiben als
\[
r(x) = \frac{s(x)}{(x-\alpha)^n},
\]
wobei die rationale Funktion $s(x)$ keine Nullstellen und keine Pole hat.
Einsetzen in die Differentialgleichung ergibt:
\[
1
=
r'(x) -2xr(x)
=
\frac{s'(x)}{(x-\alpha)^n}
-n
\frac{s(x)}{(x-\alpha)^{n+1}}
-
\frac{2xs(x)}{(x-\alpha)^n}.
\]
Multiplizieren mit $(x-\alpha)^{n+1}$ gibt
\[
(x-\alpha)^{n+1}
=
s'(x)(x-\alpha)
-
ns(x)
-
2xs(x)(x-\alpha)
\]
Setzt man $x=\alpha$ ein, verschwinden alle Terme ausser dem mittleren
auf der rechten Seite, es bleibt
\[
ns(\alpha) = 0.
\]
Dies widerspricht aber der Wahl der rationalen Funktion $s(x)$, für die
$\alpha$ keine Nullstelle ist.

Somit kann es keine rationale Funktion $r(x)$ geben, die eine Lösung der
Differentialgleichung~\eqref{buch:integrale:expx2dgl} ist und
die Funktion $e^{-x^2}$ hat keine elementare Stammfunktion.
\end{proof}

Der Satz~\ref{buch:iintegrale:satz:expx2} rechtfertigt die Einführung 
der Fehlerfunktion $\operatorname{erf}(x)$ als neue spezielle Funktion,
mit deren Hilfe die Funktion $e^{-x^2}$ integriert werden kann.



