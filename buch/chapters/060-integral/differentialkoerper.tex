%
% differentialkoerper.tex
%
% (c) 2021 Prof Dr Andreas Müller, OST Ostschweizer Fachhochschule
%
\section{Differentialkörper und das Integrationsproblem
\label{buch:integrale:section:dkoerper}}
\rhead{Differentialkörper}
Die Einführung einer neuen Funktion $\operatorname{erf}(x)$ wurde
durch die Behauptung gerechtfertigt, dass es für den Integranden
$e^{-x^2}$ keine Stammfunktion in geschlossener Form gäbe.
Die Fehlerfunktion ist bei weitem nicht die einzige mit dieser
Eigenschaft.
Doch woher weiss man, dass es keine solche Funktion gibt, und
was heisst überhaupt ``Stammfunktion in geschlossener Form''?
In diesem Abschnitt wird daher ein algebraischer Rahmen entwickelt,
in dem diese Frage sinnvoll gestellt werden kann.
Das ultimative Ziel, welches aber erst in
Abschnitt~\ref{buch:integral:section:risch} in Angriff genommen
wird, ist ein Computer-Algorithmus, der Integrale in geschlossener
Form findet oder beweist, dass dies für einen gegebenen Integranden
nicht möglich ist.

%
% rational.tex
%
% (c) 2022 Prof Dr Andreas Müller, OST Ostschweizer Fachhochschlue
%
\subsection{Rationale Funktionen und Funktionenkörper
\label{buch:integral:subsection:rational}}
Welche Funktionen sollen als Antwort auf die Frage nach einer Stammfunktion
akzeptiert werden?
Polynome in der unabhängigen Variablen $x$ sollten sicher dazu gehören,
also alles, was man mit Hilfe der Multiplikation, Addition und Subtraktion
aus Koeffizienten zum Beispiel in den rationalen Zahlen $\mathbb{Q}$ und
der unabhängigen Variablen aufbauen kann.
Doch welche weiteren Operationen sollen zugelassen werden und was lässt
sich über die entstehende Funktionenmenge aussagen?

\subsubsection{Körper}
Die kleinste Zahlenmenge, in der alle arithmetischen Operationen soweit
sinnvoll durchgeführt werden können, ist die Menge $\mathbb{Q}$ der
rationalen Zahlen.
Etwas formaler ist eine solche Menge, in der die Arithmetik uneingeschränkt
ausgeführt werden kann, ein Körper gemäss der folgenden Definition.
\index{Korper@Körper}%

\begin{definition}
\label{buch:integral:definition:koerper}
Eine {\em Körper} ist eine Menge $K$ mit zwei Verknüpfungen $+$, die Addition,
und $\cdot$, die Multiplikation,
welche die folgenden Eigenschaften haben.
\begin{center}
\renewcommand{\tabcolsep}{0pt}
\begin{tabular}{p{68mm}p{4mm}p{68mm}}
%Eigenschaften der
Addition:
\begin{enumerate}[{\bf A}.1)]
\item assoziativ: $(a+b)+c=a+(b+c)$
für alle $a,b,c\in K$
\item kommutativ: $a+b=b+a$
für alle $a,b\in K$
\item Neutrales Element der Addition: es gibt ein Element $0\in K$ mit
der Eigenschaft $a+0=a$ für alle $a\in K$
\item Additiv inverses Element: zu jedem Element $a\in K$ gibt es das Element
$-a$ mit der Eigenschaft $a+(-a)=0$.
\end{enumerate}
&&%
%Eigenschaften der
Multiplikation:
\begin{enumerate}[{\bf M}.1)]
\item assoziativ: $(a\cdot b)\cdot c=a\cdot (b\cdot c)$
für alle $a,b,c\in K$
\index{Assoziativgesetz}%
\index{assoziativ}%
\item kommutativ: $a\cdot b=b\cdot a$
für alle $a,b\in K$
\index{Kommutativgesetz}%
\index{kommutativ}%
\item Neutrales Element der Multiplikation: es gibt ein Element $1\in K$ mit
der Eigenschaft $a\cdot 1 =a$ für alle $a\in K$
\index{neutrales Element}%
\item Multiplikativ inverses Element: zu jedem Element
\index{inverses Element}%
$a\in K^*=K\setminus\{0\}$
gibt es das Element $a^{-1}$ mit der Eigenschaft $a\cdot a^{-1}=1$.
\index{Einheitengruppe}%
\index{Gruppe der invertierbaren Elemente}%
\end{enumerate}
\end{tabular}
\end{center}
\vspace{-22pt}
Ausserdem gilt das Distributivgesetz: für alle $a,b,c\in K$ gilt
$a\cdot(b+c)=a\cdot b + a\cdot c$.
\index{Disitributivgesetz}%
Die Menge $K^*$ heisst auch die {\em Einheitengruppe} oder die
{\em Gruppe der invertierbaren Elemente} des Körpers.
\end{definition}

Das Assoziativgesetz {\bf A}.1 besagt, dass Summen mit beliebig
vielen Termen ohne Klammern geschrieben werden kann, weil es nicht
darauf ankommt, in welcher Reihenfolge die Additionen ausgeführt werden.
Ebenso für das Assoziativgesetz {\bf M}.1 der Multiplikation.
Die Kommutativgesetze {\bf A}.2 und {\bf M}.2 implizieren, dass man
nicht auf die Reihenfolge der Summanden oder Faktoren achten muss.
Das Distributivgesetz schliesslich besagt, dass man Produkte ausmultiplizieren
oder gemeinsame Faktoren ausklammern kann, wie man es in der Schule
gelernt hat.

Die rellen Zahlen $\mathbb{R}$ und die komplexen Zahlen $\mathbb{C}$
bilden ebenfalls einen Körper, die von den rationalen Zahlen geerbten
Eigenschaften der Verknüpfungen setzen sich auf $\mathbb{R}$ und
$\mathbb{C}$ fort.
Es lassen sich allerdings auch Zahlkörper zwischen $\mathbb{Q}$ und
$\mathbb{R}$ konstruieren, wie das folgende Beispiel zeigt.

\begin{beispiel}
\label{buch:integral:beispiel:Qsqrt2}
Die Menge
\[
\mathbb{Q}(\!\sqrt{2})
=
\{
a+b\sqrt{2}
\;|\;
a,b\in \mathbb{Q}
\}
\]
ist eine Teilmenge von $\mathbb{R}$.
Die Rechenoperationen haben alle verlangten Eigenschaften, wenn gezeigt
werden kann, dass Produkte und Quotienten von Zahlen in $\mathbb{Q}(\!\sqrt{2})$
wieder in $\mathbb{Q}(\!\sqrt{2})$ sind.
Dazu rechnet man
\begin{align*}
(a+b\sqrt{2})
(c+d\sqrt{2})
&=
ac + 2bd + (ad+bc)\sqrt{2} \in \mathbb{Q}(\!\sqrt{2})
\intertext{und}
\frac{a+b\sqrt{2}}{c+d\sqrt{2}}
&=
\frac{a+b\sqrt{2}}{c+d\sqrt{2}}
\cdot
\frac{c-d\sqrt{2}}{c-d\sqrt{2}}
=
\frac{ac-2bd +(-ad+bc)\sqrt{2}}{c^2-2d^2}
\\
&=
\underbrace{\frac{ac-2bd}{c^2-2d^2}}_{\displaystyle\in\mathbb{Q}}
+
\underbrace{\frac{-ad+bc}{c^2-2d^2}}_{\displaystyle\in\mathbb{Q}}
\sqrt{2}
\in \mathbb{Q}(\!\sqrt{2}).
\qedhere
\end{align*}
\end{beispiel}


\subsubsection{Rationalen Funktionen}
Die als Antworten auf die Frage nach einer Stammfunktion akzeptablen
Funktionen sollten alle rationalen Zahlen sowie die unabhängige
Variable $x$ enthalten.
Ausserdem sollte man beliebige arithmetische Operationen mit
diesen Ausdrücken durchführen können.
Mit Addition, Subtraktion und Multiplikation entstehen aus den
rationalen Zahlen und der unabhängigen Variablen die Polynome $\mathbb{Q}[x]$
(siehe auch Abschnitt~\ref{buch:potenzen:section:polynome}).


\begin{definition}
Die Menge
\[
\mathbb{Q}(x)
=
\biggl\{
\frac{p(x)}{q(x)}
\;\bigg|\;
p(x),q(x)\in\mathbb{Q}[x]
\wedge
q(x)\ne 0
\biggr\},
\]
bestehend aus allen Quotienten von Polynome, deren Nenner nicht
das Nullpolynom ist, heisst der Körper der {\em rationalen Funktionen}
\index{rationale Funktion}%
mit Koeffizienten in $\mathbb{Q}$.
\end{definition}

Die Definition erlaubt, dass der Nenner Nullstellen hat, die sich in
Polen der Funktion äussern.
Die Eigenschaften eines Körpers sind sicher erfüllt, wenn wir uns
nur davon überzeugen können,
dass die arithmetischen Operationen nicht aus dieser Funktionenmenge
herausführen.
Dazu muss man nur verstehen, dass die Operation des gleichnamig Machens 
zweier Brüche auch für Nenner funktioniert, die Polynome sind, und die
Summe wzeier Brüche von Polynomen wieder in einen Bruch von Polynomen
umwandelt.




%
% erweiterungen.tex
%
% (c) 2022 Prof Dr Andreas Müller, OST Ostschweizer Fachhochschlue
%
\subsection{Körpererweiterungen
\label{buch:integral:subsection:koerpererweiterungen}}
Das Beispiel des Körpers $\mathbb{Q}(\!\sqrt{2})$ auf Seite
\pageref{buch:integral:beispiel:Qsqrt2} illustriert eine Möglichkeit,
einen kleinen Körper zu vergrössern.
Das Prinzip ist verallgemeinerungsfähig und soll in diesem Abschnitt
erarbeitet werden.

%
% algebraische Zahl-Erweiterungen
\subsubsection{Algebraische Erweiterungen}
Der Körper $\mathbb{Q}(\!\sqrt{2})$ entsteht aus dem Körper $\mathbb{Q}$
dadurch, dass man die Zahl $\sqrt{2}$ hinzufügt und alle erlaubten
arithmetischen Operationen zulässt.
Die Darstellung von Elementen von $\mathbb{Q}(\!\sqrt{2})$ als
$a+b\sqrt{2}$ ist möglich, weil die Zahl $\alpha=\sqrt{2}$ die 
algebraische Relation
\[
\alpha^2-2 = \sqrt{2}^2 -2 = 0
\]
erfüllt.
Voraussetzung für diese Aussage ist, dass es die Zahl $\sqrt{2}$ in einem
geeigneten grösseren Körper gibt. 
Die reellen oder komplexen Zahlen bilden einen solchen Körper.
Wir verallgemeinern diese Situation wie folgt.

\begin{definition}
Ist $K$ ein Körper, dann heisst ein Körper $L$ mit $K\subset L$ ein
{\em Erweiterungskörper} von $K$.
\index{Erweiterungskoerper@Erweiterungskörper}
\end{definition}

\begin{definition}
\label{buch:integral:definition:algebraisch}
Sei $K\subset L$ eine Körpererweiterung.
Das Element $\alpha\in L$ heisst {\em algebraisch} über $K$, wenn es
ein Polynom $p(x)\in K[x]$ gibt derart, dass $\alpha$ eine Nullstelle
von $p(x)$ ist, also gibt mit $p(\alpha)=0$.
Das normierte Polynom $m(x)$ geringsten Grades, welches $m(\alpha)=0$
erfüllt, heisst das {\em Minimalpolynom} von $\alpha$.
\index{Minimalpolynom}%
\end{definition}

Man sagt auch $\alpha$ ist algebraisch vom Grad $n$, wenn das Minimalpolynom
den Grad $n$ hat.
Wenn $\alpha\ne 0$ algebraisch ist, dann ist auch $1/\alpha$ algebraisch,
wie das folgende Argument zeigt.
Für das Minimalpolynom $m(x)$ von $\alpha$, ist $m(\alpha)=0$.
Teilt man diese Gleichung durch $\alpha^n$ teilt, erhält man 
\[
m_0\frac{1}{\alpha^n}
+
m_1\frac{1}{\alpha^{n-1}}
+
\ldots
+
m_{n-1}\frac{1}{\alpha}
+
1
=
0,
\]
das Polynom
\[
\hat{m}(x)
=
m_0x^n + m_1x^{n-1} + \ldots m_{n-1} x + 1
\in
K[x]
\]
hat also $\alpha^{-1}$ als Nullstelle.
Das Polynom $\hat{m}(x)$ beweist daher, dass $\alpha^{-1}$ algebraisch ist.

Die Zahl $\sqrt{2}\in\mathbb{R}$ ist also algebraisch über $\mathbb{Q}$
und jede andere Quadratwurzel von Elementen von $\mathbb{Q}$ ist
ebenfalls algebraisch über $\mathbb{Q}$.
Auch der Körper $\mathbb{Q}(\alpha)$ kann für jede andere Quadratwurzel
auf die genau gleiche Art wie für $\sqrt{2}$ konstruiert werden.

\begin{definition}
\label{buch:integral:definition:algebraischeerweiterung}
Sei $K\subset L$ eine Körpererweiterung und $\alpha\in L$ ein algebraisches
Element mit Minimalpolynom $m(x)\in K[x]$.
Dann heisst die Menge
\begin{equation}
K(\alpha)
=
\{
a_0 + a_1\alpha + \ldots +a_n\alpha^n
\;|\;
a_i\in K
\}
\label{buch:integral:eqn:algelement}
\end{equation}
mit $n=\deg m(x) - 1$ der durch {\em Adjunktion} oder Hinzufügen
von $\alpha$ erhaltene Erweiterungsköper.
\end{definition}

Wieder muss nur überprüft werden, dass jedes Produkt oder jeder
Quotient von Ausdrücken der Form~\eqref{buch:integral:eqn:algelement}
wieder in diese Form gebracht werden kann.
Dazu sei
\[
m(x)
=
m_0+m_1x + m_2x^2
+\ldots +m_{n-1}x^{n-1} + x^n
\]
das Minimalpolynom von $\alpha$.
Die Gleichung $m(\alpha)=0$ kann nach $\alpha^n$ aufgelöst werden und
liefert
\[
\alpha^n = -m_0 - m_1\alpha - m_2\alpha^2 -\ldots -m_{n-1}\alpha^{n-1}.
\]
Damit kann jede Potenz von $\alpha$ mit einem Exponenten grösser als $n$
in eine Linearkombination von Potenzen mit kleineren Exponenten
reduziert werden.
Ein Polynom in $\alpha$ kann also immer auf die
Form~\eqref{buch:integral:eqn:algelement}
gebracht werden.

Es muss aber noch gezeigt werden, dass auch der Kehrwert eines Elements
der Form~\eqref{buch:integral:eqn:algelement} in dieser Form geschrieben
werden kann.
Sei also $a(\alpha)$ so ein Element, dann sind die beiden Polynome
$a(x)$ und $m(x)$ teilerfremd, der grösste gemeinsame Teiler ist $1$.
Mit dem erweiterten euklidischen Algorithmus kann man zwei Polynome
$s(x)$ und $t(x)$ finden derart, dass $s(x)a(x)+t(x)m(x)=1$.
Setzt man $\alpha$ für $x$ ein, verschwindet das Minimalpolynom und
es bleibt
\[
s(\alpha)a(\alpha) = 1
\qquad\Rightarrow\qquad
s(\alpha) = \frac{1}{a(\alpha)}.
\]
Damit ist $s(\alpha)$ eine Darstellung von $1/a(\alpha)$ in der 
Form~\eqref{buch:integral:eqn:algelement}.

%
% Komplexe Zahlen
%
\subsubsection{Komplexe Zahlen}
Die imaginäre Einheit $i$ hat die Eigenschaft, dass $i^2=-1$, insbesondere
ist sie Nullstelle des Polynoms $m(x)=x^2+1\in\mathbb{Q}[x]$.
Die Menge $\mathbb{Q}(i)$ ist daher eine algebraische Körpererweiterung
von $\mathbb{Q}$ bestehend aus den komplexen Zahlen mit rationalem
Real- und Imaginärteil.

%
% Transzendente Körpererweiterungen
%
\subsubsection{Transzendente Erweiterungen}
Nicht alle Zahlen in $\mathbb{R}$ sind algebraisch.
Lindemann bewies 1882 einen allgemeinen Satz, aus dem folgt,
dass $\pi$ und $e$ nicht algebraisch sind, es gibt also
kein Polynom mit rationalen Koeffizienten, welches $\pi$
oder $e$ als Nullstelle hat.

\begin{definition}
Eine Zahl $\alpha\in L$ in einer Körpererweiterung $K\subset L$ 
heisst {\em transzendent}, wenn $\alpha$ nicht algebraisch ist,
wenn es also kein Polynom in $K[x]$ gibt, welches $\alpha$ als
Nullstelle hat.
\end{definition}

Die Zahlen $\pi$ und $e$ sind also transzendent.
Eine andere Art, diese Eigenschaft zu beschreiben ist zu sagen,
dass die Potenzen
\[
1=\pi^0, \pi, \pi^2,\pi^3,\dots
\]
linear unabhängig sind.
Gäbe es nämlich eine lineare Abhängigkeit, dann gäbe es Koeffizienten
$l_i$ derart, dass
\[
l_0 + l_1\pi^1 + l_2\pi^2 + \ldots + l_{n-1}\pi^{n-1} + l_{n}\pi^n = l(\pi)=0,
\]
und damit wäre dann ein Polynom gefunden, welches $\pi$ als Nullstelle hat.

Selbstverstländlich kann man zu einem transzendenten Element $\alpha$
immer noch einen Körper konstruieren, der alle Zahlen enthält, welche man
mit den arithmetischen Operationen aus $\alpha$ bilden kann.
Man kann ihn schreiben als
\[
K(\alpha)
=
\biggl\{
\frac{p(\alpha)}{q(\alpha)}
\;\bigg|\;
p(x),q(x)\in K[x] \wedge p(x)\ne 0
\biggr\},
\]
aber die Vereinfachungen zur
Form~\eqref{buch:integral:eqn:algelement}, die bei einem algebraischen
Element $\alpha$ möglich waren, können jetzt nicht mehr durchgeführt
werden.
$K\subset K(\alpha)$ ist zwar immer noch eine Körpererweiterung, aber
$K(\alpha)$ ist nicht mehr ein endlichdimensionaler Vektorraum.
Die Körpererweiterung $K\subset K(\alpha)$ heisst {\em transzendent}.

%
% rationale Funktionen als Körpererweiterungen
%
\subsubsection{Rationale Funktionen als Körpererweiterung}
Die unabhängige Variable wird bei Rechnen so behandelt, dass die
Potenzen alle linear unabhängig sind.
Dies ist die Grundlage für den Koeffizientenvergleich.
Der Körper der rationalen Funktion $K(x)$
ist also eine transzendente Körpererweiterung von $K$.

%
% Erweiterungen mit algebraischen Funktionen 
%
\subsubsection{Algebraische Funktionen}
Für das Integrationsproblem möchten wir nicht nur rationale Funktionen
verwenden können, sondern auch Wurzelfunktionen.
Wir möchten also zum Beispiel auch mit der Funktion $\sqrt{ax^2+bx+c}$
und allem, was man mit arithmetischen Operationen daraus machen kann,
arbeiten können.
Eine Körpererweiterung, die $\sqrt{ax^2+bx+c}$ enthält, enthält auch
alles, was man daraus bilden kann.
Doch wie bekommen wir die Funktion $\sqrt{ax^2+bx+c}$ in den Körper?

Die Art und Weise, wie man Wurzeln in der Schule kennenlernt ist als
eine neue Operation, die zu einer Zahl die Quadratwurzel liefert.
Diese Idee, den Körper mit einer weiteren Funktion anzureichern,
führt über nicht auf eine nützliche neue algebraische Struktur.
Wir dürfen daher $\sqrt{ax^2+bx+c}$ nicht als die Zusammensetzung
einer einzelnen neuen Funktion $\sqrt{\phantom{A}}$ mit
einem Polynom betrachten.

Die Wurzel $\sqrt{ax^2+bx+c}$ ist aber auch die Nullstelle des Polynoms
\[
p(z)
=
z^2 - [ax^2+bx+c]
\in
K(x)[z]
\]
mit Koeffizienten in $K(x)$.
Die eckigen Klammern sollen helfen, die Koeffizienten in $K(x)$
zu erkennen.
Die Funktion $\sqrt{ax^2+bx+c}$ ist also algebraisch über $K(x)$.
Einen Funktionenkörper, der die Funktion enthält, kann man also erhalten,
indem man den Körper $K(x)$ um das über $K(x)$ algebraische Element
$y=\sqrt{ax^2+bx+c}$ zu $K(x,y)=K(x,\sqrt{ax^2+bx+c}$ erweitert.
Wurzelfunktion werden daher nicht als Zusammensetzungen, sondern als
algebraische Erweiterungen eines Funktionenkörpers betrachtet.

%
% Konjugation
%
\subsubsection{Konjugation}
Die komplexen Zahlen sind die algebraische Erweiterung der reellen Zahlen
um die Nullstelle $i$ des Polynoms $m(x)=x^2+1$.
Die Zahl $-i$ ist aber auch eine Nullstelle von $m(x)$, die mit algebraischen
Mitteln nicht von $i$ unterscheidbar ist.
Die komplexe Konjugation $a+bi\mapsto a-bi$ vertauscht die beiden 
\index{Konjugation, komplexe}%
\index{komplexe Konjugation}%
Nullstellen des Minimalpolynoms.

Ähnliches gilt für die Körpererweiterung $\mathbb{Q}(\!\sqrt{2})$.
$\sqrt{2}$ und $\sqrt{2}$ sind beide Nullstellen des Minimalpolynoms
$m(x)=x^2-2$, die mit algebraischen Mitteln nicht unterschiedbar sind.
Sie haben zwar verschiedene Vorzeichen, doch ohne eine Ordnungsrelation
können diese nicht unterschieden werden.
\index{Ordnungsrelation}%
Eine Ordnungsrelation zwischen rationalen Zahlen lässt sich zwar
definieren, aber die Zahl $\sqrt{2}$ ist nicht rational, es braucht
also eine zusätzliche Annahme, zum Beispiel die Identifikation von
$\sqrt{2}$ mit einer reellen Zahl in $\mathbb{R}$, wo der Vergleich
möglich ist.

Auch in $\mathbb{Q}(\!\sqrt{2})$ ist die Konjugation
$a+b\sqrt{2}\mapsto a-b\sqrt{2}$ eine Selbstabbildung, die
die Körperoperationen respektiert.

Das Polynom $m(x)=x^2-x-1$ hat die Nullstellen
\[
\frac12 \pm\sqrt{\biggl(\frac12\biggr)^2+1}
=
\frac{1\pm\sqrt{5}}{2}
=
\left\{
\bgroup
\renewcommand{\arraystretch}{2.20}
\renewcommand{\arraycolsep}{2pt}
\begin{array}{lcl}
\displaystyle
\frac{1+\sqrt{5}}{2} &=& \phantom{-}\varphi \\
\displaystyle
\frac{1-\sqrt{5}}{2} &=& \displaystyle-\frac{1}{\varphi}.
\end{array}
\egroup
\right.
\]
Sie erfüllen die gleiche algebraische Relation $x^2=x+1$.
Sie sind sowohl im Vorzeichen wie auch im absoluten Betrag 
verschieden, beides verlangt jedoch eine Ordnungsrelation als
Voraussetzung, die uns fehlt.
Aus beiden kann man mit rationalen Operationen $\sqrt{5}$ gewinnen,
denn
\[
\sqrt{5}
=
4\varphi-1
=
-4\biggl(-\frac{1}{\varphi}\biggr)^2-1
\qquad\Rightarrow\qquad
\mathbb{Q}(\!\sqrt{5})
=
\mathbb{Q}(\varphi)
=
\mathbb{Q}(-1/\varphi).
\]
Die Abbildung $a+b\varphi\mapsto a-b/\varphi$ ist eine Selbstabbildung
des Körpers $\mathbb{Q}(\!\sqrt{5})$, welche die beiden Nullstellen 
vertauscht.

Dieses Phänomen gilt für jede algebraische Erweiterung.
Die Nullstellen des Minimalpolynoms, welches die Erweiterung
definiert, sind grundsätzlich nicht unterscheidbar.
Mit der Adjunktion einer Nullstelle enthält der Erweiterungskörper
auch alle anderen.
Sind $\alpha_1$ und $\alpha_2$ zwei Nullstellen des Minimalpolynoms,
dann definiert die Abbildung $\alpha_1\mapsto\alpha_2$ eine Selbstabbildung,
die die Nullstellen permutiert.

Die algebraische Körpererweiterung
$\mathbb{Q}(x)\subset \mathbb{Q}(x,\sqrt{ax^2+bx+c})$
ist nicht unterscheidbar von
$\mathbb{Q}(x)\subset \mathbb{Q}(x,-\!\sqrt{ax^2+bx+c})$.
Für das Integrationsproblem bedeutet dies, dass alle Methoden so
formuliert werden müssen, dass die Wahl der Nullstellen auf die
Lösung keinen Einfluss haben.



%
% diffke.tex
%
% (c) 2022 Prof Dr Andreas Müller, OST Ostschweizer Fachhochschlue
%
\subsection{Differentialkörper und ihre Erweiterungen
\label{buch:integral:subsection:diffke}}
Die Ableitung wird in den Grundvorlesungen der Analysis jeweils
als ein Grenzprozess eingeführt.
Die praktische Berechnung von Ableitungen verwendet aber praktisch
nie diese Definition, sondern fast ausschliesslich die rein algebraischen
Ableitungsregeln.
So wie die Wurzelfunktionen im letzten Abschnitt als algebraische
Körpererweiterungen erkannt wurden, muss jetzt auch für die Ableitung
eine rein algebraische Definition gefunden werden.
Die entstehende Struktur ist der Differentialkörper, der in diesem
Abschnitt definiert werden soll.

%
% Derivation
%
\subsubsection{Derivation}
Für die praktische Berechnung der Ableitung einer Funktion verwendet
man in erster Linie die bekannten Rechenregeln.
Dazu gehören für zwei Funktionen $f$ und $g$
\begin{itemize}
\item Linearität: $(\alpha f+\beta g)' = \alpha f' + \beta g'$ für
Konstanten $\alpha$, $\beta$.
\item Produktregel: $(fg)'=f'g+fg'$.
\index{Produktregel}%
\item Quotientenregel: $(f/g)' = (f'g-fg')/g^2$.
\index{Quotientenregel}%
\end{itemize}
Die ebenfalls häufig verwendete Kettenregel $(f\circ g)' = (f'\circ g) g'$
\index{Kettenregel}%
für zusammengesetzte Funktionen wird später kaum benötigt, da wir
Verkettungen durch Körpererweiterungen ersetzen wollen.
Die Ableitung hat somit die rein algebraischen Eigenschaften
einer Derivation gemäss folgender Definition.

\begin{definition}
Sei $\mathscr{F}$ ein Körper.
Eine {\em Derivation} ist eine lineare Abbildung
\index{Derivation}%
$D\colon \mathscr{F}\to\mathscr{F}$
mit der Eigenschaft
\[
D(fg) = (Df)g+f(Dg).
\]
Ein {\em Differentialkörper} ist ein Körper mit einer Derivation.
\index{Differentialkoerper@Differentialkörper}%
\end{definition}

Die Ableitung in einem Funktionenkörper ist eine Derivation,
die sich zusätzlich dadurch auszeichnet, dass $Dx=x'=1$.
Sie wird weiterhin mit dem Strich bezeichnet.

%
% Ableitungsregeln
%
\subsubsection{Ableitungsregeln}
Die Definition einer Derivation macht keine Aussagen über Quotienten,
diese kann man aber aus den Eigenschaften einer Derivation sofort
ableiten.
Wir schreiben $q=f/g$ für $f,g\in\mathscr{F}$, dann ist $f=qg$.
Nach der Kettenregel gilt
\(
f'=q'g+qg'
\).
Substituiert man darin $q=f/g$ und löst nach $q'$ auf, erhält man
\[
f'=q'g+\frac{fg'}{g}
\qquad\Rightarrow\qquad
q'=\frac1{g}\biggl(f'-\frac{fg'}{g}\biggr)
=
\frac{f'g-fg'}{g^2}.
\]


%
% Konstantenkörper
%
\subsubsection{Konstantenkörper}
Die Ableitung einer Konstanten verschwindet.
Beim Hinzufügen von Funktionen zu einem Funktionenkörper können weitere
Konstanten hinzukommen, ohne dass dies auf den ersten Blick sichtbar wird.
Zum Beispiel enthält $\mathbb{Q}(x,\!\sqrt{x+\pi})$ wegen
$(\!\sqrt{x+\pi})^2-x=\pi$ auch die Konstante $\pi$.
Eine Derivation ermöglicht dank des nachfolgenden Satzes auch,
solche Konstanten zu erkennen.

\begin{satz}
Sei $\mathscr{F}$ ein Körper und $D$ eine Derivation in $\mathscr{F}$.
Dann ist die Menge $C=\{a\in\mathscr{F}\;|\;Da=0\}$ ein Körper.
\end{satz}

\begin{proof}[Beweis]
Es muss gezeigt werden, dass Summe und Produkt von Element von $C$ 
wieder in $C$ liegen.
Wenn $Da=Db=0$, dann ist $D(a+b)=Da+Db=0$, also ist $a+b\in C$.
Für das Produkt gilt $D(ab)=(Da)b+a(Db)=0b+a0=0$, also ist auch
$ab\in C$.
\end{proof}

Die Menge $C$ heisst der {\em Konstantenkörper} von $\mathscr{F}$.
\index{Konstantenkörper}%

%
% Logarithmus und Exponantialfunktion
%
\subsubsection{Logarithmus und Exponentialfunktion}
Die Exponentialfunktion und der Logarithmus sind nicht algebraisch
über $\mathbb{Q}(x)$, sie lassen sich nicht durch eine algebraische
Gleichung charakterisieren.
Sie zeichnen sich aber durch besondere Ableitungseigenschaften aus.
Die Theorie der gewöhnlichen Differentialgleichungen garantiert,
dass eine Funktion durch eine Differentialgleichung und Anfangsbedingungen
festgelegt ist.
Für die Exponentialfunktion und der Logarithmus haben die 
Ableitungseigenschaften
\[
\exp'(x) = \exp(x)
\qquad\text{und}\qquad
x \log'(x) = 1.
\]
\index{Exponentialfunktion}%
\index{Logarithmus}%
In der algebraischen Beschreibung eines Funktionenkörpers gibt es
das Konzept des Wertes einer Funktion an einer bestimmten Stelle nicht.
Somit können keine Anfangsbedingungen vorgegeben werden.
Da die Gleichungen linear sind, sind Vielfache einer Lösung wieder
Lösungen.
Insbesondere ist mit $\exp(x)$ auch $a\exp(x)$ eine Lösung und mit
$\log(x)$ auch $a\log(x)$ für alle Konstanten $a$.

Die Eigenschaft, dass die Exponentialfunktion die Umkehrfunktion
des Logarithmus ist, lässt sich mit den Mitteln eines Differentialkörpers
nicht ausdrücken.


%
% iproblem.tex
%
% (c) 2022 Prof Dr Andreas Müller, OST Ostschweizer Fachhochschlue
%
\subsection{Das Integrationsproblem
\label{buch:integral:subsection:integrationsproblem}}
\index{Integrationsproblem}%
Die Ableitung ist ein einem Differentialkörper mit Hilfe der Ableitungsregeln
immer ausführbar, ganz ähnlich wie die Berechnung von Potenzen in einem Körper
immer ausführbar ist.
Die Umkehrung, also eine sogenannte Stammfunktion zu finden, ist dagegen
deutlich schwieriger.

\begin{definition}
\index{Stammfunktion}
Eine {\em Stammfunktion} einer Funktion $f\in\mathscr{K}$ im Funktionenkörper
$\mathscr{K}$ ist eine Funktion $F\in\mathscr{K}$ derart, dass $F'=f$.
Wir schreiben auch $F=\int f$.
\end{definition}

Zwei Stammfunktionen $F_1$ und $F_2$ einer Funktion $f\in\mathscr{K}$
haben die Eigenschaft
\[
\left.\begin{aligned}
F_1' &= f \\
F_2' &= f 
\end{aligned}\quad\right\}
\qquad
\Rightarrow
\qquad
(F_1-F_2)' = 0
\qquad\Rightarrow\qquad
F_1-F_2\in\mathscr{C},
\]
die beiden Stammfunktionen unterscheiden sich daher nur durch eine
Konstante.

\subsubsection{Stammfunktion von Polynomen}
Für Polynome ist das Problem leicht lösbar.
Aus der Ableitungsregel
\[
\frac{d}{dx} x^n = nx^{n-1}
\]
folgt, dass
\[
\int x^n = \frac{1}{n+1} x^{n+1}
\]
eine Stammfunktion von $x^n$ ist.
Da $\int$ linear ist, ergibt sich damit auch eine Stammfunktion für
ein beliebiges Polynom
\[
g(x)
=
g_0 + g_1x + g_2x^2 + \dots g_nx^n
=
\sum_{k=0}^n g_kx^k
\in\mathbb{Q}(x)
\]
angeben:
\begin{equation}
\int g(x)
=
g_0x + \frac12g_1x^2 + \frac13g_2x^3 + \dots \frac{1}{n+1}g_nx^{n+1}
=
\sum_{k=0}^n 
\frac{g_k}{k+1}x^{k+1}.
\label{buch:integral:iproblem:eqn:polyintegral}
\end{equation}

\subsubsection{Körpererweiterungen}
Obwohl die Ableitung in einem Differentialkörper immer ausgeführt werden 
kann, gibt es keine Garantie, dass es eine Stammfunktion im gleichen 
Körper gibt.
Im kleinsten denkbaren Funktionenkörper $\mathbb{Q}(x)$
haben die negativen Potenzen linearer Funktionen die Stammfunktionen
\[
\int
\frac{1}{(x-\alpha)^k}
=
\frac{1}{(-k+1)(x-\alpha)^{k-1}}
\]
für $k\ne 1$, sind also wieder in $\mathbb{Q}(x)$.
Für $k=1$ ist aber
\[
\int \frac{1}{x-\alpha}
=
\log(x-\alpha),
\]
es braucht also eine Körpererweiterung um $\log(x-\alpha)$, damit
$(x-\alpha)^{-1}$ eine Stammfunktion in $\mathbb{Q}(x,\log(x-\alpha))$
hat.


%
% irat.tex
%
% (c) 2022 Prof Dr Andreas Müller, OST Ostschweizer Fachhochschlue
%
\subsection{Integration rationaler Funktionen
\label{buch:integral:subsection:rationalefunktionen}}
Für die Integration der rationalen Funktionen lernt man in einem
Analysis-Kurs üblicherweise ein Lösungsverfahren.
Dies zeigt zunächst, dass rationale Funktionen immer eine Stammfunktion
in einem geeigneten Erweiterungskörper haben.
Es deutet aber auch an, dass Stammfunktionen eine ziemlich spezielle
Form haben, die später als
Satz von Liouville~\ref{buch:integral:satz:liouville}
ein besondere Rolle spielen wird.

%
% Aufgabenstellung
%
\subsubsection{Aufgabenstellung}
In diesem Abschnitt ist eine rationale Funktion $f(x)\in\mathbb{Q}(x)$
gegeben, deren Stammfunktion bestimmt werden soll.
Als rationale Funktion kann sie als Bruch
\begin{equation}
f(x) = \frac{p(x)}{q(x)}
\label{buch:integral:irat:eqn:quotient}
\end{equation}
mit Polynomen $p(x),q(x)\in\mathbb{Q}[x]$ geschrieben werden.
Gesucht ist ein Erweiterungskörper $\mathscr{K}\supset \mathbb{Q}(x)$ 
derart und eine Stammfunktion $F\in\mathscr{K}$ von $f$, also $F'=f$.

%
% Polynomdivision
%
\subsubsection{Polynomdivision}
Der Quotient~\eqref{buch:integral:irat:eqn:quotient} kann durch Polynomdivision
mit Rest vereinfacht werden in einen polynomialen Teil und einen echten
Bruch:
\begin{equation}
f(x)
=
g(x)
+
\frac{a(x)}{b(x)}
\label{buch:integral:irat:eqn:polydiv}
\end{equation}
mit Polynomen $g(x),a(x),b(x)\in\mathbb[Q](x)$ und $\deg a(x) < \deg b(x)$.
Für den ersten Summanden liefert
\eqref{buch:integral:iproblem:eqn:polyintegral} eine Stammfunktion.
Im Folgenden bleibt also nur noch der zweite Term zu behandeln.

%
% Partialbruchzerlegung
%
\subsubsection{Partialbruchzerlegung}
Zur Berechnung des Integral des Bruchs
in~\eqref{buch:integral:irat:eqn:polydiv} wird die Partialbruchzerlegung
benötigt.
Der Einfachheit halber nehmen wir an, dass wir den Körper $\mathbb{Q}(x)$
mit alle Nullstellen $\beta_i$ des Nenner-Polynoms $b(x)$ zu einem Körper
$\mathscr{K}$ erweitert haben, in dem Nenner in Linearfaktoren zerfällt.
Unter diesen Voraussetzungen hat die Partialbruchzerlegung die Form
\begin{equation}
\frac{a(x)}{b(x)}
=
\sum_{i=1}^m
\sum_{k=1}^{k_i}
\frac{A_{ik}}{(x-\beta_i)^k},
\label{buch:integral:irat:eqn:partialbruch}
\end{equation}
wobei $k_i$ die Vielfachheit der Nullstelle $\beta_i$ ist.
Die Koeffizienten $A_{ik}$ können zum Beispiel mit Hilfe eines linearen
Gleichungssystems bestimmt werden.

Um eine Stammfunktion zu finden, muss man also Stammfunktionen für
jeden einzelnen Summanden bestimmen.
Für Exponenten $k>1$ im Nenner eines Terms der
Partialbruchzerlegung~\eqref{buch:integral:irat:eqn:partialbruch}
kann dazu die Regel
\[
\int \frac{A_{ik}}{(x-\beta_i)^k}
=
\frac{A_{ik}}{(-k+1)(x-\beta_i)^{k-1}}
\]
verwendet werden.
Diese Stammfunktion liegt wieder in $\mathbb{Q}(x)$ liegt.

%
% Körpererweiterungen
%
\subsubsection{Körpererweiterung}
Für $k=1$ ist eine logarithmische Erweiterung um die Funktion
\begin{equation}
\int \frac{A_{i1}}{x-\alpha_i}
=
A_{i1}
\log(x-\alpha_i)
\label{buch:integral:irat:eqn:logs}
\end{equation}
nötig.
Es gibt also eine Stammfunktion in einem Erweiterungskörper, sofern
er zusätzlich alle logarithmischen Funktionen
in~\ref{buch:integral:irat:eqn:logs} enthält.
Sie hat die Form
\[
\sum_{i=1}^m A_{i1} \log(x-\beta_i),
\]
wobei $A_{i1}\in\mathbb{Q}$ ist.

Setzt man alle vorher schon gefundenen Teile der Stammfunktion zusammen,
kann man sehen, dass die Stammfunktion die Form
\begin{equation}
F(x) = v_0(x) + \sum_{i=1}^m c_i \log v_i(x)
\label{buch:integral:irat:eqn:liouvillstammfunktion}
\end{equation}
haben muss.
Dabei ist $v_0(x)\in\mathbb{Q}(x)$ und besteht aus der Stammfunktion
des polynomiellen Teils und den Stammfunktionen der Terme der Partialbruchzerlegung mit Exponenten $k>1$.
Die logarithmischen Terme bestehen aus den Konstanten $c_i=A_{i1}$ 
und den Logarithmusfunktionen $v_i(x)=x-\beta_i\in\mathbb{Q}(x)$.
Die Funktion $f(x)$ muss daher die Form
\[
f(x)
=
v_0'(x)
+
\sum_{i=1}^m c_i\frac{v'_i(x)}{v_i(x)}
\]
gehabt haben.
Die Form~\eqref{buch:integral:irat:eqn:liouvillstammfunktion}
der Stammfunktion ist nicht eine Spezialität der rationalen Funktionen.
Sie wird auch bei grösseren Funktionenkörpern immer wieder auftreten
und ist als Satz von Liouville bekannt.

%
% Minimale algebraische Erweiterung
%
\subsubsection{Minimale algebraische Erweiterung}
XXX Rothstein-Trager


%
% sqrat.tex
%
% (c) 2022 Prof Dr Andreas Müller, OST Ostschweizer Fachhochschlue
%
\subsection{Integranden der Form $R(x,\sqrt{ax^2+bx+c})$
\label{buch:integral:subsection:rxy}}
Für rationale Funktionen lässt sich immer eine Stammfunktion in einem
Erweiterungskörper angeben, der durch hinzufügen einzelner logarithmischer
Funktionen entsteht.
Die dabei verwendeten Techniken lassen sich verallgemeinern.
Zur Illustration und Motivation des später beschriebenen Risch-Algorithmus
stellen wir uns in diesem Abschnitt der Aufgabe, Integrale
mit einem Integranden zu berechnen, der eine rationale Funktion von $x$
und $\sqrt{ax^2+bx+c}$ ist.

%
% Aufgabenstellung
%
\subsubsection{Aufgabenstellung}
Eine rationale Funktion von $x$ und $\sqrt{ax^2+bx+c}$ ist ein
Element des Differentialkörpers, den man aus $\mathbb{Q}(x)$ durch
hinzufügen des Elementes
\[
y=\sqrt{ax^2+bx+c}
\]
erhält.
Eine Funktion $f\in\mathbb{Q}(x,y)$ kann geschrieben werden als Bruch
\begin{equation}
f
=
\frac{
\tilde{p}_0 + \tilde{p}_1y + \dots + \tilde{p}_n y^n
}{
\tilde{q}_0 + \tilde{q}_1y + \dots + \tilde{q}_m y^m
}
\label{buch:integral:sqrat:eqn:ftilde}
\end{equation}
mit rationalen Koeffizienten $\tilde{p}_i,\tilde{q}_i\in\mathbb{Q}(x)$.
Gesucht ist eine Stammfunktion von $f$.

%
% Algebraische Vereinfachungen
%
\subsubsection{Algebraische Vereinfachungen}
Da $x^2=ax^2+bx+c$ ein Polynom ist, sind auch alle geraden Potenzen
von $y$ Polynome in $\mathbb{Q}(x)$,
und die ungeraden Potenzen von $y$ lassen sich als Produkt aus einem
Polynom und dem Faktor $y$ schreiben.
Der Integrand~\eqref{buch:integral:sqrat:eqn:ftilde} 
lässt sich daher vereinfachen zu einem Bruch der Form
\begin{equation}
f(x)
=
\frac{p_0+p_1y}{q_0+q_1y},
\label{buch:integral:sqrat:eqn:moebius}
\end{equation}
wobei $p_i$ und $q_i$ rationale Funktionen in $\mathbb{Q}(x)$ sind.

%
% Rationalisieren
%
\subsubsection{Rationalisieren}
Unschön an der Form~\eqref{buch:integral:sqrat:eqn:moebius} ist die
Tatsache, dass $y$ sowohl im Nenner wie auch im Zähler auftreten kann.
Da aber $y$ die quadratische Identität $y^2=ax^2+bx+c$ erfüllt,
kann das $y$ im Nenner durch Erweitern mit $q_0-q_1y$ zum verschwinden
gebracht werden.
Die Rechnung ergibt
\begin{align*}
\frac{p_0+p_1y}{q_0+q_1y}
&=
\frac{p_0+p_1y}{q_0+q_1y}
\cdot
\frac{q_0-q_1y}{q_0-q_1y}
=
\frac{(p_0+p_1y)(q_0-q_1y)}{q_0^2-q_1^2y^2}
\\
&=
\frac{p_0q_0-p_1q_1(ax^2+bx+c)}{q_0^2-q_1^2(ax^2+bx+c)}
+
\frac{q_0p_1-q_1p_0}{q_0^2-q_1^2(ax^2+bx+c)} y.
\end{align*}
Die Quotienten enthalten $y$ nicht mehr, sind also in $\mathbb{Q}(x)$.
In der späteren Rechnung stellt sich heraus, dass es praktischer ist,
das $y$ im Nenner zu haben, was man durch erweitern mit $y$ wieder
unter Ausnützung von $y^2=ax^2+bx+c$ erreichen kann.
Die zu integrierende Funktion  kann also in der Form
\begin{equation}
f(x)
=
W_1 + W_2\frac{1}{y}
\end{equation}
geschrieben werden mit rationalen Funktionen
$W_1,W_2\in\mathbb{Q}(x)$.
Eine Stammfunktion von $W_1$ kann mit der Methode von
Abschnitt~\ref{buch:integral:subsection:rationalefunktionen}
gefunden werden.
Im Folgenden kümmern wir uns daher nur noch um $W_1$.

\subsubsection{Polynomdivision}

\subsubsection{Integranden der Form $p(x)/y$}

\subsubsection{Partialbruchzerlegung}

\begin{equation}
\int
\frac{1}{(x-\alpha)^k \sqrt{ax^2+bx+c}}
\label{buch:integral:sqrat:eqn:2teart}
\end{equation}

\subsubsection{Integrale der Form \eqref{buch:integral:sqrat:eqn:2teart}}






