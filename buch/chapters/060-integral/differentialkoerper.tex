%
% differentialalgebren.tex
%
% (c) 2021 Prof Dr Andreas Müller, OST Ostschweizer Fachhochschule
%
\section{Differentialkörper und der Satz von Liouville
\label{buch:integrale:section:dkoerper}}
\rhead{Differentialkörper und der Satz von Liouville}
Das Problem der Darstellbarkeit eines Integrals in geschlossener
Form verlangt zunächst einmal nach einer Definition dessen, was man
als ``geschlossene Form'' akzeptieren will.
Die sogenannten {\em elementaren Funktionen} von
Abschnitt~\ref{buch:integrale:section:elementar}
bilden dafür den theoretischen Rahmen.
Das Problem ist dann die Frage zu beantworten, ob ein Integral eine
Stammfunktion hat, die eine elementare Funktion ist.
Der Satz von Liouville von Abschnitt~\ref{buch:integrale:section:liouville}
löst das Problem.

\subsection{Eine Analogie
\label{buch:integrale:section:analogie}}
% XXX Analogie: Formel für Polynom-Nullstellen 
% XXX           Stammfunktion als elementare Funktion

\subsection{Elementare Funktionen
\label{buch:integrale:section:elementar}}


\subsubsection{Rationale Funktionen}

\subsubsection{Wurzeln}

\subsubsection{Die trigonometrischen Funktionen}

\subsection{Differentielle Algebra
\label{buch:integrale:section:dalgebra}}

\subsubsection{Ableitungsoperation}

\subsubsection{Logarithmen und Exponentiale}

\subsubsection{Elementare Körpererweiterungen}

\subsection{Der Satz von Liouville
\label{buch:integrale:section:liouville}}

\subsection{Die Fehlerfunktion ist keine elementare Funktion
\label{buch:integrale:section:fehlernichtelementar}}

