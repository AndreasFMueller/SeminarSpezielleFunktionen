%
% eulertransformation.tex
%
% (c) 2021 Prof Dr Andreas Müller, OST Ostschweizer Fachhochschule
%
\section{Euler-Transformation der hypergeometrischen Funktionen
\label{buch:integral:section:eulertransformation}}
\rhead{Euler-Transformation}
Die hypergeometrischen Funktionen wurden bisher einerseits
als Reihen mit einer speziellen Rekursionsrelation der Reihenglieder
und als Lösungen einer speziellen Art von Differentialgleichung
erkannt.
In diesem Abschnitt soll untersucht werden, ob man sie auch
auch durch Integrale definieren kann.

\subsection{Integraldarstellung der hypergeometrischen Funktion
$\mathstrut_2F_1$}

XXX An dieser Stelle Abschnitt 4.3.5 (Integraldarstellung) einfügen

\begin{satz}[Euler]
\label{buch:integrale:eulertransformation:satz}
Die hypergeometrische Funktion $\mathstrut_2F_1$ kann durch das 
Integral
\begin{equation}
\mathstrut_2F_1\biggl(\begin{matrix}a,b\\c\end{matrix};z\biggr)
=
\frac{\Gamma(c)}{\Gamma(b)\Gamma(c-b)}
\int_0^1
t^{b-1}(1-t)^{c-b-1}(1-zt)^{-a}
\,dt
\label{buch:integrale:eulertransformation:satzeqn}
\end{equation}
dargestellt werden.
\end{satz}

\subsubsection{Alternative Parametrisierungen}
Die Substitution $t=\sin^2 s$ ermöglicht eine alternative Parametrisierung
der Integraldarstellung der hypergeometrischen Funktion.
Wenden wir sie auf~\eqref{buch:integrale:eulertransformation:satzeqn}
an, erhalten wir wegen $dt = 2\cos s\sin s\,ds$
\begin{align*}
\mathstrut_2F_1\biggl(\begin{matrix}a,b\\c\end{matrix};z\biggr)
&=
\frac{\Gamma(c)}{\Gamma(b)\Gamma(c-b)}
\int_0^{\frac{\pi}2}
\sin^{2(b-1)}(s)\,
(1-\sin^2s)^{c-b-1} (1-z\sin^2 s)^{-a}
\,\cos s\sin s
\,ds
\\
&=
\frac{\Gamma(c)}{\Gamma(b)\Gamma(c-b)}
\int_0^{\frac{\pi}2}
\sin^{2b-1}(s)\,\cos^{2c-2b-1}(s)\, (1-z\sin^2 s)^{-a}
\,ds.
\end{align*}

XXX Parametrisierung für Intervall $[0,\infty)$

\subsection{Integraldarstellung als Integraltransformation}
Im vorangegangenen Abschnitt wurde gezeigt, wie sich die Funktion
$\mathstrut_2F_1$ als ein Integral des Integranden
\[
t^{b-1}(1-t)^{c-b-1} (1-xt)^{-a}
\]
ausdrücken lässt.
Der letzte Faktor $(1-xt)^{-a}$ kann mit der Binomialreihe
\begin{align*}
(1+x)^\alpha
&=
1
+ 
\alpha x
+
\frac{\alpha(\alpha-1)}{2!}x^2
+
\frac{\alpha(\alpha-1)(\alpha-2)}{3!}x^3
+
\\
&=
1
+
\frac{-\alpha}{1}(-x)
+
\frac{-\alpha(-\alpha+1)}{2!} (-x)^2
+
\frac{-\alpha(-\alpha+1)(-\alpha+2)}{3!} (-x)^3
+
\dots
\\
&=
\sum_{k=0}^\infty \frac{(-\alpha)_k}{k!} (-x)^k
=
\mathstrut_0F_1\biggl(
\begin{matrix}
\text{---}\\-\alpha
\end{matrix}
;-x
\biggr)
\end{align*}
als hypergeometrische Funktion geschrieben werden.
Die Integraldarstellung von $\mathstrut_2F_1$ kann daher auch als
\[
\mathstrut_2F_1\biggl(\begin{matrix}a,b\\c\end{matrix};z\biggr)
=
\frac{\Gamma(c)}{\Gamma(b)\Gamma(c-b)}
\int_0^1 t^{b-1}(1-t)^{c-b-1}
\,
\mathstrut_0F_1(;a;zt)\,dt
\]
Eine gewisse Ähnlichkeit zur Laplace-Transformation ist dieser
Formel nicht abzusprechen.
Die Funktion \( t^{b-1}(1-t)^{c-b-1} \) wird statt mit der
Exponentialfunktion $e^{xt} = \mathstrut_0F_0(xt)$ mit der
hypergeometrischen Funktion $\mathstrut_0F_1(;a;xt)$ multipliziert und
integriert.
Dies suggeriert, dass sich möglicherweise jede der hypergeometrischen
Funktionen $\mathstrut_{p+1}F_{q+1}$ durch ein Integral, dessen 
Integrand $\mathstrut pF_q$ enthält, ausdrücken lässt.

\begin{satz}
Es gilt
\[
\mathstrut_{p+1}F_{q+1}\biggl(
\begin{matrix}
a_1,\dots,a_{p+1}\\
b_1,\dots,b_{q+1}
\end{matrix}
;z
\biggr)
=
\frac{\Gamma(b_{q+1})}{\Gamma(a_{p+1})\Gamma(b_{q-1}-a_{p+1})}
\int_0^1
t^{a_{p+1}-1}(1-t)^{b_{q+1}-a_{p+1}-1}
\mathstrut_pF_q\biggl(
\begin{matrix}
a_1,\dots,a_p\\
b_1,\dots,b_q
\end{matrix};zt
\biggr)
\,dt
\]
\end{satz}

\begin{proof}[Beweis]
Sei $I$ das Integral auf der rechten Seite.
Wir setzen die Reihenentwicklung der Funktion $\mathstrut_pF_q$ in
die Integralformel ein und erhalten
\begin{align*}
I
&=
\int_0^1 t^{a_{p+1}-1}(1-t)^{b_{q+1}-a_{p+1}-1}
\sum_{k=0}^\infty
\frac{(a_1)_k\cdots (a_p)_k}{(b_1)_k\cdots (b_q)_k}
\frac{(zt)^k}{k!}
\,dt
\\
&=
\sum_{k=0}^\infty
\frac{(a_1)_k\cdots (a_p)_k}{(b_1)_k\cdots (b_q)_k}
\frac{z^k}{k!}
\int_0^1
t^{a_{p+1}+k-1}(1-t)^{b_{q+1}-a_{p+1}-1}
\,dt.
\intertext{Das verbleibende Integral auf der rechten Seite ist das
Beta-Integral $B(a_{p+1}+k, b_{q+1}-a_{p+1})$:
}
&=
\sum_{k=0}^\infty
\frac{(a_1)_k\cdots (a_p)_k}{(b_1)_k\cdots (b_q)_k}
\frac{z^k}{k!}
B(a_{p+1}+k, b_{q+1}-a_{p+1}).
\intertext{Mit der Rekursionsformel aus
Lemma~\ref{buch:rekursion:gamma:betareklemma}
für das Beta-Integral folgt}
&=
\sum_{k=0}^\infty
\frac{(a_1)_k\cdots (a_p)_k}{(b_1)_k\cdots (b_q)_k}
\frac{z^k}{k!}
\frac{(a_{p+1})_k}{(b_{q+1})_k} B(a_{p+1},b_{q+1}-a_{p+1})
\\
&=
\sum_{k=0}^\infty
\frac{(a_1)_k\cdots (a_{p+1})_k}{(b_1)_k\cdots (b_{q+})_k}
\frac{z^k}{k!}
\frac{\Gamma(a_{p+1})\Gamma(b_{q+1}-a_{p+1})}{\Gamma(b_{q+1})}
\\
&=
\frac{\Gamma(a_{p+1})\Gamma(b_{q+1}-a_{p+1})}{\Gamma(b_{q+1})}
\,\mathstrut_{p+1}F_{q+1}\biggl(
\begin{matrix}a_1,\dots,a_{p+1}\\
b_1,\dots,b_{q+1}
\end{matrix}; z\biggr).
\end{align*}
Auflösen nach $\mathstrut_{p+1}F_{q+1}$ ergibt die behauptete
Formel.
\end{proof}

Auch die Euler-Transformation lässt sich mit Hilfe der Substitution
$t=\sin^2 s$ in eine alternative Parametrisierung umschreiben.
Sie ist
\begin{align*}
\mathstrut_{p+1}F_{q+1}\biggl(
\begin{matrix}
a_1,\dots,a_{p+1}\\
b_1,\dots,b_{q+1}
\end{matrix}
;z
\biggr)
&=
\frac{2\Gamma(b_{q+1})}{\Gamma(a_{p+1})\Gamma(b_{q-1}-a_{p+1})}
\\
&\quad\times
\int_0^{\frac{\pi}2}
\sin^{2a_{p+1}-2}(s)\, \cos^{2b_{q+1}-2a_{p+1}-2}(s)
\,
\mathstrut_pF_q\biggl(
\begin{matrix}
a_1,\dots,a_p\\
b_1,\dots,b_q
\end{matrix};z\sin^2 s
\biggr)
\sin s\cos s
\,ds
\\
&=
\frac{2\Gamma(b_{q+1})}{\Gamma(a_{p+1})\Gamma(b_{q-1}-a_{p+1})}
\\
&\quad\times
\int_0^{\frac{\pi}2}
\sin^{2a_{p+1}-1}(s)\, \cos^{2b_{q+1}-2a_{p+1}-1}(s)
\,
\mathstrut_pF_q\biggl(
\begin{matrix}
a_1,\dots,a_p\\
b_1,\dots,b_q
\end{matrix};z\sin^2 s
\biggr)
\,ds.
\end{align*}
