%
% erweiterungen.tex
%
% (c) 2022 Prof Dr Andreas Müller, OST Ostschweizer Fachhochschlue
%
\subsection{Körpererweiterungen
\label{buch:integral:subsection:koerpererweiterungen}}
Das Beispiel des Körpers $\mathbb{Q}(\!\sqrt{2})$ auf Seite
\pageref{buch:integral:beispiel:Qsqrt2} illustriert eine Möglichkeit,
einen kleinen Körper zu vergrössern.
Das Prinzip ist verallgemeinerungsfähig und soll in diesem Abschnitt
erarbeitet werden.

%
% algebraische Zahl-Erweiterungen
\subsubsection{Algebraische Erweiterungen}
Der Körper $\mathbb{Q}(\!\sqrt{2})$ entsteht aus dem Körper $\mathbb{Q}$
dadurch, dass man die Zahl $\sqrt{2}$ hinzufügt und alle erlaubten
arithmetischen Operationen zulässt.
Die Darstellung von Elementen von $\mathbb{Q}(\!\sqrt{2})$ als
$a+b\sqrt{2}$ ist möglich, weil die Zahl $\alpha=\sqrt{2}$ die 
algebraische Relation
\[
\alpha^2-2 = \sqrt{2}^2 -2 = 0
\]
erfüllt.
Voraussetzung für diese Aussage ist, dass es die Zahl $\sqrt{2}$ in einem
geeigneten grösseren Körper gibt. 
Die reellen oder komplexen Zahlen bilden einen solchen Körper.
Wir verallemeinern diese Situation wie folgt.

\begin{definition}
Ist $K$ ein Körper, dann heisst ein Körper $L$ mit $K\subset L$ ein
{\em Erweiterungskörper} von $K$.
\index{Erweiterungskoerper@Erweiterungskörper}
\end{definition}

\begin{definition}
\label{buch:integral:definition:algebraisch}
Sei $K\subset L$ eine Körpererweiterung.
Das Element $\alpha\in L$ heisst {\em algebraisch} über $K$, wenn es
ein Polynom $p(x)\in K[x]$ gibt derart, dass $\alpha$ eine Nullstelle
von $p(x)$ ist, also gibt mit $p(\alpha)=0$.
Das normierte Polynom $m(x)$ geringsten Grades, welches $m(\alpha)=0$
erfüllt, heisst das {\em Minimalpolynom} von $\alpha$.
\index{Minimalpolynom}%
\end{definition}

Man sagt auch $\alpha$ ist algebraisch vom Grad $n$, wenn das Minimalpolynom
den Grad $n$ hat.
Wenn $\alpha\ne 0$ algebraisch ist, dann ist auch $1/\alpha$ algebraisch,
wie das folgende Argument zeigt.
Für das Minimalpolynom $m(x)$ von $\alpha$, ist $m(\alpha)=0$.
Teilt man diese Gleichung durch $\alpha^n$ teilt, erhält man 
\[
m_0\frac{1}{\alpha^n}
+
m_1\frac{1}{\alpha^{n-1}}
+
\ldots
+
m_{n-1}\frac{1}{\alpha}
+
1
=
0,
\]
das Polynom
\[
\hat{m}(x)
=
m_0x^n + m_1x^{n-1} + \ldots m_{n-1} x + 1
\in
K[x]
\]
hat also $\alpha^{-1}$ als Nullstelle.
Das Polynom $\hat{m}(x)$ beweist daher, dass $\alpha^{-1}$ algebraisch ist.

Die Zahl $\sqrt{2}\in\mathbb{R}$ ist also algebraisch über $\mathbb{Q}$
und jede andere Quadratwurzel von Elementen von $\mathbb{Q}$ ist
ebenfalls algebraisch über $\mathbb{Q}$.
Auch der Körper $\mathbb{Q}(\alpha)$ kann für jede andere Quadratwurzel
auf die genau gleiche Art wie für $\sqrt{2}$ konstruiert werden.

\begin{definition}
\label{buch:integral:definition:algebraischeerweiterung}
Sei $K\subset L$ eine Körpererweiterung und $\alpha\in L$ ein algebraisches
Element mit Minimalpolynom $m(x)\in K[x]$.
Dann heisst die Menge
\begin{equation}
K(\alpha)
=
\{
a_0 + a_1\alpha + \ldots +a_n\alpha^n
\;|\;
a_i\in K
\}
\label{buch:integral:eqn:algelement}
\end{equation}
mit $n=\deg m(x) - 1$ der durch Adjunktion von $\alpha$ erhaltene
Erweiterungsköper.
\end{definition}

Wieder muss nur überprüft werden, dass jedes Produkt oder jeder
Quotient von Ausdrücken der Form~\eqref{buch:integral:eqn:algelement}
wieder in diese Form gebracht werden kann.
Dazu sei
\[
m(x)
=
m_0+m_1x + m_2x^2
+\ldots +m_{n-1}x^{n-1} + x^n
\]
das Minimalpolynom von $\alpha$.
Die Gleichung $m(\alpha)=0$ kann nach $\alpha^n$ aufgelöst werden und
liefert
\[
\alpha^n = -m_0 - m_1\alpha - m_2\alpha^2 -\ldots -m_{n-1}\alpha^{n-1}.
\]
Damit kann jede Potenz von $\alpha$ mit einem Exponenten grösser als $n$
in eine Linearkombination von Potenzen mit kleineren Exponenten
reduziert werden.
Ein Polynom in $\alpha$ kann also immer auf die
Form~\eqref{buch:integral:eqn:algelement}
gebracht werden.

XXX Quotienten

% rationale Funktionen als Körpererweiterungen
\subsubsection{Rationale Funktionen als Körpererweiterung}

% Erweiterungen mit algebraischen Funktionen 
\subsubsection{Algebraische Funktionen}

% Transzendente Körpererweiterungen
\subsubsection{Transzendente Erweiterungen}


