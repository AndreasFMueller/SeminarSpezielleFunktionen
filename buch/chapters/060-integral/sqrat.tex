%
% sqrat.tex
%
% (c) 2022 Prof Dr Andreas Müller, OST Ostschweizer Fachhochschlue
%
\subsection{Integranden der Form $R(x,\sqrt{ax^2+bx+c})$
\label{buch:integral:subsection:rxy}}
Für rationale Funktionen lässt sich immer eine Stammfunktion in einem
Erweiterungskörper angeben, der durch hinzufügen einzelner logarithmischer
Funktionen entsteht.
Die dabei verwendeten Techniken lassen sich verallgemeinern.
Zur Illustration und Motivation des später beschriebenen Risch-Algorithmus
stellen wir uns in diesem Abschnitt der Aufgabe, Integrale
mit einem Integranden zu berechnen, der eine rationale Funktion von $x$
und $\sqrt{ax^2+bx+c}$ ist.

%
% Aufgabenstellung
%
\subsubsection{Aufgabenstellung}
Eine rationale Funktion von $x$ und $\sqrt{ax^2+bx+c}$ ist ein
Element des Differentialkörpers, den man aus $\mathbb{Q}(x)$ durch
hinzufügen des Elementes
\[
y=\sqrt{ax^2+bx+c}
\]
erhält.
Eine Funktion $f\in\mathbb{Q}(x,y)$ kann geschrieben werden als Bruch
\begin{equation}
f
=
\frac{
\tilde{p}_0 + \tilde{p}_1y + \dots + \tilde{p}_n y^n
}{
\tilde{q}_0 + \tilde{q}_1y + \dots + \tilde{q}_m y^m
}
\label{buch:integral:sqrat:eqn:ftilde}
\end{equation}
mit rationalen Koeffizienten $\tilde{p}_i,\tilde{q}_i\in\mathbb{Q}(x)$.
Gesucht ist eine Stammfunktion von $f$.

%
% Algebraische Vereinfachungen
%
\subsubsection{Algebraische Vereinfachungen}
Da $x^2=ax^2+bx+c$ ein Polynom ist, sind auch alle geraden Potenzen
von $y$ Polynome in $\mathbb{Q}(x)$,
und die ungeraden Potenzen von $y$ lassen sich als Produkt aus einem
Polynom und dem Faktor $y$ schreiben.
Der Integrand~\eqref{buch:integral:sqrat:eqn:ftilde} 
lässt sich daher vereinfachen zu einem Bruch der Form
\begin{equation}
f(x)
=
\frac{p_0+p_1y}{q_0+q_1y},
\label{buch:integral:sqrat:eqn:moebius}
\end{equation}
wobei $p_i$ und $q_i$ rationale Funktionen in $\mathbb{Q}(x)$ sind.

%
% Rationalisieren
%
\subsubsection{Rationalisieren}
Unschön an der Form~\eqref{buch:integral:sqrat:eqn:moebius} ist die
Tatsache, dass $y$ sowohl im Nenner wie auch im Zähler auftreten kann.
Da aber $y$ die quadratische Identität $y^2=ax^2+bx+c$ erfüllt,
kann das $y$ im Nenner durch Erweitern mit $q_0-q_1y$ zum verschwinden
gebracht werden.
Die Rechnung ergibt
\begin{align*}
\frac{p_0+p_1y}{q_0+q_1y}
&=
\frac{p_0+p_1y}{q_0+q_1y}
\cdot
\frac{q_0-q_1y}{q_0-q_1y}
=
\frac{(p_0+p_1y)(q_0-q_1y)}{q_0^2-q_1^2y^2}
\\
&=
\frac{p_0q_0-p_1q_1(ax^2+bx+c)}{q_0^2-q_1^2(ax^2+bx+c)}
+
\frac{q_0p_1-q_1p_0}{q_0^2-q_1^2(ax^2+bx+c)} y.
\end{align*}
Die Quotienten enthalten $y$ nicht mehr, sind also in $\mathbb{Q}(x)$.
In der späteren Rechnung stellt sich heraus, dass es praktischer ist,
das $y$ im Nenner zu haben, was man durch erweitern mit $y$ wieder
unter Ausnützung von $y^2=ax^2+bx+c$ erreichen kann.
Die zu integrierende Funktion  kann also in der Form
\begin{equation}
f(x)
=
W_1 + W_2\frac{1}{y}
\end{equation}
geschrieben werden mit rationalen Funktionen
$W_1,W_2\in\mathbb{Q}(x)$.
Eine Stammfunktion von $W_1$ kann mit der Methode von
Abschnitt~\ref{buch:integral:subsection:rationalefunktionen}
gefunden werden.
Im Folgenden kümmern wir uns daher nur noch um $W_1$.

\subsubsection{Polynomdivision}

\subsubsection{Integranden der Form $p(x)/y$}

\subsubsection{Partialbruchzerlegung}

\begin{equation}
\int
\frac{1}{(x-\alpha)^k \sqrt{ax^2+bx+c}}
\label{buch:integral:sqrat:eqn:2teart}
\end{equation}

\subsubsection{Integrale der Form \eqref{buch:integral:sqrat:eqn:2teart}}





