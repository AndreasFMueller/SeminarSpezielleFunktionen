%
% 32-erweiterungen.tex
%
% (c) 2022 Prof Dr Andreas Müller, OST Ostschweizer Fachhochschlue
%
\subsection{Körpererweiterungen
\label{buch:integral:subsection:koerpererweiterungen}}
Das Beispiel des Körpers $\mathbb{Q}(\!\sqrt{2})$ auf Seite
\pageref{buch:integral:beispiel:Qsqrt2} illustriert eine Möglichkeit,
einen kleinen Körper zu vergrössern.
Das Prinzip ist verallgemeinerungsfähig und soll in diesem Abschnitt
erarbeitet werden.

%
% algebraische Zahl-Erweiterungen
%
\subsubsection{Algebraische Erweiterungen}
Der Körper $\mathbb{Q}(\!\sqrt{2})$ entsteht aus dem Körper $\mathbb{Q}$
dadurch, dass man die Zahl $\sqrt{2}$ hinzufügt und alle erlaubten
arithmetischen Operationen zulässt.
Die Darstellung von Elementen von $\mathbb{Q}(\!\sqrt{2})$ als
$a+b\sqrt{2}$ ist möglich, weil die Zahl $\alpha=\sqrt{2}$ die 
algebraische Relation
\[
\alpha^2-2 = \sqrt{2}^2 -2 = 0
\]
erfüllt.
Voraussetzung für diese Aussage ist, dass es die Zahl $\sqrt{2}$ in einem
geeigneten grösseren Körper gibt. 
Die reellen oder komplexen Zahlen bilden einen solchen Körper.
Wir verallgemeinern diese Situation wie folgt.

\begin{definition}
Ist $K$ ein Körper, dann heisst ein Körper $L$ mit $K\subset L$ ein
{\em Erweiterungskörper} von $K$.
\index{Erweiterungskoerper@Erweiterungskörper}
\end{definition}

\begin{definition}
\label{buch:integral:definition:algebraisch}
Sei $K\subset L$ eine Körpererweiterung.
Das Element $\alpha\in L$ heisst {\em algebraisch} über $K$, wenn es
ein Polynom $p(x)\in K[x]$ gibt derart, dass $\alpha$ eine Nullstelle
von $p(x)$ ist, also mit $p(\alpha)=0$.
Das normierte Polynom $m(x)$ geringsten Grades, welches $m(\alpha)=0$
erfüllt, heisst das {\em Minimalpolynom} von $\alpha$.
\index{Minimalpolynom}%
\end{definition}

Man sagt auch, $\alpha$ ist algebraisch vom Grad $n$, wenn das Minimalpolynom
den Grad $n$ hat.
Wenn $\alpha\ne 0$ algebraisch ist, dann ist auch $1/\alpha$ algebraisch,
wie das folgende Argument zeigt.
Für das Minimalpolynom $m(x)$ von $\alpha$, ist $m(\alpha)=0$.
Teilt man diese Gleichung durch $\alpha^n$ teilt, erhält man 
\[
m_0\frac{1}{\alpha^n}
+
m_1\frac{1}{\alpha^{n-1}}
+
\ldots
+
m_{n-1}\frac{1}{\alpha}
+
1
=
0,
\]
das Polynom
\[
\hat{m}(x)
=
m_0x^n + m_1x^{n-1} + \ldots m_{n-1} x + 1
\in
K[x]
\]
hat also $\alpha^{-1}$ als Nullstelle.
Das Polynom $\hat{m}(x)$ beweist daher, dass $\alpha^{-1}$ algebraisch ist.

Die Zahl $\sqrt{2}\in\mathbb{R}$ ist also algebraisch über $\mathbb{Q}$
und jede andere Quadratwurzel von Elementen von $\mathbb{Q}$ ist
ebenfalls algebraisch über $\mathbb{Q}$.
Auch der Körper $\mathbb{Q}(\alpha)$ kann für jede andere Quadratwurzel
auf die genau gleiche Art wie für $\sqrt{2}$ konstruiert werden.

\begin{definition}
\label{buch:integral:definition:algebraischeerweiterung}
Sei $K\subset L$ eine Körpererweiterung und $\alpha\in L$ ein algebraisches
Element mit Minimalpolynom $m(x)\in K[x]$.
Dann heisst die Menge
\begin{equation}
K(\alpha)
=
\{
a_0 + a_1\alpha + \ldots +a_n\alpha^n
\;|\;
a_i\in K
\}
\label{buch:integral:eqn:algelement}
\end{equation}
mit $n=\deg m(x) - 1$ der durch {\em Adjunktion} oder Hinzufügen
von $\alpha$ erhaltene Erweiterungsköper.
\index{Adjunktion}%
\end{definition}

Wieder muss nur überprüft werden, dass jedes Produkt oder jeder
Quotient von Ausdrücken der Form~\eqref{buch:integral:eqn:algelement}
wieder in diese Form gebracht werden kann.
Dazu sei
\[
m(x)
=
m_0+m_1x + m_2x^2
+\ldots +m_{n-1}x^{n-1} + x^n
\]
das Minimalpolynom von $\alpha$.
Die Gleichung $m(\alpha)=0$ kann nach $\alpha^n$ aufgelöst werden und
liefert
\[
\alpha^n = -m_0 - m_1\alpha - m_2\alpha^2 -\ldots -m_{n-1}\alpha^{n-1}.
\]
Damit kann jede Potenz von $\alpha$ mit einem Exponenten grösser als $n$
in eine Linearkombination von Potenzen mit kleineren Exponenten
reduziert werden.
Ein Polynom in $\alpha$ kann also immer auf die
Form~\eqref{buch:integral:eqn:algelement}
gebracht werden.

Es muss aber noch gezeigt werden, dass auch der Kehrwert eines Elements
der Form~\eqref{buch:integral:eqn:algelement} in dieser Form geschrieben
werden kann.
Sei also $a(\alpha)$ so ein Element, dann sind die beiden Polynome
$a(x)$ und $m(x)$ teilerfremd, der grösste gemeinsame Teiler ist $1$.
Mit dem erweiterten euklidischen Algorithmus kann man zwei Polynome
$s(x)$ und $t(x)$ finden derart, dass $s(x)a(x)+t(x)m(x)=1$.
Setzt man $\alpha$ für $x$ ein, verschwindet das Minimalpolynom und
es bleibt
\[
s(\alpha)a(\alpha) = 1
\qquad\Rightarrow\qquad
s(\alpha) = \frac{1}{a(\alpha)}.
\]
Damit ist $s(\alpha)$ eine Darstellung von $1/a(\alpha)$ in der 
Form~\eqref{buch:integral:eqn:algelement}.

%
% Komplexe Zahlen
%
\subsubsection{Komplexe Zahlen}
Die imaginäre Einheit $i$ hat die Eigenschaft, dass $i^2=-1$, insbesondere
\index{imaginäre Einheit}%
\index{Realteil}%
\index{Imaginärteil}%
ist sie Nullstelle des Polynoms $m(x)=x^2+1\in\mathbb{Q}[x]$.
Die Menge $\mathbb{Q}(i)$ ist daher eine algebraische Körpererweiterung
von $\mathbb{Q}$ bestehend aus den komplexen Zahlen mit rationalem
Real- und Imaginärteil.

%
% Transzendente Körpererweiterungen
%
\subsubsection{Transzendente Erweiterungen}
Nicht alle Zahlen in $\mathbb{R}$ sind algebraisch.
Lindemann bewies 1882 einen allgemeinen Satz, aus dem folgt,
\index{Lindemann}%
dass $\pi$ und $e$ nicht algebraisch sind, es gibt also
kein Polynom mit rationalen Koeffizienten, welches $\pi$
oder $e$ als Nullstelle hat.
\index{e@$e$}%
\index{pi@$\pi$}%

\begin{definition}
Eine Zahl $\alpha\in L$ in einer Körpererweiterung $K\subset L$ 
heisst {\em transzendent}, wenn $\alpha$ nicht algebraisch ist,
\index{transzendent}%
wenn es also kein Polynom in $K[x]$ gibt, welches $\alpha$ als
Nullstelle hat.
\end{definition}

Die Zahlen $\pi$ und $e$ sind also transzendent.
Eine andere Art, diese Eigenschaft zu beschreiben ist zu sagen,
dass die Potenzen
\[
1=\pi^0, \pi, \pi^2,\pi^3,\dots
\]
linear unabhängig sind.
Gäbe es nämlich eine lineare Abhängigkeit, dann gäbe es Koeffizienten
$l_i$ derart, dass
\[
l_0 + l_1\pi^1 + l_2\pi^2 + \ldots + l_{n-1}\pi^{n-1} + l_{n}\pi^n = l(\pi)=0,
\]
und damit wäre dann ein Polynom gefunden, welches $\pi$ als Nullstelle hat.

Selbstverstländlich kann man zu einem transzendenten Element $\alpha$
immer noch einen Körper konstruieren, der alle Zahlen enthält, welche man
mit den arithmetischen Operationen aus $\alpha$ bilden kann.
Man kann ihn schreiben als
\[
K(\alpha)
=
\biggl\{
\frac{p(\alpha)}{q(\alpha)}
\;\bigg|\;
p(x),q(x)\in K[x] \wedge p(x)\ne 0
\biggr\},
\]
aber die Vereinfachungen zur
Form~\eqref{buch:integral:eqn:algelement}, die bei einem algebraischen
Element $\alpha$ möglich waren, können jetzt nicht mehr durchgeführt
werden.
$K\subset K(\alpha)$ ist zwar immer noch eine Körpererweiterung, aber
$K(\alpha)$ ist nicht mehr ein endlichdimensionaler Vektorraum.
Die Körpererweiterung $K\subset K(\alpha)$ heisst {\em transzendent}.

%
% rationale Funktionen als Körpererweiterungen
%
\subsubsection{Rationale Funktionen als Körpererweiterung}
Die unabhängige Variable wird bei Rechnen so behandelt, dass die
Potenzen alle linear unabhängig sind.
Dies ist die Grundlage für den Koeffizientenvergleich.
Der Körper der rationalen Funktion $K(x)$
ist also eine transzendente Körpererweiterung von $K$.

%
% Erweiterungen mit algebraischen Funktionen 
%
\subsubsection{Algebraische Funktionen}
Für das Integrationsproblem möchten wir nicht nur rationale Funktionen
verwenden können, sondern auch Wurzelfunktionen.
Wir möchten also zum Beispiel auch mit der Funktion $\sqrt{ax^2+bx+c}$
und allem, was man mit arithmetischen Operationen daraus machen kann,
arbeiten können.
Eine Körpererweiterung, die $\sqrt{ax^2+bx+c}$ enthält, enthält auch
alles, was man daraus bilden kann.
Doch wie bekommen wir die Funktion $\sqrt{ax^2+bx+c}$ in den Körper?

Die Art und Weise, wie man Wurzeln in der Schule kennenlernt ist als
eine neue Operation, die zu einer Zahl die Quadratwurzel liefert.
Diese Idee, den Körper mit einer weiteren Funktion anzureichern,
führt über nicht auf eine nützliche neue algebraische Struktur.
Wir dürfen daher $\sqrt{ax^2+bx+c}$ nicht als die Zusammensetzung
einer einzelnen neuen Funktion $\sqrt{\phantom{A}}$ mit
einem Polynom betrachten.

Die Wurzel $\sqrt{ax^2+bx+c}$ ist aber auch die Nullstelle des Polynoms
\[
p(z)
=
z^2 - [ax^2+bx+c]
\in
K(x)[z]
\]
mit Koeffizienten in $K(x)$.
Die eckigen Klammern sollen helfen, die Koeffizienten in $K(x)$
zu erkennen.
Die Funktion $\sqrt{ax^2+bx+c}$ ist also algebraisch über $K(x)$.
Einen Funktionenkörper, der die Funktion enthält, kann man nun erhalten,
indem man den Körper $K(x)$ um das über $K(x)$ algebraische Element
$y=\sqrt{ax^2+bx+c}$ zu $K(x,y)=K(x,\sqrt{ax^2+bx+c})$ erweitert.
Wurzelfunktion werden daher nicht als Zusammensetzungen von Polynomen
mit einer Wurzeloperation, sondern als
algebraische Erweiterungen eines Funktionenkörpers betrachtet.

%
% Konjugation
%
\subsubsection{Konjugation}
Die komplexen Zahlen sind die algebraische Erweiterung der reellen Zahlen
um die Nullstelle $i$ des Polynoms $m(x)=x^2+1$.
Die Zahl $-i$ ist aber auch eine Nullstelle von $m(x)$, die mit algebraischen
Mitteln nicht von $i$ unterscheidbar ist.
Die komplexe Konjugation $a+bi\mapsto a-bi$ vertauscht die beiden 
\index{Konjugation, komplexe}%
\index{komplexe Konjugation}%
Nullstellen des Minimalpolynoms.

Ähnliches gilt für die Körpererweiterung $\mathbb{Q}(\!\sqrt{2})$.
$\sqrt{2}$ und $-\sqrt{2}$ sind beide Nullstellen des Minimalpolynoms
$m(x)=x^2-2$, die mit algebraischen Mitteln nicht unterschiedbar sind.
Sie haben zwar verschiedene Vorzeichen, doch ohne eine Ordnungsrelation
können diese nicht unterschieden werden.
\index{Ordnungsrelation}%
Eine Ordnungsrelation zwischen rationalen Zahlen lässt sich zwar
definieren, aber die Zahl $\sqrt{2}$ ist nicht rational, es braucht
also eine zusätzliche Annahme, zum Beispiel die Identifikation von
$\sqrt{2}$ mit einer reellen Zahl in $\mathbb{R}$, wo der Vergleich
möglich ist.

Auch in $\mathbb{Q}(\!\sqrt{2})$ ist die Konjugation
$a+b\sqrt{2}\mapsto a-b\sqrt{2}$ eine Selbstabbildung, die
die Körperoperationen respektiert.

Als weiteres Beispiel für diese Konjugation betrachten wir
das Polynom $m(x)=x^2-x-1$ mit den Nullstellen
\[
\frac12 \pm\sqrt{\biggl(\frac12\biggr)^2+1}
=
\frac{1\pm\sqrt{5}}{2}
=
\left\{
\bgroup
\renewcommand{\arraystretch}{2.20}
\renewcommand{\arraycolsep}{2pt}
\begin{array}{lcl}
\displaystyle
\frac{1+\sqrt{5}}{2} &=& \phantom{-}\varphi \\
\displaystyle
\frac{1-\sqrt{5}}{2} &=& \displaystyle-\frac{1}{\varphi}.
\end{array}
\egroup
\right.
\]
Sie erfüllen die gleiche algebraische Relation $x^2=x+1$.
Sie sind sowohl im Vorzeichen wie auch im absoluten Betrag 
verschieden, beides verlangt jedoch eine Ordnungsrelation als
Voraussetzung, die uns fehlt.
Aus beiden kann man mit rationalen Operationen $\sqrt{5}$ gewinnen,
denn
\[
\sqrt{5}
=
4\varphi-1
=
-4\biggl(-\frac{1}{\varphi}\biggr)^2-1
\qquad\Rightarrow\qquad
\mathbb{Q}(\!\sqrt{5})
=
\mathbb{Q}(\varphi)
=
\mathbb{Q}(-1/\varphi).
\]
Die Abbildung $a+b\varphi\mapsto a-b/\varphi$ ist eine Selbstabbildung
des Körpers $\mathbb{Q}(\!\sqrt{5})$, welche die beiden Nullstellen 
vertauscht.

Dieses Phänomen gilt für jede algebraische Erweiterung.
Die Nullstellen des Minimalpolynoms, welches die Erweiterung
definiert, sind grundsätzlich nicht unterscheidbar.
Mit der Adjunktion einer Nullstelle enthält der Erweiterungskörper
auch alle anderen.
Sind $\alpha_1$ und $\alpha_2$ zwei Nullstellen des Minimalpolynoms,
dann definiert die Abbildung $\alpha_1\mapsto\alpha_2$ eine Selbstabbildung,
die die Nullstellen permutiert.

Die algebraische Körpererweiterung
$\mathbb{Q}(x)\subset \mathbb{Q}(x,\sqrt{ax^2+bx+c})$
ist nicht unterscheidbar von
$\mathbb{Q}(x)\subset \mathbb{Q}(x,-\!\sqrt{ax^2+bx+c})$.
Für das Integrationsproblem bedeutet dies, dass alle Methoden so
formuliert werden müssen, dass die Wahl der Nullstellen auf die
Lösung keinen Einfluss haben.


