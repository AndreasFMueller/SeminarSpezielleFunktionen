%
% elementar.tex
%
% (c) 2022 Prof Dr Andreas Müller, OST Ostschweizer Fachhochschlue
%
\subsection{Elementare Funktionen
\label{buch:integral:subsection:elementar}}
Etwas allgemeiner kann man sagen, dass in den
Beispielen~\eqref{buch:integration:risch:eqn:integralbeispiel2}
algebraische Erweiterungen von $\mathbb{Q}(x)$ und Erweiterungen
um Logarithmen oder Exponentialfunktionen vorgekommen sind.
Die Stammfunktionen verwenden dieselben Funktionen oder höchstens
Erweiterungen um Logarithmen von Funktionen, die man schon im
Integranden gesehen hat.

%
% Exponentielle und logarithmische Funktione
%
\subsubsection{Exponentielle und logarithmische Funktionen}
In Abschnitt~\ref{buch:integral:subsection:diffke} haben wir
bereits die Exponentialfunktion $e^x$ und die Logarithmusfunktion 
$\log x$ charakterisiert als eine Körpererweiterung durch 
Elemente, die der Differentialgleichung
\[
\exp' = \exp
\qquad\text{und}\qquad
\log' = \frac{1}{x}
\]
genügen.
Für die Stammfunktionen, die in 
Abschnitt~\ref{buch:integral:subsection:logexp}
gefunden wurden, sind aber Logarithmusfunktionen nicht von
$x$ sondern von beliebigen über $\mathbb{Q}$ algebraischen Elementen
nötig.
Um zu verstehen, wie wir diese Funktion als Körpererweiterung erhalten
könnten, betrachten wir die Ableitung einer Exponentialfunktion
$\vartheta(x) = \exp(f(x))$ und eines
Logarithmus 
$\psi(x) = \log(f(x))$, wie man sie mit der Kettenregel
berechnet hätte:
\begin{align*}
\vartheta'(x)
&=\exp(f(x)) \cdot f'(x)
&
\psi'(x)
&=
\frac{f'(x)}{f(x)}
\quad\Leftrightarrow\quad
f(x)\psi'(x)
=
f'(x).
\end{align*}
Dies motiviert die folgende Definition

\begin{definition}
\label{buch:integral:def:explog}
Sei $\mathscr{F}$ ein Differentialklörper und $f\in\mathscr{F}$.
Ein Exponentialfunktion von $f$ ist ein $\vartheta\in \mathscr{F}$mit
$\vartheta' = \vartheta f'$.
Ein Logarithmus von $f$ ist ein $\vartheta\in\mathscr{F}$ mit
$f\vartheta'=f'$.
\end{definition}

Für $f=x$ mit $f'=1$ reduziert sich die 
Definition~\ref{buch:integral:def:explog}
auf die Definition der Exponentialfunktion $\exp(x)$ und
Logarithmusfunktion $\log(x)$ auf Seite~\pageref{buch:integral:expundlog}.


%
%
%
\subsubsection{Transzendente Körpererweiterungen}
Die Wurzelfunktionen haben wir früher als algebraische Erweiterungen
eines Differentialkörpers erkannt.
Die logarithmischen und exponentiellen Elemente gemäss
Definition~\ref{buch:integral:def:explog} sind nicht algebraisch.

\begin{definition}
\label{buch:integral:def:transzendent}
Sei $\mathscr{F}\subset\mathscr{G}$ eine Körpererweiterung und
$\vartheta\in\mathscr{G}$.
$\vartheta$ heisst {\em transzendent}, wenn $\vartheta$ nicht
algebraisch ist.
\end{definition}

\begin{beispiel}
Die Funktion $f = e^x + e^{2x} + e^{x/2}$ ist sicher transzendent,
in diesem Beispiel zeigen wir, dass es mindestens drei verschiedene
Möglichkeiten gibt, eine Körpererweiterung von $\mathbb{Q}(x)$ zu
konstruieren, die $f$ enthält.

Erste Möglichkeit: $f=\vartheta_1 + \vartheta_2 + \vartheta_3$ mit
$\vartheta_1=e^x$,
$\vartheta_2=e^{2x}$
und
$\vartheta_3=e^{x/2}$.
Jedes der Elemente $\vartheta_i$ ist exponentiell über $\mathbb{Q}(x)$ und 
$f$ ist in
\[
\mathbb{Q}(x)
\subset
\mathbb{Q}(x,\vartheta_1)
\subset
\mathbb{Q}(x,\vartheta_1,\vartheta_2)
\subset
\mathbb{Q}(x,\vartheta_1,\vartheta_2,\vartheta_3)
\ni
f.
\]
Jede dieser Körpererweiterungen ist transzendent.

Zweite Möglichkeit: $\vartheta_1=e^x$ ist exponentiell über 
$\mathbb{Q}(x)$ und $\mathbb{Q}(x,\vartheta_1)$ enthält wegen
\[
(\vartheta_1^2)'
=
2\vartheta_1\vartheta_1'
=
2\vartheta_1^2,
\]
somit ist $\vartheta_1^2=\vartheta_2$ eine Exponentialfunktion von $2x$
über $\mathbb{Q}(x)$.
Das Element $\vartheta_3=e^{x/2}$ ist zwar auch exponentiell über
$\mathbb{Q}(x)$, es ist aber auch eine Nullstelle des Polynoms
$m(z)=z^2-[\vartheta_1]$.
Die Erweiterung
$\mathbb{Q}(x,\vartheta_1)\subset\mathbb{Q}(x,\vartheta_1,\vartheta_3)$
ist eine algebraische Erweiterung, die
$f=\vartheta_1 + \vartheta_1^2+\vartheta_3$ enthält.

Dritte Möglichkeit: $\vartheta_3=e^{x/2}$ ist exponentiell über
$\mathbb{Q}(x)$.
Die transzendente Körpererweiterung
\[
\mathbb{Q}(x) \subset \mathbb{Q}(x,\vartheta_3)
\]
enthält das Element
$f=\vartheta_3^4+\vartheta_3^2 + \vartheta_3 $.
\end{beispiel}

Das Beispiel zeigt, dass man nicht sagen kann, dass eine Funktion
ausschliesslich in einer algebraischen oder transzendenten Körpererweiterung
zu finden ist. 
Vielmehr gibt es für die gleiche Funktion möglicherweise verschiedene
Körpererweiterungen, die alle die Funktion enthalten können.

%
% Elementare Funktionen
%
\subsubsection{Elementare Funktionen}
Die Stammfunktionen~\eqref{buch:integration:risch:eqn:integralbeispiel2}
können aufgebaut werden, indem man dem Körper $\mathbb{Q}(x)$ schrittweise
sowohl algebraische wie auch transzendente Elemente hinzufügt,
wie in der folgenden Definition, die dies für abstrakte
Differentialkörpererweiterungen formuliert.

\begin{definition}
Eine Körpererweiterung $\mathscr{F}\subset\mathscr{G}$ heisst 
{\em transzendente elementare Erweiterung}, wenn 
$\mathscr{G} = \mathscr{F}(\vartheta_1,\dots,\vartheta_n)$ und
jedes der Element $\vartheta_i$ transzendent und logarithmisch oder
exponentiell ist über
$\mathscr{F}_{i-1}=\mathscr{F}(\vartheta_1,\dots,\vartheta_{i-1})$.
Die Körpererweiterung $\mathscr{F}\subset\mathscr{G}$ heisst
{\em elementare Erweiterung}, wenn 
$\mathscr{G} = \mathscr{F}(\vartheta_1,\dots,\vartheta_n)$ und
jedes Element $\vartheta_i$ ist entweder logarithmisch, exponentiell
oder algebraisch über $\mathscr{F}_{i-1}$.
\end{definition}

Die Funktionen, die als akzeptable Stammfunktionen für das Integrationsproblem
in Betracht kommen, sind also jene, die in einer geeigneten elementaren
Erweiterung des von $\mathbb{Q}(x)$ liegen.
Ausserdem können auch noch weitere Konstanten nötig sein, sowohl
algebraische Zahlen wie auch Konstanten wie $\pi$ oder $e$.

\begin{definition}
Sei $\mathscr{K}(x)$ der Differentialklörper der rationalen Funktionen
über dem Konstantenkörper $\mathscr{K}\supset\mathbb{Q}$, der in $\mathbb{C}$
enthalten ist.
Ist $\mathscr{F}\supset \mathscr{K}(x)$ eine transzendente elementare 
Erweiterung von $\mathscr{K}(x)$, dann heisst $\mathscr{F}$
ein Körper von {\em transzendenten elementaren Funktionen}.
Ist $\mathscr{F}$ eine elementare Erweiterung von $\mathscr{K}(x)$, dann
heisst $\mathscr{F}$ ein Körper von {\em elementaren Funktionen}.
\end{definition}

\subsubsection{Das Integrationsproblem}
Die elementaren Funktionen enthalten alle Funktionen, die sich mit
arithmetischen Operationen, Wurzeln, Exponentialfunktionen, Logarithmen und
damit auch mit trigonometrischen und hyperbolischen Funktionen und ihren
Umkehrfunktionen aus den rationalen Zahlen, der unabhängigen Variablen $x$ 
und möglicherweise einigen zusätzlichen Konstanten aufbauen lassen.
Sei also $f$ eine Funktion in einem Körper von elementaren
Funktionen
\[
\mathscr(F)
=
\mathbb{Q}(\alpha_1,\dots,\alpha_l)(x,\vartheta_1,\dots,\vartheta_n).
\]
Eine elementare Stammfunktion ist eine Funktion $F=\int f$ in einer
elementaren Körpererweiterung
\[
\mathscr{G}
=
\mathbb{Q}(\alpha_1,\dots,\alpha_l,\dots,\alpha_{l+k})
(x,\vartheta_1,\dots,\vartheta_n,\dots,\vartheta_{n+m})
\]
mit $F'=f$.
Das Ziel ist, $F$ mit Hilfe eines Algorithmus zu bestimmen.



