%
% elementar.tex
%
% (c) 2022 Prof Dr Andreas Müller, OST Ostschweizer Fachhochschlue
%
\subsection{Elementare Funktionen
\label{buch:integral:subsection:elementar}}
Etwas allgemeiner kann man sagen, dass in den
Beispielen~\eqref{buch:integration:risch:eqn:integralbeispiel2}
algebraische Erweiterungen von $\mathbb{Q}(x)$ und Erweiterungen
um Logarithmen oder Exponentialfunktionen vorgekommen sind.
Die Stammfunktionen verwenden dieselben Funktionen oder höchstens
Erweiterungen um Logarithmen von Funktionen, die man schon im
Integranden gesehen hat.

