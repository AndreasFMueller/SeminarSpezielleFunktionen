%
% irat.tex
%
% (c) 2022 Prof Dr Andreas Müller, OST Ostschweizer Fachhochschlue
%
\subsection{Integration rationaler Funktionen
\label{buch:integral:subsection:rationalefunktionen}}
Für die Integration der rationalen Funktionen lernt man in einem
Analysis-Kurs üblicherweise ein Lösungsverfahren.
Dies zeigt zunächst, dass rationale Funktionen immer eine Stammfunktion
in einem geeigneten Erweiterungskörper haben.
Es deutet aber auch an, dass Stammfunktionen eine ziemlich spezielle
Form haben, die später als
Satz von Liouville~\ref{buch:integral:satz:liouville}
ein besondere Rolle spielen wird.

%
% Aufgabenstellung
%
\subsubsection{Aufgabenstellung}
In diesem Abschnitt ist eine rationale Funktion $f(x)\in\mathbb{Q}(x)$
gegeben, deren Stammfunktion bestimmt werden soll.
Als rationale Funktion kann sie als Bruch
\begin{equation}
f(x) = \frac{p(x)}{q(x)}
\label{buch:integral:irat:eqn:quotient}
\end{equation}
mit Polynomen $p(x),q(x)\in\mathbb{Q}[x]$ geschrieben werden.
Gesucht ist ein Erweiterungskörper $\mathscr{K}\supset \mathbb{Q}(x)$ 
derart und eine Stammfunktion $F\in\mathscr{K}$ von $f$, also $F'=f$.

%
% Polynomdivision
%
\subsubsection{Polynomdivision}
Der Quotient~\eqref{buch:integral:irat:eqn:quotient} kann durch Polynomdivision
mit Rest vereinfacht werden in einen polynomialen Teil und einen echten
Bruch:
\begin{equation}
f(x)
=
g(x)
+
\frac{a(x)}{b(x)}
\label{buch:integral:irat:eqn:polydiv}
\end{equation}
mit Polynomen $g(x),a(x),b(x)\in\mathbb[Q](x)$ und $\deg a(x) < \deg b(x)$.
Für den ersten Summanden liefert
\eqref{buch:integral:iproblem:eqn:polyintegral} eine Stammfunktion.
Im Folgenden bleibt also nur noch der zweite Term zu behandeln.

%
% Partialbruchzerlegung
%
\subsubsection{Partialbruchzerlegung}
Zur Berechnung des Integral des Bruchs
in~\eqref{buch:integral:irat:eqn:polydiv} wird die Partialbruchzerlegung
benötigt.
Der Einfachheit halber nehmen wir an, dass wir den Körper $\mathbb{Q}(x)$
mit alle Nullstellen $\beta_i$ des Nenner-Polynoms $b(x)$ zu einem Körper
$\mathscr{K}$ erweitert haben, in dem Nenner in Linearfaktoren zerfällt.
Unter diesen Voraussetzungen hat die Partialbruchzerlegung die Form
\begin{equation}
\frac{a(x)}{b(x)}
=
\sum_{i=1}^m
\sum_{k=1}^{k_i}
\frac{A_{ik}}{(x-\beta_i)^k},
\label{buch:integral:irat:eqn:partialbruch}
\end{equation}
wobei $k_i$ die Vielfachheit der Nullstelle $\beta_i$ ist.
Die Koeffizienten $A_{ik}$ können zum Beispiel mit Hilfe eines linearen
Gleichungssystems bestimmt werden.

Um eine Stammfunktion zu finden, muss man also Stammfunktionen für
jeden einzelnen Summanden bestimmen.
Für Exponenten $k>1$ im Nenner eines Terms der
Partialbruchzerlegung~\eqref{buch:integral:irat:eqn:partialbruch}
kann dazu die Regel
\[
\int \frac{A_{ik}}{(x-\beta_i)^k}
=
\frac{A_{ik}}{(-k+1)(x-\beta_i)^{k-1}}
\]
verwendet werden.
Diese Stammfunktion liegt wieder in $\mathscr{K}(x)$ liegt.

%
% Körpererweiterungen
%
\subsubsection{Körpererweiterung}
Für $k=1$ ist eine logarithmische Erweiterung um die Funktion
\begin{equation}
\int \frac{A_{i1}}{x-\alpha_i}
=
A_{i1}
\log(x-\alpha_i)
\label{buch:integral:irat:eqn:logs}
\end{equation}
nötig.
Es gibt also eine Stammfunktion in einem Erweiterungskörper, sofern
er zusätzlich alle logarithmischen Funktionen
in~\ref{buch:integral:irat:eqn:logs} enthält.
Sie hat die Form
\[
\sum_{i=1}^m A_{i1} \log(x-\beta_i),
\]
wobei $A_{i1}\in\mathscr{K}$ ist.

Setzt man alle vorher schon gefundenen Teile der Stammfunktion zusammen,
kann man sehen, dass die Stammfunktion die Form
\begin{equation}
F(x) = v_0(x) + \sum_{i=1}^m c_i \log v_i(x)
\label{buch:integral:irat:eqn:liouvillstammfunktion}
\end{equation}
haben muss.
Dabei ist $v_0(x)\in\mathscr{K}(x)$ und besteht aus der Stammfunktion
des polynomiellen Teils und den Stammfunktionen der Terme der Partialbruchzerlegung mit Exponenten $k>1$.
Die logarithmischen Terme bestehen aus den Konstanten $c_i=A_{i1}$ 
und den Logarithmusfunktionen $v_i(x)=x-\beta_i\in\mathscr{K}(x)$.
Die Funktion $f(x)$ muss daher die Form
\[
f(x)
=
v_0'(x)
+
\sum_{i=1}^m c_i\frac{v'_i(x)}{v_i(x)}
\]
gehabt haben.
Die Form~\eqref{buch:integral:irat:eqn:liouvillstammfunktion}
der Stammfunktion ist nicht eine Spezialität der rationalen Funktionen.
Sie wird auch bei grösseren Funktionenkörpern immer wieder auftreten
und ist als Satz von Liouville bekannt.

%
% Minimale algebraische Erweiterung
%
\subsubsection{Minimale algebraische Erweiterung}
XXX Rothstein-Trager

