%
% risch.tex
%
% (c) 2021 Prof Dr Andreas Müller, OST Ostschweizer Fachhochschule
%
\section{Der Risch-Algorithmus
\label{buch:integral:section:risch}}
\rhead{Risch-Algorithmus}
Die Lösung des Integrationsproblem für $\mathbb{Q}(x)$ und für
$\mathbb{Q}(x,y)$ mit $y=\!\sqrt{ax^2+bx+c}$ hat gezeigt, dass
ein Differentialkörper genau die richtige Bühne für dieses Unterfangen
sein dürfte.
Die Stammfunktionen konnten in einem Erweiterungskörper gefunden
werden, der ein paar Logarithmen hinzugefügt worden sind.
Tatsächlich lässt sich in diesem Rahmen sogar ein Algorithmus
formulieren, der in einem noch zu definierenden Sinn ``elementare''
Funktionen als Stammfunktionen finden kann oder beweisen kann, dass
eine solche nicht existiert.
Dieser Abschnitt soll einen Überblick darüber geben.

%
% logexp.tex
%
% (c) 2022 Prof Dr Andreas Müller, OST Ostschweizer Fachhochschlue
%
\subsection{Log-Exp-Notation für elementare Funktionen
\label{buch:integral:subsection:logexp}}
Die Integration rationaler Funktionen hat bereits gezeigt, dass
eine Stammfunktion nicht immer im Körper der rationalen Funktionen
existiert.
Es kann notwendig sein, dem Körper logarithmische Erweiterungen der Form
$\log(x-\alpha)$ hinzuzufügen.

Es können jedoch noch ganz andere neue Funktionen auftreten, wie die
folgende Zusammenstellung einiger Stammfunktionen zeigt:
\begin{align*}
\int\frac{dx}{1+x^2}
&=
\arctan x
\\
\end{align*}







%
% elementar.tex
%
% (c) 2022 Prof Dr Andreas Müller, OST Ostschweizer Fachhochschlue
%
\subsection{Elementare Funktionen
\label{buch:integral:subsection:elementar}}
Etwas allgemeiner kann man sagen, dass in den
Beispielen~\eqref{buch:integration:risch:eqn:integralbeispiel2}
algebraische Erweiterungen von $\mathbb{Q}(x)$ und Erweiterungen
um Logarithmen oder Exponentialfunktionen vorgekommen sind.
Die Stammfunktionen verwenden dieselben Funktionen oder höchstens
Erweiterungen um Logarithmen von Funktionen, die man schon im
Integranden gesehen hat.


\input{chapters/060-integral/liouville.tex}



