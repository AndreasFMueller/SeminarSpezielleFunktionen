%
% rational.tex
%
% (c) 2022 Prof Dr Andreas Müller, OST Ostschweizer Fachhochschlue
%
\subsection{Rationale Funktionen und Funktionenkörper
\label{buch:integral:subsection:rational}}
Welche Funktionen sollen als Antwort auf die Frage nach einer Stammfunktion
akzeptiert werden?
Polynome in der unabhängigen Variablen $x$ sollten sicher dazu gehören,
also alles, was man mit Hilfe der Multiplikation, Addition und Subtraktion
aus Koeffizienten zum Beispiel in den rationalen Zahlen $\mathbb{Q}$ und
der unabhängigen Variablen aufbauen kann.
Doch welche weiteren Operationen sollen zugelassen werden und was lässt
sich über die entstehende Funktionenmenge aussagen?

\subsubsection{Körper}
Die kleinste Zahlenmenge, in der alle arithmetischen Operationen soweit
sinnvoll durchgeführt werden können, ist die Menge $\mathbb{Q}$ der
rationalen Zahlen.
Etwas formaler ist eine solche Menge, in der die Arithmetik uneingeschränkt
ausgeführt werden kann, ein Körper gemäss der folgenden Definition.
\index{Korper@Körper}%

\begin{definition}
\label{buch:integral:definition:koerper}
Eine {\em Körper} ist eine Menge $K$ mit zwei Verknüpfungen $+$, die Addition,
und $\cdot$, die Multiplikation,
welche die folgenden Eigenschaften haben.
\begin{center}
\renewcommand{\tabcolsep}{0pt}
\begin{tabular}{p{68mm}p{4mm}p{68mm}}
%Eigenschaften der
Addition:
\begin{enumerate}[{\bf A}.1)]
\item assoziativ: $(a+b)+c=a+(b+c)$
für alle $a,b,c\in K$
\item kommutativ: $a+b=b+a$
für alle $a,b\in K$
\item Neutrales Element der Addition: es gibt ein Element $0\in K$ mit
der Eigenschaft $a+0=a$ für alle $a\in K$
\item Additiv inverses Element: zu jedem Element $a\in K$ gibt es das Element
$-a$ mit der Eigenschaft $a+(-a)=0$.
\end{enumerate}
&&%
%Eigenschaften der
Multiplikation:
\begin{enumerate}[{\bf M}.1)]
\item assoziativ: $(a\cdot b)\cdot c=a\cdot (b\cdot c)$
für alle $a,b,c\in K$
\index{Assoziativgesetz}%
\index{assoziativ}%
\item kommutativ: $a\cdot b=b\cdot a$
für alle $a,b\in K$
\index{Kommutativgesetz}%
\index{kommutativ}%
\item Neutrales Element der Multiplikation: es gibt ein Element $1\in K$ mit
der Eigenschaft $a\cdot 1 =a$ für alle $a\in K$
\index{neutrales Element}%
\item Multiplikativ inverses Element: zu jedem Element
\index{inverses Element}%
$a\in K^*=K\setminus\{0\}$
gibt es das Element $a^{-1}$ mit der Eigenschaft $a\cdot a^{-1}=1$.
\index{Einheitengruppe}%
\index{Gruppe der invertierbaren Elemente}%
\end{enumerate}
\end{tabular}
\end{center}
\vspace{-22pt}
Ausserdem gilt das Distributivgesetz: für alle $a,b,c\in K$ gilt
$a\cdot(b+c)=a\cdot b + a\cdot c$.
\index{Disitributivgesetz}%
Die Menge $K^*$ heisst auch die {\em Einheitengruppe} oder die
{\em Gruppe der invertierbaren Elemente} des Körpers.
\end{definition}

Das Assoziativgesetz {\bf A}.1 besagt, dass Summen mit beliebig
vielen Termen ohne Klammern geschrieben werden kann, weil es nicht
darauf ankommt, in welcher Reihenfolge die Additionen ausgeführt werden.
Ebenso für das Assoziativgesetz {\bf M}.1 der Multiplikation.
Die Kommutativgesetze {\bf A}.2 und {\bf M}.2 implizieren, dass man
nicht auf die Reihenfolge der Summanden oder Faktoren achten muss.
Das Distributivgesetz schliesslich besagt, dass man Produkte ausmultiplizieren
oder gemeinsame Faktoren ausklammern kann, wie man es in der Schule
gelernt hat.

Die rellen Zahlen $\mathbb{R}$ und die komplexen Zahlen $\mathbb{C}$
bilden ebenfalls einen Körper, die von den rationalen Zahlen geerbten
Eigenschaften der Verknüpfungen setzen sich auf $\mathbb{R}$ und
$\mathbb{C}$ fort.
Es lassen sich allerdings auch Zahlkörper zwischen $\mathbb{Q}$ und
$\mathbb{R}$ konstruieren, wie das folgende Beispiel zeigt.

\begin{beispiel}
\label{buch:integral:beispiel:Qsqrt2}
Die Menge
\[
\mathbb{Q}(\!\sqrt{2})
=
\{
a+b\sqrt{2}
\;|\;
a,b\in \mathbb{Q}
\}
\]
ist eine Teilmenge von $\mathbb{R}$.
Die Rechenoperationen haben alle verlangten Eigenschaften, wenn gezeigt
werden kann, dass Produkte und Quotienten von Zahlen in $\mathbb{Q}(\!\sqrt{2})$
wieder in $\mathbb{Q}(\!\sqrt{2})$ sind.
Dazu rechnet man
\begin{align*}
(a+b\sqrt{2})
(c+d\sqrt{2})
&=
ac + 2bd + (ad+bc)\sqrt{2} \in \mathbb{Q}(\!\sqrt{2})
\intertext{und}
\frac{a+b\sqrt{2}}{c+d\sqrt{2}}
&=
\frac{a+b\sqrt{2}}{c+d\sqrt{2}}
\cdot
\frac{c-d\sqrt{2}}{c-d\sqrt{2}}
=
\frac{ac-2bd +(-ad+bc)\sqrt{2}}{c^2-2d^2}
\\
&=
\underbrace{\frac{ac-2bd}{c^2-2d^2}}_{\displaystyle\in\mathbb{Q}}
+
\underbrace{\frac{-ad+bc}{c^2-2d^2}}_{\displaystyle\in\mathbb{Q}}
\sqrt{2}
\in \mathbb{Q}(\!\sqrt{2}).
\qedhere
\end{align*}
\end{beispiel}

%
% Rationale Funktionen
%
\subsubsection{Rationalen Funktionen}
Die als Antworten auf die Frage nach einer Stammfunktion akzeptablen
Funktionen sollten alle rationalen Zahlen sowie die unabhängige
Variable $x$ enthalten.
Ausserdem sollte man beliebige arithmetische Operationen mit
diesen Ausdrücken durchführen können.
Mit Addition, Subtraktion und Multiplikation entstehen aus den
rationalen Zahlen und der unabhängigen Variablen die Polynome $\mathbb{Q}[x]$
(siehe auch Abschnitt~\ref{buch:potenzen:section:polynome}).


\begin{definition}
Die Menge
\[
\mathbb{Q}(x)
=
\biggl\{
\frac{p(x)}{q(x)}
\;\bigg|\;
p(x),q(x)\in\mathbb{Q}[x]
\wedge
q(x)\ne 0
\biggr\},
\]
bestehend aus allen Quotienten von Polynomen, deren Nenner nicht
das Nullpolynom ist, heisst der Körper der {\em rationalen Funktionen}
\index{rationale Funktion}%
mit Koeffizienten in $\mathbb{Q}$.
\end{definition}

Die Definition erlaubt, dass der Nenner Nullstellen hat, die sich in
Polen der Funktion äussern.
Die Eigenschaften eines Körpers sind sicher erfüllt, wenn wir uns
nur davon überzeugen können,
dass die arithmetischen Operationen nicht aus dieser Funktionenmenge
herausführen.
Dazu muss man nur verstehen, dass die Operation des gleichnamig Machens 
zweier Brüche auch für Nenner funktioniert, die Polynome sind, und die
Summe wzeier Brüche von Polynomen wieder in einen Bruch von Polynomen
umwandelt.

%
% Warum rationale Zahlen?
%
\subsubsection{Warum die Beschränkung auf rationale Zahlen?}
Aus mathematischer Sicht gibt es gute Gründe, Analysis im Körper $\mathbb{R}$
oder $\mathbb{C}$ zu betreiben.
Da Ableitung und Integral als Grenzwerte definiert sind, stellt diese
Wahl des Körpers sicher, dass die Grenzwerte auch tatsächlich existieren.
Der Fundamentalsatz der Algebra garantiert, dass über $\mathbb{C}$
jedes Polynome in Linearfaktoren zerlegt werden kann.

Der Einfachheit der Analyse in $\mathbb{R}$ oder $\mathbb{C}$ steht
die Schwierigkeit gegenüber, beliebige Elemente von $\mathbb{R}$ in
einem Computer exakt darzustellen.
Für Brüche in $\mathbb{Q}$ gibt es eine solche Darstellung durch
Paare von Ganzzahlen, wie sie die GNU Multiprecision Arithmetic Library
\cite{buch:gmp} realisiert.
Irrationale Zahlen dagegen können nur exakt gehandhabt werden, wenn
man im wesentlichen symbolisch mit ihnen rechnet. 
Die Grundlage dafür wird in
Abschnitt~\ref{buch:integral:subsection:koerpererweiterungen}
gelegt.



