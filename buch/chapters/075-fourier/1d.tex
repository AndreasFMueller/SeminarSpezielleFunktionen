%
% 1d.tex
%
% (c) 2022 Prof Dr Andreas Müller
%
\section{Fourier-Transformation
\label{buch:fourier:section:fourier}}
\rhead{Fourier-Transformation}
Im Kapitel~\ref{buch:chapter:orthogonalitaet}
wurde gezeigt, wie orthogonale Funktionenfamilien ermöglichen,
beliebige Funktionen auf besonders einfache Art und Weise
als Linearkombinationen oder Reihen von Funktionen der Familie
darzustellen.

%
% Fourier-Reihen
%
\subsection{Fourier-Reihen}
Das berühmteste solche Beispiel ist die Fourier-Reihe, die
Joseph Fourier im Rahmen seiner Studien zur Wärmeleitungsgleichung
entdeckt hat.
Sie besagt, dass unter geeigneten Regularitätsvoraussetzungen
jede $2\pi$-periodische Funktion $f(x)$ als konvergente Reihe von
trigonometrischen Funktionen in der Form
\begin{equation}
f(x)
=
\frac{a_0}2
+
\sum_{k=1}^\infty (a_k \cos kx + b_k \sin kx)
\label{buch:fourier:eqn:fourierreihe}
\end{equation}
geschrieben werden kann.
Die Koeffizienten $a_k$ und $b_k$ können mit Hilfe von Integralen
\[
\begin{aligned}
a_k
&=
\frac{1}{\pi}
\int_{-\pi}^\pi f(x)\, \cos kx\,dx
&&\text{für $k\ge 0$}
\\
b_k
&=
\frac{1}{\pi}
\int_{-\pi}^\pi f(x)\, \sin kx\,dx
&&\text{für $k\ge 1$}
\end{aligned}
\]
berechnet werden.

%
% Komplexes Skalarprodukt
%
\subsubsection{Komplexes Skalarprodukt}
Die Reihe~\eqref{buch:fourier:eqn:fourierreihe} hat zwar den Vorteil,
nur reellwertige Funktionen zu verwenden.
Die trigonometrischen Funktionen lassen sich aber mit Hilfe der
eulerschen Formel mittels
\[
\cos x = \frac{e^{ix}+e^{-ix}}{2}
\qquad
\sin x = \frac{e^{ix}-e^{-ix}}{2i}
\]
als Linearkombinationen von Exponentialfunktionen ausdrücken.
Wendet man diese Transformation auf
die Fourier-Reihe~\eqref{buch:fourier:eqn:fourierreihe}
an, erhält man eine symmetrischere, komplexe Darstellung.
Dazu ist zunächst die Definition eines Skalarproduktes für komplexe
Funktionen nötig.

\begin{definition}
\label{buch:fourier:def:cskalar}
Die bilineare Funktion
\[
\langle f,g\rangle
=
\frac{1}{2\pi}
\int_{-\pi}^\pi \overline{f(x)}g(x)\,dx
\]
ist {\em sesquilinear}, d.~h. sie ist
\begin{enumerate}
\item
linear im zweiten und konjugiert linear im ersten Argument:
\begin{align*}
\langle\lambda_1f_1+\lambda_2f_2,g\rangle
&=
\overline{\lambda_1}\langle f_1,g\rangle
+
\overline{\lambda_2}\langle f_2,g\rangle
&
\langle f,\mu_1g_1+\mu_2g_2\rangle
&=
\mu_1
\langle f,g_1\rangle
+
\mu_2
\langle f,g_2\rangle,
\end{align*}
für $\lambda_i,\mu_i\in\mathbb{C}$.
\item
Konjugiert-symmetrisch:
$\langle f,g\rangle=\overline{\langle g,f\rangle}$.
\item
Positiv definit: Falls $f\ne 0$ folgt
$\langle f,f\rangle > 0$.
\end{enumerate}
Man nennt
$\langle\;\,,\;\rangle$
ein {\em Skalarprodukt} für komplexe Funktionen.
Es liefert eine
{\em Norm}
\[
\|f\|^2
=
\langle f,f\rangle
=
\int_{-\pi}^{\pi} |f(x)|^2\,dx.
\]
\end{definition}

%
% Komlexe Fourier-Reihe
%
\subsubsection{Komplexe Fourier-Reihe}
Bezüglich des Skalarprodukts
der Definition~\ref{buch:fourier:def:cskalar}
sind die Funktionen
\[
e_k
:
\mathbb{R} \to \mathbb{C}
\colon
x\mapsto e_k(x)=e^{ikx}
\]
eine orthonormierte Familie.
Die komplexe Fourier-Theorie besagt dann,
dass eine beliebige komplexe $2\pi$-periodische Funktion $f(x)$ 
in der Form einer bezüglich der Norm $\|\,\cdot\,\|$ konvergenten Reihe
\begin{equation}
f(x)
=
\sum_{k=-\infty}^\infty c_ke^{ikx}
\label{buch:fourier:eqn:cfourierreihe}
\end{equation}
mit den Koeffizienten
\[
c_k
=
\frac{1}{2\pi}
\int_{-\pi}^\pi
f(x) e^{-ikx}\,dx.
\]

\subsubsection{Parseval-Identität}
Aus der geometrischen Motivation lässt sich auch auch die
Parseval-Identität ableiten.
Sie besagt, dass Skalarprodukt und Norm zweier Funktionen
$f(x)$ und $g(x)$ auch aus den Fourier-Koeffizienten
berechnet werden können.
Sind $d_k$ die Fourier-Koeffizienten der Funktion $g(x)$, dann gilt
\begin{align*}
\langle f,g\rangle
&=
\int_{-\pi}^\pi \overline{f(x)} g(x)\,dx
=
\sum_{k=-\infty}^\infty \overline{c_k}d_k,
\\
\|f\|^2
&=
\int_{-\pi}^\pi |f(x)|^2\,dx
=
\sum_{k=-\infty}^{\infty} |c_k|^2.
\end{align*}
Die rechten Seiten können als Skalarprodukt und Norm der Folgen
der Fourier-Koeffizienten interpretiert werden.
Die Formeln besagen dann, dass der Übergang zu den Fourier-Koeffizienten
das Skalarprodukt erhält.
Man kann die Formeln auch als die Fourier-Version des Satzes
von Pythagoras sehen.

%
% Das Fourier-Integral
%
\subsection{Fourier-Integral und Fourier-Transformation}
Gibt man die Einschränkung auf $2\pi$-periodische Funktionen
auf, gibt es keine abzählbare Familie von Basisfunktionen, mit denen
man eine beliebige Funktion als Summe darstellen können.
Die im folgenden zusammengefasste Fourier-Transformation zeigt
aber, dass eine Darstellung als Integral möglich ist.

\subsubsection{Fourier-Integral}
Die Fourier-Transformierte
\begin{equation}
\hat{f}(k)
=
(\mathscr{F}f)(k)
=
\frac{1}{\sqrt{2\pi}}
\int_{-\infty}^\infty f(x)e^{-ikx}\,dx
\label{buch:fourier:eqn:fouriertrafo}
\end{equation}
verallgemeinert die Berechnung der Fourier-Koeffizienten
auf beliebige Funktionen $f\colon\mathbb{R}\to\mathbb{R}$.
Wegen des unendlich langen Integrationsintervals sind zusätzliche
Einschränkungen nötig, damit das Integral überhaupt definiert ist.
Es genügt zu verlangen, dass
\[
\int_{-\infty}^\infty |f(x)|\,dx < \infty
\]
ist.
Die Funktionen mit dieser Eigenschaft bilden den Funktionenraum
$L^1(\mathbb{R})$.
Man beachte, dass die Funktionen
\(
e_k(x) = e^{ikx}
\)
nicht in $L^1(\mathbb{R})$ sind, die Betrachtungsweise auf dem
Intervall $[-\pi,\pi]$ lässt sich daher nicht auf den Fall des
Definitionsbereichs $\mathbb{R}$ übertragen.

Die allgemeine Theorie zeigt, dass die
Transformation~\eqref{buch:fourier:eqn:fouriertrafo}
durch das Integral
\[
f(x)
=
\frac{1}{\sqrt{2\pi}}
\int_{-\infty}^\infty (\mathscr{F}f)(k)e^{ikx}\,dk.
\]
umgekehrt wird.
Dies funktioniert wieder nur, wenn auch $\mathscr{F}f$ eine Funktion
in $L^1(\mathbb{R})$ ist.

%
% Skalarprodukt
%
\subsubsection{Skalarprodukt}
Auch die Fourier-Transformation kann im Rahmen eines Funktionenraums
mit einem Skalarprodukt beschrieben werden.
Das Skalarprodukt in diesem Fall ist
\[
\langle f,g\rangle
=
\int_{-\infty}^\infty \overline{f(x)} g(x)\,dx.
\]
Das Skalarprodukt ist wieder nur für eine eingeschränkte Menge von
Funktionen definiert, nämlich für Funktionen aus dem Funktionenraum
\[
L^2(\mathbb{R})
=
\left\{
f\colon\mathbb{R}\to\mathbb{R}
\,\left|\,
\int_{-\infty}^\infty |f(x)|^2\,dx < \infty
\right.
\right\}.
\]
In der Praxis hat man oft mit Funktionen zu tun, die nur in einem
beschränkten Intervall von $0$ verschieden sind und zudem stetig.
Solche Funktionen sind immer in $L^2(\mathbb{R})\cap L^1(\mathbb{R})$.

%
% Parseval-Identität
%
\subsubsection{Parseval-Identität}
Wie im Falle der Fourier-Reihe, ändert die Fourier-Transformation
die Länge des Vektors nicht.
Für die Fourier-Transformation lautet die Parseval-Identität
\begin{align*}
\langle f,g\rangle
&=
\int_{-\infty}^\infty
\overline{f(x)} g(x)\,dx
=
\int_{-\infty}^\infty
\overline{ \mathscr{F}f(k) }
\mathscr{F}g(k)\,dk
=
\langle \mathscr{F}f,\mathscr{F}g\rangle
=
\langle\hat{f},\hat{g}\rangle
\\
\|f\|^2
&=
\int_{-\infty}^\infty
|f(x)|^2\,dx
=
\int_{-\infty}^\infty
|\mathscr{F}f(k)|^2\,dk
=
\|\mathscr{F}f\|^2
=
\|\hat{f}\|^2,
\end{align*}
für Funktionen, für die alles definiert ist.
Die Fourier-Transformation erhält also das Skalarprodukt,
sie ist eine unitäre Abbildung.
