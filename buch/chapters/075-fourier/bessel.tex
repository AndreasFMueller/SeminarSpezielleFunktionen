%
% bessel.tex
%
% (c) 2022 Prof Dr Andreas Müller, OST Ostschweizer Fachhochschule
%
\section{Fourier-Transformation und Bessel-Funktionen
\label{buch:fourier:section:fourier-und-bessel}}
\rhead{Fourier-Transformation und Bessel-Funktionen}

Sei $f\colon \mathbb{R}^2\to\mathbb{C}$ eine auf $\mathbb{R}$ definierte
Funktion.
Die Fourier-Transformation von $f$ ist das Integral
\begin{equation}
(\mathscr{F}f)(u,v)
=
F(u,v)
=
\frac{1}{2\pi}
\int_{-\infty}^\infty
\int_{-\infty}^\infty
f(x,y) e^{i(xu+yv)}
\,dx\,dy.
\label{buch:fourier:eqn:2dfourier}
\end{equation}
Die Funktionen $e_{u,v}\colon (x,y)\mapsto e^{i(xu+yv)}$
sind die Eigenfunktionen des Laplace-Operators in kartesischen Koordinaten,
sie erfüllen
\[
\Delta e_{u,v} = (u^2+v^2) \Delta e_{u,v}.
\]
Die Fourier-Integrale sind die Skalarprodukte
\[
(\mathscr{F}f)(u,v)
=
\langle
e_{u,v},
f
\rangle,
\]
wobei das Skalarprodukt durch
\[
\langle f,g\rangle
=
\int_{-\infty}^\infty
\int_{-\infty}^\infty
\overline{f(x)} g(x)
\,dx\,dy
\]
definiert ist.

Jede Funktion in der Ebene kann auch in Polarkoordinaten ausgedrückt werden.
Die kartesischen Koordinaten können mittels
\begin{align*}
x&=r\cos\varphi
y&=r\sin\varphi
\end{align*}
durch die Polarkoordinaten $(r,\varphi)$ ausgedrückt werden.
Wir schreiben
\[
\tilde{f}(r,\varphi)
=
f(r\cos\varphi,r\sin\varphi)
\]
für die Funktion $f$ ausgedrückt in Polarkoordinaten.

In Polarkoordinaten wird das Skalarprodukt
\[
\langle f,g\rangle
=
\int_0^\infty \int_{0}^{2\pi} e^{in\varphi}
\overline{
\tilde{f}(r,\varphi)
}
\tilde{g}(r,\varphi)
r\,dr\,d\varphi.
\]
Auch die Fouriertransformation kann jetzt durch Berechnung eines
doppelten Integrals in Polarkoordinaten ermittelt werden.
Ziel dieses Abschnitts ist zu zeigen, dass auch diese Berechnung auf
Bessel-Funktionen führt.
Im Gegenzug werden sich neue Eigenschaften und Darstellungen derselben
ergeben.


\subsection{Berechnung der Fourier-Transformation in Polarkoordinaten}
Die Fourier-Transformation $(\mathscr{F}f)(u,v)$ ist eine Funktion
$\mathbb{R}^2\to\mathbb{C}$, die vom Wellenvektor $(u,v)$ abhängt.
Auch dieser Vektor kann in Polarkoordinaten ausgedrückt werden.
Für die Polarkoordinaten in der Wellenvektor-Ebene soll die Bezeichnung
$(R,\vartheta)$ verwendet werden, was auf die Transformationsgleichungen
\begin{align*}
u&=R\cos\vartheta\\
v&=R\sin\vartheta
\end{align*}
führt.
Im Exponenten der Exponentialfunktion
des Fourier-Integrals~\eqref{buch:fourier:eqn:2dfourier}
steht der Ausdruck
\[
xu+yv
=
r\cos\varphi\cdot R\cos\vartheta
+
r\sin\varphi\cdot R\sin\vartheta
=
rR\cos(\varphi-\vartheta).
\]
Mit diesen Bezeichnungen wird das
Fourier-Integral~\eqref{buch:fourier:eqn:2dfourier}
zu
\begin{align}
\tilde{F}(R,\vartheta)
&=
\frac{1}{2\pi}
\int_{0}^{\infty}
\int_{0}^{2\pi}
f(r\cos\varphi,r\sin\varphi)
e^{irR\cos(\varphi-\vartheta)}
\,d\varphi\,r\, dr
\notag
\\
&=
\frac{1}{2\pi}
\int_{0}^{\infty}
\int_{0}^{2\pi}
\tilde{f}(r,\varphi)
e^{irR\cos(\varphi-\vartheta)}
\,d\varphi\,r\, dr.
\label{buch:fourier:eqn:fouriertrafopolar}
\end{align}
Die partielle Funktion $\varphi\mapsto \tilde{f}(r,\varphi)$
ist eine $2\pi$-periodische Funktion, sie lässt sich also als
komplexe Fourier-Reihe
\begin{equation}
\tilde{f}(r,\varphi)
=
\sum_{n\in\mathbb{Z}} \hat{f}_n(r) e^{in\varphi}
\label{buch:fourier:eqn:fourierkoef}
\end{equation}
schreiben, die Funktionen $\hat{f}_n(r)$ sind die komplexen
Fourier-Koeffizienten.
Setzt man \eqref{buch:fourier:eqn:fourierkoef} in die Fourier-Transformation
\eqref{buch:fourier:eqn:fouriertrafopolar} ein, erhält man
\begin{align*}
\tilde{F}(R,\vartheta)
&=
\sum_{n\in\mathbb{Z}}
\int_0^\infty
\hat{f}_n(r)
\frac{1}{2\pi}
\int_0^{2\pi}
e^{in\varphi+irR\cos(\varphi-\vartheta)}
\,d\varphi
\,
r\,dr.
\end{align*}
Der Exponent im inneren Integral kann als
\[
in\varphi+irR\cos(\varphi-\vartheta)
=
i(n(\varphi-\vartheta)+rR\cos(\varphi-\vartheta))
+
in\vartheta,
\]
oder im Integral als
\[
\tilde{F}(R,\vartheta)
=
\sum_{n\in\mathbb{Z}}
\int_0^\infty
\hat{f}_n(r)
\frac{1}{2\pi}
\int_0^{2\pi}
e^{in(\varphi-\vartheta)+irR\cos(\varphi-\vartheta)}
e^{in\vartheta}
\,d\varphi
\,
r\,dr
\]
geschrieben werden.
Der zweite Exonentialfaktor hängt nicht von $\varphi$ ab und kann daher
aus dem Integral herausgezogen werden.
Der erste Exponentialfaktor hängt nur von $\varphi-\vartheta$ ab.
Da die Exponentialfunktion $2\pi$-periodisch ist, hat die Verschiebung
um $\vartheta$ keinen Einfluss auf den Wert des Integrals.
Die Fourier-Transformation ist daher auch
\[
\tilde{F}(R,\vartheta)
=
\sum_{n\in\mathbb{Z}}
\int_0^\infty
\hat{f}_n(r)
e^{in\vartheta}
\underbrace{
\frac{1}{2\pi}
\int_0^{2\pi}
e^{in\varphi+irR\cos\varphi}
\,d\varphi
}_{\displaystyle =:F_n(rR)}
\,
r\,dr.
\]
Die Beziehung zu den Besselfunktionen können wir daraus herstellen,
indem wir zunächst $\xi = rR$ abkürzen und dann das innere Integral
\begin{equation}
F_n(\xi)
=
\frac{1}{2\pi}
\int_{0}^{2\pi}
e^{in\varphi+i\xi\cos\varphi}
\,d\varphi
=
\frac{1}{2\pi}
\int_{0}^{2\pi}
e^{in\varphi}e^{i\xi\cos\varphi}
\,d\varphi
\label{buch:fourier:eqn:Fncosphi}
\end{equation}
auswerten.
Exponentialfunktion als Potenzreihe entwickeln:
\[
F_n(\xi)
=
\frac{1}{2\pi}
\int_0^{2\pi}
e^{in\varphi}
\sum_{k=0}^\infty
\frac{
i^k\xi^k \cos^k\varphi
}{k!}
\,d\varphi
=
\sum_{k=0}^\infty
\frac{i^k\xi^k}{k!}
\underbrace{
\frac{1}{2\pi}
\int_0^{2\pi}
e^{in\varphi}
\cos^k\varphi
\,d\varphi}_{\displaystyle =c_{n,k}}.
\]
Das Integral auf der rechten Seite ist im Wesentlichen ein
Fourier-Koeffizient der Funktion $\varphi\mapsto \cos^k\varphi$.

\subsubsection{Berechnung der Fourier-Koeffizienten von $\cos^k\varphi$}
Indem man die Kosinus-Funktion als die Linearkombination
\[
\cos\varphi
=
\frac{e^{i\varphi}+e^{-i\varphi}}2
\]
von Exponentialfunktionen ausdrückt, kann man auch die $k$-te Potenz 
mit Hilfe des binomischen Satzes als
\[
\cos^k\varphi
=
\sum_{m=0}^k
\frac{1}{2^k}
\binom{k}{m}
e^{im\varphi}e^{i(m-k)\varphi}
=
\sum_{m=0}^k
\frac{1}{2^k}
\binom{k}{m}
e^{i(2m-k)\varphi}
\]
ausdrücken.
Der Fourier-Koeffizient von $\cos^k\varphi$ ist daher das Integral
\begin{align*}
c_{n,k}
&=
\frac{1}{2\pi}
\int_0^{2\pi}
e^{in\varphi}\cos^k\varphi\,d\varphi
\\
&=
\frac{1}{2^k}
\sum_{m=0}^k
\binom{k}{m}
\frac{1}{2\pi}
\int_0^{2\pi}
e^{in\varphi}e^{i(2m-k)\varphi}
\,d\varphi
\\
&=
\frac{1}{2^k}
\sum_{m=0}^k
\binom{k}{m}
\frac{1}{2\pi}
\int_0^{2\pi}
e^{i(2m-k+n)\varphi}
\,d\varphi.
\end{align*}
Für $2m-k+n=0$ ist das Integral ein Integral der Funktion $1$ über
ein Intervall der Länge $2\pi$, zusammen mit dem Faktor $1/2\pi$ hat
es daher den Wert $1$.
Für $2m-k+n\ne 0$ ist das Integral 
\[
\frac{1}{2\pi}
\int_0^{2\pi}
e^{i(2m-k+n)\varphi}
\,d\varphi
=
\frac{1}{i}
\biggl[
\frac{e^{i(2m-k+n)\varphi}}{2m-k+n}
\biggr]_0^{2\pi}
=
0
\]
weil die Exponentialfunktion $2\pi$-periodisch ist.
Nur für $k=2m+n$ ergibt sich ein nicht verschwindender
Fourier-Koeffizient.
Eine Summe über $k\in\mathbb{N}$ kann daher auch als Summe über
$m\in\mathbb{N}$ interpretiert werden, in der $k$ durch die Formel
$k=2m+n$ gegeben wird.
Mit dieser Konvention wird
\[
c_{n,k}
=
c_{n,2m+n}
%=
%\frac{1}{2\pi}
%\int_0^{2\pi}
%e^{-i(2m+n)\varphi}
%\cos^{2m+n}\varphi
%\,d\varphi
=
\frac{1}{2^{2m+n}}
\binom{2m+n}{m}
\]
schreiben lässt.

\subsubsection{Berechnung von $F_n(\xi)$}
Die Reihe für $F_n(\xi)$ lässt sich weiter vereinfachen.
Wir verwenden wieder die Tatsache, dass sich nur für $n=-2m-k$
ein Beitrag ergibt.
Dies bedeutet, dass $k=2m+n$ sein muss, die Summe kann damit als
Summe über $m$ statt über $k$ geschrieben werden.
Somit ist
\begin{align*}
F_n(\xi)
&=
\sum_{k=0}^\infty
\frac{i^k\xi^k}{k!}
c_{n,k}
=
\sum_{m=0}^\infty
\frac{i^{2m+n}\xi^{2m+n}}{(2m+n)!}
c_{n,2m+n}
\\
&=
\sum_{m=0}^\infty
\frac{1}{2^{2m+n}}
\binom{2m+n}{m}
\frac{i^{2m+n}\xi^{2m+n}}{(2m+n)!}
\\
&=
i^n
\sum_{m=0}^\infty
\frac{(-1)^m}{(2m+n)!}
\frac{(2m+n)!}{m!\,(2m+n-m)!}
\biggl(\frac{\xi}{2}\biggr)^{2m+n}
\\
&=
i^n
\sum_{m=0}^\infty
\frac{(-1)^m}
{m!\,\Gamma(m+n+1)}
\biggl(\frac{\xi}{2}\biggr)^{2m+n}
=
i^n J_n(\xi).
\end{align*}
Die Funktionen $F_n(\xi)$ sind daher bis auf einen Phasenfaktor der
Wert $J_n(\xi)$ einer Bessel-Funktion.

\subsubsection{Berechnung der Fourier-Transformation mit Bessel-Funktionen}
Mit allen oben zusammengestellten Notationen kann die Fourier-Transformation
jetzt in Polarkoordinaten als
\[
\tilde{F}(R,\vartheta)
=
\sum_{n\in\mathbb{Z}}
e^{in\vartheta}
\int_0^\infty 
\hat{f}_n(r)
i^n
J_n(rR)
r\,dr
\]
geschrieben werden.
Dies hat tatsächlich die Form eines Skalarproduktes der Funktion
$\tilde{f}(r,\varphi)$ mit einer Funktion der Form
\[
\tilde{e}_{n,R}(r,\varphi)
=
e^{in\varphi}
J_n(rR).
\]
Letzeres sind die in Abschnitt~\ref{buch:fourier:section:2d}
versprochenen Basisfunktionen.

\subsubsection{Fourier-Reihe von $e^{i\xi\cos\varphi}$}
Die Funktionen $F_n(\xi)$ sind wegen
\[
F_n(\xi)
=
\frac{1}{2\pi}
\int_0^{2\pi}
e^{in\varphi}
e^{i\xi\cos\varphi}
\,d\varphi,
\]
daraus kann man die Fourier-Reihe von $e^{i\xi\cos\varphi}$ 
berechnen, dies wird im folgenden Satz durchgeführt.


\begin{satz}
\label{buch:fourier:satz:expinphi}
Die komplexe Fourier-Reihe der Funktion
$\varphi\mapsto \exp(i\xi\cos\varphi)$
ist
\begin{align}
e^{i\xi\cos\varphi}
&=
J_0(\xi)
+
2\sum_{n=1}^\infty i^n J_n(\xi) \cos n\varphi.
\label{buch:fourier:eqn:expinphicomplex}.
\intertext{Real- und Imaginärteil davon sind die Fourier-Reihen}
\cos(\xi\cos\varphi)
&=
J_0(\xi) + 2\sum_{m=1}^\infty (-1)^m J_{2m}(\xi) \cos2m\varphi
\label{buch:fourier:eqn:expinphireal}
\\
\sin(\xi\cos\varphi)
&=
2\sum_{m=0}^\infty (-1)^m J_{2m+1}(\xi) \cos(2m+1)\varphi.
\label{buch:fourier:eqn:expinphiimaginary}
\end{align}
\end{satz}

\begin{proof}[Beweis]
Die Fourier-Koeffizienten $F_n(\xi)$ der Funktion $e^{i\xi\cos\varphi}$
führen auf die Fourier-Reihe
\begin{align*}
e^{i\xi\cos\varphi}
&=
\sum_{n\in\mathbb{Z}} F_n(\xi) e^{in\varphi}
=
\sum_{n\in\mathbb{Z}} i^n J_n(\xi) e^{in\varphi}.
\end{align*}
Terme mit $\pm n$ können wegen
\[
\left.
\begin{aligned}
J_{-n}(\xi) &= (-1)^n J_n(\xi) 
\\
i^{-n}&=(-1)^n i^n
\end{aligned}
\quad
\right\}
\qquad\Rightarrow\qquad
i^{-n}J_{-n}(\xi) = i^n J_n(\xi)
\]
zusammengefasst werden, auf diese Weise erhält man 
\begin{align*}
e^{i\xi\cos\varphi}
&=
J_0(\xi)
+
\sum_{n=1}^\infty i^n J_n(\xi) (e^{in\varphi}+e^{-in\varphi})
=
2\sum_{n=1}^\infty i^n J_n(\xi) \cos n\varphi.
\end{align*}
Dies beweist
\eqref{buch:fourier:eqn:expinphicomplex}.

Indem man Real- und Imaginärteil trennt, kann man daraus auch
die Fourier-Reihen von $\cos(\xi\cos\varphi)$ und
$\sin(\xi\cos\varphi)$ gewinnen, sie sind
\begin{align*}
\exp(\xi\cos\varphi)
&=
J_0(\xi) + 2\sum_{n=1}^\infty i^{n} J_{n}(\xi) \cos n\varphi
\\
&=
J_0(\xi)
+
2\sum_{m=1}^\infty i^{2m}J_{2m}(\xi)\cos 2m\varphi
+
2\sum_{m=0}^\infty i^{2m+1}J_{2m+1}(\xi)\cos(2m+1)\varphi
\\
&=
J_0(\xi)
+
2\sum_{m=1}^\infty (-1)^{m}J_{2m}(\xi)\cos 2m\varphi
+
2i\sum_{m=0}^\infty (-1)^{m}J_{2m+1}(\xi)\cos(2m+1)\varphi
\\
\cos(\xi\cos\varphi)
&=
J_0(\xi)
+
2\sum_{m=1}^\infty (-1)^{m}J_{2m}(\xi)\cos 2m\varphi
\\
\sin(\xi\cos\varphi)
&=
2\sum_{m=0}^\infty (-1)^m J_{2m+1}(\xi) \cos(2m+1)\varphi.
\end{align*}
Damit sind auch die Formeln
\eqref{buch:fourier:eqn:expinphireal}
und
\eqref{buch:fourier:eqn:expinphiimaginary}
für die reellen Fourier-Reihen bewiesen.
\end{proof}

%
% Integraldarstellung der Bessel-Funktion
%
\subsection{Integraldarstellung der Bessel-Funktion}
Aus \eqref{buch:fourier:eqn:Fncosphi} kann jetzt die Integraldarstelltung
der Bessel-Funktionen gewonnen werden.
Dazu substituiert man $\varphi$ durch $\tau$ mit
$\varphi = \frac{\pi}2-\tau$
oder
$\tau=\frac{\pi}2-\varphi$
und $d\tau = -d\varphi$
im Integral und berechnet
\begin{align*}
J_n(\xi)
&=
(-i)^n
\frac{1}{2\pi}
\int_0^{2\pi}
e^{in\varphi+i\xi \cos\varphi}
\,d\varphi
\\
&=
-
(-i)^n
\frac{1}{2\pi}
\int_{\frac{\pi}2}^{-\frac{3\pi}2}
e^{in(\frac{\pi}2-\tau) + i\xi\cos(\frac{\pi}2-\tau)}
\,d\tau
\\
&=
(-i)^n
\frac{1}{2\pi}
\int^{\frac{\pi}2}_{-\frac{3\pi}2}
i^n
e^{-in\tau + i\xi\sin\tau)}
\,d\tau.
\intertext{Da der Integrand $2\pi$-periodisch ist, kann das
Integrationsintervall auf $[-\pi,\pi]$ verschoben werden, was}
&=
\frac{1}{2\pi}
\int_{-\pi}^{\pi}
e^{-in\tau + i\xi\sin\tau)}
\,d\tau.
\intertext{ergibt.
Das Integral kann in zwei Integrale}
&=
\frac{1}{2\pi}
\int_0^\pi
e^{-in\tau + i\xi\sin\tau}
\,d\tau
+
\frac{1}{2\pi}
\int_0^\pi
e^{in\tau - i\xi\sin\tau}
\,d\tau
\intertext{aufgeteilt werden,
}
&=
\frac{1}{\pi}
\int_0^\pi
\frac{
e^{-in\tau + i\xi\sin\tau}
+
e^{in\tau - i\xi\sin\tau}
}{2}
\,d\tau
\\
&=
\frac{1}{\pi}
\int_0^\pi
\frac{
e^{i(-n\tau + \xi\sin\tau)}
+
e^{-i(-n\tau + \xi\sin\tau)}
}{2}
\,d\tau
\\
&=
\frac{1}{\pi}
\int_0^\pi
\cos(n\tau - \xi\sin\tau)
\,d\tau.
\end{align*}
Damit haben wir den folgenden Satz bewiesen:

\begin{satz}[Integraldarstelltung der Bessel-Funktionen]
\label{buch:fourier:satz:bessel-integraldarstellung}
Die Bessel-Funktionen $J_n$ mit ganzzahliger Ordnung $n$ haben
die Integraldarstellung
\begin{equation}
J_n(\xi)
=
\frac{1}{\pi}
\int_0^\pi
\cos(n\tau - \xi\sin\tau)
\,d\tau.
\label{buch:fourier:eqn:bessel-integraldarstellung}
\end{equation}
\end{satz}




