%
% 4-legendredgl.tex
%
% (c) 2021 Prof Dr Andreas Müller, OST Ostschweizer Fachhochschule
%
\section{Orthogonale Polynome und Differentialgleichungen
\label{buch:orthogonal:section:orthogonale-polynome-und-dgl}}
%\kopfrechts{Differentialgleichungen orthogonaler Polynome}
Legendre hat einen ganz anderen Zugang zu den nach ihm benannten
Polynomen gefunden.
Er hat sie als die Lösungen einer speziellen Differentialgleichungen
gefunden.
In diesem Abschnitt sollen diese Funktionen mit der Potenzreihenmethode
wiedergefunden werden.
Dabei stellt sich heraus, dass diese Polynome auch Eigenfunktionen eines
selbstadjungierten Differentialgoperators sind.
Die Orthogonalität wird dann aus einer Verallgemeinerung der bekannten
Eingeschaft folgen, dass Eigenvektoren einer symmetrischen Matrix zu 
verschiedenen Eigenwerten orthogonal sind.

\kopfrechts{Differentialgleichungen orthogonaler Polynome}
%
% Legendre-Differentialgleichung
%
\subsection{Legendre-Differentialgleichung}
Die {\em Legendre-Differentialgleichung} ist die Differentialgleichung
\index{Differentialgleichung!Legendre-}%
\index{Legendre-Differentialgleichung}%
\begin{equation}
(1-x^2) y'' - 2x y' + n(n+1) y = 0
\label{buch:integral:eqn:legendre-differentialgleichung}
\end{equation}
für eine Funktion $y(x)$ auf dem Intervall $[-1,1]$.

Sei $y(x)$ eine Lösung der Differentialgleichung
\eqref{buch:integral:eqn:legendre-differentialgleichung}.
Setzt man $y_s(x)=y(-x)$ in die Differentialgleichung ein, erhält
man
\[
(1-x^2)y_s''(x) - 2x y'_s(x) + n(n+1)y_s(x)
=
(1-x^2)y''(-x) +2x y(-x) +n(n+1)y(-x).
\]
Ersetzt man $t=-x$, dann wird daraus
\[
(1-t^2)y''(t) -2t y(t) + n(n+1) y(t) = 0
\]
aus der Differentialgleichung
\eqref{buch:integral:eqn:legendre-differentialgleichung}.
Insbesondere ist die gespiegelte Funktion $y_s(x)$ ebenfalls
eine Lösung der Differentialgleichung.

Ist $y(x)$ eine Lösung der Differentialgleichung ist, dann lässt
sie sich in die Summe einer geraden und einer ungeraden Funktion
\[
\left.
\begin{aligned}
y_g(x) &= \frac{y(x)+y(-x)}{2}\\
y_u(x) &= \frac{y(x)-y(-x)}{2}
\end{aligned}
\quad
\right\}
\quad
\Rightarrow
\quad
y(x) = y_g(x) + y_u(x)
\]
zerlegen, die als Linearkombinationen der beiden Lösungen
$y(x)$ und $y_s(x)$ ebenfalls Lösungen der Differentialgleichung
sind.

%
% Potenzreihenlösungen
%
\subsubsection{Potenzreihenlösung}
Wir suchen eine Lösung in Form einer Potenzreihe um $x=0$ und 
verwenden dazu den Ansatz
\[
y(x) = a_0+a_1x+a_2x^2+ \dots = \sum_{k=0}^\infty a_kx^k.
\]
Einsetzen von $y(x)$ und den Ableitungen in die
Legendre-Differentialgleichung~\eqref{buch:integral:eqn:legendre-differentialgleichung}
ergibt die Bedingungen
\begin{align*}
(1-x^2) \sum_{k=2}^\infty k(k-1)a_kx^{k-2}
-2x\sum_{k=0}^\infty ka_kx^{k-1}
+
n(n+1)\sum_{k=0}^\infty  a_kx^k
&=
0
\\
\sum_{k=0}^\infty (k+2)(k+1)a_{k+2}x^k
-
\sum_{k=2}^\infty k(k-1)a_kx^k
-
2\sum_{k=1}^\infty ka_kx^k
+
n(n+1)\sum_{k=0}^\infty  a_kx^k
&=
0,
\end{align*}
aus denen die Koeffizienten $a_k$ bestimmt werden können.
Die Koeffizienten zur Potenz $k$ sind 
\begin{align}
k&=0:
&
&&
0&=
2a_2+n(n+1)a_0
\notag
\\
&&
&\Rightarrow&
a_2&=-\frac{n(n+1)}{2}a_0
\notag
\\
k&=1:
&
&&
0&=
6a_3-2a_1+n(n+1)a_1
\notag
\\
&&
&\Rightarrow&
a_3&= \frac{2-n(n+1)}{6}a_1
\notag
\\
k&>1:
&
&&
0&=
(k+2)(k+1)a_{k+2} -k(k-1)a_k -2ka_k +n(n+1) a_k
\notag
\\
&&
&\Rightarrow&
a_{k+2}
&=
\frac{ k(k+1)-n(n+1) }{(k+2)(k+1)}
a_k.
\label{buch:integral:legendre-dgl:eqn:akrek}
\end{align}
Wenn $a_1=0$ und $a_0\ne 1$ ist, dann ist die Funktion $y(x)$ gerade,
alle ungeraden Koeffizienten verschwinden.
Ebenso verschwinden alle geraden Koeffizienten, wenn $a_0=0$ und $a_1\ne 0$.
Für jede Lösung $y(x)$ der Differentialgleichung ist
$y_g(x)$ ein Lösung mit $a_1=0$ und $y_u(x)$ eine Lösung mit $a_0=0$.
Wir können die Diskussion der Lösungen daher auf gerade oder ungerade
Lösungen einschränken.

Gesucht ist jetzt eine Lösung in Form eines Polynoms.
In diesem Fall müssen die Koeffizienten $a_k$ ab einem
gewissen Index verschwinden.
Dies tritt nach \eqref{buch:integral:legendre-dgl:eqn:akrek} genau
dann auf, wenn der Zähler für ein $k$ verschwindet.
Folglich gibt es genau dann Polynomlösungen der Differentialgleichung,
wenn $n$ eine natürlich Zahl ist.
Ausserdem ist die Lösung ein Polynom $\bar{P}_n(x)$ vom Grad $n$.
Das Polynom soll wieder so normiert sein, dass $\bar{P}_n(1)=1$ ist.

Die Lösungen der Differentialgleichung können jetzt explizit
berechnet werden.
Zunächst ist $\bar{P}_0(x)=1$ und $\bar{P}_1(x)=x$.
Für $n=2$ setzen wir zunächst $a_0=1$ und $a_1=0$ und erhalten
\[
y(x)
=
1 + \frac{0(0+1) - 2(2+1)}{(0+2)(0+1)}a_0 x^2
=
1
-3x^2
\qquad\text{oder}\qquad
\bar{P}_3(x) = \frac12(3x^2-1).
\]
Für $n=3$ starten wir von $a_1=1$ und $a_0=0$, was zunächst $a_2=0$
impliziert.
Für $a_3$ finden wir
\[
a_3=\frac{1(1+1)-3(3+1)}{(1+2)(1+1)} = -\frac53
\qquad\Rightarrow\qquad
y(x) = x-\frac53x^3
\qquad\Rightarrow\qquad
\bar{P}_3(x) = \frac12(5x^3-3x).
\]
Dies stimmt überein mit den früher gefundenen Ausdrücken für
die Legendre-Polynome.

Die Potenzreihenlösung zeigt zwar, dass es für jedes $n\in\mathbb{N}$
eine Polynomlösung $\bar{P}_n(x)$ vom Grad $n$ gibt.
Dies kann aber nicht erklären, warum die so gefundenen Polynome
orthogonal sind.

%
% Eigenfunktionen
% 
\subsubsection{Eigenfunktionen}
Die Differentialgleichung
\eqref{buch:integral:eqn:legendre-differentialgleichung}
kann mit dem Differentialoperator
\[
D = \frac{d}{dx}(1-x^2)\frac{d}{dx}
\]
als
\[
Dy + n(n+1)y = 0
\]
geschrieben werden.
Tatsächlich ist
\[
Dy
=
\frac{d}{dx} (1-x^2) \frac{d}{dy}
=
\frac{d}{dx} (1-x^2)y'
=
(1-x^2)y'' -2x y'.
\]
Dies bedeutet, dass die Lösungen $\bar{P}_n(x)$ Eigenfunktionen
des Operators $D$ zum Eigenwert $n(n+1)$ sind:
\[
D\bar{P}_n = -n(n+1) \bar{P}_n.
\]

%
% Orthogonalität von P_n als Eigenfunktionen
%
\subsubsection{Orthogonalität von $\bar{P}_n$ als Eigenfunktionen des Operators $D$}
Ein Operator $A$ auf Funktionen heisst {\em selbstadjungiert}, wenn
für zwei beliebige Funktionen $f$ und $g$ gilt
\[
\langle Af,g\rangle = \langle f,Ag\rangle
\]
gilt.
Im vorliegenden Zusammenhang möchten wir die Eigenschaft nutzen,
dass Eigenfunktionen eines selbstadjungierten Operatores zu verschiedenen
Eigenwerten orthogonal sind.
Dazu seien $Df = \lambda f$ und $Dg=\mu g$ und wir rechnen
\begin{equation}
\renewcommand{\arraycolsep}{2pt}
\begin{array}{rcccrl}
\langle Df,g\rangle &=& \langle \lambda f,g\rangle &=& \lambda\phantom{)}\langle f,g\rangle
&\multirow{2}{*}{\hspace{3pt}$\biggl\}\mathstrut-\mathstrut$}\\
=\langle f,Dg\rangle &=& \langle f,\mu g\rangle &=& \mu\phantom{)}\langle f,g\rangle&
\\[2pt]
\hline
         0           & &                        &=& (\lambda-\mu)\langle f,g\rangle&
\end{array}
\label{buch:integrale:eqn:eigenwertesenkrecht}
\end{equation}
Da $\lambda-\mu\ne 0$ ist, muss $\langle f,g\rangle=0$ sein.

Der Operator $D$ ist selbstadjungiert, d.~h.
für zwei beliebige zweimal stetig differenzierbare Funktion $f$ und $g$
auf dem Intervall $[-1,1]$ gilt
\begin{align*}
\langle Df,g\rangle
&=
\int_{-1}^1 (Df)(x) g(x) \,dx
\\
&=
\int_{-1}^1
\biggl(\frac{d}{dx} (1-x^2)\frac{d}{dx}f(x)\biggr) g(x)
\,dx
\\
&=
\underbrace{
\biggl[
\biggl((1-x^2)\frac{d}{dx}f(x)\biggr) g(x)
\biggr]_{-1}^1
}_{\displaystyle = 0}
-
\int_{-1}^1
\biggl((1-x^2)\frac{d}{dx}f(x)\biggr) \frac{d}{dx}g(x)
\,dx
\\
&=
-
\int_{-1}^1
\biggl(\frac{d}{dx}f(x)\biggr) \biggl((1-x^2)\frac{d}{dx}g(x)\biggr)
\,dx
\\
&=
-
\underbrace{
\biggl[
f(x) \biggl((1-x^2)\frac{d}{dx}g(x)\biggr)
\biggr]_{-1}^1}_{\displaystyle = 0}
+
\int_{-1}^1
f(x) \biggl(\frac{d}{dx}(1-x^2)\frac{d}{dx}g(x)\biggr)
\,dx
\\
&=
\langle f,Dg\rangle.
\end{align*}
Dies beweist, dass $D$ selbstadjungiert ist.
Da $\bar{P}_n$ Eigenwerte des selbstadjungierten Operators $D$ zu
den verschiedenen Eigenwerten $-n(n+1)$ sind, folgt auch, dass
die $\bar{P}_n$ orthogonale Polynome vom Grad $n$ sind, die die 
gleiche Standardisierungsbedingung wie die Legendre-Polyonome
erfüllen, also ist $\bar{P}_n(x)=P_n(x)$.

%
% Legendre-Funktionen zweiter Art
%
\subsubsection{Legendre-Funktionen zweiter Art}
%Siehe Wikipedia-Artikel \url{https://de.wikipedia.org/wiki/Legendre-Polynom}
%
Die Potenzreihenmethode liefert natürlich auch Lösungen der
Legendreschen Differentialgleichung, die sich nicht als Polynome
darstellen lassen.
Ist $n$ gerade, dann liefern die Anfangswerte $a_0=0$ und $a_1=1$ 
eine ungerade Funktion, die Folge der Koeffizienten bricht
aber nicht ab, vielmehr ist
\begin{align*}
a_{k+2}
&=
\frac{k(k+1)}{(k+1)(k+2)}a_k
=
\frac{k}{k+2}a_k.
\end{align*}
Durch wiederholte Anwendung dieser Rekursionsformel findet man
\[
a_{k}
=
\frac{k-2}{k}a_{k-2}
=
\frac{k-2}{k}\frac{k-4}{k-2}a_{k-4}
=
\frac{k-2}{k}\frac{k-4}{k-2}\frac{k-6}{k-4}a_{k-6}
=
\dots
=
\frac{1}{k}a_1.
\]
Die Lösung hat daher die Reihenentwicklung
\[
Q_0(x) = x+\frac13x^3 + \frac15x^5 + \frac17x^7+\dots
=
\frac12\log \frac{1+x}{1-x}
=
\operatorname{artanh}x.
\]
Die Funktion $Q_0(x)$ heisst {\em Legendre-Funktion zweiter Art}.

Für $n=1$ wird die Reihenentwicklung $a_0=1$ und $a_1=0$ etwas
interessanter.
Die Rekursionsformel für die Koeffizienten ist
\[
a_{k+2}
=
\frac{k(k+1)-2}{(k+1)(k+2)} a_k.
\qquad\text{oder}\qquad
a_k
=
\frac{(k-1)(k-2)-2}{k(k-1)}
a_{k-2}.
\]
Man erhält der Reihe nach
\begin{align*}
a_2 &= \frac{-2}{2\cdot 1} a_0 = -1
\\
a_3 &= 0
\\
a_4 &= \frac{3\cdot 2-2}{4\cdot 3} a_2 = \frac{4}{4\cdot 3}a_2 = \frac13a_2 = -\frac13
\\
a_5 &= 0
\\
a_6 &= \frac{5\cdot 4-2}{6\cdot 5}a_4 = \frac{18}{6\cdot 5}a_4 = -\frac15
\\
a_7 &= 0
\\
a_8 &= \frac{7\cdot 6-2}{8\cdot 7}a_6 = \frac{40}{8\cdot 7} = -\frac17
\\
a_9 &= 0
\\
a_{10} &= \frac{9\cdot 8-2}{10\cdot 9}a_8 = \frac{70}{10\cdot 9} = -\frac19,
\end{align*}
woraus sich die Reihenentwicklung
\begin{align*}
y(x)
&=
-x^2 -\frac13x^4 -\frac15x^6 - \frac17x^8 -\frac19x^{10}-\dots
\\
&=
-x\biggl(x+\frac13x^3 + \frac15x^5 + \frac17x^7 + \frac19x^9+\dots\biggr)
=
-x\operatorname{artanh}x
\end{align*}
errechnen lässt.

Die {\em Legendre-Funktionen zweiter Art} $Q_n(x)$  werden allerdings
so definiert, dass gewisse Rekursionsformeln für die Legendre-Polynome,
die wir hier nicht hergeleitet haben, auch für die $Q_n(x)$ gelten.
In dieser Normierung muss statt des eben berechneten $y(x)$ die Funktion
\[
Q_1(x) = x \operatorname{artanh}x-1
\]
verwendet werden.

%
%  Laguerre-Differentialgleichung
%
\subsection{Laguerre-Differentialgleichung
\label{buch:orthogonal:subsection:laguerre-differentialgleichung}}
Die Laguerre-Gewichtsfunktion $w_{\text{Laguerre}}(x)=e^{-x}$
\index{Laguerre-Gewichtsfunktion}%
führte auf die Laguerre-Polynome $L_n(x)$, die in 
\eqref{buch:orthogonal:eqn:laguerre-polynom-hypergeometrisch}
als hypergeometrische Funktionen erkannt wurden.
Sie sind daher auch Lösungen der Differentialgleichung
der hypergeometrischen Funktion $\mathstrut_1F_1$, die in
\eqref{buch:differentialgleichungen:1f1} dargestellt ist.

Die Parameter der Darstellung von $L_n(x)$ als hypergeometrische
Funktion sind
\[
L_n(x) = \mathstrut_1F_1\biggl(
\begin{matrix}-n\\1\end{matrix}
;x
\biggr)
\qquad\Rightarrow\qquad
\left\{
\begin{aligned}
a&=-n\\
b&=1.
\end{aligned}
\right.
\]
Einsetzen dieser Parametrer in die Differentialgleichung
\eqref{buch:differentialgleichungen:1f1}
\begin{equation}
zw'' + (1-z)w'+nw=0
\label{buch:differentialgleichungen:eqn:laguerre-dgl}
\end{equation}
Dies ist die {\em Laguerre-Differentialgleichung}.
\index{Laguerre-Differentialgleichung}%
\index{Differentialgleichung!Laguerre}%
Die Laguerre-Polynome sind also Lösungen der Laguerre-Differentialgleichung,
wenn der Parameter $n$ nicht negativ ganzzahlig ist.

Die allgemeine Laguerre-Differentialgleichung lässt beliebige reelle
Werte für den Koeffizienten von $w$ zu, sie lautet
\[
zw''+(1-z)w'+\lambda w=0.
\]
Die Anfangsbedingungen für die hypergeometrische Funktion als Lösung 
\begin{align*}
L_n(0)
&=
\mathstrut_1F_1\biggl(\begin{matrix}-\lambda\\1\end{matrix}; 0\biggr) = 1
\\
L'_n(0) &=
\frac{d}{dx}
\mathstrut_1F_1\biggl(\begin{matrix}-\lambda\\1\end{matrix};0\biggr)
=
\frac{-\lambda}{1}.
\end{align*}
Der Satz
\ref{buch:differentialgleichungen:satz:1f1-dgl-loesungen}
schlägt eine zweite Lösung vor, im vorliegenden Fall mit $b=1$
ist die zweite Lösung jedoch identisch zur ersten, es muss daher
ein anderer Weg zu einer zweiten Lösung gesucht werden.

%XXX TODO: zweite Lösung der Differentialgleichung.

%
%
%
\subsubsection{Die assoziierte Laguerre-Differentialgleichung}
\index{assoziierte Laguerre-Differentialgleichung}%
\index{Laguerre-Differentialgleichung, assoziierte}%
\index{Differentialgleichung!assoziierte Laguerre-}%
Die {\em assoziierte Laguerre-Differentialgleichung} ist die
Differentialgleichung
\begin{equation}
zw'' + (\nu  +1-z)w' + \lambda w = 0,
\label{buch:differentialgleichungen:eqn:assoziierte-laguerre-dgl}
\end{equation}
also die Differentialgleichung für die hypergeometrische Funktion
$\mathstrut_1F_1$ mit Parametern $a=-\lambda$ und $b=\nu+1$.
Die Gleichung
\eqref{buch:differentialgleichungen:eqn:assoziierte-laguerre-dgl}
hat daher die Lösung
\(
\mathstrut_1F_1(-\lambda;\nu+1;x).
\)
Für natürliches $\lambda$ sind diese Lösungen Polynome
\[
L_n^{(\nu)}(x)
=
\mathstrut_1F_1\biggl(
\begin{matrix}-n\\\nu+1\end{matrix}
;x\biggr),
\]
sie heissen die {\em assoziierten Laguerre-Polynome}.
\index{assoziierte Laguerre-Polynome}%
\index{Laguerre-Polynome, assoziierte}%
