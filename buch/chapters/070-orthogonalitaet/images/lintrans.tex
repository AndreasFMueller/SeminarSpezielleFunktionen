%
% lintrans.tex -- template for standalon tikz images
%
% (c) 2021 Prof Dr Andreas Müller, OST Ostschweizer Fachhochschule
%
\documentclass[tikz]{standalone}
\usepackage{amsmath}
\usepackage{times}
\usepackage{txfonts}
\usepackage{pgfplots}
\usepackage{csvsimple}
\usetikzlibrary{arrows,intersections,math,calc}
\begin{document}
\def\skala{1}
\begin{tikzpicture}[>=latex,thick,scale=\skala]

\coordinate (Al) at (0,2);
\coordinate (Ar) at (4,2);
\coordinate (Bl) at (-4,-2);
\coordinate (Br) at (4,-2);

\fill[color=blue!20] (Al) -- (Bl) -- (Br) -- (Ar) --cycle;

\draw[->] ($(Al)+(-0.1,0)$) -- ($(Ar)+(0.4,0)$) coordinate[label={$t$}];;
\draw ($(Al)+(0,-0.1)$) -- ($(Al)+(0,0.1)$);
\draw ($(Ar)+(0,-0.1)$) -- ($(Ar)+(0,0.1)$);
\node at (Al) [above] {$0\mathstrut$};
\node at (Ar) [above] {$1\mathstrut$};


\draw[->] ($(Bl)+(-0.1,0)$) -- ($(Br)+(0.4,0)$) coordinate[label={$x$}];
\draw ($(Bl)+(0,-0.1)$) -- ($(Bl)+(0,0.1)$);
\draw ($(Br)+(0,-0.1)$) -- ($(Br)+(0,0.1)$);
\node at (Bl) [below] {$-1\mathstrut$};
\node at (Br) [below] {$1\mathstrut$};
\node at ($0.5*(Br)+0.5*(Bl)$) [below] {$0\mathstrut$};
\fill ($0.5*(Br)+0.5*(Bl)$) circle[radius=0.05];

\coordinate (S1) at (2,1.4);
\coordinate (S2) at (0,-1.4);
\coordinate (O) at (0.4,0);
\coordinate (S0) at (1,0);

\draw[->,color=blue,line width=5pt]
	($(S1)-(O)$)
	--
	($(S2)-(O)$);
\node at ($(S0)-(O)$) [above left] {$\displaystyle t\mapsto x = 2t-1$};
\draw[<-,color=blue,line width=5pt]
	($(S1)+(O)$)
	--
	($(S2)+(O)$);
\node at ($(S0)+(O)$) [below right] {$\displaystyle x\mapsto t = \frac{x+1}2$};

\end{tikzpicture}
\end{document}

