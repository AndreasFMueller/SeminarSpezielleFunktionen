%
% orthogonal.tex -- plots of legendre polynomials
%
% (c) 2021 Prof Dr Andreas Müller, OST Ostschweizer Fachhochschule
%
\documentclass[tikz]{standalone}
\usepackage{amsmath}
\usepackage{times}
\usepackage{txfonts}
\usepackage{pgfplots}
\usepackage{csvsimple}
\usetikzlibrary{arrows,intersections,math}
\begin{document}
\def\skala{6}
\input{legendrepaths.tex}
\begin{tikzpicture}[>=latex,thick,scale=\skala]

\definecolor{fone}{rgb}{1,0,0}
\definecolor{ftwo}{rgb}{1.0,0,0.8}
\definecolor{fthree}{rgb}{0.8,0,1}
\definecolor{ffour}{rgb}{0,0,1}
\definecolor{ffive}{rgb}{0,0.8,1}
\definecolor{fsix}{rgb}{0,1,1}
\definecolor{fseven}{rgb}{0,0.6,0}
\definecolor{feight}{rgb}{0.2,1,0.6}
\definecolor{fnine}{rgb}{0.6,0.8,0.2}
\definecolor{ften}{rgb}{1,0.4,0}

\def\dx{1}
\def\Dy{3}
\def\dy{3}

\begin{scope}
\clip (-1,-0.6) rectangle (1,1);

%\pgfmathparse{\Dy/(\normalizationfour*\normalizationfour)}
%\xdef\dy{\pgfmathresult}
\fill[color=ffour!70,opacity=0.5] (-1,0) -- \produktvier -- (1,0) -- cycle;

%\pgfmathparse{\Dy/(\normalizationeight*\normalizationeight)}
%\xdef\dy{\pgfmathresult}
\fill[color=fseven!70,opacity=0.5] (-1,0) -- \produktsieben -- (1,0) -- cycle;

%\pgfmathparse{\Dy/(\normalizationfour*\normalizationeight)}
%\xdef\dy{\pgfmathresult}
\fill[color=red!50,opacity=0.5] (-1,0) -- \produktortho -- (1,0) -- cycle;

%\pgfmathparse{\Dy/\normalizationfour}
%\xdef\dy{\pgfmathresult}
%\draw[line width=1.4pt,color=ffour] \legendrefour;
%
%\pgfmathparse{\Dy/\normalizationeight}
%\xdef\dy{\pgfmathresult}
%\draw[line width=1.4pt,color=feight] \legendreeight;

%\pgfmathparse{\Dy/(\normalizationfour*\normalizationeight)}
%\xdef\dy{\pgfmathresult}
\draw[line width=1.4pt,color=red] \produktortho;

\end{scope}

\draw[->] ({-1-(0.1/\skala)},0) -- ({1+(0.3/\skala)},0)
	coordinate[label={$x$}];
\draw[->] (0,{-{0.2*\Dy}-(0.1/\skala)}) -- (0,{1+(0.3/\skala)})
	coordinate[label={right:$y$}];
\foreach \x in {-1,-0.9,...,1.001}{
	\draw ({\dx*\x},{-0.1/\skala}) -- ({\dx*\x},{0.1/\skala});
}
\foreach \y in {-0.2,-0.1,0.1,0.2,0.3}{
	\draw ({-0.1/\skala},{\Dy*\y}) -- ({0.1/\skala},{\Dy*\y});
	\node at ({-0.1/\skala},{\Dy*\y}) [left] {$\mathstrut\y$};
}
\foreach \x in {-1,-0.5,0.5,1}{
	\node at ({\dx*\x},{-0.1/\skala}) [below] {$\mathstrut\x$};
}

\end{tikzpicture}
\end{document}

