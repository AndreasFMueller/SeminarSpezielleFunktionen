%
% jacobi.tex -- template for standalon tikz images
%
% (c) 2021 Prof Dr Andreas Müller, OST Ostschweizer Fachhochschule
%
\documentclass[tikz]{standalone}
\usepackage{amsmath}
\usepackage{times}
\usepackage{txfonts}
\usepackage{pgfplots}
\usepackage{csvsimple}
\usetikzlibrary{arrows,intersections,math}
\begin{document}
\def\skala{1.07}
\input{jacobipaths.tex}
\begin{tikzpicture}[>=latex,thick,scale=\skala]

\def\dx{6}
\def\dy{2}
\def\dwy{1}

\begin{scope}
	\begin{scope}
		\clip (-6,-2.55) rectangle (6,2.55);
		\fill[color=red!20] \legendreweight
			-- ({1*\dx},0) -- ({-1*\dx},0) -- cycle;
		%\draw[color=blue] \legendrea;
		\draw[color=blue] \legendreb;
		\draw[color=blue] \legendrec;
		\draw[color=blue] \legendred;
		\draw[color=blue] \legendree;
		\draw[color=blue] \legendref;
		\draw[color=blue] \legendreg;
		\draw[color=blue] \legendreh;
		\draw[color=blue] \legendrei;
		\draw[color=blue] \legendrek;
		\draw[color=red] \legendreweight;
	\end{scope}
	\draw[->] (-6.1,0) -- (6.3,0) coordinate[label={$x$}];
	\draw[->] (0,-2.55) -- (0,2.6) coordinate[label={right:$y$}];
	\node at (6,2.3) [left] {$a=0$};
	\node at (-6,2.3) [right] {$b=0$};
	\draw ({-1*\dx},-0.1) -- ({-1*\dx},0.1);
	\draw ({1*\dx},-0.1) -- ({1*\dx},0.1);
	\node at ({1*\dx},-0.1) [below] {$1$};
	\node at ({-0.97*\dx},-0.1) [below left] {$-1$};
\end{scope}

\begin{scope}[yshift=-5.6cm]
	\begin{scope}
		\clip (-6,-2.55) rectangle (6,2.55);
		\fill[color=red!20] \aoneweight
			-- ({1*\dx},0) -- ({-1*\dx},0) -- cycle;
		%\draw[color=blue] \aonea;
		\draw[color=blue] \aoneb;
		\draw[color=blue] \aonec;
		\draw[color=blue] \aoned;
		\draw[color=blue] \aonee;
		\draw[color=blue] \aonef;
		\draw[color=blue] \aoneg;
		\draw[color=blue] \aoneh;
		\draw[color=blue] \aonei;
		\draw[color=blue] \aonek;
		\draw[color=blue] \aonel;
		\draw[color=blue] \aonem;
		\draw[color=blue] \aonen;
		\draw[color=blue] \aoneo;
		\draw[color=red] \aoneweight;
	\end{scope}
	\draw[->] (-6.1,0) -- (6.3,0) coordinate[label={$x$}];
	\draw[->] (0,-2.55) -- (0,2.6) coordinate[label={right:$y$}];
	\fill[color=white,opacity=0.8] (4.9,2.1) rectangle (5.9,2.5);
	\node at (6,2.3) [left] {$a=1$};
	\node at (-6,2.3) [right] {$b=0$};
	\draw ({-1*\dx},-0.1) -- ({-1*\dx},0.1);
	\draw ({1*\dx},-0.1) -- ({1*\dx},0.1);
	\node at ({1*\dx},-0.1) [below] {$1$};
	\node at ({-0.97*\dx},-0.1) [below left] {$-1$};
\end{scope}

\begin{scope}[yshift=-11.2cm]
	\begin{scope}
		\clip (-6,-2.55) rectangle (6,2.55);
		\fill[color=red!20] \aminusoneweight
			-- ({1*\dx},3)
			-- ({1*\dx},0) -- ({-1*\dx},0) -- cycle;
		%\draw[color=blue] \aminusonea;
		\draw[color=blue] \aminusoneb;
		\draw[color=blue] \aminusonec;
		\draw[color=blue] \aminusoned;
		\draw[color=blue] \aminusonee;
		\draw[color=blue] \aminusonef;
		\draw[color=blue] \aminusoneg;
		\draw[color=blue] \aminusoneh;
		\draw[color=blue] \aminusonei;
		\draw[color=blue] \aminusonek;
		\draw[color=blue] \aminusonel;
		\draw[color=blue] \aminusonem;
		\draw[color=blue] \aminusonen;
		\draw[color=blue] \aminusoneo;
		\draw[color=red] \aminusoneweight;
	\end{scope}
	\draw[->] (-6.1,0) -- (6.3,0) coordinate[label={$x$}];
	\draw[->] (0,-2.55) -- (0,2.6) coordinate[label={right:$y$}];
	\node at (6,2.3) [left] {$a=-1$};
	\node at (-6,2.3) [right] {$b=0$};
	\draw ({-1*\dx},-0.1) -- ({-1*\dx},0.1);
	\draw ({1*\dx},-0.1) -- ({1*\dx},0.1);
	\node at ({1*\dx},-0.1) [below] {$1$};
	\node at ({-0.97*\dx},-0.1) [below left] {$-1$};
\end{scope}



\end{tikzpicture}
\end{document}

