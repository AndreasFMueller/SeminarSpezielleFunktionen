%
% laguerre.tex -- Laguerre-Polynome
%
% (c) 2021 Prof Dr Andreas Müller, OST Ostschweizer Fachhochschule
%
\documentclass[tikz]{standalone}
\usepackage{amsmath}
\usepackage{times}
\usepackage{txfonts}
\usepackage{pgfplots}
\usepackage{csvsimple}
\usetikzlibrary{arrows,intersections,math}
\begin{document}
\def\dx{1}
\def\dy{0.33333}
\input{laguerrepaths.tex}

\definecolor{fone}{rgb}{1,0,0}
\definecolor{ftwo}{rgb}{1.0,0,0.8}
\definecolor{fthree}{rgb}{0.8,0,1}
\definecolor{ffour}{rgb}{0,0,1}
\definecolor{ffive}{rgb}{0,0.8,1}
\definecolor{fsix}{rgb}{0,1,1}
\definecolor{fseven}{rgb}{0,0.6,0}
\definecolor{feight}{rgb}{0.2,1,0.6}
\definecolor{fnine}{rgb}{0.6,0.8,0.2}
\definecolor{ften}{rgb}{1,0.4,0}

\def\skala{1}
\begin{tikzpicture}[>=latex,thick,scale=\skala]


\begin{scope}
\clip (0,-5) rectangle (12,5);

\draw[color=fone] \laguerrezero;
\draw[color=ftwo] \laguerreone;
\draw[color=fthree] \laguerretwo;
\draw[color=ffour] \laguerrethree;
\draw[color=ffive] \laguerrefour;
\draw[color=fsix] \laguerrefive;
\draw[color=fseven] \laguerresix;
\draw[color=feight] \laguerreseven;
\draw[color=fnine] \laguerreeight;
\draw[color=ften] \laguerrenine;

\end{scope}

\draw[->] (-0.1,0) -- (12.4,0) coordinate[label={$x$}];
\draw[->] (0,{-5.1}) -- (0,{5.3}) coordinate[label={$y$}];

\def\marke#1#2{
	\node[color=#1] at (0.5,{5-0.42*#2}) [right] {$n=#2$};
}

\marke{fone}{0}
\marke{ftwo}{1}
\marke{fthree}{2}
\marke{ffour}{3}
\marke{ffive}{4}
\marke{fsix}{5}
\marke{fseven}{6}
\marke{feight}{7}
\marke{fnine}{8}
\marke{ften}{9}

\foreach \x in {1,...,12}{
	\draw ({\x},-0.05) -- ({\x},0.05);
}
\node at (5,-0.05) [below] {$5$};
\node at (10,-0.05) [below] {$10$};

\foreach \y in {1,...,15}{
	\draw (-0.05,{\y*\dy}) -- (0.05,{\y*\dy});
	\draw (-0.05,{-\y*\dy}) -- (0.05,{-\y*\dy});
}
\node at (-0.05,{1*\dy}) [left] {$1$};
\node at (-0.05,{5*\dy}) [left] {$5$};
\node at (-0.05,{10*\dy}) [left] {$10$};
\node at (-0.05,{15*\dy}) [left] {$15$};
\node at (-0.05,{-5*\dy}) [left] {$-5$};
\node at (-0.05,{-10*\dy}) [left] {$-10$};
\node at (-0.05,{-15*\dy}) [left] {$-15$};

\end{tikzpicture}
\end{document}

