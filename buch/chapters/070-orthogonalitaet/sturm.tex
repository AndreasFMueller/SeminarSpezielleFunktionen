%
% sturm.tex
%
% (c) 2022 Prof Dr Andreas Müller, OST Ostschweizer Fachhochschule
%
\section{Das Sturm-Liouville-Problem
\label{buch:integrale:subsection:sturm-liouville-problem}}
\rhead{Das Sturm-Liouville-Problem}
Sowohl bei den Bessel-Funktionen wie bei den Legendre-Polynomen
konnte die Orthogonalität der Funktionen dadurch gezeigt werden,
dass sie als Eigenfunktionen eines bezüglich eines geeigneten
Skalarproduktes selbstadjungierten Operators erkannt wurden.

\subsection{Differentialgleichung}
Das klassische Sturm-Liouville-Problem ist das folgende Eigenwertproblem.
Gesucht sind Lösungen der Differentialgleichung
\begin{equation}
((p(x)y'(x))' + q(x)y(x) = \lambda w(x) y(x)
\label{buch:integrale:eqn:sturm-liouville}
\end{equation}
auf dem Intervall $(a,b)$, die zusätzlich die Randbedingungen
\begin{equation}
\begin{aligned}
k_a y(a) + h_a p(a) y'(a) &= 0 \\
k_b y(b) + h_b p(b) y'(b) &= 0
\end{aligned}
\label{buch:integrale:sturm:randbedingung}
\end{equation}
erfüllen, wobei $|k_i|^2 + |h_i|^2\ne 0$ mit $i=a,b$.
Weitere Bedingungen an die Funktionen $p(x)$, $q(x)$, $w(x)$  sowie die
Lösungsfunktionen $y(x)$ sollen später geklärt werden.

\subsection{Das verallgemeinerte Eigenwertproblem für symmetrische Matrizen}
Ein zu \eqref{buch:integrale:eqn:sturm-liouville} analoges Eigenwertproblem
für Matrizen ist das folgende verallgemeinerte Eigenwertproblem.
Das gewohnte Eigenwertproblem verwendet die Matrix $B=E$.

\begin{definition}
\index{verallgemeinerter Eigenvektor}%
\index{Eigenvektor, verallgemeinerter}%
\label{buch:orthogonal:sturm:verallgemeinerter-eigenvektor}
Seien $A$ und $B$ $n\times n$-Matrizen.
$v$ heisst {\em verallgemeinerter Eigenvektor} zum Eigenwert $\lambda$,
wenn
\[
Av = \lambda Bv.
\]
\end{definition}

Für symmetrische Matrizen lässt sich dieses Problem auf ein 
Optimierungsproblem reduzieren.

\begin{satz}
Seien $A$ und $B$ symmetrische $n\times n$-Matrizen und sei ausserdem
$B$ positiv definit.
Ist $v$ ein Vektor, der die Grösse
\[
f(v)=\frac{v^tAv}{v^tBv}
\]
maximiert, ist ein verallgemeinerter Eigenvektor für die Matrizen $A$
und $B$.
\end{satz}

\begin{proof}[Beweis]
Sei $\lambda = f(v)$ der maximale Wert und $u\ne 0$ ein beliebiger Vektor. 
Da $v$ die Grösse $f(v)$ maximiert, muss die Ableitung
von $f(u+tv)$ für $t=0$ verschwinden.
Um diese Ableitung zu berechnen, bestimmen wir zunächst die Ableitung
von $(v+tu)^tM(v+tu)$  an der Stelle $t=0$ für eine beliebige
symmetrische Matrix:
\begin{align*}
\frac{d}{dt}
(v+tu)^tM(v+tu)
&=
\frac{d}{dt}\bigl(
v^tv + t(v^tMu+u^tMv) + t^2 u^tMu
\bigr)
=
v^tMu+u^tMv + 2tv^tMv
\\
\frac{d}{dt}
(v^t+tu^t)M(v+tu)
\bigg|_{t=0}
&=
v^tMu+u^tMv
=
2v^tMu
\end{align*}
Dies wenden wir jetzt auf den Quotenten $\lambda(v+tu)$ an.
\begin{align*}
\frac{d}{dt}f(v+tu)\bigg|_{t=0}
&=
\frac{d}{dt}
\frac{(v+tu)^tA(v+tu)}{(v+tu)^tB(v+tu)}\bigg|_{t=0}
\\
&=
\frac{2u^tAv(v^tBv) - 2u^tBv(v^tAv)}{(v^tBv)^2}
=
\frac{2}{v^tBv}
u^t
\biggl(
Av - \frac{v^tAv}{v^tBv} Bv
\biggr)
\\
&=
2
\frac{
u^t(
Av - \lambda Bv
)
}{v^tBv}
\end{align*}
Da $v$ ein Maximum von $\lambda(v)$ ist, verschwindet diese Ableitung
für alle Vektoren $u$, somit gilt
\[
u^t(Av-\lambda Bv)=0
\]
für alle $u$, also auch $Av=\lambda Bv$.
Dies beweist, dass $v$ ein verallgemeinerter Eigenvektor zum
Eigenwert $\lambda$ ist.
\end{proof}

\begin{satz}
Verallgemeinerte Eigenvektoren $u$ und $v$ von $A$ und $B$
zu verschiedenen Eigenwerten erfüllen $u^tBv=0$.
\end{satz}

\begin{proof}
Seien $\lambda$ und $\mu$ die Eigenwerte, also $Au=\lambda Bu$
und $Av=\mu Bv$.
Wie in \eqref{buch:integrale:eqn:eigenwertesenkrecht}
berechnen wir das Skalarprodukt auf zwei Arten
\[
\renewcommand{\arraycolsep}{2pt}
\begin{array}{rcccrl}
 u^tAv &=&u^t\lambda Bv &=& \lambda\phantom{\mathstrut-\mu)} u^tBv
	&\multirow{2}{*}{\hspace{3pt}$\bigg\}\mathstrut-\mathstrut$}\\
=v^tAu &=&v^t\mu Bu     &=&  \mu\phantom{)}u^tBv         &\\
\hline
     0 & &              &=& (\lambda - \mu)u^tBv.        &
\end{array}
\]
Da die Eigenwerte verschieden sind, ist $\lambda-\mu\ne 0$, es folgt, 
dass $u^tBv=0$ sein muss.
\end{proof}

Verallgemeinerte Eigenwerte und Eigenvektoren verhalten sich also
ganz analog zu den gewöhnlichen Eigenwerten und Eigenvektoren.
Da $B$ positiv definit ist, ist $B$ auch invertierbar.
Zudem kann $B$ zur Definition des verallgemeinerten Skalarproduktes
\[
\langle u,v\rangle_B = u^tBv
\]
verwendet werden.
Die Matrix 
\[
\tilde{A} = B^{-1}A
\]
ist bezüglich dieses Skalarproduktes selbstadjungiert, denn es gilt
\[
\langle\tilde{A}u,v\rangle_B
=
(B^{-1}Au)^t Bv
=
u^tA^t(B^{-1})^tBv
=
u^tAv
=
u^tBB^{-1}Av
=
\langle u,\tilde{A}v\rangle.
\]
Das verallgemeinerte Eigenwertproblem für symmetrische Matrizen
ist damit ein gewöhnliches Eigenwertproblem für selbstadjungierte
Matrizen des Operators $\tilde{A}$ bezüglich des verallgemeinerten
Skalarproduktes $\langle\,\;,\;\rangle_B$.

\subsection{Der Operator $L_0$ und die Randbedingung}
Die Differentialgleichung kann auch in Operatorform geschrieben werden.
Dazu schreiben wir
\[
L_0 
=
-\frac{d}{dx}p(x)\frac{d}{dx}.
\]
Bezüglich des gewöhnlichen Skalarproduktes
\[
\langle f,g\rangle
=
\int_a^b f(x)g(x)\,dx
\]
für Funktionen auf dem Intervall $[a,b]$ ist der Operator $L_0$
tatsächlich selbstadjungiert.
Mit partieller Integration rechnet man nach:
\begin{align}
\langle f,L_0g\rangle
&=
\int_a^b f(x) \biggl(-\frac{d}{dx}p(x)\frac{d}{dx}\biggr)g(x)\,dx
\notag
\\
&=
-\int_a^b f(x) \frac{d}{dx}\bigl( p(x) g'(x) \bigr)\,dx
\notag
\\
&=
-\biggl[f(x) p(x)g'(x)\biggr]_a^b
+
\int_a^b f'(x) p(x) g'(x) \,dx
\notag
\\
\langle L_0f,g\rangle
&=
-\biggl[f'(x)p(x)g(x)\biggr]_a^b
+
\int_a^b f'(x) p(x) g'(x) \,dx.
\notag
\intertext{Die beiden Skalarprodukte führen also auf das gleiche
Integral, sie unterscheiden sich nur um die Randterme}
\langle f,L_0g\rangle
-
\langle L_0f,g\rangle
&=
-f(b)p(b)g'(b) + f(a)p(a)g'(a)
+f'(b)p(b)g(b) - f'(a)p(a)g(a)
\label{buch:integrale:sturm:sabedingung}
\\
&=
-
p(b)
\left|\begin{matrix}
f(b) &g(b)\\
f'(b)&g'(b)
\end{matrix}\right|
+
p(a)
\left|\begin{matrix}
f(a) &g(a)\\
f'(a)&g'(a)
\end{matrix}\right|
\notag
\\
&=
-
\left|\begin{matrix}
f(b) &g(b)\\
p(b)f'(b)&p(b)g'(b)
\end{matrix}\right|
+
\left|\begin{matrix}
f(a) &g(a)\\
p(a)f'(a)&p(a)g'(a)
\end{matrix}\right|.
\notag
\end{align}
Um zu erreichen, dass der Operator selbstadjunigert wird, muss 
sichergestellt werden, dass entweder $p$ oder die Determinanten
an den Intervallenden verschwinden.
Dies passiert genau dann, wenn die Vektoren 
\[
\begin{pmatrix}
f(a)\\
p(a)f'(a)
\end{pmatrix}
\text{\;und\;}
\begin{pmatrix}
g(a)\\
p(a)g'(a)
\end{pmatrix}
\]
linear abhängig sind.
In zwei Dimensionen bedeutet lineare Abhängigkeit, dass es
eine nichttriviale Linearform mit Koeffizienten $h_a, k_a$ gibt,
die auf beiden Vektoren verschwindet.
Ausgeschrieben bedeutet dies, dass die Randbedingung
\eqref{buch:integrale:sturm:randbedingung}
erfüllt sein muss.

\subsection{Skalarprodukt}
Das Ziel der folgenden Abschnitte ist, das Sturm-Liouville-Problem als
Eigenwertproblem für einen selbstadjungierten Operator in einem 
Funktionenraum mit einem geeigneten Skalarprodukt zu finden.

Wir haben bereits gezeigt, dass die Randbedingung
\eqref{buch:integrale:sturm:randbedingung} sicherstellt, dass der
Operator $L_0$ für das Standardskalarprodukt selbstadjungiert ist.
Dies entspricht der Symmetrie der Matrix $A$.

Die Komponente $q(x)$ stellt keine besonderen Probleme, denn
\[
\langle f,qg\rangle
=
\int_a^b f(x)q(x)g(x)\,dx
=
\langle qf,g\rangle.
\]
Der Operator $f(x) \mapsto q(x)f(x)$, der eine Funktion mit 
der Funktion $q(x)$ multipliziert, ist also ebenfalls symmetrisch.
Dasselbe gilt für einen Operator, der mit $w(x)$ multipliziert.
Da $w(x)$ eine positive Funktion ist, ist der Operator $f(x)\mapsto w(x)f(x)$
sogar positiv definit.
Dies entspricht der Matrix $B$.
Nach den Erkenntnissen des vorangegangenen Abschnittes ist das
verallgemeinerte Eigenwertproblem daher ein Eigenwertproblem
für einen modifizierten Operator bezüglich eines alternativen
Skalarproduktes.

Als Skalarprodukt muss also das Integral
\[
\langle f,g\rangle_w
=
\int_a^b f(x)g(x)w(x)\,dx
\]
mit der Gewichtsfunktion $w(x)$ verwendet werden.
Damit dies ein vernünftiges Skalarprodukt ist, muss $w(x)>0$ im
Innerend es Intervalls sein.

\subsection{Der Vektorraum $H$}
Damit können wir jetzt die Eigenschaften der in Frage kommenden
Funktionen zusammenstellen.
Zunächst müssen sie auf dem Intervall $[a,b]$ definiert sein und
das Integral
\[
\int_a^b |f(x)|^2 w(x)\,dx < \infty
\]
muss existieren.
Wir bezeichnen den Vektorraum der Funktionen, deren Quadrat mit
der Gewichtsfunktion $w(x)$ integrierbar sind, mit
$L^2([a,b],w)$.

Damit auch $\langle qf,f\rangle_w$ und $\langle L_0f,f\rangle_w$
wohldefiniert sind, müssen zusätzlich die Integrale
\[
\int_a^b |f(x)|^2 q(x) w(x)\,dx
\qquad\text{und}\qquad
\int_a^b |f'(x)|^2 p(x) w(x)\,dx
\]
existieren.
Wir setzen daher
\[
H
=
\biggl\{
f\in L^2([a,b],w)\;\bigg|\;
\int_a^b |f'(x)|^2p(x)w(x)\,dx<\infty,
\int_a^b |f(x)|^2q(x)w(x)\,dx<\infty
\biggr\}.
\]

\subsection{Differentialoperator}
Das verallgemeinerte Eigenwertproblem für $A$ und $B$ ist ein
gewöhnliches Eigenwertproblem für die Operator $\tilde{A}=B^{-1}A$
bezüglich des modifizierten Skalarproduktes.
Das Sturm-Liouville-Problem ist also ein Eigenwertproblem im
Vektorraum $H$ mit dem Skalarprodukt $\langle\,\;,\;\rangle_w$.
Der Operator
\[
L = \frac{1}{w(x)} \biggl(-\frac{d}{dx} p(x)\frac{d}{dx} + q(x)\biggr)
\]
heisst der {\em Sturm-Liouville-Operator}.
Eine Lösung des Sturm-Liouville-Problems ist eine Funktion $y(x)$ derart,
dass 
\[
Ly = \lambda y,
\]
$\lambda$ ist der zu $y(x)$ gehörige Eigenwert.
Der Operator ist definiert auf Funktionen des im vorangegangenen Abschnitt
definierten Vektorraumes $H$.

\subsection{Beispiele}
Die meisten der früher vorgestellten Funktionenfamilien stellen sich
als Lösungen eines geeigneten Sturm-Liouville-Problems heraus.
Alle Eigenschaften aus der Sturm-Liouville-Theorie gelten daher
automatisch für diese Funktionenfamilien.

\subsubsection{Trigonometrische Funktionen}
Die trigonometrischen Funktionen sind Eigenfunktionen des Operators
$d^2/dx^2$, also eines Sturm-Liouville-Operators mit $p(x)=1$, $q(x)=0$
und $w(x)=1$.
Auf dem Intervall $(-\pi,\pi)$ können wir die Randbedingungen
\bgroup
\renewcommand{\arraycolsep}{2pt}
\[
\begin{aligned}
&
\begin{array}{lclclcl}
k_{-\pi}          &=&1,&\qquad&h_{-\pi}          &=&0\\
k_{\phantom{-}\pi}&=&1,&\qquad&h_{\phantom{-}\pi}&=&0
\end{array}
\;\bigg\}
&&\Rightarrow&
\begin{array}{lcl}
y(-\pi)          &=&0\\
y(\phantom{-}\pi)&=&0\\
\end{array}
\;\bigg\}
&\quad\Rightarrow&
y(x) &= B\sin nx
\\
&
\begin{array}{lclclcl}
k_{-\pi}          &=&0,&\qquad&h_{-\pi}          &=&1\\
k_{\phantom{-}\pi}&=&0,&\qquad&h_{\phantom{-}\pi}&=&1
\end{array}
\;\bigg\}
&&\Rightarrow&
\begin{array}{lcl}
y'(-\pi)          &=&0\\
y'(\phantom{-}\pi)&=&0\\
\end{array}
\; \bigg\}
&\quad\Rightarrow&
y(x) &= A\cos nx
\end{aligned}
\]
\egroup
verwenden.
Die Orthogonalität der Sinus- und Kosinus-Funktionen folgt jetzt
ganz ohne weitere Rechnung.

An dieser Lösung ist nicht ganz befriedigend, dass die trigonometrischen
Funktionen nicht mit einer einzigen Randbedingung gefunden werden können.
Der Ausweg ist, periodische Randbedingungen zu verlangen, also
$y(-\pi)=y(\pi)$ und $y'(-\pi)=y'(\pi)$.
Dann ist wegen
\begin{align*}
\langle f,L_0g\rangle - \langle L_0f,g\rangle
&=
-f(\pi)g'(\pi)+f(-\pi)g'(-\pi)+f'(\pi)g(\pi)-f'(-\pi)g(-\pi)
\\
&=
-f(\pi)g'(\pi)+f(\pi)g'(\pi)+f'(\pi)g(\pi)-f'(\pi)g(\pi)
=0
\end{align*}
die Bedingung~\eqref{buch:integrale:sturm:sabedingung}
ebenfalls erfüllt, $L_0$ ist in diesem Raum selbstadjungiert.

\subsubsection{Bessel-Funktionen $J_n(x)$}
Der Bessel-Operator \eqref{buch:differentialgleichungen:bessel-operator}
kann wie folgt in die Form eines Sturm-Liouville-Operators gebracht 
werden.
Zunächst rechnet man
\[
B
=
x^2\frac{d^2}{dx^2} + x\frac{d}{dx} + x^2
=
x\biggl(
x\frac{d^2}{dx^2} + \frac{d}{dx} + x
\biggr)
=
x\biggl(
\frac{d}{dx}(-x)\frac{d}{dx} + x
\biggr).
\]
Somit ist $B$ ein Sturm-Liouville-Operator für 
\begin{equation}
\begin{aligned}
p(x) &= -x \\
q(x) &= x \\
w(x) &= \frac{1}{x}.
\end{aligned}
\label{buch:orthogonal:sturm:bessel:n}
\end{equation}
Am linken Rand kann als Randbedingung 
\[
\lim_{x\to 0} p(x) y'(x) = 0
\]
verwendet werden, die für alle Bessel-Funktionen erfüllt ist.
Dies entspricht der Wahl $k_0=0$ und $h_0=1$.
Am rechten Rand für $x\to\infty$ kann man
\[
\lim_{x\to\infty} y(x)=0
\]
verlangen, was der Wahl $k_\infty=1$ und $h_\infty=0$ entspricht.
Damit ist die Bessel-Differentialgleichung erkannt als ein
Sturm-Liouville-Problem für $\lambda=n^2$.
Es folgt damit sofort, dass die Besselfunktionen orthogonale
Funktionen bezüglich des Skalarproduktes mit der Gewichtsfunktion
$w(x)=1/x$ sind.

\subsubsection{Bessel-Funktionen $J_n(s x)$}
Das Sturm-Liouville-Problem mit den Funktionen
\eqref{buch:orthogonal:sturm:bessel:n}
ist jedoch nicht die einzige Möglichkeit, die Bessel-Differentialgleichung
in ein Sturm-Liouville-Problem zu verwandeln.
Das Problem \eqref{buch:orthogonal:sturm:bessel:n} ging davon
aus, dass $n^2$ der verallgemeinerte Eigenwert sein soll.
Im Folgenden sollen hingegen die Funktionen $J_n(s x)$ für
konstantes $n$, aber verschiedene $s$ untersucht und
als orthogonal erkannt werden.

Die Funktion $y(x) = J_n(x)$ ist eine Lösung der Bessel-Differentialgleichung
\[
x^2y'' + xy' + x^2y = n^2y.
\]
Setzt man $x=s t$ und $f(t)=y(s t)$, dann wird die Ableitung 
\[
\begin{aligned}
f'(t)
&=
\frac{d}{dt}y(s t)
=
y'(s t) \cdot s
&&\Rightarrow
&
\frac{f'(t)}{s}
&=
y'(x)
\\
f''(t)
&=
\frac{d^2}{dt^2} y(s t)
=
y''(s t) \cdot s^2
&&\Rightarrow
&
\frac{f''(t)}{s^2}
&=
y''(x).
\end{aligned}
\]
Setzt man diese in die Besselsche Differentialgleichung ein,
findet man
\begin{align*}
x^2y''+xy'+x^2y
=
s^2 t^2 \frac{f''(t)}{s^2}
+
s t \frac{f'(t)}{s}
+
s^2 t^2 f(t)
&=
n^2 f(t).
\end{align*}
Damit ist gezeigt, dass die Funktionen $J_n(s x)$ Lösungen
der Differentialgleichung
\begin{equation}
x^2y'' + xy' + (s^2 x^2  - n^2) y = 0
\label{buch:orthogonal:sturm:eqn:bessellambda}
\end{equation}
ist.

Die Differentialgleichung
\eqref{buch:orthogonal:sturm:eqn:bessellambda}
soll jetzt ebenfalls als ein Sturm-Liouville-Problem betrachtet
werden, diesmal aber mit festem $n$ und $s^2$ als dem verallgemeinerten
Eigenwert.
Dazu wird
\begin{equation}
\begin{aligned}
p(x) &= -x \\
q(x) &= -\frac{n^2}{x} \\
w(x) &= x
\end{aligned}
\label{buch:orthgonal:sturm:bessel:lambdaparams}
\end{equation}
gesetzt.
Das zugehörige Sturm-Liouville-Problem ist jetzt
\[
\frac{1}{x}\biggl(
\frac{d}{dx} (-x)\frac{d}{dx} -\frac{n^2}{x}
\biggr)
y
=
\lambda y
\quad\Rightarrow\quad
y'' + \frac{1}{x}y' - \frac{n^2}{x^2}y = \lambda y,
\]
oder nach Multiplikation mit $x^2$
\begin{equation}
x^2y'' + xy' + ((-\lambda)x^2 - n^2) y = 0.
\end{equation}
Die Funktionen $J_n(sx)$ sind daher verallgemeinerte Eigenfunktionen
des Sturm-Liouville-Problems
\eqref{buch:orthgonal:sturm:bessel:lambdaparams}
für den Eigenwert $\lambda = -s^2$.

\begin{satz}[Orthogonalität der Bessel-Funktionen]
Die Bessel-Funktionen $J_n(sx)$ für verschiedene $s$ sind orthogonal
bezüglich des Skalarproduktes mit der Gewichtsfunktion $w(x)=x$,
d.~h.
\[
\int_0^\infty J_n(s_1x) J_n(s_2x) x\,dx
=
0
\]
für $s_1\ne s_2$.
\end{satz}

\begin{proof}[Beweis]
Um die Bessel-Funktionen als Lösung des Sturm-Liouville-Problems
\eqref{buch:orthgonal:sturm:bessel:lambdaparams}
zu betrachten, müssen noch geeignete Randbedingungen formuliert werden.
Für $n>0$ kann man 
$J_n(0)=0$ verwenden, also $k_0=1$ und $h_0=0$.
Für $J_0$ ist dies nicht geeignet, aber wegen $J_0'(0)=0$ kann
man für $n=0$ verwenden $k_0=0$ und $h_0=1$ wählen.

Für den rechten Rand kann man verwenden, dass die Ableitung der
Bessel-Funktionen wie $x^{-3/2}$ gegen $0$ geht, gilt
\[
\lim_{x\to\infty} p(x) J_n(sx) = 0,
\]
weil $p(x)J_n(sx)$ wie $x^{-1/2}$ gegen $0$ geht.
Dies bedeutet, dass $k_\infty=0$ und $h_\infty=1$
verwendet werden kann.
Damit sind geeignete Randbedingungen für das Sturm-Liouville-Problem
gefunden.
\end{proof}

\subsubsection{Laguerre-Polynome}
Die Laguerre-Polynome sind orthogonal bezüglich des Skalarprodukts
mit der Laguerre-Gewichtsfunktion $w(x)=e^{-x}$ und erfüllen die
Laguerre-Differentialgleichung
\eqref{buch:differentialgleichungen:eqn:laguerre-dgl}.
mit $p(x)=-xe^{-x}$ wird 
\[
\frac{1}{w(x)}
\biggl(
-
\frac{d}{dx} p(x) \frac{d}{dx}
\biggr)
=
e^x \biggl(xe^{-x}\frac{d^2}{dx^2} + (1-x)e^{-x}\frac{d}{dx}\biggr),
\]
dies sind die abgeleiteten Terme in der Laguerre-Differentialgleichung.
Der Definitionsbereich ist $(0,\infty)$.
Als Randbedingung am linken kann man $y(0)=1$ verwenden, welche
auch die Laguerre-Polynome ergeben hat.
Am rechten Rand ist die Bedingung
\[
\lim_{x\to\infty} p(x)y'(x)
=
\lim_{x\to\infty} xe^{-x} y'(x)
=
0
\]
für beliebige Polynomlösungen erfüllt, dies ist der Fall
$k_{\infty}=0$ und $h_\infty=1$.

Das zugehörige verallgemeinerte Eigenwertproblem  wird damit
\[
xy'' + (1-x)y' - \lambda y = 0,
\]
also die Laguerre-Differentialgleichung.
Somit folgt, dass die Laguerre-Polynome orthogonal sind bezüglich
des Skalarproduktes mit der Laguerre-Gewichtsfunktion.

\subsubsection{Tschebyscheff-Polynome}
Die Tschebyscheff-Polynome sind Lösungen der
Tschebyscheff-Differentialgleichung
\[
(1-x^2)y'' -xy' = n^2y
\]
auf dem Intervall $(-1,1)$.
Auch dieses Problem kann als Sturm-Liouville-Problem formuliert
werden mit
\begin{align*}
w(x) &= \frac{1}{\sqrt{1-x^2}} \\
p(x) &= \sqrt{1-x^2} \\
q(x) &= 0
\end{align*}
Das zugehörige Sturm-Liouville-Eigenwertproblem ist
\[
\frac{d}{dx}\sqrt{1-x^2}\frac{d}{dx} y(x)
=
\lambda \frac{1}{\sqrt{1-x^2}} y(x).
\]
Führt man die Ableitungen auf der linken Seite aus, entsteht die
Gleichung
\begin{align*}
\sqrt{1-x^2} y''(x) -\frac{x}{\sqrt{1-x^2}}y'(x)
&=  \lambda \frac{1}{\sqrt{1-x^2}} y(x)
\intertext{Multiplikation mit $\sqrt{1-x^2}$ ergibt}
(1-x^2)
y''(x) 
-
xy'(x)
&=
\lambda y(x).
\end{align*}
Es folgt, dass die Tschebyscheff-Polynome orthogonal sind 
bezüglich des Skalarproduktes
\[
\langle f,g\rangle = \int_{-1}^1 f(x)g(x)\frac{dx}{\sqrt{1-x^2}}.
\]

\subsubsection{Jacobi-Polynome}
TODO


\subsubsection{Hypergeometrische Differentialgleichungen}
%\url{https://encyclopediaofmath.org/wiki/Hypergeometric_equation}
Auch die Eulersche hypergeometrische Differentialgleichung
lässt sich in die Form eines Sturm-Liouville-Operators
bringen.
Dazu setzt man
\begin{align*}
p(z)
&=
z^c(z-1)^{a+b+1-c}
\\
q(z)
&=
-abz^{c-1}(z-1)^{a+b-c}
\\
w(z)
&=
z^{c-1}(z-1)^{a+b-c}.
\end{align*}
Setzt man dies in den Sturm-Liouville-Operator ein, erhält man
\begin{equation}
L
=
-\frac{d}{dz}p(z)\frac{d}{dz} + q(z)
=
-p(z)\frac{d^2}{dz^2}
-p'(z)\frac{d}{dz}
+q(z)
\label{buch:orthgonalitaet:eqn:hypersturm}
\end{equation}
Wir brauchen also
\begin{align*}
p'(z)
&=
cz^{c-1}(z-1)^{a+b+1-c}
+
(a+b+1-c)
z^c
(z-1)^{a+b-c}
\\
&=
\bigl(
c(z-1)+
(a+b+1-c)z
\bigr)
\cdot
z^{c-1}(z-1)^{a+b-c}
\\
&=
-
\bigl(
c-(a+b+1)z
\bigr)
\cdot
z^{c-1}(z-1)^{a+b-c}.
\end{align*}
Einsetzen in~\eqref{buch:orthgonalitaet:eqn:hypersturm} liefert
\begin{align*}
L
%=
%-\frac{d}{dz}p(z)\frac{d}{dz}+q(z)
&=
-z^c(z-1)^{a+b+1-c} \frac{d^2}{dz^2}
+
w(z)
(c-(a+b+1)z)
\frac{d}{dz}
-
abw(z)
\\
&=
w(z)
\biggl(
-
z(z-1)
\frac{d^2}{dz^2}
+
(c-(a+b+1)z)
\frac{d}{dz}
-ab
\biggr)
\\
&=
w(z)
\biggl(
z(1-z)
\frac{d^2}{dz^2}
+
(c-(a+b+1)z)
\frac{d}{dz}
-ab
\biggr).
\end{align*}
Die Klammer auf der rechten Seite ist tatsächlich die linke Seite der
eulerschen hypergeometrischen Differentialgleichung.

Die hypergeometrische Funktion $\mathstrut_2F_1(a,b;c;z)$ ist ein
Eigenvektor des Operators $L$ zum Eigenwert $\lambda$.
Sei jetzt $w(z)$ eine Eigenfunktion zum Eigenwert $\lambda\ne 0$,
also
\[
z(1-z)w''(z) + (c-(a+b+1)z)w'(z) - ab w(z) = \lambda w(z).
\]
Kann man $a$ und $b$ so in $a_1$ und $b_1$ ändern, dass $a+b=a_1+b_1$
gleich bleiben aber das Produkt den Wert $a_1b_1=ab-\lambda$?
$a_1$ und $b_1$ sind die Lösungen der quadratischen Gleichung
\[
x^2 - (a+b)x + ab-\lambda = 0.
\]
Alle Eigenfunktionen des Operators $L$ sind also hypergeometrische
Funktion $\mathstrut_2F_1$.

Da die Gewichtsfunktion $w(z)$ bei der Ersetzung $a\to a_1$ und $b\to b_1$
sich nicht ändert ($w(z)$ hängt nur von der Summe $a+b$ ab, welche sich
nicht ändert), sind die beide beiden Eigenfunktionen bezüglich
des Skalarproduktes mit der Gewichtsfunktion $w(z)$ orthogonal.






