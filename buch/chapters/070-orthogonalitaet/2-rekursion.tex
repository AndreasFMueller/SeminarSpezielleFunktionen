%
% 2-rekursion.tex -- drei term rekursion für orthogonale Polynome
%
% (c) 2022 Prof Dr Andreas Müller, OST Ostschweizer Fachhochschule
%
\section{Drei-Term-Rekursion für orthogonale Polynome
\label{buch:orthogonal:section:drei-term-rekursion}}
\kopfrechts{Drei-Term-Rekursion für orthogonale Polynome}
Die Berechnung der Legendre-Polynome mit Hilfe des Gram-Schmidt-Verfahrens
ist wenig hilfreich, wenn es darum geht, Werte der Polynome zu berechnen.
Glücklicherweise erfüllen orthogonale Polynome automatisch eine 
Rekursionsbeziehung mit nur drei Termen.
Zum Beispiel kann man zeigen, dass für die Legendre-Polynome die
Relation
\begin{align*}
nP_n(x) &= (2n-1)xP_{n-1}(x) - (n-1)P_{n-2}(x),\;\forall n\ge 2,
\\
P_1(x) &= x,
\\
P_0(x) &= 1.
\end{align*}
Mit so einer Rekursionsbeziehung ist es sehr einfach, die Funktionswerte
für alle $P_n(x)$ zu berechnen.

%
% Allgemeine Drei-Term-Rekursion für orthogonale Polynome
%
\subsubsection{Allgemeine Drei-Term-Rekursion für orthogonale Polynome}
Die Multiplikation mit $x$ macht aus einem Polynom vom Grad $n$ ein
Polynom vom Grad $n+1$.
Das Polynom $xp_n(x)$ lässt sich daher als Linearkombination der
Polynome $p_k(x)$ mit $k\le n+1$ schreiben.
Es muss also eine lineare Beziehung zwischen den Polynomen $p_k(x)$ und
$xp_n(x)$ geben, die man nach $p_{n+1}(x)$ auflösen kann, um eine lineare
Darstellung von $p_{n+1}(x)$ durch die $p_k(x)$ und $p_n(x)$ zu
bekommen.
A priori muss man damit rechnen, dass sehr viele Summanden nötig sind.
Der folgende Satz besagt, dass dies nicht so ist.
Überraschenderweise erfüllen die Polynome $p_n(x)$ eine Rekursionsbeziehung
mit nur drei Termen.

\begin{satz}
\index{Satz!Drei-Term-Rekursion}%
\label{buch:orthogonal:satz:drei-term-rekursion}
Eine Folge bezüglich des Skalarprodukts $\langle\,\;,\;\rangle_w$
orthogonaler Polynome $p_n$ 
mit dem Grade $\deg p_n = n$ erfüllt eine Rekursionsbeziehung der Form
\begin{equation}
p_{n+1}(x)
=
(A_nx+B_n)p_n(x) - C_np_{n-1}(x)
\label{buch:orthogonal:eqn:rekursion}
\end{equation}
für $n\ge 0$, wobei $p_{-1}(x)=0$ gesetzt wird.
Die Zahlen $A_n$, $B_n$ und $C_n$ sind reell und es ist
$A_{n-1}A_nC_n\ge 0$ für $n>0$. 
Wenn $k_n>0$ der Leitkoeffizient von $p_n(x)$ ist, dann gilt
\begin{equation}
A_n=\frac{k_{n+1}}{k_n},
\qquad
C_{n+1} = \frac{A_{n+1}}{A_n}\frac{h_{n+1}}{h_n}.
\label{buch:orthogonal:eqn:koeffizientenrelation}
\end{equation}
\end{satz}

Die Rekursionsbeziehung~\eqref{buch:orthogonal:eqn:rekursion} bedeutet,
dass sich die Werte $p_n(x)$ ausgehend von $p_1(x)$ und
$p_0(x)$ mit nur $O(n)$ Operationen ermitteln lassen.

%
% Multiplikationsoperator
%
\subsubsection{Multiplikationsoperator mit $x$}
\index{Multiplikationsoperator}%
Man kann die Relation \eqref{buch:orthogonal:eqn:rekursion}
auch nach dem Produkt $xp_n(x)$ auflösen, dann wird sie
\begin{equation}
xp_n(x)
=
\frac{1}{A_n}p_{n+1}(x)
-
\frac{B_n}{A_n}p_n(x)
+
\frac{C_n}{A_n}p_{n-1}(x).
\label{buch:orthogonal:eqn:multixrelation}
\end{equation}
Die Multiplikation mit $x$ ist eine lineare Abbildung im Raum der Funktionen,
die wir weiter unten auch $M_x$ abkürzen.
Die Relation~\eqref{buch:orthogonal:eqn:multixrelation} besagt, dass diese
Abbildung in der Basis der Polynome $p_k$ tridiagonale Form hat.

%
% Drei-Term-Rekursion für die Tschebyscheff-Polynome
%
\subsubsection{Drei-Term-Rekursion für die Tschebyscheff-Polynome}
Eine Relation der Form~\eqref{buch:orthogonal:eqn:multixrelation}
wurde bereits in 
Abschnitt~\ref{buch:potenzen:tschebyscheff:rekursionsbeziehungen}
hergeleitet.
In der Form~\eqref{buch:orthogonal:eqn:rekursion} geschrieben lautet
sie
\begin{equation}
T_{n+1}(x) = 2x\,T_n(x)-T_{n-1}(x)
\qquad\Rightarrow\qquad
xT_n(x) = \frac12 T_{n-1}(x) + \frac12 T_{n+1}(x).
\label{buch:orthogonal:eqn:Tschebrel}
\end{equation}
In der Form
\eqref{buch:orthogonal:eqn:rekursion}
müssen die Koeffizienten
$A_n=2$, $B_n=0$ und $C_n=1$ gewählt werden.
Die Matrixdarstellung des Multiplikationsoperators $M_x$ in der
Basis der Tschebyscheff-Polynome hat wegen
\eqref{buch:orthogonal:eqn:Tschebrel}
die Form
\begin{equation}
M_x
=
\begin{pmatrix}
      0&\frac12&      0&      0&      0&\dots  \\
\frac12&      0&\frac12&      0&      0&\dots  \\
      0&\frac12&      0&\frac12&      0&\dots  \\
      0&      0&\frac12&      0&\frac12&\dots  \\
      0&      0&      0&\frac12&      0&\dots  \\
 \vdots& \vdots& \vdots& \vdots& \vdots&\ddots 
\end{pmatrix}.
\label{buch:orthogonal:eqn:Mx}
\end{equation}

%
% Beweis von Satz
%
\subsubsection{Beweis von Satz~\ref{buch:orthogonal:satz:drei-term-rekursion}}
Die Relation~\eqref{buch:orthogonal:eqn:multixrelation} zeigt auch,
dass der Beweis die Koeffizienten $\langle xp_k,p_j\rangle_w$
berechnen muss.
Dabei wird wiederholt der folgende Trick verwendet.
Für jede beliebige Funktion $f$ mit $\|f\|_w^2<\infty$ ist
\[
\langle fp_k,p_j\rangle_w
=
\int_a^b \bigl(f(x)p_k(x)\bigr)\,p_j(x)\,w(x)\,dx
=
\int_a^b p_k(x)\,\bigl(f(x)p_j(x)\bigr)\,w(x)\,dx
=
\langle p_k,fp_j\rangle_w.
\]
Für $f(x)=x$ kann man weiter verwenden, dass $xp_k(x)$ ein Polynom
vom Grad $k+1$ ist.
Die Gleichheit $\langle xp_k,p_j\rangle_w=\langle p_k,xp_j\rangle_w$
ermöglicht also, den Faktor $x$ dorthin zu schieben, wo er nützlicher ist.

\begin{proof}[Beweis des Satzes]
Multipliziert man die rechte Seite von
\eqref{buch:orthogonal:eqn:rekursion} aus, dann ist der einzige Term
vom Grad $n+1$ der Term $A_nxp_n(x)$.
Der Koeffizient $A_n$ ist also dadurch festgelegt, dass
\begin{equation}
b(x)
=
p_{n+1}(x) - A_nxp_n(x)
\label{buch:orthogonal:rekbeweis}
\end{equation}
Grad $\le n$ hat.
Dazu müssen sich die Terme vom Grad $n+1$ in den Polynomen wegheben,
d.~h.~$k_{n+1}-A_nk_n=0$, woraus die erste Beziehung in
\eqref{buch:orthogonal:eqn:koeffizientenrelation} folgt.

Die Polynome $p_k$ sind durch Orthogonalisierung der Monome
$1$, $x$,\dots $x^{k}$ entstanden.
Dies bedeutet, dass $\langle p_n,x^k\rangle_w=0$ für alle $k<n$
gilt und daher auch $\langle p_n,Q\rangle_w=0$ für jedes Polynom
$Q(x)$ vom Grad $<n$.

Das Polynom $b(x)$ ist vom Grad $\le n$, es lässt sich also als
Linearkombination
\[
b(x) = \sum_{k=0}^n b_k p_k(x)
\]
der $p_k$ mit $k\le n$ schreiben.
Die Koeffizienten $b_j$ kann man erhalten, indem man 
\eqref{buch:orthogonal:rekbeweis} skalar mit $p_j$ multipliziert.
Dabei erhält man
\[
h_jb_j
=
\langle b,p_j\rangle_w
=
\langle p_{n+1},p_j\rangle_w
-
A_n\langle xp_n,p_j\rangle_w.
\]
Für $j\le n$ verschwindet der erste Term nach der Definition einer
Folge von orthogonalen Polynomen.
Den zweiten Term kann man in
\[
\langle xp_n,p_j\rangle_w
=
\langle p_n,xp_j\rangle_w
\]
umformen.
Darin ist $xp_j$ ein Polynom vom Grad $j+1$.
Für $n>j+1$ folgt, dass der zweite Term verschwindet.
Somit sind alle $b_j=0$ mit $j<n-1$, nur der Term $j=n-1$
bleibt bestehen.
Mit $B_n=b_n$ und $C_n=b_{n-1}$ bekommt man die somit die
Rekursionsbeziehung~\eqref{buch:orthogonal:eqn:rekursion}.

Indem man das Skalarprodukt von~\eqref{buch:orthogonal:eqn:rekursion}
mit $p_{n-1}$ bildet, findet man
\begin{align}
\underbrace{\langle
p_{n+1},p_{n-1}
\rangle_w}_{\displaystyle=0}
&=
\langle (A_nx+B_n)p_n+C_np_{n-1},p_{n-1} \rangle_w
\notag
\\
0
&=
A_n\langle xp_n,p_{n-1} \rangle_w
+B_n\underbrace{\langle p_n,b_{n-1}\rangle_w}_{\displaystyle=0}
-C_n\|p_{n-1}\|_w^2
\notag
\\
0
&=
A_n\langle p_n,xp_{n-1} \rangle_w
-C_n\|p_{n-1}\|_w^2.
\label{buch:orthogonal:eqn:rekbeweis2}
\end{align}
Indem man $xp_{n-1}$ als
\[
xp_{n-1}(x)
=
\frac{k_{n-1}}{k_n} p_n(x)
+
\sum_{k=0}^{n-1} d_kp_k(x)
\]
schreibt, bekommt man
\begin{align*}
\langle
p_n,
xp_{n-1}
\rangle_w
&=
\biggl\langle
p_n,
\frac{k_{n-1}}{k_n} p_n
+
\sum_{k=0}^{n-1} d_kp_k
\biggr\rangle_w
=
\frac{k_{n-1}}{k_n}h_n
+
\sum_{k=0}^{n-1} d_k\underbrace{\langle p_n,p_k\rangle_w}_{\displaystyle=0}.
\end{align*}
Eingesetzt in~\eqref{buch:orthogonal:eqn:rekbeweis2} erhält man
\[
A_n\frac{k_{n-1}}{k_n}h_n = C_n h_{n-1}
\qquad\Rightarrow\qquad
C_n
=
A_n\frac{k_{n-1}}{k_n}\frac{h_n}{h_{n-1}},
\]
damit ist auch die zweite Beziehung von
\eqref{buch:orthogonal:eqn:koeffizientenrelation}
bewiesen.
\end{proof}
