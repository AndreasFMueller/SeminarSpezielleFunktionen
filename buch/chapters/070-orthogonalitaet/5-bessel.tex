%
% 5-bessel.tex
%
% Bessel-Funktionen also orthogonale Funktionenfamilie
%
\section{Bessel-Funktionen als orthogonale Funktionenfamilie
\label{buch:orthogonalitaet:section:bessel}}
%\kopfrechts{Bessel-Funktionen}
Auch die Bessel-Funktionen sind eine orthogonale Funktionenfamilie.
Sie sind differenzierbaren Funktionen $f(r)$, die definiert
sind für $r>0$, mit $f'(r)=0$ und derart, dass für $r\to\infty$ $f(r)$
so schnell abnimmt, dass auch $rf(r)$ noch gegen $0$ strebt.
Das Skalarprodukt ist
\[
\langle f,g\rangle
=
\langle f,g\rangle_r
=
\int_0^\infty f(r) g(r)\,r\,dr,
\]
ein verallgemeinertes Skalarprodukt mit Gewichtsfunktion $w(r)=r$.
Als Operator verwenden wir
\[
A = \frac{d^2}{dr^2} + \frac{1}{r}\frac{d}{dr} + s(r),
\]
wobei $s(r)$ eine beliebige integrierbare Funktion sein kann.
Zunächst überprüfen wir, ob dieser Operator wirklich selbstadjungiert ist.
Dazu rechnen wir
\begin{align}
\langle Af,g\rangle
&=
\int_0^\infty
r\,\biggl(f''(r)+\frac1rf'(r)+s(r)f(r)\biggr) g(r)
\,dr
\notag
\\
&=
\int_0^\infty rf''(r)g(r)\,dr
+
\int_0^\infty f'(r)g(r)\,dr
+
\int_0^\infty s(r)f(r)g(r)\,dr.
\notag
\intertext{Der letzte Term ist symmetrisch in $f$ und $g$, daher
ändern wir daran weiter nichts.
Auf das erste Integral kann man partielle Integration anwenden und erhält}
&=
\biggl[rf'(r)g(r)\biggr]_0^\infty
-
\int_0^\infty f'(r)g(r) + rf'(r)g'(r)\,dr
+
\int_0^\infty f'(r)g(r)\,dr
+
\int_0^\infty s(r)f(r)g(r)\,dr.
\notag
\intertext{Der erste Term verschwindet wegen der Bedingungen an die
Funktionen $f$ und $g$.
Der erste Term im zweiten Integral hebt sich gegen das
zweite Integral weg.
Der letzte Term ist das Skalarprodukt von $f'$ und $g'$.
Somit ergibt sich
}
&=
-\langle f',g'\rangle
+
\int_0^\infty s(r) f(r)g(r)\,dr.
\label{buch:integrale:orthogonal:besselsa}
\end{align}
Vertauscht man die Rollen von $f$ und $g$, erhält man das Gleiche, da im
letzten Ausdruck~\eqref{buch:integrale:orthogonal:besselsa} die Funktionen
$f$ und $g$ symmetrisch auftreten.
Damit ist gezeigt, dass der Operator $A$ selbstadjungiert ist.
Es folgt nun, dass Eigenvektoren des Operators $A$ automatisch
orthogonal sind.
\kopfrechts{Bessel-Funktionen}

Eigenfunktionen von $A$ sind aber Lösungen der Differentialgleichung
\[
\begin{aligned}
&&
Af&=\lambda f
\\
&\Rightarrow\qquad&
f''(r) +\frac1rf'(r) + s(r)f(r) &= \lambda f(r)
\\
&\Rightarrow\qquad&
r^2f''(r) +rf'(r)+ (-\lambda r^2+s(r)r^2)f(r) &= 0
\end{aligned}
\]
sind.

Durch die Wahl $s(r)=1$ wird der Operator $A$ zum Bessel-Operator
$B$, definiert in
\eqref{buch:differentialgleichungen:bessel-operator}.
Die Lösungen der Besselschen Differentialgleichung zu verschiedenen Werten
des Parameters müssen also orthogonal sein, insbesondere sind die
Bessel-Funktion $J_\nu(r)$ und $J_\mu(r)$ orthogonal wenn $\mu\ne\nu$ ist.

