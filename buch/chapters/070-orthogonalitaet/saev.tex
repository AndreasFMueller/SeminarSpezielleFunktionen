\subsubsection{Selbstadjungierte Operatoren und Eigenvektoren}
Symmetrische Matrizen spielen eine spezielle Rolle in der
endlichdimensionalen linearen Algebra, weil sie sich immer
mit einer orthonormierten Basis diagonalisieren lassen.
In der vorliegenden Situation undendlichdimensionaler Vektorräume
brauchen wir eine angepasste Definition.

\begin{definition}
Eine lineare Selbstabbildung $A\colon V\to V$
eines Vektorrraums mit Skalarprodukt
heisst {\em selbstadjungiert}, wenn für alle Vektoren $u,v\in V$
heisst $\langle Au,v\rangle = \langle u,Av\rangle$.
\end{definition}

Es ist wohlbekannt, dass Eigenvektoren einer symmetrischen Matrix
zu verschiedenen Eigenwerten orthogonal sind.
Der Beweis ist direkt übertragbar, wir halten das Resultat hier
für spätere Verwendung fest.

\begin{satz}
\index{Satz!orthogonale Eigenvektoren}%
Sind $f$ und $g$ Eigenvektoren eines selbstadjungierten Operators $A$
zu verschiedenen Eigenwerten $\lambda$ und $\mu$, dann sind $f$ und $g$
orthogonal.
\end{satz}

\begin{proof}[Beweis]
Im vorliegenden Zusammenhang möchten wir die Eigenschaft nutzen,
dass Eigenfunktionen eines selbstadjungierten Operatores zu verschiedenen
Eigenwerten orthogonal sind.
Dazu seien $Df = \lambda f$ und $Dg=\mu g$ und wir rechnen
\begin{equation*}
\renewcommand{\arraycolsep}{2pt}
\begin{array}{rcccrl}
\langle Df,g\rangle &=& \langle \lambda f,g\rangle &=& \lambda\phantom{)}\langle f,g\rangle
&\multirow{2}{*}{\hspace{3pt}$\biggl\}\mathstrut-\mathstrut$}\\
=\langle f,Dg\rangle &=& \langle f,\mu g\rangle &=& \mu\phantom{)}\langle f,g\rangle&
\\[2pt]
\hline
         0           & &                        &=& (\lambda-\mu)\langle f,g\rangle&
\end{array}
\end{equation*}
Da $\lambda-\mu\ne 0$ ist, muss $\langle f,g\rangle=0$ sein.
\end{proof}

\begin{beispiel}
Sei $C^1([0,2\pi], \mathbb{C})=C^1(S^1,\mathbb{C})$
der Vektorraum der $2\pi$-periodischen differenzierbaren Funktionen mit
dem Skalarprodukt 
\[
\langle f,g\rangle
=
\frac{1}{2\pi}\int_0^{2\pi} \overline{f(t)}g(t)\,dt
\]
enthält die Funktionen $e_n(t) = e^{int}$.
Der Operator
\[
D=i\frac{d}{dt}
\]
ist selbstadjungiert, denn mit Hilfe von partieller Integration erhält man
\[
\langle Df,g\rangle
=
\frac{1}{2\pi}
\int_0^{2\pi}
\underbrace{
\overline{i\frac{df(t)}{dt}}
}_{\uparrow}
\underbrace{g(t)}_{\downarrow}
\,dt
=
\underbrace{
\frac{-i}{2\pi}
\biggl[
\overline{f(t)}g(t)
\biggr]_0^{2\pi}
}_{\displaystyle=0}
+
\frac{1}{2\pi}
\int_0^{2\pi}
\overline{f(t)}i\frac{dg(t)}{dt}
\,dt
=
\langle f,Dg\rangle
\]
unter Ausnützung der $2\pi$-Periodizität der Funktionen.

Die Funktionen $e_n(t)$ sind Eigenfunktionen des Operators $D$, denn
\[
De_n(t) = i\frac{d}{dt}e^{int} = -n e^{int} = -n e_n(t).
\]
Nach obigem Satz sind die Eigenfunktionen von $D$ orthogonal.
\end{beispiel}

Das Beispiel illustriert, dass orthogonale Funktionenfamilien
ein automatisches Nebenprodukt selbstadjungierter Operatoren sind.
