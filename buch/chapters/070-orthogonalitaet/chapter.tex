%
% chapter.tex -- Spezielle Funktionen definiert durch Integrale
%
% (c) 2021 Prof Dr Andreas Müller, Hochschule Rapperswil
%
% !TeX spellcheck = de_CH
\chapter{Orthogonalität
\label{buch:chapter:orthogonalitaet}}
\lhead{Orthogonalität}
\rhead{}
In der linearen Algebra lernt man, dass orthonormierte Basen für die
Lösung vektorgeometrischer Probleme, bei denen auch das Skalarprodukt
involviert ist, besonders günstig sind.
Die Zerlegung eines Vektors in einer Basis verlangt normalerweise nach
der Lösung eines linearen Gleichungssystems, für orthonormierte
Basisvektoren beschränkt sie sich auf die Berechnung von Skalarprodukten.

Oft dienen spezielle Funktionen als Basis der Lösungen einer linearen
partiellen Differentialgleichung (siehe Kapitel~\ref{buch:chapter:pde}).
Die Randbedingungen müssen dazu in der gewählten Basis von Funktionen
zerlegt werden.
Fourier ist es gelungen, die Idee des Skalarproduktes und der Orthogonalität
auf Funktionen zu verallgemeinern und so zum Beispiel das Wärmeleitungsproblem
zu lösen.

Der Orthonormalisierungsprozess von Gram-Schmidt wird damit auch auf
Funktionen anwendbar
(Abschnitt~\ref{buch:orthogonalitaet:section:orthogonale-funktionen}),
der Nutzen führt aber noch viel weiter,
wenn man ihn auf Polynome anwendet.
Da die Menge $K[x]$ der Polynome ein Vektorraum ist, führt er von der
Basis der Monome $\{1,x,x^2,\dots,x^n\}$ auf orthonormierte Polynome.
Diese haben jedoch eine ganze Reihe weiterer nützlicher Eigenschaften.
So wird in Abschnitt~\ref{buch:orthogonal:section:drei-term-rekursion}
gezeigt, dass sich die Werte aller Polynome einer solchen Familie mit
einer Rekursionsformel effizient berechnen lassen, die höchstens drei
Terme umfasst.
In Abschnitt~\ref{buch:orthogonalitaet:section:rodrigues} werden
die Rodrigues-Formeln vorgeführt, die Polynome durch Anwendung eines
Differentialoperators hervorbringen.
In Abschnitt~\ref{buch:orthogonal:section:orthogonale-polynome-und-dgl}
schliesslich wird gezeigt, dass diese Polynome auch Eigenfunktionen
eines selbstadjungierten Operators sind.
Da man in der linearen Algebra auch lernt, dass die Eigenvektoren einer
symmetrischen Matrix zu verschiedenen Eigenwerten orthogonal sind,
ist die Orthogonalität dieser Funktionen plötzlich nicht mehr überraschend.

Die Bessel-Funktionen von
Abschnitt~\ref{buch:differntialgleichungen:section:bessel}
sind auch Eigenfunktionen eines Differentialoperators.
Abschnitt~\ref{buch:orthogonalitaet:section:bessel} findet das zugehörige
Skalarprodukt, das ihn zu einem selbstadjungierten Operator macht.
Dies deutet andeutet, dass auch für andere Funktionenfamilien
eine entsprechende Konstruktion möglich ist.
Das in Abschnitt~\ref{buch:integrale:subsection:sturm-liouville-problem}
präsentierte Sturm-Liouville-Problem führt sie durch.
Das Kapitel schliesst mit dem
Abschnitt~\ref{buch:orthogonal:section:gauss-quadratur}
über die Gauss-Quadratur, welche die Eigenschaften orthogonaler Polynome
für einen besonders effizienten numerischen Integrationsalgorithmus
ausnutzt.

%
% orthogonal.tex
%
% (c) 2021 Prof Dr Andreas Müller, OST Ostschweizer Fachhochschule
%
\section{Orthogonale Funktionenfamilien
\label{buch:orthogonalitaet:section:orthogonale-funktionen}}
\rhead{Orthogonale Funktionenfamilien}
Die Fourier-Theorie basiert auf der Idee, Funktionen durch 
Funktionenreihen mit Summanden zu bilden, die im Sinne eines
Skalarproduktes orthogonal sind, welches mit Hilfe eines Integrals
definiert sind.
Solche Funktionenfamilien treten jedoch auch als Lösungen von
Differentialgleichungen auf.
Besonders interessant wird die Situation, wenn die Funktionen 
Polynome sind.
In diesem Abschnitt soll zunächst das Skalarprodukt definiert 
und an Hand von Beispielen gezeigt werden, wie verschiedenartige
interessante Familien von orthogonalen Polynomen gewonnen werden
können.

%
% Skalarprodukt
%
\subsection{Skalarprodukt}
Der reelle Vektorraum $\mathbb{R}^n$ trägt das Skalarprodukt
\[
\langle\;\,,\;\rangle
\colon
\mathbb{R}^n \times \mathbb{R}^n \to \mathbb{R}
:
(x,y)\mapsto \langle x, y\rangle = \sum_{k=1}^n x_iy_k,
\]
welches viele interessante Anwendungen ermöglicht.
Eine orthonormierte Basis macht es zum Beispiel besonders leicht,
eine Zerlegung eines Vektors in dieser Basis zu finden.
In diesem Abschnitt soll zunächst an die Eigenschaften erinnert
werden, die das Skalarprdoukt $\langle\;\,,\;\rangle$ zu einem nützlichen 
Werkzeug machen.

%
% Eigenschaften eines Skalarproduktes
%
\subsubsection{Eigenschaften eines Skalarproduktes}
Das Skalarprodukt erlaubt, die Länge eines Vektors $v$
als $|v| = \sqrt{\langle v,v\rangle}$ zu definieren.
Dies funktioniert natürlich nur, wenn die Wurzel auch immer
definiert ist, d.~h.~das Skalarprodukt eines Vektors mit sich
selbst darf nicht negativ sein.
Dazu dient die folgende Definition.

\begin{definition}
Sei $V$ ein reeller Vektorraum.
Eine bilineare Abbildung
\[
\langle\;\,,\;\rangle
\colon
V\times V
\to
\mathbb{R}
:
(u,v) \mapsto \langle u,v\rangle.
\]
heisst {\em positiv definit}, wenn für alle Vektoren $v \in V$ mit
\index{positiv definit}%
$v\ne 0 \Rightarrow \langle v,v\rangle > 0$ 
Die {\em Norm} eines Vektors $v$ ist
$|v|=\sqrt{\langle v,v\rangle}$.
\index{Norm}%
\end{definition}

Damit man mit dem Skalarprodukt sinnvoll rechnen kann, ist ausserdem
erforderlich, dass es eine einfache Beziehung zwischen 
$\langle x,y\rangle$ und $\langle y,x\rangle$ gibt.

\begin{definition}
Eine Bilinearform $\langle\;\,,\;\rangle$ heisst {\em symmetrisch}, wenn
\index{symmetrisch!Bilinearform}%
für zwei beliebige Vektoren $u,v\in V$
\[
\langle u,v\rangle = \langle v,u\rangle
\]
gilt.
\end{definition}

\begin{definition}
Ein {\em Skalarprodukt} auf einem reellen Vektorraum $V$ ist eine
\index{Skalarprodukt}%
positiv definite, symmetrische bilineare Abbildung
\[
\langle\;\,,\;\rangle
\colon
V\times V
\to
\mathbb{R}
:
(u,v) \mapsto \langle u,v\rangle.
\]
\end{definition}

Das Skalarprodukt $\langle u,v\rangle=u^tv$ auf dem Vektorraum 
$\mathbb{R}^n$ erfüllt die Definition ganz offensichtlich,
sie führt auf die Komponentendarstellung
\[
\langle u,v\rangle = u^tv = \sum_{k=1}^n u_iv_i.
\]
Weitere (verallgemeinerte) Skalarprodukte ergeben ergeben sich mit
jeder symmetrischen, positiv definiten Matrix $G$ und der Definition
$\langle u,v\rangle_G=u^tGv$.
Ein einfacher Spezialfall tritt auf, wenn $G$ eine Diagonalmatrix
$\operatorname{diag}(w_1,\dots,w_n)$
mit positiven Einträgen $w_i>0$ auf der Diagonalen ist.
In diesem Fall schreiben wir
\[
\langle u,v\rangle_w
=
u^t\operatorname{diag}(w_1,\dots,w_n)v
=
\sum_{k=1}^n u_iv_i\,w_i
\]
und nennen $\langle \;\,,\;\rangle_w$ das {\em gewichtete Skalarprodukt}
mit {\em Gewichten $w_i$}.

%
% Skalarprodukte auf Funktionenräumen
%
\subsubsection{Skalarprodukte auf Funktionenräumen}
Stetige Funktionen auf einem Intervall sind integrierbar.
Dies ermöglicht, ein Skalarprodukt auf dem reellen
Vektorraum der stetigen Funktionen auf einem Intervall zu definieren.

\begin{satz}
\label{buch:orthogonal:satz:skalarprodukt}
Sei $V$ der reelle Vektorraum $C([a,b])$ der reellwertigen, stetigen
Funktion auf dem Intervall $[a,b]$.
Dann ist 
\[
\langle\;\,,\;\rangle
\colon
C([a,b]) \times C([a,b]) \to \mathbb{R}
:
(f,g) \mapsto \langle f,g\rangle = \int_a^b f(x)g(x)\,dx.
\]
ein Skalarprodukt.
\end{satz}

\begin{proof}[Beweis]
Die Definition ist offensichtlich symmetrisch in $f$ und $g$.
Aus den Eigenschaften des Integrals ist klar, dass das Produkt
bilinear ist:
\begin{align*}
\langle \lambda_1 f_1+\lambda_2f_2,g\rangle
&=
\int_a^b (\lambda_1f_(x) +\lambda_2f_2(x))g(x)\,dx
=
\lambda_1\int_a^b f_1(x) g(x)\,dx
+
\lambda_2\int_a^b f_2(x) g(x)\,dx
\\
&=
\lambda_1\langle f_1,g\rangle
+
\lambda_2\langle f_2,g\rangle.
\end{align*}
Ausserdem ist es positiv definit, denn wenn $f(x_0) \ne 0$ ist,
dann gibt es wegen der Stetigkeit von $f$ eine Umgebung
$U=[x_0-\varepsilon,x_0+\varepsilon]$, derart, dass $|f(x)| > \frac12|f(x_0)|$
ist für alle $x\in U$.
Somit ist das Integral
\[
\langle f,f\rangle
=
\int_a^b |f(x)|^2\,dx
\ge
\int_{x_0-\varepsilon}^{x_0+\varepsilon} |f(x)|^2\,dx
\ge
\int_{x_0-\varepsilon}^{x_0+\varepsilon} \frac14|f(x_0)|^2\,dx
=
\frac{1}{4}|f(x_0)|^2\cdot 2\varepsilon
=
\frac{|f(x_0)|^2\varepsilon}{2}
>0,
\]
was beweist, dass $\langle\;,\;\rangle$ positiv definit und damit
ein Skalarprodukt ist.
\end{proof}

Die Definition des Skalarproduktes in
Satz~\ref{buch:orthogonal:satz:skalarprodukt}
kann noch etwas verallgemeinert werden, indem 
die Funktionswerte nicht überall auf dem Definitionsbereich 
gleich gewichtet werden. 

\begin{definition}
\label{buch:orthogonal:def:skalarproduktw}
Sei $w\colon [a,b]\to \mathbb{R}^+$ eine positive, stetige Funktion,
dann heisst
\[
\langle\;\,,\;\rangle_w
\colon
C([a,b]) \times C([a,b]) \to \mathbb{R}
:
(f,g) \mapsto \langle f,g\rangle_w = \int_a^b f(x)g(x)\,w(x)\,dx.
\]
das {\em gewichtete Skalarprodukt} mit {\em Gewichtsfunktion $w(x)$}.
\index{<,>@$\langle\;\,,\;\rangle$}%
\index{Gewichtsfunktion}
\end{definition}

%
% Gram-Schmidt-Orthonormalisierung
%
\subsection{Gram-Schmidt-Orthonormalisierung}
\index{Gram-Schmidt-Orthonormalisierung}%
In einem reellen Vektorraum $V$ mit Skalarprodukt $\langle\;\,,\;\rangle$
kann aus einer beliebigen Basis $b_1,\dots,b_n$ mit Hilfe des 
Gram-Schmidtschen Orthogonalisierungsverfahrens immer eine
orthonormierte Basis $\tilde{b}_1,\dots,\tilde{b}_n$ Basis
gewonnen werden.
Das Verfahren stellt sicher, dass für alle $k\le n$ gilt,
dass die Vektoren $b_i$ und $\tilde{b}_i$ die gleichen Unterräume
\[
\langle b_1,\dots,b_k\rangle
=
\langle \tilde{b}_1,\dots,\tilde{b}_k\rangle
\]
aufspannen.
Zur Vereinfachung der Formeln schreiben wir $v^0=v/|v|$ für einen
Einheitsvektor mit der Richtung $v$.
Die Vektoren $\tilde{b}_i$ können mit Hilfe der Formeln
\begin{equation}
\left.
\begin{aligned}
\tilde{b}_1
&=
(b_1)^0
\\
\tilde{b}_2
&=
\bigl(
b_2
-
\langle \tilde{b}_1,b_2\rangle \tilde{b}_1
\bigr)^0
\\
\tilde{b}_3
&=
\bigl(
b_3
-
\langle \tilde{b}_1,b_3\rangle \tilde{b}_1
-
\langle \tilde{b}_2,b_3\rangle \tilde{b}_2
\bigr)^0
\\
&\;\vdots
\\
\tilde{b}_n
&=
\bigl(
b_n
-
\langle \tilde{b}_1,b_n\rangle \tilde{b}_1
-
\langle \tilde{b}_2,b_n\rangle \tilde{b}_2
-\dots
-
\langle \tilde{b}_{n-1},b_n\rangle \tilde{b}_{n-1}
\bigr)^0
\end{aligned}
\right\}
\label{buch:orthogonal:eqn:orthonormalisierung}
\end{equation}
iterativ berechnet werden.
Dieses Verfahren lässt sich auch auf Funktionenräume anwenden.

Die Normierung ist nicht unbedingt nötig und manchmal unangenehm,
da die Norm unschöne Quadratwurzeln einführt.
Falls es genügt, eine orthogonale Basis zu finden, kann darauf
verzichtet werden, bei der Orthogonalisierung muss aber berücksichtigt
werden, dass die Vektoren $\tilde{b}_i$ jetzt nicht mehr Einheitslänge
haben.
Die Formeln
\begin{equation}
\left.
\begin{aligned}
\tilde{b}_0
&=
b_0
\\
\tilde{b}_1
&=
b_1
-
\frac{\langle b_1,\tilde{b}_0\rangle}{\langle \tilde{b}_0,\tilde{b}_0\rangle}\tilde{b}_0
\\
\tilde{b}_2
&=
b_2
-
\frac{\langle b_2,\tilde{b}_0\rangle}{\langle \tilde{b}_0,\tilde{b}_0\rangle}\tilde{b}_0
-
\frac{\langle b_2,\tilde{b}_1\rangle}{\langle \tilde{b}_1,\tilde{b}_1\rangle}\tilde{b}_1
\\
&\;\vdots
\\
\tilde{b}_n
&=
b_n
-
\frac{\langle b_n,\tilde{b}_0\rangle}{\langle \tilde{b}_0,\tilde{b}_0\rangle}\tilde{b}_0
-
\frac{\langle b_n,\tilde{b}_1\rangle}{\langle \tilde{b}_1,\tilde{b}_1\rangle}\tilde{b}_1
-
\dots
-
\frac{\langle b_n,\tilde{b}_{n-1}\rangle}{\langle \tilde{b}_{n-1},\tilde{b}_{n-1}\rangle}\tilde{b}_{n-1}
\end{aligned}
\right\}
\label{buch:orthogonal:eqn:orthogonalisierung}
\end{equation}
tragen dem Rechnung.

%
% Orthogonale und orthonormierte Funktionenfamilien
%
\subsubsection{Orthogonale und orthonormierte Funktionenfamilien}
In einem Funktionenraum kann man mit dem
Gram-Schmidt-Orthonormalisierungsprozess eine 
orthonormierte Funktionenfamilie im Sinne der folgenden Definition
finden.

\begin{definition}
Eine Folge $f_n(x)$ von Funktionen heisst {\em orthonormiert}, wenn 
\index{orthonormiert}%
\[
\langle f_n,f_m\rangle = \delta_{nm}
\]
für alle $n$ und $m$.
\end{definition}

Um so eine Funktionenfamilie zu bekommen, beginnt man mit irgend
einer Familie von linear unabhängigen Funktionen und wendet den
Orthonormalisierungsprozess~\eqref{buch:orthogonal:eqn:orthonormalisierung}
darauf an.

Die Normierung auf Norm $1$ ist nicht immer erwünscht oder nötig. 
Die Variante~\eqref{buch:orthogonal:eqn:orthogonalisierung}
des Prozesses liefert dann eine Familie von Funktion, die nur
paarweise orthogonal sind.
Wir nennen eine solche Familie orthogonal, wie die folgenden Definition
diesen Begriff beschreibt.

\begin{definition}
\label{buch:orthogonal:def:orthogonal}
Eine Folge von Funktionen $f_n(x)$ heisst {\em orthogonal} bezüglich des
\index{orthogonal}%
Skalarproduktes $\langle\,\;,\;\rangle_w$, wenn
\[
\langle f_n,f_m\rangle_w = h_n \delta_{nm}
\]
für alle $n$, $m$.
\end{definition}

Man beachte, dass die Konstanten $h_n$ zusätzlich bestimmt werden müssen.
Durch Division von $f_n$ durch $\sqrt{h_n}$ erhält man eine
orthonormierte Funktionenfamilie.

%
% Legendre-Polynome
%
\subsection{Legendre-Polynome
\label{buch:orthogonal:subsection:legendre-polynome}}
Der Gram-Schmidtsche Orthogonalisierungsprozess kann für jedes beliebige
Skalarprodukt aus der Folge $1$, $x$, $x^2,\dots$ der Monome ein
Folge von orthogonalisierten Polynomen machen.
In diesem Abschnitt rechnen wir den Fall der konstanten Gewichtsfunktion
$w(x)=1$ durch, er führt auf die sogenannten {\em Legendre-Polynome}.

Da wir auf die Normierung verzichten, brauchen wir ein anderes
Kriterium, welches die Polynome eindeutig festlegen kann.
Wir bezeichnen das Polynom vom Grad $n$, das bei diesem Prozess
entsteht, mit $P_n(x)$ und legen willkürlich aber traditionskonform
fest, dass $P_n(1)=1$ sein soll.
Die Polynome $P_n(x)$ bilden also nur eine orthogonale Funktionenfamilie
(Definition~\ref{buch:orthogonal:def:orthogonal}).

%
% Symmetrie-Eigenschaften
%
\subsubsection{Symmetrieeigenschaften}
\index{Symmetrie}%
Das Skalarprodukt berechnet ein Integral eines Produktes von zwei
Polynomen über das symmetrische Interval $[-1,1]$.
Ist die eine gerade und die andere ungerade, dann ist das
Produkt eine ungerade Funktion und das Skalarprodukt verschwindet,
die Funktionen sind bereits orthogonal.
Haben die Funktionen die gleiche Parität, 
sind sie also beide gerade oder beide ungerade, dann ist das Produkt
gerade und das Skalarprodukt ist im Allgemeinen von $0$ verschieden.
Dies zeigt, dass es doch noch etwas zu Orthogonalisieren gibt.

Die ersten beiden Funktionen sind das konstante Polynom $1$ und
das Polynom $x$.
Nach obiger Beobachtung ist das Skalarprodukt $\langle 1,x\rangle=0$,
also ist $P_1(x)=x$.
Die Graphen der entstehenden Polynome sind in
Abbildung~\ref{buch:integral:orthogonal:legendregraphen}
dargestellt.
\begin{figure}
\centering
\includegraphics{chapters/070-orthogonalitaet/images/legendre.pdf}
\caption{Graphen der Legendre-Polynome $P_n(x)$ für $n=1,\dots,10$.
\label{buch:integral:orthogonal:legendregraphen}}
\end{figure}

\begin{lemma}
\label{buch:orthogonal:lemma:symmetrie}
Die Polynome $P_{2n}(x)$ sind gerade, die Polynome $P_{2n+1}(x)$ sind
ungerade Funktionen von $x$.
\end{lemma}

\begin{proof}[Beweis]
Wir verwenden vollständige Induktion nach $n$.
Wir wissen bereits, dass $P_0(x)=1$ und $P_1(x)=x$ die verlangten
Symmetrieeigenschaften haben.
Im Sinne der Induktionsannahme nehmen wir daher an, dass die
Symmetrieeigenschaften für $P_k(x)$, $k<n$, bereits bewiesen sind.
$P_n(x)$ entsteht jetzt durch Orthogonalisierung nach der Formel
\[
P_n(x)
=
x^n
-
\langle P_{n-1},x^n\rangle P_{n-1}(x)
-
\langle P_{n-2},x^n\rangle P_{n-2}(x)
-\dots-
\langle P_1,x^n\rangle P_1(x)
-
\langle P_0,x^n\rangle P_0(x).
\]
Die Skalarprodukte
$\langle P_{n-1},x^n\rangle$,
$\langle P_{n-3},x^n\rangle$, $\dots$ verschwinden alle wegen
unterschiedlicher Parität, so dass
$P_n(x)$ eine Linearkombination der Funktionen $x^n$, $P_{n-2}(x)$,
$P_{n-4}(x)$ ist, die die gleiche Parität wie $x^n$ haben.
Also hat auch $P_n(x)$ die gleiche Parität, was das Lemma beweist.
\end{proof}

%
% Orthogonalisierung nach Gram-Schmidt
%
\subsubsection{Orthogonalisierung mit Gram-Schmidt}
Nach Lemma~\ref{buch:orthogonal:lemma:symmetrie} müssen
bei jedem Orthogonalisierungsschritt jeweils nur Polynome der
gleichen Parität berücksichtigt werden.
Die Orthogonalisierung von $x^2$ liefert daher
\[
p(x) = x^2
-
\frac{\langle x^2,P_0\rangle}{\langle P_0,P_0\rangle} P_0(x)
=
x^2 - \frac{\int_{-1}^1x^2\,dx}{\int_{-1}^11\,dx}
=
x^2 - \frac{\frac{2}{3}}{2}=x^2-\frac13.
\]
Dieses Polynom erfüllt die Standardisierungsbedingung noch 
nicht, denn $p(1)=\frac23\ne 1$.
Daraus leiten wir ab, dass
\[
P_2(x) = \frac12(3x^2-1)
\]
ist.

Für $P_3(x)$ brauchen wir nur die Skalaprodukte
\[
\left.
\begin{aligned}
\langle x^3,P_1\rangle
&=
\int_{-1}^1  x^3\cdot x\,dx
=
\biggl[\frac15x^5\biggr]_{-1}^1
=
\frac25
\qquad
\\
\langle P_1,P_1\rangle
&=
\int_{-1}^1 x^2\,dx
=
\frac23
\end{aligned}
\right\}
\qquad
\Rightarrow
\qquad
p(x) = x^3 - \frac{\;\frac25\;}{\frac23}x=x^3-\frac{3}{5}x.
\]
Die richtige Standardisierung ergibt sich,
indem man durch $p(1)=\frac25$ dividiert, also
\[
P_2(x) = \frac12(5x^3-3x).
\]

Die Berechnung weiterer Polynome verlangt, dass Skalarprodukte
$\langle x^n,P_k\rangle$ berechnet werden müssen, was wegen
der zunehmend komplizierten Form von $P_k$ etwas mühsam ist.
Wir berechnen den Fall $P_4$.
Dazu muss das Polynom $x^4$ um eine Linearkombination von
$P_2$ und $P_0(x)=1$ korrigiert werden.
Die Skalarprodukte sind
\begin{align*}
\langle x^4, P_0\rangle
&=
\int_{-1}^1 x^4\,dx = \frac25
\\
\langle P_0,P_0\rangle
&=
\int_{-1}^1 \,dx = 2
\\
\langle x^4,P_2\rangle
&=
\int_{-1}^1 \frac32x^6-\frac12 x^4\,dx
=
\biggl[\frac{3}{14}x^7-\frac{1}{10}x^5\biggr]_{-1}^1
=
\frac6{14}-\frac15
=
\frac8{35}
\\
\langle P_2,P_2\rangle
&=
\int_{-1}^1 \frac14(3x^2-1)^2\,dx
=
\int_{-1}^1 \frac14(9x^4-6x^2+1)\,dx
=
\frac14\biggl(\frac{18}{5}-4+2\biggr)
=\frac25.
\end{align*}
Daraus folgt für $p(x)$
\begin{align*}
p(x)
&=
x^4
-
\frac{\langle x^4,P_2\rangle}{\langle P_2,P_2\rangle}P_2(x)
-
\frac{\langle x^4,P_0\rangle}{\langle P_0,P_0\rangle}P_0(x)
\\
&=
x^4
-\frac47 P_2(x) - \frac15 P_0(x)
\\
&=
x^4 - \frac{6}{7}x^2 + \frac{3}{35}
\end{align*}
mit $p(1)=\frac{8}{35}$, so dass man
\[
P_4(x) =
\frac18(35x^4-30x^2+3)
\]
setzen muss.

\begin{figure}
\centering
\includegraphics{chapters/070-orthogonalitaet/images/orthogonal.pdf}
\caption{Orthogonalität der Legendre-Polynome $P_4(x)$ ({\color{blue}blau})
und $P_7(x)$ ({\color{darkgreen}grün}).
Die blaue Fläche ist die Fläche unter dem Graphen 
von $P_4(x)^2$, $P_4(x)$ muss durch die Wurzel aus diesem Flächeninhalt
geteilt werden, um ein Polynome mit Norm $1$ zu erhalten.
Für die grüne Fläche ist es $P_7(x)$.
Die rote Kurve ist der Graph der Funktion $P_4(x)\cdot P_7(x)$,
die rote Fläche ist deren Integral, sie ist $0$, d.~h.~die beiden
Funktionen sind orthogonal.
\label{buch:integral:orthogonal:legendreortho}}
\end{figure}

\begin{table}
\centering
\renewcommand{\arraystretch}{1.2}
\begin{tabular}{|>{$}c<{$}|>{$}l<{$}|}
\hline
n&P_n(x)\\
\hline
 0&1
\\
 1&x
\\
 2&\frac12(3x^2-1)
\\
 3&\frac12(5x^3-3x)
\\
 4&\frac18(35x^4-30x^2+3)
\\
 5&\frac18(63x^5-70x^3+15x)
\\
 6&\frac1{16}(231x^6-315x^4+105x^2-5)
\\
 7&\frac1{16}(429x^7-693x^5+315x^3-35x)
\\
 8&\frac1{128}(6435x^8-12012x^6+6930x^4-1260x^2+35)
\\
 9&\frac1{128}(12155x^9-25740x^7+18018x^5-4620x^3+315x)
\\
10&\frac1{256}(46189x^{10}-109395x^8+90090x^6-30030x^4+3465x^2-63)
\\[2pt]
\hline
\end{tabular}
\caption{Die Legendre-Polynome $P_n(x)$ für $n=0,1,\dots,10$ sind
orthogonale Polynome vom Grad $n$, die den Wert $P_n(1)=1$ haben.
\label{buch:integral:table:legendre-polynome}}
\end{table}

Die so konstruierten Polynome heissen die {\em Legendre-Polynome}.
\index{Legendre-Polynome}%
Die Fortführung des Verfahrens liefert die Polynome in
Tabelle~\ref{buch:integral:table:legendre-polynome}.
Die Graphen sind in Abbildung~\ref{buch:integral:orthogonal:legendregraphen}
dargestellt.
Abbildung~\ref{buch:integral:orthogonal:legendreortho} illustriert, 
dass die beiden Polynome $P_4(x)$ und $P_7(x)$ orthogonal sind.
Das Produkt $P_4(x)\cdot P_7(x)$ hat Integral $=0$.

%
% Verschiedene Gewichtsfunktionen
%
\subsection{Gewichtsfunktionen
\label{buch:orthogonal:subsection:gewichtsfunktionen}}
Das Standardskalarprodukt auf dem Raum der Funktionen auf dem
Interval $[-1,1]$ ist das Skalarprodukt mit der Gewichtsfunktion
$w(x)=1$, es führt auf die Legendre-Polynome.
Die Wahl einer anderen Gewichtsfunktion ändert natürlich
das Resultat der Orthogonalisierung.
Nullstellen und Pole der Gewichtsfunktion ändern die Menge der
Funktionen, für die das Skalarprodukt definiert ist.
Diesem Zusammenhang soll im ersten Unterabschnitt nachgegangen werden.
Danach sollen verschiedene für die Praxis relevante Gewichtsfunktionen
vorgestellt werden.

\begin{figure}
\centering
\includegraphics{chapters/070-orthogonalitaet/images/weight.pdf}
\caption{Nullstellen und Pole der Gewichtsfunktion (rot) legen Ort
und Grad von Polen und Nullstellen der Funktionen fest, die beschränkte
$\|\,\cdot\,\|_w$-Norm haben.
An den Stellen $\pm 1$ und $\pm\frac12$ hat die Gewichtsfunktion
Pole bzw.~Nullstellen mit Grad $\alpha$.
Der blaue Bereich deutet an, wie schnell die Funktion $f$ in diesem
Bereich anwachsen kann, bzw.~wie schnell nahe der Polstelle gegen $0$
gehen muss.
\label{buch:orthogonalitaet:fig:gewicht}}
\end{figure}
%
% Pole und Nullstellen der Gewichtsfunktion
%
\subsubsection{Pole und Nullstellen
\label{buch:orthogonal:pole-und-nullstellen}}
Das Skalarprodukt $\langle\,\;,\;\rangle_w$ ist nur sinnvoll
für Funktionen $f(x)$, für die die Norm $\|f\|_w$ definiert ist.
An einer Nullstelle $x_0$ der Gewichtsfunktion $w$ darf die Funktion $f$ 
einen Pol haben. 
Solange $f(x)$ für $x\to x_0$ nicht zu schnell divergiert, kann
das Produkt $|f(x)|^2 w(x)$ immer noch integrierbar sein.

Um dies etwas genauer zu quantifizieren, nehmen wir an, dass
$w(x)$ an der Stelle $x_0$ eine Nullstelle vom Grad $\alpha$ hat.
Dies bedeutet, dass $w(x) \approx C|x-x_0|^\alpha$ ist für eine geeignete
Konstante $C$ und für $|x-x_0|<\varepsilon$.
Ein Pol von $f$ vom Grad $a$ an der Stelle $x_0$ führt entsprechend auf
eine Abschätzung $|f(x)| \approx D|f(x)|^{-a}$ für $|x-x_0|<\varepsilon$.
Dann ist
\[
|f(x)|^2 w(x) \approx CD |x-x_0|^{\alpha-2a}.
\]
Für das Integral in der Nähe von $x_0$ ist
\begin{align*}
\int_{x_0-\varepsilon}^{x_0+\varepsilon}
|f(x)|^2 w(x)\,dx
&\approx 
CD
\int_{x_0-\varepsilon}^{x_0+\varepsilon}
|x-x_0|^{\alpha-2a}\,dx
\\
&=
2CD
\int_0^\varepsilon
t^{\alpha-2a}
\,dt
=
2CD
\begin{cases}
\displaystyle
\;
\biggl[\frac{t^{\alpha-2a+1}}{\alpha-2a+1}\biggr]_0^\varepsilon
&\qquad
\alpha-2a\ne-1
\\[7pt]
\displaystyle
\;
\biggl[ \log t \biggr]_0^\varepsilon
&\qquad
\text{sonst.}
\end{cases}
\end{align*}
Der Zähler $t^{\alpha-2a+1}$ divergiert für $t\to 0$ genau dann,
wenn $\alpha-2a+1<0$ oder $\alpha<2a-1$.
Auch im zweiten Fall, für $\alpha-2a+1=0$, divergiert das Integral.
Damit die Norm $\|f\|_w$ definiert ist, muss also $a<\frac12(\alpha+1)$
sein.

Ganz ähnlich führt eine Polstelle von $w$ vom Grad $\alpha$
an der Stelle $x_0$ dazu, dass $f$ dort eine Nullstelle vom Grad
$a$ haben muss.
Das Normintegral konvergiert nur, wenn $2a-\alpha > -1$ ist
oder $a > \frac12(\alpha+1)$.
 
Pole der Gewichtsfunktion schränken also ein, welche Funktionen
überhaupt der Untersuchung mit Hilfe des Skalarproduktes
$\langle\,\;,\;\rangle_w$ zugänglich sind
(Abbildung~\ref{buch:orthogonalitaet:fig:gewicht}).
Ist die Ordnung $\alpha$ des Poles grösser als $1$, dann müssen die Funktionen
eine Nullstelle mindestens vom Grad $\frac12(a+1)$ haben.
Andererseits erweitern
Nullstellen der Gewichtsfunktion die Klasse der Funktionen.
Ist die Ordnung der Nullstelle $\alpha$, dann dürfen die Funktionen einen
Pol der Ordnung kleiner als $\frac12(\alpha+1)$ haben.

\begin{lemma}
\label{buch:orthogonal:lemma:gewichtsfunktion}
Sei $w(x)\ge 0$ auf dem Intervall $(a,b)$.
Der Vektorraum $H_w$ von auf $(a,b)$ definierten Funktionen sei
\[
H_w
=
\biggl\{
f\colon(a,b) \to \mathbb{R}
\;\bigg|\;
\int_a^b |f(x)|^2 w(x)\,dx
<\infty
\biggr\}
=
L^2([a,b],w).
\]
\index{L2abw@$L^2([a,b],w)$}%
Die Funktionen $f\in H_w$ haben folgende Eigenschaften
\begin{enumerate}
\item
Ist $\xi\in[a,b]$ eine Nullstelle vom Grad $\alpha$ der Funktion $w(x)$,
dann 
\item
Ist $\xi\in[a,b]$ eine Polstelle vom Grad $a$ der Funktion $w(x)$,
dann hat $f$ eine Nullstelle mindestens from Grad 
\end{enumerate}
\end{lemma}


%
% Die Jacobische Gewichtsfunktion
%
\subsubsection{Jacobische Gewichtsfunktion}
Die Gewichtsfunktion für die Legendre-Polynome war $w(x)=1$, alle
Punkte im Intervall $(-1,1)$ hatten das gleiche Gewicht.
Diese soll jetzt ersetzt werden durch eine Gewichtsfunktion, die
den Punkten an den Intervallenden mehr oder weniger Gewicht gibt,
wobei auch zugelassen sein soll, dass die Gewichtung nicht symmetrisch
ist.

\begin{definition}
\label{buch:orthogonal:def:jacobi-gewichtsfunktion}
Die {\em Jacobi-Gewichtsfunktion} ist die Funktion
\index{Jacobi-Gewichtsfunktion}%
\[
w^{(\alpha,\beta)}
\colon (-1,1)\to\mathbb{R}
:
x\mapsto w^{(\alpha,\beta)}(x) = (1-x)^\alpha(1+x)^\beta
\]
mit $\alpha,\beta\in\mathbb{R}$.
\index{walphabeta@$w^{(\alpha,\beta)}$}%
Das Skalarprodukt zugehörige Skalarprodukt wird auch mit
\[
\langle\,\;,\;\rangle_{w^{(\alpha,\beta)}}
=
\langle\,\;,\;\rangle_{(\alpha,\beta)}
\]
\index{<,>alphabeta@$\langle \;\,,\;\rangle_{\alpha,\beta}$}%
bezeichnet und die zugehörige Norm mit
\[
\|f\|_{(\alpha,\beta)}
=
\langle f,f\rangle_{(\alpha,\beta)}
=
\int_{-1}^1 |f(x)|^2 w^{(\alpha,\beta)}(x)\,dx.
\]
\end{definition}

\begin{definition}
\label{buch:orthogonal:def:jacobi-polynome}
Die {\em Jacobi-Polynome} $P^{(\alpha,\beta)}_n(x)$ sind 
\index{Jacobi-Polynome}%
Polynome vom Grad $n$, die bezüglich des Skalarproduktes
$\langle\,\;,\;\rangle_{w^{(\alpha,\beta)}}$ orthogonal sind
und mit
\[
P_n^{(\alpha,\beta)}(1) = \binom{n+\alpha}n
\]
normiert sind.
\index{Palphabeta@$P^{(\alpha,\beta)}_n$}
\end{definition}

In Abbildung~\ref{buch:orthogonal:fig:jacobi-parameter}
ist die Abhängigkeit der Jacobi-Polynome von den Parametern $\alpha$
und $\beta$ illustriert.
Für $\alpha=\beta=0$ entsteht die Gewichtsfunktion
$w^{(0,0)}(x)=1$, die Legendre-Polynome sind also der Spezialfall
$\alpha=\beta=0$ der Jacobi-Polynome.

Der Exponent $\alpha$ in der Gewichtsfunktion $w^{(\alpha,\beta)}(x)$
steuert das Gewicht, welches Punkte am rechten Rand des Intervalls
erhalten.
Für positive Werte von $\alpha$ hat $w^{(\alpha,\beta)}(x)$ eine
Nullstelle vom Grad $\alpha$ an der Stelle $x=1$, nach
Lemma~\ref{buch:orthogonal:lemma:gewichtsfunktion}
dürfen die Funktionen einen Pole der Ordnung $<\frac12(\alpha-1)$ haben.
Je grösser $\alpha$ ist, desto weniger Gewicht haben die Punkte
am rechten Rand des Intervalls und desto schneller darf eine Funktion
für $x\to 1$ divergieren.

Für negative Werte von $\alpha$ hat $w^{(\alpha,\beta)}(x)$ einen
Pol vom Grad $-\alpha$ an der Stelle $x=1$.
Funktionen müssen daher also ein Nullstelle mindestens vom Grad
$\frac12(1-\alpha)$ haben.

\begin{figure}
\centering
\includegraphics{chapters/070-orthogonalitaet/images/lintrans.pdf}
\caption{Variablentransformation vom Intervall $[0,1]$ auf das
Intervall $[-1,1]$ zwecks Vergleich der Beta-Verteilung, die auf $[0,1]$
definiert ist, mit den Jacobi-Polynomen, die auf $[-1,1]$ definiert
sind.
\label{buch:orthogonal:fig:lintrans}}
\end{figure}
\begin{figure}
\centering
\includegraphics[width=\textwidth]{chapters/070-orthogonalitaet/images/jacobi.pdf}
\caption{Jacobi-Polynome vom Grad $1$ bis $14$ für verschiedene Werte
der Parameter $\alpha$ und $\beta$.
Je grösser $\alpha$, desto weniger Gewicht bekommen die Funktionswerte am
rechten Rand und desto grösser werden die Funktionswerte.
Für negative $\alpha$ müssen die Polynome dagegen eine Nullstelle am
rechten Rand haben.
\label{buch:orthogonal:fig:jacobi-parameter}}
\end{figure}

%
% Jacobi-Gewichtsfunktion und Beta-Verteilung
%
\subsubsection{Jacobi-Gewichtsfunktion und Beta-Verteilung
\label{buch:orthogonal:subsection:beta-verteilung}}
Die Jacobi-Gewichtsfunktion entsteht aus der Wahrscheinlichkeitsdichte
der Beta-Verteilung, die in
Abschnitt~\ref{buch:rekursion:subsection:beta-verteilung}
eingeführt wurde, mit Hilfe der linearen Variablentransformation $x = 2t-1$
oder $t=(x+1)/2$ (Abbildung~\ref{buch:orthogonal:fig:lintrans}).
\index{Beta-Verteilung}%
Das Integral mit der Jacobi-Gewichtsfunktion $w^{(\alpha,\beta)}(x)$
kann damit umgeformt werden in
\begin{align*}
\int_{-1}^1
f(x)\,w^{(\alpha,\beta)}(x)\,dx
&=
\int_0^1
f(2t-1) w^{(\alpha,\beta)}(2t-1)\,2\,dt
\\
&=
\int_0^1
f(2t-1)
(1-(2t-1))^\alpha (1+(2t-1))^\beta
\,2\,dt
\\
&=
2^{\alpha+\beta+1}
\int_0^1
f(2t-1)
\,
t^\beta
(1-t)^\alpha
\,dt
\\
&=
2^{\alpha+\beta+1}
B(\alpha+1,\beta+1)
\int_0^1
f(2t-1)
\,
\frac{
t^\beta
(1-t)^\alpha
}{B(\alpha+1,\beta+1)}
\,dt.
\end{align*}
Auf der letzten Zeile steht ein Integral mit der Wahrscheinlichkeitsdichte
der Beta-Verteilung.
Orthogonale Funktionen bezüglich der Jacobischen Gewichtsfunktion
$w^{(\alpha,\beta)}$ werden mit der genannten Substitution also
zu orthogonalen Funktionen bezüglich der Beta-Verteilung mit
Parametern $\beta+1$ und $\alpha+1$.


%
% Tschebyscheff-Gewichtsfunktion
%
\subsubsection{Tschebyscheff-Gewichtsfunktion}
Es wird später gezeigt werden, dass die Tschebyscheff-Polynome
von Abschnitt~\ref{buch:polynome:section:tschebyscheff} eine
Familie orthogonaler Polynome sein.
Das zugehörige Skalarprodukt hat die Gewichtsfunktion
\[
w_{\text{Tschebyscheff}}(x)
=
\frac{1}{\sqrt{1-x^2}}
=
\frac{1}{\sqrt{(1-x)(1+x)}}
=
(1-x)^{-\frac{1}{2}}
(1+x)^{-\frac{1}{2}}
=
w^{(-\frac12,-\frac12)}(x).
\]
Die {\em Tschebyscheff-Gewichtsfunktion} ist also ein Spezialfall der
Jacobi-Gewichtsfunktion.
\index{Tschebyscheff-Gewichtsfunktion}%

%
% Hermite-Gewichtsfunktion
%
\subsubsection{Hermite-Gewichtsfunktion}
\index{ex2@$e^{-\frac{x^2}2}$}%
Die Gewichtsfunktion
\[
w_{\text{Hermite}}(x)
=
w(x)
=
e^{-\frac{x^2}{2}}
\]
heisst die {\em Hermite-Gewichtsfunktion}.
\index{Hermite-Gewichtsfunktion}%
Sie hat keine Nullstellen und geht für $x\to\pm\infty$ so schnell
gegen $0$, dass für alle Polynome $p(x)$
\[
\int_{-\infty}^\infty |p(x)|^2 e^{-\frac{x^2}{2}}\,dx<\infty
\]
ist.
Als Definitionsintervall kann daher die ganze reelle Achse
verwendet werden, also $a=-\infty$ und $b=\infty$.
Die mit dieser Gewichtsfunktion konstruierten Polynome heissen
bei geeigneter Normierung die {\em Hermite-Polynome}.
% XXX Normierung der Hermite-Polynome festlegen
\index{Hermite-Polynome}%

%
% Laguerre-Gewichtsfunktion
%
\subsubsection{Laguerre-Gewichtsfunktion}
Ähnlich wie die Hermite-Gewichtsfunktion ist die
{\em Laguerre-Gewichtsfunktion}
\index{Laguerre-Gewichtsfunktion}%
\[
w_{\text{Laguerre}}(x)
=
e^{-x}
\]
\index{e-x@$e^{-x}$}%
auf ganz $\mathbb{R}$ definiert, und sie geht für $x\to\infty$ wieder
sehr rasch gegen $0$.
Für $x\to-\infty$ hingegen wächst sie so schnell an, dass für alle Polynome
$p(x)$ das Integral
\[
\int_{-\infty}^\infty p(x)e^{-x}\,dx
\]
unbeschränkt ist.
Die Laguerre-Gewichtsfunktion ist daher nur geeignet für den
Definitionsbereich $(0,\infty)$.
Die bezüglich der Laguerre-Gewichtsfunktion orthogonalen Polynome
heissen bei geeigneter Normierung die {\em Laguerre-Polynome}.
\index{Laguerre-Polynome}%

\input{chapters/070-orthogonalitaet/2-rekursion.tex}
\input{chapters/070-orthogonalitaet/3-rodrigues.tex}
\input{chapters/070-orthogonalitaet/4-legendredgl.tex}
%
% Bessel-Funktionen also orthogonale Funktionenfamilie
%
\section{Bessel-Funktionen als orthogonale Funktionenfamilie
\label{buch:orthogonalitaet:section:bessel}}
\rhead{Bessel-Funktionen}
Auch die Bessel-Funktionen sind eine orthogonale Funktionenfamilie.
Sie sind differenzierbaren Funktionen $f(r)$, die definiert
sind für $r>0$, mit $f'(r)=0$ und derart, dass für $r\to\infty$ $f(r)$
so schnell abnimmt, dass auch $rf(r)$ noch gegen $0$ strebt.
Das Skalarprodukt ist
\[
\langle f,g\rangle
=
\langle f,g\rangle_r
=
\int_0^\infty f(r) g(r)\,r\,dr,
\]
ein verallgemeinertes Skalarprodukt mit Gewichtsfunktion $w(r)=r$.
Als Operator verwenden wir
\[
A = \frac{d^2}{dr^2} + \frac{1}{r}\frac{d}{dr} + s(r),
\]
wobei $s(r)$ eine beliebige integrierbare Funktion sein kann.
Zunächst überprüfen wir, ob dieser Operator wirklich selbstadjungiert ist.
Dazu rechnen wir
\begin{align}
\langle Af,g\rangle
&=
\int_0^\infty
r\,\biggl(f''(r)+\frac1rf'(r)+s(r)f(r)\biggr) g(r)
\,dr
\notag
\\
&=
\int_0^\infty rf''(r)g(r)\,dr
+
\int_0^\infty f'(r)g(r)\,dr
+
\int_0^\infty s(r)f(r)g(r)\,dr.
\notag
\intertext{Der letzte Term ist symmetrisch in $f$ und $g$, daher
ändern wir daran weiter nichts.
Auf das erste Integral kann man partielle Integration anwenden und erhält}
&=
\biggl[rf'(r)g(r)\biggr]_0^\infty
-
\int_0^\infty f'(r)g(r) + rf'(r)g'(r)\,dr
+
\int_0^\infty f'(r)g(r)\,dr
+
\int_0^\infty s(r)f(r)g(r)\,dr.
\notag
\intertext{Der erste Term verschwindet wegen der Bedingungen an die
Funktionen $f$ und $g$.
Der erste Term im zweiten Integral hebt sich gegen das
zweite Integral weg.
Der letzte Term ist das Skalarprodukt von $f'$ und $g'$.
Somit ergibt sich
}
&=
-\langle f',g'\rangle
+
\int_0^\infty s(r) f(r)g(r)\,dr.
\label{buch:integrale:orthogonal:besselsa}
\end{align}
Vertauscht man die Rollen von $f$ und $g$, erhält man das Gleiche, da im
letzten Ausdruck~\eqref{buch:integrale:orthogonal:besselsa} die Funktionen
$f$ und $g$ symmetrisch auftreten.
Damit ist gezeigt, dass der Operator $A$ selbstadjungiert ist.
Es folgt nun, dass Eigenvektoren des Operators $A$ automatisch
orthogonal sind.

Eigenfunktionen von $A$ sind aber Lösungen der Differentialgleichung
\[
\begin{aligned}
&&
Af&=\lambda f
\\
&\Rightarrow\qquad&
f''(r) +\frac1rf'(r) + s(r)f(r) &= \lambda f(r)
\\
&\Rightarrow\qquad&
r^2f''(r) +rf'(r)+ (-\lambda r^2+s(r)r^2)f(r) &= 0
\end{aligned}
\]
sind.

Durch die Wahl $s(r)=1$ wird der Operator $A$ zum Bessel-Operator
$B$, definiert in
\eqref{buch:differentialgleichungen:bessel-operator}.
Die Lösungen der Besselschen Differentialgleichung zu verschiedenen Werten
des Parameters müssen also orthogonal sein, insbesondere sind die
Bessel-Funktion $J_\nu(r)$ und $J_\mu(r)$ orthogonal wenn $\mu\ne\nu$ ist.


\input{chapters/070-orthogonalitaet/6-sturm.tex}
\input{chapters/070-orthogonalitaet/7-gaussquadratur.tex}

\section*{Übungsaufgabe}
\rhead{Übungsaufgaben}
\aufgabetoplevel{chapters/070-orthogonalitaet/uebungsaufgaben}
\begin{uebungsaufgaben}
\uebungsaufgabe{701}
%\uebungsaufgabe{1}
\end{uebungsaufgaben}

