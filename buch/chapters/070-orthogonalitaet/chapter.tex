%
% chapter.tex -- Spezielle Funktionen definiert durch Integrale
%
% (c) 2021 Prof Dr Andreas Müller, Hochschule Rapperswil
%
% !TeX spellcheck = de_CH
\chapter{Orthogonalität
\label{buch:chapter:orthogonalitaet}}
\lhead{Orthogonalität}
\rhead{}
In der linearen Algebra lernt man, dass orthonormierte Basen für die
Lösung vektorgeometrischer Probleme, bei denen auch das Skalarprodukt
involviert ist, besonders günstig sind.
Die Zerlegung eines Vektors in einer Basis verlangt normalerweise nach
der Lösung eines linearen Gleichungssystems, für orthonormierte
Basisvektoren beschränkt sie sich auf die Berechnung von Skalarprodukten.

Oft dienen spezielle Funktionen als Basis der Lösungen einer linearen
partiellen Differentialgleichung (siehe Kapitel~\ref{buch:chapter:pde}).
Die Randbedingungen müssen dazu in der gewählten Basis von Funktionen
zerlegt werden.
Fourier ist es gelungen, die Idee des Skalarproduktes und der Orthogonalität
auf Funktionen zu verallgemeinern und so zum Beispiel das Wärmeleitungsproblem
zu lösen.

Der Orthonormalisierungsprozess von Gram-Schmidt wird damit auch auf
Funktionen anwendbar
(Abschnitt~\ref{buch:orthogonalitaet:section:orthogonale-funktionen}),
der Nutzen führt aber noch viel weiter.
Da $K[x]$ ein Vektorraum ist, führt er von der Basis der Monome
$\{1,x,x^2,\dots,x^n\}$ 
auf orthonormierte Polynome.
Diese haben jedoch eine ganze Reihe weiterer nützlicher Eigenschaften.
So wird in Abschnitt~\ref{buch:orthogonal:section:drei-term-rekursion}
gezeigt, dass sich die Werte aller Polynome einer solchen Familie mit
einer Rekursionsformel effizient berechnen lassen, die höchstens drei
Terme umfasst.
In Abschnitt~\ref{buch:orthogonalitaet:section:rodrigues} werden
die Rodrigues-Formeln vorgeführt, die Polynome durch Anwendung eines
Differentialoperators hervorbringen.
In Abschnitt~\ref{buch:orthogonal:section:orthogonale-polynome-und-dgl}
schliesslich wird gezeigt, dass diese Polynome auch Eigenfunktionen
eines selbstadjungierten Operators sind.
Da man in der linearen Algebra auch lernt, dass die Eigenvektoren einer
symmetrischen Matrix zu verschiedenen Eigenwerten orthogonal sind,
ist die Orthogonalität plötzlich nicht mehr überraschend.

Die Bessel-Funktionen von
Abschnitt~\ref{buch:differntialgleichungen:section:bessel}
sind auch Eigenfunktionen eines Differentialoperators.
Abschnitt~\ref{buch:orthogonalitaet:section:bessel} findet das zugehörige
Skalarprodukt, welches andeutet, dass auch für andere Funktionenfamilien
eine entsprechende Konstruktion möglich ist.
Das in Abschnitt~\ref{buch:integrale:subsection:sturm-liouville-problem}
präsentierte Sturm-Liouville-Problem führt sie durch.
Das Kapitel schliesst mit dem
Abschnitt~\ref{buch:orthogonal:section:gauss-quadratur}
über die Gauss-Quadratur, welche die Eigenschaften orthogonaler Polynome
für einen besonders effizienten numerischen Integrationsalgorithmus
ausnutzt.

%
% orthogonal.tex
%
% (c) 2021 Prof Dr Andreas Müller, OST Ostschweizer Fachhochschule
%
\section{Orthogonale Funktionenfamilien
\label{buch:orthogonalitaet:section:orthogonale-funktionen}}
\rhead{Orthogonale Funktionenfamilien}
Die Fourier-Theorie basiert auf der Idee, Funktionen durch 
Funktionenreihen mit Summanden zu bilden, die im Sinne eines
Skalarproduktes orthogonal sind, welches mit Hilfe eines Integrals
definiert sind.
Solche Funktionenfamilien treten jedoch auch als Lösungen von
Differentialgleichungen.
Besonders interessant wird die Situation, wenn die Funktionen 
Polynome sind.

%
% Skalarprodukt
%
\subsection{Skalarprodukt}
Der reelle Vektorraum $\mathbb{R}^n$ trägt das Skalarprodukt
\[
\langle\;\,,\;\rangle
\colon
\mathbb{R}^n \times \mathbb{R}^n \to \mathbb{R}
:
(x,y)\mapsto \langle x, y\rangle = \sum_{k=1}^n x_iy_k,
\]
welches viele interessante Anwendungen ermöglicht.
Eine orthonormierte Basis macht es zum Beispiel besonders leicht,
eine Zerlegung eines Vektors in dieser Basis zu finden.
In diesem Abschnitt soll zunächst an die Eigenschaften erinnert
werden, die zu einem nützlichen 

\subsubsection{Eigenschaften eines Skalarproduktes}
Das Skalarprodukt erlaubt auch, die Länge eines Vektors $v$
als $|v| = \sqrt{\langle v,v\rangle}$ zu definieren.
Dies funktioniert natürlich nur, wenn die Wurzel auch immer
definiert ist, d.~h.~das Skalarprodukt eines Vektors mit sich
selbst darf nicht negativ sein.
Dazu dient die folgende Definition.

\begin{definition}
Sei $V$ ein reeller Vektorraum.
Eine bilineare Abbildung
\[
\langle\;\,,\;\rangle
\colon
V\times V
\to
\mathbb{R}
:
(u,v) \mapsto \langle u,v\rangle.
\]
heisst {\em positiv definit}, wenn für alle Vektoren $v \in V$ mit
$v\ne 0 \Rightarrow \langle v,v\rangle > 0$ 
Die {\em Norm} eines Vektors $v$ ist
$|v|=\sqrt{\langle v,v\rangle}$.
\end{definition}

Damit man mit dem Skalarprodukt sinnvoll rechnen kann, ist ausserdem
erforderlich, dass es eine einfache Beziehung zwischen 
$\langle x,y\rangle$ und $\langle y,x\rangle$ gibt.

\begin{definition}
Ein {\em Skalarprodukt} auf einem reellen Vektorraum $V$ ist eine
positiv definite, symmetrische bilineare Abbildung
\[
\langle\;\,,\;\rangle
\colon
V\times V
\to
\mathbb{R}
:
(u,v) \mapsto \langle u,v\rangle.
\]
\end{definition}

Das Skalarprodukt $\langle u,v\rangle=u^tv$ auf dem Vektorraum 
$\mathbb{R}^n$ erfüllt die Definition ganz offensichtlich,
sie führt auf die Komponentendarstellung
\[
\langle u,v\rangle = u^tv = \sum_{k=1}^n u_iv_i.
\]
Weitere Skalarprodukte ergeben ergeben sich mit jeder symmetrischen,
positiv definiten Matrix $G$ und der Definition
$\langle u,v\rangle_G=u^tGv$.
Ein einfacher Spezialfall tritt auf, wenn $G$ eine Diagonalmatrix
$\operatorname{diag}(w_1,\dots,w_n)$
mit positiven Einträgen $w_i>0$ auf der Diagonalen ist.
In diesem Fall schreiben wir
\[
\langle u,v\rangle_w
=
u^t\operatorname{diag}(w_1,\dots,w_n)v
=
\sum_{k=1}^n u_iv_i\,w_i
\]
und nennen $\langle \;\,,\;\rangle_w$ das {\em gewichtete Skalarprodukt}
mit {\em Gewichten $w_i$}.

\subsubsection{Skalarprodukte auf Funktionenräumen}
Das Integral ermöglicht jetzt, ein Skalarprodukt auf dem reellen
Vektorraum der stetigen Funktionen auf einem Intervall zu definieren.

\begin{definition}
\label{buch:orthogonal:def:skalarprodukt}
Sei $V$ der reelle Vektorraum $C([a,b])$ der reellwertigen, stetigen
Funktion auf dem Intervall $[a,b]$.
Dann ist 
\[
\langle\;\,,\;\rangle
\colon
C([a,b]) \times C([a,b]) \to \mathbb{R}
:
(f,g) \mapsto \langle f,g\rangle = \int_a^b f(x)g(x)\,dx.
\]
ein Skalarprodukt.
\end{definition}

Die Definition ist offensichtlich symmetrisch in $f$ und $g$ und
aus den Eigenschaften des Integrals ist klar, dass das Produkt
bilinear ist:
\begin{align*}
\langle \lambda_1 f_1+\lambda_2f_2,g\rangle
&=
\int_a^b (\lambda_1f_(x) +\lambda_2f_2(x))g(x)\,dx
=
\lambda_1\int_a^b f_1(x) g(x)\,dx
+
\lambda_2\int_a^b f_2(x) g(x)\,dx
\\
&=
\lambda_1\langle f_1,g\rangle
+
\lambda_2\langle f_2,g\rangle.
\end{align*}
Ausserdem ist es positiv definit, denn wenn $f(x_0) \ne 0$ ist,
dann gibt es wegen der Stetigkeit von $f$ eine Umgebung
$U=[x_0-\varepsilon,x_0+\varepsilon]$, derart, dass $|f(x)| > \frac12|f(x_0)|$
ist für alle $x\in U$.
Somit ist das Integral
\[
\langle f,f\rangle
=
\int_a^b |f(x)|^2\,dx
\ge
\int_{x_0-\varepsilon}^{x_0+\varepsilon} |f(x)|^2\,dx
\ge
\int_{x_0-\varepsilon}^{x_0+\varepsilon} \frac14|f(x_0)|^2\,dx
=
\frac{1}{4}|f(x_0)|^2\cdot 2\varepsilon
=
\frac{|f(x_0)|^2\varepsilon}{2}
>0,
\]
was beweist, dass $\langle\;,\;\rangle$ positiv definit und damit
ein Skalarprodukt ist.

Die Definition kann noch etwas verallgemeinert werden, indem 
die Funktionswerte nicht überall auf dem Definitionsbereich 
gleich gewichtet werden. 

\begin{definition}
\label{buch:orthogonal:def:skalarproduktw}
Sei $w\colon [a,b]\to \mathbb{R}^+$ eine positive, stetige Funktion,
dann ist
\[
\langle\;\,,\;\rangle_w
\colon
C([a,b]) \times C([a,b]) \to \mathbb{R}
:
(f,g) \mapsto \langle f,g\rangle_w = \int_a^b f(x)g(x)\,w(x)\,dx.
\]
das {\em gewichtete Skalarprodukt} mit {\em Gewichtsfunktion $w(x)$}.
\end{definition}

\subsubsection{Gram-Schmidt-Orthonormalisierung}
In einem reellen Vektorraum $V$ mit Skalarprodukt $\langle\;\,,\;\rangle$
kann aus einer beleibigen Basis $b_1,\dots,b_n$ mit Hilfe des 
Gram-Schmidtschen Orthogonalisierungsverfahrens immer eine
orthonormierte Basis $\tilde{b}_1,\dots,\tilde{b}_n$ Basis
gewonnen werden.
Es stellt sicher, dass für alle $k\le n$ gilt
\[
\langle b_1,\dots,b_k\rangle
=
\langle \tilde{b}_1,\dots,\tilde{b}_k\rangle.
\]
Zur Vereinfachung der Formeln schreiben wir $v^0=v/|v|$ für einen zu
$v$ parallelen Einheitsvektor.
Die Vektoren $\tilde{b}_i$ können mit Hilfe der Formeln
\begin{align*}
\tilde{b}_1
&=
(b_1)^0
\\
\tilde{b}_2
&=
\bigl(
b_2
-
\langle \tilde{b}_1,b_2\rangle \tilde{b}_1
\bigr)^0
\\
\tilde{b}_3
&=
\bigl(
b_3
-
\langle \tilde{b}_1,b_3\rangle \tilde{b}_1
-
\langle \tilde{b}_2,b_3\rangle \tilde{b}_2
\bigr)^0
\\
&\;\vdots
\\
\tilde{b}_n
&=
\bigl(
b_n
-
\langle \tilde{b}_1,b_n\rangle \tilde{b}_1
-
\langle \tilde{b}_2,b_n\rangle \tilde{b}_2
-\dots
-
\langle \tilde{b}_{n-1},b_n\rangle \tilde{b}_{n-1}
\bigr)^0
\end{align*}
iterativ berechnet werden.
Dieses Verfahren lässt sich auch auf Funktionenräume anwenden.

Die Normierung ist nicht unbedingt nötig und manchmal unangenehm,
da die Norm unschöne Quadratwurzeln einführt.
Falls es genügt, eine orthogonale Basis zu finden, kann darauf
verzichtet werden, bei der Orthogonalisierung muss aber berücksichtigt
werden, dass die Vektoren $\tilde{b}_i$ jetzt nicht mehr Einheitslänge
haben.
Die Formeln
\begin{align*}
\tilde{b}_0
&=
b_0
\\
\tilde{b}_1
&=
b_1
-
\frac{\langle b_1,\tilde{b}_0\rangle}{\langle \tilde{b}_0,\tilde{b}_0\rangle}\tilde{b}_0
\\
\tilde{b}_2
&=
b_2
-
\frac{\langle b_2,\tilde{b}_0\rangle}{\langle \tilde{b}_0,\tilde{b}_0\rangle}\tilde{b}_0
-
\frac{\langle b_2,\tilde{b}_1\rangle}{\langle \tilde{b}_1,\tilde{b}_1\rangle}\tilde{b}_1
\\
&\;\vdots
\\
\tilde{b}_n
&=
b_n
-
\frac{\langle b_n,\tilde{b}_0\rangle}{\langle \tilde{b}_0,\tilde{b}_0\rangle}\tilde{b}_0
-
\frac{\langle b_n,\tilde{b}_1\rangle}{\langle \tilde{b}_1,\tilde{b}_1\rangle}\tilde{b}_1
-
\dots
-
\frac{\langle b_n,\tilde{b}_{n-1}\rangle}{\langle \tilde{b}_{n-1},\tilde{b}_{n-1}\rangle}\tilde{b}_{n-1}.
\end{align*}
berücksichtigen dies.

\subsubsection{Selbstadjungierte Operatoren und Eigenvektoren}
Symmetrische Matrizen spielen eine spezielle Rolle in der
endlichdimensionalen linearen Algebra, weil sie sich immer
mit einer orthonormierten Basis diagonalisieren lassen.
In der vorliegenden Situation undendlichdimensionaler Vektorräume
brauchen wir eine angepasste Definition.

\begin{definition}
Eine lineare Selbstabbildung $A\colon V\to V$
eines Vektorrraums mit Skalarprodukt
heisst {\em selbstadjungiert}, wenn für alle Vektoren $u,v\in V$
heisst $\langle Au,v\rangle = \langle u,Av\rangle$.
\end{definition}

Es ist wohlbekannt, dass Eigenvektoren einer symmetrischen Matrix
zu verschiedenen Eigenwerten orthogonal sind.
Der Beweis ist direkt übertragbar, wir halten das Resultat hier
für spätere Verwendung fest.

\begin{satz}
Sind $f$ und $g$ Eigenvektoren eines selbstadjungierten Operators $A$
zu verschiedenen Eigenwerten $\lambda$ und $\mu$, dann sind $f$ und $g$
orthogonal.
\end{satz}

\begin{proof}[Beweis]
Im vorliegenden Zusammenhang möchten wir die Eigenschaft nutzen,
dass Eigenfunktionen eines selbstadjungierten Operatores zu verschiedenen
Eigenwerten orthogonal sind.
Dazu seien $Df = \lambda f$ und $Dg=\mu g$ und wir rechnen
\begin{equation*}
\renewcommand{\arraycolsep}{2pt}
\begin{array}{rcccrl}
\langle Df,g\rangle &=& \langle \lambda f,g\rangle &=& \lambda\phantom{)}\langle f,g\rangle
&\multirow{2}{*}{\hspace{3pt}$\biggl\}\mathstrut-\mathstrut$}\\
=\langle f,Dg\rangle &=& \langle f,\mu g\rangle &=& \mu\phantom{)}\langle f,g\rangle&
\\[2pt]
\hline
         0           & &                        &=& (\lambda-\mu)\langle f,g\rangle&
\end{array}
\end{equation*}
Da $\lambda-\mu\ne 0$ ist, muss $\langle f,g\rangle=0$ sein.
\end{proof}

\begin{beispiel}
Sei $C^1([0,2\pi], \mathbb{C})=C^1(S^1,\mathbb{C})$
der Vektorraum der $2\pi$-periodischen differenzierbaren Funktionen mit
dem Skalarprodukt 
\[
\langle f,g\rangle
=
\frac{1}{2\pi}\int_0^{2\pi} \overline{f(t)}g(t)\,dt
\]
enthält die Funktionen $e_n(t) = e^{int}$.
Der Operator
\[
D=i\frac{d}{dt}
\]
ist selbstadjungiert, denn mit Hilfe von partieller Integration erhält man
\[
\langle Df,g\rangle
=
\frac{1}{2\pi}
\int_0^{2\pi}
\underbrace{
\overline{i\frac{df(t)}{dt}}
}_{\uparrow}
\underbrace{g(t)}_{\downarrow}
\,dt
=
\underbrace{
\frac{-i}{2\pi}
\biggl[
\overline{f(t)}g(t)
\biggr]_0^{2\pi}
}_{\displaystyle=0}
+
\frac{1}{2\pi}
\int_0^{2\pi}
\overline{f(t)}i\frac{dg(t)}{dt}
\,dt
=
\langle f,Dg\rangle
\]
unter Ausnützung der $2\pi$-Periodizität der Funktionen.

Die Funktionen $e_n(t)$ sind Eigenfunktionen des Operators $D$, denn
\[
De_n(t) = i\frac{d}{dt}e^{int} = -n e^{int} = -n e_n(t).
\]
Nach obigem Satz sind die Eigenfunktionen von $D$ orthogonal.
\end{beispiel}

Das Beispiel illustriert, dass orthogonale Funktionenfamilien
ein automatisches Nebenprodukt selbstadjungierter Operatoren sind.


% XXX Orthogonalisierungsproblem so formulieren, dass klar wird,
% XXX dass man ein "Normierungskriterium braucht.

Da wir auf die Normierung verzichten, brauchen wir ein anderes
Kriterium, welches die Polynome eindeutig festlegen kann.
Wir bezeichnen das Polynom vom Grad $n$, das bei diesem Prozess
entsteht, mit $P_n(x)$ und legen willkürlich aber traditionskonform
fest, dass $P_n(1)=1$ sein soll.

Das Skalarprodukt berechnet ein Integral eines Produktes von zwei
Polynomen über das symmetrische Interval $[-1,1]$.
Ist die eine gerade und die andere ungerade, dann ist das
Produkt eine ungerade Funktion und das Skalarprodukt verschwindet.
Sind beide Funktionen gerade oder ungerade, dann ist das Produkt
gerade und das Skalarprodukt ist im Allgmeinen von $0$ verschieden.
Dies zeigt, dass es tatsächlich etwas zu Orthogonalisieren gibt.

Die ersten beiden Funktionen sind das konstante Polynom $1$ und
das Polynome $x$.
Nach obiger Beobachtung ist das Skalarprodukt $\langle 1,x\rangle=0$,
also ist $P_1(x)=x$.
Die Graphen der entstehenden Polynome sind in
Abbildung~\ref{buch:integral:orthogonal:legendregraphen}
dargestellt.
\begin{figure}
\centering
\includegraphics{chapters/070-orthogonalitaet/images/legendre.pdf}
\caption{Graphen der Legendre-Polynome $P_n(x)$ für $n=1,\dots,10$.
\label{buch:integral:orthogonal:legendregraphen}}
\end{figure}

\begin{lemma}
Die Polynome $P_{2n}(x)$ sind gerade, die Polynome $P_{2n+1}(x)$ sind
ungerade Funktionen von $x$.
\end{lemma}

\begin{proof}[Beweis]
Wir verwenden vollständige Induktion nach $n$.
Wir wissen bereits, dass $P_0(x)=1$ und $P_1(x)=x$ die verlangten
Symmetrieeigenschaften haben.
Im Sinne der Induktionsannahme nehmen wir daher an, dass die
Symmetrieeigenschaften für $P_k(x)$, $k<n$, bereits bewiesen sind.
$P_n(x)$ entsteht jetzt durch Orthogonalisierung nach der Formel
\[
P_n(x)
=
x^n
-
\langle P_{n-1},x^n\rangle P_{n-1}(x)
-
\langle P_{n-2},x^n\rangle P_{n-2}(x)
-\dots-
\langle P_1,x^n\rangle P_1(x)
-
\langle P_0,x^n\rangle P_0(x).
\]
Die Skalarprodukte
$\langle P_{n-1},x^n\rangle$,
$\langle P_{n-3},x^n\rangle$, $\dots$ verschwinden alle, so dass
$P_n(x)$ eine Linearkombination der Funktionen $x^n$, $P_{n-2}(x)$,
$P_{n-4}(x)$ ist, die die gleiche Parität wie $x^n$ haben.
Also hat auch $P_n(x)$ die gleiche Parität, was das Lemma beweist.
\end{proof}

Die Ortogonalisierung von $x^2$ liefert daher
\[
p(x) = x^2
-
\frac{\langle x^2,P_0\rangle}{\langle P_0,P_0\rangle} P_0(x)
=
x^2 - \frac{\int_{-1}^1x^2\,dx}{\int_{-1}^11\,dx}
=
x^2 - \frac{\frac{2}{3}}{2}=x^2-\frac13
\]
Dieses Polynom erfüllt die Standardisierungsbedingung noch 
nicht den $p(1)=\frac23$.
Daraus leiten wir ab, dass
\[
P_2(x) = \frac12(3x^2-1)
\]
ist.

Für $P_3(x)$ brauchen wir nur die Skalaprodukte
\[
\left.
\begin{aligned}
\langle x^3,P_1\rangle
&=
\int_{-1}^1  x^3\cdot x\,dx
=
\biggl[\frac15x^5\biggr]_{-1}^1
=
\frac25
\qquad
\\
\langle P_1,P_1\rangle
&=
\int_{-1}^1 x^2\,dx
=
\frac23
\end{aligned}
\right\}
\qquad
\Rightarrow
\qquad
p(x) = x^3 - \frac{\frac25}{\frac23}x=x^3-\frac{3}{5}x
\]
Die richtige Standardisierung ergibt sich,
indem man durch $p(1)=\frac25$ dividiert, also
\[
P_2(x) = \frac12(5x^3-3x).
\]

Die Berechnung weiterer Polynome verlangt, dass Skalarprodukte
$\langle x^n,P_k\rangle$ berechnet werden müssen, was wegen
der zunehmend komplizierten Form von $P_k$ etwas mühsam ist.
Wir berechnen den Fall $P_4$.
Dazu muss das Polynom $x^4$ um eine Linearkombination von
$P_2$ und $P_0(x)=1$ korrigiert werden.
Die Skalarprodukte sind
\begin{align*}
\langle x^4, P_0\rangle
&=
\int_{-1}^1 x^4\,dx = \frac25
\\
\langle P_0,P_0\rangle
&=
\int_{-1}^1 \,dx = 2
\\
\langle x^4,P_2\rangle
&=
\int_{-1}^1 \frac32x^6-\frac12 x^4\,dx
=
\biggl[\frac{3}{14}x^7-\frac{1}{10}x^5\biggr]_{-1}^1
=
\frac6{14}-\frac15
=
\frac8{35}
\\
\langle P_2,P_2\rangle
&=
\int_{-1}^1 \frac14(3x^2-1)^2\,dx
=
\int_{-1}^1 \frac14(9x^4-6x^2+1)\,dx
=
\frac14(\frac{18}{5}-4+2)
=\frac25.
\end{align*}
Daraus folgt für $p(x)$
\begin{align*}
p(x)
&=
x^4
-
\frac{\langle x^4,P_2\rangle}{\langle P_2,P_2\rangle}P_2(x)
-
\frac{\langle x^4,P_0\rangle}{\langle P_0,P_0\rangle}P_0(x)
\\
&=
x^4
-\frac47 P_2(x) - \frac15 P_0(x)
\\
&=
x^4 - \frac{6}{7}x^2 + \frac{3}{35}
\end{align*}
mit $p(1)=\frac{8}{35}$, so dass man
\[
P_4(x) =
\frac18(35x^4-30x^2+3)
\]
setzen muss.

\begin{figure}
\centering
\includegraphics{chapters/070-orthogonalitaet/images/orthogonal.pdf}
\caption{Orthogonalität der Legendre-Polynome $P_4(x)$ ({\color{blue}blau})
und $P_7(x)$ ({\color{darkgreen}grün}).
Die blaue Fläche ist die Fläche unter dem Graphen 
von $P_4(x)^2$, $P_4(x)$ muss durch die Wurzel aus diesem Flächeninhalt
geteilt werden, um ein Polynome mit Norm $1$ zu erhalten.
Für die grüne Fläche ist es $P_7(x)$.
Die rote Kurve ist der Graph der Funktion $P_4(x)\cdot P_7(x)$,
die rote Fläche ist deren Integral, sie ist $0$, d.~h.~die beiden
Funktionen sind orthogonal.
\label{buch:integral:orthogonal:legendreortho}}
\end{figure}

\begin{table}
\centering
\renewcommand{\arraystretch}{1.2}
\begin{tabular}{|>{$}c<{$}|>{$}l<{$}|}
\hline
n&P_n(x)\\
\hline
 0&1
\\
 1&x
\\
 2&\frac12(3x^2-1)
\\
 3&\frac12(5x^3-3x)
\\
 4&\frac18(35x^4-30x^2+3)
\\
 5&\frac18(63x^5-70x^3+15x)
\\
 6&\frac1{16}(231x^6-315x^4+105x^2-5)
\\
 7&\frac1{16}(429x^7-693x^5+315x^3-35x)
\\
 8&\frac1{128}(6435x^8-12012x^6+6930x^4-1260x^2+35)
\\
 9&\frac1{128}(12155x^9-25740x^7+18018x^5-4620x^3+315x)
\\
10&\frac1{256}(46189x^{10}-109395x^8+90090x^6-30030x^4+3465x^2-63)
\\[2pt]
\hline
\end{tabular}
\caption{Die Legendre-Polynome $P_n(x)$ für $n=0,1,\dots,10$ sind
orthogonale Polynome vom Grad $n$, die den Wert $P_n(1)=1$ haben.
\label{buch:integral:table:legendre-polynome}}
\end{table}

Die so konstruierten Polynome heissen die {\em Legendre-Polynome}.
Durch weitere Durchführung des Verfahrens liefert die Polynome in
Tabelle~\ref{buch:integral:table:legendre-polynome}.
Die Graphen sind in Abbildung~\ref{buch:integral:orthogonal:legendregraphen}
dargestellt.
Abbildung~\ref{buch:integral:orthogonal:legendreortho} illustriert, 
dass die die beiden Polynome $P_4(x)$ und $P_7(x)$ orthogonal sind.
Das Produkt $P_4(x)\cdot P_7(x)$ hat Integral $=0$.

%
% Rekursionsrelation
%
\subsection{Drei-Term-Rekursion
\label{buch:orthogonal:subsection:rekursionsrelation}}
Die Berechnung der Legendre-Polynome mit Hilfe des Gram-Schmidt-Verfahrens
ist ausserordentlich mühsame wenig hilfreich, wenn es darum geht, Werte
der Polynome zu berechnen.
Glücklicherweise erfüllen orthogonale Polynome automatisch eine 
Rekursionsbeziehung mit nur drei Termen.
Zum Beispiel kann man zeigen, dass für die Legendre-Polynome die
Relation
\begin{align*}
nP_n(x) &= (2n-1)xP_{n-1}(x) - (n-1)P_{n-2}(x),\;\forall n\ge 2,
\\
P_1(x) &= x,
\\
P_0(x) &= 1.
\end{align*}
Mit so einer Rekursionsbeziehung ist es sehr einfach, die Funktionswerte
für alle $P_n(x)$ zu berechnen.

\begin{definition}
Eine Folge von Polynomen $p_n(x)$ heisst orthogonal bezüglich des
Skalarproduktes $\langle\,\;,\;\rangle_w$, wenn 
\[
\langle p_n,p_m\rangle_w = h_n \delta_{nm}
\]
für alle $n$, $m$.
\end{definition}

\subsubsection{Allgemeine Drei-Term-Rekursion für orthogonale Polynome}
Der folgende Satz besagt, dass $p_n$ eine Rekursionsbeziehung erfüllt.

\begin{satz}
\label{buch:orthogonal:satz:drei-term-rekursion}
Eine Folge bezüglich $\langle\,\;,\;\rangle_w$ orthogonaler Polynome $p_n$ 
mit dem Grade $\deg p_n = n$ erfüllt eine Rekursionsbeziehung der Form
\begin{equation}
p_{n+1}(x)
=
(A_nx+B_n)p_n(x) - C_np_{n-1}(x)
\label{buch:orthogonal:eqn:rekursion}
\end{equation}
für $n\ge 0$, wobei $p_{-1}(x)=0$ gesetzt wird.
Die Zahlen $A_n$, $B_n$ und $C_n$ sind reell und es ist
$A_{n-1}A_nC_n\ge 0$ für $n>0$. 
Wenn $k_n>0$ der Leitkoeffizient von $p_n(x)$ ist, dann gilt
\begin{equation}
A_n=\frac{k_{n+1}}{k_n},
\qquad
C_{n+1} = \frac{A_{n+1}}{A_n}\frac{h_{n+1}}{h_n}.
\label{buch:orthogonal:eqn:koeffizientenrelation}
\end{equation}
\end{satz}

\subsubsection{Multiplikationsoperator mit $x$}
Man kann die Relation auch nach dem Produkt $xp_n(x)$ auflösen, dann
wird sie
\begin{equation}
xp_n(x)
=
\frac{1}{A_n}p_{n+1}(x)
-
\frac{B_n}{A_n}p_n(x)
+
\frac{C_n}{A_n}p_{n-1}(x).
\label{buch:orthogonal:eqn:multixrelation}
\end{equation}
Die Multiplikation mit $x$ ist eine lineare Abbildung im Raum der Funktionen.
Die Relation~\eqref{buch:orthogonal:eqn:multixrelation} besagt, dass diese
Abbildung in der Basis der Polynome $p_k$ tridiagonale Form hat.

\subsubsection{Drei-Term-Rekursion für die Tschebyscheff-Polynome}
Eine Relation der Form~\eqref{buch:orthogonal:eqn:multixrelation}
wurde bereits in 
Abschnitt~\ref{buch:potenzen:tschebyscheff:rekursionsbeziehungen}
hergeleitet.
In der Form~\eqref{buch:orthogonal:eqn:rekursion} geschrieben lautet
sie
\[
T_{n+1}(x) = 2x\,T_n(x)-T_{n-1}(x).
\]
also
$A_n=2$, $B_n=0$ und $C_n=1$.

\subsubsection{Beweis von Satz~\ref{buch:orthogonal:satz:drei-term-rekursion}}
Die Relation~\eqref{buch:orthogonal:eqn:multixrelation} zeigt auch,
dass der Beweis die Koeffizienten $\langle xp_k,p_j\rangle_w$
berechnen muss.
Dabei wird wiederholt der folgende Trick verwendet.
Für jede beliebige Funktion $f$ mit $\|f\|_w^2<\infty$ ist
\[
\langle fp_k,p_j\rangle_w
=
\langle p_k,fp_j\rangle_w.
\]
Für $f(x)=x$ kann man weiter verwenden, dass $xp_k(x)$ ein Polynom
vom Grad $k+1$ ist.
Die Gleichheit $\langle xp_k,p_j\rangle_w=\langle p_k,xp_j\rangle_w$
ermöglicht also, den Faktor $x$ dorthin zu schieben, wo es nützlicher ist.

\begin{proof}[Beweis des Satzes]
Multipliziert man die rechte Seite von
\eqref{buch:orthogonal:eqn:rekursion} aus, dann ist der einzige Term
vom Grad $n+1$ der Term $A_nxp_n(x)$.
Der Koeffizient $A_n$ ist also dadurch festgelegt, dass
\begin{equation}
b(x)
=
p_{n+1}(x) - A_nxp_n(x)
\label{buch:orthogonal:rekbeweis}
\end{equation}
Grad $\le n$ hat.
Dazu müssen sich die Terme vom Grad $n+1$ in den Polynomen wegheben,
d.~h.~$k_{n+1}-A_nk_n=0$, woraus die erste Beziehung in
\eqref{buch:orthogonal:eqn:koeffizientenrelation} folgt.

Die Polynome $p_k$ sind durch Orthogonalisierung der Monome
$1$, $x$,\dots $x^{k}$ entstanden.
Dies bedeutet, dass $\langle p_n,x^k\rangle_w=0$ für alle $k<n$
gilt und daher auch $\langle p_n,Q\rangle_w=0$ für jedes Polynome
$Q(x)$ vom Grad $<n$.

Das Polynom $b(x)$ ist vom Grad $\le n$, es lässt sich also als
Linearkombination
\[
b(x) = \sum_{k=0}^n b_k p_k(x)
\]
der $p_k$ mit $k\le n$ schreiben.
Die Koeffizienten $b_j$ kann man erhalten, indem man 
\eqref{buch:orthogonal:rekbeweis} Skalar mit $p_j$ multipliziert.
Dabei erhält man
\[
h_jb_j
=
\langle b,p_j\rangle_w
=
\langle p_{n+1},p_j\rangle_w
-
A_n\langle xp_n,p_j\rangle_w.
\]
Für $j\le n$ verschwindet der erste Term nach der Definition einer
Folge von orthogonalen Polynomen.
Den zweiten Term kann man umformen in
\[
\langle xp_n,p_j\rangle_w
=
\langle p_n,xp_j\rangle_w.
\]
Darin ist $xp_j$ ein Polynom vom Grad $j+1$.
Für $n>j+1$ folgt, dass der zweite Term verschwindet.
Somit sind alle $b_j=0$ mit $j<n-1$, nur der Term $j=n-1$
bleibt bestehen.
Mit $B_n=b_n$ und $C_n=b_{n-1}$ bekommt man die somit die
Rekursionsbeziehung~\eqref{buch:orthogonal:eqn:rekursion}.

Indem man das Skalarprodukt von~\eqref{buch:orthogonal:eqn:rekursion}
mit $p_{n-1}$ bildet, findet man
\begin{align}
\underbrace{\langle
p_{n+1},p_{n-1}
\rangle_w}_{\displaystyle=0}
&=
\langle (A_nx+B_n)p_n+C_np_{n-1},p_{n-1} \rangle_w
\notag
\\
0
&=
A_n\langle xp_n,p_{n-1} \rangle_w
+B_n\underbrace{\langle p_n,b_{n-1}\rangle_w}_{\displaystyle=0}
-C_n\|p_{n-1}\|_w^2
\notag
\\
0
&=
A_n\langle p_n,xp_{n-1} \rangle_w
-C_n\|p_{n-1}\|_w^2
\label{buch:orthogonal:eqn:rekbeweis2}
\end{align}
Indem man $xp_n$ als
\[
xp_{n-1}(x)
=
\frac{k_{n-1}}{k_n} p_n(x)
+
\sum_{k=0}^{n-1} d_kp_k(x)
\]
schreibt, bekommt man
\begin{align*}
\langle
p_n,
xp_{n-1}
\rangle_w
&=
\biggl\langle
p_n,
\frac{k_{n-1}}{k_n} p_n
+
\sum_{k=0}^{n-1} d_kp_k
\biggr\rangle_w
=
\frac{k_{n-1}}{k_n}h_n
+
\sum_{k=0}^{n-1} d_k\underbrace{\langle p_n,p_k\rangle_w}_{\displaystyle=0}
\end{align*}
Eingesetzt in~\eqref{buch:orthogonal:eqn:rekbeweis2} erhält man
\[
A_n\frac{k_{n-1}}{k_n}h_n = C_n h_{n-1}
\qquad\Rightarrow\qquad
C_n
=
A_n\frac{k_{n-1}}{k_n}\frac{h_n}{h_{n-1}},
\]
damit ist auch die zweite Beziehung von
\eqref{buch:orthogonal:eqn:koeffizientenrelation}.
\end{proof}

%
% rekursion.tex -- drei term rekursion für orthogonale Polynome
%
% (c) 2022 Prof Dr Andreas Müller, OST Ostschweizer Fachhochschule
%
\section{Drei-Term-Rekursion für orthogonale Polynome
\label{buch:orthogonal:section:drei-term-rekursion}}
Die Berechnung der Legendre-Polynome mit Hilfe des Gram-Schmidt-Verfahrens
ist wenig hilfreich, wenn es darum geht, Werte der Polynome zu berechnen.
Glücklicherweise erfüllen orthogonale Polynome automatisch eine 
Rekursionsbeziehung mit nur drei Termen.
Zum Beispiel kann man zeigen, dass für die Legendre-Polynome die
Relation
\begin{align*}
nP_n(x) &= (2n-1)xP_{n-1}(x) - (n-1)P_{n-2}(x),\;\forall n\ge 2,
\\
P_1(x) &= x,
\\
P_0(x) &= 1.
\end{align*}
Mit so einer Rekursionsbeziehung ist es sehr einfach, die Funktionswerte
für alle $P_n(x)$ zu berechnen.

\begin{definition}
Eine Folge von Polynomen $p_n(x)$ heisst orthogonal bezüglich des
Skalarproduktes $\langle\,\;,\;\rangle_w$, wenn 
\[
\langle p_n,p_m\rangle_w = h_n \delta_{nm}
\]
für alle $n$, $m$.
\end{definition}

\subsection{Allgemeine Drei-Term-Rekursion für orthogonale Polynome}
Der folgende Satz besagt, dass $p_n$ eine Rekursionsbeziehung erfüllt.

\begin{satz}
\label{buch:orthogonal:satz:drei-term-rekursion}
Eine Folge bezüglich $\langle\,\;,\;\rangle_w$ orthogonaler Polynome $p_n$ 
mit dem Grade $\deg p_n = n$ erfüllt eine Rekursionsbeziehung der Form
\begin{equation}
p_{n+1}(x)
=
(A_nx+B_n)p_n(x) - C_np_{n-1}(x)
\label{buch:orthogonal:eqn:rekursion}
\end{equation}
für $n\ge 0$, wobei $p_{-1}(x)=0$ gesetzt wird.
Die Zahlen $A_n$, $B_n$ und $C_n$ sind reell und es ist
$A_{n-1}A_nC_n\ge 0$ für $n>0$. 
Wenn $k_n>0$ der Leitkoeffizient von $p_n(x)$ ist, dann gilt
\begin{equation}
A_n=\frac{k_{n+1}}{k_n},
\qquad
C_{n+1} = \frac{A_{n+1}}{A_n}\frac{h_{n+1}}{h_n}.
\label{buch:orthogonal:eqn:koeffizientenrelation}
\end{equation}
\end{satz}

\subsection{Multiplikationsoperator mit $x$}
Man kann die Relation auch nach dem Produkt $xp_n(x)$ auflösen, dann
wird sie
\begin{equation}
xp_n(x)
=
\frac{1}{A_n}p_{n+1}(x)
-
\frac{B_n}{A_n}p_n(x)
+
\frac{C_n}{A_n}p_{n-1}(x).
\label{buch:orthogonal:eqn:multixrelation}
\end{equation}
Die Multiplikation mit $x$ ist eine lineare Abbildung im Raum der Funktionen.
Die Relation~\eqref{buch:orthogonal:eqn:multixrelation} besagt, dass diese
Abbildung in der Basis der Polynome $p_k$ tridiagonale Form hat.

\subsection{Drei-Term-Rekursion für die Tschebyscheff-Polynome}
Eine Relation der Form~\eqref{buch:orthogonal:eqn:multixrelation}
wurde bereits in 
Abschnitt~\ref{buch:potenzen:tschebyscheff:rekursionsbeziehungen}
hergeleitet.
In der Form~\eqref{buch:orthogonal:eqn:rekursion} geschrieben lautet
sie
\[
T_{n+1}(x) = 2x\,T_n(x)-T_{n-1}(x).
\]
also
$A_n=2$, $B_n=0$ und $C_n=1$.

\subsection{Beweis von Satz~\ref{buch:orthogonal:satz:drei-term-rekursion}}
Die Relation~\eqref{buch:orthogonal:eqn:multixrelation} zeigt auch,
dass der Beweis die Koeffizienten $\langle xp_k,p_j\rangle_w$
berechnen muss.
Dabei wird wiederholt der folgende Trick verwendet.
Für jede beliebige Funktion $f$ mit $\|f\|_w^2<\infty$ ist
\[
\langle fp_k,p_j\rangle_w
=
\langle p_k,fp_j\rangle_w.
\]
Für $f(x)=x$ kann man weiter verwenden, dass $xp_k(x)$ ein Polynom
vom Grad $k+1$ ist.
Die Gleichheit $\langle xp_k,p_j\rangle_w=\langle p_k,xp_j\rangle_w$
ermöglicht also, den Faktor $x$ dorthin zu schieben, wo es nützlicher ist.

\begin{proof}[Beweis des Satzes]
Multipliziert man die rechte Seite von
\eqref{buch:orthogonal:eqn:rekursion} aus, dann ist der einzige Term
vom Grad $n+1$ der Term $A_nxp_n(x)$.
Der Koeffizient $A_n$ ist also dadurch festgelegt, dass
\begin{equation}
b(x)
=
p_{n+1}(x) - A_nxp_n(x)
\label{buch:orthogonal:rekbeweis}
\end{equation}
Grad $\le n$ hat.
Dazu müssen sich die Terme vom Grad $n+1$ in den Polynomen wegheben,
d.~h.~$k_{n+1}-A_nk_n=0$, woraus die erste Beziehung in
\eqref{buch:orthogonal:eqn:koeffizientenrelation} folgt.

Die Polynome $p_k$ sind durch Orthogonalisierung der Monome
$1$, $x$,\dots $x^{k}$ entstanden.
Dies bedeutet, dass $\langle p_n,x^k\rangle_w=0$ für alle $k<n$
gilt und daher auch $\langle p_n,Q\rangle_w=0$ für jedes Polynome
$Q(x)$ vom Grad $<n$.

Das Polynom $b(x)$ ist vom Grad $\le n$, es lässt sich also als
Linearkombination
\[
b(x) = \sum_{k=0}^n b_k p_k(x)
\]
der $p_k$ mit $k\le n$ schreiben.
Die Koeffizienten $b_j$ kann man erhalten, indem man 
\eqref{buch:orthogonal:rekbeweis} Skalar mit $p_j$ multipliziert.
Dabei erhält man
\[
h_jb_j
=
\langle b,p_j\rangle_w
=
\langle p_{n+1},p_j\rangle_w
-
A_n\langle xp_n,p_j\rangle_w.
\]
Für $j\le n$ verschwindet der erste Term nach der Definition einer
Folge von orthogonalen Polynomen.
Den zweiten Term kann man umformen in
\[
\langle xp_n,p_j\rangle_w
=
\langle p_n,xp_j\rangle_w.
\]
Darin ist $xp_j$ ein Polynom vom Grad $j+1$.
Für $n>j+1$ folgt, dass der zweite Term verschwindet.
Somit sind alle $b_j=0$ mit $j<n-1$, nur der Term $j=n-1$
bleibt bestehen.
Mit $B_n=b_n$ und $C_n=b_{n-1}$ bekommt man die somit die
Rekursionsbeziehung~\eqref{buch:orthogonal:eqn:rekursion}.

Indem man das Skalarprodukt von~\eqref{buch:orthogonal:eqn:rekursion}
mit $p_{n-1}$ bildet, findet man
\begin{align}
\underbrace{\langle
p_{n+1},p_{n-1}
\rangle_w}_{\displaystyle=0}
&=
\langle (A_nx+B_n)p_n+C_np_{n-1},p_{n-1} \rangle_w
\notag
\\
0
&=
A_n\langle xp_n,p_{n-1} \rangle_w
+B_n\underbrace{\langle p_n,b_{n-1}\rangle_w}_{\displaystyle=0}
-C_n\|p_{n-1}\|_w^2
\notag
\\
0
&=
A_n\langle p_n,xp_{n-1} \rangle_w
-C_n\|p_{n-1}\|_w^2
\label{buch:orthogonal:eqn:rekbeweis2}
\end{align}
Indem man $xp_n$ als
\[
xp_{n-1}(x)
=
\frac{k_{n-1}}{k_n} p_n(x)
+
\sum_{k=0}^{n-1} d_kp_k(x)
\]
schreibt, bekommt man
\begin{align*}
\langle
p_n,
xp_{n-1}
\rangle_w
&=
\biggl\langle
p_n,
\frac{k_{n-1}}{k_n} p_n
+
\sum_{k=0}^{n-1} d_kp_k
\biggr\rangle_w
=
\frac{k_{n-1}}{k_n}h_n
+
\sum_{k=0}^{n-1} d_k\underbrace{\langle p_n,p_k\rangle_w}_{\displaystyle=0}
\end{align*}
Eingesetzt in~\eqref{buch:orthogonal:eqn:rekbeweis2} erhält man
\[
A_n\frac{k_{n-1}}{k_n}h_n = C_n h_{n-1}
\qquad\Rightarrow\qquad
C_n
=
A_n\frac{k_{n-1}}{k_n}\frac{h_n}{h_{n-1}},
\]
damit ist auch die zweite Beziehung von
\eqref{buch:orthogonal:eqn:koeffizientenrelation}.
\end{proof}

%
% rodrigues.tex
%
% (c) 2022 Prof Dr Andreas Müller, OST Ostschweizer Fachhochschule
%
\section{Rodrigues-Formeln
\label{buch:orthogonalitaet:section:rodrigues}}
\rhead{Rodrigues-Formeln}
Die Drei-Term-Rekursionsformel ermöglicht Werte orthogonaler Polynome
effizient zu berechnen.
Die Rekursionsformel erhöht den Grad eines Polynoms, indem mit $x$ 
multipliziert wird.
mit der Ableitung kann man den Grad aber auch senken, man könnte daher
auch nach einer Rekursionsformel fragen, die bei einem Polynom hohen
Grades beginnt und mit Hilfe von Ableitungen zu geringeren Graden
absteigt.
Solche Formeln heissen {\em Rodrigues-Formeln} nach dem Entdecker Olinde
\index{Rodriguez, Olinde}%
Rodrigues, der eine solche Formal als erster für Legendre-Polynome
gefunden hat.

In diesem Abschnitt sei $p_n(x)$ eine bezüglich des Skalarproduktes
$\langle\,\;,\;\rangle_w$ auf dem Intervall $[a,b]$ orthogonale Familie
von Polynomen mit genaum dem Grad $\deg p_n=n$.
Die Skalarprodukte sollen 
\[
\langle p_n,p_m\rangle_w = h_n\delta_{nm}
\]
sein.

%
% Pearsonsche Differentialgleichung
%
\subsection{Pearsonsche Differentialgleichung}
Die {\em Pearsonsche Differentialgleichung} ist die Differentialgleichung
\begin{equation}
B(x) y' - A(x) y = 0,
\label{buch:orthogonal:eqn:pearson}
\end{equation}
\index{Differentialgleichung!Pearsonsche}%
\index{Pearsonsche Differentialgleichung}%
wobei $B(x)$ ein Polynom vom Grad höchstens $2$ ist und $A(x)$ ein
höchstens lineares Polynom.
Die Gleichung~\eqref{buch:orthogonal:eqn:pearson}
kann gelöst werden, wenn $y$ und $B(x)$ keine Nullstellen  haben.
Dann kann man die Gleichung umstellen in
\[
\frac{y'}{y}
=
(\log y)'
=
\frac{A(x)}{B(x)}
\qquad\Rightarrow\qquad
y
=
\exp\biggl(
\int\frac{A(x)}{B(x)}
\,dx
\biggr)
.
\]
Im Folgenden nehmen wir zusätzlich an, dass an den Intervallenden
\begin{equation}
\lim_{x\to a+} w(x)B(x) = 0,
\qquad\text{und}\qquad
\lim_{x\to b-} w(x)B(x) = 0
\end{equation}
gilt.

Falls $w(x)$ an den Intervallenden einen von $0$ verschiedenen
Grenzwert hat, bedeutet dies, dass $B(a)=B(b)=0$ sein muss.
Falls $w(x)$ am Intervallende divergiert, muss $B(x)$ dort eine
Nullstelle höherer Ordnung haben, was aber für ein Polynom
zweiten Grades nicht möglich ist.

%
% Rekursionsformel
%
\subsection{Rekursionsformel}
Multiplikation mit $B(x)$ wird den Grad eines Polynomes typischerweise 
um $2$ erhöhen, die Ableitung wird ihn wieder um $1$ reduzieren.
Etwas formeller kann man dies wie folgt formulieren:

\begin{satz}
Für alle $n\ge 0$ ist
\begin{equation}
q_n(x)
=
\frac{1}{w(x)}
\frac{d^n}{dx^n} B(x)^n w(x)
\label{buch:orthogonalitaet:rodrigues:eqn:rekursion}
\end{equation}
ein Polynom vom Grad höchstens $n$.
\end{satz}

\begin{proof}[Beweis]
Wenn $r_0(x)$ irgend eine differenzierbare Funktion ist, dann ist
\begin{align*}
\frac{d^n}{dx^n}
r_0(x) B(x)^n w(x)
&=
\frac{d^{n-1}}{dx^{n-1}}\frac{d}{dx} r_0(x) B(x)^n w(x)
\\
&=
\frac{d^{n-1}}{dx^{n-1}}
\bigl(r_0'(x)B(x)+ nr_0(x)B'(x)B(x)^{n-1}w(x) + r_0(x)B(x)^n w'(x) \bigr)
\\
&=
\frac{d^{n-1}}{dx^{n-1}}
(\underbrace{r_0'(x)B(x)+nr_0(x)B'(x)+r_0(x)A(x)}_{\displaystyle = r_1(x)})
B(x)^{n-1} w(x)
\\
&=
\frac{d^{n-1}}{dx^{n-1}} r_1(x)B^{n-1}(x) w(x).
\end{align*}
Iterativ lässt sich eine Folge von
Funktionen $r_k(x)$ definieren, für die Rekursionsformel
\begin{equation}
r_k(x) = r_{k-1}'(x)B(x) + \bigl((n+1-k)B'(x) + A(x)\bigr)r_{k-1}(x)
\label{buch:orthogonal:rodrigues:rekursion:beweis1}
\end{equation}
gilt.
Wenn $r_0(x)$ ein Polynom ist, dann sind alle Funktionen $r_k(x)$
ebenfalls Polynome.
Aus der Konstruktion kann man schliessen, dass
\[
\frac{d^n}{dx^n} r_0(x) B(x)^n w(x)
=
r_n(x) w(x).
\]
Insbesondere folgt für $r_0(x)=1$, dass die $n$-te Ableitung den
Faktor $w(x)$ enthält und dass somit $r_n(x)=q_n(x)$ ein Polynom ist.

Wir müssen auch noch den Grad von $r_k(x)$ bestimmen, wobei wir
wieder von $r_0(x)=1$ ausgehen.
Wir behaupten, dass $\deg r_k(x)\le k$ ist, und beweisen dies
mit vollständiger Induktion.
Für $k=0$ ist $\deg r_0(x) = 0 \le k$ die Induktionsverankerung.

Wir nehmen jetzt also an, dass $\deg r_{k-1}(x)\le k-1$ ist und
verwenden 
\eqref{buch:orthogonal:rodrigues:rekursion:beweis1} um den Grad zu berechnen:
\begin{equation*}
\deg r_k(x)
=
\max \bigl(
\underbrace{\deg(r_{k-1}'(x) B(x))}_{\displaystyle (k-1) -1 + 2}
,
\underbrace{\deg(r_{k-1}(x)B'(x))}_{\displaystyle \le (k-1)+1}
,
\underbrace{\deg(r_{k-1}(x)A(x))}_{\displaystyle \le (k-1)+1}
\bigr)
\le k.
\end{equation*}
Damit ist der Induktionsschritt und $\deg r_k(x)\le k$ bewiesen.
Damit ist auch gezeigt, dass $\deg q_n(x)\le n$.
\end{proof}

Die Rodrigues-Formel~\eqref{buch:orthogonalitaet:rodrigues:eqn:rekursion}
produziert eine Folge von Polynomen aufsteigenden Grades, es ist aber
noch nicht klar, dass diese Polynome bezüglich des gewählten Skalarproduktes
orthogonal sind.
Dies ist der Inhalt des folgenden Satzes.

\begin{satz}
Es gibt Konstanten $c_n$ derart, dass
\[
p_n(x)
=
\frac{c_n}{w(x)} \frac{d^n}{dx^n} \bigl(B(x)^n w(x)\bigr) 
\]
gilt.
\end{satz}

\begin{proof}[Beweis]
Wir zeigen, dass die Polynome orthogonal sind auf allen Monomen
von geringerem Grad.
\begin{align*}
\langle q_n, x^k\rangle_w
&=
\int_a^b q_n(x)x^kw(x)\,dx
\\
&=
\int_a^b \frac{1}{w(x)}
\biggl(\frac{d^n}{dx^n}\bigl(B(x)^n w(x)\bigr)\biggr)
x^k w(x)\,dx
\\
&=
\int_a^b \frac{d^n}{dx^n}\bigl(B(x)^n w(x)\bigr) x^k \,dx
\\
&=
\biggl[\frac{d^{n-1}}{dx^{n-1}}\bigl(B(x)^n w(x)\bigr) x^k \biggr]_a^b
-
\int_a^b \frac{d^{n-1}}{dx^{n-1}}\bigl(B(x)^n w(x)\bigr)kx^{k-1}\,dx
\end{align*}
Durch $n$-fache Iteration wird das Integral auf $0$ reduziert.
Es bleiben nur die eckigen Klammern stehen, doch wenn man die Produktregel
auswertet, bleibt immer mindestens ein Produkt $B(x)w(x)$ stehen,
nach den Voraussetzungen an den Grenzwert dieses Produktes an den
Intervallenden verschwinden diese Terme alle.
Damit sind die $q_n(x)$ Polynome, die $w$-orthogonal sind auf allen
$x^k$ mit $k<n$.

Die Polynome $q_k(x)$ mit $k< n$ haben Grad $<n$ und sind daher
Linearkombinationen von Monomen vom Grad $<n$.
Soeben wurde gezeigt, dass $q_n(x)$ orthogonal auf diesen Monomen
ist, also auch auf $q_k(x)$ mit $k<n$.
Damit ist gezeigt, dass Polynome $q_n(x)$ eine orthogonale Familie
von Polynomen bilden.
Durch Normierung müssen sich daraus die Polynome $p_n(x)$ ergeben.
\end{proof}

%
% Legendre-Polynome
%
\subsubsection{Legendre-Polynome}
Legendre-Polynome sind orthogonale Polynome zum Standardskalarprodukt
mit $w(x)=1$.
Die Pearsonsche Differentialgleichung ist für $A(x)=0$ immer erfüllt.
Die Randbedingung bedeutet wegen $w(x)=1$, dass $B(x)$ an den
Endpunkten des Intervalls verschwinden muss.
Da $B(x)$ ein Polynom höchstens vom Grad $2$ ist, muss $B(x)$ ein
Vielfaches von $(x-1)(x+1)=x^2-1$ sein.
Die Rodrigues-Formel für die Legendre-Polynome hat daher die Form
\[
P_n(x)
=
c_n
\frac{d^n}{dx^n}
(x^2-1)^n,
\]
darin müssen die Konstanten $c_n$ noch bestimmt werden.
In der für die Legendre-Polynome gewählten Normierung ist
\[
c_n = \frac1{2^n n!}
\qquad\text{und damit}\qquad
P_n(x)
=
\frac{1}{2^nn!}
\frac{d^n}{dx^n}
(x^2-1)^n.
\]

%
% Hermite-Polynome
%
\subsubsection{Hermite-Polynome}
Die Hermite-Polynome sind auf ganz $\mathbb{R}$ definiert und verwenden
die Gewichtsfunktion
\[
w(x) = e^{-x^2}.
\]
Für jedes beliebige Polynome $B(x)$, auch für höheren Grad als $2$, ist
\[
\lim_{x\to-\infty} B(x) w(x)
=
\lim_{x\to-\infty} B(x)e^{-x^2}
=
0
\qquad\text{und}\qquad
\lim_{x\to\infty} B(x) w(x)
=
\lim_{x\to\infty} B(x)e^{-x^2}
=
0,
\]
die Randbedingung der Pearsonschen Differentialgleichung ist also
immer erfüllt.

Die Ableitung der Gewichtsfunktion ist
\[
w'(x) = -2xe^{-x^2}.
\]
Eingesetzt in die Pearsonsche Differentialgleichung findet man
\[
\frac{w'(x)}{w(x)}
=
\frac{-2xe^{-x^2}}{e^{-x^2}}
=
\frac{-2x}{1}
\]
und daher
\[
A(x) = -2x
\qquad\text{und}\qquad
B(x) = 1.
\]
Die Gradbedingung ist also immer erfüllt und es folgt die Rodrigues-Formel
für die Hermite-Polynome
\index{Hermite-Polynom}%
\index{Polynome!Hermite}%
\begin{equation}
H_n(x)
=
c_n
e^{x^2}\frac{d^n}{dx^n} e^{-x^2}
=
(-1)^n
e^{x^2}\frac{d^n}{dx^n} e^{-x^2}.
\label{buch:orthogonal:eqn:hermite-rodrigues}
\end{equation}

Die Hermite-Polynome können mit der Rodrigues-Formel berechnet werden,
aber die Form~\eqref{buch:orthogonal:eqn:hermite-rodrigues} ist dazu
nicht gut geeignet.
Zur Vereinfachung dient die Berechnung 
\[
-\frac{d}{dx}
\bigl(
e^{-x^2}f(x)
\bigr)
=
2xe^{-x^2}f(x)
-
e^{-x^2}f'(x)
=
e^{-x^2}
\biggl(-\frac{d}{dx}+2x\biggr)
f(x),
\]
nach der der Ableitungsoperator mit dem Faktor $e^{-x^2}$ 
vertauscht werden kann, wenn er durch die grosse Klammer auf der
rechten Seite ersetzt wird.
Die Rodrigues-Formel bekommt daher die Form
\[
H_n(x) = \biggl(2x-\frac{d}{dx}\biggr)^n \cdot 1.
\]

%TODO: Relation zu hypergeometrischen Funktionen $\mathstrut_1F_1$

%\url{https://en.wikipedia.org/wiki/Rodrigues%27_formula}

%
% Jacobi-Gewichtsfunktion
%
\subsubsection{Jacobi-Gewichtsfunktion}
%(%i1) w: (1-x)^a*(1+x)^b;
%                                      a        b
%(%o1)                          (1 - x)  (x + 1)
%(%i2) diff(w,x)/w;
%                        a        b - 1            a - 1        b
%               b (1 - x)  (x + 1)      - a (1 - x)      (x + 1)
%(%o2)          -------------------------------------------------
%                                      a        b
%                               (1 - x)  (x + 1)
%(%i3) q: diff(w,x)/w;
%                        a        b - 1            a - 1        b
%               b (1 - x)  (x + 1)      - a (1 - x)      (x + 1)
%(%o3)          -------------------------------------------------
%                                      a        b
%                               (1 - x)  (x + 1)
%(%i4) ratsimp(q);
%                               (b + a) x - b + a
%(%o4)                          -----------------
%                                     2
%                                    x  - 1
%
Die Jacobi-Gewichtsfunktion 
\index{Jacobi-Gewichtsfunktion}%
\index{Gewichtsfunktion!Jacobi}%
\[
w(x)
=
w^{(\alpha,\beta)}(x)
=
(1-x)^\alpha(1+x)^\beta
\]
hat die Ableitung
\[
w'(x)
=
\beta(1-x)^\alpha(1+x)^{\beta-1}-\alpha(1-x)^{\alpha-1}(1+x)^\beta
\]
und für die linke Seite der Pearsonschen Differentialgleichung findet man
\[
\frac{w'(x)}{w(x)}
=
\frac{
\beta(1-x)^\alpha(1+x)^{\beta-1}-\alpha(1-x)^{\alpha-1}(1+x)^\beta
}{
(1-x)^\alpha(1+x)^\beta
}
=
\frac{\beta-\alpha-(\alpha+\beta)x}{1-x^2}
=
\frac{A(x)}{B(x)}.
\]
Die Polynome
\[
A(x) = \beta-\alpha-(\alpha+\beta)x
\qquad\text{und}\qquad
B(x) = 1-x^2
\]
erfüllen die Gradvoraussetzungen für eine Lösung der Pearsonschen
Differentialgleichung, die Anlass zu einer Rodrigues-Formel gibt.
Die Randbedingungen sind noch zu prüfen: $B(x)$ hat eine Nullstelle
erster Ordnung bei $\pm1$, also ist
\[
\lim_{x\to \pm1\mp} B(x)w(x) = 0
\]
genau dann, wenn $\alpha>-1$ und $\beta>-1$ gilt.
Für $\alpha>-1$ und $\beta>-1$ gibt es daher auch für die Jacobi-Polynome
eine Rodriguez-Formel der Art
\[
P^{(\alpha,\beta)}_n(x)
=
\frac{c_n}{w^{(\alpha,\beta)}(x)}
\frac{d^n}{dx^n}
\bigl((1-x^2)^{n} w^{(\alpha,\beta)}(x)\bigr).
\]
Die Konstanten $c_n$ werden durch die Normierung
% XXX in welchem Abschnitt
festgelegt.

%
% Tschebyscheff-Gewichtsfunktion
%
\subsubsection{Die Tschebyscheff-Gewichtsfunktion}
Die Tschebyscheff-Gewichtsfunktion ist der Spezialfall $a=b=-\frac12$
der Jacobi-Gewichtsfunktion.
\index{Tschebyscheff-Gewichtsfunktion}%
\index{Gewichtsfunktion!Tschebyscheff}%
Die Rodrigues-Formel für die Tschebyscheff-Polynome lautet daher
\[
T_n(x)
=
c_n\sqrt{1-x^2} \frac{d^n}{dx^n}
\frac{(1-x^2)^n}{\sqrt{1-x^2}}
=
\frac{1}{2^nn!} \sqrt{1-x^2}
\frac{d^n}{dx^n} 
\frac{(1-x^2)^n}{\sqrt{1-x^2}},
\]
wobei wir den korrekten Wert von $c_n$ nicht nachgewiesen haben.

%
% Laguerre Gewichtsfunktion
%
\subsubsection{Die Laguerre-Gewichtsfunktion}
Die Laguerre-Gewichtsfunktion
\index{Laguerre-Gewichtsfunktion}%
\index{Gewichtsfunktion!Laguerre}%
\[
w_{\text{Laguerre}}(x)
=
w(x)
=
e^{-x}
\]
hat die Ableitung
\[
w'(x) = -e^{-x},
\]
die Pearsonsche Differentialgleichung ist daher
\[
\frac{w'(x)}{w(x)}=\frac{-1}{1}.
\]
Dies suggeriert $A(x)=-1$ und $B(x)=1$ als Zähler und Nenner der rechten
Seite, aber daraus produziert die Rodrigues-Formel immer nur die konstante
Funktion.
Ausserdem ist die Randbedingung an der Stelle $x=0$ nicht erfüllt.
$B(x)$ muss so gewählt werden, dass
\[
0
=
\lim_{x\to 0+} w(x)B(x)
= 
\lim_{x\to 0+} e^{-x}B(x)
=
\lim_{x\to 0+} B(x)
=
B(0).
\]
Die Annahme einer konstanten Funktion $B(x)$ widerspricht dem.
Aus der Pearsonschen Differentialgleichung folgt $A(x)=-B(x)$.
Da $A(x)$ höchstens vom Grad 1 sein kann und $B(x)$ mindestens
vom Grad $1$ muss, folgt
\[
B(x) = x
\qquad\text{und}\qquad
A(x) = -x.
\]
Die Rodrigues-Formel liefert dann die Laguerre-Polynome als
\[
L_n(x) = c_n e^x \frac{d^n}{dx^n} x^ne^{-x}.
\]
Die Werte von $c_n$ hängen von der gewählten Normierung ab.

Mit der Rodrigues-Formel können die Laguerre-Polynome bis auf
die Normierung recht direkt berechnen.
Dazu versuchen wir die Ableitungen von $f(x)e^{-x}$ dadurch zu
berechnen, dass wir den Gewichtsfaktor $e^{-x}$ möglichst weit
nach links verschieben wie in
\begin{align*}
\frac{d}{dx}
e^{-x}
f(x)
&=
e^{-x}
\bigl( -f(x) + f'(x) \bigr)
=
e^{-x}
\biggl( -1 + \frac{d}{dx}\biggr) f.
\end{align*}
Daraus kann man ablesen, dass die Ableitung nach $x$ mit dem Faktor
$e^{-x}$ vertauscht werden kann, wenn man die Ableitung durch
$-1+d/dx$ ersetzt.
Damit kann jetzt auch die $n$-te Ableitung bestimmen:
\begin{align*}
\frac{d^n}{dx^n}e^{-x}f(x)
&=
e^{-x} \biggl(\frac{d}{dx}-1\biggr)^n f(x)
=
e^{-x} \sum_{k=0}^n (-1)^{k}\binom{n}{k}\frac{d^{n-k}}{dx^{n-k}} f(x)
\end{align*}
Dies muss jetzt auf $f(x)=x^n$ angewendet werden.
Es ergibt sich
\begin{align*}
\frac{d^n}{dx^n}e^{-x}x^n
&=
e^{-x} \sum_{k=0}^n (-1)^{k}\binom{n}{k}\frac{d^{n-k}}{dx^{n-k}} x^n
\\
&=
e^{-x} \sum_{k=0}^n (-1)^{k}\binom{n}{k}
n(n-1)(n-2)\cdots (k+1)
x^k
\\
&=
e^{-x}
\sum_{k=0}^n (-1)^k \frac{n(n-1)\cdots(n-k+1)}{k!}
\frac{n!}{k!}
x^k
\\
&=
e^{-x} n!
\sum_{k=0}^\infty
\frac{(-n)(-n+1)(-n+2)\cdot\ldots\cdot (-n+k-1)}{1\cdot 2\cdot \ldots\cdot k}
\frac{x^k}{k!}
\\
&=
e^{-x} n!
\cdot
\mathstrut_1F_1\biggl(
\begin{matrix}-n\\1\end{matrix}; x
\biggr).
\end{align*}
Die übliche Normierung für die Laguerre-Polynome ist $L_n(0)=1$,
die übereinstimmt mit dem Wert der hypergeometrischen Funktion 
an der Stelle $0$.
Wir fassen die Resultate im folgenden Satz zusammen.

\begin{satz}
Die Laguerre-Polynome vom Grad $n$ haben die Form
\begin{equation}
L_n(x)
=
\sum_{k=0}^n \binom{n}{k}\frac{(-1)^k}{k!}x^k
=
\mathstrut_1F_1\biggl(\begin{matrix}-n\\1\end{matrix};x\biggr).
\label{buch:orthogonal:eqn:laguerre-polynom-hypergeometrisch}
\end{equation}
\end{satz}
Laguerre-Polynome sind als spezielle hypergeometrische Funktionen,
für $n\le 7$ sind sie 
in Tabelle~\ref{buch:orthogonal:table:laguerre} zusammengestellt.
In Abbildung~\ref{buch:orthogonal:fig:laguerre} sind die Laguerre-Polynome
vom Grad $0$ bis $9$ dargestellt.

\begin{figure}
\centering
\includegraphics{chapters/070-orthogonalitaet/images/laguerre.pdf}
\caption{Laguerre-Polynome vom Grad $0$ bis $9$
\label{buch:orthogonal:fig:laguerre}}
\end{figure}
\begin{table}
\renewcommand{\arraystretch}{1.4}
\centering
\begin{tabular}{|>{$}c<{$}|>{$}l<{$}|}
\hline
n& L_n(x)\\
\hline
0&1\\
1&-x+1\\
2&\frac1{2!}(x^2-4x+2)\\
3&\frac{1}{3!}(-x^3+9x^2-18x+6)\\
4&\frac{1}{4!}(x^4-16x^3+72x^2-96x+24)\\
5&\frac{1}{5!}(-x^5+25x^4-200x^3+60x^2-600x+120)\\
6&\frac{1}{6!}(x^6-36x^5+450x^4-2400x^3+5400x^2-4320x+720)\\
7&\frac{1}{7!}(-x^7+49x^6-882x^5+7350x^4-29400x^3+52920x^2-35280x+5040)\\
8&\frac{1}{8!}(x^8-64x^7+1568x^6-18816x^5+117600x^4-376320x^3+564480x^2-322560x+40320)\\
\hline
\end{tabular}
\caption{Laguerre-Polynome $L_n(x)$ für $n=0,\dots,8$
\label{buch:orthogonal:table:laguerre}}
\end{table}



%
% legendredgl.tex
%
% (c) 2021 Prof Dr Andreas Müller, OST Ostschweizer Fachhochschule
%
\subsection{Orthogonale Polynome und Differentialgleichungen}
Legendre hat einen ganz anderen Zugang zu den nach ihm benannten
Polynomen gefunden.
Er hat sie gefunden als die Lösungen einer speziellen Differentialgleichungen.
In diesem Abschnitt sollen diese Funktionen mit der Potenzreihen-Methode
wiedergefunden werden.
Dabei stellt sich heraus, dass diese Polynome auch Eigenfunktionen eines
selbstadjungierten Differentialgoperator sind.
Die Orthogonalität wird dann aus einer Verallgemeinerung der bekannten
Eingeschaft folgen, dass Eigenvektoren einer symmetrischen Matrix zu 
verschiedenen Eigenwerten orthogonal sind.

\subsubsection{Legendre-Differentialgleichung}
Die {\em Legendre-Differentialgleichung} ist die Differentialgleichung
\begin{equation}
(1-x^2) y'' - 2x y' + n(n+1) y = 0
\label{buch:integral:eqn:legendre-differentialgleichung}
\end{equation}
für eine Funktion $y(x)$ auf dem Intervall $[-1,1]$.

Sei $y(x)$ eine Lösung der Differentialgleichung
\eqref{buch:integral:eqn:legendre-differentialgleichung}.
Setzt man $y_s(x)=y(-x)$ in die Differentialgleichung ein, erhält
man
\[
(1-x^2)y_s''(x) - 2x y'_s(x) + n(n+1)y_s(x)
=
(1-x^2)y''(-x) +2x y(-x) +n(n+1)y(-x).
\]
Ersetzt man $t=-x$, dann wird daraus
\[
(1-x^2)y''(t) -2t y(t) + n(n+1) y(t) = 0
\]
aus der Differentialgleichung
\eqref{buch:integral:eqn:legendre-differentialgleichung}.
Insbesondere ist die gespiegelte Funktion $y_s(x)$ ebenfalls
eine Lösung der Differentialgleichung.

Ist $y(x)$ eine Lösung der Differentialgleichung ist, dann lässt
sie sich in die Summe einer geraden und einer ungeraden Funktion
\[
\left.
\begin{aligned}
y_g(x) &= \frac{y(x)+y(-x)}{2}\\
y_u(x) &= \frac{y(x)-y(-x)}{2}
\end{aligned}
\quad
\right\}
\quad
\Rightarrow
\quad
y(x) = y_g(x) + y_u(x)
\]
zerlegen, die als Linearkombinationen der beiden Lösungen
$y(x)$ und $y_s(x)$ ebenfalls Lösungen der Differentialgleichung
sind.

\subsubsection{Potenzreihenlösung}
Wir suchen eine Lösung in Form einer Potenzreihe um $x=0$ und 
verwenden dazu den Ansatz
\[
y(x) = a_0+a_1x+a_2x^2+ \dots = \sum_{k=0}^\infty a_kx^k.
\]
\begin{align*}
(1-x^2) \sum_{k=2}^\infty k(k-1)a_kx^{k-2}
-2x\sum_{k=0}^\infty ka_kx^{k-1}
+
n(n+1)\sum_{k=0}^\infty  a_kx^k
&=
0
\\
\sum_{k=0}^\infty (k+2)(k+1)a_{k+2}x^k
-
\sum_{k=2}^\infty k(k-1)a_kx^k
-
2\sum_{k=1}^\infty ka_kx^k
+
n(n+1)\sum_{k=0}^\infty  a_kx^k
&=
0
\end{align*}
Die Koeffizienten zur Potenz $k$ sind daher
\begin{align}
k&=0:
&
0&=
2a_2+n(n+1)a_0
\notag
\\
&&
a_2&=-\frac{n(n+1)}{2}a_0
\notag
\\
k&=1:
&
0&=
6a_3-2a_1+n(n+1)a_1
\notag
\\
&&
a_3&= \frac{2-n(n+1)}{6}a_1
\notag
\\
k&>1:
&
0&=
(k+2)(k+1)a_{k+2} -k(k-1)a_k -2ka_k +n(n+1) a_k
\notag
\\
&&
a_{k+2}
&=
\frac{ k(k+1)-n(n+1) }{(k+2)(k+1)}
a_k
\label{buch:integral:legendre-dgl:eqn:akrek}
\end{align}
Wenn $a_1=0$ und $a_0\ne 1$ ist, dann ist die Funktion $y(x)$ gerade,
alle ungeraden Koeffizienten verschwinden.
Ebenso verschwinden alle geraden Koeffizienten, wenn $a_0=0$ und $a_1\ne 0$.
Für jede Lösung $y(x)$ der Differentialgleichung ist
$y_g(x)$ ein Lösung mit $a_1=0$ und $y_u(x)$ eine Lösung mit $a_0=0$.
Wir können die Diskussion der Lösungen daher auf gerade oder ungerade
Lösungen einschränken.

Gesucht ist jetzt eine Lösung in Form eines Polynoms.
In diesem Fall müssen die Koeffizienten $a_k$ ab einem
gewissen Index verschwinden.
Dies tritt nach \eqref{buch:integral:legendre-dgl:eqn:akrek} genau
dann auf, wenn der Zähler für ein $k$ verschwindet.
Folglich gibt es genau dann Polynomlösungen der Differentialgleichungen,
wenn $n$ eine natürlich Zahl ist.
Ausserdem ist die Lösung ein Polynom $\bar{P}_n(x)$ vom Grad $n$.
Das Polynom soll wieder so normiert sein, dass $\bar{P}_n(1)=1$ ist.

Die Lösungen der Differentialgleichungen können jetzt explizit
berechnet werden.
Zunächst ist $\bar{P}_0(x)=1$ und $\bar{P}_1(x)=x$.
Für $n=2$ setzen wir zunächst $a_0=1$ und $a_1=0$ und erhalten
\[
y(x)
=
1 + \frac{0(0+1) - 2(2+1)}{(0+2)(0+1)}a_0 x^2
=
1
-3x^2
\qquad\text{oder}\qquad
\bar{P}_3(x) = \frac12(3x^2-1).
\]
Für $n=3$ starten wir von $a_1=1$ und $a_0=0$, was zunächst $a_2=0$
impliziert.
Für $a_3$ finden wir
\[
a_3=\frac{1(1+1)-3(3+1)}{(1+2)(1+1)} = -\frac53
\qquad\Rightarrow\qquad
y(x) = x-\frac53x^3
\qquad\Rightarrow\qquad
\bar{P}_3(x) = \frac12(5x^3-3x).
\]
Dies stimmt überein mit den früher gefundenen Ausdrücken für
die Legendre-Polynome.

Die Potenzreihenlösung zeigt zwar, dass es für jedes $n\in\mathbb{N}$
eine Polynomlösung $\bar{P}_n(x)$ vom Grad $n$ gibt.
Dies kann aber nicht erklären, warum die so gefundenen Polynome
orthogonal sind.

\subsubsection{Eigenfunktionen}
Die Differentialgleichung
\eqref{buch:integral:eqn:legendre-differentialgleichung}
Kann mit dem Differentialoperator
\[
D = \frac{d}{dx}(1-x^2)\frac{d}{dx}
\]
als
\[
Dy + n(n+1)y = 0
\]
geschrieben werden.
Tatsächlich ist
\[
Dy
=
\frac{d}{dx} (1-x^2) \frac{d}{dy}
=
\frac{d}{dx} (1-x^2)y'
=
(1-x^2)y'' -2x y'.
\]
Dies bedeutet, dass die Lösungen $\bar{P}_n(x)$ Eigenfunktionen
des Operators $D$ zum Eigenwert $n(n+1)$ sind:
\[
D\bar{P}_n = -n(n+1) \bar{P}_n.
\]

\subsubsection{Orthogonalität von $\bar{P}_n$ als Eigenfunktionen}
Ein Operator $A$ auf Funktionen heisst {\em selbstadjungiert}, wenn
für zwei beliebige Funktionen $f$ und $g$ gilt
\[
\langle Af,g\rangle = \langle f,Ag\rangle
\]
gilt.
Im vorliegenden Zusammenhang möchten wir die Eigenschaft nutzen,
dass Eigenfunktionen eines selbstadjungierten Operatores zu verschiedenen
Eigenwerten orthogonal sind.
Dazu seien $Df = \lambda f$ und $Dg=\mu g$ und wir rechnen
\begin{equation}
\renewcommand{\arraycolsep}{2pt}
\begin{array}{rcccrl}
\langle Df,g\rangle &=& \langle \lambda f,g\rangle &=& \lambda\phantom{)}\langle f,g\rangle
&\multirow{2}{*}{\hspace{3pt}$\biggl\}\mathstrut-\mathstrut$}\\
=\langle f,Dg\rangle &=& \langle f,\mu g\rangle &=& \mu\phantom{)}\langle f,g\rangle&
\\[2pt]
\hline
         0           & &                        &=& (\lambda-\mu)\langle f,g\rangle&
\end{array}
\label{buch:integrale:eqn:eigenwertesenkrecht}
\end{equation}
Da $\lambda-\mu\ne 0$ ist, muss $\langle f,g\rangle=0$ sein.

Der Operator $D$ ist selbstadjungiert, d.~h.
für zwei beliebige zweimal stetig differenzierbare Funktion $f$ und $g$
auf dem Intervall $[-1,1]$ gilt
\begin{align*}
\langle Df,g\rangle
&=
\int_{-1}^1 (Df)(x) g(x) \,dx
\\
&=
\int_{-1}^1
\biggl(\frac{d}{dx} (1-x^2)\frac{d}{dx}f(x)\biggr) g(x)
\,dx
\\
&=
\underbrace{
\biggl[
\biggl((1-x^2)\frac{d}{dx}f(x)\biggr) g(x)
\biggr]_{-1}^1
}_{\displaystyle = 0}
-
\int_{-1}^1
\biggl((1-x^2)\frac{d}{dx}f(x)\biggr) \frac{d}{dx}g(x)
\,dx
\\
&=
-
\int_{-1}^1
\biggl(\frac{d}{dx}f(x)\biggr) \biggl((1-x^2)\frac{d}{dx}g(x)\biggr)
\,dx
\\
&=
-
\underbrace{
\biggl[
f(x) \biggl((1-x^2)\frac{d}{dx}g(x)\biggr)
\biggr]_{-1}^1}_{\displaystyle = 0}
+
\int_{-1}^1
f(x) \biggl(\frac{d}{dx}(1-x^2)\frac{d}{dx}g(x)\biggr)
\,dx
\\
&=
\langle f,Dg\rangle.
\end{align*}
Dies beweist, dass $D$ selbstadjungiert ist.
Da $\bar{P}_n$ Eigenwerte des selbstadjungierten Operators $D$ zu
den verschiedenen Eigenwerten $-n(n+1)$ sind, folgt auch, dass
die $\bar{P}_n$ orthogonale Polynome vom Grad $n$ sind, die die 
gleiche Standardierdisierungsbedingung wie die Legendre-Polyonome
erfüllen, also ist $\bar{P}_n(x)=P_n(x)$.

\subsubsection{Legendre-Funktionen zweiter Art}
%Siehe Wikipedia-Artikel \url{https://de.wikipedia.org/wiki/Legendre-Polynom}
%
Die Potenzreihenmethode liefert natürlich auch Lösungen der
Legendreschen Differentialgleichung, die sich nicht als Polynome
darstellen lassen.
Ist $n$ gerade, dann liefern die Anfangswerte $a_0=0$ und $a_1=1$ 
eine ungerade Funktion, die Folge der Koeffizienten bricht
aber nicht ab, vielmehr ist
\begin{align*}
a_{k+2}
&=
\frac{k(k+1)}{(k+1)(k+2)}a_k
=
\frac{k}{k+2}a_k.
\end{align*}
Durch wiederholte Anwendung dieser Rekursionsformel findet man
\[
a_{k}
=
\frac{k-2}{k}a_{k-2}
=
\frac{k-2}{k}\frac{k-4}{k-2}a_{k-4}
=
\frac{k-2}{k}\frac{k-4}{k-2}\frac{k-6}{k-4}a_{k-6}
=
\dots
=
\frac{1}{k}a_1.
\]
Die Lösung hat daher die Reihenentwicklung
\[
Q_0(x) = x+\frac13x^3 + \frac15x^5 + \frac17x^7+\dots
=
\frac12\log \frac{1+x}{1-x}
=
\operatorname{artanh}x.
\]
Die Funktion $Q_0(x)$ heisst {\em Legendre-Funktion zweiter Art}.

Für $n=1$ wird die Reihenentwicklung $a_0=1$ und $a_1=0$ etwas
interessanter.
Die Rekursionsformel für die Koeffizienten ist
\[
a_{k+2}
=
\frac{k(k+1)-2}{(k+1)(k+2)} a_k.
\qquad\text{oder}\qquad
a_k
=
\frac{(k-1)(k-2)-2}{k(k-1)}
a_{k-2}
\]
Man erhält der Reihe nach
\begin{align*}
a_2 &= \frac{-2}{2\cdot 1} a_0 = -1
\\
a_3 &= 0
\\
a_4 &= \frac{3\cdot 2-2}{4\cdot 3} a_2 = \frac{4}{4\cdot 3}a_2 = \frac13a_2 = -\frac13
\\
a_5 &= 0
\\
a_6 &= \frac{5\cdot 4-2}{6\cdot 5}a_4 = \frac{18}{6\cdot 5}a_4 = -\frac15
\\
a_7 &= 0
\\
a_8 &= \frac{7\cdot 6-2}{8\cdot 7}a_6 = \frac{40}{8\cdot 7} = -\frac17
\\
a_9 &= 0
\\
a_{10} &= \frac{9\cdot 8-2}{10\cdot 9}a_8 = \frac{70}{10\cdot 9} = -\frac19,
\end{align*}
woraus sich die Reihenentwicklung
\begin{align*}
y(x)
&=
-x^2 -\frac13x^4 -\frac15x^6 - \frac17x^8 -\frac19x^{10}-\dots
\\
&=
-x\biggl(x+\frac13x^3 + \frac15x^5 + \frac17x^7 + \frac19x^9+\dots\biggr)
=
-x\operatorname{artanh}x.
\end{align*}
Die {\em Legendre-Funktionen zweiter Art} $Q_n(x)$  werden allerdings
so definiert, dass gewisse Rekursionsformeln für die Legendre-Polynome,
die wir hier nicht hergeleitet haben, auch für die $Q_n(x)$ gelten.
In dieser Normierung muss statt des eben berechneten $y(x)$ die Funktion
\[
Q_1(x) = x \operatorname{artanh}x-1
\]
verwendet werden.


%
% Besselfunktionen also orthogonale Funktionenfamilie
%
\section{Bessel-Funktionen als orthogonale Funktionenfamilie}
\rhead{Bessel-Funktionen}
Auch die Besselfunktionen sind eine orthogonale Funktionenfamilie.
Sie sind Funktionen differenzierbaren Funktionen $f(r)$ für $r>0$
mit $f'(r)=0$ und für $r\to\infty$ nimmt $f(r)$ so schnell ab, dass
auch $rf(r)$ noch gegen $0$ strebt.
Das Skalarprodukt ist
\[
\langle f,g\rangle
=
\int_0^\infty r f(r) g(r)\,dr,
\]
als Operator verwenden wir
\[
A = \frac{d^2}{dr^2} + \frac{1}{r}\frac{d}{dr} + s(r),
\]
wobei $s(r)$ eine beliebige integrierbare Funktion sein kann.
Zunächst überprüfen wir, ob dieser Operator wirklich selbstadjungiert ist.
Dazu rechnen wir
\begin{align}
\langle Af,g\rangle
&=
\int_0^\infty
r\,\biggl(f''(r)+\frac1rf'(r)+s(r)f(r)\biggr) g(r)
\,dr
\notag
\\
&=
\int_0^\infty rf''(r)g(r)\,dr
+
\int_0^\infty f'(r)g(r)\,dr
+
\int_0^\infty s(r)f(r)g(r)\,dr.
\notag
\intertext{Der letzte Term ist symmetrisch in $f$ und $g$, daher
ändern wir daran weiter nichts.
Auf das erste Integral kann man partielle Integration anwenden und erhält}
&=
\biggl[rf'(r)g(r)\biggr]_0^\infty
-
\int_0^\infty f'(r)g(r) + rf'(r)g'(r)\,dr
+
\int_0^\infty f'(r)g(r)\,dr
+
\int_0^\infty s(r)f(r)g(r)\,dr.
\notag
\intertext{Der erste Term verschwindet wegen der Bedingungen an die
Funktionen $f$ und $g$.
Der erste Term im zweiten Integral hebt sich gegen das
zweite Integral weg.
Der letzte Term ist das Skalarprodukt von $f'$ und $g'$.
Somit ergibt sich
}
&=
-\langle f',g'\rangle
+
\int_0^\infty s(r) f(r)g(r)\,dr.
\label{buch:integrale:orthogonal:besselsa}
\end{align}
Vertauscht man die Rollen von $f$ und $g$, erhält man das Gleiche, da im
letzten Ausdruck~\eqref{buch:integrale:orthogonal:besselsa} die Funktionen
$f$ und $g$ symmetrische auftreten.
Damit ist gezeigt, dass der Operator $A$ selbstadjungiert ist.
Es folgt nun, dass Eigenvektoren des Operators $A$ automatisch
orthogonal sind.

Eigenfunktionen von $A$ sind aber Lösungen der Differentialgleichung
\[
\begin{aligned}
&&
Af&=\lambda f
\\
&\Rightarrow\qquad&
f''(r) +\frac1rf'(r) + s(r)f(r) &= \lambda f(r)
\\
&\Rightarrow\qquad&
r^2f''(r) +rf'(r)+ (-\lambda r^2+s(r)r^2)f(r) &= 0
\end{aligned}
\]
sind.

Durch die Wahl $s(r)=1$ wird der Operator $A$ zum Bessel-Operator
$B$ definiert in
\eqref{buch:differentialgleichungen:bessel-operator}.
Die Lösungen der Besselschen Differentialgleichung zu verschiedenen Werten
des Parameters müssen also orthogonal sein, insbesondere sind die
Besselfunktion $J_\nu(r)$ und $J_\mu(r)$ orthogonal wenn $\mu\ne\nu$ ist.


%
% sturm.tex
%
% (c) 2022 Prof Dr Andreas Müller, OST Ostschweizer Fachhochschule
%
\section{Das Sturm-Liouville-Problem
\label{buch:integrale:subsection:sturm-liouville-problem}}
\rhead{Das Sturm-Liouville-Problem}
Sowohl bei den Bessel-Funktionen wie bei den Legendre-Polynomen
konnte die Orthogonalität der Funktionen dadurch gezeigt werden,
dass sie als Eigenfunktionen eines bezüglich eines geeigneten
Skalarproduktes selbstadjungierten Operators erkannt wurden.

%
% Differentialgleichungen
%
\subsection{Differentialgleichung}
Das klassische Sturm-Liouville-Problem ist das folgende Eigenwertproblem.
Gesucht sind Lösungen der Differentialgleichung
\begin{equation}
((p(x)y'(x))' + q(x)y(x) = \lambda w(x) y(x)
\label{buch:integrale:eqn:sturm-liouville}
\end{equation}
auf dem Intervall $(a,b)$, die zusätzlich die Randbedingungen
\begin{equation}
\begin{aligned}
k_a y(a) + h_a p(a) y'(a) &= 0 \\
k_b y(b) + h_b p(b) y'(b) &= 0
\end{aligned}
\label{buch:integrale:sturm:randbedingung}
\end{equation}
erfüllen, wobei $|k_i|^2 + |h_i|^2\ne 0$ mit $i=a,b$.
Weitere Bedingungen an die Funktionen $p(x)$, $q(x)$, $w(x)$  sowie die
Lösungsfunktionen $y(x)$ sollen später geklärt werden.

%
% Das verallgemeinerte Eigenwertproblem für symmetrische Matrizen
%
\subsection{Das verallgemeinerte Eigenwertproblem für symmetrische Matrizen}
Ein zu \eqref{buch:integrale:eqn:sturm-liouville} analoges Eigenwertproblem
für Matrizen ist das folgende verallgemeinerte Eigenwertproblem.
Das gewohnte Eigenwertproblem verwendet die Matrix $B=E$.

\begin{definition}
\index{verallgemeinerter Eigenvektor}%
\index{Eigenvektor, verallgemeinerter}%
\label{buch:orthogonal:sturm:verallgemeinerter-eigenvektor}
Seien $A$ und $B$ $n\times n$-Matrizen.
$v$ heisst {\em verallgemeinerter Eigenvektor} zum Eigenwert $\lambda$,
wenn
\[
Av = \lambda Bv.
\]
\end{definition}

Für symmetrische Matrizen lässt sich dieses Problem auf ein 
Optimierungsproblem reduzieren.

\begin{satz}
Seien $A$ und $B$ symmetrische $n\times n$-Matrizen und sei ausserdem
$B$ positiv definit.
Ist $v$ ein Vektor, der die Grösse
\[
f(v)=\frac{v^tAv}{v^tBv}
\]
maximiert, ist ein verallgemeinerter Eigenvektor für die Matrizen $A$
und $B$.
\end{satz}

\begin{proof}[Beweis]
Sei $\lambda = f(v)$ der maximale Wert und $u\ne 0$ ein beliebiger Vektor. 
Da $v$ die Grösse $f(v)$ maximiert, muss die Ableitung
von $f(u+tv)$ für $t=0$ verschwinden.
Um diese Ableitung zu berechnen, bestimmen wir zunächst die Ableitung
von $(v+tu)^tM(v+tu)$  an der Stelle $t=0$ für eine beliebige
symmetrische Matrix:
\begin{align*}
\frac{d}{dt}
(v+tu)^tM(v+tu)
&=
\frac{d}{dt}\bigl(
v^tv + t(v^tMu+u^tMv) + t^2 u^tMu
\bigr)
=
v^tMu+u^tMv + 2tv^tMv
\\
\frac{d}{dt}
(v^t+tu^t)M(v+tu)
\bigg|_{t=0}
&=
v^tMu+u^tMv
=
2v^tMu
\end{align*}
Dies wenden wir jetzt auf den Quotenten $\lambda(v+tu)$ an.
\begin{align*}
\frac{d}{dt}f(v+tu)\bigg|_{t=0}
&=
\frac{d}{dt}
\frac{(v+tu)^tA(v+tu)}{(v+tu)^tB(v+tu)}\bigg|_{t=0}
\\
&=
\frac{2u^tAv(v^tBv) - 2u^tBv(v^tAv)}{(v^tBv)^2}
=
\frac{2}{v^tBv}
u^t
\biggl(
Av - \frac{v^tAv}{v^tBv} Bv
\biggr)
\\
&=
2
\frac{
u^t(
Av - \lambda Bv
)
}{v^tBv}
\end{align*}
Da $v$ ein Maximum von $\lambda(v)$ ist, verschwindet diese Ableitung
für alle Vektoren $u$, somit gilt
\[
u^t(Av-\lambda Bv)=0
\]
für alle $u$, also auch $Av=\lambda Bv$.
Dies beweist, dass $v$ ein verallgemeinerter Eigenvektor zum
Eigenwert $\lambda$ ist.
\end{proof}

\begin{satz}
Verallgemeinerte Eigenvektoren $u$ und $v$ von $A$ und $B$
zu verschiedenen Eigenwerten erfüllen $u^tBv=0$.
\end{satz}

\begin{proof}
Seien $\lambda$ und $\mu$ die Eigenwerte, also $Au=\lambda Bu$
und $Av=\mu Bv$.
Wie in \eqref{buch:integrale:eqn:eigenwertesenkrecht}
berechnen wir das Skalarprodukt auf zwei Arten
\[
\renewcommand{\arraycolsep}{2pt}
\begin{array}{rcccrl}
 u^tAv &=&u^t\lambda Bv &=& \lambda\phantom{\mathstrut-\mu)} u^tBv
	&\multirow{2}{*}{\hspace{3pt}$\bigg\}\mathstrut-\mathstrut$}\\
=v^tAu &=&v^t\mu Bu     &=&  \mu\phantom{)}u^tBv         &\\
\hline
     0 & &              &=& (\lambda - \mu)u^tBv.        &
\end{array}
\]
Da die Eigenwerte verschieden sind, ist $\lambda-\mu\ne 0$, es folgt, 
dass $u^tBv=0$ sein muss.
\end{proof}

Verallgemeinerte Eigenwerte und Eigenvektoren verhalten sich also
ganz analog zu den gewöhnlichen Eigenwerten und Eigenvektoren.
Da $B$ positiv definit ist, ist $B$ auch invertierbar.
Zudem kann $B$ zur Definition des verallgemeinerten Skalarproduktes
\[
\langle u,v\rangle_B = u^tBv
\]
verwendet werden.
Die Matrix 
\[
\tilde{A} = B^{-1}A
\]
ist bezüglich dieses Skalarproduktes selbstadjungiert, denn es gilt
\[
\langle\tilde{A}u,v\rangle_B
=
(B^{-1}Au)^t Bv
=
u^tA^t(B^{-1})^tBv
=
u^tAv
=
u^tBB^{-1}Av
=
\langle u,\tilde{A}v\rangle.
\]
Das verallgemeinerte Eigenwertproblem für symmetrische Matrizen
ist damit ein gewöhnliches Eigenwertproblem für selbstadjungierte
Matrizen des Operators $\tilde{A}$ bezüglich des verallgemeinerten
Skalarproduktes $\langle\,\;,\;\rangle_B$.

%
% Der Operator L_0 und die Randbedingung
%
\subsection{Der Operator $L_0$ und die Randbedingung}
Die Differentialgleichung kann auch in Operatorform geschrieben werden.
Dazu schreiben wir
\[
L_0 
=
-\frac{d}{dx}p(x)\frac{d}{dx}.
\]
Bezüglich des gewöhnlichen Skalarproduktes
\[
\langle f,g\rangle
=
\int_a^b f(x)g(x)\,dx
\]
für Funktionen auf dem Intervall $[a,b]$ ist der Operator $L_0$
tatsächlich selbstadjungiert.
Mit partieller Integration rechnet man nach:
\begin{align}
\langle f,L_0g\rangle
&=
\int_a^b f(x) \biggl(-\frac{d}{dx}p(x)\frac{d}{dx}\biggr)g(x)\,dx
\notag
\\
&=
-\int_a^b f(x) \frac{d}{dx}\bigl( p(x) g'(x) \bigr)\,dx
\notag
\\
&=
-\biggl[f(x) p(x)g'(x)\biggr]_a^b
+
\int_a^b f'(x) p(x) g'(x) \,dx
\notag
\\
\langle L_0f,g\rangle
&=
-\biggl[f'(x)p(x)g(x)\biggr]_a^b
+
\int_a^b f'(x) p(x) g'(x) \,dx.
\notag
\intertext{Die beiden Skalarprodukte führen also auf das gleiche
Integral, sie unterscheiden sich nur um die Randterme}
\langle f,L_0g\rangle
-
\langle L_0f,g\rangle
&=
-f(b)p(b)g'(b) + f(a)p(a)g'(a)
+f'(b)p(b)g(b) - f'(a)p(a)g(a)
\label{buch:integrale:sturm:sabedingung}
\\
&=
-
p(b)
\left|\begin{matrix}
f(b) &g(b)\\
f'(b)&g'(b)
\end{matrix}\right|
+
p(a)
\left|\begin{matrix}
f(a) &g(a)\\
f'(a)&g'(a)
\end{matrix}\right|
\notag
\\
&=
-
\left|\begin{matrix}
f(b) &g(b)\\
p(b)f'(b)&p(b)g'(b)
\end{matrix}\right|
+
\left|\begin{matrix}
f(a) &g(a)\\
p(a)f'(a)&p(a)g'(a)
\end{matrix}\right|.
\notag
\end{align}
Um zu erreichen, dass der Operator selbstadjunigert wird, muss 
sichergestellt werden, dass entweder $p$ oder die Determinanten
an den Intervallenden verschwinden.
Dies passiert genau dann, wenn die Vektoren 
\[
\begin{pmatrix}
f(a)\\
p(a)f'(a)
\end{pmatrix}
\text{\;und\;}
\begin{pmatrix}
g(a)\\
p(a)g'(a)
\end{pmatrix}
\]
linear abhängig sind.
In zwei Dimensionen bedeutet lineare Abhängigkeit, dass es
eine nichttriviale Linearform mit Koeffizienten $h_a, k_a$ gibt,
die auf beiden Vektoren verschwindet.
Ausgeschrieben bedeutet dies, dass die Randbedingung
\eqref{buch:integrale:sturm:randbedingung}
erfüllt sein muss.

%
% Skalarprodukt
%
\subsection{Skalarprodukt}
Das Ziel der folgenden Abschnitte ist, das Sturm-Liouville-Problem als
Eigenwertproblem für einen selbstadjungierten Operator in einem 
Funktionenraum mit einem geeigneten Skalarprodukt zu finden.

Wir haben bereits gezeigt, dass die Randbedingung
\eqref{buch:integrale:sturm:randbedingung} sicherstellt, dass der
Operator $L_0$ für das Standardskalarprodukt selbstadjungiert ist.
Dies entspricht der Symmetrie der Matrix $A$.

Die Komponente $q(x)$ stellt keine besonderen Probleme, denn
\[
\langle f,qg\rangle
=
\int_a^b f(x)q(x)g(x)\,dx
=
\langle qf,g\rangle.
\]
Der Operator $f(x) \mapsto q(x)f(x)$, der eine Funktion mit 
der Funktion $q(x)$ multipliziert, ist also ebenfalls symmetrisch.
Dasselbe gilt für einen Operator, der mit $w(x)$ multipliziert.
Da $w(x)$ eine positive Funktion ist, ist der Operator $f(x)\mapsto w(x)f(x)$
sogar positiv definit.
Dies entspricht der Matrix $B$.
Nach den Erkenntnissen des vorangegangenen Abschnittes ist das
verallgemeinerte Eigenwertproblem daher ein Eigenwertproblem
für einen modifizierten Operator bezüglich eines alternativen
Skalarproduktes.

Als Skalarprodukt muss also das Integral
\[
\langle f,g\rangle_w
=
\int_a^b f(x)g(x)w(x)\,dx
\]
mit der Gewichtsfunktion $w(x)$ verwendet werden.
Damit dies ein vernünftiges Skalarprodukt ist, muss $w(x)>0$ im
Innerend es Intervalls sein.

%
% Der Vektorraum H
%
\subsection{Der Vektorraum $H$}
Damit können wir jetzt die Eigenschaften der in Frage kommenden
Funktionen zusammenstellen.
Zunächst müssen sie auf dem Intervall $[a,b]$ definiert sein und
das Integral
\[
\int_a^b |f(x)|^2 w(x)\,dx < \infty
\]
muss existieren.
Wir bezeichnen den Vektorraum der Funktionen, deren Quadrat mit
der Gewichtsfunktion $w(x)$ integrierbar sind, mit
$L^2([a,b],w)$.

Damit auch $\langle qf,f\rangle_w$ und $\langle L_0f,f\rangle_w$
wohldefiniert sind, müssen zusätzlich die Integrale
\[
\int_a^b |f(x)|^2 q(x) w(x)\,dx
\qquad\text{und}\qquad
\int_a^b |f'(x)|^2 p(x) w(x)\,dx
\]
existieren.
Wir setzen daher
\[
H
=
\biggl\{
f\in L^2([a,b],w)\;\bigg|\;
\int_a^b |f'(x)|^2p(x)w(x)\,dx<\infty,
\int_a^b |f(x)|^2q(x)w(x)\,dx<\infty
\biggr\}.
\]

%
% Der Sturm-Liouville-Differentialoperator
%
\subsection{Der Sturm-Liouville-Differentialoperator}
Das verallgemeinerte Eigenwertproblem für $A$ und $B$ ist ein
gewöhnliches Eigenwertproblem für die Operator $\tilde{A}=B^{-1}A$
bezüglich des modifizierten Skalarproduktes.
Das Sturm-Liouville-Problem ist also ein Eigenwertproblem im
Vektorraum $H$ mit dem Skalarprodukt $\langle\,\;,\;\rangle_w$.
Der Operator
\[
L = \frac{1}{w(x)} \biggl(-\frac{d}{dx} p(x)\frac{d}{dx} + q(x)\biggr)
\]
heisst der {\em Sturm-Liouville-Operator}.
Eine Lösung des Sturm-Liouville-Problems ist eine Funktion $y(x)$ derart,
dass 
\[
Ly = \lambda y,
\]
$\lambda$ ist der zu $y(x)$ gehörige Eigenwert.
Der Operator ist definiert auf Funktionen des im vorangegangenen Abschnitt
definierten Vektorraumes $H$.

%
% Beispiele
%
\subsection{Beispiele}
Die meisten der früher vorgestellten Funktionenfamilien stellen sich
als Lösungen eines geeigneten Sturm-Liouville-Problems heraus.
Alle Eigenschaften aus der Sturm-Liouville-Theorie gelten daher
automatisch für diese Funktionenfamilien.

%
% Trignometrische Funktionen
%
\subsubsection{Trigonometrische Funktionen}
Die trigonometrischen Funktionen sind Eigenfunktionen des Operators
$d^2/dx^2$, also eines Sturm-Liouville-Operators mit $p(x)=1$, $q(x)=0$
und $w(x)=1$.
Auf dem Intervall $(-\pi,\pi)$ können wir die Randbedingungen
\bgroup
\renewcommand{\arraycolsep}{2pt}
\[
\begin{aligned}
&
\begin{array}{lclclcl}
k_{-\pi}          &=&1,&\qquad&h_{-\pi}          &=&0\\
k_{\phantom{-}\pi}&=&1,&\qquad&h_{\phantom{-}\pi}&=&0
\end{array}
\;\bigg\}
&&\Rightarrow&
\begin{array}{lcl}
y(-\pi)          &=&0\\
y(\phantom{-}\pi)&=&0\\
\end{array}
\;\bigg\}
&\quad\Rightarrow&
y(x) &= B\sin nx
\\
&
\begin{array}{lclclcl}
k_{-\pi}          &=&0,&\qquad&h_{-\pi}          &=&1\\
k_{\phantom{-}\pi}&=&0,&\qquad&h_{\phantom{-}\pi}&=&1
\end{array}
\;\bigg\}
&&\Rightarrow&
\begin{array}{lcl}
y'(-\pi)          &=&0\\
y'(\phantom{-}\pi)&=&0\\
\end{array}
\; \bigg\}
&\quad\Rightarrow&
y(x) &= A\cos nx
\end{aligned}
\]
\egroup
verwenden.
Die Orthogonalität der Sinus- und Kosinus-Funktionen folgt jetzt
ganz ohne weitere Rechnung.

An dieser Lösung ist nicht ganz befriedigend, dass die trigonometrischen
Funktionen nicht mit einer einzigen Randbedingung gefunden werden können.
Der Ausweg ist, periodische Randbedingungen zu verlangen, also
$y(-\pi)=y(\pi)$ und $y'(-\pi)=y'(\pi)$.
Dann ist wegen
\begin{align*}
\langle f,L_0g\rangle - \langle L_0f,g\rangle
&=
-f(\pi)g'(\pi)+f(-\pi)g'(-\pi)+f'(\pi)g(\pi)-f'(-\pi)g(-\pi)
\\
&=
-f(\pi)g'(\pi)+f(\pi)g'(\pi)+f'(\pi)g(\pi)-f'(\pi)g(\pi)
=0
\end{align*}
die Bedingung~\eqref{buch:integrale:sturm:sabedingung}
ebenfalls erfüllt, $L_0$ ist in diesem Raum selbstadjungiert.

%
% Bessel-Funktionen J_n(x)
%
\subsubsection{Bessel-Funktionen $J_n(x)$}
Der Bessel-Operator \eqref{buch:differentialgleichungen:bessel-operator}
kann wie folgt in die Form eines Sturm-Liouville-Operators gebracht 
werden.
Zunächst rechnet man
\[
B
=
x^2\frac{d^2}{dx^2} + x\frac{d}{dx} + x^2
=
x\biggl(
x\frac{d^2}{dx^2} + \frac{d}{dx} + x
\biggr)
=
x\biggl(
\frac{d}{dx}(-x)\frac{d}{dx} + x
\biggr).
\]
Somit ist $B$ ein Sturm-Liouville-Operator für 
\begin{equation}
\begin{aligned}
p(x) &= -x \\
q(x) &= x \\
w(x) &= \frac{1}{x}.
\end{aligned}
\label{buch:orthogonal:sturm:bessel:n}
\end{equation}
Am linken Rand kann als Randbedingung 
\[
\lim_{x\to 0} p(x) y'(x) = 0
\]
verwendet werden, die für alle Bessel-Funktionen erfüllt ist.
Dies entspricht der Wahl $k_0=0$ und $h_0=1$.
Am rechten Rand für $x\to\infty$ kann man
\[
\lim_{x\to\infty} y(x)=0
\]
verlangen, was der Wahl $k_\infty=1$ und $h_\infty=0$ entspricht.
Damit ist die Bessel-Differentialgleichung erkannt als ein
Sturm-Liouville-Problem für $\lambda=n^2$.
Es folgt damit sofort, dass die Besselfunktionen orthogonale
Funktionen bezüglich des Skalarproduktes mit der Gewichtsfunktion
$w(x)=1/x$ sind.

%
% Bessel-Funktionen J_n(sx)
%
\subsubsection{Bessel-Funktionen $J_n(s x)$}
Das Sturm-Liouville-Problem mit den Funktionen
\eqref{buch:orthogonal:sturm:bessel:n}
ist jedoch nicht die einzige Möglichkeit, die Bessel-Differentialgleichung
in ein Sturm-Liouville-Problem zu verwandeln.
Das Problem \eqref{buch:orthogonal:sturm:bessel:n} ging davon
aus, dass $n^2$ der verallgemeinerte Eigenwert sein soll.
Im Folgenden sollen hingegen die Funktionen $J_n(s x)$ für
konstantes $n$, aber verschiedene $s$ untersucht und
als orthogonal erkannt werden.

Die Funktion $y(x) = J_n(x)$ ist eine Lösung der Bessel-Differentialgleichung
\[
x^2y'' + xy' + x^2y = n^2y.
\]
Setzt man $x=s t$ und $f(t)=y(s t)$, dann wird die Ableitung 
\[
\begin{aligned}
f'(t)
&=
\frac{d}{dt}y(s t)
=
y'(s t) \cdot s
&&\Rightarrow
&
\frac{f'(t)}{s}
&=
y'(x)
\\
f''(t)
&=
\frac{d^2}{dt^2} y(s t)
=
y''(s t) \cdot s^2
&&\Rightarrow
&
\frac{f''(t)}{s^2}
&=
y''(x).
\end{aligned}
\]
Setzt man diese in die Besselsche Differentialgleichung ein,
findet man
\begin{align*}
x^2y''+xy'+x^2y
=
s^2 t^2 \frac{f''(t)}{s^2}
+
s t \frac{f'(t)}{s}
+
s^2 t^2 f(t)
&=
n^2 f(t).
\end{align*}
Damit ist gezeigt, dass die Funktionen $J_n(s x)$ Lösungen
der Differentialgleichung
\begin{equation}
x^2y'' + xy' + (s^2 x^2  - n^2) y = 0
\label{buch:orthogonal:sturm:eqn:bessellambda}
\end{equation}
ist.

Die Differentialgleichung
\eqref{buch:orthogonal:sturm:eqn:bessellambda}
soll jetzt ebenfalls als ein Sturm-Liouville-Problem betrachtet
werden, diesmal aber mit festem $n$ und $s^2$ als dem verallgemeinerten
Eigenwert.
Dazu wird
\begin{equation}
\begin{aligned}
p(x) &= -x \\
q(x) &= -\frac{n^2}{x} \\
w(x) &= x
\end{aligned}
\label{buch:orthgonal:sturm:bessel:lambdaparams}
\end{equation}
gesetzt.
Das zugehörige Sturm-Liouville-Problem ist jetzt
\[
\frac{1}{x}\biggl(
\frac{d}{dx} (-x)\frac{d}{dx} -\frac{n^2}{x}
\biggr)
y
=
\lambda y
\quad\Rightarrow\quad
y'' + \frac{1}{x}y' - \frac{n^2}{x^2}y = \lambda y,
\]
oder nach Multiplikation mit $x^2$
\begin{equation}
x^2y'' + xy' + ((-\lambda)x^2 - n^2) y = 0.
\end{equation}
Die Funktionen $J_n(sx)$ sind daher verallgemeinerte Eigenfunktionen
des Sturm-Liouville-Problems
\eqref{buch:orthgonal:sturm:bessel:lambdaparams}
für den Eigenwert $\lambda = -s^2$.

\begin{satz}[Orthogonalität der Bessel-Funktionen]
Die Bessel-Funktionen $J_n(sx)$ für verschiedene $s$ sind orthogonal
bezüglich des Skalarproduktes mit der Gewichtsfunktion $w(x)=x$,
d.~h.
\[
\int_0^\infty J_n(s_1x) J_n(s_2x) x\,dx
=
0
\]
für $s_1\ne s_2$.
\end{satz}

\begin{proof}[Beweis]
Um die Bessel-Funktionen als Lösung des Sturm-Liouville-Problems
\eqref{buch:orthgonal:sturm:bessel:lambdaparams}
zu betrachten, müssen noch geeignete Randbedingungen formuliert werden.
Für $n>0$ kann man 
$J_n(0)=0$ verwenden, also $k_0=1$ und $h_0=0$.
Für $J_0$ ist dies nicht geeignet, aber wegen $J_0'(0)=0$ kann
man für $n=0$ verwenden $k_0=0$ und $h_0=1$ wählen.

Für den rechten Rand kann man verwenden, dass die Ableitung der
Bessel-Funktionen wie $x^{-3/2}$ gegen $0$ geht, gilt
\[
\lim_{x\to\infty} p(x) J_n(sx) = 0,
\]
weil $p(x)J_n(sx)$ wie $x^{-1/2}$ gegen $0$ geht.
Dies bedeutet, dass $k_\infty=0$ und $h_\infty=1$
verwendet werden kann.
Damit sind geeignete Randbedingungen für das Sturm-Liouville-Problem
gefunden.
\end{proof}

%
% Laguerre-Polynome
%
\subsubsection{Laguerre-Polynome}
Die Laguerre-Polynome sind orthogonal bezüglich des Skalarprodukts
mit der Laguerre-Gewichtsfunktion $w(x)=e^{-x}$ und erfüllen die
Laguerre-Differentialgleichung
\eqref{buch:differentialgleichungen:eqn:laguerre-dgl}.
mit $p(x)=-xe^{-x}$ wird 
\[
\frac{1}{w(x)}
\biggl(
-
\frac{d}{dx} p(x) \frac{d}{dx}
\biggr)
=
e^x \biggl(xe^{-x}\frac{d^2}{dx^2} + (1-x)e^{-x}\frac{d}{dx}\biggr),
\]
dies sind die abgeleiteten Terme in der Laguerre-Differentialgleichung.
Der Definitionsbereich ist $(0,\infty)$.
Als Randbedingung am linken kann man $y(0)=1$ verwenden, welche
auch die Laguerre-Polynome ergeben hat.
Am rechten Rand ist die Bedingung
\[
\lim_{x\to\infty} p(x)y'(x)
=
\lim_{x\to\infty} xe^{-x} y'(x)
=
0
\]
für beliebige Polynomlösungen erfüllt, dies ist der Fall
$k_{\infty}=0$ und $h_\infty=1$.

Das zugehörige verallgemeinerte Eigenwertproblem  wird damit
\[
xy'' + (1-x)y' - \lambda y = 0,
\]
also die Laguerre-Differentialgleichung.
Somit folgt, dass die Laguerre-Polynome orthogonal sind bezüglich
des Skalarproduktes mit der Laguerre-Gewichtsfunktion.

%
% Tschebyscheff-Polynome
%
\subsubsection{Tschebyscheff-Polynome}
Die Tschebyscheff-Polynome sind Lösungen der
Tschebyscheff-Differentialgleichung
\[
(1-x^2)y'' -xy' = n^2y
\]
auf dem Intervall $(-1,1)$.
Auch dieses Problem kann als Sturm-Liouville-Problem formuliert
werden mit
\begin{align*}
w(x) &= \frac{1}{\sqrt{1-x^2}} \\
p(x) &= \sqrt{1-x^2} \\
q(x) &= 0
\end{align*}
Das zugehörige Sturm-Liouville-Eigenwertproblem ist
\[
\frac{d}{dx}\sqrt{1-x^2}\frac{d}{dx} y(x)
=
\lambda \frac{1}{\sqrt{1-x^2}} y(x).
\]
Führt man die Ableitungen auf der linken Seite aus, entsteht die
Gleichung
\begin{align*}
\sqrt{1-x^2} y''(x) -\frac{x}{\sqrt{1-x^2}}y'(x)
&=  \lambda \frac{1}{\sqrt{1-x^2}} y(x)
\intertext{Multiplikation mit $\sqrt{1-x^2}$ ergibt}
(1-x^2)
y''(x) 
-
xy'(x)
&=
\lambda y(x).
\end{align*}
Es folgt, dass die Tschebyscheff-Polynome orthogonal sind 
bezüglich des Skalarproduktes
\[
\langle f,g\rangle = \int_{-1}^1 f(x)g(x)\frac{dx}{\sqrt{1-x^2}}.
\]

%
% Jacobi-Polynome
%
\subsubsection{Jacobi-Polynome}
TODO

%
% Hypergeometrische Differentialgleichungen
%
\subsubsection{Hypergeometrische Differentialgleichungen}
%\url{https://encyclopediaofmath.org/wiki/Hypergeometric_equation}
Auch die Eulersche hypergeometrische Differentialgleichung
lässt sich in die Form eines Sturm-Liouville-Operators
bringen.
Dazu setzt man
\begin{align*}
p(z)
&=
z^c(z-1)^{a+b+1-c}
\\
q(z)
&=
-abz^{c-1}(z-1)^{a+b-c}
\\
w(z)
&=
z^{c-1}(z-1)^{a+b-c}.
\end{align*}
Setzt man dies in den Sturm-Liouville-Operator ein, erhält man
\begin{equation}
L
=
-\frac{d}{dz}p(z)\frac{d}{dz} + q(z)
=
-p(z)\frac{d^2}{dz^2}
-p'(z)\frac{d}{dz}
+q(z)
\label{buch:orthgonalitaet:eqn:hypersturm}
\end{equation}
Wir brauchen also
\begin{align*}
p'(z)
&=
cz^{c-1}(z-1)^{a+b+1-c}
+
(a+b+1-c)
z^c
(z-1)^{a+b-c}
\\
&=
\bigl(
c(z-1)+
(a+b+1-c)z
\bigr)
\cdot
z^{c-1}(z-1)^{a+b-c}
\\
&=
-
\bigl(
c-(a+b+1)z
\bigr)
\cdot
z^{c-1}(z-1)^{a+b-c}.
\end{align*}
Einsetzen in~\eqref{buch:orthgonalitaet:eqn:hypersturm} liefert
\begin{align*}
L
%=
%-\frac{d}{dz}p(z)\frac{d}{dz}+q(z)
&=
-z^c(z-1)^{a+b+1-c} \frac{d^2}{dz^2}
+
w(z)
(c-(a+b+1)z)
\frac{d}{dz}
-
abw(z)
\\
&=
w(z)
\biggl(
-
z(z-1)
\frac{d^2}{dz^2}
+
(c-(a+b+1)z)
\frac{d}{dz}
-ab
\biggr)
\\
&=
w(z)
\biggl(
z(1-z)
\frac{d^2}{dz^2}
+
(c-(a+b+1)z)
\frac{d}{dz}
-ab
\biggr).
\end{align*}
Die Klammer auf der rechten Seite ist tatsächlich die linke Seite der
eulerschen hypergeometrischen Differentialgleichung.

Die hypergeometrische Funktion $\mathstrut_2F_1(a,b;c;z)$ ist ein
Eigenvektor des Operators $L$ zum Eigenwert $\lambda$.
Sei jetzt $w(z)$ eine Eigenfunktion zum Eigenwert $\lambda\ne 0$,
also
\[
z(1-z)w''(z) + (c-(a+b+1)z)w'(z) - ab w(z) = \lambda w(z).
\]
Kann man $a$ und $b$ so in $a_1$ und $b_1$ ändern, dass $a+b=a_1+b_1$
gleich bleiben aber das Produkt den Wert $a_1b_1=ab-\lambda$?
$a_1$ und $b_1$ sind die Lösungen der quadratischen Gleichung
\[
x^2 - (a+b)x + ab-\lambda = 0.
\]
Alle Eigenfunktionen des Operators $L$ sind also hypergeometrische
Funktion $\mathstrut_2F_1$.

Da die Gewichtsfunktion $w(z)$ bei der Ersetzung $a\to a_1$ und $b\to b_1$
sich nicht ändert ($w(z)$ hängt nur von der Summe $a+b$ ab, welche sich
nicht ändert), sind die beide beiden Eigenfunktionen bezüglich
des Skalarproduktes mit der Gewichtsfunktion $w(z)$ orthogonal.







%
% Anwendung: Gauss-Quadratur
%
\section{Anwendung: Gauss-Quadratur}
\rhead{Gauss-Quadratur}
Orthogonale Polynome haben eine etwas unerwartet Anwendung in einem
von Gauss erdachten numerischen Integrationsverfahren.
Es basiert auf der Beobachtung, dass viele Funktionen sich sehr
gut durch Polynome approximieren lassen.
Wenn man also sicherstellt, dass ein Verfahren für Polynome
sehr gut funktioniert, darf man auch davon ausgehen, dass es für
andere Funktionen nicht allzu schlecht sein wird.

\subsection{Interpolationspolynome}
Zu einer stetigen Funktion $f(x)$ auf dem Intervall $[-1,1]$ 
ist ein Polynome vom Grad $n$ gesucht, welches in den Punkten
$x_0<x_1<\dots<x_n$ die Funktionswerte $f(x_i)$ annimmt.
Ein solches Polynom $p(x)$ hat $n+1$ Koeffizienten, die aus dem
linearen Gleichungssystem der $n+1$ Gleichungen $p(x_i)=f(x_i)$ 
ermittelt werden können.

Das Interpolationspolynom $p(x)$ lässt sich abera uch direkt 
angeben.
Dazu konstruiert man zuerst die Polynome
\[
l_i(x)
=
\frac{
(x-x_0)(x-x_1)\cdots\widehat{(x-x_i)}\cdots (x-x_n)
}{
(x_i-x_0)(x_i-x_1)\cdots\widehat{(x_i-x_i)}\cdots (x_i-x_n)
}
\]
vom Grad $n$, wobei der Hut bedeutet, dass diese Faktoren
im Produkt wegzulassen sind.
Die Polynome $l_i(x)$ haben die Eigenschaft
\[
l_i(x_j) = \delta_{ij}
=
\begin{cases}
1&\qquad i=j\\
0&\qquad\text{sonst}.
\end{cases}
\]
Die Linearkombination
\[
p(x) = \sum_{i=0}^n f(x_i)l_i(x)
\]
ist dann ein Polynom vom Grad $n$, welches am den Stellen $x_j$
die Werte
\[
p(x_j) 
=
\sum_{i=0}^n f(x_i)l_i(x_j)
=
\sum_{i=0}^n f(x_i)\delta_{ij}
=
f(x_j)
\]
hat, das Polynome $p(x)$ ist also das gesuchte Interpolationspolynom.

\subsection{Integrationsverfahren auf der Basis von Interpolation}
Das Integral einer stetigen Funktion $f(x)$ auf dem Intervall $[-1,1]$
kann mit Hilfe des Interpolationspolynoms approximiert werden.
Wenn $|f(x)-p(x)|<\varepsilon$ ist im Intervall $[-1,1]$, dann gilt
für die Integrale
\[
\biggl|\int_{-1}^1 f(x)\,dx -\int_{-1}^1p(x)\,dx\biggr|
\le
\int_{-1}^1 |f(x)-p(x)|\,dx
\le
2\varepsilon.
\]
Ein Interpolationspolynom mit kleinem Fehler liefert also auch
eine gute Approximation für das Integral.

Da das Interpolationspolynome durch die Funktionswerte $f(x_i)$
bestimmt ist, muss auch das Integral allein aus diesen Funktionswerten
berechnet werden können.
Tatsächlich ist
\begin{equation}
\int_{-1}^1 p(x)\,dx
=
\int_{-1}^1 \sum_{i=0}^n f(x_i)l_i(x)\,dx
=
\sum_{i=0}^n f(x_i)
\underbrace{\int_{-1}^1
l_i(x)\,dx}_{\displaystyle = A_i}.
\label{buch:integral:gaussquadratur:eqn:Aidef}
\end{equation}
Das Integral von $f(x)$ wird also durch eine mit den Zahlen $A_i$
gewichtete Summe
\[
\int_{-1}^1 f(x)\,dx
\approx
\sum_{i=1}^n f(x_i)A_i
\]
approximiert.

\subsection{Integrationsverfahren, die für Polynome exakt sind}
Ein Polynom vom Grad $2n$ hat $2n+1$ Koeffizienten.
Um das Polynom durch ein Interpolationspolynom exakt wiederzugeben,
braucht man $2n+1$ Stützstellen.
Andererseits gilt
\[
\int_{-1}^1 a_{2n}x^{2n} + a_{2n-1}x^{2n-1} + \dots + a_2x^2 + a_1x + a_0\,dx
=
\int_{-1}^1 a_{2n}x^{2n} + a_{2n-2}x^{2n-2}+\dots +a_2x^2 +a_0\,dx,
\]
das Integral ist also bereits durch die $n+1$ Koeffizienten mit geradem
Index bestimmt.
Es sollte daher möglich sein, aus $n+1$ Funktionswerten eines beliebigen
Polynoms vom Grad höchstens $2n$ an geeignet gewählten Stützstellen das
Integral exakt zu bestimmen.

\begin{beispiel}
Wir versuchen dies für quadratische Polynome durchzuführen, also 
für $n=1$.
Gesucht sind also zwei Werte $x_i$, $i=0,1$ und Gewichte $A_i$, $i=0,1$
derart, dass für jedes quadratische Polynome $p(x)=a_2x^2+a_1x+a_0$ 
das Integral durch
\[
\int_{-1}^1 p(x)\,dx
=
A_0 p(x_0) + A_1 p(x_1)
\]
gebeben ist.
Indem wir für $p(x)$ die Polynome $1$, $x$, $x^2$ und $x^3$ einsetzen,
erhalten wir vier Gleichungen
\[
\begin{aligned}
p(x)&=\rlap{$1$}\phantom{x^2}\colon& 2       &= A_0\phantom{x_0}+ A_1     \\
p(x)&=x^{\phantom{2}}\colon& 0       &= A_0x_0   + A_1x_1  \\
p(x)&=x^2\colon& \frac23 &= A_0x_0^2 + A_1x_1^2\\
p(x)&=x^3\colon& 0       &= A_0x_0^3 + A_1x_1^3.
\end{aligned}
\]
Dividiert man die zweite und vierte Gleichung in der Form
\[
\left.
\begin{aligned}
A_0x_0 &= -A_1x_1\\
A_0x_0^2 &= -A_1x_1^2
\end{aligned}
\quad
\right\}
\quad
\Rightarrow
\quad
x_0^2=x_1^2
\quad
\Rightarrow
\quad
x_1=-x_0.
\]
Indem wir dies in die zweite Gleichung einsetzen, finden wir 
\[
0 = A_0x_0 + A_1x_1 = A_0x_1 -A_1x_0 = (A_0-A_1)x_0
\quad\Rightarrow\quad
A_0=A_1.
\]
Aus der ersten Gleichung folgt jetzt
\[
2= A_0+A_1 = 2A_0 \quad\Rightarrow\quad A_0 = 1.
\]
Damit bleiben nur noch die Werte von $x_i$ zu bestimmen, was 
mit Hilfe der zweiten Gleichung geschehen kann:
\[
\frac23 = A_0x_0^2 + A_1x_1^2 = 2x_0^2
\quad\Rightarrow\quad
x_0 = \frac{1}{\sqrt{3}}, x_1 = -\frac{1}{\sqrt{3}}
\]
Damit ist das Problem gelöst: das Integral eines Polynoms vom Grad 3
im Interval $[-1,1]$ ist exakt gegeben durch
\[
\int_{-1}^1 p(x)\,dx
=
p\biggl(-\frac{1}{\sqrt{3}}\biggr)
+
p\biggl(\frac{1}{\sqrt{3}}\biggr).
\]
Das Integral kann also durch nur zwei Auswertungen des Polynoms
exakt bestimmt werden.

Im Laufe der Lösung des Gleichungssystems wurden die Gewichte $A_i$
mit bestimmt.
Es ist aber auch möglich, die Gewichte zu bestimmen, wenn man die
Stützstellen kennt.
Nach \eqref{buch:integral:gaussquadratur:eqn:Aidef}
sind sie die $A_i$ gegeben als Integrale der Polynome
$l_i(x)$, die im vorliegenden Fall linear sind:
\begin{align*}
l_0(x)
&=
\frac{x-x_1}{x_0-x_1}
=
\frac{x-\frac1{\sqrt{3}}}{-\frac{2}{\sqrt{3}}}
=
\frac12(1-\sqrt{3}x)
\\
l_1(x)
&=
\frac{x-x_0}{x_1-x_0}
=
\frac{x+\frac1{\sqrt{3}}}{\frac{2}{\sqrt{3}}}
=
\frac12(1+\sqrt{3}x)
\end{align*}
Diese haben die Integrale
\[
\int_{-1}^1\frac12(1\pm\sqrt{3}x)\,dx
=
\int_{-1}^1 \frac12\,dx
=
1,
\]
da das Polynom $x$ verschwindendes Integral hat.
Dies stimmt mit $A_0=A_1=1$ überein.
\label{buch:integral:beispiel:gaussquadraturn1}
\end{beispiel}

Das eben vorgestellt Verfahren kann natürlich auf beliebiges $n$
verallgemeinert werden.
Allerdings ist die Rechnung zur Bestimmung der Stützstellen und
Gewichte sehr mühsam.

\subsection{Stützstellen und Orthogonalpolynome}
Sei $R_n=\{p(X)\in\mathbb{R}[X] \mid \deg p\le n\}$ der Vektorraum
der Polynome vom Grad $n$.

\begin{satz}
\label{buch:integral:satz:gaussquadratur}
Sei $p$ ein Polynom vom Grad $n$, welches auf allen Polynomen in $R_{n-1}$
orthogonal sind.
Seien ausserdem $x_0<x_1<\dots<x_n$ Stützstellen im Intervall $[-1,1]$ 
und $A_i\in\mathbb{R}$ Gewichte derart dass
\[
\int_{-1}^1 f(x)\,dx =
\sum_{i=0}^n A_if(x_i)
\]
für jedes Polynom $f$ vom Grad höchstens $2n-1$, dann sind die Zahlen
$x_i$ die Nullstellen des Polynoms $p$.
\end{satz}

\begin{proof}[Beweis]
Sei $f(x)$ ein beliebiges Polynom vom Grad $2n-1$.
Nach dem Polynomdivisionsalgorithmus gibt es
Polynome $q,r\in R_{n-1}$ derart, dass $f=qp+r$.
Dann ist das Integral von $f$ gegeben durch
\[
\int_{-1}^1 f(x)\,dx
=
\int_{-1}^1q(x) p(x)\,dx + \int_{-1}^1 r(x)\,dx
=
\langle q,p\rangle + \int_{-1}^1 r(x)\,dx.
\]
Da $p\perp R_{n-1}$ folgt insbesondere, dass $\langle q,p\rangle=0$.

Da die Integrale auch aus den Werten in den Stützstellen berechnet
werden können, muss auch
\[
0
=
\int_{-1}^1 q(x)p(x)\,dx
=
\sum_{i=0}^n q(x_i)p(x_i)
\]
für jedes beliebige Polynom $q\in R_{n-1}$ gelten.
Da man für $q$ die Interpolationspolynome $l_j(x)$ verwenden
kann, den Grad $n-1$ haben, folgt
\[
0
=
\sum_{i=0}^n
l_j(x_i)p(x_i)
=
\sum_{i=0}^n \delta_{ij}p(x_i),
\]
die Stützstellen $x_i$ müssen also die Nullstellen des Polynoms
$p(x)$ sein.
\end{proof}

Der Satz~\ref{buch:integral:satz:gaussquadratur} begründet das
{\em Gausssche Quadraturverfahren}.
Die in Abschnitt~\ref{buch:integral:section:orthogonale-polynome}
bestimmten Legendre-Polynome $P_n$ haben die im Satz
verlangte Eigenschaft,
dass sie auf allen Polynomen geringeren Grades orthogonal sind.
Wählt man die $n$ Nullstellen von $P_n$ als Stützstellen, erhält man 
automatisch ein Integrationsverfahren, welches für Polynome vom Grad
$2n-1$ exakt ist.

\begin{beispiel}
Das Legendre-Polynom $P_2(x) = \frac12(3x^2-1)$ hat die
Nullstellen $x=\pm1/\sqrt{3}$, dies sind genau die im Beispiel
auf Seite~\pageref{buch:integral:beispiel:gaussquadraturn1} befundenen
Sützstellen.
\end{beispiel}

\subsection{Fehler der Gauss-Quadratur}
Das Gausssche Quadraturverfahren mit $n$ Stützstellen berechnet
Integrale von Polynomen bis zum Grad $2n-1$ exakt.
Für eine beliebige Funktion kann man die folgende Fehlerabschätzung
angeben \cite[theorem 7.3.4, p.~497]{buch:numal}.

\begin{satz}
Seien $x_i$ die Stützstellen und $A_i$ die Gewichte einer
Gaussschen Quadraturformel mit $n+1$ Stützstellen und sei $f$
eine auf dem Interval $[-1,1]$ $2n+2$-mal stetig differenzierbare
Funktion, dann ist der $E$ Fehler des Integrals
\[
\int_{-1}^1 f(x)\,dx = \sum_{i=0}^n A_i f(x_i) + E
\]
gegeben durch
\begin{equation}
E = \frac{f^{(2n+2)}(\xi)}{(2n+2)!}\int_{-1}^1 l(x)^2\,dx,
\label{buch:integral:gaussquadratur:eqn:fehlerformel}
\end{equation}
wobei $l(x)=(x-x_0)(x-x_1)\dots(x-x_n)$  und $\xi$ ein geeigneter
Wert im Intervall $[-1,1]$ ist.
\end{satz}

Dank dem Faktor $(2n+2)!$ im Nenner von
\eqref{buch:integral:gaussquadratur:eqn:fehlerformel}
geht der Fehler für grosses $n$ sehr schnell gegen $0$.
Man kann auch zeigen, dass die mit Gauss-Quadratur mit $n+1$
Stützstellen berechneten Näherungswerte eines Integrals einer
stetigen Funktion $f(x)$ für $n\to\infty$ immer gegen den wahren
Wert des Integrals konvergieren.

\begin{table}
\def\u#1{\underline{#1}}
\centering
\begin{tabular}{|>{$}c<{$}|>{$}r<{$}|>{$}r<{$}|}
\hline
           n & \text{Gauss-Quadratur} & \text{Trapezregel} \\
\hline
\phantom{0}2 & 0.\u{95}74271077563381 & 0.\u{95}63709682242596 \\
\phantom{0}4 & 0.\u{95661}28333449730 & 0.\u{956}5513401768598 \\
\phantom{0}6 & 0.\u{9566114}812034364 & 0.\u{956}5847489712136 \\
\phantom{0}8 & 0.\u{956611477}5028123 & 0.\u{956}5964425360520 \\
          10 & 0.\u{9566114774905}637 & 0.\u{9566}018550715587 \\
          12 & 0.\u{956611477490518}7 & 0.\u{9566}047952369826 \\
          14 & 0.\u{95661147749051}72 & 0.\u{9566}065680717177 \\
          16 & 0.\u{956611477490518}7 & 0.\u{9566}077187127541 \\
          18 & 0.\u{956611477490518}3 & 0.\u{9566}085075898731 \\
          20 & 0.\u{956611477490518}4 & 0.\u{9566}090718697414 \\
\hline
      \infty & 0.9566114774905183 & 0.9566114774905183 \\
\hline
\end{tabular}
\caption{Integral von $\sqrt{1-x^2}$ zwischen $-\frac12$ und $\frac12$ 
berechnet mit Gauss-Quadratur und der Trapezregel, aber mit zehnmal
so vielen Stützstellen.
Bereits mit 12 Stützstellen erreicht die Gauss-Quadratur
Maschinengenauigkeit, die Trapezregel liefert auch mit 200 Stützstellen
nicht mehr als 4 korrekte Nachkommastellen.
\label{buch:integral:gaussquadratur:table0.5}}
\end{table}

%\begin{table}
%\def\u#1{\underline{#1}}
%\centering
%\begin{tabular}{|>{$}c<{$}|>{$}r<{$}|>{$}r<{$}|}
%\hline
%           n & \text{Gauss-Quadratur} & \text{Trapezregel} \\
%\hline
%\phantom{0}2 & 1.\u{5}379206741571556 & 1.\u{5}093105464758343 \\
%\phantom{0}4 & 1.\u{51}32373472933831 & 1.\u{51}13754509594814 \\
%\phantom{0}6 & 1.\u{512}1624557410367 & 1.\u{51}17610879524799 \\
%\phantom{0}8 & 1.\u{51207}93479994321 & 1.\u{51}18963282632112 \\
%          10 & 1.\u{51207}13859966004 & 1.\u{51}19589735776959 \\
%          12 & 1.\u{512070}5317779943 & 1.\u{51}19930161260693 \\
%          14 & 1.\u{5120704}334802813 & 1.\u{5120}135471596636 \\
%          16 & 1.\u{5120704}216176006 & 1.\u{5120}268743889558 \\
%          18 & 1.\u{5120704}201359081 & 1.\u{5120}360123137213 \\
%          20 & 1.\u{5120704199}459651 & 1.\u{5120}425490275837 \\
%\hline
%      \infty & 1.5120704199172947 & 1.5120704199172947 \\
%\hline
%\end{tabular}
%\end{table}

%\begin{table}
%\def\u#1{\underline{#1}}
%\centering
%\begin{tabular}{|>{$}c<{$}|>{$}r<{$}|>{$}r<{$}|}
%\hline
%           n & \text{Gauss-Quadratur} & \text{Trapezregel} \\
%\hline
%\phantom{0}2 & 1.\u{}6246862220133462 & 1.\u{5}597986803933712 \\
%\phantom{0}4 & 1.\u{5}759105515463101 & 1.\u{56}63563456168101 \\
%\phantom{0}6 & 1.\u{5}706630058381434 & 1.\u{56}77252866190838 \\
%\phantom{0}8 & 1.\u{56}94851106536780 & 1.\u{568}2298707696152 \\
%          10 & 1.\u{56}91283195332679 & 1.\u{568}4701957758742 \\
%          12 & 1.\u{56}90013806299465 & 1.\u{568}6030805941198 \\
%          14 & 1.\u{5689}515434853885 & 1.\u{568}6841603070025 \\
%          16 & 1.\u{5689}306507843050 & 1.\u{568}7372230731711 \\
%          18 & 1.\u{5689}214761291217 & 1.\u{568}7738235496322 \\
%          20 & 1.\u{56891}73062385982 & 1.\u{568}8001228530786 \\
%\hline
%      \infty & 1.5689135396691616 & 1.5689135396691616 \\
%\hline
%\end{tabular}
%\end{table}

\begin{table}
\def\u#1{\underline{#1}}
\centering
\begin{tabular}{|>{$}c<{$}|>{$}r<{$}|>{$}r<{$}|}
\hline
           n & \text{Gauss-Quadratur} & \text{Trapezregel} \\
\hline
\phantom{0}2 & 1.\u{}6321752373234928 & 1.\u{5}561048774629949 \\
\phantom{0}4 & 1.\u{57}98691557134743 & 1.\u{5}660124134617943 \\
\phantom{0}6 & 1.\u{57}35853681692993 & 1.\u{5}683353001877542 \\
\phantom{0}8 & 1.\u{57}19413565928206 & 1.\u{5}692627503425400 \\
          10 & 1.\u{57}13388119633434 & 1.\u{5}697323578543481 \\
          12 & 1.\u{57}10710489948883 & 1.\u{570}0051217458713 \\
          14 & 1.\u{570}9362135398341 & 1.\u{570}1784766276063 \\
          16 & 1.\u{570}8621102742815 & 1.\u{570}2959121005231 \\
          18 & 1.\u{570}8186779483588 & 1.\u{570}3793521168343 \\
          20 & 1.\u{5707}919411931615 & 1.\u{570}4408749735932 \\
\hline
      \infty & 1.5707367072605671 & 1.5707367072605671 \\
\hline
\end{tabular}
\caption{Integral von $\sqrt{1-x^2}$ zwischen $-0.999$ und $0.999$ 
berechnet mit Gauss-Quadratur und der Trapezregel, aber mit zehnmal
so vielen Stützstellen.
Wegen der divergierenden Steigung des Integranden bei $\pm 1$ tun
sich beide Verfahren sehr schwer. 
Trotzdem erreich die Gauss-Quadrator 4 korrekte Nachkommastellen
mit 20 Stütztstellen, während die Trapezregel auch mit 200 Stützstellen
nur 3 korrekte Nachkommastellen findet.
\label{buch:integral:gaussquadratur:table0.999}}
\end{table}

\begin{figure}
\centering
\includegraphics{chapters/060-integral/gq/gq.pdf}
\caption{Approximationsfehler des
Integrals~\eqref{buch:integral:gaussquadratur:bspintegral}
in Abhängigkeit von $a$.
Die Divergenz der Ableitung des Integranden an den Intervallenden
$\pm 1$ führt zu schlechter Konvergenz des Verfahrens, wenn $a$
nahe an $1$ ist.
\label{buch:integral:gaussquadratur:fehler}}
\end{figure}

Zur Illustration der Genauigkeit der Gauss-Quadratur berechnen wir
das Integral
\begin{equation}
\int_{-a}^a \sqrt{1-x^2}\,dx
=
\arcsin a + a \sqrt{1-a^2}
\label{buch:integral:gaussquadratur:bspintegral}
\end{equation}
mit Gauss-Quadratur einerseits und dem Trapezverfahren
andererseits.
Da Gauss-Quadratur mit sehr viel weniger Sützstellen auskommt,
berechnen wir die Trapeznäherung mit zehnmal so vielen Stützstelln.
In den Tabellen~\ref{buch:integral:gaussquadratur:table0.5}
und
\ref{buch:integral:gaussquadratur:table0.999}
sind die Resultate zusammengestellt.
Für $a =\frac12$ zeigt
Tabelle~\ref{buch:integral:gaussquadratur:table0.5}
die sehr schnelle Konvergenz der Gauss-Quadratur, schon mit
12 Stützstellen wird Maschinengenauigkeit erreicht.
Das Trapezverfahren dagegen erreicht auch mit 200 Stützstellen nur
4 korrekte Nachkommastellen.

An den Stellen $x=\pm 1$ divergiert die Ableitung des Integranden
des Integrals \eqref{buch:integral:gaussquadratur:bspintegral}.
Da grösste und kleinste Stützstelle der Gauss-Quadratur immer
deutlich vom Rand des Intervalls entfernt ist, kann das Verfahren
diese ``schwierigen'' Stellen nicht erkennen.
Tabelle~\ref{buch:integral:gaussquadratur:table0.999} zeigt, wie
die Konvergenz des Verfahrens in diesem Fall sehr viel schlechter ist.
Dies zeigt auch der Graph in
Abbildung~\ref{buch:integral:gaussquadratur:fehler}.

\subsection{Skalarprodukte mit Gewichtsfunktion}
Die Nullstellen der Legendre-Polynome ergaben ein gutes
Integrationsverfahren für Polynome auf einem beschränkten
Intervall.
Die Beispiele haben aber auch gezeigt, dass Stellen, wo die
Ableitung des Integranden divergiert, die Genauigkeit stark
beeinträchtigen können.
Ausserdem ist das Verfahren nicht anwendbar auf uneigentliche
Integrale.

\subsubsection{Umgang mit Singularitäten}
Die Lösung des Problems mit Stellen mit divergenter Ableitung
besteht darin, die Stützstellen in der Nähe dieser Stellen
zu konzentrieren.
Die Verwendung einer Gewichtsfunktion $w(x)$ kann genau dies
erreichen.
Statt das Integral einer Funktion $f(x)$ zu bestimmen, 
kann man $f(x)=g(x)w(x)$ schreiben, wobei $w(x)$ so
gewählt werden soll, dass das Verhalten der Steigung an
den Intervallenden gut wiedergibt.
Dies ist mit einer Jacobischen Gewichtsfunktion immer möglich.
Statt der Nullstellen der Legendre-Polynome sind dann die
Nullstellen der Jacobi-Polynome  und die Funktionswete von $g(x)$
an diesen Stellen zu verwenden,  die Gewichte sind
die Integrale von $l_i(x) P^{(\alpha,\beta)}(x)$.

\subsubsection{Uneigentliche Integrale}
Die Berechnung eines uneigentlichen Integrals auf dem Intervall
$(0,\infty)$ oder $(-\infty,\infty)$ ist aus mehreren Gründen nicht
direkt mit dem früher beschriebenen Gauss-Quadraturverfahren
möglich.

Die Stützstellen, die bei der Gauss-Quadratur in einem Intervall
$(a,b)$ verwendet werden, entstehen dadurch, dass man die Nullstellen
der Legendre-Polynome in $(-1,1)$ auf das Intervall $(a,b)$
skaliert.
Dies führt offensichtlich nicht zum Erfolg, wenn ein oder beide
Intervallgrenzen unendlich sind.
Dieses Problem kann dadurch gelöst werden, dass man das unendliche
Intervall $(a,\infty)$ mit
\[
x =  a + \frac{1-t}{t}
\]
auf das Intervall $[0,1]$ transformiert.

Will man beim Intervall $(0,\infty)$ bleiben, dann ist zu beachten,
dass das Integral eines Polynomes immer divergent ist, es ist also
auf jeden Fall nötig, den Integranden durch Funktionen zu approximieren,
die genügend schnell gegen $0$ gehen.
Polynome beliebigen Grades können verwendet werden, wenn sie mit
einer Funktion multipliziert werden, die schneller als jedes Polynom
gegen $0$ geht, so dass das Integral immer noch konvergiert.
Die Funktionen $e^{-x}$ für das Intervall $(0,\infty)$ oder
$e^{-x^2}$ für das Intervall $(-\infty,\infty)$ kommen dafür in Frage.

Um das Integral von $f(x)$ im Intervall $(0,\infty)$ zu berechnen,
schreibt man daher zunächst
\[
\int_0^\infty f(x)\,dx
=
\int_0^\infty g(x)e^{-x}\,dx
=
\int_0^\infty g(x) w(x)\,dx
\quad\text{mit}\quad
w(x)=e^{-x}
\text{ und }
g(x)=f(x)e^x.
\]
Dann approximiert $g(x)$ man durch ein Interpolationspolynom,
so wie man das bei der Gauss-Quadratur gemacht hat.
Als Stützstellen müssen dazu die Nullstellen der Laguerre-Polynome
verwendet werden.
Als Gewichte $w_i$ sind die Integrale der $l_i(x)e^{-x}$
zu verwenden.







\section*{Übungsaufgabe}
\rhead{Übungsaufgaben}
\aufgabetoplevel{chapters/070-orthogonalitaet/uebungsaufgaben}
\begin{uebungsaufgaben}
\uebungsaufgabe{701}
%\uebungsaufgabe{1}
\end{uebungsaufgaben}

