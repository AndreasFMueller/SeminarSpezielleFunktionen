%
% rodrigues.tex
%
% (c) 2022 Prof Dr Andreas Müller, OST Ostschweizer Fachhochschule
%
\section{Rodrigues-Formeln
\label{buch:orthogonalitaet:section:rodrigues}}
Die Drei-Term-Rekursionsformel ermöglicht Werte orthogonaler Polynome
effizient zu berechnen.
Die Rekursionsformel erhöht den Grad eines Polynoms, indem mit $x$ 
multipliziert wird.
mit der Ableitung kann man den Grad aber auch senken, man könnte daher
auch nach einer Rekursionsformel fragen, die bei einem Polynom hohen
Grades beginnt und mit Hilfe von Ableitungen zu geringeren Graden
absteigt.
Solche Formeln heissen Rodrigues-Formeln nach dem Entdecker Olinde
Rodrigues, der eine solche Formal als erster für Legendre-Polynome
gefunden hat.

In diesem Abschnitt sei $p_n(x)$ eine bezüglich des Skalarproduktes
$\langle\,\;,\;\rangle_w$ auf dem Intervall $[a,b]$ orthogonale Familie
von Polynomen mit genaum dem Grad $\deg p_n=n$.
Die Skalarprodukte sollen 
\[
\langle p_n,p_m\rangle_w = h_n\delta_{nm}
\]
sein.

\subsection{Pearsonsche Differentialgleichung}
Die {\em Pearsonsche Differentialgleichung} ist die Differentialgleichung
\begin{equation}
B(x) y' - A(x) y = 0,
\label{buch:orthogonal:eqn:pearson}
\end{equation}
wobei $B(x)$ ein Polynom vom Grad höchstens $2$ ist und $A(x)$ ein
höchstens lineares Polynom.
Die Gleichung~\eqref{buch:orthogonal:eqn:pearson}
kann gelöst werden, wenn $y$ und $B(x)$ keine Nullstellen  haben.
Dann kann man die Gleichung umstellen in
\[
\frac{y'}{y}
=
(\log y)'
=
\frac{A(x)}{B(x)}
\qquad\Rightarrow\qquad
y = \exp\biggl( \int\frac{A(x)}{B(x)}\biggr)\,dx.
\]
Im folgenden nehmen wir zusätzlich an, dass
\begin{equation}
\lim_{x\to a+} w(x)B(x) = 0,
\qquad\text{und}\qquad
\lim_{x\to b-} w(x)B(x) = 0.
\end{equation}
Falls $w(x)$ an den Intervallenden einen von $0$ verschiedenen
Grenzwert hat, bedeutet dies, dass $B(a)=B(b)=0$ sein muss.
Falls $w(x)$ am Intervallende divergiert, muss $B(x)$ dort eine
Nullstelle höherer Ordnung haben, was aber für ein Polynom
zweiten Grades nicht möglich ist.

\subsection{Rekursionsformel}
Multiplikation mit $B(x)$ wird den Grad eines Polynomes typischerweise 
um $2$ erhöhen, die Ableitung wird ihn wieder um $1$ reduzieren.
Etwas formeller kann man dies wie folgt formulieren:

\begin{satz}
Für alle $n\ge 0$ ist
\[
q_n(x)
=
\frac{1}{w(x)}
\frac{d^n}{dx^n} B(x)^n w(x)
\]
ein Polynom vom Grad höchstens $n$.
\end{satz}

\begin{proof}[Beweis]
Wenn $r_0(x)$ irgend eine differenzierbare Funktion ist, dann ist
\begin{align*}
\frac{d^n}{dx^n}
r_0(x) B(x)^n w(x)
&=
\frac{d^{n-1}}{dx^{n-1}}\frac{d}{dx} r_0(x) B(x)^n w(x)
\\
&=
\frac{d^{n-1}}{dx^{n-1}}
\bigl(r_0'(x)B(x)+ nB'(x)B(x)^{n-1}w(x) + B(x)^n w'(x) \bigr)
\\
&=
\frac{d^{n-1}}{dx^{n-1}}
(r_0'(x)B(x)+nB'(x)+A(x)) B(x)^{n-1} w(x)
=
\frac{d^{n-1}}{dx^{n-1}} r_1(x)B^{n-1}(x) w(x).
\end{align*}
Für die Funktionen $r_k$ gilt die Rekursionsformel
\begin{equation}
r_k(x) = r_{k-1}'(x)B(x) + kB'(x) + A(x).
\label{buch:orthogonal:rodrigues:rekursion:beweis1}
\end{equation}
Wenn $r_0(x)$ ein Polynom ist, dann sind alle Funktionen $r_k(x)$
ebenfalls Polynome.
Durch wiederholte Anwendung dieser Formel kann man schliessen, dass 
\[
\frac{d^n}{dx^n} r_0(x) B(x)^n w(x)
=
r_n(x) w(x).
\]
Insbesondere folgt für $r_0(x)=1$, dass man durch $w(x)$ dividieren kann
und dass $r_n(x)=q_n(x)$.

Wir müssen auch noch den Grad von $r_k(x)$ bestimmen.
Dazu verwenden wir 
\eqref{buch:orthogonal:rodrigues:rekursion:beweis1} und berechnen den
Grad:
\begin{equation*}
\deg r_k(x)
=
\max \bigl(
\underbrace{\deg(r_{k-1}'(x) B(x))}_{\displaystyle \deg r_{k-1}(x) -1 + 2}
,
\underbrace{\deg(B'(x))}_{\displaystyle \le 1}
,
\underbrace{\deg(A(x))}_{\displaystyle \le 1}
\bigr)
\le \max r_{k-1}(x) + 1.
\end{equation*}
Aus $\deg r_0(x)=0$ kann man jetzt ablesen, dass $\deg r_k(x)\le k$ ist.
Damit ist gezeigt, dass $\deg q_n(x)\le n$.
\end{proof}

\begin{satz}
Es gibt Konstanten $c_n$ derart, dass
\[
p_n(x)
=
\frac{c_n}{w(x)} \frac{d^n}{dx^n} \bigl(B(x)^n w(x)\bigr) 
\]
gilt.
\end{satz}

\begin{proof}[Beweis]
Wir müssen zeigen, dass die Polynome orthogonal sind auf allen Monomen
von geringerem Grad.
\begin{align*}
\langle q_n, x^k\rangle_w
&=
\int_a^b q_n(x)x^kw(x)\,dx
\\
&=
\int_a^b \frac{1}{w(x)}\frac{d^n}{dx^n}(B(x)^n w(x)) x^k w(x)\,dx
\\
&=
\int_a^b \frac{d^n}{dx^n}(B(x)^n w(x)) x^k \,dx
\\
&=
\biggl[\frac{d^{n-1}}{dx^{n-1}}(B(x)^n w(x)) x^k \biggr]_a^b
-
\int_a^b \frac{d^{n-1}}{dx^{n-1}}(B(x)^n w(x))kx^{k-1}\,dx
\end{align*}
Durch $n$-fache Iteration wird das Integral auf $0$ reduziert.
Es bleiben nur die eckigen Klammern stehen, doch wenn man die Produktregel
auswertet, bleibt immer mindestens ein Produkt $B(x)w(x)$ stehen,
nach den Voraussetzungen an den Grenzwert dieses Produktes an den
Intervallenden verschwinden diese Terme alle.
Damit sind die $q_n(x)$ Polynome, die $w$-orthogonal sind auf allen
$x^k$ mit $k<n$, also Vielfache der $w$-Orthgonalpolynome.
\end{proof}

\subsubsection{Legendre-Polynome}
Legendre-Polynome sind orthogonale Polynome zum Standardskalarprodukt
mit $w(x)=1$.
Die Pearsonsche Differentialgleichung ist für $A(x)=0$ immer erfüllt.
Die Randbedingung bedeutet wegen $w(x)=1$, dass $B(x)$ an den
Endpunkten des Intervalls verschwinden muss.
Da $B(x)$ ein Polynom höchstens vom Grad $2$ ist, muss $B(x)$ ein
Vielfaches von $(x-1)(x+1)=x^-1$ sein.
Die Rodrigues-Formel für die Legendre-Polynome hat daher die Form
\[
P_n(x)
=
c_n
\frac{d^n}{dx^n}
(x^2-1)^n,
\]
darin müssen die Konstanten $c_n$ noch bestimmt werden.
In der für die Legendre-Polynome gewählten Normierung ist
\[
c_n = \frac1{2^n n!}
\qquad\text{und damit}\qquad
P_n(x)
=
\frac{1}{2^nn!}
\frac{d^n}{dx^n}
(x^2-1)^n.
\]

\subsubsection{Hermite-Polynome}
TODO

\url{https://en.wikipedia.org/wiki/Rodrigues%27_formula}


