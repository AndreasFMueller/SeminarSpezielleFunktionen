%
% buchcover.tex -- Cover für das Buch Numerik
%
% (c) 2018 Prof Dr Andreas Müller, Hochschule Rapperswil
%
\documentclass[11pt]{standalone}
\usepackage{tikz}
\usepackage{times}
\usepackage{geometry}
\usepackage{german}
\usepackage[utf8]{inputenc}
\usepackage[T1]{fontenc}
\usepackage{times}
\usepackage{amsmath,amscd}
\usepackage{amssymb}
\usepackage{amsfonts}
\usepackage{txfonts}
\usepackage{ifthen}
\usetikzlibrary{math}
\geometry{papersize={402mm,278mm},total={405mm,278mm},top=72.27pt, bottom=0pt, left=72.27pt, right=0pt}
\newboolean{guidelines}
\setboolean{guidelines}{false}
\setboolean{guidelines}{true}

\begin{document}
\begin{tikzpicture}[>=latex, scale=1]
\tikzmath{
	real \ruecken, \einschlag, \gelenk, \breite, \hoehe;
	\ruecken = 3.0;
	\einschlag = 1.6;
	\gelenk = 0.8;
	\breite = 16.7;
	\hoehe = 24.6;
	real \bogengreite, \bogenhoehe;
	\bogenbreite = 2 * (\breite + \einschlag + \gelenk) + \ruecken;
	\bogenhoehe = 2 * \einschlag + \hoehe;
}

%\clip (0,0) circle (6);

\draw[fill=blue](0,0) rectangle({\bogenbreite},{\bogenhoehe});
\hsize=13.6cm

\begin{scope}
\clip (0,0) rectangle({\bogenbreite},{\bogenhoehe});
%\clip (0,0) rectangle ({\bogenbreite},20.2);
%\node at (18.7,8.9) [scale=5.0]{\includegraphics{nozzle-hell.jpg}};
%\node at (18.7,8.9) [scale=5.0]{\includegraphics{nozzle-hell3.jpg}};
%\node at (18.7,8.9) [scale=5.0]{\includegraphics{nozzle.jpg}};
%\node at ({\bogenbreite/2},13.0) {\includegraphics[width=42cm]{matrix.pdf}};
\end{scope}

\node at ({\einschlag+2*\gelenk+\ruecken+1.5*\breite},24.3)
	[color=white,scale=1]
	{\hbox to\hsize{\hfill%
	\sf \fontsize{24}{24}\selectfont Mathematisches Seminar}};

\node at ({\einschlag+2*\gelenk+\ruecken+1.5*\breite},21.9)
	[color=white,scale=1]
	{\hbox to\hsize{\hfill%
	\sf \fontsize{44}{44}\selectfont Spezielle Funktionen}};

\node at ({\einschlag+2*\gelenk+\ruecken+1.5*\breite},19.7)
	[color=white,scale=1]
	{\hbox to\hsize{\hfill%
	\sf \fontsize{13}{5}\selectfont Andreas Müller}};

\node at ({\einschlag+2*\gelenk+\ruecken+1.5*\breite},18.4)
	[color=white,scale=1]
	{\hbox to\hsize{\hfill%
	\sf \fontsize{13}{5}\selectfont
	Joshua Baer,			% E
	Marius Baumann,			% E
	Reto Fritsche,			% E (2)
	Alain Keller%,			% E
%	Ahmet Güzel%,			% E
	}};

\node at ({\einschlag+2*\gelenk+\ruecken+1.5*\breite},17.75)
	[color=white,scale=1]
	{\hbox to\hsize{\hfill%
	\sf \fontsize{13}{5}\selectfont
	Marc Kühne,
	Robine Luchsinger,		% B
	Naoki Pross,			% E
	Thomas Reichlin%,                % B
	}};

\node at ({\einschlag+2*\gelenk+\ruecken+1.5*\breite},17.1)
	[color=white,scale=1]
	{\hbox to\hsize{\hfill%
	\sf \fontsize{13}{5}\selectfont
	Michael Schmid,		% MSE
	Pascal Andreas Schmid,		% B
	Adrian Schuler%,
	}};
 
\node at ({\einschlag+2*\gelenk+\ruecken+1.5*\breite},16.45)
	[color=white,scale=1]
	{\hbox to\hsize{\hfill%
	\sf \fontsize{13}{5}\selectfont
	Thierry Schwaller,		% E
	Michael Steiner,		% E
	Tim Tönz,			% E
	Fabio Viecelli%,		% B
	}};

\node at ({\einschlag+2*\gelenk+\ruecken+1.5*\breite},15.8)
	[color=white,scale=1]
	{\hbox to\hsize{\hfill%
	\sf \fontsize{13}{5}\selectfont
	Lukas Zogg%,			% B
	%
	}};

\node at ({\einschlag+2*\gelenk+\ruecken+1.5*\breite},15.15)
	[color=white,scale=1]
	{\hbox to\hsize{\hfill%
	\sf \fontsize{13}{5}\selectfont
	%Reto Wildhaber%			% B
	}};
 
%\node at (0,3) [color=white] {\sf \LARGE Mathematisches Seminar 2017};

% Rücken
\node at ({\bogenbreite/2 + 0.00},18.5) [color=white,rotate=-90]
	{\sf\fontsize{35}{0}\selectfont Spezielle Funktionen\strut};

% Buchrückseite
\node at ({\einschlag+0.5*\breite},18.6) [color=white] {\sf
\fontsize{13}{16}\selectfont
\vbox{%
\parindent=0pt
%\raggedright
Das Mathematische Seminar der Ostschweizer Fachhochschule
in Rapperswil hat sich im Frühjahrssemester 2022 dem Thema
Spezielle Funktionen
zugewandt.
Ziel war, die grosse Vielfalt von speziellen Funktionen und
Funktionenfamilien zu ergründen, die im Laufe der Zeit für die
verschiedensten Anwendungen erdacht wurden.
Dieses Buch bringt das Skript des Vorlesungsteils mit den von den
Seminarteilnehmern beigetragenen Seminararbeiten zusammen.
}};


\ifthenelse{\boolean{guidelines}}{
\draw[white] (0,{\einschlag})--({\bogenbreite},{\einschlag});
\draw[white] (0,{\bogenhoehe-\einschlag})--({\bogenbreite},{\bogenhoehe-\einschlag});

\draw[white] ({\einschlag},0)--({\einschlag},{\bogenhoehe});
\draw[white] ({\einschlag+\breite},0)--({\einschlag+\breite},{\bogenhoehe});
\draw[white] ({\einschlag+\breite+\gelenk},0)--({\einschlag+\breite+\gelenk},{\bogenhoehe});
\draw[white] ({\bogenbreite-\einschlag-\breite-\gelenk},0)--({\bogenbreite-\einschlag-\breite-\gelenk},{\bogenhoehe});
\draw[white] ({\bogenbreite-\einschlag-\breite},0)--({\bogenbreite-\einschlag-\breite},{\bogenhoehe});
\draw[white] ({\bogenbreite-\einschlag},0)--({\bogenbreite-\einschlag},{\bogenhoehe});
}{}

\end{tikzpicture}
\end{document}
