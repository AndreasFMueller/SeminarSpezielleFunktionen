%
% funktionen.tex -- slide template
%
% (c) 2021 Prof Dr Andreas Müller, OST Ostschweizer Fachhochschule
%
\bgroup
\begin{frame}[t]
\setlength{\abovedisplayskip}{5pt}
\setlength{\belowdisplayskip}{5pt}
\frametitle{Spezielle Funktionen}
%\begin{center}
\renewcommand\arraystretch{1.8}
\begin{tabular}{ll}
Aufgabe&Lösung\\
\hline
Gleichung $x^n=a$ lösen &
\uncover<2->{{\color<9->{red}Wurzelfunktion} $x = \sqrt[n]{a}$}
\\
Kreis beschreiben &
\uncover<3->{{\color<9->{red}trignometrische Funktionen} $\sin t$, $\cos t$}
\\
Differentialgleichung $y' = ay$ lösen &
\uncover<4->{{\color<9->{red}Exponentialfunktion} $y(x) = \exp (ax)$}
\\
Exponentialgleichung $a^x=b$ lösen &
\uncover<5->{{\color<9->{red}Logarithmus} $x = \log_a b$}
\\
Integral $\displaystyle \int_0^x e^{-t^2}\,dt$ berechnen &
\uncover<6->{Keine Lösung $\rightarrow \displaystyle
{\color<9->{red}\operatorname{erf}(x)} = \int_0^xe^{-t^2}\,dt$}
\\
Fakultät $x!$ auf $x\in\mathbb{C}$ ausdehnen &
\uncover<7->{{\color<9->{red}Gamma-Funktion} $\Gamma(x)$}
\\
$x^2y'' +xy' + (x^2-a^2)y=0$ &
\uncover<8->{{\color<9->{red}Bessel-Funktionen} $J_a$, $Y_a$, $I_a$, $K_a$}
\end{tabular}
%\end{center}
\end{frame}
\egroup
