%
% ordnungsstatistik.tex -- Ordnungsstatistik
%
% (c) 2021 Prof Dr Andreas Müller, OST Ostschweizer Fachhochschule
%
\bgroup
\begin{frame}[t]
\setlength{\abovedisplayskip}{5pt}
\setlength{\belowdisplayskip}{5pt}
\frametitle{Ordnungstatistik}
\vspace{-10pt}
\begin{block}{Angeordnete Stichprobe}
\[
X_{1:n}
\le
X_{2:n}
\le
\dots
\le
X_{(n-1):n}
\le
X_{n:n}
\]
$X_{k:n} = \mathstrut$der $k$-te von $n$ Werten
\end{block}
\vspace{-20pt}
\begin{columns}[t,onlytextwidth]
\begin{column}{0.44\textwidth}
\uncover<2->{%
\begin{block}{Verteilungsfunktion}
\begin{align*}
F_{X_{k:n}}(x)
&=
P(X_{k:n} \le x)
\\
&\uncover<3->{=
P\bigl(
|\{i\;|\; {\color<4>{red}X_i\le x}\}| \ge k
\bigr)}
\\
&\uncover<5->{=
P(\text{Anzahl $A_i$}\ge k)}
\\
&\uncover<9->{=
P(K\ge k)}
\\
\uncover<6->{
F_{X_i}(x)&= P(X_i\le x)}\uncover<7->{ = P(A_i)}\uncover<10->{ = p}
}
\end{align*}
\uncover<4->{$A_i=\{X_i\le x\}$}\uncover<7->{ ist ein Beroulli- Experiment 
\uncover<10->{mit Eintretens- wahrscheinlichkeit $p$}
\end{block}}
\end{column}
\begin{column}{0.52\textwidth}
\uncover<8->{%
\begin{block}{Wiederholtes Bernoulli-Experiment}
$K=\mathstrut$Anzahl $k$, für die $A$ eingetreten
ist\only<11->{, ist binomialverteilt:}
\begin{align*}
\uncover<12->{P(K=k) 
&=
\phantom{\sum_{i=k}^n\mathstrut}
\binom{n}{k} p^k (1-p)^{n-k}
}
\\
\uncover<13->{
P(K\ge k)
&=
\sum_{i=k}^n
\binom{n}{i} p^i (1-p)^{n-i}
}
\\
\uncover<14->{
&=
\sum_{i=k}^n
\binom{n}{i} F_X(x)^i (1-F_X(x))^{n-i}
}
\end{align*}
\end{block}}
\end{column}
\end{columns}
\end{frame}
\egroup
