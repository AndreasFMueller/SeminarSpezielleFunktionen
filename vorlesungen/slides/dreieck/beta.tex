%
% beta.tex -- slide template
%
% (c) 2021 Prof Dr Andreas Müller, OST Ostschweizer Fachhochschule
%
\bgroup
\begin{frame}[t]
\setlength{\abovedisplayskip}{5pt}
\setlength{\belowdisplayskip}{5pt}
\frametitle{Beta-Verteilung}
\vspace{-20pt}
\begin{columns}[t,onlytextwidth]
\begin{column}{0.40\textwidth}
\begin{block}{Ordnungsstatistik}
\begin{align*}
\varphi(x)
&=
{\color{blue}N} x^{k-1} (1-x)^{n-k}
\\
&\uncover<8->{
=
\beta_{k,n-k+1}(x)
}
\end{align*}
\end{block}
\uncover<8->{%
\begin{block}{Risch-Algorithmus}
Die Beta-Verteilungen haben ausser in Spezialfällen
keine Stammfunktion in geschlossener Form.
\end{block}}
\end{column}
\begin{column}{0.56\textwidth}
\uncover<2->{%
\begin{definition}
Beta-Verteilung
\[
\beta_{a,b}(x)
=
\begin{cases}
\displaystyle
\uncover<7->{
{\color{blue}
\frac{1}{B(a,b)}
}
}
x^{a-1}(1-x)^{b-1}
&0\le x\le 1
\\
0&\text{sonst}
\end{cases}
\]
\end{definition}}
\uncover<3->{%
\begin{block}{Normierung}
\begin{align*}
{\color{blue}\frac{1}{{N}}}
&\uncover<4->{=
\int_{-\infty}^\infty \beta_{a,b}(x)\,dx}
\\
&\uncover<5->{=
\int_{0}^1 x^{a-1}(1-x)^{b-1}\,dx}
\\
&\uncover<6->{=
B(a,b)}
\end{align*}
\end{block}}
\end{column}
\end{columns}
\end{frame}
\egroup
