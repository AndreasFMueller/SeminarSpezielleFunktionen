%
% minmax.tex -- Minimum und Maximum
%
% (c) 2021 Prof Dr Andreas Müller, OST Ostschweizer Fachhochschule
%
\bgroup
\begin{frame}[t]
\setlength{\abovedisplayskip}{5pt}
\setlength{\belowdisplayskip}{5pt}
\frametitle{Minimum und Maximum}
\vspace{-20pt}
\begin{columns}[t,onlytextwidth]
\begin{column}{0.48\textwidth}
\begin{block}{Maximum}
Verteilungsfunktion von
\[
Z=\operatorname{max}(X_1,\dots,X_n)
\]
\begin{align*}
\uncover<3->{
F_Z(x)
&=
P(Z\le x)}
\\
\uncover<4->{
&=
P(X_1\le x\wedge\dots\wedge X_n\le x)
}
\\
\uncover<5->{
&=
P(X_1\le x)\cdot \ldots\cdot P(X_n\le x)
}
\\
\uncover<6->{
&=
F_X(x)^n
}
\end{align*}
\end{block}
\end{column}
\begin{column}{0.48\textwidth}
\uncover<2->{%
\begin{block}{Minimum}
Verteilungsfunktion von
\[
Z=\operatorname{min}(X_1,\dots,X_n)
\]
\begin{align*}
\uncover<7->{
F_Z(x)
&=
P(Z\le x)
}
\\
\uncover<8->{
&=P(\overline{
X_1\le x\wedge\dots\wedge X_n \le x
})
}
\\
\uncover<9->{
&=
1-P(
X_1> x\wedge\dots\wedge X_n > x
)
}
\\
\uncover<10->{
&=
1-(P(X_1>x)\cdot\ldots\cdot P(X_n>x))
}
\\
\uncover<11->{
&=
1-(1-F_X(x))^n
}
\end{align*}
\end{block}}
\end{column}
\end{columns}
\end{frame}
\egroup
