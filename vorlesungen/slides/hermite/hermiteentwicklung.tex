%
% hermiteentwicklung.tex -- slide template
%
% (c) 2021 Prof Dr Andreas Müller, OST Ostschweizer Fachhochschule
%
\bgroup
\begin{frame}[t]
\setlength{\abovedisplayskip}{5pt}
\setlength{\belowdisplayskip}{5pt}
\frametitle{Beliebige Polynome}
\vspace{-20pt}
\begin{columns}[t,onlytextwidth]
\begin{column}{0.48\textwidth}
\begin{block}{Polynom}
\[
P(x)
=
p_0 + p_1x + p_2x^2 + \dots + p_nx^n
\]
\uncover<2->{%
als Linearkombination von Hermite-Polynome schreiben:
\begin{align*}
P(x)
&=
a_0H_0(x)% + a_1H_1(x)
+ \dots + a_nH_n(x)
\\
&=
a_0\cdot 1
\\
&\quad + a_1\cdot 2x
\\
&\quad + a_2\cdot(4x^2-2) 
\\
&\quad + a_3\cdot(8x^3-12x)
\\
&\quad + a_4\cdot(16x^4-48x^2+12)
\\
&\quad\;\;\vdots
\\
&\quad + a_n(2^nx^n + \dots)
\end{align*}}
\end{block}
\end{column}
\begin{column}{0.48\textwidth}
\uncover<3->{%
\begin{block}{Koeffizientenvergleich}
führt auf ein Gleichungssystem
\begin{center}
\begin{tabular}{|>{$}r<{$}>{$}r<{$}>{$}r<{$}>{$}r<{$}>{$}r<{$}>{$}c<{$}|>{$}c<{$}|}
\hline
a_0&a_1&a_2&a_3&a_4&\dots&\\
\hline
 1&  0&  0&  0&  0&\dots&p_0\\
 0&  2&  0&  0&  0&\dots&p_1\\
-2&  0&  4&  0&  0&\dots&p_2\\
 0&-12&  0&  8&  0&\dots&p_3\\
12&  0&-48&  0& 16&\dots&p_4\\
\vdots&\vdots&\vdots&\vdots&\vdots&\ddots&\vdots\\
\hline
\end{tabular}
\end{center}
\uncover<4->{%
Dreiecksmatrix}\uncover<5->{, Diagonalelement
$\ne 0$}
\uncover<6->{$\Rightarrow$ 
$\exists$ eindeutige Lösung}
\end{block}}
\end{column}
\end{columns}
\end{frame}
\egroup
