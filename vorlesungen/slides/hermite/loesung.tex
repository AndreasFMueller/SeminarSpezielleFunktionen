%
% loesung.tex -- slide template
%
% (c) 2021 Prof Dr Andreas Müller, OST Ostschweizer Fachhochschule
%
\bgroup
\begin{frame}[t]
\setlength{\abovedisplayskip}{5pt}
\setlength{\belowdisplayskip}{5pt}
\frametitle{Lösung}
\vspace{-20pt}
\begin{columns}[t,onlytextwidth]
\begin{column}{0.48\textwidth}
\begin{block}{Frage}
Für welche Polynome $P(t)$ kann man eine Stammfunktion
\[
\int
P(t)e^{-\frac{t^2}2}
\,dt
\]
in geschlossener Form angeben?
\end{block}
\uncover<2->{%
\begin{block}{``Hermite-Antwort''}
\[
\int H_n(x)e^{-x^2}\,dx
\]
kann genau für $n>0$ in geschlossener Form angegeben werden.
\end{block}}
\end{column}
\begin{column}{0.48\textwidth}
\uncover<3->{%
\begin{block}{Allgemein}
\begin{align*}
\int P(x)e^{-x^2}\,dx
&\uncover<4->{=
\int \sum_{k=0}^n a_kH_k(x)e^{-x^2}\,dx}
\\
\uncover<5->{
&=
\sum_{k=0}^n
a_k
\int
H_k(x)e^{-x^2}\,dx
}
\\
\uncover<6->{
&=
a_0\operatorname{erf}(x) + C
}
\\
\uncover<6->{
&\hspace*{2mm} + \sum_{k=1}^n a_k\int H_k(x)e^{-x^2}\,dx
}
\end{align*}
\end{block}}
\uncover<7->{%
\begin{theorem}
Das Integral von $P(x)e^{-x^2}$ ist genau dann elementar darstellbar, wenn
$a_0=0$
\end{theorem}}
\end{column}
\end{columns}
\end{frame}
\egroup
