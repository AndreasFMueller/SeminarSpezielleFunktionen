%
% normalintegrale.tex -- 
%
% (c) 2021 Prof Dr Andreas Müller, OST Ostschweizer Fachhochschule
%
\bgroup
\begin{frame}[t]
\setlength{\abovedisplayskip}{5pt}
\setlength{\belowdisplayskip}{5pt}
\frametitle{Integranden $P(t)e^{-t^2}$}
\vspace{-20pt}
\begin{columns}[t,onlytextwidth]
\begin{column}{0.48\textwidth}
\begin{block}{Frage}
Für welche Polynome $P(t)$ kann man eine Stammfunktion
\[
\int
P(t)e^{-t^2}
\,dt
\]
in geschlossener Form angeben?
\end{block}
\uncover<4->{%
\begin{block}{Allgemeine Antwort}
Satz von Liouville und
Risch- Algorithmus können entscheiden, ob es eine elementare Stammfunktion gibt
\end{block}}
\end{column}
\begin{column}{0.48\textwidth}
\uncover<2->{%
\begin{block}{Negativbeispiel}
$P(t) = 1$, das Normalverteilungsintegral
\[
F(x)
=
\frac{1}{\sqrt{2\pi}} \int_{-\infty}^x e^{-t^2}\,dt
\]
ist nicht elementar darstellbar.
\end{block}}
\uncover<3->{%
\begin{block}{Positivbeispiel}
$P(t)=t$. Wegen
\begin{align*}
\frac{d}{dx}e^{-x^2}
&=
-xe^{-x^2}
\intertext{ist}
\int te^{-t^2}\,dt
&=
-e^{-x^2}+C
\end{align*}
elementar darstellbar.
\end{block}}
\end{column}
\end{columns}
\end{frame}
\egroup
