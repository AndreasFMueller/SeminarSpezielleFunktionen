%
% skalarprodukt.tex -- slide template
%
% (c) 2021 Prof Dr Andreas Müller, OST Ostschweizer Fachhochschule
%
\bgroup
\begin{frame}[t]
\setlength{\abovedisplayskip}{5pt}
\setlength{\belowdisplayskip}{5pt}
\frametitle{Skalarprodukt}
\vspace{-20pt}
\begin{columns}[t,onlytextwidth]
\begin{column}{0.48\textwidth}
\begin{block}{Orthogonale Zerlegung}
Orthogonale $H_k$ normalisieren:
\[
\tilde{H}_k(x) = \frac{1}{\|H_k\|_w} H_k(x)
\]
mit Gewichtsfunktion $w(x)=e^{-x^2}$
\end{block}
\begin{block}{``Hermite''-Analyse}
\begin{align*}
P(x)
&=
\sum_{k=1}^\infty a_k H_k(x)
=
\sum_{k=1}^\infty \tilde{a}_k \tilde{H}_k(x)
\\
\tilde{a}_k
&=
\| H_k\|_w\, a_k
\\
a_k
&=
\frac{1}{\|H_k\|}
\langle \tilde{H}_k, P\rangle_w
=
\frac{1}{\|H_k\|^2}
\langle H_k, P\rangle_w
\end{align*}
\end{block}
\end{column}
\begin{column}{0.48\textwidth}
\begin{block}{Integrationsproblem}
Bedingung:
\begin{align*}
a_0=0
\qquad\Leftrightarrow\qquad
\langle H_0,P\rangle_w
&=
0
\\
\int_{-\infty}^\infty
P(t) w(t)  \,dt
=
\int_{-\infty}^\infty
P(t) e^{-t^2}  \,dt
&=
0
\end{align*}
\end{block}
\begin{theorem}
Das Integral von $P(t)e^{-t^2}$ ist in geschlossener Form darstellbar
genau dann, wenn
\[
\int_{-\infty}^\infty P(t)e^{-t^2}\,dt = 0
\]
\end{theorem}
\end{column}
\end{columns}
\end{frame}
\egroup
