%
% klothoide.tex -- Klothoide
%
% (c) 2021 Prof Dr Andreas Müller, OST Ostschweizer Fachhochschule
%
\bgroup
\begin{frame}[t]
\setlength{\abovedisplayskip}{5pt}
\setlength{\belowdisplayskip}{5pt}
\frametitle{Klothoide}
\vspace{-20pt}
\begin{columns}[t,onlytextwidth]
\begin{column}{0.48\textwidth}
\begin{block}{Krümmung der Euler-Spirale}
\begin{align*}
\frac{d}{dt}\gamma_1(t)
&=
\dot{\gamma}_1(t)
=
\begin{pmatrix}
\cos t^2\\
\sin t^2
\end{pmatrix}
\intertext{\uncover<2->{Bogenlänge:}}
\uncover<2->{
|\dot{\gamma}_1(t)|
&=
\sqrt{\cos^2 t^2 + \sin^2 t^2}
=
1
}
\intertext{\uncover<3->{Polarwinkel:}}
\uncover<3->{
\varphi&=t^2
\intertext{\uncover<4->{Krümmung:}}
\uncover<4->{
\frac{d\varphi}{dt}
&=
2t
}
}
\end{align*}
\uncover<5->{%
$\Rightarrow$ Krümmung ist proportional zur Bogenlänge
}
\end{block}
\end{column}
\begin{column}{0.48\textwidth}
\uncover<6->{%
\begin{block}{Definition}
Eine Kurve, deren Krümmung proportional zur Bogenlänge ist, heisst
{\em Klothoide}
\end{block}}
\uncover<7->{%
\begin{block}{Anwendung}
\begin{itemize}
\item<8->
Strassenbau: Um mit konstanter Geschwindigkeit auf einer
Klothoide zu fahren, muss man das Lenkrad mit konstanter Geschwindigkeit
drehen
\item<9->
Apfel + Sparschäler
\end{itemize}
\end{block}}
\end{column}
\end{columns}
\end{frame}
\egroup
