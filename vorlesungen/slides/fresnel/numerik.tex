%
% numerik.tex -- slide template
%
% (c) 2021 Prof Dr Andreas Müller, OST Ostschweizer Fachhochschule
%
\bgroup
\begin{frame}[t]
\setlength{\abovedisplayskip}{5pt}
\setlength{\belowdisplayskip}{5pt}
\frametitle{Numerik}
\vspace{-20pt}
\begin{columns}[t,onlytextwidth]
\begin{column}{0.48\textwidth}
\begin{block}{Taylor-Reihe}
\begin{align*}
\sin t^2
&=
\sum_{k=0}^\infty
(-1)^k \frac{t^{2k+1}}{(2k+1)!}
\\
%\int \sin t^2\,dt
\uncover<2->{
S(t)
&=
\sum_{k=0}^\infty
(-1)^k \frac{t^{4k+3}}{(2k+1)!(4n+3)}
}
\\
\cos t^2
&=
\sum_{k=0}^\infty
(-1)^k \frac{t^{2k}}{(2k)!}
\\
%\int \sin t^2\,dt
\uncover<3->{
C(t)
&=
\sum_{k=0}^\infty
(-1)^k \frac{t^{4k+1}}{(2k)!(4k+1)}
}
\end{align*}
\end{block}
\end{column}
\begin{column}{0.48\textwidth}
\uncover<4->{
\begin{block}{Differentialgleichung}
\[
\dot{\gamma}(t)
=
\begin{pmatrix}
\sin t^2\\ \cos t^2
\end{pmatrix}
\]
\end{block}}
\uncover<5->{%
\begin{block}{Hypergeometrische Reihen}
\begin{align*}
\uncover<6->{%
S(t)
&=
\frac{\pi z^3}{6}\,
\mathstrut_1F_2\biggl(
\begin{matrix}\frac34\\\frac32,\frac74\end{matrix}
;
-\frac{\pi^2z^4}{16}
\biggr)
}
\\
\uncover<7->{
C(t)
&=
z\,
\mathstrut_1F_2\biggl(
\begin{matrix}\frac14\\\frac12,\frac54\end{matrix}
;
-\frac{\pi^2z^4}{16}
\biggr)}
\end{align*}
\end{block}}
\end{column}
\end{columns}
\end{frame}
\egroup
