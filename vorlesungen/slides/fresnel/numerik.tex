%
% numerik.tex -- numerische Berechnung der Fresnel Integrale
%
% (c) 2021 Prof Dr Andreas Müller, OST Ostschweizer Fachhochschule
%
\bgroup
\begin{frame}[t]
\setlength{\abovedisplayskip}{5pt}
\setlength{\belowdisplayskip}{5pt}
\frametitle{Numerik}
\vspace{-20pt}
\begin{columns}[t,onlytextwidth]
\begin{column}{0.48\textwidth}
\begin{block}{Taylor-Reihe}
\begin{align*}
\sin t^{\uncover<2->{\color<2>{red}2}}
&=
\sum_{k=0}^\infty
(-1)^k \frac{t^{
\ifthenelse{\boolean{presentation}}{\only<1>{2k+1}}{}
\only<2->{\color<2>{red}4k+2}
}
}{
(2k+1)!
}
\\
%\int \sin t^2\,dt
\uncover<4->{
S_1(t)
&=
\sum_{k=0}^\infty
(-1)^k \frac{t^{4k+3}}{(2k+1)!(4n+3)}
}
\\
\cos t^{\uncover<3->{\color<3>{red}2}}
&=
\sum_{k=0}^\infty
(-1)^k \frac{t^{
\ifthenelse{\boolean{presentation}}{\only<-2>{2k}}{}
\only<3->{\color<3>{red}4k}}
}{
(2k)!
}
\\
%\int \sin t^2\,dt
\uncover<5->{
C_1(t)
&=
\sum_{k=0}^\infty
(-1)^k \frac{t^{4k+1}}{(2k)!(4k+1)}
}
\end{align*}
\end{block}
\end{column}
\begin{column}{0.48\textwidth}
\uncover<6->{
\begin{block}{Differentialgleichung}
\[
\dot{\gamma}_1(t)
=
\begin{pmatrix}
\cos t^2\\ \sin t^2
\end{pmatrix}
\uncover<7->{
\;
\to
\;
\gamma_1(t)
=
\begin{pmatrix}
C_1(t)\\S_1(t)
\end{pmatrix}
}
\]
\end{block}}
\uncover<8->{%
\begin{block}{Hypergeometrische Reihen}
\begin{align*}
\uncover<9->{%
S(t)
&=
\frac{\pi z^3}{6}
\cdot
\mathstrut_1F_2\biggl(
\begin{matrix}\frac34\\\frac32,\frac74\end{matrix}
;
-\frac{\pi^2z^4}{16}
\biggr)
}
\\
\uncover<10->{
C(t)
&=
z
\cdot
\mathstrut_1F_2\biggl(
\begin{matrix}\frac14\\\frac12,\frac54\end{matrix}
;
-\frac{\pi^2z^4}{16}
\biggr)}
\end{align*}
\end{block}}
\end{column}
\end{columns}
\uncover<11->{%
\begin{block}{Komplexe Fehlerfunktion}
\[
\left.
\begin{matrix}
S(z)\\
C(z)
\end{matrix}
\right\}
=
\frac{1\pm i}{4}
\left(
\operatorname{erf}\biggl({\frac{1+i}2}\sqrt{\pi}z\biggr)
\mp i
\operatorname{erf}\biggl({\frac{1-i}2}\sqrt{\pi}z\biggr)
\right)
\]
\end{block}}
\end{frame}
\egroup
