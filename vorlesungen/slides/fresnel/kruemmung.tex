%
% kruemmung.tex -- Kruemmung
%
% (c) 2021 Prof Dr Andreas Müller, OST Ostschweizer Fachhochschule
%
\bgroup
\begin{frame}[t]
\setlength{\abovedisplayskip}{5pt}
\setlength{\belowdisplayskip}{5pt}
\frametitle{Krümmung einer Kurve}
\vspace{-20pt}
\begin{columns}[t,onlytextwidth]
\begin{column}{0.48\textwidth}
\begin{block}{Krümmungsradius}
Bogen und Radius:
\[
s=r\cdot\Delta\varphi
\uncover<2->{
\quad
\Rightarrow
\quad
r
=
\frac{s}{\Delta\varphi}
}
\]
\end{block}
\vspace*{-12pt}
\uncover<3->{
\begin{block}{Krümmung}
Je grösser der Krümmungsradius, desto kleiner die Krümmung:
\[
\kappa = \frac{1}{r}
\]
\end{block}}
\vspace*{-12pt}
\uncover<5->{%
\begin{block}{Definition}
Änderungsgeschwindigkeit des Polarwinkels der Tangente
\[
\kappa
=
\frac{1}{r}
\uncover<6->{=
\frac{\Delta\varphi}{s}}
\uncover<7->{=
\frac{d\varphi}{dt}}
\]
\end{block}}
\end{column}
\begin{column}{0.48\textwidth}
\begin{center}
\begin{tikzpicture}[>=latex,thick]

\begin{scope}
\clip (-1,-1) rectangle (4,4);

\def\r{3}
\def\winkel{30}

\fill[color=blue!20] (0,0) -- (0:\r) arc (0:\winkel:\r) -- cycle;
\node[color=blue] at ({0.5*\winkel}:{0.5*\r}) {$\Delta\varphi$};

\draw[line width=0.3pt] (0,0) circle[radius=\r];

\draw[->] (0,0) -- (0:\r);
\draw[->] (0,0) -- (\winkel:\r);

\uncover<4->{
\draw[->] (0:\r) -- ($(0:\r)+(90:0.7*\r)$);
\draw[->] (\winkel:\r) -- ($(\winkel:\r)+({90+\winkel}:0.7*\r)$);
}

\draw[color=red,line width=1.4pt] (0:\r) arc (0:\winkel:\r);
\node[color=red] at ({0.5*\winkel}:\r) [left] {$s$};
\fill[color=red] (0:\r) circle[radius=0.05];
\fill[color=red] (\winkel:\r) circle[radius=0.05];

\node at (\winkel:{0.5*\r}) [above] {$r$};
\node at (0:{0.5*\r}) [below] {$r$};
\end{scope}

\end{tikzpicture}
\end{center}
\uncover<4->{%
Für $\varphi$ kann man auch den Polarwinkel des Tangentialvektors nehmen
}
\end{column}
\end{columns}
\end{frame}
\egroup
